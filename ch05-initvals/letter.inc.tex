

% \begin{quote}

% The use of a fixed method would greatly simplify an algorithm design in any standalone autonomous toolbox. The results in Table II do not help to uncover whether such a strategy is permissible. To solve this dilemma, two histograms of differences $\Delta = \model{best} - \model{2nd}$ need to be built and shown: one for $\model{best} = \model{FRIR}$ and the other for $\model{best} \neq \model{FRIR}$.
% \end{quote}

% We agree that a single method would greatly simplify the implementation of a toolbox.
% However, we think that using multiple initial value strategies is a valuable approach since this makes a toolbox more robust to different conditions (SNR, frequency resolution, system order, \ldots) as it reduces the likelihood that a local optimum is hit.
% This can e.g. be seen in subplot 2 of \figref{fig:distancesStress} (which is discussed further on).
% A similar approach w.r.t. to combine initial values is followed in some toolboxes, e.g. \cite{FDIDENT,TDIDENT}.

% Many nature-inspired optimization algorithms (such as Particle Swarm Optimization (PSO)) and stochastic optimization algorithms even use hundreds or thousands of randomized initial values, in part to avoid local optima.
% But, obviously, using many initial values comes at a (considerable) increase in computational cost.

% While this discussion and this inherent trade-off is very interesting, we think this is outside of the scope of the paper where we only wish to demonstrate that smoothing techniques such as RFIR are a worthwhile source of initial values, but not necessarily the only one.
