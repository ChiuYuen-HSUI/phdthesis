
\newcommand{\reviewer}[1]{\paragraph{#1}}

\reviewer{Reviewer \#2 (Remarks for the Author):}
\begin{quote}
The authors propose two smoothing techniques, which improve the initial estimates for transfer function identification. Figure 2 and section VI are appreciated. 

Comments: 

1. Please treat carefully the optimization problems. Just presenting the cost functions is not enough. 

3. Which are the optimization algorithms? Clear steps must be provided. 
\end{quote}
Note that the minimization of the cost function (5) in the paper has been extensively studied in the literature (e.g. \cite[Section 9.11]{Pintelon2001} which has been referenced in the manuscript).
We have used the well-known Levenberg-Marquardt optimization algorithm to perform the non-convex optimization.
This information has also been added to the paper.

\begin{quote}
2. Do your smoothing techniques guarantee and/or keep the minimum? 
\end{quote}
The cost function involving the smoothed estimate is different from the cost function with the raw data. 
Obviously, in that case the minimum is expected to be shifted.
However, the result from this cost function is used as an \emph{initial} estimate for the cost function with the raw data. 
In the end, only the cost function with the raw data is retained.
The initial estimate does \emph{not} change the minimum of a cost function.

\begin{quote}
4. The transparency of the paper is suffering because the authors do not present how the theoretical results from sections II and III are applied in section IV. 

5. The same comment is also applied for section V, i.e., low transparency. 
\end{quote}

The theoretical considerations in sections II and III are applied in section IV.
In section III, the initial estimation procedures are enumerated, and a distinct notation for each of them is given:
\begin{itemize}
  \item use of the true parameters: $\model{0}$,
  \item use of existing techniques: $\model{exist}$,
  \item use of the LPM with truncation: $\model{trunc}$,
  \item use of the RFIR: $\model{RFIR}$.
\end{itemize}
Then, the results of the latter three initialization techniques are compared to the results with $\model{0}$ in Section IV (simulations) and Section V (measurements).

\begin{quote}
6. I am not sure if section IV is actually necessary. You may drop it out. 
\end{quote}

The presence of section IV is crucial for the article, and will not be skipped, for the following reason. 
The goal of the article is to demonstrate that the use of smoothing techniques increases the success rate the minimization of the ML cost function. 
While Sections II and III introduce the methods used, Section IV actually \emph{does} the demonstration of the improvement. 
It is important to include the simulation results, because these allow to make Monte Carlo simulations at various SNR values, to obtain statistically meaningful results. 
It would be much more difficult to apply a Monte Carlo method to the measurements, since a huge number of experiments must be performed, with different experimental conditions (SNR), on a variety of different systems.

\begin{quote}
2) It is unclear how generalizable the procedure introduced by Authors is, and what is its domain of applicability. Based on section IV Demonstration (simulated data), a reader may be convinced that RFIR is the absolute winner.

So, it comes as an unpleasant surprise in section V Experimental Results to learn that RFIR only wins in $50\%$ of cases (Table I). The natural question of ``why so different'' is never discussed. It seems that a lack of ground truth in a real experiment is correctly addressed by using a response from an almost noiseless system (60dB, p.18). Could it be that in simulation two low-pass filters were investigated (each with $n_{\theta} = 5$ parameters, p.13), while in experiment pass-band filter was tested with $n_{\theta} = 9$ adjustable parameters? Having the same system tested in simulations and in experiment could partially answer this question. 

However, the results in Table I do not necessarily nullify a usefulness of the RFIR as a starting point. It may be that in the cases when $\model{RFIR}$ is not the best estimate, a difference between $\model{best}$ and $\model{RFIR}$ is acceptably small. If for cases when $\model{best} = \model{RFIR}$ is much better than the second best estimate $G_{2nd}$, then it makes sense to always use RFIR, as a penalty for wrong choice would be mild. 

\end{quote}

The fact that ‘only’ $50\%$ of the RFIR estimates turn out to be the best estimates does not mean that the remaining $50\%$ from RFIR are bad estimates. 
It simply means that, for the remaining $50\%$, the other methods (trunc and exist) happened to give a slightly smaller cost function. 
Since Table I is misleading, we have decided to remove it from the paper.

In \figref{fig:overview} we can compare $\model{2nd}$ and $\model{best}$.
By inspecting the different estimates for  each of the repeated experiments (\figref{fig:overview}), it can easily be seen that choosing $\model{RFIR}$ while it is not $\model{best}$ only leads to a very modest performance degradation.
In \figref{fig:histogram}, the different methods are compared in this respect by subtracting the performance of the $\model{best}$ from the performance of the respective methods.
From that figure it can be seen that there is little performance to gain \emph{or} lose from choosing $\model{RFIR}$ over $\model{best}$ when RFIR is not the `best' method (as measured by the cost function $V$).
The outcome is a lot less favorable for the alternative methods.

\begin{figure}
  \centering
  \setlength{\figurewidth}{0.85\columnwidth}
  \setlength{\figureheight}{0.68\figurewidth}
  % This file was created by matlab2tikz.
%
%workdir  : /Users/egon/Dropbox/VUB/PhD/Mikaya/initialvalues
%stack    : ../../../../../../../../Users/egon/Dropbox/VUB/PhD/Mikaya/initialvalues/processAllMeasurements (10)
%git info : commit 45c2d89a4e4044db9de2834c5c1c86bec69e4864
%           Author: Egon Geerardyn <egon.geerardyn@gmail.com>
%           Date:   Tue Oct 7 09:18:45 2014 +0200
%           
%               estimParamMeas: duplicate code removed
%           
%            estimateParametricMeas.m | 9 +--------
%            1 file changed, 1 insertion(+), 8 deletions(-)
%           
%
%
\begin{tikzpicture}
\begin{axis}[%
name=fit,
width=\figurewidth,
height=0.5\figureheight,
scale only axis,
xmin=1,
xmax=100,
xmajorgrids,
xtick={15,23,41,53,69,88,100},
xticklabels={},
ymode=log,
ymin=4000,
ymax=100000,
yminorticks=true,
ylabel={Cost $V(\model{\bullet})$},
ymajorgrids,
yminorgrids,
axis x line*=bottom,
axis y line*=left,
]
\addplot [best,line width=1pt] table[]{\thisDir/data/overview-cost-val/cost-best.tsv};
\label{leg:best}
% \addlegendentry{$\hat{G}_{best}$};

\addplot [RFIR,RFIRmark,only marks] table[]{\thisDir/data/overview-cost-val/cost-rfir.tsv};
\label{leg:RFIR}
% \addlegendentry{$\hat{G}_{RFIR}$};

\addplot [LPMTrunc,LPMTruncInitmark,only marks] table[]{\thisDir/data/overview-cost-val/cost-trunc.tsv};
\label{leg:trunc}
% \addlegendentry{$\hat{G}_{trunc}$};

\addplot [existing,existingInitmark,only marks] table[]{\thisDir/data/overview-cost-val/cost-exist.tsv};
\label{leg:exist}
% \addlegendentry{$\hat{G}_{existing}$};

\addplot [G0Hat,G0Hatmark,only marks,medsmallmarkers] table[]{\thisDir/data/overview-cost-val/cost-true.tsv};
% \addlegendentry{$\hat{G}_0$};
\label{leg:VXI}

\end{axis}


\begin{axis}[%
name=valid,
width=\figurewidth,
height=0.5\figureheight,
anchor=above north west,
at={(fit.below south west)},
scale only axis,
xmin=1,
xmax=100,
xmajorgrids,
xtick={15,23,41,53,69,88,100},
ymode=log,
ymin=0.4,
ymax=90,
yminorticks=true,
xlabel={Repetition},
ylabel={$\norm[2]{\model{\bullet} - \model{VXI}}$},
ymajorgrids,
yminorgrids,
axis x line*=bottom,
axis y line*=left
]
\addplot [dashed] table[]{\thisDir/data/overview-cost-val/sys-h2norm.tsv};
\label{leg:reference}
\addplot [best,line width=1pt] table[]{\thisDir/data/overview-cost-val/distance-best.tsv};
\addplot [RFIR,RFIRmark,only marks] table[]{\thisDir/data/overview-cost-val/distance-rfir.tsv};
\addplot [LPMTrunc,LPMTruncInitmark,only marks] table[]{\thisDir/data/overview-cost-val/distance-trunc.tsv};
\addplot [existing,existingInitmark,only marks] table[]{\thisDir/data/overview-cost-val/distance-exist.tsv};
\addplot [G0Hat,G0Hatmark,only marks,medsmallmarkers] table[]{\thisDir/data/overview-cost-val/distance-true.tsv};
\end{axis}
\end{tikzpicture}%

  \caption{Cost function $V(\model{\bullet})$ and validation distance of the model estimated in each repetition of the measurement.
  The same data as in Fig.~10 and Fig.~11 of the paper is presented.
  Based on the cost function, $\model{RFIR}$~\legref{leg:RFIR} is not always the best estimate $\model{best}$~\legref{leg:best}.
  Especially in the validation plot (bottom), $\model{RFIR}$ performs (almost) as well or even better than $\model{exist}$~\legref{leg:exist} and $\model{trunc}$~\legref{leg:trunc}.
  Both latter methods often produce models that perform poorly compared to $\| G_{VXI} \|$~\legref{leg:reference} and $\model{VXI}$~\legref{leg:VXI}.
  This means that in our limited experimental study, the performance degradation of choosing RFIR to produce initial values over the studied alternatives is negligible.}
  \label{fig:overview}
\end{figure}

\begin{figure}
  \centering
  \setlength{\figurewidth}{0.75\columnwidth}
  \setlength{\figureheight}{0.60\figurewidth}
  % This file was created by matlab2tikz.
%
%workdir  : /Users/egon/Dropbox/VUB/PhD/Mikaya/initialvalues
%stack    : ../../../../../../../../Users/egon/Dropbox/VUB/PhD/Mikaya/initialvalues/processAllMeasurements (10)
%git info : commit 45c2d89a4e4044db9de2834c5c1c86bec69e4864
%           Author: Egon Geerardyn <egon.geerardyn@gmail.com>
%           Date:   Tue Oct 7 09:18:45 2014 +0200
%
%               estimParamMeas: duplicate code removed
%
%            estimateParametricMeas.m | 9 +--------
%            1 file changed, 1 insertion(+), 8 deletions(-)
%
%
%
\begin{tikzpicture}
\begin{axis}[%
name=Estim,
width=0.5\figurewidth,
height=\figureheight,
scale only axis,
unbounded coords=jump,
xmin=0.5,
xmax=3.5,
grid=both,
title={\footnotesize{Estimation: $V_{\bullet} = V(\model{\bullet})$}},
x tick label as interval,
xtick={0.5,1.5,2.5,3.5},
xticklabels={{$\model{exist}$},{$\model{trunc}$},{$\model{RFIR}$}},
ymode=log,
ymin=0.001,
ymax=100000,
yminorticks=true,
ylabel={$V_{\bullet} - V_{\mathrm{best}}$},
axis background/.style={fill=white},
axis x line*=bottom,
axis y line*=left
]

\addplot[degradationBox, forget plot] table[row sep=crcr] {%
x	y\\
0.625	9393.62\\
1.375	9393.62\\
1.375	3526.08\\
0.625	3526.08\\
}--cycle;
\addplot [degradationMedian, forget plot] table[row sep=crcr]{%
1.375	5011.06\\
0.625	5011.06\\
};
\addplot [degradationMarks, forget plot] table[row sep=crcr]{%
0.922277	0.033605\\
1.13317	0.190822\\
1.0527	4.06321\\
1.14798	4.82057\\
0.942883	5.48263\\
0.925083	12.455\\
1.02548	12.7229\\
0.967593	13.9686\\
1.11363	16.3418\\
1.0346	21.2306\\
0.908182	22.5912\\
0.857845	216.493\\
0.984566	1553.31\\
1.04435	2534.01\\
1.15773	2562.14\\
1.0265	2627.87\\
1.06589	2871.46\\
1.06832	3181.67\\
0.875817	3526.08\\
1.10459	3553.44\\
1.14669	3580.93\\
0.927736	3860.74\\
0.924988	3867.56\\
0.991749	3886.67\\
0.866629	3964.33\\
0.876369	4061.22\\
1.13007	4143.91\\
0.997072	4249.59\\
0.924052	4305.99\\
0.935867	4314.09\\
0.991615	4318.47\\
1.06812	4352.7\\
0.89015	4395.58\\
1.14381	4476.02\\
0.953398	4529.72\\
0.957848	4939.79\\
0.99829	5006.96\\
1.09277	5015.16\\
1.08029	5243.86\\
0.912788	5324.21\\
1.0032	5538.86\\
0.988741	5805.83\\
1.05155	6032.33\\
1.05756	6141.9\\
1.09436	6317.17\\
0.844383	6475.02\\
0.901184	6514.9\\
1.1361	7628.9\\
1.01268	7780.28\\
1.10057	7837.68\\
1.08409	7868.77\\
1.05339	8678.66\\
1.01529	8862.45\\
1.08974	9064.78\\
0.949984	9141.84\\
0.900703	9393.62\\
1.00464	9506.79\\
1.00903	9575.88\\
1.0529	9652.71\\
1.11381	10254.4\\
1.0029	10385\\
1.15552	10403.7\\
1.03435	10711.7\\
0.940191	10787.2\\
1.00074	11304\\
0.909025	12337.9\\
1.00862	12346.2\\
1.01205	12684.5\\
0.937867	13376.8\\
1.10252	13967.7\\
1.05978	14435.5\\
0.858345	15783\\
0.892796	16345.6\\
0.899061	23972.7\\
};
\addplot[degradationBox, forget plot] table[row sep=crcr] {%
x	y\\
1.625	4985.6\\
2.375	4985.6\\
2.375	36.1076\\
1.625	36.1076\\
}--cycle;
\addplot [degradationMedian, forget plot] table[row sep=crcr]{%
2.375	230.098\\
1.625	230.098\\
};
\addplot [degradationMarks, forget plot] table[row sep=crcr]{%
2.07081	0.00208325\\
2.06555	0.0429928\\
2.08573	0.080705\\
2.06684	0.132026\\
1.83887	3.95937\\
2.10837	4.19375\\
2.0362	5.12476\\
2.14417	7.39222\\
1.90572	9.39305\\
2.09681	11.0485\\
2.12762	12.787\\
2.09265	17.1415\\
2.03467	17.3682\\
2.02207	18.894\\
2.04932	22.024\\
1.87406	22.915\\
1.90604	27.123\\
1.95031	29.5564\\
2.01471	38.2913\\
1.85415	39.4592\\
1.87189	45.6268\\
1.95267	50.4529\\
1.85917	54.2021\\
1.96901	55.0901\\
1.95223	67.0147\\
2.14927	71.4587\\
2.08156	76.2027\\
1.8726	76.8336\\
2.00315	94.798\\
2.02529	97.2349\\
1.94441	129.464\\
1.94249	139.836\\
1.87556	141.836\\
2.0833	147.277\\
1.93196	149.683\\
1.85233	156.424\\
1.93233	230.098\\
2.11591	289.831\\
2.14784	320.596\\
2.01355	938.98\\
1.84854	1131.67\\
2.13748	1187.54\\
2.09318	2499.61\\
1.96814	2933.79\\
2.04595	3224.42\\
2.02493	3426.69\\
2.0904	3605.18\\
1.93503	3618.9\\
1.94672	3848.38\\
2.03591	4136.4\\
1.9253	4220.87\\
1.84069	4462.4\\
2.01157	4627.27\\
1.90929	4891.11\\
2.01638	4910.46\\
1.9732	5211.02\\
2.0156	5534.71\\
2.00222	5982.69\\
1.99062	6832.38\\
2.15292	7025.27\\
1.92599	7082.94\\
1.86105	7552.33\\
1.92043	7758.79\\
2.15554	7771.62\\
2.10393	8062.32\\
1.90238	8521.9\\
1.85906	9205.49\\
1.94386	9959.45\\
2.11268	10043.5\\
2.15744	10385\\
1.88701	10829.7\\
2.03905	14012.5\\
2.0974	83868.8\\
};

\addplot[degradationBox, forget plot] table[row sep=crcr] {%
x	y\\
2.625	76.7968\\
3.375	76.7968\\
3.375	13.4679\\
2.625	13.4679\\
}--cycle;
\addplot [degradationMedian, forget plot] table[row sep=crcr]{%
3.375	27.8619\\
2.625	27.8619\\
};
\addplot [degradationMarks, forget plot] table[row sep=crcr]{%
3.15729	3.25727\\
2.90997	3.28234\\
3.10705	3.68391\\
3.04718	6.20922\\
2.96872	6.22396\\
2.88322	9.0092\\
2.85018	9.03611\\
2.88393	9.34701\\
2.88649	9.68312\\
2.87276	11.0841\\
2.98003	11.9244\\
2.94129	13.1149\\
3.08462	13.1685\\
2.91207	13.5677\\
2.91651	15.0955\\
3.10151	15.1192\\
3.00764	15.2633\\
3.15085	15.3574\\
3.0555	18.3496\\
3.06533	18.7947\\
2.95795	18.9305\\
3.14119	19.4318\\
3.12683	19.8344\\
3.11083	20.1674\\
3.08737	23.413\\
3.0081	27.4831\\
3.13914	27.8619\\
2.87018	28.152\\
3.12764	29.9847\\
2.94934	31.6225\\
3.1446	32.5802\\
3.142	38.8809\\
3.01376	41.6544\\
2.84522	43.8465\\
3.16307	53.9795\\
3.15044	54.1607\\
2.92824	64.9624\\
3.09067	65.7946\\
3.11931	72.972\\
3.1246	73.155\\
2.94859	87.7222\\
3.08119	105.747\\
3.15014	109.139\\
2.99794	119.826\\
2.94498	173.778\\
2.8475	202.868\\
3.10686	207.263\\
2.95689	210.604\\
2.88951	251.989\\
2.89472	386.196\\
3.08223	696.356\\
2.89102	988.211\\
2.9141	3093.3\\
};
\addplot [percentileLines, forget plot] table[row sep=crcr]{%
0	88317.9\\
4	88317.9\\
};
\addplot [percentileLines, forget plot] table[row sep=crcr]{%
0	4538.94\\
4	4538.94\\
};
\addplot [percentileLines, forget plot] table[row sep=crcr]{%
0	4329\\
4	4329\\
};
\end{axis}

\begin{axis}[%
name=Valid,
anchor=north west,
at={(Estim.right of north east)},
xshift=1em,
width=0.5\figurewidth,
height=\figureheight,
scale only axis,
unbounded coords=jump,
xmin=0.5,
xmax=3.5,
grid=both,
x tick label as interval,
xtick={0.5,1.5,2.5,3.5},
xticklabels={{$\model{exist}$},{$\model{trunc}$},{$\model{RFIR}$}},
ymode=log,
ymin=0.0001,
ymax=100,
yminorticks=true,
ylabel={${D_{\bullet} - D}_{\mathrm{best}}$},
title={\footnotesize Validation: $D_{\bullet} = \| \model{\bullet} - G_{\mathrm{VXI}} \|_2$},
axis x line*=bottom,
axis y line*=right,
legend to name=leg:init:secondBest,
legend columns=-1,
legend cell align=center,
legend style={draw=none}
]

\addplot[enhancementBox, forget plot] table[row sep=crcr] {%
x	y\\
0.625	0.104862\\
1.375	0.104862\\
1.375	0.011776\\
0.625	0.011776\\
}--cycle;
\addplot [enhancementMedian, forget plot] table[row sep=crcr]{%
1.375	0.0323978\\
0.625	0.0323978\\
};
\addplot [enhancementMarks, forget plot] table[row sep=crcr]{%
1.0493	0.000948768\\
1.15562	0.0226032\\
0.910995	0.0421923\\
0.937528	0.167532\\
};

\addplot[enhancementBox, forget plot] table[row sep=crcr] {%
x	y\\
1.625	0.154826\\
2.375	0.154826\\
2.375	0.0763623\\
1.625	0.0763623\\
}--cycle;
\addplot [enhancementMedian, forget plot] table[row sep=crcr]{%
2.375	0.111799\\
1.625	0.111799\\
};
\addplot [enhancementMarks, forget plot] table[row sep=crcr]{%
1.91483	0.00159476\\
2.08192	0.0430308\\
1.8504	0.0763623\\
1.93836	0.0813026\\
1.96155	0.0948962\\
1.86118	0.128702\\
2.05686	0.143308\\
2.09727	0.154826\\
2.03958	0.268334\\
2.16016	13.0713\\
};

\addplot[enhancementBox] table[row sep=crcr] {%
x	y\\
2.875	0.361718\\
3.375	0.361718\\
3.375	0.078234\\
2.875	0.078234\\
}--cycle;
%\addlegendentry{boxplot $X_{\bullet} < X_{\mathrm{best}}$}
\addlegendentry{and}

\addplot [enhancementMedian, forget plot] table[row sep=crcr]{%
3.375	0.177147\\
2.875	0.177147\\
};
% \addlegendentry{$X_{\bullet} < X_{\mathrm{best}}$}

\addplot [enhancementMarks] table[row sep=crcr]{%
3.08631	0.000908397\\
2.95689	0.0240998\\
2.89041	0.038931\\
3.01417	0.0500605\\
2.92749	0.052977\\
2.92303	0.0665624\\
3.10365	0.0899057\\
3.01096	0.119278\\
2.97867	0.128891\\
3.14069	0.132046\\
3.11793	0.1615\\
3.09935	0.171268\\
2.9993	0.183025\\
3.14567	0.213321\\
2.9509	0.252252\\
2.86419	0.273097\\
3.05034	0.284965\\
2.95392	0.333224\\
3.0083	0.390213\\
3.05455	0.418286\\
2.97926	0.43449\\
3.07361	0.969427\\
2.90533	1.59776\\
2.95782	13.4128\\
};
\addlegendentry{$X_{\bullet} < X_{\mathrm{best}}$}

\addlegendimage{empty legend}
\addlegendentry{\vspace{2em}}

\addplot[degradationBox, forget plot] table[row sep=crcr] {%
x	y\\
0.625	21.1252\\
1.375	21.1252\\
1.375	11.6063\\
0.625	11.6063\\
}--cycle;
\addplot [degradationMedian, forget plot] table[row sep=crcr]{%
1.375	12.9791\\
0.625	12.9791\\
};
\addplot [degradationMarks, forget plot] table[row sep=crcr]{%
0.887198	0.13575\\
0.883733	0.195773\\
0.880203	0.215907\\
1.11122	0.258142\\
0.936531	0.31366\\
1.00937	0.330849\\
1.11384	0.386375\\
1.02971	2.14982\\
0.957671	7.38296\\
0.953742	9.22212\\
1.06507	9.47533\\
0.860294	9.92974\\
0.949693	10.2786\\
1.03298	10.5201\\
0.885001	10.806\\
1.12473	10.9766\\
0.98867	11.092\\
0.926371	11.6063\\
0.980495	11.6821\\
1.10848	11.7794\\
0.953962	11.805\\
1.00172	11.8383\\
0.987888	11.9813\\
1.14107	12.0191\\
1.06703	12.0845\\
0.88436	12.2874\\
1.1519	12.3576\\
0.906502	12.5381\\
1.10127	12.5504\\
0.862838	12.5799\\
1.09837	12.7479\\
0.989981	12.8807\\
1.11409	12.8833\\
1.09563	12.9343\\
0.955771	12.9448\\
0.963631	13.0135\\
0.987283	13.0217\\
0.848556	13.0522\\
1.00052	13.5716\\
1.15372	13.904\\
1.13831	14.0275\\
1.04045	14.6608\\
0.892037	14.993\\
1.10545	15.1419\\
0.873603	15.7625\\
1.11061	17.4435\\
1.06485	19.2345\\
1.04837	19.3425\\
0.996781	20.1077\\
1.01052	20.117\\
1.01056	20.2522\\
1.03728	20.5913\\
1.13444	21.1252\\
0.849343	21.4307\\
1.02902	22.4413\\
1.16309	22.4416\\
0.968188	22.5387\\
1.14128	23.0535\\
0.865202	23.1932\\
0.904096	23.3183\\
0.854033	23.3395\\
0.838013	23.5941\\
1.14552	23.6036\\
0.910712	23.6388\\
0.949658	23.8141\\
0.992471	23.8201\\
1.02439	23.9607\\
1.04429	24.0795\\
0.999157	24.5959\\
1.05687	24.8603\\
};

\addplot[degradationBox, forget plot] table[row sep=crcr]{%
x	y\\
1.625	14.1311\\
2.375	14.1311\\
2.375	0.530686\\
1.625	0.530686\\
}--cycle;
\addplot [degradationMedian, forget plot] table[row sep=crcr]{%
2.375	6.81802\\
1.625	6.81802\\
};
\addplot [degradationMarks, forget plot] table[row sep=crcr]{%
2.01848	0.000251818\\
2.06215	0.000558771\\
2.05682	0.00161405\\
2.04724	0.0710554\\
1.88924	0.091754\\
1.96946	0.114643\\
1.9221	0.281593\\
1.92777	0.311959\\
1.86084	0.341869\\
2.03453	0.370976\\
2.09148	0.409017\\
2.10186	0.431962\\
2.08185	0.45912\\
1.96165	0.493978\\
1.84934	0.512762\\
1.95035	0.517082\\
1.90033	0.571496\\
2.09091	0.608902\\
2.13221	0.681324\\
2.08043	0.742926\\
2.03626	0.757944\\
1.91526	0.880168\\
1.86727	0.913659\\
2.12971	1.01118\\
1.94887	1.01684\\
1.87384	1.07951\\
1.96749	1.10944\\
2.00882	1.12209\\
2.01012	1.13445\\
2.13439	1.92016\\
1.936	6.50271\\
2.09635	6.81802\\
2.07263	9.69644\\
1.89501	10.3565\\
1.9068	10.6324\\
2.00262	10.9362\\
2.16252	11.2214\\
1.97823	11.3713\\
2.05275	11.8515\\
2.11201	11.9711\\
1.85074	12.4174\\
2.08119	12.4272\\
2.00732	12.628\\
2.095	12.6522\\
1.8855	12.6754\\
1.8647	13.6112\\
1.96105	13.974\\
2.10867	14.1834\\
1.8712	14.6479\\
2.08998	14.7139\\
1.90906	15.4034\\
1.99554	15.7126\\
1.87366	15.819\\
2.03901	16.0831\\
2.00778	16.199\\
1.91965	17.9733\\
2.16176	18.8728\\
1.91335	18.9788\\
1.84056	19.1298\\
1.87118	21.0041\\
1.89027	21.6238\\
1.9057	23.8199\\
2.14712	61.7874\\
};

\addplot[degradationBox] table[row sep=crcr] {%
x	y\\
2.625	0.332278\\
3.125	0.332278\\
3.125	0.107788\\
2.625	0.107788\\
}--cycle;
% \addlegendentry{boxplot $X_{\bullet} > X_{\mathrm{best}}$}
\addlegendentry{and}

\addplot [degradationMedian, forget plot]  table[row sep=crcr]{%
3.125	0.182529\\
2.625	0.182529\\
};
% \addlegendentry{$X_{\bullet} > X_{\mathrm{best}}$}

\addplot [degradationMarks]  table[row sep=crcr]{%
3.04706	0.00149983\\
2.8864	0.0301416\\
2.9612	0.0318119\\
3.12009	0.0336713\\
3.09039	0.0411459\\
3.13232	0.0839664\\
3.02877	0.100556\\
3.04978	0.110198\\
3.04547	0.111012\\
2.95122	0.118648\\
3.12903	0.136062\\
3.15593	0.14847\\
2.86822	0.158895\\
2.90584	0.158988\\
2.9272	0.182529\\
3.14742	0.184652\\
3.04926	0.200623\\
3.08482	0.202014\\
3.15389	0.233636\\
3.06578	0.281606\\
3.06513	0.311427\\
3.12906	0.325348\\
2.95706	0.353069\\
3.09003	0.405457\\
2.89513	0.422351\\
3.07449	0.426608\\
2.8799	0.470346\\
3.01656	0.588477\\
3.12259	0.837442\\
};
\addlegendentry{$X_{\bullet} > X_{\mathrm{best}}$}

\addlegendimage{empty legend}
\addlegendentry{\vspace{2em}}

\addplot [percentileLines, forget plot]  table[row sep=crcr]{%
0	63.8272\\
4	63.8272\\
};
\addplot [percentileLines, forget plot]  table[row sep=crcr]{%
0	1.94482\\
4	1.94482\\
};
\addplot [percentileLines]   table[row sep=crcr]{%
0	1.21778\\
4	1.21778\\
};
\addlegendentry{$\min, \mathrm{median}, \max \; X_{\bullet}$}

\end{axis}
\end{tikzpicture}%

  \caption{
  Performance degradation/enhancement ($X_{\bullet}-X_{\mathrm{best}}$) for choosing a given method $\bullet$ instead of the `best' method for the measurement example in the paper.
  In red/orange the positive values ($X_{\bullet} > X_{\mathrm{best}}$), the performance degradation for using $\model{\bullet}$ over $\model{best}$ is shown.
  The negative values ($X_{\bullet} < X_{\mathrm{best}}$), i.e. the performance enhancement (for using $\model{\bullet}$ instead of $\model{best}$) is shown in green/blue.
  Zero values (i.e. $X_{\bullet} = X_{\mathrm{best}}$) are not shown, hence the data shows when method $\bullet$ is \emph{not} selected as the `best' method.
  The dashed lines are the minimum, median and maximum of $X_{\bullet}$ to give a sense of scale.
  The performance degradation of choosing $\model{RFIR}$ instead of $\model{best}$ is smaller than the corresponding performance degradation for the other methods.
  The performance enhancement from choosing $\model{RFIR}$ over $\model{best}$, is almost on par with the corresponding performance degradation (i.e. there is little to gain \emph{or} lose from choosing $\model{RFIR}$ over $\model{best}$).
  For $\model{trunc}$ and $\model{exist}$, there is much less performance to gain from picking those over $\model{best}$ and there is a lot more to lose.
  }
  \label{fig:histogram}
\end{figure}

\begin{quote}

The use of a fixed method would greatly simplify an algorithm design in any standalone autonomous toolbox. The results in Table II do not help to uncover whether such a strategy is permissible. To solve this dilemma, two histograms of differences $\Delta = \model{best} - \model{2nd}$ need to be built and shown: one for $\model{best} = \model{FRIR}$ and the other for $\model{best} \neq \model{FRIR}$. 
\end{quote}

We agree that a single method would greatly simplify the implementation of a toolbox.
However, we think that using multiple initial value strategies is a valuable approach since this makes a toolbox more robust to different conditions (SNR, frequency resolution, system order, \ldots) as it reduces the likelihood that a local optimum is hit.
This can e.g. be seen in subplot 2 of \figref{fig:distancesStress} (which is discussed further on).
A similar approach w.r.t. to combine initial values is followed in some toolboxes, e.g. \cite{FDIDENT,TDIDENT}.

Many nature-inspired optimization algorithms (such as Particle Swarm Optimization (PSO)) and stochastic optimization algorithms even use hundreds or thousands of randomized initial values, in part to avoid local optima.
But, obviously, using many initial values comes at a (considerable) increase in computational cost.

While this discussion and this inherent trade-off is very interesting, we think this is outside of the scope of the paper where we only wish to demonstrate that smoothing techniques such as RFIR are a worthwhile source of initial values, but not necessarily the only one.

\begin{quote}
The abstract promises a better starting point for a minimization of a non-convex cost function. However, no effort is undertaken in the paper to show that improvement is expected ``in most cases'' (or, under which conditions). The boundaries between different domains of starting points leading to different local minima may be very complicated. In fact, the higher the dimension of a search space is ($n_{\theta}$), the more likely it is that the boundaries are more complicated. 

It is obvious that in some cases where subsets of starting points stretch over the boundaries, a particular strategy for selecting a starting point may lead to improvement. However, the paper doesn't explain why this should happen ``in most cases''. A reader would gain more confidence in a robustness of the introduced method if its boundary of applicability were clearly spelled out (for example: the method applicable to filters of order less than ... based on extensive testing). 
\end{quote}

We agree that the complicated boundaries between attraction regions of the different local optima of a cost function makes this a hard problem.
Moreover, the intricate relationship between the actual pole locations, signal SNR, choice of excitation signal, \ldots and shape of the cost function makes that determining clear boundaries on the usability of a starting value seems intractable to us.

On the one hand, an obvious limitation to the presented initialization techniques is that they share the limitations of the smoothing techniques: if the non-parametric estimate is a worse representation than the raw data (e.g. heavily biased), it is very unlikely that the initial estimate will outperform the existing estimates.
In particular for RFIR, a simulation study in~\cite{Chen2013} suggests that RFIR handles systems with model orders up to at least $30$, even for small datasets $N\leq 500$.
For LPM (with or without time-truncation), no comparable studies are available to our knowledge, but the LPM itself has been used successfully in diverse practical applications.

For the parametric estimates, we ran some additional simulations to verify the usability of the proposed initial values.
In particular, we used
\begin{itemize}
  \item low-pass, high-pass, band-pass and band-stop 
  \item Chebyshev Type I, Chebyshev Type II, Elliptical and Butterworth
\end{itemize}
discrete-time filters (see \figref{fig:bodeplots}) of tenth degree generated by the MATLAB functions \texttt{cheby1}, \texttt{cheby2}, \texttt{ellip} and \texttt{butter} and an SNR of $20 \unit{dB}$ and $N=1024$ measured samples.
Note: the band-pass and band-stop filters actually have a model order of 20.

The input excitation and the disturbing output noise were both white and Gaussian.
The results of a Monte Carlo simulation consisting of $100$ runs (as described above) have been summarized qualitatively in the paper (due to limited space) and are discussed below in more detail.

\TODO{add table with different transfer functions, generation code, etc. for stress test}

\begin{figure}[p]
  \setlength{\figurewidth}{0.75\columnwidth}
  \setlength{\figureheight}{0.6\figurewidth}
  \centering
  %\ref{legBodes}
  % This file was created by matlab2tikz.
%
%workdir  : /Users/egon/Dropbox/VUB/PhD/Mikaya/initialvalues
%stack    : ../../../../../../../../Users/egon/Dropbox/VUB/PhD/Mikaya/initialvalues/testMonteCarloSystemsSmall (10)
%git info : commit 7bee89de2a914143a08c391cf0cbb846b1db5777
%           Author: Egon Geerardyn <egon.geerardyn@gmail.com>
%           Date:   Mon Feb 2 18:15:16 2015 +0100
%           
%               lo SNR stress test with Cheby
%           
%            printSummary.m               | 10 +++++----
%            summary.m                    |  2 ++
%            testMonteCarloSystemsSmall.m | 51 ++++++++++++++++++++++----------------------
%            3 files changed, 34 insertions(+), 29 deletions(-)
%           
%
%
%

\tikzset{plotnum/.style={color=black!30,align=left,font=\scriptsize,anchor=north west}}
\pgfplotsset{cheby1/.style={ymin=-80,ymax=10,ytick={0,-60},extra y ticks={10},extra y tick labels={}}}
\pgfplotsset{cheby2/.style={ymin=-60,ymax=5,ytick={0,-40},extra y ticks={5},extra y tick labels={}}}
\pgfplotsset{ellip/.style={ymin=-80,ymax=10,ytick={0,-60},extra y ticks={10},extra y tick labels={}}}
\pgfplotsset{butter/.style={ymin=-80,ymax=15,ytick={0,-60},extra y ticks={15},extra y tick labels={}}}

\begin{tikzpicture}

\pgfplotsset{every axis legend/.style={anchor=west}}

\begin{groupplot}[%
group style={%
  group name=bodes,
  group size=4 by 4,
  horizontal sep=0.2cm,
  vertical sep=0.2cm,
  x descriptions at=edge bottom,
  y descriptions at=edge left},
xlabel={$\omega \axisunit{rad/s}$},
scale only axis,
xmin=0,
xmax=3.1415,
xtick={0,1.2566,1.5708,1.885,3.1416},
xticklabels={{},{$\frac{2\pi}{5}$},{},{$\frac{3\pi}{5}$},{}},
grid=major,
ylabel={$|G| \axisunit{dB}$},
height=0.25\figureheight,
width=0.25\figurewidth,
axis x line*=bottom,
axis y line*=left]

% ================================================
\nextgroupplot[cheby1, legend columns=-1,
  legend cell align=left, legend style={at={(rel axis cs:0.5,1.05)}, anchor=south west}]
\node[plotnum,anchor=south west] at (rel axis cs:0,0) {1};
\addplot [color=G0Hat,solid,forget plot]
  table{\thisDir/fig/data/stress-frf/bodeplots-1.tsv};
\addplot [color=G0Hat,solid]
  table{\thisDir/fig/data/stress-frf/bodeplots-2.tsv};
\addlegendentry{\model{\trueSymbol}};
\label{leg:bode:model0}

\addplot [color=existing,solid,forget plot]
  table{\thisDir/fig/data/stress-frf/bodeplots-3.tsv};
\addplot [color=existing,solid]
  table{\thisDir/fig/data/stress-frf/bodeplots-4.tsv};
\addlegendentry{\model{exist}};
\label{leg:bode:modelExisting}

\addplot [color=LPMTrunc,solid,forget plot]
  table{\thisDir/fig/data/stress-frf/bodeplots-5.tsv};
\addplot [color=LPMTrunc,solid]
  table{\thisDir/fig/data/stress-frf/bodeplots-6.tsv};
  \addlegendentry{\model{trunc}};
\label{leg:bode:modelTrunc}

\addplot [color=RFIR,solid,forget plot]
  table{\thisDir/fig/data/stress-frf/bodeplots-7.tsv};
\addplot [color=RFIR,solid]
  table{\thisDir/fig/data/stress-frf/bodeplots-8.tsv};
\addlegendentry{\model{RFIR}};
\label{leg:bode:model:RFIR}

% \addplot [color=best,dotted]
%   table{\thisDir/fig/data/stress-frf/bodeplots-9.tsv};
% \addplot [color=best,dotted]
%   table{\thisDir/fig/data/stress-frf/bodeplots-10.tsv};
% \addlegendentry{\model{best}};
% \label{leg:bode:best}
\addplot [exact]
  table{\thisDir/fig/data/stress-frf/bodeplots-11.tsv};
\addlegendentry{$\true{G}$};
\label{leg:bode:true}


% ================================================
\nextgroupplot[cheby1]
\node[plotnum,anchor=south] at (rel axis cs:0.5,0) {2};
\addplot [color=G0Hat,solid]
  table{\thisDir/fig/data/stress-frf/bodeplots-12.tsv};
\addplot [color=G0Hat,solid]
  table{\thisDir/fig/data/stress-frf/bodeplots-13.tsv};
\addplot [color=existing,solid]
  table{\thisDir/fig/data/stress-frf/bodeplots-14.tsv};
\addplot [color=existing,solid]
  table{\thisDir/fig/data/stress-frf/bodeplots-15.tsv};
\addplot [color=LPMTrunc,solid]
  table{\thisDir/fig/data/stress-frf/bodeplots-16.tsv};
\addplot [color=LPMTrunc,solid]
  table{\thisDir/fig/data/stress-frf/bodeplots-17.tsv};
\addplot [color=RFIR,solid]
  table{\thisDir/fig/data/stress-frf/bodeplots-18.tsv};
\addplot [color=RFIR,solid]
  table{\thisDir/fig/data/stress-frf/bodeplots-19.tsv};
% \addplot [color=best,dotted]
%   table{\thisDir/fig/data/stress-frf/bodeplots-20.tsv};
% \addplot [color=best,dotted]
%   table{\thisDir/fig/data/stress-frf/bodeplots-21.tsv};
\addplot [exact]
  table{\thisDir/fig/data/stress-frf/bodeplots-22.tsv};

% ================================================
\nextgroupplot[cheby1]
\node[plotnum,anchor=south east] at (rel axis cs:1,0) {3};
\addplot [color=G0Hat,solid]
  table{\thisDir/fig/data/stress-frf/bodeplots-23.tsv};
\addplot [color=G0Hat,solid]
  table{\thisDir/fig/data/stress-frf/bodeplots-24.tsv};
\addplot [color=existing,solid]
  table{\thisDir/fig/data/stress-frf/bodeplots-25.tsv};
\addplot [color=existing,solid]
  table{\thisDir/fig/data/stress-frf/bodeplots-26.tsv};
\addplot [color=LPMTrunc,solid]
  table{\thisDir/fig/data/stress-frf/bodeplots-27.tsv};
\addplot [color=LPMTrunc,solid]
  table{\thisDir/fig/data/stress-frf/bodeplots-28.tsv};
\addplot [color=RFIR,solid]
  table{\thisDir/fig/data/stress-frf/bodeplots-29.tsv};
\addplot [color=RFIR,solid]
  table{\thisDir/fig/data/stress-frf/bodeplots-30.tsv};
% \addplot [color=best,dotted]
%   table{\thisDir/fig/data/stress-frf/bodeplots-31.tsv};
% \addplot [color=best,dotted]
%   table{\thisDir/fig/data/stress-frf/bodeplots-32.tsv};
\addplot [exact]
  table{\thisDir/fig/data/stress-frf/bodeplots-33.tsv};

% ================================================
\nextgroupplot[cheby1]
\node[plotnum,anchor=south east] at (rel axis cs:1,0) {4};
\addplot [color=G0Hat,solid]
  table{\thisDir/fig/data/stress-frf/bodeplots-34.tsv};
\addplot [color=G0Hat,solid]
  table{\thisDir/fig/data/stress-frf/bodeplots-35.tsv};
\addplot [color=existing,solid]
  table{\thisDir/fig/data/stress-frf/bodeplots-36.tsv};
\addplot [color=existing,solid]
  table{\thisDir/fig/data/stress-frf/bodeplots-37.tsv};
\addplot [color=LPMTrunc,solid]
  table{\thisDir/fig/data/stress-frf/bodeplots-38.tsv};
\addplot [color=LPMTrunc,solid]
  table{\thisDir/fig/data/stress-frf/bodeplots-39.tsv};
\addplot [color=RFIR,solid]
  table{\thisDir/fig/data/stress-frf/bodeplots-40.tsv};
\addplot [color=RFIR,solid]
  table{\thisDir/fig/data/stress-frf/bodeplots-41.tsv};
% \addplot [color=best,dotted]
%   table{\thisDir/fig/data/stress-frf/bodeplots-42.tsv};
% \addplot [color=best,dotted]
%   table{\thisDir/fig/data/stress-frf/bodeplots-43.tsv};
\addplot [exact]
  table{\thisDir/fig/data/stress-frf/bodeplots-44.tsv};

% ================================================
% ================================================
\nextgroupplot[cheby2]
\node[plotnum,anchor=south west] at (rel axis cs:0,0) {5};
\addplot [color=G0Hat,solid]
  table{\thisDir/fig/data/stress-frf/bodeplots-45.tsv};
\addplot [color=G0Hat,solid]
  table{\thisDir/fig/data/stress-frf/bodeplots-46.tsv};
\addplot [color=existing,solid]
  table{\thisDir/fig/data/stress-frf/bodeplots-47.tsv};
\addplot [color=existing,solid]
  table{\thisDir/fig/data/stress-frf/bodeplots-48.tsv};
\addplot [color=LPMTrunc,solid]
  table{\thisDir/fig/data/stress-frf/bodeplots-49.tsv};
\addplot [color=LPMTrunc,solid]
  table{\thisDir/fig/data/stress-frf/bodeplots-50.tsv};
\addplot [color=RFIR,solid]
  table{\thisDir/fig/data/stress-frf/bodeplots-51.tsv};
\addplot [color=RFIR,solid]
  table{\thisDir/fig/data/stress-frf/bodeplots-52.tsv};
% \addplot [color=best,dotted]
%   table{\thisDir/fig/data/stress-frf/bodeplots-53.tsv};
% \addplot [color=best,dotted]
%   table{\thisDir/fig/data/stress-frf/bodeplots-54.tsv};
\addplot [exact]
  table{\thisDir/fig/data/stress-frf/bodeplots-55.tsv};

% ================================================
\nextgroupplot[cheby2]
\node[plotnum,anchor=south] at (rel axis cs:0.5,0) {6};
\addplot [color=G0Hat,solid]
  table{\thisDir/fig/data/stress-frf/bodeplots-56.tsv};
\addplot [color=G0Hat,solid]
  table{\thisDir/fig/data/stress-frf/bodeplots-57.tsv};
\addplot [color=existing,solid]
  table{\thisDir/fig/data/stress-frf/bodeplots-58.tsv};
\addplot [color=existing,solid]
  table{\thisDir/fig/data/stress-frf/bodeplots-59.tsv};
\addplot [color=LPMTrunc,solid]
  table{\thisDir/fig/data/stress-frf/bodeplots-60.tsv};
\addplot [color=LPMTrunc,solid]
  table{\thisDir/fig/data/stress-frf/bodeplots-61.tsv};
\addplot [color=RFIR,solid]
  table{\thisDir/fig/data/stress-frf/bodeplots-62.tsv};
\addplot [color=RFIR,solid]
  table{\thisDir/fig/data/stress-frf/bodeplots-63.tsv};
% \addplot [color=best,dotted]
%   table{\thisDir/fig/data/stress-frf/bodeplots-64.tsv};
% \addplot [color=best,dotted]
%   table{\thisDir/fig/data/stress-frf/bodeplots-65.tsv};
\addplot [exact]
  table{\thisDir/fig/data/stress-frf/bodeplots-66.tsv};

% ================================================
\nextgroupplot[cheby2]
\node[plotnum,anchor=south east] at (rel axis cs:1,0) {7};
\addplot [color=G0Hat,solid]
  table{\thisDir/fig/data/stress-frf/bodeplots-67.tsv};
\addplot [color=G0Hat,solid]
  table{\thisDir/fig/data/stress-frf/bodeplots-68.tsv};
\addplot [color=existing,solid]
  table{\thisDir/fig/data/stress-frf/bodeplots-69.tsv};
\addplot [color=existing,solid]
  table{\thisDir/fig/data/stress-frf/bodeplots-70.tsv};
\addplot [color=LPMTrunc,solid]
  table{\thisDir/fig/data/stress-frf/bodeplots-71.tsv};
\addplot [color=LPMTrunc,solid]
  table{\thisDir/fig/data/stress-frf/bodeplots-72.tsv};
\addplot [color=RFIR,solid]
  table{\thisDir/fig/data/stress-frf/bodeplots-73.tsv};
\addplot [color=RFIR,solid]
  table{\thisDir/fig/data/stress-frf/bodeplots-74.tsv};
% \addplot [color=best,dotted]
%   table{\thisDir/fig/data/stress-frf/bodeplots-75.tsv};
% \addplot [color=best,dotted]
%   table{\thisDir/fig/data/stress-frf/bodeplots-76.tsv};
\addplot [exact]
  table{\thisDir/fig/data/stress-frf/bodeplots-77.tsv};

% ================================================
\nextgroupplot[cheby2]
\node[plotnum,anchor=south east] at (rel axis cs:1,0) {8};
\addplot [color=G0Hat,solid]
  table{\thisDir/fig/data/stress-frf/bodeplots-78.tsv};
\addplot [color=G0Hat,solid]
  table{\thisDir/fig/data/stress-frf/bodeplots-79.tsv};
\addplot [color=existing,solid]
  table{\thisDir/fig/data/stress-frf/bodeplots-80.tsv};
\addplot [color=existing,solid]
  table{\thisDir/fig/data/stress-frf/bodeplots-81.tsv};
\addplot [color=LPMTrunc,solid]
  table{\thisDir/fig/data/stress-frf/bodeplots-82.tsv};
\addplot [color=LPMTrunc,solid]
  table{\thisDir/fig/data/stress-frf/bodeplots-83.tsv};
\addplot [color=RFIR,solid]
  table{\thisDir/fig/data/stress-frf/bodeplots-84.tsv};
\addplot [color=RFIR,solid]
  table{\thisDir/fig/data/stress-frf/bodeplots-85.tsv};
% \addplot [color=best,dotted]
%   table{\thisDir/fig/data/stress-frf/bodeplots-86.tsv};
% \addplot [color=best,dotted]
%   table{\thisDir/fig/data/stress-frf/bodeplots-87.tsv};
\addplot [exact]
  table{\thisDir/fig/data/stress-frf/bodeplots-88.tsv};

% ================================================
% ================================================
\nextgroupplot[ellip]
\node[plotnum,anchor=south west] at (rel axis cs:0,0) {9};
\addplot [color=G0Hat,solid]
  table{\thisDir/fig/data/stress-frf/bodeplots-89.tsv};
\addplot [color=G0Hat,solid]
  table{\thisDir/fig/data/stress-frf/bodeplots-90.tsv};
\addplot [color=existing,solid]
  table{\thisDir/fig/data/stress-frf/bodeplots-91.tsv};
\addplot [color=existing,solid]
  table{\thisDir/fig/data/stress-frf/bodeplots-92.tsv};
\addplot [color=LPMTrunc,solid]
  table{\thisDir/fig/data/stress-frf/bodeplots-93.tsv};
\addplot [color=LPMTrunc,solid]
  table{\thisDir/fig/data/stress-frf/bodeplots-94.tsv};
\addplot [color=RFIR,solid]
  table{\thisDir/fig/data/stress-frf/bodeplots-95.tsv};
\addplot [color=RFIR,solid]
  table{\thisDir/fig/data/stress-frf/bodeplots-96.tsv};
% \addplot [color=best,dotted]
%   table{\thisDir/fig/data/stress-frf/bodeplots-97.tsv};
% \addplot [color=best,dotted]
%   table{\thisDir/fig/data/stress-frf/bodeplots-98.tsv};
\addplot [exact]
  table{\thisDir/fig/data/stress-frf/bodeplots-99.tsv};

% ================================================
\nextgroupplot[ellip]
\node[plotnum,anchor=south] at (rel axis cs:0.5,0) {10};
\addplot [color=G0Hat,solid]
  table{\thisDir/fig/data/stress-frf/bodeplots-100.tsv};
\addplot [color=G0Hat,solid]
  table{\thisDir/fig/data/stress-frf/bodeplots-101.tsv};
\addplot [color=existing,solid]
  table{\thisDir/fig/data/stress-frf/bodeplots-102.tsv};
\addplot [color=existing,solid]
  table{\thisDir/fig/data/stress-frf/bodeplots-103.tsv};
\addplot [color=LPMTrunc,solid]
  table{\thisDir/fig/data/stress-frf/bodeplots-104.tsv};
\addplot [color=LPMTrunc,solid]
  table{\thisDir/fig/data/stress-frf/bodeplots-105.tsv};
\addplot [color=RFIR,solid]
  table{\thisDir/fig/data/stress-frf/bodeplots-106.tsv};
\addplot [color=RFIR,solid]
  table{\thisDir/fig/data/stress-frf/bodeplots-107.tsv};
% \addplot [color=best,dotted]
%   table{\thisDir/fig/data/stress-frf/bodeplots-108.tsv};
% \addplot [color=best,dotted]
%   table{\thisDir/fig/data/stress-frf/bodeplots-109.tsv};
\addplot [exact]
  table{\thisDir/fig/data/stress-frf/bodeplots-110.tsv};

% ================================================
\nextgroupplot[ellip]
\node[plotnum,anchor=south east] at (rel axis cs:1,0) {11};
\addplot [color=G0Hat,solid]
  table{\thisDir/fig/data/stress-frf/bodeplots-111.tsv};
\addplot [color=G0Hat,solid]
  table{\thisDir/fig/data/stress-frf/bodeplots-112.tsv};
\addplot [color=existing,solid]
  table{\thisDir/fig/data/stress-frf/bodeplots-113.tsv};
\addplot [color=existing,solid]
  table{\thisDir/fig/data/stress-frf/bodeplots-114.tsv};
\addplot [color=LPMTrunc,solid]
  table{\thisDir/fig/data/stress-frf/bodeplots-115.tsv};
\addplot [color=LPMTrunc,solid]
  table{\thisDir/fig/data/stress-frf/bodeplots-116.tsv};
\addplot [color=RFIR,solid]
  table{\thisDir/fig/data/stress-frf/bodeplots-117.tsv};
\addplot [color=RFIR,solid]
  table{\thisDir/fig/data/stress-frf/bodeplots-118.tsv};
% \addplot [color=best,dotted]
%   table{\thisDir/fig/data/stress-frf/bodeplots-119.tsv};
% \addplot [color=best,dotted]
%   table{\thisDir/fig/data/stress-frf/bodeplots-120.tsv};
\addplot [exact]
  table{\thisDir/fig/data/stress-frf/bodeplots-121.tsv};

% ================================================
\nextgroupplot[ellip]
\node[plotnum,anchor=south east] at (rel axis cs:1,0) {12};
\addplot [color=G0Hat,solid]
  table{\thisDir/fig/data/stress-frf/bodeplots-122.tsv};
\addplot [color=G0Hat,solid]
  table{\thisDir/fig/data/stress-frf/bodeplots-123.tsv};
\addplot [color=existing,solid]
  table{\thisDir/fig/data/stress-frf/bodeplots-124.tsv};
\addplot [color=existing,solid]
  table{\thisDir/fig/data/stress-frf/bodeplots-125.tsv};
\addplot [color=LPMTrunc,solid]
  table{\thisDir/fig/data/stress-frf/bodeplots-126.tsv};
\addplot [color=LPMTrunc,solid]
  table{\thisDir/fig/data/stress-frf/bodeplots-127.tsv};
\addplot [color=RFIR,solid]
  table{\thisDir/fig/data/stress-frf/bodeplots-128.tsv};
\addplot [color=RFIR,solid]
  table{\thisDir/fig/data/stress-frf/bodeplots-129.tsv};
% \addplot [color=best,dotted]
%   table{\thisDir/fig/data/stress-frf/bodeplots-130.tsv};
% \addplot [color=best,dotted]
%   table{\thisDir/fig/data/stress-frf/bodeplots-131.tsv};
\addplot [exact]
  table{\thisDir/fig/data/stress-frf/bodeplots-132.tsv};


% ================================================
% ================================================
\nextgroupplot[butter]
\node[plotnum,anchor=south west] at (rel axis cs:0,0) {13};
\addplot [color=G0Hat,solid]
  table{\thisDir/fig/data/stress-frf/bodeplots-133.tsv};
\addplot [color=G0Hat,solid]
  table{\thisDir/fig/data/stress-frf/bodeplots-134.tsv};
\addplot [color=existing,solid]
  table{\thisDir/fig/data/stress-frf/bodeplots-135.tsv};
\addplot [color=existing,solid]
  table{\thisDir/fig/data/stress-frf/bodeplots-136.tsv};
\addplot [color=LPMTrunc,solid]
  table{\thisDir/fig/data/stress-frf/bodeplots-137.tsv};
\addplot [color=LPMTrunc,solid]
  table{\thisDir/fig/data/stress-frf/bodeplots-138.tsv};
\addplot [color=RFIR,solid]
  table{\thisDir/fig/data/stress-frf/bodeplots-139.tsv};
\addplot [color=RFIR,solid]
  table{\thisDir/fig/data/stress-frf/bodeplots-140.tsv};
% \addplot [color=best,dotted]
%   table{\thisDir/fig/data/stress-frf/bodeplots-141.tsv};
% \addplot [color=best,dotted]
%   table{\thisDir/fig/data/stress-frf/bodeplots-142.tsv};
\addplot [exact]
  table{\thisDir/fig/data/stress-frf/bodeplots-143.tsv};

% ================================================
\nextgroupplot[butter]
\node[plotnum,anchor=south] at (rel axis cs:0.5,0) {14};
\addplot [color=G0Hat,solid]
  table{\thisDir/fig/data/stress-frf/bodeplots-144.tsv};
\addplot [color=G0Hat,solid]
  table{\thisDir/fig/data/stress-frf/bodeplots-145.tsv};
\addplot [color=existing,solid]
  table{\thisDir/fig/data/stress-frf/bodeplots-146.tsv};
\addplot [color=existing,solid]
  table{\thisDir/fig/data/stress-frf/bodeplots-147.tsv};
\addplot [color=LPMTrunc,solid]
  table{\thisDir/fig/data/stress-frf/bodeplots-148.tsv};
\addplot [color=LPMTrunc,solid]
  table{\thisDir/fig/data/stress-frf/bodeplots-149.tsv};
\addplot [color=RFIR,solid]
  table{\thisDir/fig/data/stress-frf/bodeplots-150.tsv};
\addplot [color=RFIR,solid]
  table{\thisDir/fig/data/stress-frf/bodeplots-151.tsv};
% \addplot [color=best,dotted]
%   table{\thisDir/fig/data/stress-frf/bodeplots-152.tsv};
% \addplot [color=best,dotted]
%   table{\thisDir/fig/data/stress-frf/bodeplots-153.tsv};
\addplot [exact]
  table{\thisDir/fig/data/stress-frf/bodeplots-154.tsv};

% ================================================
\nextgroupplot[butter]
\node[plotnum,anchor=south east] at (rel axis cs:1,0) {15};
\addplot [color=G0Hat,solid]
  table{\thisDir/fig/data/stress-frf/bodeplots-155.tsv};
\addplot [color=G0Hat,solid]
  table{\thisDir/fig/data/stress-frf/bodeplots-156.tsv};
\addplot [color=existing,solid]
  table{\thisDir/fig/data/stress-frf/bodeplots-157.tsv};
\addplot [color=existing,solid]
  table{\thisDir/fig/data/stress-frf/bodeplots-158.tsv};
\addplot [color=LPMTrunc,solid]
  table{\thisDir/fig/data/stress-frf/bodeplots-159.tsv};
\addplot [color=LPMTrunc,solid]
  table{\thisDir/fig/data/stress-frf/bodeplots-160.tsv};
\addplot [color=RFIR,solid]
  table{\thisDir/fig/data/stress-frf/bodeplots-161.tsv};
\addplot [color=RFIR,solid]
  table{\thisDir/fig/data/stress-frf/bodeplots-162.tsv};
% \addplot [color=best,dotted]
%   table{\thisDir/fig/data/stress-frf/bodeplots-163.tsv};
% \addplot [color=best,dotted]
%   table{\thisDir/fig/data/stress-frf/bodeplots-164.tsv};
\addplot [exact]
  table{\thisDir/fig/data/stress-frf/bodeplots-165.tsv};

% ================================================
\nextgroupplot[butter]
\node[plotnum,anchor=south east] at (rel axis cs:1,0) {16};
\addplot [color=G0Hat,solid]
  table{\thisDir/fig/data/stress-frf/bodeplots-166.tsv};
\addplot [color=G0Hat,solid]
  table{\thisDir/fig/data/stress-frf/bodeplots-167.tsv};
\addplot [color=existing,solid]
  table{\thisDir/fig/data/stress-frf/bodeplots-168.tsv};
\addplot [color=existing,solid]
  table{\thisDir/fig/data/stress-frf/bodeplots-169.tsv};
\addplot [color=LPMTrunc,solid]
  table{\thisDir/fig/data/stress-frf/bodeplots-170.tsv};
\addplot [color=LPMTrunc,solid]
  table{\thisDir/fig/data/stress-frf/bodeplots-171.tsv};
\addplot [color=RFIR,solid]
  table{\thisDir/fig/data/stress-frf/bodeplots-172.tsv};
\addplot [color=RFIR,solid]
  table{\thisDir/fig/data/stress-frf/bodeplots-173.tsv};
% \addplot [color=best,dotted]
%   table{\thisDir/fig/data/stress-frf/bodeplots-174.tsv};
% \addplot [color=best,dotted]
%   table{\thisDir/fig/data/stress-frf/bodeplots-175.tsv};
\addplot [exact]
  table{\thisDir/fig/data/stress-frf/bodeplots-176.tsv};

\end{groupplot}

% \node at ($(bodes c4r3.north east) + (1.5cm,0pt)$) {\pgfplotslegendfromname{legBodes}};

\end{tikzpicture}

  \caption{Bode plots of the tested filters~\legref{leg:bode:true}. Top to bottom: Chebyshev I, Chebyshev II, Elliptical and Butterworth characteristic.
  Left to right: low-pass, band-pass, high-pass and band-stop.
  For each filter, two of the parametric estimates $\model{\bullet}$ are shown per method.
  }
  \label{fig:bodeplots}
\end{figure}

\begin{figure}[p]
  \setlength{\figurewidth}{0.75\columnwidth}
  \setlength{\figureheight}{0.6\figurewidth}
  \centering
  % This file was created by matlab2tikz.
%
%workdir  : /Users/egon/Dropbox/VUB/PhD/Mikaya/initialvalues
%stack    : ../../../../../../../../Users/egon/Dropbox/VUB/PhD/Mikaya/initialvalues/testMonteCarloSystemsSmall (10)
%git info : commit 7bee89de2a914143a08c391cf0cbb846b1db5777
%           Author: Egon Geerardyn <egon.geerardyn@gmail.com>
%           Date:   Mon Feb 2 18:15:16 2015 +0100
%           
%               lo SNR stress test with Cheby
%           
%            printSummary.m               | 10 +++++----
%            summary.m                    |  2 ++
%            testMonteCarloSystemsSmall.m | 51 ++++++++++++++++++++++----------------------
%            3 files changed, 34 insertions(+), 29 deletions(-)
%           
%
%
%
\usepgfplotslibrary{groupplots}
\tikzset{plotnum/.style={black!30,align=left,font=\scriptsize,anchor=north west}}
\def\dataDir{\thisDir/data/stress-distances}

\begin{tikzpicture}

\pgfplotsset{every axis legend/.style={anchor=west}}



\begin{groupplot}[%
group style={%
  group name=dist,
  group size=4 by 4,
  horizontal sep=0.2cm,
  vertical sep=0.2cm,
  x descriptions at=edge bottom,
  y descriptions at=edge left},
xlabel={Run},
scale only axis,
xmin=1,
grid=major,
ytick={1,0.1,0.01},
xtick={25,50,75},
extra x ticks={100},
extra x tick labels={},
extra y ticks={0.02},
extra y tick labels={},
xmax=100,
ylabel={$D_{\bullet}$},
ymin=0.0009,
ymax=1,
ymode=log,
height=0.25\figureheight,
width=0.25\figurewidth,
axis x line*=bottom,
axis y line*=left]

% ================================================
\nextgroupplot[legend columns=-1, legend cell align=left, legend style={at={(rel axis cs:0.5,1.05)}, anchor=south west}]
\node[plotnum] at (rel axis cs:0,1) {1};

\addplot [G0Hat    , solid] table{\dataDir/distances-1.tsv};
\addlegendentry{\model{\trueSymbol}}
\label{leg:dist:model0}

\addplot [existing , solid] table{\dataDir/distances-2.tsv};
\addlegendentry{\model{exist}}
\label{leg:dist:modelExist}

\addplot [LPMTrunc , solid] table{\dataDir/distances-3.tsv};
\addlegendentry{\model{trunc}}
\label{leg:dist:modelTrunc}

\addplot [RFIR     , solid] table{\dataDir/distances-4.tsv};
\addlegendentry{\model{RFIR}}
\label{leg:dist:modelRFIR}

\addplot [best     , dashed] table{\dataDir/distances-5.tsv};
\addlegendentry{\model{best}}
\label{leg:dist:modelbest}

% ================================================
\nextgroupplot
\node[plotnum] at (rel axis cs:0,1) {2};
\addplot [G0Hat    , solid  , forget plot] table{\dataDir/distances-6.tsv};
\addplot [existing , solid  , forget plot] table{\dataDir/distances-7.tsv};
\addplot [LPMTrunc , solid  , forget plot] table{\dataDir/distances-8.tsv};
\addplot [RFIR     , solid  , forget plot] table{\dataDir/distances-9.tsv};
\addplot [best     , dashed , forget plot] table{\dataDir/distances-10.tsv};

% ================================================
\nextgroupplot
\node[plotnum] at (rel axis cs:0,1) {3};
\addplot [G0Hat    , solid  , forget plot] table{\dataDir/distances-11.tsv};
\addplot [existing , solid  , forget plot] table{\dataDir/distances-12.tsv};
\addplot [LPMTrunc , solid  , forget plot] table{\dataDir/distances-13.tsv};
\addplot [RFIR     , solid  , forget plot] table{\dataDir/distances-14.tsv};
\addplot [best     , dashed , forget plot] table{\dataDir/distances-15.tsv};

% ================================================
\nextgroupplot
\node[plotnum] at (rel axis cs:0,1) {4};
\addplot [G0Hat    , solid  , forget plot] table{\dataDir/distances-16.tsv};
\addplot [existing , solid  , forget plot] table{\dataDir/distances-17.tsv};
\addplot [LPMTrunc , solid  , forget plot] table{\dataDir/distances-18.tsv};
\addplot [RFIR     , solid  , forget plot] table{\dataDir/distances-19.tsv};
\addplot [best     , dashed , forget plot] table{\dataDir/distances-20.tsv};

% ================================================
% ================================================
\nextgroupplot
\node[plotnum] at (rel axis cs:0,1) {5};
\addplot [G0Hat    , solid  , forget plot] table{\dataDir/distances-21.tsv};
\addplot [existing , solid  , forget plot] table{\dataDir/distances-22.tsv};
\addplot [LPMTrunc , solid  , forget plot] table{\dataDir/distances-23.tsv};
\addplot [RFIR     , solid  , forget plot] table{\dataDir/distances-24.tsv};
\addplot [best     , dashed , forget plot] table{\dataDir/distances-25.tsv};

% ================================================
\nextgroupplot
\node[plotnum] at (rel axis cs:0,1) {6};
\addplot [G0Hat    , solid  , forget plot] table{\dataDir/distances-26.tsv};
\addplot [existing , solid  , forget plot] table{\dataDir/distances-27.tsv};
\addplot [LPMTrunc , solid  , forget plot] table{\dataDir/distances-28.tsv};
\addplot [RFIR     , solid  , forget plot] table{\dataDir/distances-29.tsv};
\addplot [best     , dashed , forget plot] table{\dataDir/distances-30.tsv};

% ================================================
\nextgroupplot
\node[plotnum] at (rel axis cs:0,1) {7};
\addplot [G0Hat    , solid  , forget plot] table{\dataDir/distances-31.tsv};
\addplot [existing , solid  , forget plot] table{\dataDir/distances-32.tsv};
\addplot [LPMTrunc , solid  , forget plot] table{\dataDir/distances-33.tsv};
\addplot [RFIR     , solid  , forget plot] table{\dataDir/distances-34.tsv};
\addplot [best     , dashed , forget plot] table{\dataDir/distances-35.tsv};

% ================================================
\nextgroupplot
\node[plotnum] at (rel axis cs:0,1) {8};
\addplot [G0Hat    , solid  , forget plot] table{\dataDir/distances-36.tsv};
\addplot [existing , solid  , forget plot] table{\dataDir/distances-37.tsv};
\addplot [LPMTrunc , solid  , forget plot] table{\dataDir/distances-38.tsv};
\addplot [RFIR     , solid  , forget plot] table{\dataDir/distances-39.tsv};
\addplot [best     , dashed , forget plot] table{\dataDir/distances-40.tsv};

% ================================================
% ================================================
\nextgroupplot
\node[plotnum] at (rel axis cs:0,1) {9};
\addplot [G0Hat    , solid  , forget plot] table{\dataDir/distances-41.tsv};
\addplot [existing , solid  , forget plot] table{\dataDir/distances-42.tsv};
\addplot [LPMTrunc , solid  , forget plot] table{\dataDir/distances-43.tsv};
\addplot [RFIR     , solid  , forget plot] table{\dataDir/distances-44.tsv};
\addplot [best     , dashed , forget plot] table{\dataDir/distances-45.tsv};

% ================================================
\nextgroupplot
\node[plotnum] at (rel axis cs:0,1) {10};
\addplot [G0Hat    , solid  , forget plot] table{\dataDir/distances-46.tsv};
\addplot [existing , solid  , forget plot] table{\dataDir/distances-47.tsv};
\addplot [LPMTrunc , solid  , forget plot] table{\dataDir/distances-48.tsv};
\addplot [RFIR     , solid  , forget plot] table{\dataDir/distances-49.tsv};
\addplot [best     , dashed , forget plot] table{\dataDir/distances-50.tsv};

% ================================================
\nextgroupplot
\node[plotnum] at (rel axis cs:0,1) {11};
\addplot [G0Hat    , solid  , forget plot] table{\dataDir/distances-51.tsv};
\addplot [existing , solid  , forget plot] table{\dataDir/distances-52.tsv};
\addplot [LPMTrunc , solid  , forget plot] table{\dataDir/distances-53.tsv};
\addplot [RFIR     , solid  , forget plot] table{\dataDir/distances-54.tsv};
\addplot [best     , dashed , forget plot] table{\dataDir/distances-55.tsv};

% ================================================
\nextgroupplot
\node[plotnum] at (rel axis cs:0,1) {12};
\addplot [G0Hat    , solid  , forget plot] table{\dataDir/distances-56.tsv};
\addplot [existing , solid  , forget plot] table{\dataDir/distances-57.tsv};
\addplot [LPMTrunc , solid  , forget plot] table{\dataDir/distances-58.tsv};
\addplot [RFIR     , solid  , forget plot] table{\dataDir/distances-59.tsv};
\addplot [best     , dashed , forget plot] table{\dataDir/distances-60.tsv};


% ================================================
% ================================================
\nextgroupplot
\node[plotnum] at (rel axis cs:0,1) {13};
\addplot [G0Hat    , solid  , forget plot] table{\dataDir/distances-61.tsv};
\addplot [existing , solid  , forget plot] table{\dataDir/distances-62.tsv};
\addplot [LPMTrunc , solid  , forget plot] table{\dataDir/distances-63.tsv};
\addplot [RFIR     , solid  , forget plot] table{\dataDir/distances-64.tsv};
\addplot [best     , dashed , forget plot] table{\dataDir/distances-65.tsv};

% ================================================
\nextgroupplot
\node[plotnum] at (rel axis cs:0,1) {14};
\addplot [G0Hat    , solid  , forget plot] table{\dataDir/distances-66.tsv};
\addplot [existing , solid  , forget plot] table{\dataDir/distances-67.tsv};
\addplot [LPMTrunc , solid  , forget plot] table{\dataDir/distances-68.tsv};
\addplot [RFIR     , solid  , forget plot] table{\dataDir/distances-69.tsv};
\addplot [best     , dashed , forget plot] table{\dataDir/distances-70.tsv};

% ================================================
\nextgroupplot
\node[plotnum] at (rel axis cs:0,1) {15};
\addplot [G0Hat    , solid  , forget plot] table{\dataDir/distances-71.tsv};
\addplot [existing , solid  , forget plot] table{\dataDir/distances-72.tsv};
\addplot [LPMTrunc , solid  , forget plot] table{\dataDir/distances-73.tsv};
\addplot [RFIR     , solid  , forget plot] table{\dataDir/distances-74.tsv};
\addplot [best     , dashed , forget plot] table{\dataDir/distances-75.tsv};

% ================================================
\nextgroupplot
\node[plotnum] at (rel axis cs:0,1) {16};
\addplot [G0Hat    , solid  , forget plot] table{\dataDir/distances-76.tsv};
\addplot [existing , solid  , forget plot] table{\dataDir/distances-77.tsv};
\addplot [LPMTrunc , solid  , forget plot] table{\dataDir/distances-78.tsv};
\addplot [RFIR     , solid  , forget plot] table{\dataDir/distances-79.tsv};
\addplot [best     , dashed , forget plot] table{\dataDir/distances-80.tsv};

\end{groupplot}

%\node at ($(dist c4r3.north east) +(1.5cm,0pt)$) {\ref{legDistances}};

\end{tikzpicture}%

  \caption{Empirical cumulative distribution function of the validation distance $D_{\bullet} = \norm{\model{\bullet} - G_0}$ for the different tested systems (different numbered plots) and the different initial values (colors).
  Note that since $\norm{G_0}=1$ is chosen, $D_{\bullet}$ immediately indicates the relative RMS error of the estimates.
  }
  \label{fig:distancesStress}
\end{figure}

The transfer functions are shown in \figref{fig:bodeplots} while the empirical cumulative validation distance is shown in \figref{fig:distancesStress}.

Our general conclusion from these simulations is that RFIR performs as well (or a lot better) than the existing techniques when good models are observed.

{\footnotesize
A few additional remarks are appropriate with respect to these results:
\begin{itemize}
  \item In many of the tested situations (subplots 1,3,13-16), all methods yield a good model quality.
  \item In some situations (subplots 8, 10, 12), the model quality is generally poor. 
  The fact that for subplots 8 and 12, the `ideal' starting value \model{0} also provides poor models means the data is not informative enough to reasonably fit a good model.
  \item In all cases except 8, 10 and 12, the RFIR performed as well or better than the existing methods.
  \item In subplot 10, the model quality of the RFIR is slightly worse than the already poor existing techniques.
  This is a underlying limitation of the RFIR used: $n_g=200$ taps was used, however the true impulse response has not decayed significantly in that interval.
  Essentially, the particular RFIR used introduces a significant bias in the non-parametric description of the filter and hence the obtained initial value is unreliable.
  \item The RFIR performs (significantly) better than the existing methods in various cases (subplot 4-7, 9, 11).
  \item Since the `best' method \legref{leg:dist:modelBest} yields better models more often than a single method, e.g. in subplots 2, 4 and 6, this illustrates that combining initialization schemes improves the model quality.
  Particularly, subplot 2 implies that \model{best} contains initial estimates of all different techniques.
  \item For all of the tested systems (except perhaps subplot 2), the truncated LPM is not advisable over the existing methods.
\end{itemize}
}
