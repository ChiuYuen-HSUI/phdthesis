% % BEGIN OF TIKZ ONLY
% \usepackage{tikz,pgfplots}
% \usepackage{pgfplotstable}
% \pgfplotsset{compat=newest}
% \usetikzlibrary{external}
% %\tikzset{external/optimize=false}
% \tikzset{external/mode=list and make}
% \tikzexternalize[prefix=tikz/]
% \tikzsetfigurename{figure}
% \pgfplotsset{compat=newest}
\pgfplotsset{tick style={black!30},grid style={black!10}}
\pgfplotsset{every axis legend/.append style={font=\footnotesize}}
\pgfplotsset{every axis label/.append style={font=\footnotesize}}
\pgfplotsset{every tick label/.append style={font=\footnotesize}}
\pgfplotsset{every axis title/.append style={font=\footnotesize}}
\pgfplotsset{every axis post/.style={unbounded coords=jump}}
\pgfplotsset{every axis title/.append style={at={(0.5,0.95)}}}

\definecolor{LPMTrunc}{named}{TangoSkyBlue2}
\definecolor{LPMTruncInit}{named}{TangoSkyBlue2}
\definecolor{RFIR}{named}{TangoScarletRed3}
\definecolor{RFIRInit}{named}{TangoScarletRed1}
\definecolor{existing}{named}{TangoChameleon3}
\definecolor{G0Hat}{named}{TangoOrange2}
\definecolor{GVXI}{named}{G0Hat}
\definecolor{reference}{named}{black}

\definecolor{heuristic}{named}{TangoPlum2}
\definecolor{observed}{named}{TangoOrange3}

\definecolor{best}{named}{TangoAluminium6}

\definecolor{G0HatFill}{named}{TangoOrange1}
\definecolor{GVXIFill}{named}{G0HatFill}
\definecolor{LPMTruncFill}{named}{TangoSkyBlue1}
\definecolor{RFIRFill}{named}{TangoScarletRed1}
\definecolor{existingFill}{named}{TangoChameleon1}
\definecolor{bestFill}{named}{TangoAluminium3}

\definecolor{FRFMean}{named}{TangoPlum3}
\definecolor{FRFSingle}{named}{TangoPlum1}
\definecolor{FRFNoise}{named}{TangoChocolate1}

\pgfplotsset{FRFMean/.style={color=FRFMean,mark=*,mark options={solid},only marks,medsmallmarkers}}
\pgfplotsset{FRFSingle/.style={color=FRFSingle,mark=square*,mark options={solid},only marks,smallmarkers}}
\pgfplotsset{FRFNoise/.style={color=FRFNoise,mark=x,mark options={solid},only marks,smallmarkers}}

\pgfset{number format/1000 sep={\,}}

\pgfplotsset{bandwidth/.style={area style,fill=TangoButter1,draw=TangoButter2,fill opacity=0.5}}
\pgfplotsset{goodestimate/.style={color=TangoChameleon2,line join=round}}
\pgfplotsset{badestimate/.style={color=TangoScarletRed2,solid,line join=round}}
\pgfplotsset{exact/.style={color=black,dashed,line width=0.75pt,line join=round}}

\pgfplotsset{LPMTruncmark/.append style={mark=*,mark options={solid}}}
\pgfplotsset{RFIRmark/.append style={mark=square*,mark options={solid}}}
\pgfplotsset{existingmark/.append style={mark=triangle*,mark options={solid}}}
\pgfplotsset{existingInitmark/.append style={mark=triangle,mark options={solid}}}
\pgfplotsset{LPMTruncInitmark/.append style={mark=o,mark options={solid}}}
\pgfplotsset{RFIRInitmark/.append style={mark=square,mark options={solid}}}
\pgfplotsset{bestmark/.append style={mark=diamond*,mark options={solid}}}
\pgfplotsset{G0Hatmark/.append style={mark=asterisk,mark options={solid}}}

\pgfplotsset{GVXI/.style={color=GVXI,line width=1pt}}
\pgfplotsset{G0Hat/.style={color=G0Hat,line width=1.5pt}}
\pgfplotsset{existing/.style={color=existing}}
\pgfplotsset{LPMTruncInit/.style={color=LPMTruncInit,densely dashed}}
\pgfplotsset{LPMTrunc/.style={color=LPMTrunc}}
\pgfplotsset{RFIRInit/.style={color=RFIRInit,densely dashed}}
\pgfplotsset{RFIR/.style={color=RFIR}}
\pgfplotsset{best/.style={color=best}}

\pgfplotsset{smallmarkers/.append style={mark size=0.75pt}}
\pgfplotsset{medsmallmarkers/.append style={mark size=0.5pt}}
\pgfplotsset{tinymarkers/.append style={mark size=0.25pt}}
\pgfplotsset{extremelytinymarkers/.append style={mark size=0.05pt}}


\tikzset{annotation/.style={align=left,draw=black!0.2,font=\scriptsize,fill=white,fill opacity=0.8}}
% http://tex.stackexchange.com/questions/83487/pgfplotstable-converting-zeros-to-in-a-knitr-inline-table
\pgfplotstableset{%
	zerofill=true,
	after row=[3pt],
                  every head row/.style={before row=\toprule, after row={\\\midrule}},
                  every last row/.style={after row=\bottomrule}
	assign column name/.code={%
        \pgfkeyssetvalue{/pgfplots/table/column name}{\multicolumn{1}{c}{\multirow{2}{*}{#1}} }%
    },
	columns/method/.style={string type,column name={\shortstack{\textsc{Method}\\\phantom{0}}}},
	columns/P000/.style={column type=r,column name={\shortstack[r]{\textsc{Min.}\\{$0\%$}}}},
	columns/P025/.style={column type=r,column name={\shortstack[r]{\vphantom{?}\\\textsc{$25\%$}}}},
	columns/P050/.style={column type=r,column name={\shortstack[r]{\textsc{Median}\\{$50\%$}}}},
	columns/P075/.style={column type=r,column name={\shortstack[r]{\vphantom{?}\\\textsc{$75\%$ }}}},
	columns/P100/.style={column type=r,column name={\shortstack[r]{\textsc{Max.}\\{$100\%$}}}},
	columns/contribution/.style={column name={\shortstack[r]{\textsc{Contribution}\\ \vphantom{0}}},
                    	postproc cell content/.append code={\pgfkeysalso{@cell content/.add={}{\%}}},
	}
}

% % END OF TIKZ ONLY
% \newcommand{\legref}[2][1]{(\ref{#2})}

%  \newlength\figureheight
%  \newlength\figurewidth
%  \setlength\figureheight{7cm}
%  \setlength\figurewidth{10cm}

%  \newlength\onecolumnwidth
%\title{Improved Initial Estimates Via FRF Smoothing Techniques for Parametric Identification of LTI Systems}

% \begin{abstract}
% Good initial values are crucial to obtain solutions of non-convex optimization problems. When estimating the transfer function of physical systems from measured noisy data, obtaining good initial parameter estimates is therefore a primordial step.
% In this paper, it is shown that smoothing the measured frequency response function (FRF) of a linear time-invariant system enhances the construction of initial estimates significantly, resulting in the optimization schemes to converge to a better optimum. 
% This is achieved with minimal user interaction.

% Two smoothing techniques, the time-truncated local polynomial method (LPM) and the regularized finite impulse response (RFIR), are compared with the existing generalized total least squares (GTLS) and the bootstrapped total least squares (BTLS) initial estimates. 
% The improvement attributable to smoothing is demonstrated by a simulation and by measurements of an electrical filter. 
% The results ultimately show that the parametric models obtained using the proposed starting values are much more likely to give a good description of the measured system and hence lead to more useful models.
% \end{abstract}

\setlength\onecolumnwidth{\columnwidth}

%\comment{JL: 
%Focus of the article (this is what the article must reflect):

%\begin{itemize}
%\item demonstrate that a smoothed FRF, obtained as described in \citep{Lumori2014}, used as an initial estimate increases the success rate of a parametric LTI TF identification
%\item
%\comment{JL:  emphasize the importance of extending the SNR range for which the success rate is acceptable, yielding a reliable parametric estimator to be used with as little user interaction as possible.
%\end{itemize}
%}
