\TODO{
  \begin{itemize}
    \item fix labels for chapter
    \item add section from Benchmark paper
    \item fix some bib entries
    \item Hinf macro aanpassen voor PDF support (indien dit niet conflicteert met glossaries)
  \end{itemize}
}
\pgfplotsset{compat=newest}
\pgfplotsset{tick style={black!30},grid style={black!10}}
\pgfplotsset{every axis legend/.append style={font=\footnotesize}}
\pgfplotsset{every axis label/.append style={font=\footnotesize}}
\pgfplotsset{every tick label/.append style={font=\footnotesize}}
\pgfplotsset{every axis title/.append style={font=\footnotesize}}
\pgfplotsset{every axis post/.style={unbounded coords=jump}}
\pgfplotsset{every axis title/.append style={at={(0.5,0.95)}}}

\definecolor{LPMTrunc}{named}{TangoSkyBlue2}
\definecolor{LPMTruncInit}{named}{TangoSkyBlue2}
\definecolor{RFIR}{named}{TangoScarletRed3}
\definecolor{RFIRInit}{named}{TangoScarletRed1}
\definecolor{existing}{named}{TangoChameleon3}
\definecolor{G0Hat}{named}{TangoOrange2}
\definecolor{GVXI}{named}{G0Hat}
\definecolor{reference}{named}{black}

\definecolor{heuristic}{named}{TangoPlum2}
\definecolor{observed}{named}{TangoOrange3}

\definecolor{best}{named}{TangoAluminium6}

\definecolor{G0HatFill}{named}{TangoOrange1}
\definecolor{GVXIFill}{named}{G0HatFill}
\definecolor{LPMTruncFill}{named}{TangoSkyBlue1}
\definecolor{RFIRFill}{named}{TangoScarletRed1}
\definecolor{existingFill}{named}{TangoChameleon1}
\definecolor{bestFill}{named}{TangoAluminium3}

\definecolor{FRFMean}{named}{TangoPlum3}
\definecolor{FRFSingle}{named}{TangoPlum1}
\definecolor{FRFNoise}{named}{TangoChocolate1}

\pgfplotsset{FRFMean/.style={color=FRFMean,mark=*,mark options={solid},only marks,medsmallmarkers}}
\pgfplotsset{FRFSingle/.style={color=FRFSingle,mark=square*,mark options={solid},only marks,smallmarkers}}
\pgfplotsset{FRFNoise/.style={color=FRFNoise,mark=x,mark options={solid},only marks,smallmarkers}}

\pgfset{number format/1000 sep={\,}}

\pgfplotsset{bandwidth/.style={area style,fill=TangoButter1,draw=TangoButter2,fill opacity=0.5}}
\pgfplotsset{goodestimate/.style={color=TangoChameleon2,line join=round}}
\pgfplotsset{badestimate/.style={color=TangoScarletRed2,solid,line join=round}}
\pgfplotsset{exact/.style={color=black,dashed,line width=0.75pt,line join=round}}

\pgfplotsset{LPMTruncmark/.append style={mark=*,mark options={solid}}}
\pgfplotsset{RFIRmark/.append style={mark=square*,mark options={solid}}}
\pgfplotsset{existingmark/.append style={mark=triangle*,mark options={solid}}}
\pgfplotsset{existingInitmark/.append style={mark=triangle,mark options={solid}}}
\pgfplotsset{LPMTruncInitmark/.append style={mark=o,mark options={solid}}}
\pgfplotsset{RFIRInitmark/.append style={mark=square,mark options={solid}}}
\pgfplotsset{bestmark/.append style={mark=diamond*,mark options={solid}}}
\pgfplotsset{G0Hatmark/.append style={mark=asterisk,mark options={solid}}}

\pgfplotsset{GVXI/.style={color=GVXI,line width=1pt}}
\pgfplotsset{G0Hat/.style={color=G0Hat,line width=1.5pt}}
\pgfplotsset{existing/.style={color=existing}}
\pgfplotsset{LPMTruncInit/.style={color=LPMTruncInit,densely dashed}}
\pgfplotsset{LPMTrunc/.style={color=LPMTrunc}}
\pgfplotsset{RFIRInit/.style={color=RFIRInit,densely dashed}}
\pgfplotsset{RFIR/.style={color=RFIR}}
\pgfplotsset{best/.style={color=best}}

\pgfplotsset{smallmarkers/.append style={mark size=0.75pt}}
\pgfplotsset{medsmallmarkers/.append style={mark size=0.5pt}}
\pgfplotsset{tinymarkers/.append style={mark size=0.25pt}}
\pgfplotsset{extremelytinymarkers/.append style={mark size=0.05pt}}


\tikzset{annotation/.style={align=left,draw=black!0.2,font=\scriptsize,fill=white,fill opacity=0.8}}
% http://tex.stackexchange.com/questions/83487/pgfplotstable-converting-zeros-to-in-a-knitr-inline-table
\pgfplotstableset{%
	zerofill=true,
	after row=[3pt],
                  every head row/.style={before row=\toprule, after row={\\\midrule}},
                  every last row/.style={after row=\bottomrule}
	assign column name/.code={%
        \pgfkeyssetvalue{/pgfplots/table/column name}{\multicolumn{1}{c}{\multirow{2}{*}{#1}} }%
    },
	columns/method/.style={string type,column name={\shortstack{\textsc{Method}\\\phantom{0}}}},
	columns/P000/.style={column type=r,column name={\shortstack[r]{\textsc{Min.}\\{$0\%$}}}},
	columns/P025/.style={column type=r,column name={\shortstack[r]{\vphantom{?}\\\textsc{$25\%$}}}},
	columns/P050/.style={column type=r,column name={\shortstack[r]{\textsc{Median}\\{$50\%$}}}},
	columns/P075/.style={column type=r,column name={\shortstack[r]{\vphantom{?}\\\textsc{$75\%$ }}}},
	columns/P100/.style={column type=r,column name={\shortstack[r]{\textsc{Max.}\\{$100\%$}}}},
	columns/contribution/.style={column name={\shortstack[r]{\textsc{Contribution}\\ \vphantom{0}}},
                    	postproc cell content/.append code={\pgfkeysalso{@cell content/.add={}{\%}}},
	}
}


\newcommand{\wnValue}{\ensuremath{0.95 \unit{rad/s}}}
\newcommand{\dampingValue}{\ensuremath{0.025}}
\newcommand{\TsValue}{\ensuremath{1 \unit{s}}}

\section{Introduction}
\label{sec:Intro}

Robustness is of key importance in feedback controlled systems, since feedback can lead to performance degradation or even closed-loop instability. 
An important example includes an \gls{AVIS}, where feedback is used to isolate high-precision equipment from external disturbances. 
The underlying feedback control principle is skyhook damping~\citep{Karnopp1995}. 
However, the performance of skyhook damping is limited by high-frequent parasitic resonance phenomena. 
Such model uncertainties can be taken into account explicitly in a robust control design~\citep{Zhang2005HybridAvis} for approaches based on \Hinf{} optimization.
However, the uncertainty estimation in these references is based on rough prior assumptions and may be inaccurate. 
On the one hand, this can lead to potentially dangerous results, since no stability and performance guarantees can be given if the estimated uncertainty is too small. 
On the other hand, this can also lead to overly conservative controllers if the estimated uncertainty is too large.

Several approaches have been developed to determine bounds on the model uncertainty. First, model validation techniques have been developed~\citep[see e.g.][]{Smith1992,Xu1999}.
However, these results are typically overly optimistic~\citep{Oomen2009UncEstim}.
Second, model error modeling approaches based on parametric system identification~\citep{Ljung1999MEM} with explicit characterization of bias and variance errors have been proposed, but these require a significant amount of user intervention.
Third, non-parametric identification approaches have been adopted, e.g.~\citep{vandeWal2002,deVries1994}. 
In~\citep{vandeWal2002}, an identified frequency response function is used directly to evaluate the \Hinf{} norm on a discrete frequency grid. 
In~\citep{deVries1994}, an extended method is presented that bounds the error in between frequency points in a worst-case fashion. 
However, such worst-case methods are well-known to be overly conservative~\citep[Section 9.5.2]{Vinnicombe2001}. 
Fourth, recently a data-driven \Hinf{} norm estimation has been developed in~\citep{Wahlberg2010,Oomen2014ILH}.
These methods rely on a sequence of iterative experiments and directly delivers an estimate of the \Hinf{} norm, and combines an optimal experiment design while essentially taking inter-grid errors into account.
A recent application of this method in~\citep{Oomen2014ILH} has revealed that these iterative methods lead to significantly more accurate \Hinf{} estimates compared with traditional frequency response-based methods, thereby underlining the importance of the inter-grid error.
Unfortunately, these methods require a sequence of dedicated iterative experiments.
This increases the measurement time significantly, especially for \gls{MIMO} plants.
When measurement time is expensive, this is a severe impediment.

Although important steps have been made in model-error-modeling, existing methods involve a trade-off between accuracy, measurement time, and user intervention.
The aim of the present paper is to develop a golden mean between traditional non-parametric techniques (third approach) and a full parametric model (second approach).
The proposed $\Hinf{}$ norm estimation procedure exploits recent developments in frequency domain estimation the \gls{LPM}~\citep{Schoukens2009LPM} and \gls{LRM}~\citep{McKelvey2012LRM} as introduced in~\citep{Geerardyn2014IFAC,Geerardyn2014ISMA}.
Before, these \gls{LPM} and \gls{LRM} techniques have been used to enhance at-grid frequency response estimates.
In the present paper, these methods are extended to provide both an at-grid and inter-grid evaluation of the peak value, as is required for the $\Hinf{}$ norm. 
The performance of these local parametric models with respect to robust control design is investigated, and bounds on the required measurement time to obtain a reliable estimate of the \Hinf{} norm are obtained.
Preliminary results appear in~\citep{Geerardyn2014IFAC,Geerardyn2014ISMA}.
The present paper presents the full procedure, including more theoretical treatment of the $\Hinf{}$ norm estimation procedure, and a description of the required measurement time.

\vspace{-1em}
\paragraph*{Outline}
\label{par:toc}
The problem is formulated in \secref{sec:Problem}.
In \secref{sec:LPMHinf} the novel method to estimate the $\Hinf{}$ norm by means of the \gls{LRM} is introduced.
In \secref{sec:simulation} the method is illustrated on simulations, and the required measurement time is determined.
The technique is demonstrated on measurements of an \gls{AVIS} in \secref{sec:measurement}.
Finally, the results and some challenges are discussed in \secref{sec:conclusion}.

\glsresetall % reset the usage of all abbreviations s.t. intro and body get the explanation at first use

\section{Problem Formulation}
\label{sec:Problem}
Robust control based on \Hinf{} optimization requires a nominal parametric model $\estimated\Plant$ and a bound on the model error $\Delta$. 
After the model $\estimated\Plant$ is determined, either through first principles modeling or system identification, it remains to determine a bound on the model error $\Delta$.
The \Hinf{} norm of the error system $\Delta$, i.e.
\begin{equation}
  \gamma 
     = 
       \infnorm{\Delta} 
    \isdef 
       \sup_{\omega} 
         \abs{\Delta(\omega)}
  \label{eq:HinfNormSISO}
\end{equation}
for a stable \gls{SISO} system, or its weighted form $\infnorm{W \Delta V}$, is a measure that serves an important purpose in many robust control design methodologies that specifies the `size' of the model class for which a controller is to be designed~\citep{Skogestad2005}.

To estimate $\gamma$, only $\Delta(\omega_{\star})$ is of interest, where
$ \omega_{\star} 
     \isdef 
       \arg
         \sup_{\omega} 
           \abs{\Delta(\omega)}
  \text{.}
  \label{eq:omega-star}
$
However, in a finite measurement time, non-parametric approaches only allow access to a finite and discrete grid of frequencies $\omega_k \in \DFTGrid$.
As such, \eqref{eq:HinfNormSISO} can be rewritten as
\begin{equation}
  \gamma \isdef \sup_{\omega} \abs{\Delta(\omega)}
         = \sup 
             \set{
                \ezbrace{\sup_{\omega \in \DFTGrid  } \abs{\Delta(\omega)}}{\gamma_{\mathrm{FRF}}}
                \;,\;
                \ezbrace{
                \sup_{\omega \in \InterGrid} \abs{\Delta(\omega)}}{\gamma_{\mathrm{IG}}}}
  \label{eq:gamma-split-grid}
\end{equation}
where $\Omega \isdef \OpenInterval{0}{\frac{2\pi}{\Ts}}$ is the measured frequency range for a sampling time $\Ts$.
Two cases can be distinguished:
\begin{enumerate}
  \item $\omega_{\star} \in \DFTGrid$ such that $\abs{\Delta(\omega_{\star})} = \gamma =\gamma_{\mathrm{FRF}}$ is observed directly on the discrete frequency grid,
  \item $\omega_{\star} \not\in \DFTGrid$, i.e. the extremal value $\abs{\Delta(\omega_{\star})}  = \gamma = \gamma_{\mathrm{IG}}$ is determined by the inter-grid behavior.
\end{enumerate}
The use of $\gamma_{\mathrm{FRF}}$ is standard in \gls{FRF}-based approaches, e.g.~\citep{vandeWal2002}.
Its expected value over different realizations of the input signal $\expectedValue{\gamma_{\mathrm{FRF}}} \leq \gamma$ due to the limited frequency resolution.
Such an underestimate may lead to overly optimistic controller designs.

The main contribution of this paper is to  estimate the inter-grid term $\gamma_{\mathrm{IG}}$ without the need for a (global) parametric model.
Instead, a  realistic value of $\gamma$ is obtained using parametric models that are valid over small frequency ranges.

\subsection{Set-up}
\TODO{overgang $\Delta(q)$ naar output-error}

Consider an output-error \gls{LTI} discrete-time \gls{SISO} system $\Delta(q)$ excited by an input signal $u_{\Delta}(n)$.
For an infinitely long data record, this can be described in the time-domain as
\begin{equation}
  y_{\Delta}(n) 
    = 
      y_{\Delta0}
      + v(n) 
    = 
      \Delta(q)u_{\Delta}(n) 
      + H(q)e(n)
  \label{eq:sysDeltaInfData}
\end{equation}
where $q^{-1}$ is the lag operator, $e(n)$ is white Gaussian noise with unity variance, and both $\Delta(q)$ and $H(q)$ are stable causal real-rational functions.
For a limited data record ($n \in \set{0,\ldots,N-1}$), the transient contributions $t_{\Delta}(n)$ lead to:
\begin{equation}
    y_{\Delta}(n) 
      = 
        \Delta(q)u(n) 
        + H(q)e(n) 
        + t_{\Delta}(n)
    \text{.}
    \label{eq:sysDeltaFiniteData}
\end{equation}
By applying the \gls{DFT}
\begin{equation}
  X(k) 
       = 
         \frac{1}
              {\sqrt{N}} 
         \sum_{n=0}^{N-1} 
            x(n) \exp \left( \frac{- j 2 \pi k n }{N} \right)
         \quad   
         \forall k \in \mathbb{Z}
  \label{eq:DFT}
\end{equation} 
to both sides of~\eqref{eq:sysDeltaFiniteData}, one obtains:
\begin{equation}
  Y_{\Delta}(k) 
                = 
                  \Delta(\omega_k) U_{\Delta}(k) 
                  + T_{\Delta}(\omega_k) 
                  + V(k) 
  \text{.}
  \label{eq:sysDeltaFiniteDataFD}
\end{equation}
The index $k$ corresponds to the $k^{\text{th}}$ frequency bin with frequency $\omega_{k} = 2\pi k / (N \Ts)$ such that the frequency grid $\DFTGrid \isdef \set{\left.\omega_k \right| k = 1,\ldots,N-1}$.

The presented results in this paper directly extend to closed-loop systems, as is shown in \secref{sec:control}.
The results in this paper are presented in discrete time, but the extension to the continuous time case is straightforward.

\section{\Hinf{} Norm Estimation Procedure}
\label{sec:LPMHinf}
To estimate $\gamma_{\mathrm{IG}}$, we propose to identify local parametric models in the frequency domain.
The frequency axis is partitioned into continuous segments $\Omega_k \isdef \OpenInterval{ \omega_{k-1} }{ \omega_{k+1} }$ in view of the local validity of the local models.
In particular, the inter-grid aspect of the $\Hinf$ norm is estimated as
\begin{equation}
  \gamma_{\mathrm{IG}} = 
    \sup \set{ \left. 
                \sup_{\omega \in \Omega_k} \abs{\LocalModel{\Delta}_k(\omega)}
                \right|
                k = 1,\ldots,N-1
                }
   \label{eq:gamma-inter-grid}
\end{equation}
where the $\LocalModel{\Delta}_{k}$ is the local model around the ($k$th) frequency $\omega_k$.
An example of such local models is shown in \figref{fig:interpol-quantities}.
Two alternative methods allow to evaluate~\eqref{eq:gamma-inter-grid} practically.

\begin{figure}
  \centering
  \setlength{\figurewidth}{0.8\columnwidth}
  \setlength{\figureheight}{0.68\figurewidth}
  % This actually uses an LPM(1,2,-1) model to FRF data. Since the FRF contains no transient, the input can be mocked to a 1-spectrum and the output to the FRF. It can be seen that DOF = 0, as all models pass through the same points, this is logical as a parabola is fitted through 3 points.
\begin{tikzpicture}

% \setlength{\figurewidth}{0.8\figurewidth}

\begin{axis}[%
name=delta,
width=\figurewidth,
height=0.64\figureheight,
unbounded coords=jump,
scale only axis,
xtick={0.83776,0.90757,0.97738,1.0472,1.117,1.1868},
xtick={\empty},
%xticklabels={$\omega_1$,$\omega_2$,$\omega_3$,$\omega_4$,$\omega_5$,$\omega_6$},
xticklabels={},
ytick={22.5},
yticklabels={},
xmin=0.78,
xmax=1.25,
xminorticks=true,
xmajorgrids=false,
%xlabel={Frequency $\omega$ \axisunit{rad/s}},
ymin=1.6,
ymax=26,
ylabel={Amplitude $\abs{\Delta}$}
% axis x line*=bottom,
% axis y line*=left,
]

\addplot[ycomb,mark=none,color=gray!25,forget plot]
table[row sep=crcr]{
0.837758040957278 12.6525858136145  1\\
0.907571211037051 19.7498852795118  2\\
0.977384381116825 21.817938976794   3\\
1.0471975511966 12.6615323622864    4\\
1.11701072127637 7.76219534512876   5\\
1.18682389135614 4.43297964103573   6\\
};

\addplot [firstgrid, forget plot]
table[row sep=crcr]{
x   y   index\\
0.837758040957278 12.6525858136145  1\\
};

\addplot [secondgrid, forget plot]
table[row sep=crcr]{
x   y   index\\
0.907571211037051 19.7498852795118  2\\
};

\addplot [thirdgrid, forget plot]
table[row sep=crcr]{
x   y   index\\
0.977384381116825 21.817938976794   3\\
};

\addplot [fourthgrid, forget plot]
table[row sep=crcr]{
x   y   index\\
1.0471975511966 12.6615323622864    4\\
};

\addplot [fifthgrid, forget plot]
table[row sep=crcr]{
x   y   index\\
1.11701072127637 7.76219534512876   5\\
};

\addplot [sixthgrid, forget plot]
table[row sep=crcr]{
x   y   index\\
1.18682389135614 4.43297964103573   6\\
};

% \addplot [grid, forget plot]
% table[row sep=crcr]{
% x   y   index\\
% 0.767944870877505 8.94875028468152  0\\
% 1.25663706143592 1.84419128906495   7\\
% 1.32645023151569 -0.282194626651631 8\\
% };

\addplot [dashed, black, forget plot]
table[row sep=crcr]{
x     y\\
0     22.5\\
2     22.5\\
};

% Model evaluations

% \addplot[domain=0.628319:0.767945,first]{70.1362*x^2 + -68.5975*x + 20.2658};
% \addplot[domain=0.698132:0.837758,second]{134.831*x^2 + -163.445*x + 54.9505};

\addplot[domain=0.767945:0.907571,first] { 348.128*x^2 + -505.936*x + 192.175 };
\addplot[domain=0.837758:0.977384,second]{-515.939*x^2 +  1002.14*x + -464.796};
\addplot[domain=0.907571:1.047200,third] {-1151.49*x^2 +  2200.13*x + -1028.56};
\addplot[domain=0.977384:1.117010,fourth]{ 436.723*x^2 + -1015.34*x + 597     };
\addplot[domain=1.047200:1.186820,fifth] { 161.075*x^2 + -418.778*x + 274.566 };
\addplot[domain=1.117010:1.256640,sixth] { 75.9587*x^2 + -222.684*x + 161.728 };

% \addplot[domain=1.18682:1.32645,first]{47.4368*x^2 + -152.992*x + 119.19};

\node[first]  (D1) at (axis cs:0.837758,10)   {$\hphantom{_1}\LocalModel{\Delta}_1$};
\node[second] (D2) at (axis cs:0.907571,17)   {$\hphantom{_2}\LocalModel{\Delta}_2$};
\node[third]  (D3) at (axis cs:0.977384,24) {$\hphantom{_3}\LocalModel{\Delta}_3$};
\node[fourth] (D4) at (axis cs:1.0472,10)     {$\hphantom{_4}\LocalModel{\Delta}_4$};
\node[fifth]  (D5) at (axis cs:1.11701,10)    {$\hphantom{_5}\LocalModel{\Delta}_5$};
\node[sixth]  (D6) at (axis cs:1.18682,6.5)     {$\hphantom{_6}\LocalModel{\Delta}_6$};

\node[color=black] (gamma) at (axis cs:1.15,21) {$\gamma_{\mathrm{IG}}$};

\end{axis}

\begin{axis}[%
name=Omega,
anchor=north west,
at={($(delta.south west) - (0pt,0.25em)$)},
width=\figurewidth,
height=0.18\figureheight,
unbounded coords=jump,
scale only axis,
xtick={0.83776,0.90757,0.97738,1.0472,1.117,1.1868},
xticklabels={},
ytick={0.75},
yticklabels={$\Omega_k$},
xmin=0.78,
xmax=1.25,
xminorticks=true,
yminorticks=false,
xmajorgrids,
ymin=-3.25,
ymax=4.75,
%ylabel={$\Omega_k$}
% axis x line*=bottom,
% axis y line*=left,
]

% Segmentation

\addplot[first,range]  coordinates { (0.771436,1.25) (0.904081,1.25) };
\addplot[second,range] coordinates { (0.841249,0.25) (0.973894,0.25) };
\addplot[third,range]  coordinates { (0.911062,1.25) (1.04371,1.25) };
\addplot[fourth,range] coordinates { (0.980875,0.25) (1.11352,0.25) };
\addplot[fifth,range]  coordinates { (1.05069,1.25)  (1.18333,1.25) };
\addplot[sixth,range]  coordinates { (1.1205,0.25)   (1.25315,0.25) };

\node[first]  (W1) at (axis cs:0.837758,3)      {$\hphantom{_1}\Omega_1$};
\node[second] (W2) at (axis cs:0.907571,-1.55)  {$\hphantom{_2}\Omega_2$};
\node[third]  (W3) at (axis cs:0.977384,3)      {$\hphantom{_3}\Omega_3$};
\node[fourth] (W4) at (axis cs:1.0472,-1.55)    {$\hphantom{_4}\Omega_4$};
\node[fifth]  (W5) at (axis cs:1.11701,3)       {$\hphantom{_5}\Omega_5$};
\node[sixth]  (W6) at (axis cs:1.18682,-1.55)   {$\hphantom{_6}\Omega_6$};

\end{axis}

\begin{axis}[%
name=kk,
anchor=north west,
at={(Omega.south west)},
width=\figurewidth,
height=0.18\figureheight,
unbounded coords=jump,
scale only axis,
xtick={0.83776,0.90757,0.97738,1.0472,1.117,1.1868},
xticklabels={$\omega_1$,$\omega_2$,$\omega_3$,$\omega_4$,$\omega_5$,$\omega_6$},
ytick={2.5,-1.5},
yticklabels={$k_L(\omega)=$,$k_R(\omega)=$},
xmin=0.78,
xmax=1.25,
xminorticks=true,
yminorticks=false,
xmajorgrids,
xlabel={Frequency $\omega$ $[\mathrm{rad/s}]$},
ymin=-3.3,
ymax=4.7,
% axis x line*=bottom,
% axis y line*=left,
]

% Segmentation

\addplot[first,range] coordinates { (0.75000,0.10) (0.83776,0.10) };
\addplot[first,range] coordinates { (0.83776,1.40) (0.904081,1.40) };

\addplot[second,range] coordinates { (0.841249,0.10) (0.90757,0.10) };
\addplot[second,range] coordinates { (0.90757,1.40) (0.973894,1.40) };

\addplot[third,range] coordinates { (0.911062,0.10) (0.97738,0.10) };
\addplot[third,range] coordinates { (0.97738,1.40) (1.04371,1.40) };

\addplot[fourth,range] coordinates { (0.980875,0.10) (1.0472,0.10) };
\addplot[fourth,range] coordinates { (1.0472,1.40) (1.11352,1.40) };

\addplot[fifth,range] coordinates { (1.05069,0.10) (1.117,0.10) };
\addplot[fifth,range] coordinates { (1.117,1.40) (1.18333,1.40) };

\addplot[sixth,range] coordinates { (1.1205,0.10) (1.1868,0.10) };
\addplot[sixth,range] coordinates { (1.1868,1.40) (1.25315,1.40) };


% \addplot[second,range] coordinates { (0.841249,0.25) (0.973894,0.25) };
% \addplot[third,range] coordinates { (0.911062,1.25) (1.04371,1.25) };
% \addplot[first,range] coordinates { (0.980875,0.25) (1.11352,0.25) };
% \addplot[second,range] coordinates { (1.05069,1.25) (1.18333,1.25) };
% \addplot[third,range] coordinates { (1.1205,0.25) (1.25315,0.25) };

%
% 1       2       3       4      5     6   
% 0.83776,0.90757,0.97738,1.0472,1.117,1.1868


% interval centers
      % 0.66323
      % 0.73304
      % 0.80285
      % 0.87266
      % 0.94248
      %  1.0123
      %  1.0821
      %  1.1519
      %  1.2217
      %  1.2915
      %  1.3614

\node[first]  (L1) at (axis cs:0.87266,3)  {$1$};
\node[second] (L2) at (axis cs:0.94248,3)  {$2$};
\node[third]  (L3) at (axis cs:1.0123,3)   {$3$};
\node[fourth] (L4) at (axis cs:1.0821,3)   {$4$};
\node[fifth]  (L5) at (axis cs:1.1519,3)   {$5$};
\node[sixth]  (L6) at (axis cs:1.2217,3)   {$6$};

\node[first]  (R1) at (axis cs:0.80285,-1.5)  {$1$};
\node[second] (R2) at (axis cs:0.87266,-1.5)  {$2$};
\node[third]  (R3) at (axis cs:0.94248,-1.5)  {$3$};
\node[fourth] (R4) at (axis cs:1.0123,-1.5)   {$4$};
\node[fifth]  (R5) at (axis cs:1.0821,-1.5)   {$5$};
\node[sixth]  (R6) at (axis cs:1.1519,-1.5)   {$6$};

\end{axis}


\end{tikzpicture}%

  \caption{Illustration of the local models $\LocalModel{\Delta}_k$ around each frequency bin $\omega_k$ for $k\in\set{1,\ldots,6}$.
  The bottom part shows the equivalence of the frequency segments $\Omega_k$ and the intervals $\set{\omega \given{k_L(\omega) = k \vee k_R(\omega) = k}}$.}
  \TODO{top subplot has a line connecting the last to the first point...}
\label{fig:interpol-quantities}
\end{figure}

\vspace{-3em}
\paragraph*{Direct method}
% \subsubsection{Direct method}
One can implement the definition~\eqref{eq:gamma-inter-grid} of $\gamma_{\mathrm{IG}}$ directly.
Hence, it boils down to solving the optimization problems
\begin{equation}
  % \gamma_{\mathrm{IG}}(k) 
  % = 
      \max_{\omega \in \Omega_k} 
      \abs{\LocalModel{\Delta}_k(\omega)}
      \label{eq:gamma-direct}
\end{equation}
for $k \in \set{1,\ldots,N-1}$.
These are tractable since $\omega$ is one-dimensional and restricted to the intervals $\Omega_k$.
Unfortunately, this approach offers no insight in the behavior of $\abs{\Delta(\omega)}$, e.g. around the peak amplitude.

\vspace{-2em}
\paragraph*{Interpolation method}
To allow inspection of $\abs{\Delta(\omega)}$ in~\eqref{eq:HinfNormSISO} for arbitrary values of $\omega \in \Omega$, we propose an interpolation approach to estimate $\gamma$ and to allow visual inspection of $\abs{\Delta(\omega)}$.
It can be understood intuitively that due to the limited validity domain of the local models, an estimate of $\Delta(\omega)$ can only be obtained from $\LocalModel{\Delta}_k(\omega)$ when $\abs{\omega_k-\omega}$ is small.
To this end, two functions that indicate respectively the left and the right adjacent local models for frequency $\omega \in \Omega$ are defined:
\begin{align}
  k_L(\omega) & 
                \isdef
                  \max
                  \set{k 
                    \given{
                      \omega_k \in \DFTGrid, 
                      \omega_k \leq \omega}
                  }\\
  k_R(\omega) & 
                \isdef
                  \min
                  \set{k 
                    \given{
                      \omega_k \in \DFTGrid, 
                      \omega_k \geq \omega}
                  }
  \text{.}
  \label{eq:kL-kR-indexing-functions}
\end{align}
Using the $k_L(\omega)$ and $k_R(\omega)$ functions, two candidate values for $\abs{\Delta(\omega)}$ are obtained: $\abs{\LocalModel{\Delta}_{k_L(\omega)}(\omega)}$ and $\abs{\LocalModel{\Delta}_{k_R(\omega)}(\omega)}$.
In view of obtaining the maximum amplitude, these are `interpolated' to obtain a single estimate:
\begin{equation}
  \abs{\Delta_{\mathrm{LRM}}(\omega)}
    \isdef
    \sup \set{
       \abs{\LocalModel{\Delta}_{k_L(\omega)}(\omega)},
        \abs{\LocalModel{\Delta}_{k_R(\omega)}(\omega)}
    }
    \label{eq:DeltaLRMInterpolation}
\end{equation}
by retaining the largest both.
It follows that
\begin{align}
  \widehat{\norm[]{\Delta_{\mathrm{LRM}}}}_{\infty} 
    & \isdef
    \sup_{\omega\in\Omega}
    \abs{\Delta_{\mathrm{LRM}}(\omega)} \\
   &  =
      \sup_{\omega\in\Omega} \; 
      \sup \set{
                 \abs{\LocalModel{\Delta}_{k_L(\omega)}(\omega)}, 
                 \abs{\LocalModel{\Delta}_{k_R(\omega)}(\omega)}}
  \label{eq:gammaInterpolOptim}
  \text{.}
\end{align}

\vspace{-2em}
\paragraph*{Remarks:}
The optimization problems~\eqref{eq:gammaInterpolOptim} and~\eqref{eq:gamma-direct} can, e.g. be solved by griding the frequency $\omega$.
The peak amplitudes obtained using~\eqref{eq:gammaInterpolOptim} and~\eqref{eq:gamma-direct} are identical since the local models $\LocalModel{\Delta}_k$ are evaluated over identical frequency ranges
$% \begin{equation}
  \Omega_k = \set{\omega \given{k_L(\omega) = k \vee k_R(\omega) = k}}
$ % \end{equation}
for both methods (but in different order) and aggregated using the supremum operator, which is indifferent to the order of its arguments.
In \figref{fig:interpol-quantities} the functions $k_L(\omega)$ and $k_R(\omega)$ are visualized and the equivalence with $\Omega_k$ is apparent.

\subsection{The \glsentrydesc{LRM}}
\label{sec:LRM}
The local models $\LocalModel{\Delta}_k$ are estimated using the \gls{LRM}~\citep{McKelvey2012LRM}.
The \gls{LRM} exploits the fact that the transfer function $\Delta(\omega)$ and the leakage term $T_{\Delta}(\omega)$  in \eqref{eq:sysDeltaFiniteDataFD} are smooth functions of the frequency.
As such, $\Delta$ and $T_{\Delta}$ are approximated locally by rational functions around each bin $k$:
\begin{align}
  \Delta(\omega_{k+r})     
    &\approx 
        \frac{\sum_{i=0}^{N_{B}} b_{i}(k) r^{i}}{1 + \sum_{i=1}^{N_{A}} a_{i}(k) r^{i}} 
      = 
        \frac{B_k(r)}{A_k(r)} 
      \isdef 
        \LocalModel{\Delta}_{k}(\omega_{k+r})
      \label{eq:LRMTaylorSystem} \\
  T_{\Delta}(\omega_{k+r}) 
    &\approx 
        \frac{\sum_{i=0}^{N_{T}} t_{i}(k) r^{i}}{1 + \sum_{i=1}^{N_{A}} a_{i}(k) r^{i}}  
      = 
        \frac{T_k'(r)}{A_k(r)} 
      \isdef 
        \LocalModel{T}_{k}(\omega_{k+r})
  \label{eq:LRMTaylorTransient}
  \text{.}
\end{align}
Its local parameters $a_{i}(k)$, $b_{i}(k)$ and $t_{i}(k)$ are estimated by considering~\eqref{eq:sysDeltaFiniteDataFD} in the local window $\LocalWindow = \set{-N_W,\ldots,+N_W}$ around each bin $k$.
This yields
\begin{equation}
  Y_{\Delta}(k+r) 
                 \approx
                    \Delta(\omega_{k+r}) U_{\Delta}(k+r) 
                    + T_{\Delta}(\omega_{k+r}) 
    \quad
    \forall r \in \LocalWindow
    \text{.}
    \label{eq:LocalWindowSpectra}
\end{equation}
By substituting the local models $\LocalModel{\Delta}_k$ and $\LocalModel{T}_k$, one obtains
\begin{align}
  Y_{\Delta}(k+r) 
                  & \approx
                    \LocalModel{\Delta}_k(\omega_{k+r}) U_{\Delta}(k+r) 
                    + \LocalModel{T}_k(\omega_{k+r}) 
                    & \forall r \in \LocalWindow \\
                  & = 
                    \frac{B_k(r)}{A_k(r)} U_{\Delta}(k+r) 
                    + 
                    \frac{T'_k(r)}{A_k(r)} 
                    & \forall r \in \LocalWindow
    \text{.}
    \label{eq:LocalWindowSpectraLRM}
\end{align}
To make this expression linear in the parameters $a_i(k)$, both sides are multiplied by $A_k(r)$ as in~\citep{Levy1959}.
By doing so, the local linear-least-squares cost function
\begin{equation}
  \sum_{r=-N_W}^{N_W} 
  \left| 
    A_k(r) Y_{\Delta}(k+r) - B_k(r) U_{\Delta}(k+r) - T'_{k}(r)
  \right|^2
  \label{eq:localCostLRM}
\end{equation}
 around each bin $k$ can be constructed.
We denote $\lrm{N_W,N_B,N_A,N_T}$ to be the \gls{LRM} with the orders $N_B$, $N_A$, $N_T$ and bandwidth $N_W$ defined above.
For $N_A=0$ the \gls{LRM} simplifies to the \gls{LPM} which we shall denote as $\lpm{N_W,N_B,N_T}$.

For a more detailed overview of  the \gls{LPM} and \gls{LRM} see~\citep{Schoukens2009LPM,Pintelon2010LPM1,Pintelon2010LPM2,McKelvey2012LRM}.


\section{Simulation}
\label{sec:simulation}
To allow the generalization of the results of the simulations, we first introduce a way to decompose complex systems into simpler systems.
The behavior of these simpler systems then allows to predict the behavior for these more complex systems.

\TODO{link to chapter 2: decoupling of information matrix}

Any real-rational system $\Delta$ can be decomposed in a parallel structure of lower-order sub-systems.
As a result, systems can often be well-represented, in a small frequency range, by a sum of a few low-order sub-systems (often first or second order for e.g. mechanical systems)~\citep[Section 2.2.2]{Gawronski2004}.
To estimate $\norm{\Delta}$ mainly the resonant sub-systems are of interest.
Assuming such resonances are sufficiently separated in the frequency,~\citep{Schoukens2013LPMerror,Geerardyn2013TIM,Gawronski2004} show that results for a single resonance are applicable to more complex systems.
This is done by considering only the dominant resonance -- \emph{viz.} the one that is the most relevant for the $\Hinf{}$ controller design.
By normalizing the conclusions for the time constant $\tau$ (or the $3\unit{dB}$ bandwidth), the results are generally valid for different values of the damping $\xi$.
Specifically, the system's $\tau \isdef (\xi\omega_{\mathrm{n}})^{-1}$ dictates the required measurement time~\citep{Schoukens2013LPMerror}, where $\xi$ is the relative damping and $\omega_{\mathrm{n}}$ the undamped natural frequency.

\subsection{A concrete simulation}
Consider a discrete time system $\Delta$ (see \figref{fig:exampleIntergrid}) as unmodeled dynamics in an output-error setting:
\begin{equation}
  \Delta(z) 
    =  \frac{0.45373 z + 0.44752}
            {z^2 - 1.0595 z + 0.96079}
  \label{eq:O2sysDT}
  \text{,}
\end{equation}
which has $\tau = 50 \unit{samples}$ (i.e. $\xi=0.02$ and $\omega_{\mathrm{n}} = 1 \unit{rad/s}$).
The input signal $u_{\Delta}$ is white Gaussian noise with unit variance and the disturbing noise variance $\sigma_v^2$ is chosen such that a \gls{SNR} 
\begin{equation}
  \SNR = \frac{\sqrt{\inv{N}\sum_n y_{\Delta0}^2(n) }}%
              {\sqrt{\inv{N}\sum_n v^2(n) }}
       = \frac{\sigma_{y_{\Delta0}}}{\sigma_v}
       \approx \frac{\norm[2]{\Delta} \sigma_{u_{\Delta}}}{\sigma_v}
  \label{eq:SNR}
\end{equation}
of $10$ is obtained.

Classical \gls{SA} techniques~\citep{Bendat1980} can estimate the \gls{FRF} of $\Delta$.
This is done by first splitting the input/output signals into $N_S$ segments of ${N}/{N_S}$ samples, windowing the segments to reduce leakage and applying the \gls{DFT}.
By dividing the resulting output/input spectra, the \gls{FRF} is obtained.
In this paper a Hann window and $N_S=1$ are used.
Note that $N_S>1$  increases the measurement time by a factor $N_S$ to attain the same frequency resolution (but would improve  transient and noise rejection).

\figref{fig:exampleIntergrid} illustrates for $N=97$ that $\gamma_{\mathrm{FRF}}$ for the \gls{SA}, $\lpm{5,4,4}$ and $\lrm{5,2,2,2}$ approaches yield unreliable estimates of $\infnorm{\Delta}$, since the frequency resolution is too coarse.
The specific orders of  \gls{LRM}/\gls{LPM} have been chosen to  keep $N_W$ small while not restricting ourselves to linear models ($N_B=1$). 
Slightly different orders (e.g. $\lrm{4,1,1,1}$) yield comparable results.

\begin{figure}
  \centering
  \setlength{\figurewidth}{0.75\columnwidth}
  \setlength{\figureheight}{0.68\figurewidth}
  % This file was created by matlab2tikz v0.4.7 (commit f204e544bf36a123713b8051de724c0d3f0daa30) running on MATLAB 8.3.
% Copyright (c) 2008--2014, Nico Schlömer <nico.schloemer@gmail.com>
% All rights reserved.
% Minimal pgfplots version: 1.3
% 
% The latest updates can be retrieved from
%   http://www.mathworks.com/matlabcentral/fileexchange/22022-matlab2tikz
% where you can also make suggestions and rate matlab2tikz.
% 
% workdir  : /Users/egon/Dropbox/VUB/PhD/LPMHinf/Code/MCLARX
% stack    : /Users/egon/Dropbox/VUB/PhD/LPMHinf/Code/MCLARX/test_hinf_length_mc_paper.m (50)
% git info : commit a8ab01207283cbc86f952083a2d49e011faa50b4
%            Author: Egon Geerardyn <egon.geerardyn@gmail.com>
%            Date:   Mon Aug 18 11:33:44 2014 +0200
%            
%                Hinf vs runlength: MC for paper
%            
%             test_hinf_length_mc_paper.m | 172 ++++++++++++++++++++++++++------------------
%             1 file changed, 102 insertions(+), 70 deletions(-)
%            
% 
% 
%
\begin{tikzpicture}


\begin{axis}[%
width=0.5\figurewidth,
height=\figureheight,
unbounded coords=jump,
scale only axis,
xmin=0.625,
xmax=1.375,
xmajorgrids,
xminorgrids,
title={$\gamma_{\mathrm{FRF}}$},
xlabel={$\omega$ \axisunit{rad/s}},
ymin=0,
ytick={20,10,0,24,14},
xtick={0.75,1,1.25},
ymajorgrids,
yminorgrids,
ymax=25,
name=gammaFRF,
ylabel={Amplitude $\abs{\Delta(\omega)}$}
]

\addplot [truesys,hinfnorm,forget plot]
  table[]{\thisDir/figs/example-true-hinf.tsv};
\label{leg:example-true-Hinf}

\addplot [truesys]
  table[]{\thisDir/figs/example-true.tsv};
\label{leg:example-true}

\addplot [SA,hinfnorm]
  table[]{\thisDir/figs/example-sa-hinf.tsv};
\label{leg:example-SA-hinfnorm}
\addplot [SAgrid]
  table[]{\thisDir/figs/example-sa-frf.tsv};
\label{leg:example-SA}

\addplot [LRM, hinfnorm, dash phase=2pt]
  table[]{\thisDir/figs/example-lrm-gammafrf.tsv};
\label{leg:example-LRM-hinfnorm}

\addplot [LRMgrid]
  table[]{\thisDir/figs/example-lrm-frf.tsv};
\label{leg:example-LRM}

\addplot [LPM, hinfnorm]
  table[]{\thisDir/figs/example-lpm-gammafrf.tsv};
\label{leg:example-LPM-hinfnorm}

\addplot [LPMgrid]
  table[]{\thisDir/figs/example-lpm-frf.tsv};
\label{leg:example-LPM}

\node at (axis cs:0.75,15.5) [color=SA] {SA};
\node at (axis cs:0.75,12.9) [color=LRM] {LRM};
\node at (axis cs:0.75,11) [color=LPM] {LPM};
\node at (axis cs:0.75,22) [color=truesys] {$\hinfnorm{\Delta}$};

\end{axis}

\begin{axis}[%
anchor=north west,
at={(gammaFRF.north east)},
width=0.5\figurewidth,
height=\figureheight,
scale only axis,
xmin=0.625,
xmax=1.375,
ytick={20,10,0,24,14},
yticklabels={},
xtick={0.75,1,1.25},
xmajorgrids,
xminorgrids,
title={$\gamma_{\mathrm{IG}}$},
xlabel={$\omega$ \axisunit{rad/s}},
ymin=0,
ymax=25,
ymajorgrids,
yminorgrids,
name=gammaIG,
xshift=1em,
%ylabel={Amplitude $\abs{\Delta}$}
]

\addplot [truesys,hinfnorm]
  table[]{\thisDir/figs/example-true-hinf.tsv};

\addplot [truesys,forget plot]
  table[]{\thisDir/figs/example-true.tsv};


\addplot [SA,hinfnorm,forget plot]
  table[]{\thisDir/figs/example-sa-hinf.tsv};
\addplot [SAgrid,forget plot]
  table[]{\thisDir/figs/example-sa-frf.tsv};

\addplot [LRM,hinfnorm,forget plot,dash phase=2pt]
  table[]{\thisDir/figs/example-lrm-gammaig.tsv};
\label{leg:example-LRM-interpol-Hinf}

\addplot [LRM,interpol,forget plot]
  table[]{\thisDir/figs/example-lrm-interpol.tsv};
\addplot [LRMgrid,forget plot]
  table[]{\thisDir/figs/example-lrm-frf.tsv};

%placeholder
\addplot [LRMgrid,interpol,forget plot]
  table[row sep=crcr]{%
10	10\\
10	10\\
};
\label{leg:example-LRM-interpol}

\addplot [LPM,hinfnorm,forget plot]
  table[]{\thisDir/figs/example-lpm-gammaig.tsv};
\addplot [LPM,interpol,forget plot]
  table[]{\thisDir/figs/example-lpm-interpol.tsv};
\addplot [LPMgrid,forget plot]
  table[]{\thisDir/figs/example-lpm-frf.tsv};

%placeholder
\addplot [LPMgrid,interpol,forget plot]
  table[row sep=crcr]{%
  10	10\\
  10	10\\
};
\label{leg:example-LPM-interpol}

\node at (axis cs:0.75,15.5) [color=SA] {SA};
\node at (axis cs:0.75,22) [color=LRM] {LRM};
\node at (axis cs:0.75,11.7) [color=LPM] {LPM};
\node at (axis cs:1.25,22) [color=truesys] {$\hinfnorm{\Delta}$};

\end{axis}
\end{tikzpicture}%

  %% This file was created by matlab2tikz v0.4.6 (commit 8d91398ecbfc95798cb61f48865d0104fd3f821d) running on MATLAB 8.2.
% Copyright (c) 2008--2014, Nico Schlömer <nico.schloemer@gmail.com>
% All rights reserved.
% Minimal pgfplots version: 1.3
% 
% The latest updates can be retrieved from
%   http://www.mathworks.com/matlabcentral/fileexchange/22022-matlab2tikz
% where you can also make suggestions and rate matlab2tikz.
% 
\begin{tikzpicture}

\begin{axis}[%
width=\figurewidth,
height=\figureheight,
unbounded coords=jump,
%scale only axis,
%xmode=log,
xmin=0.35,
xmax=1.3,
xminorticks=true,
xlabel={Frequency \axisunit{rad/s}},
xmajorgrids,
xminorgrids,
ymin=0,
ymax=30,
ylabel={Amplitude \axisunit{dB}},
ymajorgrids,
%axis x line*=bottom,
%axis y line*=left,
legend style={at={(0.05,0.95)},anchor=north west,draw=black,fill=white,legend cell align=left,font=\tiny}
]
\addplot [truesys]
  table[row sep=crcr]{
0.0698131700797732	0.045375683852626	\\
0.0767944870877505	0.0549366275306511	\\
0.0837758040957278	0.0654209684346938	\\
0.0907571211037051	0.0768322440905536	\\
0.0977384381116825	0.0891743163969068	\\
0.10471975511966	0.102451375927103	\\
0.111701072127637	0.116667946621249	\\
0.118682389135614	0.131828890878523	\\
0.125663706143592	0.147939415062751	\\
0.132645023151569	0.165005075434294	\\
0.139626340159546	0.183031784522257	\\
0.146607657167524	0.202025817952517	\\
0.153588974175501	0.221993821749038	\\
0.160570291183478	0.242942820125791	\\
0.167551608191456	0.264880223789532	\\
0.174532925199433	0.287813838773729	\\
0.18151424220741	0.311751875826985	\\
0.188495559215388	0.336702960380364	\\
0.195476876223365	0.36267614311862	\\
0.202458193231342	0.389680911185167	\\
0.20943951023932	0.417727200049285	\\
0.216420827247297	0.446825406069365	\\
0.223402144255274	0.476986399785517	\\
0.230383461263251	0.508221539979957	\\
0.237364778271229	0.540542688544463	\\
0.244346095279206	0.573962226198319	\\
0.251327412287183	0.608493069101712	\\
0.258308729295161	0.644148686415861	\\
0.265290046303138	0.680943118860796	\\
0.272271363311115	0.718890998329471	\\
0.279252680319093	0.758007568618609	\\
0.28623399732707	0.798308707342471	\\
0.293215314335047	0.839810949100126	\\
0.300196631343025	0.88253150997275	\\
0.307177948351002	0.926488313432458	\\
0.314159265358979	0.971700017752141	\\
0.321140582366957	1.01818604501028	\\
0.328121899374934	1.06596661179418	\\
0.335103216382911	1.11506276171303	\\
0.342084533390889	1.16549639983947	\\
0.349065850398866	1.21729032920996	\\
0.356047167406843	1.27046828952416	\\
0.363028484414821	1.32505499819416	\\
0.370009801422798	1.38107619390854	\\
0.376991118430775	1.43855868288904	\\
0.383972435438753	1.49753038803351	\\
0.39095375244673	1.55802040115447	\\
0.397935069454707	1.62005903854225	\\
0.404916386462684	1.68367790010046	\\
0.411897703470662	1.74890993232526	\\
0.418879020478639	1.81578949542217	\\
0.425860337486616	1.88435243488402	\\
0.432841654494594	1.95463615788032	\\
0.439822971502571	2.02667971484414	\\
0.446804288510548	2.10052388667748	\\
0.453785605518526	2.17621127803665	\\
0.460766922526503	2.25378641720607	\\
0.46774823953448	2.3332958631164	\\
0.474729556542458	2.41478832012103	\\
0.481710873550435	2.49831476120619	\\
0.488692190558412	2.58392856038023	\\
0.49567350756639	2.67168563506436	\\
0.502654824574367	2.7616445993973	\\
0.509636141582344	2.85386692946031	\\
0.516617458590322	2.94841714154404	\\
0.523598775598299	3.04536298469895	\\
0.530580092606276	3.14477564895401	\\
0.537561409614253	3.24672999074625	\\
0.544542726622231	3.35130477728228	\\
0.551524043630208	3.4585829517585	\\
0.558505360638185	3.56865192159773	\\
0.565486677646163	3.68160387212293	\\
0.57246799465414	3.79753610839225	\\
0.579449311662117	3.91655142826289	\\
0.586430628670095	4.03875853014665	\\
0.593411945678072	4.16427245937592	\\
0.600393262686049	4.29321509762025	\\
0.607374579694027	4.42571570039934	\\
0.614355896702004	4.561911488438	\\
0.621337213709981	4.70194829941977	\\
0.628318530717959	4.84598130764073	\\
0.635299847725936	4.99417582016866	\\
0.642281164733913	5.14670815940178	\\
0.649262481741891	5.30376664343532	\\
0.656243798749868	5.46555267742696	\\
0.663225115757845	5.63228197125704	\\
0.670206432765823	5.80418590127215	\\
0.6771877497738	5.98151303686404	\\
0.684169066781777	6.16453085617252	\\
0.691150383789754	6.35352767942913	\\
0.698131700797732	6.54881485354423	\\
0.705113017805709	6.75072922766719	\\
0.712094334813686	6.9596359668713	\\
0.719075651821664	7.17593176013139	\\
0.726056968829641	7.40004848976753	\\
0.733038285837618	7.6324574430131	\\
0.740019602845596	7.87367416295683	\\
0.747000919853573	8.12426405661205	\\
0.75398223686155	8.38484890330716	\\
0.760963553869528	8.65611443828931	\\
0.767944870877505	8.93881922608364	\\
0.774926187885482	9.23380508792667	\\
0.78190750489346	9.5420094102804	\\
0.788888821901437	9.86447974056574	\\
0.795870138909414	10.2023911762669	\\
0.802851455917391	10.5570671798792	\\
0.809832772925369	10.9300046111542	\\
0.816814089933346	11.3229039665366	\\
0.823795406941323	11.7377060593788	\\
0.830776723949301	12.1766366646257	\\
0.837758040957278	12.6422609763884	\\
0.844739357965256	13.1375500427689	\\
0.851720674973233	13.665961533235	\\
0.85870199198121	14.2315369795546	\\
0.865683308989187	14.839016364134	\\
0.872664625997165	15.4939671338678	\\
0.879645943005142	16.2029150288334	\\
0.886627260013119	16.9734407923245	\\
0.893608577021097	17.8141518331839	\\
0.900589894029074	18.7343087630001	\\
0.907571211037051	19.7425835952564	\\
0.914552528045029	20.8437236463356	\\
0.921533845053006	22.0303558445135	\\
0.928515162060983	23.2644179939997	\\
0.93549647906896	24.4413708521668	\\
0.942477796076938	25.3479719345953	\\
0.949459113084915	25.694622637891	\\
0.956440430092892	25.3216930827679	\\
0.96342174710087	24.3702127454847	\\
0.970403064108847	23.1292567021555	\\
0.977384381116824	21.8201665054904	\\
0.984365698124802	20.5538792995138	\\
0.991347015132779	19.3717146337776	\\
0.998328332140756	18.2824190435281	\\
1.00530964914873	17.281741641294	\\
1.01229096615671	16.3611378046015	\\
1.01927228316469	15.5113300466072	\\
1.02625360017267	14.7236444702485	\\
1.03323491718064	13.9904227412495	\\
1.04021623418862	13.3050651874542	\\
1.0471975511966	12.6619389333792	\\
1.05417886820457	12.0562478879792	\\
1.06116018521255	11.4839030254295	\\
1.06814150222053	10.941406625032	\\
1.07512281922851	10.4257538208759	\\
1.08210413623648	9.93435074562962	\\
1.08908545324446	9.4649471708484	\\
1.09606677025244	9.01558128552699	\\
1.10304808726042	8.58453442909683	\\
1.11002940426839	8.17029390727458	\\
1.11701072127637	7.77152234523675	\\
1.12399203828435	7.38703232572919	\\
1.13097335529233	7.01576530670775	\\
1.1379546723003	6.65677401454684	\\
1.14493598930828	6.3092076703723	\\
1.15191730631626	5.9722995354615	\\
1.15889862332423	5.64535636331829	\\
1.16587994033221	5.32774942646523	\\
1.17286125734019	5.01890684971016	\\
1.17984257434817	4.71830703222224	\\
1.18682389135614	4.4254729810354	\\
1.19380520836412	4.13996741079023	\\
1.2007865253721	3.86138849035672	\\
1.20776784238008	3.58936613779235	\\
1.21474915938805	3.32355878192681	\\
1.22173047639603	3.063650522545	\\
1.22871179340401	2.80934863229919	\\
1.23569311041199	2.560381352626	\\
1.24267442741996	2.31649594346226	\\
1.24965574442794	2.07745695276844	\\
1.25663706143592	1.84304467701321	\\
1.26361837844389	1.61305378806202	\\
1.27059969545187	1.3872921054886	\\
1.27758101245985	1.16557949632954	\\
1.28456232946783	0.947746886826053	\\
1.2915436464758	0.733635372825006	\\
1.29852496348378	0.523095417315403	\\
1.30550628049176	0.315986125107372	\\
1.31248759749974	0.112174585965931	\\
1.31946891450771	-0.0884647213758853	\\
1.32645023151569	-0.286050470941746	\\
1.33343154852367	-0.480694982773514	\\
1.34041286553164	-0.672504672821844	\\
1.34739418253962	-0.861580463945302	\\
1.3543754995476	-1.0480181619713	\\
1.36135681655558	-1.23190880031257	\\
1.36833813356355	-1.41333895623222	\\
1.37531945057153	-1.59239104150464	\\
1.38230076757951	-1.76914356991347	\\
1.38928208458749	-1.94367140376238	\\
1.39626340159546	-2.11604598134016	\\
1.40324471860344	-2.28633552707578	\\
1.41022603561142	-2.45460524593852	\\
1.4172073526194	-2.62091750347787	\\
1.42418866962737	-2.78533199275597	\\
1.43116998663535	-2.94790588930118	\\
1.43815130364333	-3.1086939950992	\\
1.4451326206513	-3.26774887253873	\\
1.45211393765928	-3.42512096914231	\\
1.45909525466726	-3.580858733833	\\
1.46607657167524	-3.73500872541678	\\
1.47305788868321	-3.88761571390017	\\
1.48003920569119	-4.03872277520344	\\
1.48702052269917	-4.18837137978107	\\
1.49400183970715	-4.33660147561494	\\
1.50098315671512	-4.48345156600567	\\
1.5079644737231	-4.62895878254932	\\
1.51494579073108	-4.77315895365456	\\
1.52192710773906	-4.91608666892569	\\
1.52890842474703	-5.05777533970814	\\
1.53588974175501	-5.19825725607041	\\
1.54287105876299	-5.3375636404723	\\
1.54985237577096	-5.47572469835097	\\
1.55683369277894	-5.61276966583608	\\
1.56381500978692	-5.74872685478988	\\
1.5707963267949	-5.88362369535196	\\
1.57777764380287	-6.01748677615541	\\
1.58475896081085	-6.15034188236763	\\
1.59174027781883	-6.28221403169755	\\
1.59872159482681	-6.41312750850102	\\
1.60570291183478	-6.54310589610577	\\
1.61268422884276	-6.67217210746884	\\
1.61966554585074	-6.80034841427084	\\
1.62664686285871	-6.92765647454428	\\
1.63362817986669	-7.05411735892648	\\
1.64060949687467	-7.17975157562023	\\
1.64759081388265	-7.30457909414162	\\
1.65457213089062	-7.42861936792571	\\
1.6615534478986	-7.55189135586033	\\
1.66853476490658	-7.67441354280891	\\
1.67551608191456	-7.79620395918306	\\
1.68249739892253	-7.91728019961914	\\
1.68947871593051	-8.03765944081033	\\
1.69646003293849	-8.1573584585434	\\
1.70344134994647	-8.27639364398357	\\
1.71042266695444	-8.39478101925175	\\
1.71740398396242	-8.51253625233198	\\
1.7243853009704	-8.6296746713474	\\
1.73136661797837	-8.74621127823946	\\
1.73834793498635	-8.86216076188259	\\
1.74532925199433	-8.9775375106658	\\
1.75231056900231	-9.09235562456968	\\
1.75929188601028	-9.20662892676665	\\
1.76627320301826	-9.32037097476967	\\
1.77325452002624	-9.43359507115366	\\
1.78023583703422	-9.54631427387289	\\
1.78721715404219	-9.65854140619541	\\
1.79419847105017	-9.77028906627504	\\
1.80117978805815	-9.88156963638068	\\
1.80816110506613	-9.99239529180028	\\
1.8151424220741	-10.1027780094373	\\
1.82212373908208	-10.212729576116	\\
1.82910505609006	-10.3222615966106	\\
1.83608637309803	-10.4313855014138	\\
1.84306769010601	-10.5401125542573	\\
1.85004900711399	-10.6484538593999	\\
1.85703032412197	-10.7564203686924	\\
1.86401164112994	-10.8640228884349	\\
1.87099295813792	-10.9712720860357	\\
1.8779742751459	-11.078178496483	\\
1.88495559215388	-11.1847525286407	\\
1.89193690916185	-11.2910044713777	\\
1.89891822616983	-11.3969444995398	\\
1.90589954317781	-11.5025826797745	\\
1.91288086018578	-11.6079289762167	\\
1.91986217719376	-11.7129932560431	\\
1.92684349420174	-11.8177852949057	\\
1.93382481120972	-11.9223147822486	\\
1.94080612821769	-12.0265913265197	\\
1.94778744522567	-12.1306244602814	\\
1.95476876223365	-12.2344236452283	\\
1.96175007924163	-12.3379982771199	\\
1.9687313962496	-12.4413576906329	\\
1.97571271325758	-12.5445111641409	\\
1.98269403026556	-12.6474679244271	\\
1.98967534727354	-12.750237151337	\\
1.99665666428151	-12.8528279823753	\\
2.00363798128949	-12.9552495172544	\\
2.01061929829747	-13.0575108224002	\\
2.01760061530545	-13.1596209354194	\\
2.02458193231342	-13.2615888695353	\\
2.0315632493214	-13.3634236179968	\\
2.03854456632938	-13.4651341584662	\\
2.04552588333735	-13.5667294573911	\\
2.05250720034533	-13.6682184743659	\\
2.05948851735331	-13.7696101664878	\\
2.06646983436129	-13.8709134927129	\\
2.07345115136926	-13.9721374182176	\\
2.08043246837724	-14.0732909187709	\\
2.08741378538522	-14.1743829851215	\\
2.0943951023932	-14.2754226274076	\\
2.10137641940117	-14.3764188795926	\\
2.10835773640915	-14.4773808039326	\\
2.11533905341713	-14.5783174954821	\\
2.1223203704251	-14.6792380866428	\\
2.12930168743308	-14.7801517517617	\\
2.13628300444106	-14.8810677117829	\\
2.14326432144904	-14.981995238962	\\
2.15024563845701	-15.0829436616456	\\
2.15722695546499	-15.1839223691253	\\
2.16420827247297	-15.2849408165707	\\
2.17118958948095	-15.3860085300487	\\
2.17817090648892	-15.4871351116358	\\
2.1851522234969	-15.5883302446307	\\
2.19213354050488	-15.6896036988746	\\
2.19911485751286	-15.7909653361856	\\
2.20609617452083	-15.8924251159169	\\
2.21307749152881	-15.9939931006451	\\
2.22005880853679	-16.0956794619991	\\
2.22704012554476	-16.1974944866357	\\
2.23402144255274	-16.2994485823745	\\
2.24100275956072	-16.4015522844981	\\
2.2479840765687	-16.503816262231	\\
2.25496539357667	-16.606251325405	\\
2.26194671058465	-16.7088684313239	\\
2.26892802759263	-16.8116786918382	\\
2.27590934460061	-16.914693380642	\\
2.28289066160858	-17.0179239408052	\\
2.28987197861656	-17.1213819925541	\\
2.29685329562454	-17.2250793413139	\\
2.30383461263251	-17.3290279860294	\\
2.31081592964049	-17.4332401277777	\\
2.31779724664847	-17.5377281786917	\\
2.32477856365645	-17.6425047712086	\\
2.33175988066442	-17.7475827676651	\\
2.3387411976724	-17.8529752702566	\\
2.34572251468038	-17.9586956313817	\\
2.35270383168836	-18.0647574643938	\\
2.35968514869633	-18.171174654783	\\
2.36666646570431	-18.2779613718118	\\
2.37364778271229	-18.3851320806325	\\
2.38062909972027	-18.4927015549118	\\
2.38761041672824	-18.600684889993	\\
2.39459173373622	-18.7090975166277	\\
2.4015730507442	-18.8179552153077	\\
2.40855436775217	-18.9272741312355	\\
2.41553568476015	-19.0370707899691	\\
2.42251700176813	-19.1473621137807	\\
2.42949831877611	-19.2581654387749	\\
2.43647963578408	-19.3694985328085	\\
2.44346095279206	-19.4813796142642	\\
2.45044226980004	-19.5938273717276	\\
2.45742358680802	-19.7068609846253	\\
2.46440490381599	-19.8205001448835	\\
2.47138622082397	-19.9347650796702	\\
2.47836753783195	-20.0496765752925	\\
2.48534885483993	-20.1652560023188	\\
2.4923301718479	-20.2815253420096	\\
2.49931148885588	-20.3985072141389	\\
2.50629280586386	-20.5162249063002	\\
2.51327412287183	-20.6347024047937	\\
2.52025543987981	-20.7539644272033	\\
2.52723675688779	-20.8740364567769	\\
2.53421807389577	-20.9949447787344	\\
2.54119939090374	-21.1167165186364	\\
2.54818070791172	-21.2393796829604	\\
2.5551620249197	-21.362963202038	\\
2.56214334192768	-21.4874969755259	\\
2.56912465893565	-21.6130119205926	\\
2.57610597594363	-21.7395400230208	\\
2.58308729295161	-21.867114391443	\\
2.59006860995959	-21.9957693149451	\\
2.59704992696756	-22.1255403242957	\\
2.60403124397554	-22.2564642570798	\\
2.61101256098352	-22.3885793270421	\\
2.61799387799149	-22.5219251979738	\\
2.62497519499947	-22.6565430625056	\\
2.63195651200745	-22.7924757262069	\\
2.63893782901543	-22.9297676974262	\\
2.6459191460234	-23.0684652833539	\\
2.65290046303138	-23.2086166928316	\\
2.65988178003936	-23.3502721464872	\\
2.66686309704734	-23.4934839948345	\\
2.67384441405531	-23.6383068450381	\\
2.68082573106329	-23.7847976971206	\\
2.68780704807127	-23.9330160904705	\\
2.69478836507924	-24.0830242616006	\\
2.70176968208722	-24.2348873142113	\\
2.7087509990952	-24.3886734027296	\\
2.71573231610318	-24.5444539306257	\\
2.72271363311115	-24.7023037649597	\\
2.72969495011913	-24.8623014687791	\\
2.73667626712711	-25.0245295531765	\\
2.74365758413509	-25.1890747510409	\\
2.75063890114306	-25.3560283147766	\\
2.75762021815104	-25.5254863405539	\\
2.76460153515902	-25.6975501219759	\\
2.771582852167	-25.8723265364189	\\
2.77856416917497	-26.0499284677327	\\
2.78554548618295	-26.2304752694782	\\
2.79252680319093	-26.4140932734529	\\
2.7995081201989	-26.6009163489098	\\
2.80648943720688	-26.7910865186481	\\
2.81347075421486	-26.9847546390424	\\
2.82045207122284	-27.1820811521241	\\
2.82743338823081	-27.3832369190501	\\
2.83441470523879	-27.588404145733	\\
2.84139602224677	-27.7977774131008	\\
2.84837733925475	-28.0115648264643	\\
2.85535865626272	-28.2299893008494	\\
2.8623399732707	-28.4532900019959	\\
2.86932129027868	-28.6817239661194	\\
2.87630260728666	-28.9155679256199	\\
2.88328392429463	-29.1551203728397	\\
2.89026524130261	-29.400703899946	\\
2.89724655831059	-29.652667860269	\\
2.90422787531856	-29.911391405308	\\
2.91120919232654	-30.1772869625264	\\
2.91819050933452	-30.4508042325361	\\
2.9251718263425	-30.7324348010049	\\
2.93215314335047	-31.0227174815301	\\
2.93913446035845	-31.3222445319875	\\
2.94611577736643	-31.631668920078	\\
2.95309709437441	-31.951712856064	\\
2.96007841138238	-32.2831778648529	\\
2.96705972839036	-32.6269567394988	\\
2.97404104539834	-32.9840478091106	\\
2.98102236240632	-33.3555720733223	\\
2.98800367941429	-33.7427939129584	\\
2.99498499642227	-34.1471462964	\\
3.00196631343025	-34.5702616833119	\\
3.00894763043822	-35.0140102100945	\\
3.0159289474462	-35.4805472651324	\\
3.02291026445418	-35.9723732847248	\\
3.02989158146216	-36.4924096060224	\\
3.03687289847013	-37.0440956206048	\\
3.04385421547811	-37.6315144484726	\\
3.05083553248609	-38.2595571200753	\\
3.05781684949407	-38.934139077018	\\
3.06479816650204	-39.6624878828695	\\
3.07177948351002	-40.453527133994	\\
};
\addlegendentry{$\Delta$}

\addplot [LPMgrid, interpol]
table[row sep=crcr]{
10 10 \\ %placeholder
11 10\\
};

\addplot [LPMgrid, forget plot]
  table[row sep=crcr]{
0	-inf	\\
0.0698131700797732	2.31108760159992	\\
0.139626340159546	1.50492569621872	\\
0.20943951023932	1.95187394085855	\\
0.279252680319093	1.33658011195956	\\
0.349065850398866	0.504255301361752	\\
0.418879020478639	0.98782092400802	\\
0.488692190558412	1.52218314683421	\\
0.558505360638185	2.47087110084254	\\
0.628318530717959	2.92479966145322	\\
0.698131700797732	5.80424696192568	\\
0.767944870877505	9.80635821312046	\\
0.837758040957278	16.9923663843421	\\
0.907571211037051	13.5718812719946	\\
0.977384381116825	15.3885012766331	\\
1.0471975511966	14.7877782996466	\\
1.11701072127637	18.2882427538693	\\
1.18682389135614	4.80366815975026	\\
1.25663706143592	1.92230313251952	\\
1.32645023151569	-1.90794985773834	\\
1.39626340159546	-2.96488275819053	\\
1.46607657167524	-3.40580537021634	\\
1.53588974175501	-0.161285569754511	\\
1.60570291183478	-1.73742795337489	\\
1.67551608191456	-8.77576638107342	\\
1.74532925199433	-11.6142566897058	\\
1.8151424220741	-25.2259834171908	\\
1.88495559215388	-2.61579842686768	\\
1.95476876223365	-3.52206482364875	\\
2.02458193231342	-9.52822738405763	\\
2.0943951023932	-13.9100045283269	\\
2.16420827247297	-20.3168630139538	\\
2.23402144255274	-12.2912738487053	\\
2.30383461263251	-16.3859575664363	\\
2.37364778271229	-6.5133575346315	\\
2.44346095279206	-10.6387873347506	\\
2.51327412287183	-6.96595059533882	\\
2.58308729295161	-16.3975147112536	\\
2.65290046303138	-11.0470753831006	\\
2.72271363311115	-6.17912375560223	\\
2.79252680319093	-9.09392322002384	\\
2.8623399732707	-7.83121238616985	\\
2.93215314335047	-13.6485770614948	\\
3.00196631343025	-13.7936623179094	\\
3.07177948351002	-6.05767325451575	\\
};
\addlegendentry{$\abs{\Delta_{\mathrm{LPM}}}$}

\addplot [LRMgrid, interpol]
table[row sep=crcr]{
10 10 \\ %placeholder
11 10\\
};

\addplot [LRMgrid, forget plot]
  table[row sep=crcr]{
0	0.986417629553671	\\
0.0698131700797732	1.37672522055726	\\
0.139626340159546	1.86018472199117	\\
0.20943951023932	1.76752286431793	\\
0.279252680319093	0.568870173251582	\\
0.349065850398866	0.213568824382719	\\
0.418879020478639	1.00272008265142	\\
0.488692190558412	2.05186465188825	\\
0.558505360638185	3.22288073096945	\\
0.628318530717959	4.07032669513166	\\
0.698131700797732	5.80088957407497	\\
0.767944870877505	8.32244450636728	\\
0.837758040957278	12.4228045774123	\\
0.907571211037051	19.7234564198129	\\
0.977384381116825	21.1380442257445	\\
1.0471975511966	12.5473177097757	\\
1.11701072127637	7.94548443658294	\\
1.18682389135614	4.37469485480625	\\
1.25663706143592	0.541051420686472	\\
1.32645023151569	-1.40321536799081	\\
1.39626340159546	-2.76733486461552	\\
1.46607657167524	-3.1491922462568	\\
1.53588974175501	-4.40227527954903	\\
1.60570291183478	-5.96619447832785	\\
1.67551608191456	-8.89515200623487	\\
1.74532925199433	-12.3617538484781	\\
1.8151424220741	-8.94502066592128	\\
1.88495559215388	-5.71455217163913	\\
1.95476876223365	-8.76229200748787	\\
2.02458193231342	-10.9958828439363	\\
2.0943951023932	-11.0944719071164	\\
2.16420827247297	-14.7100249806446	\\
2.23402144255274	-13.9580668417838	\\
2.30383461263251	-11.0329301947376	\\
2.37364778271229	-12.5638142795047	\\
2.44346095279206	-15.6847678836759	\\
2.51327412287183	-27.613408237032	\\
2.58308729295161	-25.008772343106	\\
2.65290046303138	-26.8456349750616	\\
2.72271363311115	-18.2938884599623	\\
2.79252680319093	-19.8902771492208	\\
2.8623399732707	-18.2788663924916	\\
2.93215314335047	-19.4534811149452	\\
3.00196631343025	-26.1980004824653	\\
3.07177948351002	-24.7236524590019	\\
};
\addlegendentry{$\abs{\Delta_{\mathrm{LRM}}}$}

\addplot [SAgrid]
  table[row sep=crcr]{
0	0.510539663424993	\\
0.0698131700797732	7.72615455613385	\\
0.139626340159546	2.24956510457827	\\
0.20943951023932	2.24806492859574	\\
0.279252680319093	5.56243672345386	\\
0.349065850398866	1.27216540902612	\\
0.418879020478639	0.819227178060999	\\
0.488692190558412	1.68474881754219	\\
0.558505360638185	2.79806484714396	\\
0.628318530717959	4.55013886775328	\\
0.698131700797732	6.45823203509423	\\
0.767944870877505	7.08640335231377	\\
0.837758040957278	15.7752866189055	\\
0.907571211037051	17.9355622463041	\\
0.977384381116825	19.7359467917969	\\
1.0471975511966	13.8534135063362	\\
1.11701072127637	11.1392267397541	\\
1.18682389135614	11.167077565966	\\
1.25663706143592	0.0482058316579241	\\
1.32645023151569	-0.576636891539977	\\
1.39626340159546	-2.63953301818106	\\
1.46607657167524	-1.44355647457115	\\
1.53588974175501	-0.528136937978729	\\
1.60570291183478	-2.6341252014924	\\
1.67551608191456	-6.0790752945187	\\
1.74532925199433	-14.5774577669548	\\
1.8151424220741	-11.8960341688611	\\
1.88495559215388	-7.77656683751462	\\
1.95476876223365	-5.48692556617914	\\
2.02458193231342	-10.8699111829475	\\
2.0943951023932	-11.1230176160389	\\
2.16420827247297	5.17572966466912	\\
2.23402144255274	-1.18569646908639	\\
2.30383461263251	-10.4087047019207	\\
2.37364778271229	-11.7015960457992	\\
2.44346095279206	-5.63301319506837	\\
2.51327412287183	-9.43558745865738	\\
2.58308729295161	-14.6314142248177	\\
2.65290046303138	-7.72718709562491	\\
2.72271363311115	2.53636735660353	\\
2.79252680319093	-4.56375252433378	\\
2.8623399732707	-6.7468684345057	\\
2.93215314335047	-20.688996129188	\\
3.00196631343025	-11.4966858915958	\\
3.07177948351002	-6.65373473940696	\\
};
\addlegendentry{$\abs{\Delta_{\mathrm{SA}}}$}

\addplot [LPM, interpol, forget plot]
  table[row sep=crcr]{
0.0698131700797732	2.31108760159992	\\
0.0767944870877505	2.10240087947614	\\
0.0837758040957278	1.9252264001816	\\
0.0907571211037051	1.77967911511649	\\
0.0977384381116825	1.66505080410121	\\
0.10471975511966	1.57985760756151	\\
0.111701072127637	1.52194361784296	\\
0.118682389135614	1.48862489112435	\\
0.125663706143592	1.47685400676704	\\
0.132645023151569	1.48338544210901	\\
0.139626340159546	1.50492569621872	\\
0.146607657167524	1.53825766198992	\\
0.153588974175501	1.58033453872474	\\
0.160570291183478	1.62834339119502	\\
0.167551608191456	1.67974173397295	\\
0.174532925199433	1.73227223525305	\\
0.18151424220741	1.78396110157405	\\
0.188495559215388	1.83310534483564	\\
0.195476876223365	1.87825333547482	\\
0.202458193231342	1.918182102917	\\
0.20943951023932	1.95187394085855	\\
0.216420827247297	1.97849410565743	\\
0.223402144255274	1.99737079419964	\\
0.230383461263251	2.00797814937289	\\
0.237364778271229	2.00992274486379	\\
0.244346095279206	2.00293381825799	\\
0.251327412287183	1.98685742442808	\\
0.258308729295161	1.96165464484886	\\
0.265290046303138	1.92740399089786	\\
0.272271363311115	1.88430816061697	\\
0.279252680319093	1.33658011195956	\\
0.28623399732707	1.30043417521182	\\
0.293215314335047	1.26334870927235	\\
0.300196631343025	1.22529330267912	\\
0.307177948351002	1.18623770207699	\\
0.314159265358979	1.14615179639537	\\
0.321140582366957	1.105005601779	\\
0.328121899374934	1.06276924743798	\\
0.335103216382911	1.01941296259565	\\
0.342084533390889	0.974907064727745	\\
0.349065850398866	0.504255301361752	\\
0.356047167406843	0.508267942203304	\\
0.363028484414821	0.516796919576791	\\
0.370009801422798	0.529924828335083	\\
0.376991118430775	0.547749771109636	\\
0.383972435438753	0.586991830642205	\\
0.39095375244673	0.662374177784386	\\
0.397935069454707	0.740448413950844	\\
0.404916386462684	0.820900032876409	\\
0.411897703470662	0.903443365651242	\\
0.418879020478639	0.98782092400802	\\
0.425860337486616	1.14032945158328	\\
0.432841654494594	1.16118385738645	\\
0.439822971502571	1.24978576977674	\\
0.446804288510548	1.33945229371312	\\
0.453785605518526	1.43004947872129	\\
0.460766922526503	1.52146385076236	\\
0.46774823953448	1.61360092757855	\\
0.474729556542458	1.70638375304179	\\
0.481710873550435	1.79975146570746	\\
0.488692190558412	1.52218314683421	\\
0.49567350756639	1.59863395238864	\\
0.502654824574367	1.68063523180501	\\
0.509636141582344	1.76796491230363	\\
0.516617458590322	1.86043802632025	\\
0.523598775598299	1.95790377293258	\\
0.530580092606276	2.06024211815014	\\
0.537561409614253	2.16736006230451	\\
0.544542726622231	2.27918768975417	\\
0.551524043630208	2.39567410398769	\\
0.558505360638185	2.47087110084254	\\
0.565486677646163	2.62095652032053	\\
0.57246799465414	2.77545048739597	\\
0.579449311662117	2.93410537496118	\\
0.586430628670095	3.09667668525844	\\
0.593411945678072	3.26292365006071	\\
0.600393262686049	3.43260978501729	\\
0.607374579694027	3.60550339728434	\\
0.614355896702004	3.78137804563278	\\
0.621337213709981	3.96001295241427	\\
0.628318530717959	2.92479966145322	\\
0.635299847725936	3.33470864427943	\\
0.642281164733913	3.52049449935521	\\
0.649262481741891	3.78604248202754	\\
0.656243798749868	4.08489340465576	\\
0.663225115757845	4.38852799279113	\\
0.670206432765823	4.69630388646527	\\
0.6771877497738	5.00759788328816	\\
0.684169066781777	5.32181017069348	\\
0.691150383789754	5.63836789278599	\\
0.698131700797732	5.80424696192568	\\
0.705113017805709	6.16781077799976	\\
0.712094334813686	6.54302999065891	\\
0.719075651821664	6.92813835534832	\\
0.726056968829641	7.32142991761322	\\
0.733038285837618	7.72128407906962	\\
0.740019602845596	8.13307713659901	\\
0.747000919853573	8.55014455670471	\\
0.75398223686155	8.96789383081608	\\
0.760963553869528	9.38647758551537	\\
0.767944870877505	9.80635821312046	\\
0.774926187885482	15.8115477926685	\\
0.78190750489346	16.1482645508656	\\
0.788888821901437	16.4258405372507	\\
0.795870138909414	16.6491148400395	\\
0.802851455917391	16.8217547983456	\\
0.809832772925369	16.9464572257327	\\
0.816814089933346	17.0250874005672	\\
0.823795406941323	17.0587700816707	\\
0.830776723949301	17.0479414025374	\\
0.837758040957278	16.9923663843421	\\
0.844739357965256	16.8911233473168	\\
0.851720674973233	16.7425531420243	\\
0.85870199198121	16.5441672986467	\\
0.865683308989187	16.2925041774081	\\
0.872664625997165	15.9829148950199	\\
0.879645943005142	15.6092493339277	\\
0.886627260013119	15.1633935122777	\\
0.893608577021097	14.6345764446513	\\
0.900589894029074	14.00830400294	\\
0.907571211037051	13.5718812719946	\\
0.914552528045029	13.1797488631955	\\
0.921533845053006	12.727813961015	\\
0.928515162060983	12.2078300505237	\\
0.93549647906896	11.6092827148919	\\
0.942477796076938	10.9583118136066	\\
0.949459113084915	12.3117693434393	\\
0.956440430092892	13.3667169780741	\\
0.96342174710087	14.2015634943303	\\
0.970403064108847	14.864905317746	\\
0.977384381116824	15.3885012766331	\\
0.984365698124802	15.7937979847162	\\
0.991347015132779	16.0954779242372	\\
0.998328332140756	16.3034939965325	\\
1.00530964914873	16.4242680254966	\\
1.01229096615671	16.4613876483302	\\
1.01927228316469	16.4159710794851	\\
1.02625360017267	16.2867806756591	\\
1.03323491718064	16.0701100734921	\\
1.04021623418862	15.759423100654	\\
1.0471975511966	14.7877782996466	\\
1.05417886820457	14.7976705115324	\\
1.06116018521255	15.0757192999404	\\
1.06814150222053	15.6762903965388	\\
1.07512281922851	16.2191598585882	\\
1.08210413623648	16.7021588413349	\\
1.08908545324446	17.1263597384769	\\
1.09606677025244	17.4942277571309	\\
1.10304808726042	17.8086566283838	\\
1.11002940426839	18.0724814160955	\\
1.11701072127637	18.2882427538693	\\
1.12399203828435	18.4580807409252	\\
1.13097335529233	18.5836934090547	\\
1.1379546723003	18.6663250197412	\\
1.14493598930828	18.7067656008809	\\
1.15191730631626	18.7053517194944	\\
1.15889862332423	18.661963066269	\\
1.16587994033221	18.5760119217723	\\
1.17286125734019	18.4464240758154	\\
1.17984257434817	18.2716110051584	\\
1.18682389135614	4.80366815975026	\\
1.19380520836412	5.0154531392343	\\
1.2007865253721	4.67050641900931	\\
1.20776784238008	4.32534073654824	\\
1.21474915938805	3.98018787048818	\\
1.22173047639603	3.6352781143525	\\
1.22871179340401	3.29083927104591	\\
1.23569311041199	2.9470966052715	\\
1.24267442741996	2.60427420279098	\\
1.24965574442794	2.26259828458217	\\
1.25663706143592	1.92230313251952	\\
1.26361837844389	1.58364039946559	\\
1.27059969545187	1.24689269702287	\\
1.27758101245985	0.912392471603539	\\
1.28456232946783	0.580547280588519	\\
1.2915436464758	0.251872641029024	\\
1.29852496348378	-0.0729663980671376	\\
1.30550628049176	-0.393103980888839	\\
1.31248759749974	-0.707411918066612	\\
1.31946891450771	-1.01442820244989	\\
1.32645023151569	-1.90794985773834	\\
1.33343154852367	-2.13456390988108	\\
1.34041286553164	-2.33338475678761	\\
1.34739418253962	-2.50293457707721	\\
1.3543754995476	-2.6421016144094	\\
1.36135681655558	-2.75017019517469	\\
1.36833813356355	-2.82682863522348	\\
1.37531945057153	-2.87215495778901	\\
1.38230076757951	-2.88658302235325	\\
1.38928208458749	-2.87085382043529	\\
1.39626340159546	-2.96488275819053	\\
1.40324471860344	-2.63999129000149	\\
1.41022603561142	-2.75655167039383	\\
1.4172073526194	-2.86551297216596	\\
1.42418866962737	-2.85119591300361	\\
1.43116998663535	-2.79386906675506	\\
1.43815130364333	-2.72522231572719	\\
1.4451326206513	-2.64473740083906	\\
1.45211393765928	-2.5515971151641	\\
1.45909525466726	-2.44474653702486	\\
1.46607657167524	-2.32296576173803	\\
1.47305788868321	-0.455141010518673	\\
1.48003920569119	-0.35303333731099	\\
1.48702052269917	-0.268561048251172	\\
1.49400183970715	-0.201504771396799	\\
1.50098315671512	-0.151709519522342	\\
1.5079644737231	-0.11908918832313	\\
1.51494579073108	-0.103630722912442	\\
1.52192710773906	-0.105398405440326	\\
1.52890842474703	-0.124538639722005	\\
1.53588974175501	-0.161285569754511	\\
1.54287105876299	-0.215967860970693	\\
1.54985237577096	-0.289016992629513	\\
1.55683369277894	-0.380977456520952	\\
1.56381500978692	-0.492519333743644	\\
1.5707963267949	-0.62445383347432	\\
1.57777764380287	-0.777752535120271	\\
1.58475896081085	-0.953571293069217	\\
1.59174027781883	-1.15328006397755	\\
1.59872159482681	-1.37850033380238	\\
1.60570291183478	-1.73742795337489	\\
1.61268422884276	-1.89319726871827	\\
1.61966554585074	-2.06783409357053	\\
1.62664686285871	-2.26252278586202	\\
1.63362817986669	-2.47862671565196	\\
1.64060949687467	-2.7177209831612	\\
1.64759081388265	-2.98163280092615	\\
1.65457213089062	-3.27249180889896	\\
1.6615534478986	-3.59279336471292	\\
1.66853476490658	-3.94547892777797	\\
1.67551608191456	-8.77576638107342	\\
1.68249739892253	-7.25123927407367	\\
1.68947871593051	-7.720353479214	\\
1.69646003293849	-8.19108895051568	\\
1.70344134994647	-8.66403950777453	\\
1.71042266695444	-9.14005780628736	\\
1.71740398396242	-9.62026959085819	\\
1.7243853009704	-10.1060719212473	\\
1.73136661797837	-10.5991074866821	\\
1.73834793498635	-11.1012034045069	\\
1.74532925199433	-11.6142566897058	\\
1.75231056900231	-11.6214410255709	\\
1.75929188601028	-12.6798721741599	\\
1.76627320301826	-13.2341187382908	\\
1.77325452002624	-13.8013640352941	\\
1.78023583703422	-14.3771732975349	\\
1.78721715404219	-14.9522614766337	\\
1.79419847105017	-15.5100230119457	\\
1.80117978805815	-16.0237320325948	\\
1.80816110506613	-16.4546130398856	\\
1.8151424220741	-25.2259834171908	\\
1.82212373908208	-5.12687196009659	\\
1.82910505609006	-4.66381667155684	\\
1.83608637309803	-4.25710966159193	\\
1.84306769010601	-3.90103956828915	\\
1.85004900711399	-3.59090102437943	\\
1.85703032412197	-3.32282346268363	\\
1.86401164112994	-3.09363045894878	\\
1.87099295813792	-2.90072692490997	\\
1.8779742751459	-2.74200985949801	\\
1.88495559215388	-2.61579842686768	\\
1.89193690916185	-2.52077973177444	\\
1.89891822616983	-2.45596736527494	\\
1.90589954317781	-2.42067043739388	\\
1.91288086018578	-2.41447135184319	\\
1.91986217719376	-2.43721100662566	\\
1.92684349420174	-2.48898043037241	\\
1.93382481120972	-2.57011809188168	\\
1.94080612821769	-2.68121224272852	\\
1.94778744522567	-2.82310764335585	\\
1.95476876223365	-3.52206482364875	\\
1.96175007924163	-3.74286775070647	\\
1.9687313962496	-4.00320663358224	\\
1.97571271325758	-4.30569036827427	\\
1.98269403026556	-4.65337401684718	\\
1.98967534727354	-5.04975204212741	\\
1.99665666428151	-5.49868985876839	\\
2.00363798128949	-6.00422552799955	\\
2.01061929829747	-6.57010818152941	\\
2.01760061530545	-7.19881429280736	\\
2.02458193231342	-7.88955032988656	\\
2.0315632493214	-9.72561698815036	\\
2.03854456632938	-9.94908090156247	\\
2.04552588333735	-10.2027115862076	\\
2.05250720034533	-10.4910329172024	\\
2.05948851735331	-10.8191848957972	\\
2.06646983436129	-11.1931660814959	\\
2.07345115136926	-11.6201612522913	\\
2.08043246837724	-12.1089986314278	\\
2.08741378538522	-12.670810094614	\\
2.0943951023932	-13.9100045283269	\\
2.10137641940117	-14.0499759353973	\\
2.10835773640915	-14.1084809937333	\\
2.11533905341713	-14.084144129079	\\
2.1223203704251	-13.9800376553073	\\
2.12930168743308	-13.8031379025413	\\
2.13628300444106	-13.563189910683	\\
2.14326432144904	-13.2713366874637	\\
2.15024563845701	-12.9388616658703	\\
2.15722695546499	-12.5762595927799	\\
2.16420827247297	-20.3168630139538	\\
2.17118958948095	-12.8915855647883	\\
2.17817090648892	-12.8107426699296	\\
2.1851522234969	-12.7339572295715	\\
2.19213354050488	-12.661028700291	\\
2.19911485751286	-12.5917405154294	\\
2.20609617452083	-12.5258607906136	\\
2.21307749152881	-12.4631427870061	\\
2.22005880853679	-12.4033251890633	\\
2.22704012554476	-12.3461322492036	\\
2.23402144255274	-12.2912738487053	\\
2.24100275956072	-12.238445522143	\\
2.2479840765687	-12.187328491524	\\
2.25496539357667	-12.1375897557352	\\
2.26194671058465	-12.0888822807121	\\
2.26892802759263	-12.0408453355941	\\
2.27590934460061	-11.9931050197147	\\
2.28289066160858	-11.9452750242601	\\
2.28987197861656	-11.8969576704285	\\
2.29685329562454	-11.8477452625869	\\
2.30383461263251	-16.3859575664363	\\
2.31081592964049	-11.6175010957554	\\
2.31779724664847	-10.5526397420301	\\
2.32477856365645	-9.67008588368435	\\
2.33175988066442	-8.9334746315476	\\
2.3387411976724	-8.31673481158612	\\
2.34572251468038	-7.80087397897984	\\
2.35270383168836	-7.37184992429297	\\
2.35968514869633	-7.01917078756395	\\
2.36666646570431	-6.73496526332553	\\
2.37364778271229	-6.5133575346315	\\
2.38062909972027	-6.35004404575068	\\
2.38761041672824	-6.24200828035913	\\
2.39459173373622	-6.18733400765399	\\
2.4015730507442	-6.18509296623796	\\
2.40855436775217	-6.23529341146684	\\
2.41553568476015	-6.33888371367505	\\
2.42251700176813	-6.49781189022826	\\
2.42949831877611	-6.71514894782689	\\
2.43647963578408	-6.9952927088612	\\
2.44346095279206	-10.6387873347506	\\
2.45044226980004	-4.94077250947936	\\
2.45742358680802	-4.89511971785043	\\
2.46440490381599	-4.91104749533207	\\
2.47138622082397	-4.98887619052516	\\
2.47836753783195	-5.13024892837899	\\
2.48534885483993	-5.33824750464152	\\
2.4923301718479	-5.61762014999363	\\
2.49931148885588	-5.97515717141101	\\
2.50629280586386	-6.42027914421845	\\
2.51327412287183	-6.96595059533882	\\
2.52025543987981	-7.63011877837556	\\
2.52723675688779	-8.43804064064443	\\
2.53421807389577	-9.42618211155832	\\
2.54119939090374	-10.6490166748736	\\
2.54818070791172	-12.1912646393329	\\
2.5551620249197	-14.1893515564427	\\
2.56214334192768	-16.8540814826331	\\
2.56912465893565	-17.9512744196524	\\
2.57610597594363	-17.1747131195298	\\
2.58308729295161	-16.3975147112536	\\
2.59006860995959	-15.6237932052433	\\
2.59704992696756	-14.8522423455634	\\
2.60403124397554	-14.0800030735608	\\
2.61101256098352	-13.3046390828492	\\
2.61799387799149	-12.5249689458416	\\
2.62497519499947	-11.74122640217	\\
2.63195651200745	-10.9548542213633	\\
2.63893782901543	-10.168137261409	\\
2.6459191460234	-9.38380969853733	\\
2.65290046303138	-11.0470753831006	\\
2.65988178003936	-7.960703088044	\\
2.66686309704734	-7.54792465204707	\\
2.67384441405531	-7.20004325950038	\\
2.68082573106329	-6.91034243212823	\\
2.68780704807127	-6.67366397908324	\\
2.69478836507924	-6.48606904793871	\\
2.70176968208722	-6.34460285895926	\\
2.7087509990952	-6.24713175111907	\\
2.71573231610318	-6.19223281160464	\\
2.72271363311115	-6.17912375560223	\\
2.72969495011913	-6.20762569310494	\\
2.73667626712711	-6.27815504340828	\\
2.74365758413509	-6.39174383385784	\\
2.75063890114306	-6.55009047247711	\\
2.75762021815104	-6.75564630967955	\\
2.76460153515902	-7.01174750060335	\\
2.771582852167	-7.3228077217247	\\
2.77856416917497	-7.69459662368018	\\
2.78554548618295	-8.13464406517431	\\
2.79252680319093	-9.09392322002384	\\
2.7995081201989	-3.3891910650284	\\
2.80648943720688	-3.85679372709785	\\
2.81347075421486	-4.33031693064862	\\
2.82045207122284	-4.80988856768806	\\
2.82743338823081	-5.29569897786513	\\
2.83441470523879	-5.78801582351548	\\
2.84139602224677	-6.28719927475686	\\
2.84837733925475	-6.79371620991532	\\
2.85535865626272	-7.30815138187899	\\
2.8623399732707	-7.83121238616985	\\
2.86932129027868	-8.36372359717097	\\
2.87630260728666	-8.90660170543094	\\
2.88328392429463	-9.4608016066407	\\
2.89026524130261	-10.0272154284897	\\
2.89724655831059	-10.6064983950016	\\
2.90422787531856	-11.1987817319755	\\
2.91120919232654	-11.8032138557446	\\
2.91819050933452	-12.4172475028975	\\
2.9251718263425	-13.035569594257	\\
2.93215314335047	-13.6485770614948	\\
2.93913446035845	-14.2403973804453	\\
2.94611577736643	-14.7867572178401	\\
2.95309709437441	-15.2536614279393	\\
2.96007841138238	-15.5987875007658	\\
2.96705972839036	-15.7778846999588	\\
2.97404104539834	-15.7565003320054	\\
2.98102236240632	-15.52247654606	\\
2.98800367941429	-15.0912124486475	\\
2.99498499642227	-14.4996700513884	\\
3.00196631343025	-13.7936623179094	\\
3.00894763043822	-13.016388977009	\\
3.0159289474462	-12.2024106629395	\\
3.02291026445418	-11.3764985600588	\\
3.02989158146216	-10.5550446215742	\\
3.03687289847013	-9.74819002997589	\\
3.04385421547811	-8.96176751357518	\\
3.05083553248609	-8.19878515654017	\\
3.05781684949407	-7.46046471818858	\\
3.06479816650204	-6.74693368197524	\\
3.07177948351002	-6.05767325451575	\\
};
%\addlegendentry{$\Delta_{\mathrm{LPM}}$ (int)}

\addplot [LRM, interpol, forget plot]
  table[row sep=crcr]{
0.0698131700797732	1.37672522055726	\\
0.0767944870877505	1.46749189353346	\\
0.0837758040957278	1.50734998894967	\\
0.0907571211037051	1.54801090957562	\\
0.0977384381116825	1.58952513329001	\\
0.10471975511966	1.6319487536573	\\
0.111701072127637	1.67534432123733	\\
0.118682389135614	1.72915793491467	\\
0.125663706143592	1.78934261810093	\\
0.132645023151569	1.85283148696942	\\
0.139626340159546	1.86018472199117	\\
0.146607657167524	1.90968627048085	\\
0.153588974175501	1.96074289723481	\\
0.160570291183478	2.01350499812548	\\
0.167551608191456	2.06814641573573	\\
0.174532925199433	2.12486928566256	\\
0.18151424220741	2.1839101390371	\\
0.188495559215388	2.24554765831613	\\
0.195476876223365	2.31011263358334	\\
0.202458193231342	2.37800088383699	\\
0.20943951023932	1.76752286431793	\\
0.216420827247297	1.76635380241555	\\
0.223402144255274	1.76370629896542	\\
0.230383461263251	1.75945351432335	\\
0.237364778271229	1.75345314271749	\\
0.244346095279206	1.74554492109149	\\
0.251327412287183	1.73554765737123	\\
0.258308729295161	1.72325566769666	\\
0.265290046303138	1.70843448272245	\\
0.272271363311115	1.69081564470446	\\
0.279252680319093	0.568870173251582	\\
0.28623399732707	0.559795586201915	\\
0.293215314335047	0.551440305014978	\\
0.300196631343025	0.543768714909902	\\
0.307177948351002	0.536753608823801	\\
0.314159265358979	0.530375518594667	\\
0.321140582366957	0.524622228694909	\\
0.328121899374934	0.519488449472817	\\
0.335103216382911	0.5149756332425	\\
0.342084533390889	0.51109192180121	\\
0.349065850398866	0.213568824382719	\\
0.356047167406843	0.265295112741398	\\
0.363028484414821	0.317131474741927	\\
0.370009801422798	0.369230949013684	\\
0.376991118430775	0.421736499870349	\\
0.383972435438753	0.478749880695148	\\
0.39095375244673	0.589761082509426	\\
0.397935069454707	0.697079160250212	\\
0.404916386462684	0.801329330270732	\\
0.411897703470662	0.903052433708922	\\
0.418879020478639	1.00272008265142	\\
0.425860337486616	1.1629325643429	\\
0.432841654494594	1.26854448903214	\\
0.439822971502571	1.3712763091533	\\
0.446804288510548	1.47174792808244	\\
0.453785605518526	1.57048240783757	\\
0.460766922526503	1.66792665098222	\\
0.46774823953448	1.76446718608599	\\
0.474729556542458	1.86044237367616	\\
0.481710873550435	1.95615196123299	\\
0.488692190558412	2.05186465188825	\\
0.49567350756639	2.33006863690201	\\
0.502654824574367	2.42794734239544	\\
0.509636141582344	2.52554291690956	\\
0.516617458590322	2.62315246013299	\\
0.523598775598299	2.72104542691386	\\
0.530580092606276	2.81946918090472	\\
0.537561409614253	2.91865354676304	\\
0.544542726622231	3.01881458783635	\\
0.551524043630208	3.12015778228312	\\
0.558505360638185	3.22288073096945	\\
0.565486677646163	3.32717550117337	\\
0.57246799465414	3.4332306882996	\\
0.579449311662117	3.54123326147084	\\
0.586430628670095	3.65137024661334	\\
0.593411945678072	3.76383029148013	\\
0.600393262686049	3.8788051502396	\\
0.607374579694027	3.99649112027214	\\
0.614355896702004	4.11709046029404	\\
0.621337213709981	4.2408128165975	\\
0.628318530717959	4.07032669513166	\\
0.635299847725936	4.21866295702176	\\
0.642281164733913	4.37168693836446	\\
0.649262481741891	4.52962170335104	\\
0.656243798749868	4.69270523023295	\\
0.663225115757845	4.86119186124398	\\
0.670206432765823	5.03535392712888	\\
0.6771877497738	5.21548357351418	\\
0.684169066781777	5.40189482117808	\\
0.691150383789754	5.59492589812646	\\
0.698131700797732	5.80088957407497	\\
0.705113017805709	6.02818053077482	\\
0.712094334813686	6.26280853031625	\\
0.719075651821664	6.5051709096723	\\
0.726056968829641	6.75570546747241	\\
0.733038285837618	7.01489473235387	\\
0.740019602845596	7.28327101081629	\\
0.747000919853573	7.56142234327751	\\
0.75398223686155	7.84999952966609	\\
0.760963553869528	8.14972442625685	\\
0.767944870877505	8.32244450636728	\\
0.774926187885482	8.65529765071187	\\
0.78190750489346	8.993383121058	\\
0.788888821901437	9.34770229268412	\\
0.795870138909414	9.71960600936745	\\
0.802851455917391	10.1106213998402	\\
0.809832772925369	10.5224834520809	\\
0.816814089933346	10.95717364398	\\
0.823795406941323	11.4169674626694	\\
0.830776723949301	11.9044931450947	\\
0.837758040957278	12.4228045774123	\\
0.844739357965256	13.0426007128689	\\
0.851720674973233	13.5873276788227	\\
0.85870199198121	14.201439539059	\\
0.865683308989187	14.8856111874701	\\
0.872664625997165	15.626247187345	\\
0.879645943005142	16.431727242025	\\
0.886627260013119	17.3119383165273	\\
0.893608577021097	18.2782426430863	\\
0.900589894029074	19.3428410114381	\\
0.907571211037051	19.7234564198129	\\
0.914552528045029	21.3173912037417	\\
0.921533845053006	22.5503635709871	\\
0.928515162060983	23.7780328338804	\\
0.93549647906896	24.8393614756772	\\
0.942477796076938	25.4710773379938	\\
0.949459113084915	25.4352956458947	\\
0.956440430092892	24.7905872289095	\\
0.96342174710087	23.8935640286182	\\
0.970403064108847	22.7391748799252	\\
0.977384381116824	21.5143026723329	\\
0.984365698124802	20.0509195474134	\\
0.991347015132779	18.9123562282718	\\
0.998328332140756	17.8727057575634	\\
1.00530964914873	16.9309912033284	\\
1.01229096615671	16.0800951303971	\\
1.01927228316469	15.2931268318945	\\
1.02625360017267	14.561890284766	\\
1.03323491718064	13.8791702230515	\\
1.04021623418862	13.2387009572587	\\
1.0471975511966	12.5473177097757	\\
1.05417886820457	12.009431018314	\\
1.06116018521255	11.4687254435631	\\
1.06814150222053	10.9559622693328	\\
1.07512281922851	10.4682143993024	\\
1.08210413623648	10.0029509899148	\\
1.08908545324446	9.55796935165995	\\
1.09606677025244	9.13133993807662	\\
1.10304808726042	8.72136168349203	\\
1.11002940426839	8.3265255280221	\\
1.11701072127637	7.945484436583	\\
1.12399203828435	7.57702858805419	\\
1.13097335529233	7.22006469832297	\\
1.1379546723003	6.87359866362954	\\
1.14493598930828	6.53672088293195	\\
1.15191730631626	6.20859375163946	\\
1.15889862332423	5.88844092322699	\\
1.16587994033221	5.57553801705711	\\
1.17286125734019	5.2692045157346	\\
1.17984257434817	4.96879664779988	\\
1.18682389135614	4.37469485480625	\\
1.19380520836412	4.0900210590915	\\
1.2007865253721	3.81142891407256	\\
1.20776784238008	3.53849162150408	\\
1.21474915938805	3.27080433645011	\\
1.22173047639603	3.00798057031841	\\
1.22871179340401	2.74964883642497	\\
1.23569311041199	2.49544947816702	\\
1.24267442741996	2.24503162798243	\\
1.24965574442794	1.99805025328152	\\
1.25663706143592	0.541051420686472	\\
1.26361837844389	0.87118689082763	\\
1.27059969545187	0.599557661882386	\\
1.27758101245985	0.332388995462509	\\
1.28456232946783	0.0697371722409912	\\
1.2915436464758	-0.188305165987401	\\
1.29852496348378	-0.441606304592483	\\
1.30550628049176	-0.689992379862701	\\
1.31248759749974	-0.933243956371371	\\
1.31946891450771	-1.17109229422022	\\
1.32645023151569	-1.40321536799081	\\
1.33343154852367	-1.62923373417851	\\
1.34041286553164	-1.8487063904131	\\
1.34739418253962	-2.0611268302294	\\
1.3543754995476	-2.26591957385506	\\
1.36135681655558	-2.42100060367028	\\
1.36833813356355	-2.49796658741025	\\
1.37531945057153	-2.57079706003327	\\
1.38230076757951	-2.63977846786491	\\
1.38928208458749	-2.70519579889515	\\
1.39626340159546	-2.76733486461552	\\
1.40324471860344	-2.75866459892507	\\
1.41022603561142	-2.80863559202362	\\
1.4172073526194	-2.8567804160482	\\
1.42418866962737	-2.903185911994	\\
1.43116998663535	-2.94793247725715	\\
1.43815130364333	-2.99109448312333	\\
1.4451326206513	-3.03274063694835	\\
1.45211393765928	-3.07293429173836	\\
1.45909525466726	-3.11173370482379	\\
1.46607657167524	-3.1491922462568	\\
1.47305788868321	-3.18535855641284	\\
1.48003920569119	-3.22027665097733	\\
1.48702052269917	-3.25398596999469	\\
1.49400183970715	-3.28652136585151	\\
1.50098315671512	-3.31791302285046	\\
1.5079644737231	-3.34818629825372	\\
1.51494579073108	-3.37736147112304	\\
1.52192710773906	-3.40545338067881	\\
1.52890842474703	-3.43247092984092	\\
1.53588974175501	-4.40227527954903	\\
1.54287105876299	-4.47329176060981	\\
1.54985237577096	-4.5463636284685	\\
1.55683369277894	-4.62184358904875	\\
1.56381500978692	-4.70010586448711	\\
1.5707963267949	-4.78155221541942	\\
1.57777764380287	-4.8666186788933	\\
1.58475896081085	-4.95578325898407	\\
1.59174027781883	-5.04957485738447	\\
1.59872159482681	-5.14858380018433	\\
1.60570291183478	-5.96619447832785	\\
1.61268422884276	-6.06838142044154	\\
1.61966554585074	-6.17516809656604	\\
1.62664686285871	-6.2872662021594	\\
1.63362817986669	-6.40548183923528	\\
1.64060949687467	-6.53073622731745	\\
1.64759081388265	-6.66409142868082	\\
1.65457213089062	-6.80678264514472	\\
1.6615534478986	-6.96025920682092	\\
1.66853476490658	-7.12623717768076	\\
1.67551608191456	-8.89515200623487	\\
1.68249739892253	-8.99817256339156	\\
1.68947871593051	-9.10092560382822	\\
1.69646003293849	-9.20375742692312	\\
1.70344134994647	-9.30703919233645	\\
1.71042266695444	-9.41117268980383	\\
1.71740398396242	-9.51659722405782	\\
1.7243853009704	-9.62379788642829	\\
1.73136661797837	-9.73331555550749	\\
1.73834793498635	-9.84575905745709	\\
1.74532925199433	-12.3617538484781	\\
1.75231056900231	-10.5447930487217	\\
1.75929188601028	-10.3043742220494	\\
1.76627320301826	-10.0852116350417	\\
1.77325452002624	-9.88426945297545	\\
1.78023583703422	-9.69903238940492	\\
1.78721715404219	-9.52739660893621	\\
1.79419847105017	-9.36758644868638	\\
1.80117978805815	-9.21809012218353	\\
1.80816110506613	-9.07760953068691	\\
1.8151424220741	-8.94502066592128	\\
1.82212373908208	-6.31073548716307	\\
1.82910505609006	-6.19729856531109	\\
1.83608637309803	-6.0987843859379	\\
1.84306769010601	-6.01347777363964	\\
1.85004900711399	-5.93995747811675	\\
1.85703032412197	-5.87704715748845	\\
1.86401164112994	-5.82377595009956	\\
1.87099295813792	-5.77934692904819	\\
1.8779742751459	-5.74311201205751	\\
1.88495559215388	-5.71455217163913	\\
1.89193690916185	-5.69326203336902	\\
1.89891822616983	-5.67893815839057	\\
1.90589954317781	-5.67137048278994	\\
1.91288086018578	-5.67043653717599	\\
1.91986217719376	-5.67609820157253	\\
1.92684349420174	-5.68840087073949	\\
1.93382481120972	-5.70747502057566	\\
1.94080612821769	-5.73354028502263	\\
1.94778744522567	-5.76691228358823	\\
1.95476876223365	-8.76229200748787	\\
1.96175007924163	-8.83841362064044	\\
1.9687313962496	-8.91266934923021	\\
1.97571271325758	-8.98551143845907	\\
1.98269403026556	-9.05737128339217	\\
1.98967534727354	-9.12866801670452	\\
1.99665666428151	-9.19981669994388	\\
2.00363798128949	-9.27123645661914	\\
2.01061929829747	-9.34335888485754	\\
2.01760061530545	-9.41663711105241	\\
2.02458193231342	-9.491555896629	\\
2.0315632493214	-10.498522714037	\\
2.03854456632938	-10.5806097018278	\\
2.04552588333735	-10.6569309779831	\\
2.05250720034533	-10.7284105391058	\\
2.05948851735331	-10.7958503189678	\\
2.06646983436129	-10.8599520670605	\\
2.07345115136926	-10.9213359094222	\\
2.08043246837724	-10.9805561005093	\\
2.08741378538522	-11.0381144557687	\\
2.0943951023932	-11.0944719071164	\\
2.10137641940117	-11.1500585672855	\\
2.10835773640915	-11.2052826329572	\\
2.11533905341713	-11.2605384037105	\\
2.1223203704251	-11.3162136433644	\\
2.12930168743308	-11.3726964585144	\\
2.13628300444106	-11.4303818089452	\\
2.14326432144904	-11.4896776841644	\\
2.15024563845701	-11.5510108593252	\\
2.15722695546499	-11.6148319474729	\\
2.16420827247297	-14.7100249806446	\\
2.17118958948095	-14.6415164055732	\\
2.17817090648892	-14.5781629052597	\\
2.1851522234969	-14.5118683669363	\\
2.19213354050488	-14.4425770757376	\\
2.19911485751286	-14.3702107740051	\\
2.20609617452083	-14.2946700234557	\\
2.21307749152881	-14.2158350353803	\\
2.22005880853679	-14.1335660521485	\\
2.22704012554476	-14.0477033407427	\\
2.23402144255274	-13.9580668417838	\\
2.24100275956072	-11.6836024987408	\\
2.2479840765687	-11.6015512159627	\\
2.25496539357667	-11.5228005462722	\\
2.26194671058465	-11.4469193837182	\\
2.26892802759263	-11.3735242280079	\\
2.27590934460061	-11.3022718614442	\\
2.28289066160858	-11.2328533655694	\\
2.28987197861656	-11.1649892385556	\\
2.29685329562454	-11.0984254345915	\\
2.30383461263251	-11.0329301947376	\\
2.31081592964049	-10.9682915788021	\\
2.31779724664847	-10.9043156427362	\\
2.32477856365645	-10.8408252383626	\\
2.33175988066442	-10.7776594441973	\\
2.3387411976724	-10.7146736698629	\\
2.34572251468038	-10.6517405144551	\\
2.35270383168836	-10.588751503775	\\
2.35968514869633	-10.5256198856673	\\
2.36666646570431	-10.4622847304737	\\
2.37364778271229	-12.5638142795047	\\
2.38062909972027	-12.6648523887818	\\
2.38761041672824	-12.7678250063012	\\
2.39459173373622	-12.8733069372021	\\
2.4015730507442	-12.981887275943	\\
2.40855436775217	-13.0941800564227	\\
2.41553568476015	-13.2108354718496	\\
2.42251700176813	-13.3325520850902	\\
2.42949831877611	-13.4600905064894	\\
2.43647963578408	-13.5942891045928	\\
2.44346095279206	-15.6847678836759	\\
2.45044226980004	-15.767796903104	\\
2.45742358680802	-15.8557346809425	\\
2.46440490381599	-15.94898421107	\\
2.47138622082397	-16.0479857624472	\\
2.47836753783195	-16.1532186118107	\\
2.48534885483993	-16.2652016682637	\\
2.4923301718479	-16.3844921629223	\\
2.49931148885588	-16.5116811063414	\\
2.50629280586386	-16.6473834833472	\\
2.51327412287183	-27.613408237032	\\
2.52025543987981	-24.2730474449373	\\
2.52723675688779	-24.4356158264731	\\
2.53421807389577	-24.5739676307904	\\
2.54119939090374	-24.6896904126477	\\
2.54818070791172	-24.7845066484928	\\
2.5551620249197	-24.8602242507337	\\
2.56214334192768	-24.9186894771356	\\
2.56912465893565	-24.9617448401098	\\
2.57610597594363	-24.9911937000544	\\
2.58308729295161	-25.008772343106	\\
2.59006860995959	-22.6501972804543	\\
2.59704992696756	-23.3200889203858	\\
2.60403124397554	-23.9408080659732	\\
2.61101256098352	-24.5116098647206	\\
2.61799387799149	-24.9726781767465	\\
2.62497519499947	-24.949983092034	\\
2.63195651200745	-24.9247497969354	\\
2.63893782901543	-24.8979139188183	\\
2.6459191460234	-24.8703315526578	\\
2.65290046303138	-26.8456349750616	\\
2.65988178003936	-17.4293810460967	\\
2.66686309704734	-17.5286850845656	\\
2.67384441405531	-17.6262340955927	\\
2.68082573106329	-17.7224648961165	\\
2.68780704807127	-17.8177868475534	\\
2.69478836507924	-17.9125880146449	\\
2.70176968208722	-18.0072405763987	\\
2.7087509990952	-18.1021056537854	\\
2.71573231610318	-18.1975376859367	\\
2.72271363311115	-18.2938884599623	\\
2.72969495011913	-18.3915108769017	\\
2.73667626712711	-18.4907625157372	\\
2.74365758413509	-18.5920090367539	\\
2.75063890114306	-18.6956274425082	\\
2.75762021815104	-18.8020091862896	\\
2.76460153515902	-18.9115630802214	\\
2.771582852167	-19.0247179023003	\\
2.77856416917497	-19.1419245253159	\\
2.78554548618295	-19.2636572781552	\\
2.79252680319093	-19.8902771492208	\\
2.7995081201989	-16.9015485772656	\\
2.80648943720688	-17.0518029225636	\\
2.81347075421486	-17.2020959287921	\\
2.82045207122284	-17.352763862219	\\
2.82743338823081	-17.5041082880053	\\
2.83441470523879	-17.6563950691698	\\
2.84139602224677	-17.8098521203536	\\
2.84837733925475	-17.9646657514575	\\
2.85535865626272	-18.1209753870046	\\
2.8623399732707	-18.2788663924916	\\
2.86932129027868	-15.5784913099865	\\
2.87630260728666	-15.9659132694952	\\
2.88328392429463	-16.3629066115529	\\
2.89026524130261	-16.7700893760257	\\
2.89724655831059	-17.1880447146434	\\
2.90422787531856	-17.6172979173596	\\
2.91120919232654	-18.0582832540659	\\
2.91819050933452	-18.5112968790449	\\
2.9251718263425	-18.976430770835	\\
2.93215314335047	-19.4534811149452	\\
2.93913446035845	-19.7788356225223	\\
2.94611577736643	-20.4402398329119	\\
2.95309709437441	-20.9467014622671	\\
2.96007841138238	-21.4580739746864	\\
2.96705972839036	-21.9697670973852	\\
2.97404104539834	-22.475329658035	\\
2.98102236240632	-22.9660436334246	\\
2.98800367941429	-23.4306209684756	\\
2.99498499642227	-23.8551810811248	\\
3.00196631343025	-24.2237513022052	\\
3.00894763043822	-26.5643498881227	\\
3.0159289474462	-26.7751458941	\\
3.02291026445418	-26.8257893630003	\\
3.02989158146216	-26.727889478249	\\
3.03687289847013	-26.505618722747	\\
3.04385421547811	-26.189075367709	\\
3.05083553248609	-25.8079938452494	\\
3.05781684949407	-25.387817329947	\\
3.06479816650204	-24.9482663554269	\\
3.07177948351002	-24.7236524590019	\\
};
%\addlegendentry{$\Delta_{\mathrm{LRM}}$ (int)}

\addplot [LPM, hinfnorm]
  table[row sep=crcr]{
0.0698131700797732	18.7067656008809	\\
3.07177948351002	18.7067656008809	\\
};
\addlegendentry{$\left\|\Delta_{\mathrm{LPM}}\right\|_{\infty}$}

\addplot [LRM, hinfnorm]
  table[row sep=crcr]{
0.0698131700797732	25.4710773379938	\\
3.07177948351002	25.4710773379938	\\
};
\addlegendentry{$\left\|\Delta_{\mathrm{LRM}}\right\|_{\infty}$}

\addplot [SA, hinfnorm]
  table[row sep=crcr]{
0.0698131700797732	19.7359467917969	\\
3.07177948351002	19.7359467917969	\\
};
\addlegendentry{$\left\|\Delta_{\mathrm{SA}}\right\|_{\infty}$}

\addplot [color=black,hinfnorm]
  table[row sep=crcr]{
0.0698131700797732  25.694622637891 \\
3.07177948351002    25.694622637891 \\
};
\addlegendentry{$\left\|\Delta\right\|_{\infty}$}

\end{axis}
\end{tikzpicture}%

  \caption{Simulation example revealing that $\gamma_{\mathrm{FRF}}$ (left) for both the \glsfirst{SA}~\legref{leg:example-SA} and \gls{LRM}~\legref{leg:example-LRM} underestimate the true $\gamma = \norm{\Delta} = 24$~\legref{leg:example-true-Hinf} due to limited frequency resolution.
  Using the proposed \gls{LRM}-based approach, both the at-grid~($\gamma_{\mathrm{FRF}}$) and intergrid~($\gamma_{\mathrm{IG}}$) estimate are improved.
  The interpolated \gls{LRM}~\legref{leg:example-LRM-interpol} and true $\Delta$~\legref{leg:example-true} almost coincide.
  Hence, $\widehat{\norm[]{\Delta_{\mathrm{LRM}}}}_{\infty}$~\legref{leg:example-LRM-interpol-Hinf} approaches the theoretical $\gamma=24$.}
  \label{fig:exampleIntergrid}
\end{figure}

It can be observed for $\gamma_{\mathrm{IG}}$ in \figref{fig:exampleIntergrid} that the \gls{LRM} is able to model the resonance well and yields a useful estimate $\infnorm{\Delta_{\mathrm{LRM}}} = 24$ in the simulation.
Intuitively, the \gls{LRM} locally approximates the resonant pole which allows for a reliable inter-grid estimate.
The \gls{LPM}, on the other hand, does not provide a reliable estimate.

\subsection{Study of the experiment length}
Frequency resolution, and hence the experiment length, is key to obtain a reliable \Hinf{} norm estimate.
In this section, we determine the minimal experiment length needed for the proposed approach by means of simulations.

We simulate the system \eqref{eq:O2sysDT} for $16\,100$ samples and discard the first $100$ samples to randomize the initial conditions.
To mimic the effect of progressively measuring longer data records, the first remaining $N$ samples are used to estimate $\infnorm{\Delta}$ as in \secref{sec:LPMHinf}.
The values of $N \in \ClosedInterval{35}{16 000}$ are chosen such that $\expectedValue{\gamma_{\mathrm{FRF}}(N)}$ is a non-decreasing function of $N$.
These simulations are then repeated in a Monte Carlo simulation of $n_{\mathrm{MC}}=200$ runs where each run has a different realization of  $u_{\Delta}(n)$ and $v(n)$.
The resulting sample average and standard deviation $\sigma$ are shown in \figref{fig:MC-run-length}.

\begin{figure}
  \centering
  \setlength{\figurewidth}{0.75\columnwidth}
  \setlength{\figureheight}{0.68\figurewidth}
  % This file was created by matlab2tikz v0.4.7 (commit f204e544bf36a123713b8051de724c0d3f0daa30) running on MATLAB 8.3.
% Copyright (c) 2008--2014, Nico Schlömer <nico.schloemer@gmail.com>
% All rights reserved.
% Minimal pgfplots version: 1.3
% 
% The latest updates can be retrieved from
%   http://www.mathworks.com/matlabcentral/fileexchange/22022-matlab2tikz
% where you can also make suggestions and rate matlab2tikz.
% 
% workdir  : /Users/egon/Dropbox/VUB/PhD/LPMHinf/Code/MCLARX
% stack    : /Users/egon/Dropbox/VUB/PhD/LPMHinf/Code/MCLARX/test_hinf_length_mc_paper.m (50)
% git info : commit a8ab01207283cbc86f952083a2d49e011faa50b4
%            Author: Egon Geerardyn <egon.geerardyn@gmail.com>
%            Date:   Mon Aug 18 11:33:44 2014 +0200
%            
%                Hinf vs runlength: MC for paper
%            
%             test_hinf_length_mc_paper.m | 172 ++++++++++++++++++++++++++------------------
%             1 file changed, 102 insertions(+), 70 deletions(-)
%            
% 
% 
%
% defining custom colors
% \definecolor{mycolor1}{rgb}{0.92941,0.83137,0.00000}%
% \definecolor{mycolor2}{rgb}{0.53333,0.54118,0.52157}%
% \definecolor{mycolor3}{rgb}{0.44706,0.62353,0.81176}%
% \definecolor{mycolor4}{rgb}{0.20392,0.39608,0.64314}%
% \definecolor{mycolor5}{rgb}{0.93725,0.16078,0.16078}%
%
\begin{tikzpicture}

\begin{axis}[%
width=\figurewidth,
height=\figureheight,
scale only axis,
xmode=log,
xmin=35,
xmax=16000,
x tick label style={align=center},
xtick={100,1000,10000},
xticklabels={{$100$\\$2\timeconst$},{$1\,000$\\$20\timeconst$},{$10\,000$\\$200\timeconst$}},
xmajorgrids,
xminorgrids,
xminorticks=true,
extra y ticks={24},
ytick={0,10,20,30},
xlabel={Data length $N$ \axisunit{samples}},
ymin=0,
ymax=30,
ymajorgrids,
yminorgrids,
ylabel={Amplitude $\svdMax{\Delta(\omega_{\star})}$}
%legend style={draw=black,fill=white,legend cell align=left}
]

\addplot [gammafrf,mediummarkers] table[]{\thisDir/data/mc-runlength/E-gamma-frf.tsv};
\label{leg:runlength-frf}
%\addlegendentry{True Value (DFT)};

\addplot [reference]
  table[row sep=crcr]{%
97	0\\
97	50\\
};
\label{leg:runlength-special-case}
%\addlegendentry{N = 97};

%
% \addplot[area legend,solid,fill=mycolor3,opacity=5.000000e-01,draw=mycolor3]
% table[row sep=crcr] {%
% x y\\
% 35    1.81560713340585\\
% 41    1.56999410222808\\
% 47    1.81697531891568\\
% 53    2.53084101382831\\
% 60    3.0503503812844\\
% 66    3.15404480553949\\
% 72    3.61302193648694\\
% 79    3.93791305243213\\
% 91    4.36665733649132\\
% 97    4.51074644526152\\
% 116   5.28016880076408\\
% 123   5.56024576016332\\
% 135   5.98807230661911\\
% 141   6.20960076645457\\
% 160   6.96441005675687\\
% 167   7.33623831088588\\
% 185   7.83513351531706\\
% 204   8.42201653223282\\
% 211   8.66382432181177\\
% 229   9.21791021262874\\
% 248   9.87134948479537\\
% 255   10.1288996693274\\
% 273   10.5115253203875\\
% 292   10.9255376593992\\
% 299   11.0943398226093\\
% 317   11.53575522997\\
% 336   11.9940566534107\\
% 343   12.1679102478731\\
% 361   12.4814327085875\\
% 380   12.9865246385691\\
% 387   13.147679437555\\
% 405   13.4755627977348\\
% 431   14.0683630532143\\
% 449   14.4517566969513\\
% 475   14.8709495733971\\
% 493   15.2111720405974\\
% 519   15.7506792618461\\
% 537   16.0714271226282\\
% 563   16.5151675148748\\
% 581   16.6933114133843\\
% 607   17.060951891627\\
% 625   17.3101243539131\\
% 651   17.5865090609801\\
% 669   17.8721414644464\\
% 695   18.1607923957859\\
% 713   18.3640285870386\\
% 739   18.6669701565261\\
% 757   18.8401990393886\\
% 783   19.0895538487323\\
% 801   19.2320913733712\\
% 827   19.4851484273624\\
% 845   19.6534922353075\\
% 871   19.8858296861036\\
% 889   20.0245986070564\\
% 915   20.1954107924043\\
% 933   20.3355578454656\\
% 977   20.6500407118445\\
% 1021  20.9309830719126\\
% 1065  21.1803846264483\\
% 1109  21.3881277385729\\
% 1153  21.5592064271881\\
% 1197  21.7207735140979\\
% 1241  21.8967315241155\\
% 1285  22.0506766510044\\
% 1329  22.2025197068129\\
% 1373  22.3234146746556\\
% 1417  22.4622025374827\\
% 1461  22.5785166144417\\
% 1505  22.6382321271423\\
% 1549  22.7093211940836\\
% 1593  22.7696842495787\\
% 1637  22.8445862518237\\
% 1681  22.9055785295265\\
% 1725  22.9672231387976\\
% 1769  23.0375893414858\\
% 1813  23.1073478847728\\
% 1857  23.1647127824911\\
% 1901  23.2174893740375\\
% 1945  23.2616548539506\\
% 1990  23.2851987104153\\
% 2034  23.3248272138905\\
% 2078  23.3631038319337\\
% 2122  23.3895773977022\\
% 2166  23.4134123280743\\
% 2210  23.4422261819587\\
% 2254  23.4632121252416\\
% 2298  23.4874771866855\\
% 2342  23.5150457101234\\
% 2386  23.5411834593371\\
% 2430  23.5560173485216\\
% 2474  23.5674224093588\\
% 2518  23.5747943596037\\
% 2562  23.5875057472283\\
% 2606  23.6005773451139\\
% 2650  23.6162153638289\\
% 2694  23.6302824356788\\
% 2738  23.6455642096851\\
% 2782  23.6572460680993\\
% 2826  23.6648944021098\\
% 2870  23.6710977973009\\
% 2914  23.6796287704932\\
% 2958  23.691776467382\\
% 3000  23.7064916662788\\
% 3908  23.8025013148402\\
% 5559  23.862553279883\\
% 6630  23.8873027626911\\
% 16000 23.995142285077\\
% 16000 24.2401116799406\\
% 6630  24.1763974180633\\
% 5559  24.1598117716954\\
% 3908  24.0959776994886\\
% 3000  24.0410542905627\\
% 2958  24.0375147578285\\
% 2914  24.0402079250619\\
% 2870  24.036935405524\\
% 2826  24.0284831991962\\
% 2782  24.0204367220456\\
% 2738  24.0128024582589\\
% 2694  24.0082103526\\
% 2650  24.0003302748751\\
% 2606  23.9948378415657\\
% 2562  23.989910283706\\
% 2518  23.9875613743855\\
% 2474  23.9819151771506\\
% 2430  23.9723434264391\\
% 2386  23.9679286235254\\
% 2342  23.9602134089791\\
% 2298  23.9558275793117\\
% 2254  23.9446375996646\\
% 2210  23.93683751804\\
% 2166  23.9340604701998\\
% 2122  23.9223898573715\\
% 2078  23.9007845578171\\
% 2034  23.9012966990911\\
% 1990  23.8988473660501\\
% 1945  23.8646790922119\\
% 1901  23.8401741492027\\
% 1857  23.820309955149\\
% 1813  23.7992453679796\\
% 1769  23.7836543462212\\
% 1725  23.7672087378416\\
% 1681  23.750481724566\\
% 1637  23.709922620313\\
% 1593  23.700220262426\\
% 1549  23.6444207170249\\
% 1505  23.5983186163522\\
% 1461  23.5643398055694\\
% 1417  23.5434604813283\\
% 1373  23.4947882247654\\
% 1329  23.4153385608772\\
% 1285  23.3695807205837\\
% 1241  23.2872821789\\
% 1197  23.1776674566771\\
% 1153  23.0853864456019\\
% 1109  23.0152088566975\\
% 1065  22.9702360578761\\
% 1021  22.8960437201898\\
% 977   22.7672031282098\\
% 933   22.5902762402738\\
% 915   22.5263599272658\\
% 889   22.4106475158437\\
% 871   22.3525985070646\\
% 845   22.2343034361605\\
% 827   22.1577473164143\\
% 801   22.0566459838074\\
% 783   21.9851686773425\\
% 757   21.8048622655595\\
% 739   21.7491610575607\\
% 713   21.6332738391341\\
% 695   21.5115119614712\\
% 669   21.3796733821955\\
% 651   21.3149966783791\\
% 625   21.0630943535243\\
% 607   20.908508121555\\
% 581   20.6699326334698\\
% 563   20.5010404522234\\
% 537   20.2426058150109\\
% 519   20.1430651645238\\
% 493   19.8473494119751\\
% 475   19.6010491185152\\
% 449   19.2262012109257\\
% 431   19.0434329407977\\
% 405   18.8831794950682\\
% 387   18.5168238207693\\
% 380   18.4039932592543\\
% 361   18.1271787348029\\
% 343   17.6872508944613\\
% 336   17.5435302232125\\
% 317   17.3792453376541\\
% 299   17.0257169658561\\
% 292   16.8110516660221\\
% 273   16.2922036298524\\
% 255   15.8822408257973\\
% 248   15.677120823265\\
% 229   15.1917364701432\\
% 211   14.6749308201787\\
% 204   14.5094039031603\\
% 185   13.9280449781014\\
% 167   13.2009217704541\\
% 160   13.0053058108001\\
% 141   12.1722927287192\\
% 135   11.9988751438759\\
% 123   11.450419206995\\
% 116   11.090889186263\\
% 97    10.0999095285891\\
% 91    9.83373235911541\\
% 79    9.09922613037694\\
% 72    8.76508058013312\\
% 66    9.67935241173424\\
% 60    8.89127799833867\\
% 53    8.60458273471775\\
% 47    9.06454222431114\\
% 41    8.97714143813841\\
% 35    8.11695641445667\\
% };

% \addlegendentry{data4};

% \addplot [color=mycolor4,dotted]
%   table[row sep=crcr]{%
% 35    4.96628177393126\\
% 41    5.27356777018325\\
% 47    5.44075877161341\\
% 53    5.56771187427303\\
% 60    5.97081418981154\\
% 66    6.41669860863687\\
% 72    6.18905125831003\\
% 79    6.51856959140453\\
% 91    7.10019484780336\\
% 97    7.30532798692533\\
% 116   8.18552899351356\\
% 123   8.50533248357918\\
% 135   8.99347372524751\\
% 141   9.19094674758688\\
% 160   9.98485793377849\\
% 167   10.26858004067\\
% 185   10.8815892467092\\
% 204   11.4657102176966\\
% 211   11.6693775709952\\
% 229   12.204823341386\\
% 248   12.7742351540302\\
% 255   13.0055702475623\\
% 273   13.4018644751199\\
% 292   13.8682946627107\\
% 299   14.0600283942327\\
% 317   14.457500283812\\
% 336   14.7687934383116\\
% 343   14.9275805711672\\
% 361   15.3043057216952\\
% 380   15.6952589489117\\
% 387   15.8322516291622\\
% 405   16.1793711464015\\
% 431   16.555897997006\\
% 449   16.8389789539385\\
% 475   17.2359993459562\\
% 493   17.5292607262862\\
% 519   17.9468722131849\\
% 537   18.1570164688195\\
% 563   18.5081039835491\\
% 581   18.681622023427\\
% 607   18.984730006591\\
% 625   19.1866093537187\\
% 651   19.4507528696796\\
% 669   19.625907423321\\
% 695   19.8361521786286\\
% 713   19.9986512130863\\
% 739   20.2080656070434\\
% 757   20.3225306524741\\
% 783   20.5373612630374\\
% 801   20.6443686785893\\
% 827   20.8214478718884\\
% 845   20.943897835734\\
% 871   21.1192140965841\\
% 889   21.2176230614501\\
% 915   21.3608853598351\\
% 933   21.4629170428697\\
% 977   21.7086219200271\\
% 1021  21.9135133960512\\
% 1065  22.0753103421622\\
% 1109  22.2016682976352\\
% 1153  22.322296436395\\
% 1197  22.4492204853875\\
% 1241  22.5920068515077\\
% 1285  22.7101286857941\\
% 1329  22.808929133845\\
% 1373  22.9091014497105\\
% 1417  23.0028315094055\\
% 1461  23.0714282100056\\
% 1505  23.1182753717472\\
% 1549  23.1768709555542\\
% 1593  23.2349522560024\\
% 1637  23.2772544360684\\
% 1681  23.3280301270463\\
% 1725  23.3672159383196\\
% 1769  23.4106218438535\\
% 1813  23.4532966263762\\
% 1857  23.4925113688201\\
% 1901  23.5288317616201\\
% 1945  23.5631669730812\\
% 1990  23.5920230382327\\
% 2034  23.6130619564908\\
% 2078  23.6319441948754\\
% 2122  23.6559836275369\\
% 2166  23.673736399137\\
% 2210  23.6895318499994\\
% 2254  23.7039248624531\\
% 2298  23.7216523829986\\
% 2342  23.7376295595512\\
% 2386  23.7545560414313\\
% 2430  23.7641803874804\\
% 2474  23.7746687932547\\
% 2518  23.7811778669946\\
% 2562  23.7887080154672\\
% 2606  23.7977075933398\\
% 2650  23.808272819352\\
% 2694  23.8192463941394\\
% 2738  23.829183333972\\
% 2782  23.8388413950725\\
% 2826  23.846688800653\\
% 2870  23.8540166014125\\
% 2914  23.8599183477776\\
% 2958  23.8646456126052\\
% 3000  23.8737729784208\\
% 3908  23.9492395071644\\
% 5559  24.0111825257892\\
% 6630  24.0318500903772\\
% 16000 24.1176269825088\\
% };
% \addlegendentry{Factory:LPM(5,2,2) (DFT)};


\addplot[LPMshade] table[] {\thisDir/data/mc-runlength/LPM-IG-uncertainty.tsv};
\label{leg:runlength-LPM-uncertainty}
%\addlegendentry{data6};

\addplot [LPMgrid,interpol,medsmallmarkers] table[]{\thisDir/data/mc-runlength/LPM-IG.tsv};
%\addlegendentry{Factory:LPM(5,2,2)};
\label{leg:runlength-LPM}


% 
% \addplot[area legend,solid,fill=mycolor5,opacity=5.000000e-01,draw=mycolor5]
% table[row sep=crcr] {%
% x y\\
% 35    3.99856997854726\\
% 41    4.43419280619551\\
% 47    5.33539108370522\\
% 53    6.24663693529195\\
% 60    7.00251386207033\\
% 66    7.73761038298303\\
% 72    8.63859771199307\\
% 79    9.47165840342075\\
% 91    10.7120632181946\\
% 97    11.7218074977735\\
% 116   13.3657953761622\\
% 123   14.0810920278741\\
% 135   14.6896758602484\\
% 141   15.5807089186708\\
% 160   16.5287500937621\\
% 167   17.0676907263333\\
% 185   18.0488404920805\\
% 204   18.5865963659318\\
% 211   19.0618591986806\\
% 229   19.6723314471738\\
% 248   19.9789354756151\\
% 255   20.2777060104645\\
% 273   20.7110117233687\\
% 292   20.8703384138973\\
% 299   21.1441954061682\\
% 317   21.4505378108037\\
% 336   21.5258703329214\\
% 343   21.6835383799795\\
% 361   21.9367003378412\\
% 380   22.0106874367409\\
% 387   22.1208859539571\\
% 405   22.2857639473276\\
% 431   22.4418855449587\\
% 449   22.5759499962955\\
% 475   22.6522674180268\\
% 493   22.7642325217223\\
% 519   22.8527803910971\\
% 537   22.9090798700585\\
% 563   23.0005135793708\\
% 581   23.0421956697691\\
% 607   23.0994886110111\\
% 625   23.1566730429276\\
% 651   23.1807737058295\\
% 669   23.2309320885834\\
% 695   23.2534529453161\\
% 713   23.3110189706393\\
% 739   23.3100332093008\\
% 757   23.3606722997316\\
% 783   23.3599566437383\\
% 801   23.4012467195739\\
% 827   23.4022473998143\\
% 845   23.4528759448223\\
% 871   23.4514581250825\\
% 889   23.4860219791268\\
% 915   23.4860126394255\\
% 933   23.5155539551283\\
% 977   23.5397416848769\\
% 1021  23.5703001200671\\
% 1065  23.5939124993033\\
% 1109  23.6228078313688\\
% 1153  23.6437588457624\\
% 1197  23.6589783762477\\
% 1241  23.6706054177789\\
% 1285  23.6798785549011\\
% 1329  23.6874490080291\\
% 1373  23.6969901763419\\
% 1417  23.7082686452029\\
% 1461  23.7223885029244\\
% 1505  23.7281357471112\\
% 1549  23.7340217837155\\
% 1593  23.7417496294606\\
% 1637  23.7473535157963\\
% 1681  23.7539963789325\\
% 1725  23.764191753833\\
% 1769  23.7699871806193\\
% 1813  23.7740091585566\\
% 1857  23.7784328026732\\
% 1901  23.7808720318496\\
% 1945  23.7849311938689\\
% 1990  23.7899216406005\\
% 2034  23.7910354149547\\
% 2078  23.7909069092353\\
% 2122  23.7907552477666\\
% 2166  23.788960383452\\
% 2210  23.7888594758005\\
% 2254  23.7910476166057\\
% 2298  23.794361541281\\
% 2342  23.7986491994456\\
% 2386  23.8023411135815\\
% 2430  23.8023065778923\\
% 2474  23.8019387059051\\
% 2518  23.8052800236333\\
% 2562  23.8101568993093\\
% 2606  23.8148201961974\\
% 2650  23.8180891355401\\
% 2694  23.8203970556185\\
% 2738  23.8216902259026\\
% 2782  23.8235063650783\\
% 2826  23.8259244265614\\
% 2870  23.8265041493174\\
% 2914  23.8270067564726\\
% 2958  23.8285355645068\\
% 3000  23.831071490141\\
% 3908  23.8506982109347\\
% 5559  23.8740315755862\\
% 6630  23.8855951933328\\
% 16000 23.9762405152338\\
% 16000 24.2144371749198\\
% 6630  24.1672271896646\\
% 5559  24.1508258255053\\
% 3908  24.1322128140983\\
% 3000  24.1180885682495\\
% 2958  24.1144535905078\\
% 2914  24.1156969159128\\
% 2870  24.1136181284035\\
% 2826  24.1107655770191\\
% 2782  24.1081183519384\\
% 2738  24.1057738349357\\
% 2694  24.1040971421681\\
% 2650  24.1036960638427\\
% 2606  24.1032083457032\\
% 2562  24.1020454672504\\
% 2518  24.1008338416104\\
% 2474  24.1007289644224\\
% 2430  24.0986205798349\\
% 2386  24.095537791811\\
% 2342  24.0945551983956\\
% 2298  24.0940846752377\\
% 2254  24.0947910069458\\
% 2210  24.0942125994443\\
% 2166  24.0932044002592\\
% 2122  24.0912664673482\\
% 2078  24.0926591430655\\
% 2034  24.0936391363507\\
% 1990  24.0930728823839\\
% 1945  24.0874616160208\\
% 1901  24.0827558984725\\
% 1857  24.0785655710949\\
% 1813  24.0787876229339\\
% 1769  24.0784140568109\\
% 1725  24.0717866016554\\
% 1681  24.0641944071553\\
% 1637  24.0614215739908\\
% 1593  24.0586124804984\\
% 1549  24.0519043394904\\
% 1505  24.0459766164961\\
% 1461  24.0400465170796\\
% 1417  24.0366176084191\\
% 1373  24.033313972892\\
% 1329  24.0287064127282\\
% 1285  24.0196908815231\\
% 1241  24.003231500849\\
% 1197  23.9889515671138\\
% 1153  23.9757304657656\\
% 1109  23.9626043112948\\
% 1065  23.9479562677061\\
% 1021  23.9359629336592\\
% 977   23.9205657551303\\
% 933   23.9091270880969\\
% 915   23.8948297898889\\
% 889   23.8843527754559\\
% 871   23.8736707801433\\
% 845   23.8602060949378\\
% 827   23.8450404938031\\
% 801   23.8432525017195\\
% 783   23.8140634738169\\
% 757   23.8099189365562\\
% 739   23.7851935680987\\
% 713   23.7639943167378\\
% 695   23.7266568748194\\
% 669   23.7067910784045\\
% 651   23.6673786350706\\
% 625   23.6530123483114\\
% 607   23.596031667625\\
% 581   23.5830597014005\\
% 563   23.5242604951328\\
% 537   23.5080088835209\\
% 519   23.4064890694206\\
% 493   23.3551903099296\\
% 475   23.2769694049357\\
% 449   23.2226865547694\\
% 431   23.1019038796709\\
% 405   23.0356121694315\\
% 387   22.8761317339887\\
% 380   22.7408584880227\\
% 361   22.7557063323859\\
% 343   22.5467307419294\\
% 336   22.385086215896\\
% 317   22.3519476348046\\
% 299   22.093843418271\\
% 292   21.8677581290684\\
% 273   21.7918481751758\\
% 255   21.41709852496\\
% 248   21.0877403114459\\
% 229   20.9582429420132\\
% 211   20.3587655678454\\
% 204   19.9883810056831\\
% 185   19.6986450215658\\
% 167   18.7545746546618\\
% 160   18.2402503020402\\
% 141   17.6126419388426\\
% 135   16.4015921785017\\
% 123   16.0701718413999\\
% 116   15.4582070500784\\
% 97    14.1941961624609\\
% 91    12.8147417142325\\
% 79    11.7970032679097\\
% 72    11.0332271269866\\
% 66    9.71406481712909\\
% 60    9.10283104534936\\
% 53    8.83488224464421\\
% 47    7.41000723657569\\
% 41    6.41154661366687\\
% 35    5.84215624945584\\
% };

% \addlegendentry{data8};

% \addplot [color=black!20!red,dotted]
%   table[row sep=crcr]{%
% 35    4.92036311400155\\
% 41    5.42286970993119\\
% 47    6.37269916014046\\
% 53    7.54075958996808\\
% 60    8.05267245370984\\
% 66    8.72583760005606\\
% 72    9.83591241948984\\
% 79    10.6343308356652\\
% 91    11.7634024662136\\
% 97    12.9580018301172\\
% 116   14.4120012131203\\
% 123   15.075631934637\\
% 135   15.5456340193751\\
% 141   16.5966754287567\\
% 160   17.3845001979011\\
% 167   17.9111326904975\\
% 185   18.8737427568232\\
% 204   19.2874886858075\\
% 211   19.710312383263\\
% 229   20.3152871945935\\
% 248   20.5333378935305\\
% 255   20.8474022677123\\
% 273   21.2514299492722\\
% 292   21.3690482714828\\
% 299   21.6190194122196\\
% 317   21.9012427228042\\
% 336   21.9554782744087\\
% 343   22.1151345609545\\
% 361   22.3462033351136\\
% 380   22.3757729623818\\
% 387   22.4985088439729\\
% 405   22.6606880583796\\
% 431   22.7718947123148\\
% 449   22.8993182755325\\
% 475   22.9646184114813\\
% 493   23.059711415826\\
% 519   23.1296347302589\\
% 537   23.2085443767897\\
% 563   23.2623870372518\\
% 581   23.3126276855848\\
% 607   23.3477601393181\\
% 625   23.4048426956195\\
% 651   23.4240761704501\\
% 669   23.4688615834939\\
% 695   23.4900549100677\\
% 713   23.5375066436886\\
% 739   23.5476133886997\\
% 757   23.5852956181439\\
% 783   23.5870100587776\\
% 801   23.6222496106467\\
% 827   23.6236439468087\\
% 845   23.6565410198801\\
% 871   23.6625644526129\\
% 889   23.6851873772914\\
% 915   23.6904212146572\\
% 933   23.7123405216126\\
% 977   23.7301537200036\\
% 1021  23.7531315268631\\
% 1065  23.7709343835047\\
% 1109  23.7927060713318\\
% 1153  23.809744655764\\
% 1197  23.8239649716808\\
% 1241  23.8369184593139\\
% 1285  23.8497847182121\\
% 1329  23.8580777103786\\
% 1373  23.8651520746169\\
% 1417  23.872443126811\\
% 1461  23.881217510002\\
% 1505  23.8870561818036\\
% 1549  23.892963061603\\
% 1593  23.9001810549795\\
% 1637  23.9043875448935\\
% 1681  23.9090953930439\\
% 1725  23.9179891777442\\
% 1769  23.9242006187151\\
% 1813  23.9263983907453\\
% 1857  23.928499186884\\
% 1901  23.931813965161\\
% 1945  23.9361964049448\\
% 1990  23.9414972614922\\
% 2034  23.9423372756527\\
% 2078  23.9417830261504\\
% 2122  23.9410108575574\\
% 2166  23.9410823918556\\
% 2210  23.9415360376224\\
% 2254  23.9429193117758\\
% 2298  23.9442231082594\\
% 2342  23.9466021989206\\
% 2386  23.9489394526962\\
% 2430  23.9504635788636\\
% 2474  23.9513338351637\\
% 2518  23.9530569326219\\
% 2562  23.9561011832799\\
% 2606  23.9590142709503\\
% 2650  23.9608925996914\\
% 2694  23.9622470988933\\
% 2738  23.9637320304192\\
% 2782  23.9658123585084\\
% 2826  23.9683450017902\\
% 2870  23.9700611388604\\
% 2914  23.9713518361927\\
% 2958  23.9714945775073\\
% 3000  23.9745800291952\\
% 3908  23.9914555125165\\
% 5559  24.0124287005458\\
% 6630  24.0264111914987\\
% 16000 24.0953388450768\\
% };
% \addlegendentry{Factory:LRM(6,2,2,2) (DFT)};


\addplot[LRMshade] table[] {\thisDir/data/mc-runlength/LRM-IG-uncertainty.tsv};
\label{leg:runlength-LRM-uncertainty}
%\addlegendentry{data10};

\addplot [truesys,hinfnorm]
  table[row sep=crcr]{%
35	23.9713487917402\\
16000	23.9713487917402\\
};
\label{leg:runlength-true}
%\addlegendentry{True Value (theoretical)}; 

\addplot [LRMgrid,interpol,medsmallmarkers] table[]{\thisDir/data/mc-runlength/LRM-IG.tsv};
%\addlegendentry{Factory:LRM(6,2,2,2)};
\label{leg:runlength-LRM}

\node at (axis cs:200,26.5) [color=LRM] {LRM};
\node at (axis cs:600,17) [color=LPM,rotate=50] {LPM};
\node at (axis cs:160,15) [rotate=50] {$\expectedValue{\gammaAtGrid}$};
\node at (axis cs:10000,25.75) [color=truesys] {$\hinfnorm{\Delta}$};

\end{axis}
\end{tikzpicture}%

  %\input{figures/runLength.tikz}
  \caption{The $\Hinf{}$ norm~\legref{leg:runlength-true} is underestimated significantly for short experiments ($N<20\tau$) by $\expectedValue{\gamma_{\mathrm{FRF}}}$~\legref{leg:runlength-frf} and the \gls{LPM}~\legref{leg:runlength-LPM}.
  However, the proposed $\widehat{\norm[]{\Delta_{\mathrm{LRM}}}}_{\infty}$~\legref{leg:runlength-LRM} is already reliable from much shorter experiments ($N \geq 4 \tau$). A single observation for $N=97$~\legref{leg:runlength-special-case} is shown in \figref{fig:exampleIntergrid}. The shaded areas show the $\pm \sigma$ intervals which are mostly due to the noise.
  }
\label{fig:MC-run-length}
\end{figure}

It can be seen in \figref{fig:MC-run-length} that $N \geq 20\tau$ is required to attain a frequency grid that is dense enough to obtain a reliable $\gamma_{\mathrm{FRF}}$, i.e. $\expectedValue{\gamma_{\mathrm{FRF}}} \approx \gamma$.
Starting from about $N=4 \text{ to } 5\tau$, the interpolated \gls{LRM} starts to yield good estimates for $\gamma_{\mathrm{IG}}$.
For long experiments $(N \to \infty)$, the true $\norm{\Delta}$ is recovered.
Practically, this implies that the interpolated \gls{LRM} beats the on-grid estimate using a data record that is only a fraction of the length.

\section{Application to an \glsentryname{AVIS}}
\label{sec:measurement}
The proposed method is illustrated on the \gls{AVIS} shown in \figref{fig:avis}.
The \gls{AVIS} consists of a support fixed to the ground and a payload platform, suspended on airmounts, that is free to move in six degrees of freedom.
Six geophones measure the payload velocity and eight linear motors allow to actively compensate for vibration of the platform.
To facilitate the presentation, only the vertical translation, i.e. a \gls{SISO} system,  is considered.

\subsection{Measurement \& control framework}
\label{sec:control}
The experiments on the \gls{AVIS} are carried out in closed-loop (\figref{fig:tpc}) with $r_2 = 0$.
The controller $\Controller=\experimental\Controller$  stabilizes the \gls{AVIS} but yields only moderate vibration isolation.
The $r_1$ input consists of five periods of a random phase multisine~\citep{Geerardyn2013TIM} of $65\,536$ samples with excited bins such that $\omega_k \approx 1.001 \omega_{k-1}$  and $\Ts=1\unit{ms}$.

\begin{figure}
 \centering
 \begin{tikzpicture}[scale=1,auto, >=stealth]
    \matrix[ampersand replacement=\&, row sep=0.3cm, column sep=0.6cm] {
%% FIRST ROW
        \node (r1) {$r_1$}; \&
        \&
        \&
        \&
        \node (guide) {}; \&
        \&
        \&
        \&
        \node        (uout) {$u$};  \\
%% SECOND ROW
        \node         (r2) {$r_2$}; \&
        \&
        \node[sum]    (sum2)  {};   \&
        \node[block]  (C)  {$C$};   \&
        \node[sum]    (sum1)  {};   \&
        \node[branch] (u)  {};      \&
        \node[block,color=modelset]  (P)  {$\ModelSet$};   \&
        \node[branch] (y)  {};      \&
        \node       (yout) {$y$};   \\
    };

    \draw [connector] (r2)   -- (sum2);
    \draw [connector] (sum2) -- node {$e_{\vphantom{c}}$}  (C);
    \draw [connector] (C)    -- node {$u_c$}  (sum1);
    \draw [connector] (sum1) -- (P);
    \draw [connector] (P)    -- (yout);
    \draw [connector] (r1)   -| (sum1);
    \draw [connector] (u)    |- (uout);
    \draw [connector] (y)    -- ++(0,-2em) -| node[near end,name=min] {\rotatebox{90}{$-$}} (sum2);

   \begin{pgfonlayer}{background}
     \node[groupbox=TPC, fit=(sum2) (y) (C) (min) (guide)] (TPC) {};
     \node at (TPC.north) [color=TPC, above] {$\ClosedLoop{\ModelSet[],\Controller}$};
   \end{pgfonlayer}

  \end{tikzpicture}

 \caption{The considered feedback configuration $\ClosedLoop{\ModelSet[],\Controller}$.}
 \label{fig:tpc}
\end{figure}

\begin{figure}
 \centering
 \begin{tikzpicture}[scale=1, >=stealth]
    \matrix[ampersand replacement=\&, row sep=0.2cm, column sep=0.32cm] {
%% FIRST ROW
        \node        (r1)   {$r_1$};              \&[-0.5em]
        \&
        \&
        \&
        \&
        \&
        \&
        \node[block] (NCp)   {$N_c$};              \&[-0.5em]
        \&[-0.5em]
        \node[block] (DCp)   {$D_c$};              \\
%% SECOND ROW
        \&
        \&
        \&
        \&
        \&
        \&
        \&
        \&
        \node[block] (Delta) {$\Delta$};          \\
%% THIRD ROW
        \node         (r2)   {$r_2$};             \&
        \&
        \node[sum]    (sum2) {};                  \&
        \node[block]  (DC)   {$D_c^{-1}$};        \&
        \node[block]  (NC)   {$N_c$};             \&
        \node[sum]    (sum1) {};                  \&
        \node[sum]    (u)    {};                  \&
        \node[block]  (DP)   {$\hat{D}^ {-1}$};   \&
        \&
        \node[block]  (NP)   {$\hat{N}$};         \&
        \node[sum]    (y)    {};                  \&
        \node         (yout) {$y$};               \\
    };

    \draw [connector] (r1)   -| (sum1);

    \draw [connector] (r2)   -- (sum2);
    \draw [connector] (sum2) -- (DC);
    \draw [connector] (DC)   -- (NC);
    \draw [connector] (NC)   -- (sum1);
    \draw [connector] (sum1) --  node[pos=0.3,above] {$u$} (u);
    \draw [connector] (u)    -- (DP);
    \draw [connector] (DP)   -- (NP);
    \draw [connector] (NP)   -- (y);
    
    \draw [connector] (y)    -- (yout);
    
    \draw [connector] (DP)    -| node [below,name=ud] {$u_{\Delta}$} (Delta);
    \draw [connector] (Delta) |- node [above,name=yd] {$y_{\Delta}$} (DCp) ;
    \draw [connector] (Delta) |- (NCp) ;
    \draw [connector] (DCp)   -| (y) ;
    \draw [connector] (NCp)   -| (u) ;

    \draw [connector] (y) -- ++(0.5em,0) -- ++(0,-2em) -| node[near end,left] {\rotatebox{90}{$-$}} (sum2);

    \begin{pgfonlayer}{background}

      \node[groupbox=controller,fit=(NC) (DC)]                            (C) {};
      \node[groupbox=plant,fit=(NP) (DP)]                           (P) {};
      \node[groupbox=plantset,fit=(P) (Delta) (DCp) (NCp) (y) (u)] (setP) {};

      \node at (C.north)      [color=controller, above]   {$C$};
      \node at (P.north west) [color=plant, above]  {$\hat{P}$};
      \node at (setP.north)   [color=plantset, above] {$\mathcal{P}$};

    \end{pgfonlayer}
\end{tikzpicture}

 \caption{Dual-Youla parametrization of the closed-loop set-up  with controller $\Controller$,  model set $\ModelSet[]$ and nominal plant model $\estimated\Plant$.}
\label{fig:dualYoula}
\end{figure}

The parametrization of the plant model set $\ModelSet[]$ conforms to the dual-\YK{} framework~\citep{Hansen1989,Anderson1998} that parametrizes $\ModelSet[]$ in terms of  all plants $P$ that are stabilized by the controller $C$:
\begin{equation}
  \ModelSet[] 
    \isdef 
      \set{
          \left. 
            \frac{\hat{N} + D_c \Delta}
                 {\hat{D} - N_c \Delta} 
          \right| 
            \hinfnorm{\Delta} \leq \gamma
        }
  \label{eq:dualYoulaModelSet}
  \text{,}
\end{equation}
where $C=N_c D_c^{-1}$ and $\estimated\Plant = \estimated{N}\estimated{D}^{-1}$ are decomposed as \gls{RCF} (see \figref{fig:dualYoula}).
To estimate $\Delta$ and $\hinfnorm{\Delta}$, the signals $u_{\Delta}$ and $y_{\Delta}$ in~\figref{fig:dualYoula} are required.
These can be computed from $u$ and $y$ directly~\citep{Anderson1998}.
In addition, $u_{\Delta}$ is noise free under standard assumptions~\citep{Hansen1989}, hence the method of \secref{sec:LPMHinf} applies.

The non-uniqueness of the \gls{RCF} can be exploited to satisfy additional conditions.
In view of the robust control criterion
\begin{equation}
  \Crit(\Plant,\Controller) 
    \isdef 
      \infnorm{W \ClosedLoop{\Plant,\Controller} V}
  \label{eq:controlCrit}
  \text{,}
\end{equation}
the \glspl{RCF} are constructed such that
$
  \worstCase\Crit \left( \ModelSet[], \experimental\Controller \right)
  \leq
  \Crit \left( \estimated\Plant, \experimental\Controller \right)
  + \gamma
$
where $\worstCase\Crit(\ModelSet[],C) \isdef \sup_{P\in\ModelSet[]} \Crit(P,C)$~\citep{Oomen2012SIRP}.
% Obviously, $\Crit( \true\Plant, \experimental\Controller) \leq \worstCase\Crit ( \ModelSet[], \experimental\Controller )$ is guaranteed when $\true\Plant \in \ModelSet[]$.
I.e. even for a weighted control criterion \eqref{eq:controlCrit}, there is no need to incorporate weights $W'$ and $V'$ into the estimation of the uncertainty bound as $\infnorm{W' \Delta V'}$.
In this paper, an eighth-order $\estimated\Plant$~(\figref{fig:avis-frf}) is estimated.
Some dynamics are left unmodeled and are hence part of $\Delta$~(\figref{fig:avisMeas}).

\begin{figure}
 \centering
    \setlength{\figurewidth}{0.75\columnwidth}
    \setlength{\figureheight}{0.68\figurewidth}
    % This file was created by matlab2tikz v0.4.3 (commit 2110609f2d993ab5074ce6b8f3e699d75d77a6b6).
% Copyright (c) 2008--2013, Nico Schlömer <nico.schloemer@gmail.com>
% All rights reserved.
% 
% The latest updates can be retrieved from
%   http://www.mathworks.com/matlabcentral/fileexchange/22022-matlab2tikz
% where you can also make suggestions and rate matlab2tikz.
% 
\begin{tikzpicture}

\begin{axis}[%
width=\figurewidth,
height=\figureheight,
scale only axis,
xmode=log,
xmin=9.9,
xmax=500,
ymajorgrids,
xmajorgrids,
xminorgrids,
xminorticks=true,
ymin=-70,
ymax=16,
ylabel={Amplitude $\abs{P}$ \axisunit{dB}},
xlabel={Frequency $\omega$  \axisunit{Hz}},
name=plot0
]

\addplot [paramPhat]
table[row sep=crcr] {
% 0.0152587890625 23.7722447525977\\
% 0.0457763671875 22.3080847897672\\
% 0.0762939453125 20.2741569707957\\
% 0.1068115234375 18.2257529897462\\
% 0.1373291015625 16.3079974146639\\
% 0.1678466796875 14.5181085972517\\
% 0.1983642578125 12.8172522087776\\
% 0.2288818359375 11.1608733787134\\
% 0.2593994140625 9.50397723607956\\
% 0.2899169921875 7.79899510770321\\
% 0.3204345703125 5.99029715556577\\
% 0.3509521484375 4.00469298301403\\
% 0.3814697265625 1.73365389099158\\
% 0.4119873046875 -1.00426577865468\\
% 0.4425048828125 -4.54966533123516\\
% 0.4730224609375 -9.51643537709583\\
% 0.5035400390625 -13.8964915122691\\
% 0.5340576171875 -9.50421779076407\\
% 0.5645751953125 -5.12612382304163\\
% 0.5950927734375 -2.12859036094432\\
% 0.6256103515625 0.0795243024252674\\
% 0.6561279296875 1.81283230611915\\
% 0.6866455078125 3.23551080308591\\
% 0.7171630859375 4.44140167314812\\
% 0.7476806640625 5.48852713143906\\
% 0.7781982421875 6.41503322856487\\
% 0.8087158203125 7.24723490797363\\
% 0.8392333984375 8.00399673011526\\
% 0.8697509765625 8.69927256044658\\
% 0.9002685546875 9.34365587383497\\
% 0.9307861328125 9.94536758467433\\
% 0.9613037109375 10.5109086205\\
% 0.9918212890625 11.0455044893486\\
% 1.0223388671875 11.5534162767064\\
% 1.0528564453125 12.0381632797992\\
% 1.0833740234375 12.5026856473201\\
% 1.1138916015625 12.9494653430929\\
% 1.1444091796875 13.3806175660085\\
% 1.1749267578125 13.7979608445429\\
% 1.2054443359375 14.2030714862825\\
% 1.2359619140625 14.5973263807037\\
% 1.2664794921875 14.9819370161158\\
% 1.2969970703125 15.3579767887555\\
% 1.3275146484375 15.7264031341724\\
% 1.3580322265625 16.0880756219148\\
% 1.3885498046875 16.4437708743058\\
% 1.4190673828125 16.7941949657516\\
% 1.4495849609375 17.1399938082556\\
% 1.4801025390625 17.4817619163763\\
% 1.5106201171875 17.8200498601774\\
% 1.5411376953125 18.1553706503383\\
% 1.5716552734375 18.4882052502356\\
% 1.6021728515625 18.8190073716575\\
% 1.6326904296875 19.1482076811222\\
% 1.6632080078125 19.4762175204909\\
% 1.6937255859375 19.8034322272109\\
% 1.7242431640625 20.1302341249468\\
% 1.7547607421875 20.4569952437229\\
% 1.7852783203125 20.7840798193475\\
% 1.8157958984375 21.1118466143154\\
% 1.8463134765625 21.4406510961916\\
% 1.8768310546875 21.7708475043434\\
% 1.9073486328125 22.1027908315403\\
% 1.9378662109375 22.4368387431394\\
% 1.9683837890625 22.7733534531064\\
% 1.9989013671875 23.112703572744\\
% 2.0294189453125 23.4552659444922\\
% 2.0599365234375 23.8014274692076\\
% 2.0904541015625 24.1515869305938\\
% 2.1209716796875 24.5061568144554\\
% 2.1514892578125 24.8655651125802\\
% 2.1820068359375 25.2302570904768\\
% 2.2125244140625 25.600696983754\\
% 2.2430419921875 25.9773695680442\\
% 2.2735595703125 26.3607815198056\\
% 2.3040771484375 26.7514624470049\\
% 2.3345947265625 27.1499654151986\\
% 2.3651123046875 27.556866719776\\
% 2.3956298828125 27.9727645504501\\
% 2.4261474609375 28.398276047329\\
% 2.4566650390625 28.8340320419987\\
% 2.4871826171875 29.2806684882203\\
% 2.5177001953125 29.738813182202\\
% 2.5482177734375 30.2090658074338\\
% 2.5787353515625 30.6919685550836\\
% 2.6092529296875 31.1879634947029\\
% 2.6397705078125 31.6973314186676\\
% 2.6702880859375 32.2201049841374\\
% 2.7008056640625 32.7559466142805\\
% 2.7313232421875 33.3039789538858\\
% 2.7618408203125 33.8625532646832\\
% 2.7923583984375 34.4289403934151\\
% 2.8228759765625 34.998932818317\\
% 2.8533935546875 35.5663602956612\\
% 2.8839111328125 36.1225545670644\\
% 2.9144287109375 36.6558610950166\\
% 2.9449462890625 37.1513927585097\\
% 2.9754638671875 37.5913317342492\\
% 3.0059814453125 37.9561374983115\\
% 3.0364990234375 38.2268760492636\\
% 3.0670166015625 38.388435532638\\
% 3.0975341796875 38.4327465267617\\
% 3.1280517578125 38.3607353513796\\
% 3.1585693359375 38.1820974821149\\
% 3.1890869140625 37.9130018010624\\
% 3.2196044921875 37.5727752169157\\
% 3.2501220703125 37.1808111366764\\
% 3.2806396484375 36.7544592218602\\
% 3.3111572265625 36.3080196085472\\
% 3.3416748046875 35.8525763470607\\
% 3.3721923828125 35.3963117506141\\
% 3.4027099609375 34.9450157498753\\
% 3.4332275390625 34.5026165737593\\
% 3.4637451171875 34.0716497331852\\
% 3.4942626953125 33.6536382712904\\
% 3.5247802734375 33.249385659551\\
% 3.5552978515625 32.8591941702117\\
% 3.5858154296875 32.4830241866703\\
% 3.6163330078125 32.1206086666063\\
% 3.6468505859375 31.771534437305\\
% 3.6773681640625 31.4352993686595\\
% 3.7078857421875 31.1113521929172\\
% 3.7384033203125 30.7991199322042\\
% 3.7689208984375 30.4980265233318\\
% 3.7994384765625 30.2075052166461\\
% 3.8299560546875 29.9270065898856\\
% 3.8604736328125 29.6560034886193\\
% 3.8909912109375 29.3939938260787\\
% 3.9215087890625 29.1405019050057\\
% 3.9520263671875 28.8950787316553\\
% 3.9825439453125 28.6573016549751\\
% 4.0130615234375 28.4267735662919\\
% 4.0435791015625 28.2031218251764\\
% 4.0740966796875 27.9859970274679\\
% 4.1046142578125 27.775071695985\\
% 4.1351318359375 27.5700389491584\\
% 4.1656494140625 27.3706111848061\\
% 4.1961669921875 27.1765188034565\\
% 4.2266845703125 26.9875089865579\\
% 4.2572021484375 26.8033445385226\\
% 4.2877197265625 26.6238027970888\\
% 4.3182373046875 26.448674613411\\
% 4.3487548828125 26.2777634012001\\
% 4.3792724609375 26.1108842528586\\
% 4.4097900390625 25.9478631196883\\
% 4.4403076171875 25.7885360527342\\
% 4.4708251953125 25.6327485005659\\
% 4.5013427734375 25.4803546602147\\
% 4.5318603515625 25.3312168775128\\
% 4.5623779296875 25.1852050931961\\
% 4.5928955078125 25.0421963312868\\
% 4.6234130859375 24.9020742264696\\
% 4.6539306640625 24.7647285873746\\
% 4.6844482421875 24.6300549928949\\
% 4.7149658203125 24.4979544188742\\
% 4.7454833984375 24.3683328927045\\
% 4.7760009765625 24.2411011735653\\
% 4.8065185546875 24.1161744562232\\
% 4.8370361328125 23.9934720964781\\
% 4.8675537109375 23.8729173565049\\
% 4.8980712890625 23.7544371684859\\
% 4.9285888671875 23.6379619150655\\
% 4.9591064453125 23.5234252252845\\
% 4.9896240234375 23.4107637847641\\
% 5.0201416015625 23.2999171590175\\
% 5.0506591796875 23.1908276288606\\
% 5.0811767578125 23.0834400369808\\
% 5.1116943359375 22.9777016448047\\
% 5.1422119140625 22.8735619988752\\
% 5.1727294921875 22.7709728060182\\
% 5.2032470703125 22.6698878166361\\
% 5.2337646484375 22.5702627155228\\
% 5.2642822265625 22.4720550196442\\
% 5.2947998046875 22.3752239823726\\
% 5.3253173828125 22.2797305037082\\
% 5.3558349609375 22.1855370460544\\
% 5.3863525390625 22.0926075551535\\
% 5.4168701171875 22.0009073858157\\
% 5.4473876953125 21.9104032321067\\
% 5.4779052734375 21.8210630616861\\
% 5.5084228515625 21.7328560540075\\
% 5.5389404296875 21.6457525421218\\
% 5.5694580078125 21.5597239578368\\
% 5.5999755859375 21.4747427800109\\
% 5.6304931640625 21.3907824857722\\
% 5.6610107421875 21.3078175044717\\
% 5.6915283203125 21.2258231741913\\
% 5.7220458984375 21.1447757006444\\
% 5.7525634765625 21.0646521183131\\
% 5.7830810546875 20.9854302536835\\
% 5.8135986328125 20.9070886904451\\
% 5.8441162109375 20.8296067365336\\
% 5.8746337890625 20.7529643929023\\
% 5.9051513671875 20.6771423239184\\
% 5.9356689453125 20.6021218292832\\
% 5.9661865234375 20.5278848173871\\
% 5.9967041015625 20.4544137800129\\
% 6.0272216796875 20.3816917683075\\
% 6.0577392578125 20.3097023699493\\
% 6.0882568359375 20.2384296874399\\
% 6.1187744140625 20.1678583174577\\
% 6.1492919921875 20.0979733312107\\
% 6.1798095703125 20.0287602557339\\
% 6.2103271484375 19.9602050560772\\
% 6.2408447265625 19.8922941183343\\
% 6.2713623046875 19.8250142334668\\
% 6.3018798828125 19.7583525818799\\
% 6.3323974609375 19.6922967187082\\
% 6.3629150390625 19.6268345597746\\
% 6.3934326171875 19.5619543681849\\
% 6.4239501953125 19.4976447415258\\
% 6.4544677734375 19.4338945996339\\
% 6.4849853515625 19.370693172906\\
% 6.5155029296875 19.3080299911225\\
% 6.5460205078125 19.2458948727579\\
% 6.5765380859375 19.1842779147538\\
% 6.6070556640625 19.1231694827303\\
% 6.6375732421875 19.0625602016145\\
% 6.6680908203125 19.0024409466643\\
% 6.6986083984375 18.9428028348697\\
% 6.7291259765625 18.8836372167118\\
% 6.7596435546875 18.8249356682611\\
% 6.7901611328125 18.7666899836022\\
% 6.8206787109375 18.7088921675647\\
% 6.8511962890625 18.6515344287496\\
% 6.8817138671875 18.5946091728347\\
% 6.9122314453125 18.5381089961471\\
% 6.9427490234375 18.4820266794898\\
% 6.9732666015625 18.4263551822117\\
% 7.0037841796875 18.3710876365075\\
% 7.0343017578125 18.3162173419396\\
% 7.0648193359375 18.2617377601706\\
% 7.0953369140625 18.2076425098967\\
% 7.1258544921875 18.1539253619737\\
% 7.1563720703125 18.1005802347269\\
% 7.1868896484375 18.0476011894361\\
% 7.2174072265625 17.9949824259891\\
% 7.2479248046875 17.9427182786955\\
% 7.2784423828125 17.8908032122542\\
% 7.3089599609375 17.8392318178679\\
% 7.3394775390625 17.7879988094988\\
% 7.3699951171875 17.7370990202583\\
% 7.4005126953125 17.6865273989263\\
% 7.4310302734375 17.6362790065947\\
% 7.4615478515625 17.5863490134282\\
% 7.4920654296875 17.5367326955395\\
% 7.5225830078125 17.487425431974\\
% 7.5531005859375 17.4384227017976\\
% 7.5836181640625 17.389720081286\\
% 7.6141357421875 17.3413132412099\\
% 7.6446533203125 17.2931979442118\\
% 7.6751708984375 17.2453700422729\\
% 7.7056884765625 17.1978254742634\\
% 7.7362060546875 17.1505602635765\\
% 7.7667236328125 17.1035705158395\\
% 7.7972412109375 17.0568524167007\\
% 7.8277587890625 17.0104022296899\\
% 7.8582763671875 16.9642162941472\\
% 7.8887939453125 16.9182910232204\\
% 7.9193115234375 16.8726229019257\\
% 7.9498291015625 16.8272084852716\\
% 7.9803466796875 16.7820443964416\\
% 8.0108642578125 16.7371273250358\\
% 8.0413818359375 16.6924540253663\\
% 8.0718994140625 16.6480213148072\\
% 8.1024169921875 16.6038260721958\\
% 8.1329345703125 16.5598652362826\\
% 8.1634521484375 16.5161358042302\\
% 8.1939697265625 16.472634830157\\
% 8.2244873046875 16.4293594237263\\
% 8.2550048828125 16.3863067487773\\
% 8.2855224609375 16.3434740219981\\
% 8.3160400390625 16.3008585116377\\
% 8.3465576171875 16.2584575362571\\
% 8.3770751953125 16.2162684635168\\
% 8.4075927734375 16.1742887090009\\
% 8.4381103515625 16.1325157350745\\
% 8.4686279296875 16.0909470497756\\
% 8.4991455078125 16.0495802057382\\
% 8.5296630859375 16.0084127991474\\
% 8.5601806640625 15.9674424687237\\
% 8.5906982421875 15.9266668947367\\
% 8.6212158203125 15.8860837980471\\
% 8.6517333984375 15.8456909391749\\
% 8.6822509765625 15.8054861173949\\
% 8.7127685546875 15.7654671698568\\
% 8.7432861328125 15.72563197073\\
% 8.7738037109375 15.685978430372\\
% 8.8043212890625 15.6465044945199\\
% 8.8348388671875 15.6072081435041\\
% 8.8653564453125 15.5680873914829\\
% 8.8958740234375 15.529140285699\\
% 8.9263916015625 15.4903649057547\\
% 8.9569091796875 15.4517593629078\\
% 8.9874267578125 15.4133217993855\\
% 9.0179443359375 15.3750503877171\\
% 9.0484619140625 15.3369433300839\\
% 9.0789794921875 15.298998857687\\
% 9.1094970703125 15.2612152301306\\
% 9.1400146484375 15.2235907348225\\
% 9.1705322265625 15.1861236863892\\
% 9.2010498046875 15.1488124261069\\
% 9.2315673828125 15.111655321347\\
% 9.2620849609375 15.0746507650352\\
% 9.2926025390625 15.0377971751251\\
% 9.3231201171875 15.001092994085\\
% 9.3536376953125 14.9645366883972\\
% 9.3841552734375 14.9281267480708\\
% 9.4146728515625 14.8918616861655\\
% 9.4451904296875 14.8557400383283\\
% 9.4757080078125 14.8197603623414\\
% 9.5062255859375 14.7839212376807\\
% 9.5367431640625 14.748221265086\\
% 9.5672607421875 14.7126590661409\\
% 9.5977783203125 14.677233282864\\
% 9.6282958984375 14.6419425773086\\
% 9.6588134765625 14.6067856311732\\
% 9.6893310546875 14.5717611454208\\
% 9.7198486328125 14.5368678399078\\
% 9.7503662109375 14.5021044530208\\
9.7808837890625 14.4674697413231\\
9.8114013671875 14.4329624792088\\
9.8419189453125 14.3985814585653\\
9.8724365234375 14.3643254884439\\
9.9029541015625 14.3301933947372\\
9.9334716796875 14.2961840198655\\
9.9639892578125 14.2622962224688\\
9.9945068359375 14.228528877107\\
10.0250244140625 14.1948808739665\\
10.0555419921875 14.1613511185736\\
10.0860595703125 14.1279385315143\\
10.1165771484375 14.0946420481607\\
10.1470947265625 14.0614606184031\\
10.1776123046875 14.0283932063886\\
10.2081298828125 13.9954387902654\\
10.2386474609375 13.9625963619322\\
10.2691650390625 13.9298649267941\\
10.2996826171875 13.8972435035233\\
10.3302001953125 13.8647311238251\\
10.3607177734375 13.8323268322089\\
10.3912353515625 13.8000296857646\\
10.4217529296875 13.7678387539438\\
10.4522705078125 13.7357531183448\\
10.4827880859375 13.7037718725037\\
10.5133056640625 13.6718941216888\\
10.5438232421875 13.6401189826999\\
10.5743408203125 13.6084455836716\\
10.6048583984375 13.5768730638813\\
10.6353759765625 13.5454005735605\\
10.6658935546875 13.5140272737106\\
10.6964111328125 13.4827523359222\\
10.7269287109375 13.4515749421985\\
10.7574462890625 13.4204942847822\\
10.7879638671875 13.3895095659856\\
10.8184814453125 13.3586199980252\\
10.8489990234375 13.3278248028585\\
10.8795166015625 13.2971232120247\\
10.9100341796875 13.266514466489\\
10.9405517578125 13.2359978164894\\
10.9710693359375 13.2055725213867\\
11.0015869140625 13.175237849518\\
11.0321044921875 13.1449930780526\\
11.0626220703125 13.1148374928508\\
11.0931396484375 13.0847703883262\\
11.1236572265625 13.0547910673096\\
11.1541748046875 13.0248988409165\\
11.1846923828125 12.9950930284171\\
11.2152099609375 12.9653729571079\\
11.2457275390625 12.9357379621872\\
11.2762451171875 12.9061873866321\\
11.3067626953125 12.876720581078\\
11.3372802734375 12.847336903701\\
11.3677978515625 12.8180357201012\\
11.3983154296875 12.7888164031902\\
11.4288330078125 12.7596783330788\\
11.4593505859375 12.7306208969682\\
11.4898681640625 12.7016434890426\\
11.5203857421875 12.6727455103641\\
11.5509033203125 12.643926368769\\
11.5814208984375 12.6151854787671\\
11.6119384765625 12.5865222614421\\
11.6424560546875 12.5579361443536\\
11.6729736328125 12.529426561442\\
11.7034912109375 12.5009929529339\\
11.7340087890625 12.4726347652501\\
11.7645263671875 12.444351450915\\
11.7950439453125 12.4161424684678\\
11.8255615234375 12.3880072823747\\
11.8560791015625 12.3599453629436\\
11.8865966796875 12.3319561862398\\
11.9171142578125 12.3040392340031\\
11.9476318359375 12.276193993567\\
11.9781494140625 12.2484199577786\\
12.0086669921875 12.2207166249204\\
12.0391845703125 12.1930834986332\\
12.0697021484375 12.1655200878409\\
12.1002197265625 12.1380259066756\\
12.1307373046875 12.1106004744055\\
12.1612548828125 12.0832433153624\\
12.1917724609375 12.0559539588717\\
12.2222900390625 12.0287319391832\\
12.2528076171875 12.0015767954029\\
12.2833251953125 11.9744880714262\\
12.3138427734375 11.9474653158724\\
12.3443603515625 11.92050808202\\
12.3748779296875 11.8936159277431\\
12.4053955078125 11.8667884154491\\
12.4359130859375 11.8400251120177\\
12.4664306640625 11.8133255887401\\
12.4969482421875 11.7866894212599\\
12.5274658203125 11.7601161895151\\
12.5579833984375 11.7336054776803\\
12.5885009765625 11.7071568741107\\
12.6190185546875 11.6807699712867\\
12.6495361328125 11.6544443657591\\
12.6800537109375 11.6281796580957\\
12.7105712890625 11.6019754528288\\
12.7410888671875 11.5758313584027\\
12.7716064453125 11.5497469871235\\
12.8021240234375 11.5237219551083\\
12.8326416015625 11.4977558822361\\
12.8631591796875 11.4718483920992\\
12.8936767578125 11.4459991119555\\
12.9241943359375 11.4202076726812\\
12.9547119140625 11.3944737087249\\
12.9852294921875 11.368796858062\\
13.0157470703125 11.3431767621495\\
13.0462646484375 11.3176130658826\\
13.0767822265625 11.2921054175508\\
13.1072998046875 11.2666534687952\\
13.1378173828125 11.241256874567\\
13.1683349609375 11.2159152930852\\
13.1988525390625 11.190628385797\\
13.2293701171875 11.1653958173365\\
13.2598876953125 11.1402172554864\\
13.2904052734375 11.1150923711383\\
13.3209228515625 11.0900208382547\\
13.3514404296875 11.0650023338314\\
13.3819580078125 11.0400365378604\\
13.4124755859375 11.0151231332933\\
13.4429931640625 10.9902618060052\\
13.4735107421875 10.9654522447596\\
13.5040283203125 10.9406941411732\\
13.5345458984375 10.9159871896818\\
13.5650634765625 10.8913310875061\\
13.5955810546875 10.8667255346189\\
13.6260986328125 10.8421702337115\\
13.6566162109375 10.8176648901621\\
13.6871337890625 10.7932092120036\\
13.7176513671875 10.768802909892\\
13.7481689453125 10.7444456970759\\
13.7786865234375 10.7201372893656\\
13.8092041015625 10.6958774051034\\
13.8397216796875 10.6716657651338\\
13.8702392578125 10.6475020927743\\
13.9007568359375 10.6233861137871\\
13.9312744140625 10.5993175563502\\
13.9617919921875 10.5752961510297\\
13.9923095703125 10.5513216307524\\
14.0228271484375 10.5273937307787\\
14.0533447265625 10.5035121886758\\
14.0838623046875 10.4796767442912\\
14.1143798828125 10.455887139727\\
14.1448974609375 10.4321431193141\\
14.1754150390625 10.4084444295873\\
14.2059326171875 10.3847908192599\\
14.2364501953125 10.3611820391997\\
14.2669677734375 10.3376178424043\\
14.2974853515625 10.3140979839777\\
14.3280029296875 10.2906222211067\\
14.3585205078125 10.2671903130375\\
14.3890380859375 10.2438020210529\\
14.4195556640625 10.22045710845\\
14.4500732421875 10.1971553405177\\
14.4805908203125 10.1738964845149\\
14.5111083984375 10.1506803096488\\
14.5416259765625 10.1275065870537\\
14.5721435546875 10.1043750897697\\
14.6026611328125 10.0812855927223\\
14.6331787109375 10.0582378727017\\
14.6636962890625 10.0352317083424\\
14.6942138671875 10.012266880104\\
14.7247314453125 9.98934317025066\\
14.7552490234375 9.96646036283231\\
14.7857666015625 9.9436182436653\\
14.8162841796875 9.92081660031357\\
14.8468017578125 9.89805522207\\
14.8773193359375 9.87533389993805\\
14.9078369140625 9.85265242661365\\
14.9383544921875 9.83001059646725\\
14.9688720703125 9.80740820552622\\
14.9993896484375 9.78484505145737\\
15.0299072265625 9.7623209335498\\
15.0604248046875 9.73983565269793\\
15.0909423828125 9.71738901138468\\
15.1214599609375 9.69498081366498\\
15.1519775390625 9.6726108651495\\
15.1824951171875 9.65027897298842\\
15.2130126953125 9.62798494585566\\
15.2435302734375 9.60572859393307\\
15.2740478515625 9.583509728895\\
15.3045654296875 9.561328163893\\
15.3350830078125 9.53918371354067\\
15.3656005859375 9.5170761938988\\
15.3961181640625 9.49500542246062\\
15.4266357421875 9.47297121813725\\
15.4571533203125 9.45097340124339\\
15.4876708984375 9.42901179348313\\
15.5181884765625 9.40708621793591\\
15.5487060546875 9.38519649904277\\
15.5792236328125 9.36334246259269\\
15.6097412109375 9.34152393570905\\
15.6402587890625 9.31974074683639\\
15.6707763671875 9.29799272572725\\
15.7012939453125 9.27627970342915\\
15.7318115234375 9.25460151227182\\
15.7623291015625 9.23295798585454\\
15.7928466796875 9.21134895903353\\
15.8233642578125 9.1897742679097\\
15.8538818359375 9.16823374981642\\
15.8843994140625 9.14672724330747\\
15.9149169921875 9.12525458814505\\
15.9454345703125 9.10381562528812\\
15.9759521484375 9.08241019688072\\
16.0064697265625 9.0610381462405\\
16.0369873046875 9.03969931784732\\
16.0675048828125 9.01839355733217\\
16.0980224609375 8.99712071146595\\
16.1285400390625 8.97588062814859\\
16.1590576171875 8.95467315639825\\
16.1895751953125 8.9334981463406\\
16.2200927734375 8.91235544919827\\
16.2506103515625 8.89124491728043\\
16.2811279296875 8.87016640397246\\
16.3116455078125 8.84911976372578\\
16.3421630859375 8.82810485204775\\
16.3726806640625 8.80712152549171\\
16.4031982421875 8.78616964164721\\
16.4337158203125 8.76524905913021\\
16.4642333984375 8.7443596375735\\
16.4947509765625 8.72350123761718\\
16.5252685546875 8.70267372089935\\
16.5557861328125 8.68187695004671\\
16.5863037109375 8.66111078866546\\
16.6168212890625 8.64037510133222\\
16.6473388671875 8.61966975358509\\
16.6778564453125 8.59899461191467\\
16.7083740234375 8.57834954375548\\
16.7388916015625 8.55773441747715\\
16.7694091796875 8.53714910237591\\
16.7999267578125 8.51659346866614\\
16.8304443359375 8.49606738747194\\
16.8609619140625 8.4755707308189\\
16.8914794921875 8.45510337162595\\
16.9219970703125 8.43466518369715\\
16.9525146484375 8.41425604171381\\
16.9830322265625 8.39387582122649\\
17.0135498046875 8.37352439864721\\
17.0440673828125 8.35320165124173\\
17.0745849609375 8.33290745712187\\
17.1051025390625 8.31264169523793\\
17.1356201171875 8.29240424537123\\
17.1661376953125 8.27219498812669\\
17.1966552734375 8.25201380492555\\
17.2271728515625 8.23186057799806\\
17.2576904296875 8.21173519037637\\
17.2882080078125 8.19163752588742\\
17.3187255859375 8.17156746914593\\
17.3492431640625 8.15152490554753\\
17.3797607421875 8.13150972126179\\
17.4102783203125 8.11152180322553\\
17.4407958984375 8.09156103913611\\
17.4713134765625 8.07162731744471\\
17.5018310546875 8.05172052734986\\
17.5323486328125 8.03184055879088\\
17.5628662109375 8.0119873024415\\
17.5933837890625 7.99216064970347\\
17.6239013671875 7.97236049270026\\
17.6544189453125 7.95258672427089\\
17.6849365234375 7.93283923796372\\
17.7154541015625 7.91311792803038\\
17.7459716796875 7.89342268941975\\
17.7764892578125 7.87375341777194\\
17.8070068359375 7.8541100094125\\
17.8375244140625 7.83449236134646\\
17.8680419921875 7.81490037125262\\
17.8985595703125 7.79533393747781\\
17.9290771484375 7.77579295903122\\
17.9595947265625 7.75627733557886\\
17.9901123046875 7.73678696743791\\
18.0206298828125 7.71732175557132\\
18.0511474609375 7.69788160158236\\
18.0816650390625 7.67846640770924\\
18.1121826171875 7.6590760768198\\
18.1427001953125 7.63971051240622\\
18.1732177734375 7.62036961857987\\
18.2037353515625 7.60105330006607\\
18.2342529296875 7.58176146219908\\
18.2647705078125 7.56249401091696\\
18.2952880859375 7.54325085275665\\
18.3258056640625 7.52403189484894\\
18.3563232421875 7.50483704491362\\
18.3868408203125 7.48566621125467\\
18.4173583984375 7.46651930275532\\
18.4478759765625 7.44739622887348\\
18.4783935546875 7.42829689963687\\
18.5089111328125 7.40922122563846\\
18.5394287109375 7.39016911803183\\
18.5699462890625 7.37114048852659\\
18.6004638671875 7.3521352493839\\
18.6309814453125 7.3331533134119\\
18.6614990234375 7.31419459396143\\
18.6920166015625 7.2952590049215\\
18.7225341796875 7.27634646071503\\
18.7530517578125 7.25745687629449\\
18.7835693359375 7.23859016713771\\
18.8140869140625 7.21974624924357\\
18.8446044921875 7.20092503912794\\
18.8751220703125 7.18212645381942\\
18.9056396484375 7.16335041085535\\
18.9361572265625 7.14459682827768\\
18.9666748046875 7.125865624629\\
18.9971923828125 7.10715671894853\\
19.0277099609375 7.08847003076823\\
19.0582275390625 7.06980548010881\\
19.0887451171875 7.05116298747598\\
19.1192626953125 7.03254247385653\\
19.1497802734375 7.01394386071459\\
19.1802978515625 6.99536706998786\\
19.2108154296875 6.97681202408391\\
19.2413330078125 6.95827864587645\\
19.2718505859375 6.93976685870177\\
19.3023681640625 6.92127658635502\\
19.3328857421875 6.90280775308673\\
19.3634033203125 6.88436028359922\\
19.3939208984375 6.86593410304305\\
19.4244384765625 6.84752913701364\\
19.4549560546875 6.82914531154775\\
19.4854736328125 6.81078255312009\\
19.5159912109375 6.79244078863994\\
19.5465087890625 6.7741199454478\\
19.5770263671875 6.75581995131207\\
19.6075439453125 6.73754073442577\\
19.6380615234375 6.71928222340328\\
19.6685791015625 6.70104434727712\\
19.6990966796875 6.68282703549474\\
19.7296142578125 6.66463021791537\\
19.7601318359375 6.64645382480689\\
19.7906494140625 6.6282977868427\\
19.8211669921875 6.61016203509864\\
19.8516845703125 6.59204650104995\\
19.8822021484375 6.57395111656827\\
19.9127197265625 6.55587581391859\\
19.9432373046875 6.53782052575635\\
19.9737548828125 6.51978518512443\\
20.0042724609375 6.50176972545029\\
20.0347900390625 6.48377408054303\\
20.0653076171875 6.46579818459061\\
20.0958251953125 6.44784197215695\\
20.1263427734375 6.42990537817908\\
20.1568603515625 6.41198833796449\\
20.1873779296875 6.39409078718827\\
20.2178955078125 6.37621266189035\\
20.2484130859375 6.3583538984729\\
20.2789306640625 6.34051443369755\\
20.3094482421875 6.32269420468276\\
20.3399658203125 6.30489314890121\\
20.3704833984375 6.28711120417712\\
20.4010009765625 6.26934830868375\\
20.4315185546875 6.25160440094071\\
20.4620361328125 6.23387941981156\\
20.4925537109375 6.21617330450117\\
20.5230712890625 6.19848599455327\\
20.5535888671875 6.18081742984796\\
20.5841064453125 6.16316755059928\\
20.6146240234375 6.14553629735274\\
20.6451416015625 6.12792361098292\\
20.6756591796875 6.11032943269106\\
20.7061767578125 6.09275370400278\\
20.7366943359375 6.07519636676555\\
20.7672119140625 6.05765736314657\\
20.7977294921875 6.04013663563027\\
20.8282470703125 6.02263412701618\\
20.8587646484375 6.00514978041654\\
20.8892822265625 5.98768353925414\\
20.9197998046875 5.97023534726001\\
20.9503173828125 5.95280514847126\\
20.9808349609375 5.93539288722887\\
21.0113525390625 5.91799850817557\\
21.0418701171875 5.90062195625357\\
21.0723876953125 5.88326317670251\\
21.1029052734375 5.86592211505734\\
21.1334228515625 5.8485987171462\\
21.1639404296875 5.83129292908835\\
21.1944580078125 5.81400469729206\\
21.2249755859375 5.79673396845264\\
21.2554931640625 5.77948068955037\\
21.2860107421875 5.76224480784847\\
21.3165283203125 5.74502627089113\\
21.3470458984375 5.72782502650156\\
21.3775634765625 5.71064102277996\\
21.4080810546875 5.69347420810163\\
21.4385986328125 5.67632453111502\\
21.4691162109375 5.65919194073982\\
21.4996337890625 5.64207638616508\\
21.5301513671875 5.62497781684731\\
21.5606689453125 5.6078961825086\\
21.5911865234375 5.59083143313484\\
21.6217041015625 5.57378351897381\\
21.6522216796875 5.55675239053339\\
21.6827392578125 5.53973799857976\\
21.7132568359375 5.52274029413563\\
21.7437744140625 5.50575922847843\\
21.7742919921875 5.48879475313859\\
21.8048095703125 5.47184681989774\\
21.8353271484375 5.45491538078707\\
21.8658447265625 5.43800038808551\\
21.8963623046875 5.42110179431809\\
21.9268798828125 5.40421955225429\\
21.9573974609375 5.38735361490626\\
21.9879150390625 5.37050393552727\\
22.0184326171875 5.35367046760995\\
22.0489501953125 5.3368531648848\\
22.0794677734375 5.32005198131844\\
22.1099853515625 5.30326687111208\\
22.1405029296875 5.28649778869987\\
22.1710205078125 5.26974468874741\\
22.2015380859375 5.25300752615007\\
22.2320556640625 5.23628625603154\\
22.2625732421875 5.21958083374221\\
22.2930908203125 5.20289121485766\\
22.3236083984375 5.1862173551772\\
22.3541259765625 5.16955921072225\\
22.3846435546875 5.152916737735\\
22.4151611328125 5.13628989267676\\
22.4456787109375 5.11967863222663\\
22.4761962890625 5.10308291327997\\
22.5067138671875 5.08650269294698\\
22.5372314453125 5.06993792855124\\
22.5677490234375 5.05338857762833\\
22.5982666015625 5.03685459792438\\
22.6287841796875 5.02033594739469\\
22.6593017578125 5.00383258420232\\
22.6898193359375 4.98734446671676\\
22.7203369140625 4.97087155351248\\
22.7508544921875 4.9544138033677\\
22.7813720703125 4.93797117526287\\
22.8118896484375 4.92154362837953\\
22.8424072265625 4.90513112209883\\
22.8729248046875 4.88873361600028\\
22.9034423828125 4.87235106986048\\
22.9339599609375 4.85598344365172\\
22.9644775390625 4.83963069754083\\
22.9949951171875 4.8232927918878\\
23.0255126953125 4.8069696872446\\
23.0560302734375 4.79066134435382\\
23.0865478515625 4.77436772414755\\
23.1170654296875 4.75808878774606\\
23.1475830078125 4.74182449645659\\
23.1781005859375 4.72557481177218\\
23.2086181640625 4.70933969537039\\
23.2391357421875 4.69311910911218\\
23.2696533203125 4.67691301504068\\
23.3001708984375 4.66072137538\\
23.3306884765625 4.64454415253409\\
23.3612060546875 4.62838130908558\\
23.3917236328125 4.61223280779461\\
23.4222412109375 4.59609861159771\\
23.4527587890625 4.57997868360663\\
23.4832763671875 4.56387298710727\\
23.5137939453125 4.54778148555851\\
23.5443115234375 4.53170414259114\\
23.5748291015625 4.51564092200674\\
23.6053466796875 4.49959178777659\\
23.6358642578125 4.4835567040406\\
23.6663818359375 4.46753563510624\\
23.6968994140625 4.45152854544744\\
23.7274169921875 4.43553539970357\\
23.7579345703125 4.4195561626784\\
23.7884521484375 4.40359079933896\\
23.8189697265625 4.38763927481464\\
23.8494873046875 4.3717015543961\\
23.8800048828125 4.35577760353423\\
23.9105224609375 4.33986738783917\\
23.9410400390625 4.3239708730793\\
23.9715576171875 4.30808802518024\\
24.0020751953125 4.29221881022387\\
24.0325927734375 4.27636319444734\\
24.0631103515625 4.26052114424211\\
24.0936279296875 4.24469262615296\\
24.1241455078125 4.22887760687703\\
24.1546630859375 4.21307605326291\\
24.1851806640625 4.19728793230965\\
24.2156982421875 4.18151321116586\\
24.2462158203125 4.16575185712871\\
24.2767333984375 4.1500038376431\\
24.3072509765625 4.13426912030068\\
24.3377685546875 4.11854767283894\\
24.3682861328125 4.10283946314035\\
24.3988037109375 4.08714445923141\\
24.4293212890625 4.07146262928178\\
24.4598388671875 4.05579394160342\\
24.4903564453125 4.04013836464967\\
24.5208740234375 4.02449586701443\\
24.5513916015625 4.00886641743123\\
24.5819091796875 3.99324998477242\\
24.6124267578125 3.97764653804833\\
24.6429443359375 3.96205604640634\\
24.6734619140625 3.94647847913016\\
24.7039794921875 3.9309138056389\\
24.7344970703125 3.91536199548631\\
24.7650146484375 3.89982301835988\\
24.7955322265625 3.88429684408009\\
24.8260498046875 3.8687834425996\\
24.8565673828125 3.85328278400241\\
24.8870849609375 3.83779483850309\\
24.9176025390625 3.82231957644594\\
24.9481201171875 3.80685696830429\\
24.9786376953125 3.79140698467965\\
25.0091552734375 3.77596959630091\\
25.0396728515625 3.7605447740237\\
25.0701904296875 3.74513248882947\\
25.1007080078125 3.72973271182483\\
25.1312255859375 3.71434541424076\\
25.1617431640625 3.69897056743187\\
25.1922607421875 3.68360814287567\\
25.2227783203125 3.66825811217179\\
25.2532958984375 3.65292044704133\\
25.2838134765625 3.637595119326\\
25.3143310546875 3.62228210098757\\
25.3448486328125 3.60698136410701\\
25.3753662109375 3.59169288088384\\
25.4058837890625 3.57641662363544\\
25.4364013671875 3.56115256479627\\
25.4669189453125 3.54590067691729\\
25.4974365234375 3.53066093266517\\
25.5279541015625 3.51543330482164\\
25.5584716796875 3.50021776628284\\
25.5889892578125 3.48501429005857\\
25.6195068359375 3.46982284927167\\
25.6500244140625 3.45464341715738\\
25.6805419921875 3.43947596706259\\
25.7110595703125 3.42432047244524\\
25.7415771484375 3.4091769068737\\
25.7720947265625 3.39404524402602\\
25.8026123046875 3.37892545768937\\
25.8331298828125 3.36381752175939\\
25.8636474609375 3.34872141023952\\
25.8941650390625 3.33363709724039\\
25.9246826171875 3.31856455697919\\
25.9552001953125 3.30350376377907\\
25.9857177734375 3.28845469206846\\
26.0162353515625 3.27341731638055\\
26.0467529296875 3.25839161135259\\
26.0772705078125 3.24337755172534\\
26.1077880859375 3.22837511234243\\
26.1383056640625 3.21338426814982\\
26.1688232421875 3.19840499419515\\
26.1993408203125 3.18343726562719\\
26.2298583984375 3.16848105769521\\
26.2603759765625 3.15353634574848\\
26.2908935546875 3.13860310523558\\
26.3214111328125 3.12368131170396\\
26.3519287109375 3.10877094079924\\
26.3824462890625 3.09387196826474\\
26.4129638671875 3.07898436994087\\
26.4434814453125 3.06410812176461\\
26.4739990234375 3.04924319976894\\
26.5045166015625 3.03438958008222\\
26.5350341796875 3.01954723892781\\
26.5655517578125 3.00471615262335\\
26.5960693359375 2.98989629758034\\
26.6265869140625 2.97508765030357\\
26.6571044921875 2.96029018739055\\
26.6876220703125 2.94550388553106\\
26.7181396484375 2.93072872150655\\
26.7486572265625 2.91596467218965\\
26.7791748046875 2.9012117145437\\
26.8096923828125 2.88646982562216\\
26.8402099609375 2.87173898256813\\
26.8707275390625 2.85701916261386\\
26.9012451171875 2.84231034308023\\
26.9317626953125 2.82761250137627\\
26.9622802734375 2.81292561499862\\
26.9927978515625 2.79824966153109\\
27.0233154296875 2.78358461864413\\
27.0538330078125 2.76893046409435\\
27.0843505859375 2.75428717572409\\
27.1148681640625 2.73965473146085\\
27.1453857421875 2.7250331093169\\
27.1759033203125 2.71042228738871\\
27.2064208984375 2.69582224385662\\
27.2369384765625 2.68123295698421\\
27.2674560546875 2.66665440511795\\
27.2979736328125 2.65208656668672\\
27.3284912109375 2.63752942020131\\
27.3590087890625 2.62298294425399\\
27.3895263671875 2.60844711751809\\
27.4200439453125 2.5939219187475\\
27.4505615234375 2.57940732677624\\
27.4810791015625 2.56490332051803\\
27.5115966796875 2.55040987896585\\
27.5421142578125 2.53592698119148\\
27.5726318359375 2.5214546063451\\
27.6031494140625 2.50699273365483\\
27.6336669921875 2.4925413424263\\
27.6641845703125 2.47810041204226\\
27.6947021484375 2.4636699219621\\
27.7252197265625 2.4492498517215\\
27.7557373046875 2.43484018093195\\
27.7862548828125 2.42044088928038\\
27.8167724609375 2.40605195652871\\
27.8472900390625 2.39167336251346\\
27.8778076171875 2.37730508714536\\
27.9083251953125 2.36294711040892\\
27.9388427734375 2.34859941236203\\
27.9693603515625 2.33426197313559\\
27.9998779296875 2.31993477293308\\
28.0303955078125 2.30561779203017\\
28.0609130859375 2.29131101077435\\
28.0914306640625 2.27701440958453\\
28.1219482421875 2.26272796895066\\
28.1524658203125 2.24845166943333\\
28.1829833984375 2.23418549166342\\
28.2135009765625 2.21992941634168\\
28.2440185546875 2.20568342423837\\
28.2745361328125 2.19144749619293\\
28.3050537109375 2.17722161311354\\
28.3355712890625 2.16300575597679\\
28.3660888671875 2.14879990582731\\
28.3966064453125 2.13460404377739\\
28.4271240234375 2.12041815100666\\
28.4576416015625 2.10624220876164\\
28.4881591796875 2.09207619835552\\
28.5186767578125 2.0779201011677\\
28.5491943359375 2.06377389864344\\
28.5797119140625 2.04963757229356\\
28.6102294921875 2.03551110369407\\
28.6407470703125 2.02139447448582\\
28.6712646484375 2.00728766637415\\
28.7017822265625 1.99319066112858\\
28.7322998046875 1.97910344058243\\
28.7628173828125 1.96502598663248\\
28.7933349609375 1.95095828123873\\
28.8238525390625 1.93690030642389\\
28.8543701171875 1.92285204427327\\
28.8848876953125 1.90881347693421\\
28.9154052734375 1.89478458661598\\
28.9459228515625 1.88076535558931\\
28.9764404296875 1.86675576618611\\
29.0069580078125 1.85275580079915\\
29.0374755859375 1.83876544188175\\
29.0679931640625 1.82478467194746\\
29.0985107421875 1.81081347356973\\
29.1290283203125 1.79685182938162\\
29.1595458984375 1.78289972207548\\
29.1900634765625 1.76895713440261\\
29.2205810546875 1.75502404917303\\
29.2510986328125 1.7411004492551\\
29.2816162109375 1.72718631757525\\
29.3121337890625 1.71328163711768\\
29.3426513671875 1.69938639092407\\
29.3731689453125 1.68550056209324\\
29.4036865234375 1.6716241337809\\
29.4342041015625 1.65775708919936\\
29.4647216796875 1.64389941161719\\
29.4952392578125 1.63005108435897\\
29.5257568359375 1.61621209080499\\
29.5562744140625 1.60238241439097\\
29.5867919921875 1.58856203860779\\
29.6173095703125 1.57475094700115\\
29.6478271484375 1.56094912317135\\
29.6783447265625 1.547156550773\\
29.7088623046875 1.53337321351471\\
29.7393798828125 1.51959909515885\\
29.7698974609375 1.50583417952125\\
29.8004150390625 1.49207845047096\\
29.8309326171875 1.47833189192994\\
29.8614501953125 1.46459448787282\\
29.8919677734375 1.45086622232662\\
29.9224853515625 1.43714707937048\\
29.9530029296875 1.42343704313543\\
29.9835205078125 1.40973609780407\\
30.0140380859375 1.39604422761035\\
30.0445556640625 1.38236141683933\\
30.0750732421875 1.36868764982684\\
30.1055908203125 1.35502291095932\\
30.1361083984375 1.34136718467351\\
30.1666259765625 1.3277204554562\\
30.1971435546875 1.31408270784403\\
30.2276611328125 1.30045392642313\\
30.2581787109375 1.286834095829\\
30.2886962890625 1.27322320074619\\
30.3192138671875 1.25962122590806\\
30.3497314453125 1.24602815609655\\
30.3802490234375 1.23244397614193\\
30.4107666015625 1.21886867092256\\
30.4412841796875 1.20530222536468\\
30.4718017578125 1.19174462444209\\
30.5023193359375 1.17819585317603\\
30.5328369140625 1.16465589663483\\
30.5633544921875 1.15112473993379\\
30.5938720703125 1.13760236823482\\
30.6243896484375 1.12408876674633\\
30.6549072265625 1.11058392072291\\
30.6854248046875 1.09708781546517\\
30.7159423828125 1.08360043631946\\
30.7464599609375 1.07012176867769\\
30.7769775390625 1.05665179797707\\
30.8074951171875 1.0431905096999\\
30.8380126953125 1.02973788937334\\
30.8685302734375 1.01629392256921\\
30.8990478515625 1.00285859490378\\
30.9295654296875 0.989431892037521\\
30.9600830078125 0.976013799674875\\
30.9906005859375 0.962604303564104\\
31.0211181640625 0.949203389497025\\
31.0516357421875 0.935811043308812\\
31.0821533203125 0.922427250877778\\
31.1126708984375 0.909051998125187\\
31.1431884765625 0.895685271015031\\
31.1737060546875 0.882327055553818\\
31.2042236328125 0.868977337790369\\
31.2347412109375 0.855636103815622\\
31.2652587890625 0.842303339762416\\
31.2957763671875 0.828979031805291\\
31.3262939453125 0.815663166160318\\
31.3568115234375 0.802355729084829\\
31.3873291015625 0.789056706877309\\
31.4178466796875 0.775766085877104\\
31.4483642578125 0.762483852464316\\
31.4788818359375 0.749209993059513\\
31.5093994140625 0.735944494123638\\
31.5399169921875 0.722687342157721\\
31.5704345703125 0.709438523702759\\
31.6009521484375 0.696198025339466\\
31.6314697265625 0.68296583368816\\
31.6619873046875 0.669741935408458\\
31.6925048828125 0.656526317199223\\
31.7230224609375 0.643318965798282\\
31.7535400390625 0.630119867982279\\
31.7840576171875 0.616929010566478\\
31.8145751953125 0.603746380404589\\
31.8450927734375 0.590571964388589\\
31.8756103515625 0.577405749448535\\
31.9061279296875 0.564247722552365\\
31.9366455078125 0.551097870705766\\
31.9671630859375 0.537956180951949\\
31.9976806640625 0.52482264037149\\
32.0281982421875 0.511697236082176\\
32.0587158203125 0.498579955238799\\
32.0892333984375 0.485470785032988\\
32.1197509765625 0.472369712693059\\
32.1502685546875 0.459276725483808\\
32.1807861328125 0.446191810706369\\
32.2113037109375 0.433114955698027\\
32.2418212890625 0.420046147832083\\
32.2723388671875 0.406985374517611\\
32.3028564453125 0.393932623199385\\
32.3333740234375 0.380887881357653\\
32.3638916015625 0.367851136507984\\
32.3944091796875 0.354822376201119\\
32.4249267578125 0.341801588022786\\
32.4554443359375 0.328788759593547\\
32.4859619140625 0.315783878568652\\
32.5164794921875 0.302786932637848\\
32.5469970703125 0.289797909525259\\
32.5775146484375 0.276816796989191\\
32.6080322265625 0.263843582821988\\
32.6385498046875 0.250878254849891\\
32.6690673828125 0.237920800932853\\
32.6995849609375 0.224971208964427\\
32.7301025390625 0.212029466871525\\
32.7606201171875 0.199095562614403\\
32.7911376953125 0.186169484186393\\
32.8216552734375 0.173251219613768\\
32.8521728515625 0.160340756955663\\
32.8826904296875 0.147438084303823\\
32.9132080078125 0.134543189782554\\
32.9437255859375 0.121656061548487\\
32.9742431640625 0.108776687790506\\
33.0047607421875 0.0959050567295678\\
33.0352783203125 0.0830411566185232\\
33.0657958984375 0.070184975742045\\
33.0963134765625 0.0573365024164123\\
33.1268310546875 0.0444957249894221\\
33.1573486328125 0.0316626318402386\\
33.1878662109375 0.0188372113791975\\
33.2183837890625 0.00601945204774484\\
33.2489013671875 -0.00679065768175723\\
33.2794189453125 -0.0195931293061396\\
33.3099365234375 -0.0323879742915886\\
33.3404541015625 -0.045175204073775\\
33.3709716796875 -0.0579548300579764\\
33.4014892578125 -0.0707268636192309\\
33.4320068359375 -0.0834913161024609\\
33.4625244140625 -0.0962481988226039\\
33.4930419921875 -0.10899752306477\\
33.5235595703125 -0.121739300084358\\
33.5540771484375 -0.134473541107179\\
33.5845947265625 -0.147200257329597\\
33.6151123046875 -0.159919459918659\\
33.6456298828125 -0.172631160012243\\
33.6761474609375 -0.185335368719126\\
33.7066650390625 -0.198032097119229\\
33.7371826171875 -0.210721356263588\\
33.7677001953125 -0.223403157174625\\
33.7982177734375 -0.236077510846192\\
33.8287353515625 -0.248744428243717\\
33.8592529296875 -0.26140392030433\\
33.8897705078125 -0.27405599793699\\
33.9202880859375 -0.286700672022603\\
33.9508056640625 -0.299337953414173\\
33.9813232421875 -0.311967852936865\\
34.0118408203125 -0.324590381388172\\
34.0423583984375 -0.337205549538034\\
34.0728759765625 -0.349813368128934\\
34.1033935546875 -0.362413847876045\\
34.1339111328125 -0.37500699946734\\
34.1644287109375 -0.387592833563683\\
34.1949462890625 -0.400171360799006\\
34.2254638671875 -0.412742591780354\\
34.2559814453125 -0.425306537088048\\
34.2864990234375 -0.437863207275804\\
34.3170166015625 -0.450412612870835\\
34.3475341796875 -0.462954764373936\\
34.3780517578125 -0.475489672259656\\
34.4085693359375 -0.488017346976349\\
34.4390869140625 -0.500537798946373\\
34.4696044921875 -0.513051038566098\\
34.5001220703125 -0.52555707620608\\
34.5306396484375 -0.538055922211155\\
34.5611572265625 -0.550547586900583\\
34.5916748046875 -0.563032080568084\\
34.6221923828125 -0.575509413482013\\
34.6527099609375 -0.58797959588545\\
34.6832275390625 -0.600442637996298\\
34.7137451171875 -0.612898550007392\\
34.7442626953125 -0.625347342086602\\
34.7747802734375 -0.637789024376948\\
34.8052978515625 -0.650223606996747\\
34.8358154296875 -0.662651100039602\\
34.8663330078125 -0.675071513574635\\
34.8968505859375 -0.687484857646498\\
34.9273681640625 -0.699891142275543\\
34.9578857421875 -0.712290377457873\\
34.9884033203125 -0.724682573165475\\
35.0189208984375 -0.737067739346309\\
35.0494384765625 -0.749445885924417\\
35.0799560546875 -0.761817022799994\\
35.1104736328125 -0.774181159849555\\
35.1409912109375 -0.786538306925952\\
35.1715087890625 -0.798888473858542\\
35.2020263671875 -0.811231670453234\\
35.2325439453125 -0.823567906492624\\
35.2630615234375 -0.835897191736047\\
35.2935791015625 -0.848219535919753\\
35.3240966796875 -0.860534948756912\\
35.3546142578125 -0.87284343993777\\
35.3851318359375 -0.885145019129741\\
35.4156494140625 -0.897439695977441\\
35.4461669921875 -0.909727480102856\\
35.4766845703125 -0.922008381105414\\
35.5072021484375 -0.934282408562084\\
35.5377197265625 -0.94654957202741\\
35.5682373046875 -0.958809881033673\\
35.5987548828125 -0.97106334509099\\
35.6292724609375 -0.98330997368731\\
35.6597900390625 -0.995549776288637\\
35.6903076171875 -1.00778276233901\\
35.7208251953125 -1.02000894126065\\
35.7513427734375 -1.03222832245403\\
35.7818603515625 -1.04444091529797\\
35.8123779296875 -1.05664672914973\\
35.8428955078125 -1.06884577334509\\
35.8734130859375 -1.08103805719844\\
35.9039306640625 -1.09322359000285\\
35.9344482421875 -1.10540238103021\\
35.9649658203125 -1.11757443953124\\
35.9954833984375 -1.12973977473565\\
36.0260009765625 -1.14189839585215\\
36.0565185546875 -1.15405031206863\\
36.0870361328125 -1.16619553255214\\
36.1175537109375 -1.17833406644906\\
36.1480712890625 -1.19046592288509\\
36.1785888671875 -1.20259111096547\\
36.2091064453125 -1.21470963977492\\
36.2396240234375 -1.22682151837783\\
36.2701416015625 -1.23892675581825\\
36.3006591796875 -1.25102536112006\\
36.3311767578125 -1.26311734328699\\
36.3616943359375 -1.27520271130271\\
36.3922119140625 -1.28728147413095\\
36.4227294921875 -1.29935364071551\\
36.4532470703125 -1.31141921998042\\
36.4837646484375 -1.32347822082993\\
36.5142822265625 -1.33553065214869\\
36.5447998046875 -1.34757652280174\\
36.5753173828125 -1.35961584163461\\
36.6058349609375 -1.37164861747345\\
36.6363525390625 -1.38367485912502\\
36.6668701171875 -1.39569457537687\\
36.6973876953125 -1.40770777499728\\
36.7279052734375 -1.41971446673551\\
36.7584228515625 -1.43171465932167\\
36.7889404296875 -1.44370836146701\\
36.8194580078125 -1.45569558186381\\
36.8499755859375 -1.46767632918558\\
36.8804931640625 -1.47965061208706\\
36.9110107421875 -1.49161843920433\\
36.9415283203125 -1.50357981915489\\
36.9720458984375 -1.51553476053768\\
37.0025634765625 -1.52748327193321\\
37.0330810546875 -1.53942536190361\\
37.0635986328125 -1.55136103899271\\
37.0941162109375 -1.56329031172606\\
37.1246337890625 -1.57521318861113\\
37.1551513671875 -1.58712967813719\\
37.1856689453125 -1.59903978877555\\
37.2161865234375 -1.61094352897955\\
37.2467041015625 -1.62284090718466\\
37.2772216796875 -1.63473193180849\\
37.3077392578125 -1.64661661125093\\
37.3382568359375 -1.65849495389419\\
37.3687744140625 -1.67036696810285\\
37.3992919921875 -1.68223266222399\\
37.4298095703125 -1.69409204458715\\
37.4603271484375 -1.7059451235045\\
37.4908447265625 -1.71779190727088\\
37.5213623046875 -1.7296324041638\\
37.5518798828125 -1.74146662244361\\
37.5823974609375 -1.7532945703535\\
37.6129150390625 -1.76511625611953\\
37.6434326171875 -1.77693168795084\\
37.6739501953125 -1.78874087403955\\
37.7044677734375 -1.80054382256091\\
37.7349853515625 -1.81234054167338\\
37.7655029296875 -1.82413103951859\\
37.7960205078125 -1.83591532422155\\
37.8265380859375 -1.8476934038906\\
37.8570556640625 -1.85946528661751\\
37.8875732421875 -1.87123098047756\\
37.9180908203125 -1.88299049352959\\
37.9486083984375 -1.89474383381604\\
37.9791259765625 -1.90649100936304\\
38.0096435546875 -1.91823202818045\\
38.0401611328125 -1.92996689826195\\
38.0706787109375 -1.94169562758507\\
38.1011962890625 -1.95341822411127\\
38.1317138671875 -1.96513469578598\\
38.1622314453125 -1.9768450505387\\
38.1927490234375 -1.98854929628299\\
38.2232666015625 -2.00024744091661\\
38.2537841796875 -2.01193949232153\\
38.2843017578125 -2.02362545836399\\
38.3148193359375 -2.03530534689457\\
38.3453369140625 -2.04697916574827\\
38.3758544921875 -2.0586469227445\\
38.4063720703125 -2.07030862568722\\
38.4368896484375 -2.08196428236493\\
38.4674072265625 -2.09361390055078\\
38.4979248046875 -2.10525748800257\\
38.5284423828125 -2.11689505246288\\
38.5589599609375 -2.12852660165906\\
38.5894775390625 -2.1401521433033\\
38.6199951171875 -2.15177168509271\\
38.6505126953125 -2.16338523470935\\
38.6810302734375 -2.17499279982033\\
38.7115478515625 -2.18659438807779\\
38.7420654296875 -2.19819000711903\\
38.7725830078125 -2.20977966456649\\
38.8031005859375 -2.22136336802787\\
38.8336181640625 -2.23294112509615\\
38.8641357421875 -2.24451294334968\\
38.8946533203125 -2.25607883035215\\
38.9251708984375 -2.26763879365274\\
38.9556884765625 -2.27919284078611\\
38.9862060546875 -2.29074097927248\\
39.0167236328125 -2.30228321661767\\
39.0472412109375 -2.31381956031317\\
39.0777587890625 -2.32535001783616\\
39.1082763671875 -2.33687459664957\\
39.1387939453125 -2.34839330420219\\
39.1693115234375 -2.35990614792863\\
39.1998291015625 -2.37141313524942\\
39.2303466796875 -2.38291427357106\\
39.2608642578125 -2.39440957028606\\
39.2913818359375 -2.40589903277299\\
39.3218994140625 -2.41738266839654\\
39.3524169921875 -2.42886048450757\\
39.3829345703125 -2.44033248844314\\
39.4134521484375 -2.45179868752656\\
39.4439697265625 -2.46325908906748\\
39.4744873046875 -2.4747137003619\\
39.5050048828125 -2.4861625286922\\
39.5355224609375 -2.49760558132724\\
39.5660400390625 -2.5090428655224\\
39.5965576171875 -2.52047438851959\\
39.6270751953125 -2.53190015754728\\
39.6575927734375 -2.54332017982068\\
39.6881103515625 -2.55473446254162\\
39.7186279296875 -2.56614301289868\\
39.7491455078125 -2.57754583806723\\
39.7796630859375 -2.5889429452095\\
39.8101806640625 -2.60033434147458\\
39.8406982421875 -2.61172003399845\\
39.8712158203125 -2.62310002990412\\
39.9017333984375 -2.6344743363016\\
39.9322509765625 -2.64584296028793\\
39.9627685546875 -2.6572059089473\\
39.9932861328125 -2.668563189351\\
40.0238037109375 -2.67991480855759\\
40.0543212890625 -2.69126077361281\\
40.0848388671875 -2.70260109154969\\
40.1153564453125 -2.71393576938861\\
40.1458740234375 -2.72526481413734\\
40.1763916015625 -2.73658823279102\\
40.2069091796875 -2.7479060323323\\
40.2374267578125 -2.75921821973128\\
40.2679443359375 -2.77052480194566\\
40.2984619140625 -2.7818257859207\\
40.3289794921875 -2.79312117858928\\
40.3594970703125 -2.804410986872\\
40.3900146484375 -2.81569521767714\\
40.4205322265625 -2.82697387790077\\
40.4510498046875 -2.83824697442671\\
40.4815673828125 -2.84951451412669\\
40.5120849609375 -2.86077650386027\\
40.5426025390625 -2.87203295047496\\
40.5731201171875 -2.88328386080624\\
40.6036376953125 -2.89452924167759\\
40.6341552734375 -2.90576909990055\\
40.6646728515625 -2.91700344227474\\
40.6951904296875 -2.92823227558792\\
40.7257080078125 -2.93945560661599\\
40.7562255859375 -2.95067344212309\\
40.7867431640625 -2.96188578886164\\
40.8172607421875 -2.97309265357228\\
40.8477783203125 -2.98429404298403\\
40.8782958984375 -2.99548996381425\\
40.9088134765625 -3.00668042276874\\
40.9393310546875 -3.01786542654172\\
40.9698486328125 -3.0290449818159\\
41.0003662109375 -3.04021909526254\\
41.0308837890625 -3.05138777354143\\
41.0614013671875 -3.06255102330098\\
41.0919189453125 -3.07370885117824\\
41.1224365234375 -3.08486126379893\\
41.1529541015625 -3.0960082677775\\
41.1834716796875 -3.10714986971713\\
41.2139892578125 -3.1182860762098\\
41.2445068359375 -3.12941689383633\\
41.2750244140625 -3.14054232916641\\
41.3055419921875 -3.1516623887586\\
41.3360595703125 -3.16277707916043\\
41.3665771484375 -3.17388640690839\\
41.3970947265625 -3.18499037852799\\
41.4276123046875 -3.19608900053378\\
41.4581298828125 -3.20718227942941\\
41.4886474609375 -3.21827022170762\\
41.5191650390625 -3.22935283385035\\
41.5496826171875 -3.2404301223287\\
41.5802001953125 -3.25150209360301\\
41.6107177734375 -3.26256875412288\\
41.6412353515625 -3.27363011032721\\
41.6717529296875 -3.28468616864424\\
41.7022705078125 -3.29573693549155\\
41.7327880859375 -3.30678241727618\\
41.7633056640625 -3.31782262039454\\
41.7938232421875 -3.32885755123256\\
41.8243408203125 -3.33988721616567\\
41.8548583984375 -3.35091162155881\\
41.8853759765625 -3.36193077376654\\
41.9158935546875 -3.372944679133\\
41.9464111328125 -3.38395334399197\\
41.9769287109375 -3.39495677466693\\
42.0074462890625 -3.40595497747107\\
42.0379638671875 -3.41694795870728\\
42.0684814453125 -3.42793572466827\\
42.0989990234375 -3.43891828163651\\
42.1295166015625 -3.44989563588441\\
42.1600341796875 -3.46086779367415\\
42.1905517578125 -3.47183476125785\\
42.2210693359375 -3.4827965448776\\
42.2515869140625 -3.49375315076544\\
42.2821044921875 -3.5047045851434\\
42.3126220703125 -3.51565085422358\\
42.3431396484375 -3.52659196420811\\
42.3736572265625 -3.53752792128926\\
42.4041748046875 -3.5484587316494\\
42.4346923828125 -3.55938440146107\\
42.4652099609375 -3.57030493688701\\
42.4957275390625 -3.5812203440802\\
42.5262451171875 -3.59213062918385\\
42.5567626953125 -3.60303579833147\\
42.5872802734375 -3.61393585764689\\
42.6177978515625 -3.6248308132443\\
42.6483154296875 -3.63572067122825\\
42.6788330078125 -3.64660543769371\\
42.7093505859375 -3.65748511872609\\
42.7398681640625 -3.66835972040128\\
42.7703857421875 -3.67922924878564\\
42.8009033203125 -3.6900937099361\\
42.8314208984375 -3.70095310990014\\
42.8619384765625 -3.71180745471583\\
42.8924560546875 -3.72265675041184\\
42.9229736328125 -3.73350100300751\\
42.9534912109375 -3.74434021851286\\
42.9840087890625 -3.75517440292863\\
43.0145263671875 -3.76600356224624\\
43.0450439453125 -3.77682770244795\\
43.0755615234375 -3.78764682950677\\
43.1060791015625 -3.79846094938656\\
43.1365966796875 -3.809270068042\\
43.1671142578125 -3.82007419141869\\
43.1976318359375 -3.8308733254531\\
43.2281494140625 -3.84166747607269\\
43.2586669921875 -3.85245664919582\\
43.2891845703125 -3.86324085073189\\
43.3197021484375 -3.87402008658132\\
43.3502197265625 -3.88479436263555\\
43.3807373046875 -3.89556368477713\\
43.4112548828125 -3.90632805887972\\
43.4417724609375 -3.91708749080805\\
43.4722900390625 -3.92784198641808\\
43.5028076171875 -3.93859155155694\\
43.5333251953125 -3.94933619206296\\
43.5638427734375 -3.96007591376571\\
43.5943603515625 -3.97081072248602\\
43.6248779296875 -3.98154062403607\\
43.6553955078125 -3.9922656242193\\
43.6859130859375 -4.00298572883051\\
43.7164306640625 -4.01370094365589\\
43.7469482421875 -4.02441127447305\\
43.7774658203125 -4.03511672705097\\
43.8079833984375 -4.04581730715013\\
43.8385009765625 -4.05651302052249\\
43.8690185546875 -4.06720387291147\\
43.8995361328125 -4.07788987005208\\
43.9300537109375 -4.08857101767086\\
43.9605712890625 -4.09924732148594\\
43.9910888671875 -4.10991878720702\\
44.0216064453125 -4.1205854205355\\
44.0521240234375 -4.13124722716437\\
44.0826416015625 -4.14190421277838\\
44.1131591796875 -4.15255638305393\\
44.1436767578125 -4.16320374365917\\
44.1741943359375 -4.17384630025402\\
44.2047119140625 -4.18448405849017\\
44.2352294921875 -4.19511702401114\\
44.2657470703125 -4.20574520245227\\
44.2962646484375 -4.21636859944073\\
44.3267822265625 -4.22698722059564\\
44.3572998046875 -4.23760107152799\\
44.3878173828125 -4.24821015784067\\
44.4183349609375 -4.25881448512857\\
44.4488525390625 -4.26941405897857\\
44.4793701171875 -4.28000888496949\\
44.5098876953125 -4.29059896867224\\
44.5404052734375 -4.30118431564976\\
44.5709228515625 -4.31176493145707\\
44.6014404296875 -4.32234082164127\\
44.6319580078125 -4.3329119917416\\
44.6624755859375 -4.34347844728943\\
44.6929931640625 -4.35404019380835\\
44.7235107421875 -4.36459723681406\\
44.7540283203125 -4.37514958181453\\
44.7845458984375 -4.38569723430996\\
44.8150634765625 -4.39624019979283\\
44.8455810546875 -4.40677848374785\\
44.8760986328125 -4.41731209165208\\
44.9066162109375 -4.42784102897488\\
44.9371337890625 -4.43836530117804\\
44.9676513671875 -4.44888491371561\\
44.9981689453125 -4.45939987203413\\
45.0286865234375 -4.46991018157252\\
45.0592041015625 -4.48041584776212\\
45.0897216796875 -4.4909168760268\\
45.1202392578125 -4.50141327178286\\
45.1507568359375 -4.51190504043913\\
45.1812744140625 -4.52239218739696\\
45.2117919921875 -4.53287471805029\\
45.2423095703125 -4.54335263778557\\
45.2728271484375 -4.55382595198191\\
45.3033447265625 -4.56429466601099\\
45.3338623046875 -4.57475878523715\\
45.3643798828125 -4.58521831501739\\
45.3948974609375 -4.59567326070141\\
45.4254150390625 -4.60612362763157\\
45.4559326171875 -4.61656942114299\\
45.4864501953125 -4.62701064656352\\
45.5169677734375 -4.63744730921379\\
45.5474853515625 -4.64787941440718\\
45.5780029296875 -4.65830696744995\\
45.6085205078125 -4.66872997364113\\
45.6390380859375 -4.67914843827261\\
45.6695556640625 -4.68956236662917\\
45.7000732421875 -4.69997176398847\\
45.7305908203125 -4.71037663562108\\
45.7611083984375 -4.7207769867905\\
45.7916259765625 -4.73117282275319\\
45.8526611328125 -4.75195097004912\\
45.9136962890625 -4.77271111942035\\
45.9747314453125 -4.79345331266694\\
46.0357666015625 -4.81417759147808\\
46.0968017578125 -4.83488399743281\\
46.1578369140625 -4.8555725720006\\
46.2188720703125 -4.876243356542\\
46.2799072265625 -4.89689639230923\\
46.3409423828125 -4.9175317204468\\
46.4019775390625 -4.93814938199223\\
46.4630126953125 -4.95874941787645\\
46.5240478515625 -4.97933186892457\\
46.5850830078125 -4.99989677585644\\
46.6461181640625 -5.02044417928722\\
46.7071533203125 -5.04097411972798\\
46.7681884765625 -5.0614866375863\\
46.8292236328125 -5.08198177316682\\
46.8902587890625 -5.10245956667188\\
46.9512939453125 -5.12292005820202\\
47.0123291015625 -5.14336328775657\\
47.0733642578125 -5.1637892952343\\
47.1343994140625 -5.18419812043387\\
47.1954345703125 -5.20458980305439\\
47.2564697265625 -5.22496438269611\\
47.3175048828125 -5.2453218988608\\
47.3785400390625 -5.26566239095243\\
47.4395751953125 -5.28598589827761\\
47.5006103515625 -5.30629246004622\\
47.5616455078125 -5.32658211537188\\
47.6226806640625 -5.34685490327252\\
47.6837158203125 -5.36711086267088\\
47.7447509765625 -5.38735003239508\\
47.8057861328125 -5.40757245117908\\
47.8668212890625 -5.42777815766324\\
47.9278564453125 -5.44796719039485\\
47.9888916015625 -5.46813958782861\\
48.0499267578125 -5.48829538832713\\
48.1109619140625 -5.50843463016146\\
48.1719970703125 -5.52855735151158\\
48.2330322265625 -5.54866359046691\\
48.2940673828125 -5.56875338502679\\
48.3551025390625 -5.58882677310098\\
48.4161376953125 -5.60888379251015\\
48.4771728515625 -5.62892448098636\\
48.5382080078125 -5.64894887617355\\
48.5992431640625 -5.66895701562802\\
48.6602783203125 -5.68894893681887\\
48.7213134765625 -5.70892467712853\\
48.7823486328125 -5.72888427385319\\
48.8433837890625 -5.7488277642033\\
48.9044189453125 -5.76875518530403\\
48.9654541015625 -5.78866657419565\\
49.0264892578125 -5.80856196783413\\
49.0875244140625 -5.82844140309149\\
49.1485595703125 -5.84830491675628\\
49.2095947265625 -5.86815254553403\\
49.2706298828125 -5.88798432604773\\
49.3316650390625 -5.90780029483824\\
49.3927001953125 -5.9276004883647\\
49.4537353515625 -5.94738494300506\\
49.5147705078125 -5.96715369505644\\
49.5758056640625 -5.98690678073556\\
49.6368408203125 -6.00664423617928\\
49.6978759765625 -6.02636609744487\\
49.7589111328125 -6.04607240051053\\
49.8199462890625 -6.06576318127583\\
49.8809814453125 -6.08543847556207\\
49.9420166015625 -6.10509831911276\\
50.0030517578125 -6.12474274759394\\
50.0640869140625 -6.14437179659473\\
50.1251220703125 -6.16398550162761\\
50.1861572265625 -6.18358389812893\\
50.2471923828125 -6.20316702145926\\
50.3082275390625 -6.2227349069038\\
50.3692626953125 -6.2422875896728\\
50.4302978515625 -6.26182510490194\\
50.4913330078125 -6.28134748765275\\
50.5523681640625 -6.30085477291296\\
50.6134033203125 -6.32034699559697\\
50.6744384765625 -6.33982419054616\\
50.7354736328125 -6.35928639252934\\
50.7965087890625 -6.37873363624307\\
50.8575439453125 -6.39816595631212\\
50.9185791015625 -6.41758338728978\\
50.9796142578125 -6.43698596365831\\
51.0406494140625 -6.45637371982928\\
51.1016845703125 -6.4757466901439\\
51.1627197265625 -6.49510490887349\\
51.2237548828125 -6.51444841021979\\
51.2847900390625 -6.53377722831534\\
51.3458251953125 -6.55309139722382\\
51.4068603515625 -6.57239095094049\\
51.4678955078125 -6.59167592339248\\
51.5289306640625 -6.61094634843918\\
51.5899658203125 -6.63020225987261\\
51.6510009765625 -6.64944369141777\\
51.7120361328125 -6.66867067673295\\
51.7730712890625 -6.68788324941018\\
51.8341064453125 -6.70708144297548\\
51.8951416015625 -6.72626529088932\\
51.9561767578125 -6.74543482654683\\
52.0172119140625 -6.76459008327828\\
52.0782470703125 -6.78373109434933\\
52.1392822265625 -6.80285789296144\\
52.2003173828125 -6.82197051225215\\
52.2613525390625 -6.84106898529548\\
52.3223876953125 -6.86015334510223\\
52.3834228515625 -6.8792236246203\\
52.4444580078125 -6.89827985673508\\
52.5054931640625 -6.91732207426974\\
52.5665283203125 -6.93635030998557\\
52.6275634765625 -6.95536459658231\\
52.6885986328125 -6.97436496669849\\
52.7496337890625 -6.99335145291172\\
52.8106689453125 -7.01232408773909\\
52.8717041015625 -7.03128290363737\\
52.9327392578125 -7.05022793300346\\
52.9937744140625 -7.06915920817463\\
53.0548095703125 -7.08807676142888\\
53.1158447265625 -7.10698062498523\\
53.1768798828125 -7.12587083100402\\
53.2379150390625 -7.14474741158728\\
53.2989501953125 -7.16361039877899\\
53.3599853515625 -7.18245982456543\\
53.4210205078125 -7.20129572087546\\
53.4820556640625 -7.22011811958081\\
53.5430908203125 -7.23892705249644\\
53.6041259765625 -7.25772255138084\\
53.6651611328125 -7.27650464793626\\
53.7261962890625 -7.29527337380912\\
53.7872314453125 -7.31402876059019\\
53.8482666015625 -7.33277083981503\\
53.9093017578125 -7.35149964296415\\
53.9703369140625 -7.37021520146338\\
54.0313720703125 -7.3889175466842\\
54.0924072265625 -7.40760670994391\\
54.1534423828125 -7.42628272250609\\
54.2144775390625 -7.44494561558073\\
54.2755126953125 -7.46359542032462\\
54.3365478515625 -7.48223216784163\\
54.3975830078125 -7.50085588918293\\
54.4586181640625 -7.51946661534738\\
54.5196533203125 -7.53806437728174\\
54.5806884765625 -7.55664920588095\\
54.6417236328125 -7.57522113198846\\
54.7027587890625 -7.5937801863965\\
54.7637939453125 -7.61232639984631\\
54.8248291015625 -7.6308598030285\\
54.8858642578125 -7.64938042658326\\
54.9468994140625 -7.66788830110068\\
55.0079345703125 -7.68638345712098\\
55.0689697265625 -7.70486592513484\\
55.1300048828125 -7.72333573558362\\
55.1910400390625 -7.74179291885971\\
55.2520751953125 -7.76023750530667\\
55.3131103515625 -7.77866952521964\\
55.3741455078125 -7.79708900884553\\
55.4351806640625 -7.8154959863833\\
55.4962158203125 -7.83389048798423\\
55.5572509765625 -7.85227254375221\\
55.6182861328125 -7.87064218374395\\
55.6793212890625 -7.88899943796931\\
55.7403564453125 -7.9073443363915\\
55.8013916015625 -7.92567690892738\\
55.8624267578125 -7.94399718544772\\
55.9234619140625 -7.96230519577745\\
55.9844970703125 -7.98060096969589\\
56.0455322265625 -7.99888453693708\\
56.1065673828125 -8.01715592718995\\
56.1676025390625 -8.03541517009867\\
56.2286376953125 -8.05366229526279\\
56.2896728515625 -8.07189733223759\\
56.3507080078125 -8.09012031053431\\
56.4117431640625 -8.10833125962039\\
56.4727783203125 -8.12653020891969\\
56.5338134765625 -8.14471718781281\\
56.5948486328125 -8.16289222563726\\
56.6558837890625 -8.18105535168778\\
56.7169189453125 -8.19920659521656\\
56.7779541015625 -8.21734598543346\\
56.8389892578125 -8.2354735515063\\
56.9000244140625 -8.25358932256105\\
56.9610595703125 -8.27169332768215\\
57.0220947265625 -8.28978559591267\\
57.0831298828125 -8.30786615625461\\
57.1441650390625 -8.32593503766912\\
57.2052001953125 -8.34399226907674\\
57.2662353515625 -8.36203787935765\\
57.3272705078125 -8.38007189735189\\
57.3883056640625 -8.39809435185961\\
57.4493408203125 -8.41610527164131\\
57.5103759765625 -8.43410468541807\\
57.5714111328125 -8.45209262187178\\
57.6324462890625 -8.47006910964541\\
57.6934814453125 -8.48803417734316\\
57.7545166015625 -8.50598785353083\\
57.8155517578125 -8.52393016673589\\
57.8765869140625 -8.54186114544784\\
57.9376220703125 -8.55978081811839\\
57.9986572265625 -8.57768921316167\\
58.0596923828125 -8.59558635895452\\
58.1207275390625 -8.61347228383666\\
58.1817626953125 -8.63134701611094\\
58.2427978515625 -8.64921058404356\\
58.3038330078125 -8.66706301586433\\
58.3648681640625 -8.68490433976685\\
58.4259033203125 -8.70273458390877\\
58.4869384765625 -8.72055377641197\\
58.5479736328125 -8.73836194536284\\
58.6090087890625 -8.75615911881247\\
58.6700439453125 -8.77394532477689\\
58.7310791015625 -8.79172059123724\\
58.7921142578125 -8.80948494614009\\
58.8531494140625 -8.82723841739755\\
58.9141845703125 -8.84498103288758\\
58.9752197265625 -8.86271282045414\\
59.0362548828125 -8.88043380790746\\
59.0972900390625 -8.89814402302422\\
59.1583251953125 -8.91584349354782\\
59.2193603515625 -8.93353224718851\\
59.2803955078125 -8.9512103116237\\
59.3414306640625 -8.96887771449811\\
59.4024658203125 -8.98653448342401\\
59.4635009765625 -9.00418064598144\\
59.5245361328125 -9.0218162297184\\
59.5855712890625 -9.0394412621511\\
59.6466064453125 -9.05705577076414\\
59.7076416015625 -9.07465978301071\\
59.7686767578125 -9.09225332631288\\
59.8297119140625 -9.10983642806168\\
59.8907470703125 -9.12740911561743\\
59.9517822265625 -9.14497141630992\\
60.0128173828125 -9.16252335743856\\
60.0738525390625 -9.18006496627265\\
60.1348876953125 -9.19759627005158\\
60.1959228515625 -9.215117295985\\
60.2569580078125 -9.23262807125308\\
60.3179931640625 -9.2501286230067\\
60.3790283203125 -9.2676189783676\\
60.4400634765625 -9.28509916442867\\
60.5010986328125 -9.30256920825409\\
60.5621337890625 -9.32002913687958\\
60.6231689453125 -9.33747897731257\\
60.6842041015625 -9.35491875653244\\
60.7452392578125 -9.37234850149068\\
60.8062744140625 -9.38976823911109\\
60.8673095703125 -9.40717799629005\\
60.9283447265625 -9.42457779989665\\
60.9893798828125 -9.44196767677292\\
61.0504150390625 -9.45934765373401\\
61.1114501953125 -9.47671775756844\\
61.1724853515625 -9.49407801503823\\
61.2335205078125 -9.51142845287916\\
61.2945556640625 -9.52876909780093\\
61.3555908203125 -9.54609997648737\\
61.4166259765625 -9.56342111559663\\
61.4776611328125 -9.5807325417614\\
61.5386962890625 -9.5980342815891\\
61.5997314453125 -9.61532636166202\\
61.6607666015625 -9.63260880853763\\
61.7218017578125 -9.64988164874867\\
61.7828369140625 -9.66714490880338\\
61.8438720703125 -9.6843986151857\\
61.9049072265625 -9.70164279435547\\
61.9659423828125 -9.71887747274863\\
62.0269775390625 -9.73610267677737\\
62.0880126953125 -9.75331843283037\\
62.1490478515625 -9.77052476727295\\
62.2100830078125 -9.78772170644732\\
62.2711181640625 -9.80490927667272\\
62.3321533203125 -9.82208750424562\\
62.3931884765625 -9.83925641543993\\
62.4542236328125 -9.85641603650716\\
62.5152587890625 -9.87356639367668\\
62.5762939453125 -9.89070751315579\\
62.6373291015625 -9.90783942113002\\
62.6983642578125 -9.92496214376327\\
62.7593994140625 -9.942075707198\\
62.8204345703125 -9.95918013755544\\
62.8814697265625 -9.97627546093572\\
62.9425048828125 -9.99336170341813\\
63.0035400390625 -10.0104388910613\\
63.0645751953125 -10.0275070499033\\
63.1256103515625 -10.0445662059618\\
63.1866455078125 -10.0616163852347\\
63.2476806640625 -10.0786576136996\\
63.3087158203125 -10.0956899173143\\
63.3697509765625 -10.1127133220175\\
63.4307861328125 -10.129727853728\\
63.4918212890625 -10.1467335383456\\
63.5528564453125 -10.1637304017511\\
63.6138916015625 -10.1807184698065\\
63.6749267578125 -10.197697768355\\
63.7359619140625 -10.2146683232214\\
63.7969970703125 -10.2316301602124\\
63.8580322265625 -10.2485833051163\\
63.9190673828125 -10.2655277837035\\
63.9801025390625 -10.2824636217269\\
64.0411376953125 -10.2993908449216\\
64.1021728515625 -10.3163094790055\\
64.1632080078125 -10.3332195496792\\
64.2242431640625 -10.3501210826261\\
64.2852783203125 -10.3670141035132\\
64.3463134765625 -10.3838986379904\\
64.4073486328125 -10.4007747116913\\
64.4683837890625 -10.4176423502333\\
64.5294189453125 -10.4345015792175\\
64.5904541015625 -10.4513524242291\\
64.6514892578125 -10.4681949108376\\
64.7125244140625 -10.4850290645968\\
64.7735595703125 -10.5018549110451\\
64.8345947265625 -10.5186724757057\\
64.8956298828125 -10.5354817840868\\
64.9566650390625 -10.5522828616817\\
65.0177001953125 -10.5690757339688\\
65.0787353515625 -10.5858604264121\\
65.1397705078125 -10.6026369644614\\
65.2008056640625 -10.6194053735519\\
65.2618408203125 -10.6361656791053\\
65.3228759765625 -10.6529179065292\\
65.3839111328125 -10.6696620812173\\
65.4449462890625 -10.6863982285504\\
65.5059814453125 -10.7031263738954\\
65.5670166015625 -10.7198465426064\\
65.6280517578125 -10.7365587600244\\
65.6890869140625 -10.7532630514778\\
65.7501220703125 -10.7699594422821\\
65.8111572265625 -10.7866479577405\\
65.8721923828125 -10.8033286231439\\
65.9332275390625 -10.8200014637712\\
65.9942626953125 -10.8366665048891\\
66.0552978515625 -10.8533237717528\\
66.1163330078125 -10.8699732896058\\
66.1773681640625 -10.8866150836802\\
66.2384033203125 -10.9032491791968\\
66.2994384765625 -10.9198756013653\\
66.3604736328125 -10.9364943753847\\
66.4215087890625 -10.953105526443\\
66.4825439453125 -10.9697090797177\\
66.5435791015625 -10.9863050603759\\
66.6046142578125 -11.0028934935747\\
66.6656494140625 -11.0194744044606\\
66.7266845703125 -11.0360478181708\\
66.7877197265625 -11.0526137598322\\
66.8487548828125 -11.0691722545627\\
66.9097900390625 -11.0857233274704\\
66.9708251953125 -11.1022670036543\\
67.0318603515625 -11.1188033082043\\
67.0928955078125 -11.1353322662015\\
67.1539306640625 -11.1518539027181\\
67.2149658203125 -11.168368242818\\
67.2760009765625 -11.1848753115563\\
67.3370361328125 -11.2013751339804\\
67.3980712890625 -11.2178677351291\\
67.4591064453125 -11.2343531400336\\
67.5201416015625 -11.2508313737173\\
67.5811767578125 -11.2673024611961\\
67.6422119140625 -11.2837664274782\\
67.7032470703125 -11.3002232975649\\
67.7642822265625 -11.3166730964503\\
67.8253173828125 -11.3331158491215\\
67.8863525390625 -11.3495515805589\\
67.9473876953125 -11.3659803157362\\
68.0084228515625 -11.3824020796209\\
68.0694580078125 -11.398816897174\\
68.1304931640625 -11.4152247933505\\
68.1915283203125 -11.4316257930995\\
68.2525634765625 -11.4480199213642\\
68.3135986328125 -11.4644072030821\\
68.3746337890625 -11.4807876631855\\
68.4356689453125 -11.4971613266013\\
68.4967041015625 -11.513528218251\\
68.5577392578125 -11.5298883630515\\
68.6187744140625 -11.5462417859145\\
68.6798095703125 -11.5625885117474\\
68.7408447265625 -11.5789285654529\\
68.8018798828125 -11.5952619719292\\
68.8629150390625 -11.6115887560707\\
68.9239501953125 -11.6279089427674\\
68.9849853515625 -11.6442225569056\\
69.0460205078125 -11.6605296233678\\
69.1070556640625 -11.676830167033\\
69.1680908203125 -11.6931242127769\\
69.2291259765625 -11.7094117854718\\
69.2901611328125 -11.7256929099868\\
69.3511962890625 -11.7419676111883\\
69.4122314453125 -11.7582359139399\\
69.4732666015625 -11.7744978431025\\
69.5343017578125 -11.7907534235344\\
69.5953369140625 -11.8070026800919\\
69.6563720703125 -11.8232456376288\\
69.7174072265625 -11.8394823209972\\
69.7784423828125 -11.8557127550471\\
69.8394775390625 -11.8719369646269\\
69.9005126953125 -11.8881549745836\\
69.9615478515625 -11.9043668097625\\
70.0225830078125 -11.920572495008\\
70.0836181640625 -11.936772055163\\
70.1446533203125 -11.9529655150699\\
70.2056884765625 -11.9691528995701\\
70.2667236328125 -11.9853342335043\\
70.3277587890625 -12.0015095417129\\
70.3887939453125 -12.0176788490357\\
70.4498291015625 -12.0338421803127\\
70.5108642578125 -12.0499995603834\\
70.5718994140625 -12.0661510140879\\
70.6329345703125 -12.0822965662662\\
70.6939697265625 -12.0984362417588\\
70.7550048828125 -12.1145700654069\\
70.8160400390625 -12.1306980620523\\
70.8770751953125 -12.1468202565377\\
70.9381103515625 -12.1629366737067\\
70.9991455078125 -12.1790473384041\\
71.0601806640625 -12.1951522754761\\
71.1212158203125 -12.2112515097702\\
71.1822509765625 -12.2273450661356\\
71.2432861328125 -12.2434329694232\\
71.3043212890625 -12.2595152444857\\
71.3653564453125 -12.2755919161779\\
71.4263916015625 -12.2916630093568\\
71.4874267578125 -12.3077285488816\\
71.5484619140625 -12.3237885596141\\
71.6094970703125 -12.3398430664185\\
71.6705322265625 -12.3558920941619\\
71.7315673828125 -12.3719356677143\\
71.7926025390625 -12.3879738119485\\
71.8536376953125 -12.4040065517407\\
71.9146728515625 -12.4200339119703\\
71.9757080078125 -12.4360559175201\\
72.0367431640625 -12.4520725932766\\
72.0977783203125 -12.46808396413\\
72.1588134765625 -12.4840900549742\\
72.2198486328125 -12.5000908907073\\
72.2808837890625 -12.5160864962313\\
72.3419189453125 -12.5320768964528\\
72.4029541015625 -12.5480621162824\\
72.4639892578125 -12.5640421806357\\
72.5250244140625 -12.5800171144327\\
72.5860595703125 -12.5959869425981\\
72.6470947265625 -12.6119516900618\\
72.7081298828125 -12.6279113817588\\
72.7691650390625 -12.6438660426291\\
72.8302001953125 -12.6598156976181\\
72.8912353515625 -12.6757603716769\\
72.9522705078125 -12.6917000897619\\
73.0133056640625 -12.7076348768355\\
73.0743408203125 -12.7235647578658\\
73.1353759765625 -12.7394897578271\\
73.1964111328125 -12.7554099016996\\
73.2574462890625 -12.7713252144698\\
73.3184814453125 -12.7872357211309\\
73.3795166015625 -12.8031414466821\\
73.4405517578125 -12.8190424161298\\
73.5015869140625 -12.8349386544868\\
73.5626220703125 -12.8508301867728\\
73.6236572265625 -12.8667170380148\\
73.6846923828125 -12.8825992332466\\
73.7457275390625 -12.8984767975096\\
73.8067626953125 -12.9143497558524\\
73.8677978515625 -12.9302181333311\\
73.9288330078125 -12.9460819550096\\
73.9898681640625 -12.9619412459593\\
74.0509033203125 -12.9777960312597\\
74.1119384765625 -12.9936463359983\\
74.1729736328125 -13.0094921852706\\
74.2340087890625 -13.0253336041805\\
74.2950439453125 -13.0411706178399\\
74.3560791015625 -13.0570032513696\\
74.4171142578125 -13.0728315298988\\
74.4781494140625 -13.0886554785653\\
74.5391845703125 -13.1044751225158\\
74.6002197265625 -13.1202904869061\\
74.6612548828125 -13.1361015969008\\
74.7222900390625 -13.1519084776737\\
74.7833251953125 -13.1677111544079\\
74.8443603515625 -13.183509652296\\
74.9053955078125 -13.1993039965398\\
74.9664306640625 -13.2150942123509\\
75.0274658203125 -13.2308803249506\\
75.0885009765625 -13.24666235957\\
75.1495361328125 -13.2624403414499\\
75.2105712890625 -13.2782142958413\\
75.2716064453125 -13.2939842480055\\
75.3326416015625 -13.3097502232136\\
75.3936767578125 -13.3255122467473\\
75.4547119140625 -13.3412703438987\\
75.5157470703125 -13.3570245399705\\
75.5767822265625 -13.3727748602757\\
75.6378173828125 -13.3885213301384\\
75.6988525390625 -13.4042639748933\\
75.7598876953125 -13.4200028198861\\
75.8209228515625 -13.4357378904735\\
75.8819580078125 -13.4514692120232\\
75.9429931640625 -13.4671968099142\\
76.0040283203125 -13.4829207095368\\
76.0650634765625 -13.4986409362926\\
76.1260986328125 -13.5143575155947\\
76.1871337890625 -13.5300704728678\\
76.2481689453125 -13.5457798335481\\
76.3092041015625 -13.5614856230838\\
76.4007568359375 -13.5850376671876\\
76.4923095703125 -13.6085818194803\\
76.5838623046875 -13.632118166\\
76.6754150390625 -13.6556467928462\\
76.7669677734375 -13.6791677861802\\
76.8585205078125 -13.7026812322262\\
76.9500732421875 -13.7261872172714\\
77.0416259765625 -13.7496858276665\\
77.1331787109375 -13.7731771498269\\
77.2247314453125 -13.7966612702322\\
77.3162841796875 -13.8201382754277\\
77.4078369140625 -13.8436082520241\\
77.4993896484375 -13.8670712866984\\
77.5909423828125 -13.8905274661941\\
77.6824951171875 -13.9139768773219\\
77.7740478515625 -13.9374196069597\\
77.8656005859375 -13.9608557420535\\
77.9571533203125 -13.9842853696173\\
78.0487060546875 -14.0077085767336\\
78.1402587890625 -14.031125450554\\
78.2318115234375 -14.0545360782989\\
78.3233642578125 -14.0779405472585\\
78.4149169921875 -14.1013389447925\\
78.5064697265625 -14.1247313583306\\
78.5980224609375 -14.1481178753727\\
78.6895751953125 -14.1714985834891\\
78.7811279296875 -14.1948735703205\\
78.8726806640625 -14.2182429235783\\
78.9642333984375 -14.2416067310449\\
79.0557861328125 -14.2649650805735\\
79.1473388671875 -14.288318060088\\
79.2388916015625 -14.3116657575838\\
79.3304443359375 -14.335008261127\\
79.4219970703125 -14.3583456588548\\
79.5135498046875 -14.3816780389754\\
79.6051025390625 -14.4050054897681\\
79.6966552734375 -14.4283280995829\\
79.7882080078125 -14.4516459568403\\
79.8797607421875 -14.4749591500317\\
79.9713134765625 -14.4982677677188\\
80.0628662109375 -14.5215718985334\\
80.1544189453125 -14.5448716311772\\
80.2459716796875 -14.5681670544216\\
80.3375244140625 -14.5914582571074\\
80.4290771484375 -14.6147453281441\\
80.5206298828125 -14.6380283565101\\
80.6121826171875 -14.6613074312519\\
80.7037353515625 -14.6845826414836\\
80.7952880859375 -14.7078540763866\\
80.8868408203125 -14.7311218252092\\
80.9783935546875 -14.7543859772654\\
81.0699462890625 -14.7776466219353\\
81.1614990234375 -14.8009038486634\\
81.2530517578125 -14.8241577469586\\
81.3446044921875 -14.8474084063933\\
81.4361572265625 -14.8706559166026\\
81.5277099609375 -14.8939003672833\\
81.6192626953125 -14.9171418481935\\
81.7108154296875 -14.9403804491512\\
81.8023681640625 -14.9636162600337\\
81.8939208984375 -14.9868493707763\\
81.9854736328125 -15.0100798713717\\
82.0770263671875 -15.0333078518682\\
82.1685791015625 -15.0565334023695\\
82.2601318359375 -15.0797566130325\\
82.3516845703125 -15.1029775740668\\
82.4432373046875 -15.1261963757331\\
82.5347900390625 -15.1494131083419\\
82.6263427734375 -15.1726278622519\\
82.7178955078125 -15.1958407278688\\
82.8094482421875 -15.2190517956437\\
82.9010009765625 -15.2422611560715\\
82.9925537109375 -15.265468899689\\
83.0841064453125 -15.2886751170737\\
83.1756591796875 -15.3118798988417\\
83.2672119140625 -15.3350833356459\\
83.3587646484375 -15.358285518174\\
83.4503173828125 -15.3814865371469\\
83.5418701171875 -15.4046864833164\\
83.6334228515625 -15.4278854474629\\
83.7249755859375 -15.4510835203939\\
83.8165283203125 -15.4742807929408\\
83.9080810546875 -15.4974773559576\\
83.9996337890625 -15.5206733003178\\
84.0911865234375 -15.5438687169122\\
84.1827392578125 -15.5670636966462\\
84.2742919921875 -15.5902583304376\\
84.3658447265625 -15.6134527092134\\
84.4573974609375 -15.6366469239072\\
84.5489501953125 -15.6598410654565\\
84.6405029296875 -15.6830352247994\\
84.7320556640625 -15.7062294928721\\
84.8236083984375 -15.7294239606052\\
84.9151611328125 -15.7526187189207\\
85.0067138671875 -15.7758138587288\\
85.0982666015625 -15.7990094709246\\
85.1898193359375 -15.8222056463841\\
85.2813720703125 -15.8454024759612\\
85.3729248046875 -15.8686000504835\\
85.4644775390625 -15.891798460749\\
85.5560302734375 -15.9149977975218\\
85.6475830078125 -15.9381981515282\\
85.7391357421875 -15.9613996134528\\
85.8306884765625 -15.9846022739339\\
85.9222412109375 -16.0078062235597\\
86.0137939453125 -16.0310115528632\\
86.1053466796875 -16.0542183523183\\
86.1968994140625 -16.0774267123345\\
86.2884521484375 -16.1006367232528\\
86.3800048828125 -16.1238484753401\\
86.4715576171875 -16.1470620587846\\
86.5631103515625 -16.1702775636904\\
86.6546630859375 -16.1934950800725\\
86.7462158203125 -16.216714697851\\
86.8377685546875 -16.2399365068457\\
86.9293212890625 -16.2631605967706\\
87.0208740234375 -16.2863870572278\\
87.1124267578125 -16.3096159777015\\
87.2039794921875 -16.3328474475522\\
87.2955322265625 -16.35608155601\\
87.3870849609375 -16.3793183921685\\
87.4786376953125 -16.4025580449783\\
87.5701904296875 -16.4258006032399\\
87.6617431640625 -16.4490461555971\\
87.7532958984375 -16.4722947905299\\
87.8448486328125 -16.4955465963474\\
87.9364013671875 -16.5188016611803\\
88.0279541015625 -16.5420600729733\\
88.1195068359375 -16.5653219194775\\
88.2110595703125 -16.5885872882426\\
88.3026123046875 -16.6118562666086\\
88.3941650390625 -16.635128941698\\
88.4857177734375 -16.6584054004066\\
88.5772705078125 -16.6816857293958\\
88.6688232421875 -16.7049700150831\\
88.7603759765625 -16.7282583436334\\
88.8519287109375 -16.7515508009498\\
88.9434814453125 -16.774847472664\\
89.0350341796875 -16.7981484441272\\
89.1265869140625 -16.8214538003995\\
89.2181396484375 -16.8447636262406\\
89.3096923828125 -16.8680780060992\\
89.4012451171875 -16.8913970241026\\
89.4927978515625 -16.914720764046\\
89.5843505859375 -16.9380493093818\\
89.6759033203125 -16.9613827432082\\
89.7674560546875 -16.9847211482579\\
89.8590087890625 -17.0080646068865\\
89.9505615234375 -17.0314132010607\\
90.0421142578125 -17.0547670123464\\
90.1336669921875 -17.0781261218958\\
90.2252197265625 -17.1014906104353\\
90.3167724609375 -17.1248605582526\\
90.4083251953125 -17.1482360451833\\
90.4998779296875 -17.1716171505977\\
90.5914306640625 -17.1950039533874\\
90.6829833984375 -17.2183965319507\\
90.7745361328125 -17.2417949641793\\
90.8660888671875 -17.2651993274428\\
90.9576416015625 -17.288609698575\\
91.0491943359375 -17.3120261538579\\
91.1407470703125 -17.3354487690069\\
91.2322998046875 -17.358877619155\\
91.3238525390625 -17.3823127788367\\
91.4154052734375 -17.4057543219719\\
91.5069580078125 -17.4292023218494\\
91.5985107421875 -17.4526568511096\\
91.6900634765625 -17.4761179817276\\
91.7816162109375 -17.4995857849957\\
91.8731689453125 -17.5230603315053\\
91.9647216796875 -17.5465416911288\\
92.0562744140625 -17.570029933001\\
92.1478271484375 -17.5935251255001\\
92.2393798828125 -17.6170273362288\\
92.3309326171875 -17.6405366319941\\
92.4224853515625 -17.664053078788\\
92.5140380859375 -17.6875767417667\\
92.6055908203125 -17.71110768523\\
92.6971435546875 -17.7346459726004\\
92.7886962890625 -17.7581916664015\\
92.8802490234375 -17.7817448282361\\
92.9718017578125 -17.8053055187641\\
93.0633544921875 -17.8288737976802\\
93.1549072265625 -17.8524497236901\\
93.2464599609375 -17.8760333544881\\
93.3380126953125 -17.8996247467325\\
93.4295654296875 -17.9232239560219\\
93.5211181640625 -17.9468310368701\\
93.6126708984375 -17.9704460426817\\
93.7042236328125 -17.9940690257263\\
93.7957763671875 -18.0177000371125\\
93.8873291015625 -18.0413391267619\\
93.9788818359375 -18.0649863433823\\
94.0704345703125 -18.0886417344404\\
94.1619873046875 -18.1123053461345\\
94.2535400390625 -18.1359772233661\\
94.3450927734375 -18.159657409712\\
94.4366455078125 -18.1833459473947\\
94.5281982421875 -18.2070428772536\\
94.6197509765625 -18.230748238715\\
94.7113037109375 -18.2544620697618\\
94.8028564453125 -18.2781844069029\\
94.8944091796875 -18.3019152851421\\
94.9859619140625 -18.3256547379462\\
95.0775146484375 -18.3494027972133\\
95.1690673828125 -18.3731594932403\\
95.2606201171875 -18.3969248546894\\
95.3521728515625 -18.4206989085554\\
95.4437255859375 -18.4444816801313\\
95.5352783203125 -18.468273192974\\
95.6268310546875 -18.49207346887\\
95.7183837890625 -18.5158825277995\\
95.8099365234375 -18.5397003879012\\
95.9014892578125 -18.5635270654359\\
95.9930419921875 -18.5873625747502\\
96.0845947265625 -18.6112069282391\\
96.1761474609375 -18.6350601363089\\
96.2677001953125 -18.6589222073392\\
96.3592529296875 -18.6827931476447\\
96.4508056640625 -18.7066729614363\\
96.5423583984375 -18.7305616507821\\
96.6339111328125 -18.7544592155679\\
96.7254638671875 -18.7783656534573\\
96.8170166015625 -18.8022809598514\\
96.9085693359375 -18.8262051278481\\
97.0001220703125 -18.850138148201\\
97.0916748046875 -18.8740800092785\\
97.1832275390625 -18.8980306970213\\
97.2747802734375 -18.9219901949011\\
97.3663330078125 -18.9459584838782\\
97.4578857421875 -18.9699355423583\\
97.5494384765625 -18.9939213461506\\
97.6409912109375 -19.0179158684237\\
97.7325439453125 -19.0419190796632\\
97.8240966796875 -19.0659309476275\\
97.9156494140625 -19.0899514373043\\
98.0072021484375 -19.1139805108669\\
98.0987548828125 -19.1380181276303\\
98.1903076171875 -19.1620642440068\\
98.2818603515625 -19.1861188134621\\
98.3734130859375 -19.2101817864711\\
98.4649658203125 -19.2342531104741\\
98.5565185546875 -19.2583327298323\\
98.6480712890625 -19.2824205857841\\
98.7396240234375 -19.3065166164009\\
98.8311767578125 -19.3306207565441\\
98.9227294921875 -19.3547329378212\\
99.0142822265625 -19.3788530885427\\
99.1058349609375 -19.4029811336794\\
99.1973876953125 -19.4271169948201\\
99.2889404296875 -19.4512605901293\\
99.3804931640625 -19.4754118343064\\
99.4720458984375 -19.4995706385442\\
99.5635986328125 -19.5237369104891\\
99.6551513671875 -19.5479105542017\\
99.7467041015625 -19.5720914701178\\
99.8382568359375 -19.5962795550113\\
99.9298095703125 -19.620474701957\\
100.021362304688 -19.644676800295\\
100.112915039062 -19.6688857355967\\
100.204467773438 -19.6931013896309\\
100.296020507813 -19.7173236403327\\
100.387573242188 -19.7415523617728\\
100.479125976562 -19.7657874241293\\
100.570678710938 -19.7900286936602\\
100.662231445313 -19.8142760326791\\
100.753784179688 -19.8385292995318\\
100.845336914062 -19.8627883485759\\
100.936889648438 -19.887053030162\\
101.028442382812 -19.9113231906181\\
101.119995117188 -19.935598672236\\
101.211547851562 -19.9598793132614\\
101.303100585938 -19.984164947886\\
101.394653320312 -20.0084554062437\\
101.486206054688 -20.03275051441\\
101.577758789062 -20.0570500944045\\
101.669311523438 -20.0813539641983\\
101.760864257812 -20.1056619377249\\
101.852416992188 -20.1299738248951\\
101.943969726562 -20.1542894316177\\
102.035522460938 -20.1786085598241\\
102.127075195312 -20.2029310074985\\
102.218627929688 -20.2272565687137\\
102.310180664062 -20.2515850336722\\
102.401733398438 -20.2759161887546\\
102.493286132812 -20.300249816573\\
102.584838867188 -20.324585696033\\
102.676391601562 -20.348923602401\\
102.767944335938 -20.373263307381\\
102.859497070312 -20.3976045791986\\
102.951049804688 -20.4219471826933\\
103.042602539062 -20.4462908794205\\
103.134155273438 -20.4706354277624\\
103.225708007812 -20.4949805830493\\
103.317260742188 -20.5193260976909\\
103.408813476562 -20.5436717213195\\
103.500366210938 -20.5680172009438\\
103.591918945312 -20.5923622811158\\
103.683471679688 -20.6167067041103\\
103.775024414062 -20.641050210118\\
103.866577148438 -20.6653925374526\\
103.958129882812 -20.6897334227733\\
104.049682617188 -20.7140726013231\\
104.141235351562 -20.738409807183\\
104.232788085938 -20.7627447735445\\
104.324340820312 -20.7870772329996\\
104.415893554688 -20.8114069178506\\
104.507446289063 -20.8357335604397\\
104.598999023438 -20.8600568934998\\
104.690551757812 -20.8843766505277\\
104.782104492188 -20.9086925661806\\
104.873657226563 -20.9330043766967\\
104.965209960938 -20.957311820342\\
105.056762695312 -20.9816146378842\\
105.148315429688 -21.0059125730936\\
105.239868164063 -21.0302053732757\\
105.331420898438 -21.054492789833\\
105.422973632812 -21.0787745788606\\
105.514526367188 -21.1030505017758\\
105.606079101563 -21.1273203259833\\
105.697631835938 -21.151583825578\\
105.789184570312 -21.1758407820884\\
105.880737304688 -21.2000909852593\\
105.972290039063 -21.2243342338802\\
106.063842773438 -21.2485703366573\\
106.155395507812 -21.2727991131344\\
106.246948242188 -21.2970203946633\\
106.338500976563 -21.3212340254264\\
106.430053710938 -21.3454398635149\\
106.521606445312 -21.3696377820643\\
106.613159179688 -21.3938276704496\\
106.704711914063 -21.4180094355449\\
106.796264648438 -21.4421830030486\\
106.887817382812 -21.4663483188784\\
107.009887695312 -21.4985558513712\\
107.131958007812 -21.5307486475087\\
107.254028320312 -21.5629267758538\\
107.376098632812 -21.5950903927154\\
107.498168945312 -21.6272397500433\\
107.620239257812 -21.6593752038681\\
107.742309570312 -21.6914972233202\\
107.864379882812 -21.7236064002658\\
107.986450195312 -21.755703459601\\
108.108520507812 -21.7877892702468\\
108.230590820312 -21.8198648568904\\
108.352661132812 -21.8519314125235\\
108.474731445312 -21.8839903118286\\
108.596801757812 -21.9160431254713\\
108.718872070312 -21.9480916353569\\
108.840942382812 -21.9801378509177\\
108.963012695312 -22.0121840264982\\
109.085083007812 -22.0442326799133\\
109.207153320312 -22.0762866122582\\
109.329223632812 -22.1083489290541\\
109.451293945312 -22.1404230628226\\
109.573364257812 -22.1725127971836\\
109.695434570312 -22.2046222925834\\
109.817504882812 -22.2367561137646\\
109.939575195312 -22.2689192590985\\
110.061645507812 -22.3011171919084\\
110.183715820312 -22.3333558739253\\
110.305786132812 -22.3656418010225\\
110.427856445312 -22.397982041392\\
110.549926757812 -22.4303842763342\\
110.671997070312 -22.4628568438468\\
110.794067382812 -22.4954087852123\\
110.916137695312 -22.5280498947985\\
111.038208007813 -22.5607907733024\\
111.160278320313 -22.5936428846851\\
111.282348632813 -22.626618617063\\
111.404418945313 -22.6597313478415\\
111.526489257813 -22.6929955133958\\
111.648559570313 -22.7264266836275\\
111.770629882813 -22.7600416417458\\
111.892700195313 -22.7938584696491\\
112.014770507813 -22.827896639304\\
112.136840820313 -22.8621771105454\\
112.258911132813 -22.8967224357461\\
112.380981445313 -22.9315568718289\\
112.503051757813 -22.9667065001148\\
112.625122070313 -23.0021993545262\\
112.747192382813 -23.0380655586752\\
112.869262695313 -23.074337472383\\
112.991333007813 -23.1110498481759\\
113.113403320313 -23.1482399982963\\
113.235473632813 -23.1859479727424\\
113.357543945313 -23.2242167488045\\
113.479614257813 -23.2630924324926\\
113.601684570313 -23.3026244721345\\
113.723754882813 -23.3428658842662\\
113.845825195313 -23.3838734917003\\
113.967895507813 -23.4257081733476\\
114.089965820313 -23.4684351249297\\
114.212036132813 -23.5121241291414\\
114.334106445312 -23.5568498330368\\
114.456176757812 -23.602692029377\\
114.578247070312 -23.6497359372927\\
114.700317382812 -23.6980724757907\\
114.822387695312 -23.7477985212183\\
114.944458007812 -23.7990171366161\\
115.066528320312 -23.8518377566876\\
115.188598632812 -23.9063763065729\\
115.310668945312 -23.9627552252996\\
115.432739257812 -24.0211033551221\\
115.554809570312 -24.081555645184\\
115.676879882812 -24.1442526010217\\
115.798950195312 -24.2093393889994\\
115.921020507812 -24.2769644749892\\
116.043090820312 -24.3472776370267\\
116.165161132812 -24.4204271390071\\
116.287231445312 -24.4965557823558\\
116.409301757812 -24.5757954592414\\
116.531372070312 -24.6582597066819\\
116.653442382812 -24.7440335960187\\
116.775512695312 -24.8331600742359\\
116.897583007812 -24.9256215872953\\
117.019653320312 -25.0213154435396\\
117.141723632812 -25.1200208994191\\
117.263793945312 -25.2213553567557\\
117.385864257812 -25.3247163515371\\
117.507934570312 -25.4292052246507\\
117.630004882812 -25.533527603048\\
117.752075195312 -25.6358653305042\\
117.874145507812 -25.7337147558971\\
117.996215820312 -25.8236881868824\\
118.118286132812 -25.9012802733539\\
118.240356445312 -25.9606111487156\\
118.362426757812 -25.9941757005738\\
118.484497070312 -25.9926548281235\\
118.606567382812 -25.9448776253104\\
118.728637695312 -25.8380522112707\\
118.850708007812 -25.6583836901128\\
118.972778320312 -25.3921372711244\\
119.094848632812 -25.0270613688373\\
119.216918945312 -24.5538876518007\\
119.338989257812 -23.9674730099728\\
119.461059570312 -23.2671698362274\\
119.583129882812 -22.4562476619913\\
119.705200195312 -21.5405241122656\\
119.827270507812 -20.5265966736953\\
119.949340820312 -19.4200810189975\\
120.071411132812 -18.2240962224974\\
120.193481445312 -16.9380154463358\\
120.315551757812 -15.5563147097339\\
120.437622070312 -14.0672134636391\\
120.559692382812 -12.4506523885986\\
120.681762695312 -10.6748886231076\\
120.803833007812 -8.6904262603648\\
120.925903320312 -6.41885016252514\\
121.047973632812 -3.73247266949523\\
121.170043945312 -0.426422477759784\\
121.292114257812 3.68886244760438\\
121.414184570312 7.15226132312156\\
121.536254882812 5.43735979821494\\
121.658325195312 2.07813270463017\\
121.780395507812 -0.513350405700571\\
121.902465820312 -2.43027793503653\\
122.024536132812 -3.89499196738402\\
122.146606445312 -5.05064123518807\\
122.268676757812 -5.98565445313844\\
122.390747070312 -6.75680187325555\\
122.512817382812 -7.4022029358515\\
122.634887695312 -7.94849734284505\\
122.756958007812 -8.4149421953524\\
122.879028320312 -8.8158581786532\\
123.001098632812 -9.1621522939878\\
123.123168945312 -9.46230168773189\\
123.245239257812 -9.72301036634169\\
123.367309570312 -9.94966022282853\\
123.489379882812 -10.1466286170048\\
123.611450195312 -10.3175169477476\\
123.733520507812 -10.4653183810691\\
123.855590820312 -10.5925430667222\\
123.977661132812 -10.7013130658858\\
124.099731445312 -10.7934353161043\\
124.221801757813 -10.8704584166243\\
124.343872070313 -10.9337173222267\\
124.465942382813 -10.9843688819929\\
124.588012695313 -11.0234203631985\\
124.710083007813 -11.051752541142\\
124.832153320313 -11.0701385369315\\
124.954223632813 -11.0792592970861\\
125.076293945313 -11.0797163979714\\
125.198364257813 -11.0720427020349\\
125.320434570313 -11.0567112760491\\
125.442504882813 -11.0341428933537\\
125.564575195313 -11.0047123747889\\
125.686645507813 -10.9687539712443\\
125.808715820313 -10.9265659505776\\
125.930786132813 -10.878414520265\\
126.052856445313 -10.8245371924135\\
126.174926757813 -10.7651456781677\\
126.296997070313 -10.7004283828884\\
126.419067382813 -10.6305525609163\\
126.541137695313 -10.5556661785748\\
126.663208007813 -10.4758995258095\\
126.785278320313 -10.3913666101099\\
126.907348632813 -10.3021663608068\\
127.029418945313 -10.2083836672454\\
127.151489257813 -10.1100902705089\\
127.273559570313 -10.0073455251679\\
127.395629882813 -9.90019704482626\\
127.517700195313 -9.7886812429401\\
127.639770507813 -9.67282377842015\\
127.761840820312 -9.55263991382203\\
127.883911132812 -9.42813479244233\\
128.005981445312 -9.29930363931542\\
128.128051757812 -9.16613188991992\\
128.250122070312 -9.02859524931546\\
128.372192382812 -8.88665968341936\\
128.494262695312 -8.74028134316646\\
128.616333007812 -8.5894064213598\\
128.738403320312 -8.43397094108387\\
128.860473632812 -8.27390047360323\\
128.982543945312 -8.10910978267756\\
129.104614257812 -7.93950239117525\\
129.226684570312 -7.76497006472642\\
129.348754882812 -7.5853922059067\\
129.470825195312 -7.40063515104244\\
129.592895507812 -7.21055136015037\\
129.714965820312 -7.01497848871809\\
129.837036132812 -6.8137383279538\\
129.959106445312 -6.60663559771814\\
130.081176757812 -6.39345657352699\\
130.203247070312 -6.17396752569426\\
130.325317382812 -5.94791294475684\\
130.447387695312 -5.71501352265873\\
130.569458007812 -5.47496385360705\\
130.691528320312 -5.22742981184235\\
130.813598632812 -4.97204555554296\\
130.935668945312 -4.70841009640255\\
131.057739257812 -4.43608336268788\\
131.179809570312 -4.15458166933468\\
131.301879882812 -3.86337249127601\\
131.423950195312 -3.56186841499563\\
131.546020507812 -3.24942011736308\\
131.668090820312 -2.92530818905669\\
131.790161132812 -2.58873358102885\\
131.912231445312 -2.23880640503469\\
132.034301757812 -1.87453276162607\\
132.156372070312 -1.4947991996713\\
132.278442382812 -1.09835432937242\\
132.400512695312 -0.683787016321906\\
132.522583007812 -0.249500481099783\\
132.644653320312 0.206318471919305\\
132.766723632812 0.685735946150345\\
132.888793945312 1.19111400539205\\
133.010864257812 1.72516092329188\\
133.132934570312 2.29098863321735\\
133.255004882812 2.89217585088217\\
133.377075195312 3.53283195068009\\
133.499145507812 4.21764944849688\\
133.621215820312 4.95191777423164\\
133.743286132812 5.7414393050894\\
133.865356445312 6.59222229178346\\
133.987426757812 7.50968677229799\\
134.109497070312 8.49683454713151\\
134.231567382812 9.55027733947334\\
134.353637695312 10.6520864834957\\
134.475708007812 11.7546204317891\\
134.597778320312 12.7581804200879\\
134.719848632812 13.4974557099427\\
134.841918945312 13.7839862445727\\
134.963989257812 13.5275175532321\\
135.086059570312 12.8159896067378\\
135.208129882812 11.8374262167942\\
135.330200195312 10.7579537424089\\
135.452270507812 9.67815584264172\\
135.574340820312 8.6462319646411\\
135.696411132812 7.68039237729813\\
135.818481445312 6.78414731812663\\
135.940551757812 5.95453650314464\\
136.062622070312 5.18614341324034\\
136.184692382812 4.47294621813066\\
136.306762695312 3.80912178212533\\
136.428833007812 3.18935732981996\\
136.550903320312 2.6089364637373\\
136.672973632812 2.06372624059566\\
136.795043945312 1.55012497494687\\
136.917114257812 1.06499839295513\\
137.039184570312 0.605616429755209\\
137.161254882812 0.169595658653562\\
137.283325195312 -0.245151088272792\\
137.405395507812 -0.640457897887978\\
137.557983398438 -1.10970026837943\\
137.710571289062 -1.55384051546071\\
137.863159179688 -1.97526113761954\\
138.015747070312 -2.37603119632752\\
138.168334960938 -2.75795722339538\\
138.320922851562 -3.12262460537557\\
138.473510742188 -3.4714313946196\\
138.626098632812 -3.80561608119275\\
138.778686523438 -4.12628052992623\\
138.931274414062 -4.43440902854686\\
139.083862304688 -4.73088419182187\\
139.236450195312 -5.01650031056415\\
139.389038085938 -5.29197461302121\\
139.541625976562 -5.55795681164403\\
139.694213867188 -5.81503723431109\\
139.846801757812 -6.06375378103994\\
139.999389648438 -6.30459790142907\\
140.151977539062 -6.53801975177734\\
140.304565429688 -6.76443266191656\\
140.457153320312 -6.9842170186521\\
140.609741210938 -7.19772365408969\\
140.762329101562 -7.40527681208309\\
140.914916992188 -7.60717675381995\\
141.067504882812 -7.80370205359778\\
141.220092773438 -7.99511162767718\\
141.372680664062 -8.18164653237974\\
141.525268554688 -8.36353156204726\\
141.677856445312 -8.54097667287146\\
141.830444335938 -8.71417825476852\\
141.983032226562 -8.88332027026417\\
142.135620117188 -9.04857527666498\\
142.288208007812 -9.21010534552532\\
142.440795898438 -9.36806289150541\\
142.593383789062 -9.52259142109259\\
142.745971679688 -9.67382621027913\\
142.898559570312 -9.82189491911245\\
143.051147460938 -9.96691815002757\\
143.203735351562 -10.1090099560075\\
143.356323242188 -10.2482783038748\\
143.508911132812 -10.3848254973772\\
143.661499023438 -10.5187485641745\\
143.814086914062 -10.6501396103568\\
143.966674804688 -10.7790861457029\\
144.119262695312 -10.9056713825267\\
144.271850585938 -11.0299745106422\\
144.424438476562 -11.1520709506978\\
144.577026367188 -11.2720325878894\\
144.729614257812 -11.3899279878465\\
144.882202148438 -11.5058225962985\\
145.034790039062 -11.6197789239594\\
145.187377929688 -11.7318567179273\\
145.339965820312 -11.8421131207587\\
145.492553710938 -11.9506028182687\\
145.645141601562 -12.0573781770009\\
145.797729492188 -12.1624893722215\\
145.950317382812 -12.2659845072119\\
146.102905273438 -12.3679097245599\\
146.255493164062 -12.4683093100841\\
146.408081054688 -12.5672257899711\\
146.560668945312 -12.6647000216493\\
146.713256835938 -12.7607712788787\\
146.865844726562 -12.8554773314927\\
147.018432617188 -12.948854520191\\
147.171020507812 -13.0409378267476\\
147.323608398438 -13.1317609399669\\
147.476196289062 -13.2213563176938\\
147.628784179688 -13.3097552451585\\
147.781372070312 -13.3969878899119\\
147.933959960938 -13.4830833535885\\
148.086547851562 -13.5680697207153\\
148.239135742188 -13.6519741047651\\
148.391723632812 -13.7348226916406\\
148.544311523438 -13.8166407807579\\
148.696899414062 -13.8974528238887\\
148.849487304688 -13.9772824619044\\
149.002075195312 -14.0561525595582\\
149.154663085938 -14.1340852384291\\
149.307250976562 -14.2111019081432\\
149.459838867188 -14.2872232959794\\
149.612426757812 -14.362469474959\\
149.765014648438 -14.4368598905107\\
149.917602539062 -14.5104133857984\\
150.070190429688 -14.5831482257897\\
150.222778320312 -14.6550821201409\\
150.375366210938 -14.7262322449666\\
150.527954101562 -14.7966152635595\\
150.680541992188 -14.86624734612\\
150.833129882812 -14.9351441885516\\
150.985717773438 -15.0033210303754\\
151.138305664062 -15.0707926718123\\
151.290893554688 -15.137573490079\\
151.443481445312 -15.2036774549403\\
151.596069335938 -15.2691181435592\\
151.748657226562 -15.3339087546811\\
151.901245117188 -15.3980621221891\\
152.053833007812 -15.4615907280615\\
152.206420898438 -15.5245067147655\\
152.359008789062 -15.5868218971131\\
152.511596679688 -15.6485477736104\\
152.664184570312 -15.7096955373228\\
152.816772460938 -15.7702760862828\\
152.969360351562 -15.8303000334626\\
153.121948242188 -15.8897777163324\\
153.274536132812 -15.9487192060264\\
153.427124023438 -16.0071343161342\\
153.579711914062 -16.0650326111364\\
153.732299804688 -16.1224234145017\\
153.884887695312 -16.1793158164609\\
154.037475585938 -16.2357186814741\\
154.190063476562 -16.2916406554048\\
154.342651367188 -16.3470901724141\\
154.495239257812 -16.40207546159\\
154.647827148438 -16.456604553321\\
154.800415039062 -16.5106852854277\\
154.953002929688 -16.5643253090634\\
155.105590820312 -16.6175320943925\\
155.258178710938 -16.6703129360581\\
155.410766601562 -16.7226749584474\\
155.563354492188 -16.7746251207636\\
155.715942382812 -16.8261702219138\\
155.868530273438 -16.877316905219\\
156.021118164062 -16.9280716629562\\
156.173706054688 -16.9784408407371\\
156.326293945312 -17.0284306417327\\
156.478881835938 -17.0780471307491\\
156.631469726562 -17.1272962381603\\
156.784057617188 -17.1761837637051\\
156.936645507812 -17.2247153801526\\
157.089233398438 -17.2728966368427\\
157.241821289062 -17.3207329631058\\
157.394409179688 -17.3682296715666\\
157.546997070312 -17.4153919613375\\
157.699584960938 -17.4622249211051\\
157.852172851562 -17.5087335321145\\
158.004760742188 -17.5549226710551\\
158.157348632812 -17.6007971128521\\
158.309936523438 -17.6463615333671\\
158.462524414062 -17.6916205120113\\
158.615112304688 -17.7365785342747\\
158.767700195312 -17.7812399941748\\
158.920288085938 -17.8256091966269\\
159.072875976562 -17.8696903597408\\
159.225463867188 -17.9134876170438\\
159.378051757812 -17.9570050196361\\
159.530639648438 -18.0002465382778\\
159.683227539062 -18.0432160654129\\
159.835815429688 -18.0859174171305\\
159.988403320312 -18.1283543350664\\
160.140991210938 -18.1705304882473\\
160.293579101562 -18.2124494748794\\
160.446166992188 -18.2541148240832\\
160.598754882812 -18.2955299975769\\
160.751342773438 -18.3366983913099\\
160.903930664062 -18.3776233370483\\
161.056518554688 -18.4183081039134\\
161.209106445312 -18.4587558998763\\
161.361694335938 -18.4989698732086\\
161.514282226562 -18.5389531138912\\
161.666870117188 -18.5787086549834\\
161.819458007812 -18.618239473952\\
161.972045898438 -18.657548493964\\
162.124633789062 -18.6966385851411\\
162.277221679688 -18.7355125657812\\
162.429809570312 -18.7741732035435\\
162.582397460938 -18.8126232166025\\
162.734985351562 -18.8508652747692\\
162.887573242188 -18.8889020005818\\
163.040161132812 -18.9267359703665\\
163.192749023438 -18.9643697152692\\
163.345336914062 -19.0018057222599\\
163.497924804688 -19.0390464351093\\
163.650512695312 -19.07609425534\\
163.803100585938 -19.1129515431514\\
163.955688476562 -19.1496206183211\\
164.108276367188 -19.1861037610815\\
164.260864257812 -19.2224032129742\\
164.413452148438 -19.2585211776815\\
164.566040039063 -19.2944598218366\\
164.718627929688 -19.3302212758123\\
164.871215820312 -19.3658076344902\\
165.023803710938 -19.4012209580093\\
165.176391601562 -19.4364632724962\\
165.328979492188 -19.471536570776\\
165.481567382812 -19.506442813066\\
165.634155273438 -19.5411839276511\\
165.786743164063 -19.5757618115433\\
165.939331054688 -19.6101783311235\\
166.091918945312 -19.6444353227685\\
166.244506835938 -19.6785345934614\\
166.397094726562 -19.7124779213885\\
166.549682617188 -19.7462670565195\\
166.702270507812 -19.7799037211756\\
166.854858398438 -19.8133896105821\\
167.007446289063 -19.8467263934091\\
167.160034179688 -19.8799157122979\\
167.312622070312 -19.9129591843756\\
167.465209960938 -19.945858401757\\
167.617797851562 -19.978614932035\\
167.770385742188 -20.0112303187589\\
167.922973632812 -20.0437060819016\\
168.106079101562 -20.0824948005133\\
168.289184570312 -20.1210871589841\\
168.472290039063 -20.1594856549203\\
168.655395507812 -20.1976927410397\\
168.838500976562 -20.2357108261937\\
169.021606445312 -20.2735422763602\\
169.204711914063 -20.3111894156104\\
169.387817382812 -20.3486545270489\\
169.570922851562 -20.385939853728\\
169.754028320312 -20.423047599538\\
169.937133789062 -20.4599799300735\\
170.120239257812 -20.4967389734762\\
170.303344726562 -20.5333268212558\\
170.486450195312 -20.5697455290888\\
170.669555664062 -20.605997117597\\
170.852661132813 -20.6420835731044\\
171.035766601562 -20.6780068483758\\
171.218872070312 -20.7137688633353\\
171.401977539062 -20.7493715057665\\
171.585083007813 -20.7848166319948\\
171.768188476562 -20.8201060675519\\
171.951293945312 -20.8552416078244\\
172.134399414062 -20.8902250186842\\
172.317504882813 -20.925058037105\\
172.500610351562 -20.9597423717615\\
172.683715820312 -20.9942797036151\\
172.866821289062 -21.0286716864838\\
173.049926757813 -21.0629199475989\\
173.233032226562 -21.0970260881475\\
173.416137695312 -21.1309916838017\\
173.599243164062 -21.1648182852347\\
173.782348632813 -21.1985074186248\\
173.965454101562 -21.232060586147\\
174.148559570312 -21.2654792664521\\
174.331665039062 -21.2987649151353\\
174.514770507813 -21.3319189651927\\
174.697875976562 -21.3649428274674\\
174.880981445312 -21.3978378910851\\
175.064086914062 -21.4306055238786\\
175.247192382813 -21.4632470728034\\
175.430297851562 -21.4957638643428\\
175.613403320312 -21.5281572049037\\
175.796508789062 -21.5604283812031\\
175.979614257813 -21.5925786606462\\
176.162719726562 -21.6246092916949\\
176.345825195312 -21.6565215042284\\
176.528930664062 -21.6883165098954\\
176.712036132812 -21.7199955024585\\
176.895141601562 -21.7515596581301\\
177.078247070312 -21.7830101359016\\
177.261352539062 -21.8143480778643\\
177.444458007812 -21.8455746095238\\
177.627563476563 -21.8766908401071\\
177.810668945312 -21.9076978628626\\
177.993774414062 -21.938596755354\\
178.176879882812 -21.9693885797469\\
178.359985351563 -22.0000743830902\\
178.543090820312 -22.0306551975901\\
178.726196289062 -22.0611320408793\\
178.909301757812 -22.0915059162791\\
179.092407226563 -22.121777813057\\
179.275512695312 -22.1519487066778\\
179.458618164062 -22.1820195590499\\
179.641723632812 -22.2119913187663\\
179.824829101563 -22.2418649213396\\
180.007934570312 -22.2716412894334\\
180.191040039062 -22.3013213330873\\
180.374145507812 -22.3309059499386\\
180.557250976563 -22.3603960254382\\
180.740356445312 -22.3897924330623\\
180.923461914062 -22.4190960345199\\
181.106567382812 -22.448307679956\\
181.289672851563 -22.4774282081501\\
181.472778320312 -22.506458446711\\
181.655883789062 -22.5353992122676\\
181.838989257812 -22.5642513106555\\
182.022094726563 -22.5930155371\\
182.205200195312 -22.6216926763956\\
182.388305664062 -22.650283503081\\
182.571411132812 -22.6787887816116\\
182.754516601563 -22.7072092665278\\
182.937622070312 -22.7355457026202\\
183.120727539062 -22.7637988250917\\
183.303833007812 -22.7919693597156\\
183.486938476562 -22.8200580229914\\
183.670043945312 -22.8480655222973\\
183.853149414062 -22.8759925560392\\
184.036254882812 -22.9038398137971\\
184.219360351562 -22.9316079764692\\
184.402465820313 -22.959297716412\\
184.585571289062 -22.9869096975785\\
184.768676757812 -23.0144445756535\\
184.951782226562 -23.0419029981866\\
185.134887695313 -23.0692856047216\\
185.317993164062 -23.0965930269247\\
185.501098632812 -23.1238258887097\\
185.684204101562 -23.1509848063601\\
185.867309570313 -23.1780703886504\\
186.050415039062 -23.2050832369636\\
186.233520507812 -23.2320239454074\\
186.416625976562 -23.2588931009278\\
186.599731445313 -23.2856912834208\\
186.782836914062 -23.3124190658417\\
186.965942382812 -23.3390770143127\\
187.149047851562 -23.3656656882282\\
187.332153320313 -23.3921856403584\\
187.515258789062 -23.4186374169509\\
187.698364257812 -23.4450215578301\\
187.881469726562 -23.4713385964954\\
188.064575195313 -23.4975890602169\\
188.247680664062 -23.5237734701301\\
188.430786132812 -23.5498923413282\\
188.613891601562 -23.575946182953\\
188.796997070313 -23.6019354982843\\
188.980102539062 -23.6278607848274\\
189.163208007812 -23.6537225343994\\
189.346313476562 -23.6795212332132\\
189.529418945313 -23.7052573619611\\
189.712524414062 -23.7309313958961\\
189.895629882812 -23.7565438049117\\
190.078735351562 -23.7820950536209\\
190.261840820312 -23.8075856014335\\
190.444946289063 -23.8330159026317\\
190.628051757812 -23.8583864064447\\
190.811157226562 -23.8836975571224\\
190.994262695312 -23.9089497940068\\
191.177368164063 -23.934143551603\\
191.360473632812 -23.9592792596489\\
191.543579101562 -23.9843573431831\\
191.726684570312 -24.0093782226123\\
191.909790039063 -24.034342313777\\
192.092895507812 -24.0592500280167\\
192.276000976562 -24.0841017722333\\
192.459106445312 -24.1088979489539\\
192.642211914063 -24.1336389563923\\
192.825317382812 -24.1583251885095\\
193.008422851562 -24.1829570350733\\
193.191528320312 -24.2075348817168\\
193.374633789063 -24.2320591099956\\
193.557739257812 -24.256530097445\\
193.740844726562 -24.2809482176349\\
193.923950195312 -24.3053138402248\\
194.107055664063 -24.3296273310178\\
194.290161132812 -24.353889052013\\
194.473266601562 -24.3780993614578\\
194.656372070312 -24.4022586138989\\
194.839477539063 -24.4263671602329\\
195.022583007812 -24.4504253477552\\
195.205688476562 -24.4744335202096\\
195.388793945312 -24.498392017835\\
195.571899414063 -24.5223011774135\\
195.755004882812 -24.5461613323162\\
195.938110351562 -24.5699728125491\\
196.121215820312 -24.5937359447974\\
196.304321289063 -24.6174510524706\\
196.487426757812 -24.6411184557448\\
196.670532226562 -24.6647384716064\\
196.853637695312 -24.6883114138934\\
197.036743164062 -24.7118375933372\\
197.219848632813 -24.7353173176034\\
197.402954101562 -24.7587508913316\\
197.586059570312 -24.7821386161748\\
197.769165039062 -24.8054807908386\\
197.952270507813 -24.8287777111191\\
198.135375976562 -24.8520296699406\\
198.318481445312 -24.8752369573928\\
198.501586914062 -24.8983998607669\\
198.715209960938 -24.925367530933\\
198.928833007813 -24.9522756237362\\
199.142456054688 -24.9791245813341\\
199.356079101562 -25.0059148409813\\
199.569702148438 -25.032646835101\\
199.783325195312 -25.059320991357\\
199.996948242188 -25.0859377327227\\
200.210571289062 -25.1124974775505\\
200.424194335938 -25.1390006396387\\
200.637817382813 -25.1654476282979\\
200.851440429688 -25.1918388484163\\
201.065063476562 -25.2181747005234\\
201.278686523438 -25.2444555808528\\
201.492309570312 -25.2706818814041\\
201.705932617188 -25.2968539900034\\
201.919555664062 -25.322972290363\\
202.133178710938 -25.34903716214\\
202.346801757813 -25.3750489809938\\
202.560424804688 -25.4010081186429\\
202.774047851562 -25.42691494292\\
202.987670898438 -25.4527698178275\\
203.201293945312 -25.4785731035903\\
203.414916992188 -25.5043251567095\\
203.628540039062 -25.5300263300137\\
203.842163085938 -25.5556769727104\\
204.055786132812 -25.5812774304362\\
204.269409179688 -25.6068280453061\\
204.483032226563 -25.6323291559619\\
204.696655273438 -25.6577810976202\\
204.910278320312 -25.6831842021193\\
205.123901367188 -25.7085387979651\\
205.337524414062 -25.7338452103768\\
205.551147460938 -25.7591037613316\\
205.764770507812 -25.7843147696083\\
205.978393554688 -25.8094785508308\\
206.192016601563 -25.8345954175104\\
206.405639648438 -25.8596656790878\\
206.619262695312 -25.8846896419741\\
206.832885742188 -25.9096676095914\\
207.046508789062 -25.9345998824122\\
207.260131835938 -25.9594867579993\\
207.473754882812 -25.9843285310437\\
207.687377929688 -26.0091254934027\\
207.901000976563 -26.0338779341376\\
208.114624023438 -26.0585861395497\\
208.328247070312 -26.0832503932172\\
208.541870117188 -26.1078709760302\\
208.755493164062 -26.132448166226\\
208.969116210938 -26.1569822394236\\
209.182739257812 -26.1814734686575\\
209.396362304688 -26.2059221244109\\
209.609985351563 -26.230328474649\\
209.823608398438 -26.2546927848511\\
210.037231445312 -26.2790153180424\\
210.250854492188 -26.3032963348256\\
210.464477539062 -26.3275360934114\\
210.678100585938 -26.3517348496494\\
210.891723632812 -26.3758928570576\\
211.105346679688 -26.4000103668519\\
211.318969726562 -26.4240876279756\\
211.532592773438 -26.4481248871273\\
211.746215820313 -26.4721223887896\\
211.959838867188 -26.4960803752567\\
212.173461914062 -26.5199990866613\\
212.387084960938 -26.5438787610021\\
212.600708007812 -26.5677196341698\\
212.814331054688 -26.5915219399736\\
213.027954101562 -26.6152859101666\\
213.241577148438 -26.639011774471\\
213.455200195313 -26.6626997606035\\
213.668823242188 -26.6863500942996\\
213.882446289062 -26.7099629993378\\
214.096069335938 -26.7335386975632\\
214.309692382812 -26.7570774089118\\
214.523315429688 -26.7805793514327\\
214.736938476562 -26.8040447413114\\
214.950561523438 -26.8274737928924\\
215.164184570313 -26.850866718701\\
215.377807617188 -26.8742237294655\\
215.591430664062 -26.8975450341382\\
215.805053710938 -26.9208308399173\\
216.018676757812 -26.9440813522672\\
216.232299804688 -26.9672967749393\\
216.445922851562 -26.9904773099924\\
216.659545898438 -27.0136231578127\\
216.873168945312 -27.0367345171334\\
217.086791992188 -27.0598115850542\\
217.300415039063 -27.0828545570605\\
217.514038085938 -27.1058636270421\\
217.727661132812 -27.1288389873125\\
217.941284179688 -27.1517808286262\\
218.154907226562 -27.1746893401977\\
218.368530273438 -27.1975647097191\\
218.582153320312 -27.2204071233775\\
218.795776367188 -27.2432167658725\\
219.009399414063 -27.2659938204334\\
219.223022460938 -27.2887384688361\\
219.436645507812 -27.3114508914195\\
219.650268554688 -27.3341312671023\\
219.863891601562 -27.3567797733987\\
220.077514648438 -27.3793965864348\\
220.291137695312 -27.4019818809643\\
220.504760742188 -27.4245358303837\\
220.718383789063 -27.4470586067481\\
220.932006835938 -27.4695503807859\\
221.145629882812 -27.4920113219139\\
221.359252929688 -27.514441598252\\
221.572875976562 -27.5368413766379\\
221.786499023438 -27.5592108226409\\
222.000122070312 -27.5815501005766\\
222.213745117188 -27.6038593735204\\
222.427368164063 -27.6261388033215\\
222.640991210938 -27.6483885506163\\
222.854614257812 -27.6706087748421\\
223.068237304688 -27.6927996342497\\
223.281860351562 -27.7149612859171\\
223.495483398438 -27.7370938857619\\
223.709106445312 -27.7591975885543\\
223.922729492188 -27.7812725479292\\
224.136352539062 -27.8033189163992\\
224.349975585938 -27.8253368453662\\
224.563598632813 -27.8473264851335\\
224.777221679688 -27.8692879849182\\
224.990844726562 -27.8912214928623\\
225.204467773438 -27.9131271560447\\
225.418090820312 -27.9350051204923\\
225.631713867188 -27.9568555311917\\
225.845336914062 -27.9786785320997\\
226.058959960938 -28.000474266155\\
226.272583007813 -28.0222428752886\\
226.486206054688 -28.0439845004346\\
226.699829101562 -28.0656992815409\\
226.913452148438 -28.0873873575793\\
227.127075195312 -28.1090488665563\\
227.340698242188 -28.1306839455228\\
227.554321289062 -28.1522927305847\\
227.767944335938 -28.1738753569122\\
227.981567382813 -28.1954319587499\\
228.195190429688 -28.2169626694267\\
228.408813476562 -28.2384676213648\\
228.622436523438 -28.2599469460898\\
228.836059570312 -28.2814007742394\\
229.049682617188 -28.302829235573\\
229.293823242188 -28.3272880095616\\
229.537963867188 -28.3517140107206\\
229.782104492188 -28.3761074286201\\
230.026245117188 -28.4004684513967\\
230.270385742188 -28.4247972657706\\
230.514526367188 -28.4490940570617\\
230.758666992188 -28.4733590092061\\
231.002807617188 -28.4975923047725\\
231.246948242188 -28.5217941249775\\
231.491088867188 -28.5459646497016\\
231.735229492188 -28.5701040575047\\
231.979370117188 -28.5942125256411\\
232.223510742188 -28.6182902300749\\
232.467651367188 -28.6423373454942\\
232.711791992188 -28.6663540453263\\
232.955932617188 -28.6903405017519\\
233.200073242188 -28.7142968857192\\
233.444213867188 -28.7382233669582\\
233.688354492188 -28.7621201139942\\
233.932495117188 -28.7859872941618\\
234.176635742188 -28.809825073618\\
234.420776367188 -28.8336336173559\\
234.664916992188 -28.8574130892176\\
234.909057617188 -28.8811636519071\\
235.153198242188 -28.904885467003\\
235.397338867188 -28.9285786949714\\
235.641479492188 -28.9522434951781\\
235.885620117188 -28.9758800259009\\
236.129760742188 -28.9994884443418\\
236.373901367188 -29.0230689066388\\
236.618041992188 -29.0466215678778\\
236.862182617188 -29.0701465821044\\
237.106323242188 -29.0936441023349\\
237.350463867188 -29.1171142805682\\
237.594604492188 -29.1405572677967\\
237.838745117188 -29.1639732140173\\
238.082885742188 -29.1873622682427\\
238.327026367188 -29.2107245785115\\
238.571166992188 -29.2340602918995\\
238.815307617188 -29.2573695545299\\
239.059448242188 -29.2806525115836\\
239.303588867188 -29.3039093073096\\
239.547729492188 -29.3271400850349\\
239.791870117188 -29.3503449871748\\
240.036010742188 -29.3735241552423\\
240.280151367188 -29.3966777298585\\
240.524291992188 -29.4198058507613\\
240.768432617188 -29.4429086568158\\
241.012573242188 -29.465986286023\\
241.256713867188 -29.4890388755294\\
241.500854492188 -29.512066561636\\
241.744995117188 -29.5350694798076\\
241.989135742188 -29.5580477646811\\
242.233276367188 -29.5810015500751\\
242.477416992188 -29.6039309689979\\
242.721557617188 -29.6268361536567\\
242.965698242188 -29.6497172354655\\
243.209838867188 -29.6725743450539\\
243.453979492188 -29.6954076122753\\
243.698120117188 -29.7182171662151\\
243.942260742188 -29.7410031351985\\
244.186401367188 -29.7637656467989\\
244.430541992188 -29.7865048278457\\
244.674682617188 -29.8092208044319\\
244.918823242188 -29.831913701922\\
245.162963867188 -29.8545836449595\\
245.407104492188 -29.8772307574745\\
245.651245117188 -29.8998551626912\\
245.895385742188 -29.9224569831349\\
246.139526367188 -29.94503634064\\
246.383666992188 -29.9675933563563\\
246.627807617188 -29.9901281507568\\
246.871948242188 -30.0126408436442\\
247.116088867188 -30.0351315541583\\
247.360229492188 -30.0576004007823\\
247.604370117188 -30.0800475013503\\
247.848510742188 -30.1024729730534\\
248.092651367188 -30.1248769324464\\
248.336791992188 -30.1472594954545\\
248.580932617188 -30.1696207773799\\
248.825073242188 -30.191960892908\\
249.069213867188 -30.2142799561135\\
249.313354492188 -30.2365780804671\\
249.557495117188 -30.2588553788415\\
249.801635742188 -30.2811119635175\\
250.045776367188 -30.3033479461899\\
250.289916992188 -30.3255634379738\\
250.534057617188 -30.3477585494102\\
250.778198242188 -30.3699333904722\\
251.022338867188 -30.3920880705702\\
251.266479492188 -30.4142226985583\\
251.510620117188 -30.4363373827396\\
251.754760742188 -30.4584322308716\\
251.998901367188 -30.4805073501722\\
252.243041992188 -30.5025628473249\\
252.487182617188 -30.524598828484\\
252.731323242188 -30.5466153992806\\
252.975463867188 -30.5686126648272\\
253.219604492188 -30.5905907297233\\
253.463745117188 -30.6125496980606\\
253.707885742188 -30.6344896734279\\
253.952026367188 -30.6564107589164\\
254.196166992188 -30.6783130571248\\
254.440307617188 -30.7001966701637\\
254.684448242188 -30.7220616996613\\
254.928588867188 -30.7439082467677\\
255.172729492188 -30.7657364121599\\
255.416870117188 -30.7875462960465\\
255.661010742188 -30.8093379981727\\
255.905151367188 -30.8311116178245\\
256.149291992188 -30.8528672538334\\
256.393432617188 -30.8746050045815\\
256.637573242188 -30.8963249680053\\
256.881713867188 -30.9180272416007\\
257.125854492188 -30.939711922427\\
257.369995117188 -30.9613791071117\\
257.614135742188 -30.9830288918545\\
257.858276367188 -31.0046613724318\\
258.102416992188 -31.026276644201\\
258.346557617188 -31.0478748021043\\
258.590698242188 -31.0694559406735\\
258.834838867188 -31.0910201540334\\
259.078979492188 -31.1125675359067\\
259.323120117188 -31.1340981796172\\
259.567260742188 -31.1556121780944\\
259.841918945312 -31.1797956453327\\
260.116577148438 -31.2039582942315\\
260.391235351562 -31.2281002553536\\
260.665893554688 -31.2522216586564\\
260.940551757812 -31.2763226334989\\
261.215209960938 -31.3004033086483\\
261.489868164062 -31.3244638122868\\
261.764526367188 -31.348504272018\\
262.039184570312 -31.3725248148739\\
262.313842773438 -31.396525567321\\
262.588500976562 -31.4205066552668\\
262.863159179688 -31.4444682040664\\
263.137817382812 -31.4684103385283\\
263.412475585938 -31.4923331829211\\
263.687133789062 -31.5162368609792\\
263.961791992188 -31.5401214959094\\
264.236450195312 -31.5639872103965\\
264.511108398438 -31.5878341266093\\
264.785766601562 -31.6116623662067\\
265.060424804688 -31.6354720503435\\
265.335083007812 -31.6592632996761\\
265.609741210938 -31.683036234368\\
265.884399414062 -31.7067909740961\\
266.159057617188 -31.7305276380556\\
266.433715820312 -31.7542463449659\\
266.708374023438 -31.7779472130761\\
266.983032226562 -31.8016303601703\\
267.257690429688 -31.8252959035733\\
267.532348632812 -31.8489439601553\\
267.807006835938 -31.872574646338\\
268.081665039062 -31.8961880780992\\
268.356323242188 -31.9197843709781\\
268.630981445312 -31.9433636400807\\
268.905639648438 -31.9669260000846\\
269.180297851562 -31.9904715652442\\
269.454956054688 -32.0140004493954\\
269.729614257812 -32.037512765961\\
270.004272460938 -32.0610086279552\\
270.278930664062 -32.0844881479886\\
270.553588867188 -32.107951438273\\
270.828247070312 -32.1313986106261\\
271.102905273438 -32.1548297764762\\
271.377563476562 -32.1782450468669\\
271.652221679688 -32.2016445324618\\
271.926879882812 -32.225028343549\\
272.201538085938 -32.2483965900456\\
272.476196289062 -32.2717493815024\\
272.750854492188 -32.2950868271079\\
273.025512695312 -32.3184090356934\\
273.300170898438 -32.3417161157367\\
273.574829101562 -32.3650081753671\\
273.849487304688 -32.388285322369\\
274.124145507812 -32.4115476641868\\
274.398803710938 -32.4347953079287\\
274.673461914062 -32.4580283603712\\
274.948120117188 -32.481246927963\\
275.222778320312 -32.5044511168293\\
275.497436523438 -32.5276410327759\\
275.772094726562 -32.550816781293\\
276.046752929688 -32.5739784675597\\
276.321411132812 -32.5971261964475\\
276.596069335938 -32.6202600725244\\
276.870727539062 -32.643380200059\\
277.145385742188 -32.6664866830244\\
277.420043945312 -32.6895796251017\\
277.694702148438 -32.7126591296841\\
277.969360351562 -32.7357252998809\\
278.244018554688 -32.7587782385208\\
278.518676757812 -32.7818180481559\\
278.793334960938 -32.8048448310656\\
279.067993164062 -32.8278586892597\\
279.342651367188 -32.8508597244829\\
279.617309570312 -32.8738480382176\\
279.891967773438 -32.8968237316879\\
280.166625976562 -32.9197869058633\\
280.441284179688 -32.942737661462\\
280.715942382812 -32.9656760989545\\
280.990600585938 -32.988602318567\\
281.265258789062 -33.0115164202851\\
281.539916992188 -33.0344185038571\\
281.814575195312 -33.0573086687972\\
282.089233398438 -33.0801870143896\\
282.363891601562 -33.1030536396909\\
282.638549804688 -33.1259086435342\\
282.913208007812 -33.1487521245322\\
283.187866210938 -33.1715841810804\\
283.462524414062 -33.1944049113604\\
283.737182617188 -33.2172144133433\\
284.011840820312 -33.240012784793\\
284.286499023438 -33.2628001232689\\
284.561157226562 -33.2855765261297\\
284.835815429688 -33.3083420905365\\
285.110473632812 -33.3310969134554\\
285.385131835938 -33.3538410916613\\
285.659790039062 -33.3765747217407\\
285.934448242188 -33.3992979000948\\
286.209106445312 -33.4220107229426\\
286.483764648438 -33.4447132863239\\
286.758422851562 -33.4674056861026\\
287.033081054688 -33.4900880179694\\
287.307739257812 -33.5127603774449\\
287.582397460938 -33.5354228598826\\
287.857055664062 -33.558075560472\\
288.131713867188 -33.5807185742413\\
288.406372070312 -33.6033519960606\\
288.681030273438 -33.6259759206446\\
288.955688476562 -33.6485904425555\\
289.230346679688 -33.6711956562061\\
289.505004882812 -33.6937916558625\\
289.779663085938 -33.7163785356471\\
290.054321289062 -33.738956389541\\
290.359497070312 -33.7640324220813\\
290.664672851562 -33.7890975559322\\
290.969848632812 -33.8141519194251\\
291.275024414062 -33.8391956406945\\
291.580200195312 -33.8642288476828\\
291.885375976562 -33.889251668145\\
292.190551757812 -33.914264229653\\
292.495727539062 -33.9392666596006\\
292.800903320312 -33.9642590852081\\
293.106079101562 -33.9892416335265\\
293.411254882812 -34.0142144314421\\
293.716430664062 -34.0391776056814\\
294.021606445312 -34.0641312828153\\
294.326782226562 -34.0890755892635\\
294.631958007812 -34.1140106512992\\
294.937133789062 -34.1389365950533\\
295.242309570312 -34.163853546519\\
295.547485351562 -34.188761631556\\
295.852661132812 -34.2136609758952\\
296.157836914062 -34.2385517051425\\
296.463012695312 -34.2634339447837\\
296.768188476562 -34.2883078201887\\
297.073364257812 -34.3131734566154\\
297.378540039062 -34.3380309792146\\
297.683715820312 -34.3628805130337\\
297.988891601562 -34.3877221830213\\
298.294067382812 -34.4125561140314\\
298.599243164062 -34.4373824308276\\
298.904418945312 -34.4622012580873\\
299.209594726562 -34.4870127204056\\
299.514770507812 -34.5118169423003\\
299.819946289062 -34.536614048215\\
300.125122070312 -34.5614041625241\\
300.430297851562 -34.5861874095365\\
300.735473632812 -34.6109639135001\\
301.040649414062 -34.6357337986055\\
301.345825195312 -34.6604971889903\\
301.651000976562 -34.6852542087432\\
301.956176757812 -34.7100049819082\\
302.261352539062 -34.7347496324886\\
302.566528320312 -34.7594882844507\\
302.871704101562 -34.7842210617287\\
303.176879882812 -34.8089480882278\\
303.482055664062 -34.833669487829\\
303.787231445312 -34.8583853843927\\
304.092407226562 -34.8830959017629\\
304.397583007812 -34.9078011637712\\
304.702758789062 -34.9325012942406\\
305.007934570312 -34.9571964169899\\
305.313110351562 -34.9818866558374\\
305.618286132812 -35.0065721346051\\
305.923461914062 -35.0312529771224\\
306.228637695312 -35.0559293072303\\
306.533813476562 -35.0806012487856\\
306.838989257812 -35.1052689256642\\
307.144165039062 -35.1299324617657\\
307.449340820312 -35.1545919810172\\
307.754516601562 -35.1792476073771\\
308.059692382812 -35.2038994648392\\
308.364868164062 -35.2285476774367\\
308.670043945312 -35.253192369246\\
308.975219726562 -35.2778336643907\\
309.280395507812 -35.3024716870458\\
309.585571289062 -35.3271065614412\\
309.890747070312 -35.3517384118661\\
310.195922851562 -35.3763673626724\\
310.501098632812 -35.4009935382794\\
310.806274414062 -35.4256170631771\\
311.111450195312 -35.4502380619305\\
311.416625976562 -35.4748566591834\\
311.721801757812 -35.4994729796622\\
312.026977539062 -35.5240871481804\\
312.332153320312 -35.5486992896419\\
312.637329101562 -35.5733095290455\\
312.942504882812 -35.5979179914884\\
313.247680664062 -35.6225248021705\\
313.552856445312 -35.6471300863981\\
313.858032226562 -35.6717339695883\\
314.163208007812 -35.6963365772725\\
314.468383789062 -35.7209380351004\\
314.773559570312 -35.7455384688446\\
315.078735351562 -35.7701380044036\\
315.383911132812 -35.7947367678068\\
315.689086914062 -35.8193348852177\\
315.994262695312 -35.8439324829384\\
316.299438476562 -35.8685296874134\\
316.604614257812 -35.8931266252337\\
316.909790039062 -35.9177234231407\\
317.214965820312 -35.9423202080303\\
317.520141601562 -35.9669171069572\\
317.825317382812 -35.9915142471383\\
318.130493164062 -36.0161117559575\\
318.435668945312 -36.0407097609693\\
318.740844726562 -36.065308389903\\
319.046020507812 -36.0899077706667\\
319.351196289062 -36.1145080313516\\
319.656372070312 -36.1391093002359\\
319.961547851562 -36.1637117057891\\
320.266723632812 -36.1883153766757\\
320.571899414062 -36.2129204417602\\
320.907592773438 -36.2399877776999\\
321.243286132812 -36.2670571288752\\
321.578979492188 -36.2941286676766\\
321.914672851562 -36.3212025668125\\
322.250366210938 -36.3482789993157\\
322.586059570312 -36.3753581385501\\
322.921752929688 -36.4024401582177\\
323.257446289062 -36.429525232365\\
323.593139648438 -36.4566135353906\\
323.928833007812 -36.4837052420516\\
324.264526367188 -36.5108005274703\\
324.600219726562 -36.5378995671418\\
324.935913085938 -36.5650025369403\\
325.271606445312 -36.5921096131268\\
325.607299804688 -36.6192209723554\\
325.942993164062 -36.6463367916809\\
326.278686523438 -36.6734572485656\\
326.614379882813 -36.7005825208865\\
326.950073242188 -36.7277127869428\\
327.285766601563 -36.7548482254623\\
327.621459960938 -36.7819890156096\\
327.957153320312 -36.8091353369925\\
328.292846679688 -36.83628736967\\
328.628540039062 -36.8634452941591\\
328.964233398438 -36.8906092914425\\
329.299926757812 -36.9177795429759\\
329.635620117188 -36.9449562306954\\
329.971313476562 -36.9721395370251\\
330.307006835938 -36.9993296448847\\
330.642700195312 -37.0265267376967\\
330.978393554688 -37.0537309993946\\
331.314086914062 -37.0809426144301\\
331.649780273438 -37.1081617677809\\
331.985473632812 -37.1353886449586\\
332.321166992188 -37.1626234320164\\
332.656860351563 -37.1898663155568\\
332.992553710938 -37.2171174827399\\
333.328247070313 -37.2443771212909\\
333.663940429688 -37.2716454195082\\
333.999633789062 -37.2989225662718\\
334.335327148438 -37.3262087510509\\
334.671020507812 -37.3535041639121\\
335.006713867188 -37.3808089955281\\
335.342407226562 -37.4081234371853\\
335.678100585938 -37.4354476807924\\
336.013793945312 -37.462781918889\\
336.349487304688 -37.4901263446535\\
336.685180664062 -37.517481151912\\
337.020874023438 -37.5448465351467\\
337.356567382812 -37.5722226895043\\
337.692260742188 -37.5996098108049\\
338.027954101562 -37.6270080955509\\
338.363647460938 -37.6544177409351\\
338.699340820313 -37.6818389448502\\
339.035034179688 -37.7092719058974\\
339.370727539063 -37.7367168233956\\
339.706420898438 -37.76417389739\\
340.042114257812 -37.7916433286618\\
340.377807617188 -37.8191253187367\\
340.713500976562 -37.8466200698948\\
341.049194335938 -37.8741277851795\\
341.384887695312 -37.901648668407\\
341.720581054688 -37.9291829241759\\
342.056274414062 -37.9567307578765\\
342.391967773438 -37.9842923757005\\
342.727661132812 -38.0118679846507\\
343.063354492188 -38.039457792551\\
343.399047851562 -38.0670620080557\\
343.734741210938 -38.0946808406601\\
344.070434570312 -38.1223145007098\\
344.406127929688 -38.1499631994115\\
344.741821289063 -38.1776271488426\\
345.077514648438 -38.2053065619617\\
345.413208007813 -38.2330016526192\\
345.748901367188 -38.2607126355671\\
346.084594726562 -38.2884397264702\\
346.420288085938 -38.3161831419163\\
346.755981445312 -38.3439430994271\\
347.091674804688 -38.3717198174687\\
347.427368164062 -38.3995135154631\\
347.763061523438 -38.4273244137985\\
348.098754882812 -38.4551527338409\\
348.434448242188 -38.482998697945\\
348.770141601562 -38.5108625294658\\
349.105834960938 -38.5387444527694\\
349.441528320312 -38.5666446932453\\
349.777221679688 -38.5945634773172\\
350.112915039062 -38.6225010324556\\
350.448608398438 -38.6504575871885\\
350.784301757813 -38.6784333711145\\
351.119995117188 -38.706428614914\\
351.486206054688 -38.73699134716\\
351.852416992188 -38.7675778165248\\
352.218627929688 -38.7981883270036\\
352.584838867188 -38.8288231841977\\
352.951049804688 -38.8594826953331\\
353.317260742188 -38.8901671692811\\
353.683471679688 -38.9208769165776\\
354.049682617188 -38.9516122494436\\
354.415893554688 -38.9823734818058\\
354.782104492188 -39.0131609293166\\
355.148315429688 -39.0439749093753\\
355.514526367188 -39.0748157411494\\
355.880737304688 -39.1056837455954\\
356.246948242188 -39.1365792454805\\
356.613159179688 -39.1675025654046\\
356.979370117188 -39.198454031822\\
357.345581054688 -39.2294339730637\\
357.711791992188 -39.2604427193601\\
358.078002929688 -39.2914806028633\\
358.444213867188 -39.3225479576708\\
358.810424804688 -39.353645119848\\
359.176635742188 -39.3847724274524\\
359.542846679688 -39.4159302205569\\
359.909057617188 -39.4471188412742\\
360.275268554688 -39.4783386337811\\
360.641479492188 -39.509589944343\\
361.007690429688 -39.5408731213389\\
361.373901367188 -39.5721885152866\\
361.740112304688 -39.6035364788685\\
362.106323242188 -39.6349173669569\\
362.472534179688 -39.6663315366404\\
362.838745117188 -39.6977793472508\\
363.204956054688 -39.7292611603888\\
363.571166992188 -39.760777339952\\
363.937377929688 -39.792328252162\\
364.303588867188 -39.8239142655919\\
364.669799804688 -39.855535751195\\
365.036010742188 -39.8871930823324\\
365.402221679688 -39.9188866348025\\
365.768432617188 -39.9506167868695\\
366.134643554688 -39.9823839192935\\
366.500854492188 -40.01418841536\\
366.867065429688 -40.0460306609099\\
367.233276367188 -40.0779110443707\\
367.599487304688 -40.1098299567871\\
367.965698242188 -40.1417877918523\\
368.331909179688 -40.1737849459399\\
368.698120117188 -40.2058218181364\\
369.064331054688 -40.2378988102731\\
369.430541992188 -40.2700163269598\\
369.796752929688 -40.3021747756178\\
370.162963867188 -40.3343745665138\\
370.529174804688 -40.3666161127946\\
370.895385742188 -40.3988998305213\\
371.261596679688 -40.4312261387048\\
371.627807617188 -40.4635954593416\\
371.994018554688 -40.4960082174495\\
372.360229492188 -40.5284648411047\\
372.726440429688 -40.5609657614786\\
373.092651367188 -40.5935114128754\\
373.458862304688 -40.6261022327705\\
373.825073242188 -40.6587386618486\\
374.191284179688 -40.6914211440433\\
374.557495117188 -40.7241501265764\\
374.923706054688 -40.7569260599984\\
375.289916992188 -40.7897493982289\\
375.656127929688 -40.8226205985981\\
376.022338867188 -40.8555401218887\\
376.388549804688 -40.8885084323781\\
376.754760742188 -40.9215259978819\\
377.120971679688 -40.9545932897966\\
377.487182617188 -40.987710783145\\
377.853393554688 -41.0208789566199\\
378.219604492188 -41.0540982926304\\
378.585815429688 -41.0873692773475\\
378.952026367188 -41.1206924007511\\
379.318237304688 -41.1540681566771\\
379.684448242188 -41.1874970428659\\
380.050659179688 -41.2209795610109\\
380.416870117188 -41.2545162168078\\
380.783081054688 -41.2881075200051\\
381.149291992188 -41.3217539844548\\
381.515502929688 -41.3554561281639\\
381.912231445312 -41.3920302221232\\
382.308959960938 -41.4286709444247\\
382.705688476562 -41.4653789706128\\
383.102416992188 -41.5021549822623\\
383.499145507812 -41.5389996670619\\
383.895874023438 -41.5759137188982\\
384.292602539062 -41.6128978379413\\
384.689331054688 -41.6499527307323\\
385.086059570313 -41.6870791102715\\
385.482788085938 -41.724277696108\\
385.879516601563 -41.7615492144319\\
386.276245117188 -41.7988943981663\\
386.672973632812 -41.8363139870622\\
387.069702148438 -41.8738087277943\\
387.466430664062 -41.9113793740584\\
387.863159179688 -41.9490266866706\\
388.259887695312 -41.986751433668\\
388.656616210938 -42.0245543904112\\
389.053344726562 -42.0624363396885\\
389.450073242188 -42.1003980718216\\
389.846801757812 -42.1384403847734\\
390.243530273438 -42.1765640842575\\
390.640258789063 -42.2147699838495\\
391.036987304688 -42.2530589051001\\
391.433715820313 -42.2914316776507\\
391.830444335938 -42.3298891393499\\
392.227172851562 -42.3684321363732\\
392.623901367188 -42.407061523344\\
393.020629882812 -42.4457781634568\\
393.417358398438 -42.4845829286029\\
393.814086914062 -42.5234766994977\\
394.210815429688 -42.5624603658108\\
394.607543945312 -42.601534826298\\
395.004272460938 -42.6407009889361\\
395.401000976562 -42.6799597710591\\
395.797729492188 -42.7193120994982\\
396.194458007813 -42.758758910723\\
396.591186523438 -42.798301150986\\
396.987915039063 -42.8379397764697\\
397.384643554688 -42.8776757534357\\
397.781372070312 -42.9175100583769\\
398.178100585938 -42.957443678173\\
398.574829101562 -42.9974776102479\\
398.971557617188 -43.0376128627303\\
399.368286132812 -43.0778504546176\\
399.765014648438 -43.1181914159422\\
400.161743164062 -43.1586367879416\\
400.558471679688 -43.1991876232308\\
400.955200195312 -43.2398449859788\\
401.351928710938 -43.2806099520874\\
401.748657226562 -43.3214836093746\\
402.145385742188 -43.36246705776\\
402.542114257813 -43.403561409455\\
402.938842773438 -43.4447677891556\\
403.335571289062 -43.4860873342397\\
403.732299804688 -43.5275211949673\\
404.129028320312 -43.5690705346856\\
404.525756835938 -43.6107365300366\\
404.922485351562 -43.6525203711703\\
405.319213867188 -43.6944232619608\\
405.715942382812 -43.7364464202273\\
406.112670898438 -43.7785910779586\\
406.509399414062 -43.8208584815434\\
406.906127929688 -43.8632498920036\\
407.302856445312 -43.905766585233\\
407.699584960938 -43.9484098522406\\
408.096313476563 -43.9911809993989\\
408.493041992188 -44.0340813486965\\
408.889770507812 -44.0771122379966\\
409.286499023438 -44.1202750213001\\
409.683227539062 -44.1635710690141\\
410.079956054688 -44.2070017682261\\
410.476684570312 -44.2505685229835\\
410.873413085938 -44.2942727545786\\
411.270141601562 -44.3381159018401\\
411.666870117188 -44.3820994214298\\
412.063598632812 -44.4262247881462\\
412.490844726563 -44.473904760801\\
412.918090820312 -44.521752864025\\
413.345336914062 -44.5697710128976\\
413.772583007812 -44.617961148679\\
414.199829101562 -44.6663252392996\\
414.627075195313 -44.7148652798593\\
415.054321289062 -44.7635832931391\\
415.481567382812 -44.8124813301237\\
415.908813476562 -44.8615614705372\\
416.336059570312 -44.9108258233896\\
416.763305664063 -44.9602765275374\\
417.190551757812 -45.0099157522567\\
417.617797851562 -45.0597456978291\\
418.045043945313 -45.1097685961425\\
418.472290039062 -45.1599867113052\\
418.899536132812 -45.2104023402746\\
419.326782226562 -45.2610178135015\\
419.754028320312 -45.3118354955887\\
420.181274414063 -45.3628577859664\\
420.608520507812 -45.414087119583\\
421.035766601562 -45.4655259676134\\
421.463012695313 -45.5171768381835\\
421.890258789062 -45.569042277113\\
422.317504882813 -45.6211248686766\\
422.744750976562 -45.6734272363829\\
423.171997070312 -45.7259520437731\\
423.599243164063 -45.7787019952401\\
424.026489257812 -45.8316798368661\\
424.453735351562 -45.8848883572833\\
424.880981445312 -45.9383303885543\\
425.308227539062 -45.9920088070759\\
425.735473632813 -46.045926534505\\
426.162719726562 -46.1000865387081\\
426.589965820312 -46.1544918347352\\
427.017211914063 -46.2091454858183\\
427.444458007812 -46.2640506043956\\
427.871704101562 -46.3192103531622\\
428.298950195312 -46.3746279461478\\
428.726196289062 -46.4303066498223\\
429.153442382813 -46.4862497842306\\
429.580688476562 -46.5424607241564\\
430.007934570312 -46.5989429003175\\
430.435180664062 -46.6556998005918\\
430.862426757812 -46.7127349712759\\
431.289672851563 -46.770052018378\\
431.716918945312 -46.8276546089436\\
432.144165039062 -46.8855464724185\\
432.571411132813 -46.9437314020473\\
432.998657226562 -47.0022132563101\\
433.425903320312 -47.0609959603977\\
433.853149414062 -47.1200835077277\\
434.280395507812 -47.179479961501\\
434.707641601563 -47.2391894563017\\
435.134887695312 -47.2992161997404\\
435.562133789062 -47.3595644741434\\
435.989379882812 -47.4202386382882\\
436.416625976562 -47.4812431291873\\
436.843872070313 -47.5425824639221\\
437.271118164062 -47.6042612415275\\
437.698364257812 -47.6662841449299\\
438.125610351563 -47.7286559429395\\
438.552856445312 -47.791381492299\\
438.980102539062 -47.85446573979\\
439.407348632812 -47.9179137244004\\
439.834594726562 -47.9817305795518\\
440.261840820313 -48.0459215353923\\
440.689086914062 -48.1104919211538\\
441.116333007812 -48.1754471675783\\
441.543579101563 -48.2407928094128\\
441.970825195312 -48.3065344879773\\
442.398071289063 -48.3726779538068\\
442.825317382812 -48.4392290693705\\
443.283081054688 -48.5109929956236\\
443.740844726562 -48.5832392520331\\
444.198608398438 -48.6559754994336\\
444.656372070313 -48.7292095621617\\
445.114135742188 -48.8029494325119\\
445.571899414062 -48.8772032753342\\
446.029663085938 -48.951979432775\\
446.487426757812 -49.0272864291691\\
446.945190429688 -49.1031329760852\\
447.402954101562 -49.1795279775312\\
447.860717773438 -49.2564805353243\\
448.318481445312 -49.3339999546297\\
448.776245117188 -49.4120957496756\\
449.234008789063 -49.4907776496472\\
449.691772460938 -49.5700556047667\\
450.149536132812 -49.6499397925646\\
450.607299804688 -49.730440624347\\
451.065063476562 -49.8115687518657\\
451.522827148438 -49.8933350741953\\
451.980590820312 -49.9757507448235\\
452.438354492188 -50.0588271789604\\
452.896118164062 -50.1425760610709\\
453.353881835938 -50.2270093526375\\
453.811645507813 -50.3121393001556\\
454.269409179688 -50.3979784433692\\
454.727172851562 -50.4845396237497\\
455.184936523438 -50.5718359932215\\
455.642700195312 -50.65988102314\\
456.100463867188 -50.7486885135235\\
456.558227539062 -50.8382726025414\\
457.015991210938 -50.928647776262\\
457.473754882812 -51.0198288786585\\
457.931518554688 -51.1118311218739\\
458.389282226563 -51.204670096744\\
458.847045898438 -51.2983617835728\\
459.304809570312 -51.3929225631585\\
459.762573242188 -51.4883692280602\\
460.220336914062 -51.5847189940985\\
460.678100585938 -51.6819895120763\\
461.135864257813 -51.7801988797048\\
461.593627929688 -51.8793656537181\\
462.051391601562 -51.9795088621498\\
462.509155273438 -52.0806480167484\\
462.966918945313 -52.1828031254941\\
463.424682617188 -52.2859947051822\\
463.882446289062 -52.3902437940237\\
464.340209960938 -52.4955719642094\\
464.797973632812 -52.6020013343749\\
465.255737304688 -52.7095545818914\\
465.713500976563 -52.8182549548952\\
466.171264648438 -52.9281262839576\\
466.629028320312 -53.0391929932765\\
467.086791992188 -53.1514801112577\\
467.544555664062 -53.2650132803298\\
468.002319335938 -53.3798187658151\\
468.460083007812 -53.4959234636503\\
468.917846679688 -53.6133549067219\\
469.375610351562 -53.7321412695447\\
469.833374023438 -53.852311370972\\
470.291137695313 -53.9738946745817\\
470.748901367188 -54.0969212863273\\
471.206665039062 -54.2214219489873\\
471.664428710938 -54.3474280328732\\
472.122192382812 -54.4749715221841\\
472.579956054688 -54.6040849963015\\
473.037719726562 -54.7348016052194\\
473.526000976562 -54.8760376333023\\
474.014282226562 -55.0191775213944\\
474.502563476562 -55.1642633608846\\
474.990844726562 -55.3113377404796\\
475.479125976562 -55.4604436455377\\
475.967407226562 -55.6116243364712\\
476.455688476562 -55.7649232028744\\
476.943969726562 -55.920383589518\\
477.432250976562 -56.0780485897675\\
477.920532226562 -56.2379608013034\\
478.408813476562 -56.4001620382507\\
478.897094726562 -56.5646929929357\\
479.385375976562 -56.7315928394783\\
479.873657226562 -56.9008987702737\\
480.361938476562 -57.0726454551086\\
480.850219726562 -57.2468644111771\\
481.338500976562 -57.4235832705989\\
481.826782226562 -57.6028249301781\\
482.315063476562 -57.7846065660846\\
482.803344726562 -57.9689384938717\\
483.291625976562 -58.1558228517973\\
483.779907226562 -58.3452520828057\\
484.268188476562 -58.5372071878211\\
484.756469726562 -58.73165572028\\
485.244750976562 -58.9285494892421\\
485.733032226562 -59.1278219361509\\
486.221313476562 -59.3293851486833\\
486.709594726562 -59.5331264745042\\
487.197875976562 -59.7389046987021\\
487.686157226562 -59.946545751932\\
488.174438476562 -60.1558379227932\\
488.662719726562 -60.3665265589072\\
489.151000976562 -60.5783082580378\\
489.639282226562 -60.7908245751701\\
490.127563476562 -61.0036553057858\\
490.615844726562 -61.216311451822\\
491.104125976562 -61.4282280371341\\
491.592407226562 -61.6387570154447\\
492.080688476562 -61.8471606065521\\
492.568969726562 -62.0526055050341\\
493.057250976562 -62.2541585260577\\
493.545532226562 -62.4507843773489\\
494.033813476562 -62.6413463616487\\
494.522094726562 -62.8246109003729\\
495.010375976562 -62.9992568003043\\
495.498657226562 -63.1638901291362\\
495.986938476562 -63.3170653889687\\
496.475219726562 -63.4573133506706\\
496.963500976562 -63.5831754220846\\
497.451782226562 -63.6932437808365\\
497.940063476562 -63.7862057554933\\
498.428344726562 -63.8608901752001\\
498.916625976562 -63.9163127511635\\
499.404907226562 -63.9517171432758\\
499.893188476562 -63.9666083268071\\
499.984741210938 -63.9670947665445\\
};
%\addlegendentry{Model $\hat{P}$};
\label{m2t:plantModel}

\addplot [frf, forget plot]
table[row sep=crcr]{
9.0179443359375 15.3883733662002\\
%9.0484619140625 15.5178110316006\\
%9.0789794921875 15.5054723429697\\
9.1094970703125 15.2138403366343\\
%9.1400146484375 15.227116879061\\
%9.1705322265625 15.2378600512608\\
9.2010498046875 15.1446661364503\\
%9.2315673828125 15.0711236189099\\
%9.2620849609375 15.2252231462627\\
9.2926025390625 15.1643565400779\\
%9.3231201171875 15.0288444172769\\
%9.3536376953125 15.0311357429839\\
9.3841552734375 14.9538609143735\\
%9.4146728515625 14.989279807746\\
%9.4451904296875 14.9813603992828\\
9.4757080078125 14.9670006235486\\
%9.5062255859375 14.7258695141575\\
%9.5367431640625 14.7570502261695\\
9.5672607421875 14.7965967791337\\
%9.5977783203125 14.5782591480695\\
%9.6282958984375 14.8342572560874\\
9.6588134765625 14.5751782533543\\
%9.6893310546875 14.6186835752451\\
%9.7198486328125 14.4556966334705\\
9.7503662109375 14.5393369991675\\
%9.7808837890625 14.2921375297925\\
%9.8114013671875 14.5767937977799\\
9.8419189453125 14.3446214852055\\
%9.8724365234375 14.2445110413124\\
%9.9029541015625 14.3569141422931\\
9.9334716796875 14.3486004651505\\
%9.9639892578125 14.2346730690829\\
%9.9945068359375 14.2262426073276\\
10.0250244140625 14.131142115802\\
%10.0555419921875 14.0479324138739\\
%10.0860595703125 14.1967848820025\\
10.1165771484375 14.0491948086043\\
%10.1470947265625 14.0171945040817\\
%10.1776123046875 14.1417212662678\\
10.2081298828125 13.8916473842004\\
%10.2386474609375 13.9793033477931\\
%10.2691650390625 14.13792832583\\
10.2996826171875 14.0644677919739\\
%10.3302001953125 13.5642605902816\\
%10.3607177734375 13.8476710568569\\
10.3912353515625 13.6407362838619\\
%10.4217529296875 13.5650143492925\\
%10.4522705078125 13.6842438583045\\
10.4827880859375 13.6397650835149\\
%10.5133056640625 13.5352160739943\\
%10.5438232421875 13.713138032561\\
10.5743408203125 13.4967188813377\\
%10.6048583984375 13.5766197496464\\
%10.6353759765625 13.6131336879554\\
10.6658935546875 13.5869059933975\\
%10.6964111328125 13.394162173085\\
%10.7269287109375 13.2723271404641\\
10.7574462890625 13.4027268777199\\
%10.7879638671875 13.325522607391\\
%10.8184814453125 13.4039293020112\\
10.8489990234375 13.2458097621548\\
%10.8795166015625 13.3218423795096\\
%10.9100341796875 13.1761698666843\\
10.9405517578125 13.1249073781763\\
%10.9710693359375 13.284076028156\\
%11.0015869140625 13.1603997279633\\
11.0321044921875 13.1328266063105\\
%11.0626220703125 13.0402292772754\\
%11.0931396484375 13.0666434380439\\
11.1236572265625 12.9098447997215\\
%11.1541748046875 12.9500899334331\\
%11.1846923828125 12.9350512046263\\
11.2152099609375 12.8407283446893\\
%11.2457275390625 12.9591770672384\\
%11.2762451171875 12.9001224366914\\
11.3067626953125 12.7468997630622\\
%11.3372802734375 12.8115253479625\\
%11.3677978515625 12.7423662366874\\
11.3983154296875 12.5940488151502\\
%11.4288330078125 12.8009070114976\\
%11.4593505859375 12.7369500709224\\
11.4898681640625 12.4147567372064\\
%11.5203857421875 12.6894218096852\\
%11.5509033203125 12.7335902674375\\
11.5814208984375 12.6547253792365\\
%11.6119384765625 12.3962170212643\\
%11.6424560546875 12.5627083863953\\
11.6729736328125 12.2361758983956\\
%11.7034912109375 12.489749115698\\
%11.7340087890625 12.4454268419936\\
11.7645263671875 12.5944176392559\\
%11.7950439453125 12.2111764396883\\
%11.8255615234375 12.1222576155065\\
11.8560791015625 12.1955055366049\\
%11.8865966796875 12.3461806543683\\
%11.9171142578125 12.3655218867708\\
11.9476318359375 12.3668884126823\\
%11.9781494140625 12.10872064879\\
%12.0086669921875 12.1763880639569\\
12.0391845703125 12.2828437331442\\
%12.0697021484375 12.2729499348623\\
%12.1002197265625 12.2132008016579\\
12.1307373046875 12.0641286547836\\
%12.1612548828125 11.8268298268559\\
%12.1917724609375 12.0669479717062\\
12.2222900390625 12.1132468176934\\
%12.2528076171875 11.9168931275984\\
%12.2833251953125 11.8162442380214\\
12.3138427734375 11.839004449425\\
%12.3443603515625 11.7318453886122\\
%12.3748779296875 11.676624674333\\
12.4053955078125 11.6776797558409\\
%12.4359130859375 11.7183655619522\\
%12.4664306640625 11.8233977942072\\
12.4969482421875 11.7825142439688\\
%12.5274658203125 11.5676501939818\\
%12.5579833984375 11.6444384417772\\
12.5885009765625 11.6088996152533\\
%12.6190185546875 11.4799913955944\\
%12.6495361328125 11.6596403096701\\
12.6800537109375 11.3494481329573\\
%12.7105712890625 11.4628139820706\\
%12.7410888671875 11.3860050579384\\
12.7716064453125 11.5408901373582\\
12.8021240234375 11.2926931266681\\
%12.8326416015625 11.2967960111019\\
%12.8631591796875 11.2973508229727\\
12.8936767578125 11.31527428286\\
%12.9241943359375 11.1971798234197\\
%12.9547119140625 11.3240112412887\\
12.9852294921875 11.2384251100733\\
%13.0157470703125 11.1823829205399\\
%13.0462646484375 11.1175576506843\\
13.0767822265625 11.117342519125\\
%13.1072998046875 11.1749028525287\\
%13.1378173828125 11.0116808420722\\
13.1683349609375 10.9827668948657\\
%13.1988525390625 11.1363951079835\\
%13.2293701171875 10.9090691358996\\
13.2598876953125 11.0604933325232\\
%13.2904052734375 10.8145999720665\\
%13.3209228515625 10.9804019025969\\
13.3514404296875 10.9899061096212\\
%13.3819580078125 10.8283519353396\\
%13.4124755859375 10.8050582954775\\
13.4429931640625 10.7596487633243\\
%13.4735107421875 10.8758961859718\\
%13.5040283203125 10.8626850311997\\
13.5345458984375 10.7525884573852\\
%13.5650634765625 10.7927877850543\\
%13.5955810546875 10.6737449996182\\
13.6260986328125 10.6786271141833\\
%13.6566162109375 10.7087237743308\\
%13.6871337890625 10.5234871169312\\
13.7176513671875 10.7075655856894\\
%13.7481689453125 10.6398458608961\\
%13.7786865234375 10.5099114220498\\
13.8092041015625 10.5697220308866\\
%13.8397216796875 10.5121473612536\\
%13.8702392578125 10.4191648261682\\
13.9007568359375 10.5080985063385\\
%13.9312744140625 10.4566324641128\\
%13.9617919921875 10.3642976333774\\
13.9923095703125 10.371691729121\\
%14.0228271484375 10.37715371538\\
%14.0533447265625 10.3735466486108\\
14.0838623046875 10.2256868813761\\
%14.1143798828125 10.3019698472544\\
%14.1448974609375 10.2687604543491\\
14.1754150390625 10.2134596420106\\
%14.2059326171875 10.2268839452642\\
%14.2364501953125 10.1888086604056\\
14.2669677734375 10.1105987526246\\
%14.2974853515625 10.2084475144267\\
%14.3280029296875 10.1737405533972\\
14.3585205078125 10.0748228714634\\
%14.3890380859375 10.0411212671081\\
%14.4195556640625 9.97506043012875\\
14.4500732421875 9.93676840585966\\
%14.4805908203125 10.0200657871031\\
%14.5111083984375 9.9692048123832\\
14.5416259765625 9.93032149091461\\
%14.5721435546875 9.91481654732701\\
%14.6026611328125 9.94797849964181\\
14.6331787109375 9.86931682026702\\
%14.6636962890625 9.86003411743309\\
%14.6942138671875 9.82002033230065\\
14.7247314453125 9.7713109846078\\
%14.7552490234375 9.76967862467846\\
%14.7857666015625 9.79943455238269\\
14.8162841796875 9.79322797578364\\
%14.8468017578125 9.68945094059105\\
%14.8773193359375 9.65188022911721\\
14.9078369140625 9.65915192848293\\
%14.9383544921875 9.60556201384753\\
%14.9688720703125 9.59513998887646\\
14.9993896484375 9.584901593157\\
%15.0299072265625 9.55502240853204\\
%15.0604248046875 9.51727030512863\\
15.0909423828125 9.5468195245526\\
%15.1214599609375 9.52902933990344\\
%15.1519775390625 9.4621052527876\\
15.1824951171875 9.42982041123727\\
%15.2130126953125 9.43850466133416\\
%15.2435302734375 9.35933344765208\\
15.2740478515625 9.37936129021552\\
%15.3045654296875 9.34769435536782\\
%15.3350830078125 9.32977850765655\\
15.3656005859375 9.28922816805637\\
%15.3961181640625 9.24988462623945\\
%15.4266357421875 9.26968802935969\\
15.4571533203125 9.1789254500103\\
%15.4876708984375 9.21445938963307\\
%15.5181884765625 9.13061810925301\\
15.5487060546875 9.14335494344999\\
%15.5792236328125 9.12881606557032\\
%15.6097412109375 9.09798704425685\\
15.6402587890625 9.01458824428936\\
%15.6707763671875 9.04990478563223\\
%15.7012939453125 9.07207738643674\\
15.7318115234375 9.00206211294758\\
%15.7623291015625 9.01533865176097\\
%15.7928466796875 8.96139170613033\\
15.8233642578125 8.96591940309524\\
%15.8538818359375 8.88612878487115\\
%15.8843994140625 8.93230768370432\\
15.9149169921875 8.89086228535261\\
%15.9454345703125 8.90076983409136\\
%15.9759521484375 8.84025871922599\\
16.0064697265625 8.8386425560111\\
%16.0369873046875 8.81407850616621\\
%16.0675048828125 8.78951620786673\\
16.0980224609375 8.73642694849358\\
%16.1285400390625 8.68476790646615\\
%16.1590576171875 8.69946691997312\\
16.1895751953125 8.70864492977116\\
%16.2200927734375 8.63294869350303\\
%16.2506103515625 8.67347673727293\\
16.2811279296875 8.58867782162645\\
%16.3116455078125 8.59099872364534\\
%16.3421630859375 8.56662716537795\\
16.3726806640625 8.56958740021183\\
%16.4031982421875 8.56623071128404\\
%16.4337158203125 8.48470759151854\\
16.4642333984375 8.5004887187695\\
%16.4947509765625 8.45604413600805\\
%16.5252685546875 8.49351750365192\\
16.5557861328125 8.42562518499296\\
%16.5863037109375 8.39252352220953\\
%16.6168212890625 8.41581613575442\\
16.6473388671875 8.33188744635297\\
%16.6778564453125 8.37996611348724\\
%16.7083740234375 8.29161229789794\\
16.7388916015625 8.31815915394247\\
%16.7694091796875 8.26106779404551\\
%16.7999267578125 8.27440656358451\\
16.8304443359375 8.25826346321409\\
%16.8609619140625 8.21687979342772\\
%16.8914794921875 8.20183471479249\\
16.9219970703125 8.1703489743494\\
%16.9525146484375 8.13532431778814\\
%16.9830322265625 8.1501622924279\\
17.0135498046875 8.10163489677913\\
%17.0440673828125 8.09992786105892\\
%17.0745849609375 8.05972393692733\\
17.1051025390625 8.07318390276582\\
%17.1356201171875 8.00853289517557\\
%17.1661376953125 8.00500031906488\\
17.1966552734375 8.01582689431817\\
%17.2271728515625 7.94997066585089\\
%17.2576904296875 7.93527815470469\\
17.2882080078125 7.9099058318206\\
%17.3187255859375 7.88574538766953\\
%17.3492431640625 7.88031880284162\\
17.3797607421875 7.86726693598821\\
%17.4102783203125 7.84922577140804\\
%17.4407958984375 7.82627445688699\\
17.4713134765625 7.78068503919014\\
%17.5018310546875 7.7708846968611\\
%17.5323486328125 7.76643998180693\\
17.5628662109375 7.74303460830907\\
%17.5933837890625 7.71985741687097\\
%17.6239013671875 7.70144763501666\\
17.6544189453125 7.64345504719728\\
%17.6849365234375 7.67683068118723\\
%17.7154541015625 7.60065939234094\\
17.7459716796875 7.64013556428332\\
%17.7764892578125 7.61918587935591\\
%17.8070068359375 7.57628688574203\\
17.8375244140625 7.56514981607788\\
%17.8680419921875 7.52245931387005\\
%17.8985595703125 7.51753713796838\\
17.9290771484375 7.46360037123629\\
%17.9595947265625 7.45027281316571\\
%17.9901123046875 7.42804247067744\\
18.0206298828125 7.43794798545891\\
%18.0511474609375 7.40416842173118\\
%18.0816650390625 7.40580892866301\\
18.1121826171875 7.33533021797001\\
%18.1427001953125 7.31628607914133\\
%18.1732177734375 7.30339949072031\\
18.2037353515625 7.30178249441228\\
%18.2342529296875 7.27939684802147\\
%18.2647705078125 7.24955669414424\\
18.2952880859375 7.24096264437333\\
%18.3258056640625 7.22292274612312\\
%18.3563232421875 7.1910737184856\\
18.3868408203125 7.18942271926552\\
%18.4173583984375 7.16063156396644\\
%18.4478759765625 7.16001975754283\\
18.4783935546875 7.11455073196965\\
%18.5089111328125 7.09139975660589\\
%18.5394287109375 7.06381946725364\\
18.5699462890625 7.05868770663816\\
%18.6004638671875 7.03141942992076\\
%18.6309814453125 7.04254286665003\\
18.6614990234375 6.99913537239685\\
%18.6920166015625 7.01536730361461\\
%18.7225341796875 6.95872689693856\\
18.7530517578125 6.96323382730203\\
%18.7835693359375 6.92824314777805\\
%18.8140869140625 6.89918470281098\\
18.8446044921875 6.87535760041064\\
%18.8751220703125 6.88301832560463\\
%18.9056396484375 6.84844959564694\\
18.9361572265625 6.80796388759036\\
%18.9666748046875 6.79767904661686\\
%18.9971923828125 6.79140239292069\\
19.0277099609375 6.79492785547228\\
%19.0582275390625 6.73381345771188\\
%19.0887451171875 6.72892191162171\\
19.1192626953125 6.71016378441174\\
%19.1497802734375 6.68830563608221\\
%19.1802978515625 6.67449401710642\\
19.2108154296875 6.62216558394817\\
%19.2413330078125 6.64567735671461\\
%19.2718505859375 6.62651434058703\\
19.3023681640625 6.57744934292329\\
%19.3328857421875 6.58419512983353\\
%19.3634033203125 6.54019014342083\\
19.3939208984375 6.53090910079855\\
%19.4244384765625 6.51858345025359\\
%19.4549560546875 6.4953482530259\\
19.4854736328125 6.46950046864979\\
%19.5159912109375 6.44659379684663\\
%19.5465087890625 6.43512408793261\\
19.5770263671875 6.4176861949339\\
%19.6075439453125 6.40461627217974\\
%19.6380615234375 6.39083539971426\\
19.6685791015625 6.34289958218459\\
%19.6990966796875 6.34469585103769\\
%19.7296142578125 6.32139638139925\\
19.7601318359375 6.3118784792783\\
%19.7906494140625 6.30242779407846\\
%19.8211669921875 6.28138463457731\\
19.8516845703125 6.23456469348361\\
%19.8822021484375 6.24129580524165\\
%19.9127197265625 6.19771072190274\\
19.9432373046875 6.18606761676446\\
%19.9737548828125 6.20101886180957\\
%20.0042724609375 6.19131781850602\\
20.0347900390625 6.14013932696686\\
%20.0653076171875 6.1358840939577\\
%20.0958251953125 6.096422405269\\
20.1263427734375 6.08614644293816\\
%20.1568603515625 6.07133897863974\\
%20.1873779296875 6.02529775543007\\
20.2178955078125 6.02267343265256\\
%20.2484130859375 5.98650297895684\\
%20.2789306640625 5.98141834787141\\
20.3094482421875 5.95605321289528\\
%20.3399658203125 5.9601172165315\\
%20.3704833984375 5.93449648363223\\
20.4010009765625 5.91946543630328\\
%20.4315185546875 5.89318890358116\\
%20.4620361328125 5.87019992130025\\
20.4925537109375 5.83785261445026\\
%20.5230712890625 5.84158720447673\\
%20.5535888671875 5.81031535147416\\
20.5841064453125 5.79727455696572\\
%20.6146240234375 5.79195944721915\\
%20.6451416015625 5.76796581900526\\
20.6756591796875 5.73558833053229\\
%20.7061767578125 5.71848175879366\\
%20.7366943359375 5.72225991481536\\
20.7672119140625 5.68901143767969\\
%20.7977294921875 5.68247747892254\\
%20.8282470703125 5.66376819183741\\
20.8587646484375 5.63526761773316\\
%20.8892822265625 5.60860285776493\\
%20.9197998046875 5.60780305586061\\
20.9503173828125 5.57372746112152\\
%20.9808349609375 5.58101120477042\\
%21.0113525390625 5.54655448675271\\
21.0418701171875 5.5275866256671\\
%21.0723876953125 5.52025026674032\\
%21.1029052734375 5.48448910985194\\
21.1334228515625 5.48559957750233\\
%21.1639404296875 5.4389062198878\\
%21.1944580078125 5.43514433222433\\
21.2249755859375 5.4207585852703\\
%21.2554931640625 5.39501658008246\\
%21.2860107421875 5.3753726708132\\
21.3165283203125 5.36719331112901\\
%21.3470458984375 5.34672500585361\\
%21.3775634765625 5.32689757647271\\
21.4080810546875 5.28337310906523\\
%21.4385986328125 5.29348975767132\\
%21.4691162109375 5.26569913520697\\
21.4996337890625 5.26388796050703\\
%21.5301513671875 5.24935624372011\\
%21.5606689453125 5.21193812142974\\
21.5911865234375 5.21207710914358\\
%21.6217041015625 5.18087475917028\\
%21.6522216796875 5.15819962518447\\
21.6827392578125 5.14906903733964\\
%21.7132568359375 5.14096859076832\\
%21.7437744140625 5.11479256652513\\
21.7742919921875 5.09563514518882\\
%21.8048095703125 5.07744517059931\\
%21.8353271484375 5.07437541817231\\
21.8658447265625 5.05529327682709\\
%21.8963623046875 5.01864695829057\\
%21.9268798828125 5.01528740823662\\
21.9573974609375 4.99409744821139\\
%21.9879150390625 4.98291421121599\\
%22.0184326171875 4.96659327079667\\
22.0489501953125 4.93969775771366\\
%22.0794677734375 4.9233055296527\\
%22.1099853515625 4.91254210257405\\
22.1405029296875 4.87574227012762\\
%22.1710205078125 4.85120813177684\\
%22.2015380859375 4.84773073096484\\
22.2320556640625 4.81602159717314\\
%22.2625732421875 4.8126141013429\\
%22.2930908203125 4.7923652531525\\
22.3236083984375 4.76598797024981\\
%22.3541259765625 4.78103890927915\\
%22.3846435546875 4.74626832610736\\
22.4151611328125 4.73168379288171\\
%22.4456787109375 4.70829573998412\\
%22.4761962890625 4.6927797899663\\
22.5067138671875 4.65445563058208\\
%22.5372314453125 4.6534098454474\\
%22.5677490234375 4.63510813051835\\
22.5982666015625 4.62660683482836\\
%22.6287841796875 4.60912200056715\\
%22.6593017578125 4.57050726266067\\
22.6898193359375 4.5519698607868\\
%22.7203369140625 4.54445011699271\\
%22.7508544921875 4.53049575118457\\
22.7813720703125 4.50499942463553\\
%22.8118896484375 4.49556147571667\\
%22.8424072265625 4.47978932149674\\
22.8729248046875 4.4643324367803\\
%22.9034423828125 4.44383574205399\\
%22.9339599609375 4.42515430150437\\
22.9644775390625 4.40113448689492\\
%22.9949951171875 4.38033563500463\\
%23.0255126953125 4.37274530263047\\
23.0560302734375 4.36161574444939\\
%23.0865478515625 4.33851120961361\\
%23.1170654296875 4.31304640933926\\
23.1475830078125 4.29464588157643\\
%23.1781005859375 4.28955623390421\\
%23.2086181640625 4.26580183358082\\
23.2391357421875 4.24075522653611\\
%23.2696533203125 4.24216358817501\\
%23.3001708984375 4.23505485712714\\
23.3306884765625 4.19190315725183\\
%23.3612060546875 4.17193590810544\\
%23.3917236328125 4.16904287533464\\
23.4222412109375 4.15235817041756\\
%23.4527587890625 4.16031836499129\\
%23.4832763671875 4.12753213515298\\
23.5137939453125 4.0865118017624\\
%23.5443115234375 4.06787513686832\\
%23.5748291015625 4.05011081125871\\
23.6053466796875 4.03102609176336\\
%23.6358642578125 4.0404884190659\\
%23.6663818359375 4.02473261519942\\
23.6968994140625 3.9886484362953\\
%23.7274169921875 3.98446104320936\\
%23.7579345703125 3.9625089585989\\
23.7884521484375 3.93281935501864\\
%23.8189697265625 3.91626847748719\\
%23.8494873046875 3.91303975500438\\
23.8800048828125 3.88136366738547\\
%23.9105224609375 3.86321052275539\\
%23.9410400390625 3.86352996993004\\
23.9715576171875 3.84745038479423\\
%24.0020751953125 3.82025634423447\\
%24.0325927734375 3.80707597473584\\
24.0631103515625 3.7798100523437\\
%24.0936279296875 3.77232359641486\\
%24.1241455078125 3.74737917035628\\
24.1546630859375 3.73428347745735\\
%24.1851806640625 3.72328997345761\\
%24.2156982421875 3.68820265585878\\
24.2462158203125 3.67515493311039\\
%24.2767333984375 3.67180096030479\\
%24.3072509765625 3.63485548012776\\
24.3377685546875 3.64737064625763\\
%24.3682861328125 3.62527990098944\\
%24.3988037109375 3.58176049477663\\
24.4293212890625 3.59196109102966\\
%24.4598388671875 3.56813909870446\\
%24.4903564453125 3.53801002914629\\
24.5208740234375 3.52010094154533\\
%24.5513916015625 3.50622756793518\\
%24.5819091796875 3.49726899102173\\
24.6124267578125 3.45875271391598\\
%24.6429443359375 3.44267377749302\\
%24.6734619140625 3.44236745599129\\
24.7039794921875 3.41872586646692\\
%24.7344970703125 3.42091005319391\\
%24.7650146484375 3.37679251401364\\
24.7955322265625 3.34900303063821\\
%24.8260498046875 3.35726645745934\\
%24.8565673828125 3.33851185148121\\
24.8870849609375 3.33271510403777\\
%24.9176025390625 3.29522793557319\\
%24.9481201171875 3.27385302016098\\
24.9786376953125 3.26551474651904\\
%25.0091552734375 3.21770914605493\\
%25.0396728515625 3.23891102172378\\
25.0701904296875 3.2250019155654\\
%25.1007080078125 3.20556780136615\\
%25.1312255859375 3.19131687799001\\
25.1617431640625 3.15976509123361\\
%25.1922607421875 3.14703120442098\\
%25.2227783203125 3.13547896090823\\
25.2532958984375 3.11535726560356\\
%25.2838134765625 3.11427215332357\\
%25.3143310546875 3.07676026190181\\
25.3448486328125 3.07623389917217\\
%25.3753662109375 3.04865197329552\\
%25.4058837890625 3.05435128027153\\
25.4364013671875 3.01764491239264\\
%25.4669189453125 2.99691796756754\\
%25.4974365234375 2.99463009354672\\
25.5279541015625 2.96988820654636\\
%25.5584716796875 2.9557399382426\\
%25.5889892578125 2.9269919784721\\
25.6195068359375 2.91219419182307\\
%25.6500244140625 2.90365186994126\\
%25.6805419921875 2.88768098792496\\
25.7110595703125 2.89103816367595\\
%25.7415771484375 2.84442162364779\\
%25.7720947265625 2.83093794739137\\
25.8026123046875 2.8334049793772\\
%25.8331298828125 2.79937078653437\\
%25.8636474609375 2.76938964959766\\
25.8941650390625 2.77157546146315\\
%25.9246826171875 2.7543175914951\\
%25.9552001953125 2.71718499430058\\
25.9857177734375 2.72361448285629\\
%26.0162353515625 2.6983787347357\\
%26.0467529296875 2.70120189533583\\
26.0772705078125 2.6684501747936\\
%26.1077880859375 2.6491446559495\\
%26.1383056640625 2.6498339354031\\
26.1688232421875 2.6378187096302\\
%26.1993408203125 2.58017220447358\\
%26.2298583984375 2.60076892099949\\
26.2603759765625 2.5882775814712\\
%26.2908935546875 2.54714955211592\\
%26.3214111328125 2.52011137502708\\
26.3519287109375 2.52793676944871\\
%26.3824462890625 2.50797145314431\\
%26.4129638671875 2.51678979540271\\
26.4434814453125 2.47282376263935\\
%26.4739990234375 2.4493736634463\\
%26.5045166015625 2.48218819177054\\
26.5350341796875 2.42869027783533\\
%26.5655517578125 2.40839020892754\\
%26.5960693359375 2.36351932970285\\
26.6265869140625 2.36952740886732\\
%26.6571044921875 2.33172491291978\\
%26.6876220703125 2.3565852544158\\
26.7181396484375 2.3215223885274\\
%26.7486572265625 2.31566790794243\\
%26.7791748046875 2.3058997150534\\
26.8096923828125 2.32098078662065\\
%26.8402099609375 2.26765984700639\\
%26.8707275390625 2.25745942604176\\
26.9012451171875 2.23035871316962\\
%26.9317626953125 2.257281941366\\
%26.9622802734375 2.20448626642186\\
26.9927978515625 2.16376664622835\\
%27.0233154296875 2.19459002574611\\
%27.0538330078125 2.14553822310534\\
27.0843505859375 2.16113828999393\\
%27.1148681640625 2.16807165243032\\
%27.1453857421875 2.09641037632938\\
27.1759033203125 2.09527684441665\\
%27.2064208984375 2.07772616442263\\
%27.2369384765625 2.07568813716079\\
27.2674560546875 2.05112765188328\\
%27.2979736328125 2.06207870720143\\
%27.3284912109375 2.01003112354689\\
27.3590087890625 2.01981990182576\\
%27.3895263671875 1.99721990065808\\
%27.4200439453125 1.97840142278233\\
27.4505615234375 1.98005075992042\\
%27.4810791015625 1.94282872885338\\
%27.5115966796875 1.95620375328595\\
27.5421142578125 1.89751824175718\\
%27.5726318359375 1.90993164110798\\
%27.6031494140625 1.88933702177775\\
27.6336669921875 1.87079980657149\\
%27.6641845703125 1.8445404404978\\
%27.6947021484375 1.83351228958947\\
27.7252197265625 1.83319715851062\\
%27.7557373046875 1.81521355004536\\
%27.7862548828125 1.81854749990634\\
27.8167724609375 1.7777705206885\\
%27.8472900390625 1.76595393367523\\
%27.8778076171875 1.7543877052541\\
27.9083251953125 1.71921680850006\\
%27.9388427734375 1.68226843728727\\
%27.9693603515625 1.70872459743633\\
27.9998779296875 1.68474120555713\\
%28.0303955078125 1.68920842733226\\
%28.0609130859375 1.63657170754294\\
28.0914306640625 1.63580537122721\\
%28.1219482421875 1.60757964256805\\
%28.1524658203125 1.60837264050552\\
28.1829833984375 1.58459186379383\\
%28.2135009765625 1.58474656924635\\
%28.2440185546875 1.56903669373721\\
28.2745361328125 1.52955646620909\\
%28.3050537109375 1.52239943986674\\
%28.3355712890625 1.51881674184583\\
28.3660888671875 1.50188002093164\\
%28.3966064453125 1.48636879041557\\
%28.4271240234375 1.47964973394778\\
28.4576416015625 1.42882372815559\\
%28.4881591796875 1.43526366883028\\
%28.5186767578125 1.40746120982885\\
28.5491943359375 1.43192540442532\\
%28.5797119140625 1.41446118708245\\
%28.6102294921875 1.36876851885385\\
28.6407470703125 1.33994707770749\\
%28.6712646484375 1.34632841806389\\
%28.7017822265625 1.35106954402741\\
28.7322998046875 1.32830663028243\\
%28.7628173828125 1.31249979412314\\
%28.7933349609375 1.30330880284419\\
28.8238525390625 1.28928553554771\\
%28.8543701171875 1.27065879285583\\
%28.8848876953125 1.25710895338263\\
28.9154052734375 1.22732575806594\\
%28.9459228515625 1.2098082335645\\
%28.9764404296875 1.20012845636969\\
29.0069580078125 1.17174527237606\\
%29.0374755859375 1.17202471244495\\
%29.0679931640625 1.16852048660727\\
29.0985107421875 1.14583379960709\\
%29.1290283203125 1.13552582192265\\
%29.1595458984375 1.12372095140101\\
29.1900634765625 1.08883108329766\\
%29.2205810546875 1.0937770145175\\
%29.2510986328125 1.07901097437549\\
29.2816162109375 1.05428159529403\\
%29.3121337890625 1.02519191543801\\
%29.3426513671875 1.02138829624421\\
29.3731689453125 0.995143475050482\\
%29.4036865234375 0.998157499957303\\
%29.4342041015625 0.971754039755901\\
29.4647216796875 0.976851959152831\\
%29.4952392578125 0.950125334591465\\
%29.5257568359375 0.92499045731176\\
29.5562744140625 0.927972934834152\\
%29.5867919921875 0.914183986973577\\
%29.6173095703125 0.89230410622606\\
29.6478271484375 0.885530136122724\\
%29.6783447265625 0.843858973959284\\
%29.7088623046875 0.845444774589319\\
29.7393798828125 0.863659326768291\\
%29.7698974609375 0.833363160557058\\
%29.8004150390625 0.816686606002995\\
29.8309326171875 0.78543094265422\\
%29.8614501953125 0.80070774446509\\
%29.8919677734375 0.754167905373169\\
29.9224853515625 0.74133272120136\\
%29.9530029296875 0.746458171354747\\
%29.9835205078125 0.704669929869692\\
30.0140380859375 0.702636047133783\\
%30.0445556640625 0.691540098054785\\
%30.0750732421875 0.669470976572428\\
30.1055908203125 0.653000202949948\\
%30.1361083984375 0.626504063252264\\
%30.1666259765625 0.630299648761105\\
30.1971435546875 0.598933268167191\\
%30.2276611328125 0.610256494943505\\
%30.2581787109375 0.602703399443592\\
30.2886962890625 0.571943741006912\\
%30.3192138671875 0.571425402953827\\
%30.3497314453125 0.560201643260572\\
30.3802490234375 0.532007741683584\\
%30.4107666015625 0.521831944141936\\
%30.4412841796875 0.504346370263738\\
30.4718017578125 0.468410390717019\\
%30.5023193359375 0.467695984097029\\
%30.5328369140625 0.452227255738679\\
30.5633544921875 0.433020664469921\\
%30.5938720703125 0.415607349564674\\
%30.6243896484375 0.42288005814346\\
30.6549072265625 0.408784442802952\\
%30.6854248046875 0.371633159765655\\
%30.7159423828125 0.36459357954486\\
30.7464599609375 0.349475825828835\\
%30.7769775390625 0.335385266164693\\
%30.8074951171875 0.321333294873956\\
30.8380126953125 0.315308112610045\\
%30.8685302734375 0.302432289556484\\
%30.8990478515625 0.275696253509977\\
30.9295654296875 0.281439510963585\\
%30.9600830078125 0.277159111075642\\
%30.9906005859375 0.228112786412453\\
31.0211181640625 0.251903816516127\\
%31.0516357421875 0.210652066054229\\
%31.0821533203125 0.193067725477915\\
31.1126708984375 0.164374245802171\\
%31.1431884765625 0.207325837073648\\
%31.1737060546875 0.167325569453251\\
31.2042236328125 0.153458028863472\\
%31.2347412109375 0.106166761657985\\
%31.2652587890625 0.11976304398526\\
31.2957763671875 0.118668350516633\\
%31.3262939453125 0.131845922220351\\
%31.3568115234375 0.0858414838095025\\
31.3873291015625 0.053788959178647\\
%31.4178466796875 0.0489867956281246\\
%31.4483642578125 0.0257384676980328\\
31.4788818359375 0.00452384165770724\\
%31.5093994140625 0.0272348289832033\\
%31.5399169921875 -0.0274438524616427\\
31.5704345703125 -0.0113282269410031\\
%31.6009521484375 -0.0564771694913196\\
%31.6314697265625 -0.0851312910608747\\
31.6619873046875 -0.0557683884125104\\
%31.6925048828125 -0.0614300365778558\\
%31.7230224609375 -0.0830697532237466\\
31.7535400390625 -0.0940874560725586\\
%31.7840576171875 -0.135347733698421\\
%31.8145751953125 -0.165369821664111\\
31.8450927734375 -0.16363654181573\\
%31.8756103515625 -0.177488600223202\\
%31.9061279296875 -0.180391621221528\\
31.9366455078125 -0.181673781350733\\
%31.9671630859375 -0.195166351631459\\
%31.9976806640625 -0.207905873237146\\
32.0281982421875 -0.268059834594144\\
%32.0587158203125 -0.269961932928816\\
%32.0892333984375 -0.277933783565842\\
32.1197509765625 -0.272823456765374\\
%32.1502685546875 -0.30288945270999\\
%32.1807861328125 -0.319113642586386\\
32.2113037109375 -0.324083315888869\\
%32.2418212890625 -0.338134095265546\\
%32.2723388671875 -0.319608269729059\\
32.3028564453125 -0.342025422287413\\
%32.3333740234375 -0.375523916496746\\
%32.3638916015625 -0.372427258693211\\
32.3944091796875 -0.405814951780144\\
%32.4249267578125 -0.403996306841377\\
%32.4554443359375 -0.424137508416066\\
32.4859619140625 -0.448673289528018\\
%32.5164794921875 -0.490530621985693\\
%32.5469970703125 -0.478882156139898\\
32.5775146484375 -0.468981363575276\\
%32.6080322265625 -0.516055222256989\\
%32.6385498046875 -0.50973366392629\\
32.6690673828125 -0.525141455074155\\
%32.6995849609375 -0.529932346085868\\
%32.7301025390625 -0.572517620188054\\
32.7606201171875 -0.581506687560684\\
%32.7911376953125 -0.606063562242522\\
%32.8216552734375 -0.621681870122357\\
32.8521728515625 -0.611949909294206\\
%32.8826904296875 -0.620146192073276\\
%32.9132080078125 -0.628669261475073\\
32.9437255859375 -0.641192960769742\\
%32.9742431640625 -0.673433538770005\\
%33.0047607421875 -0.694083304190909\\
33.0352783203125 -0.700820652095453\\
%33.0657958984375 -0.707053668897261\\
%33.0963134765625 -0.71136493988795\\
33.1268310546875 -0.73906338928194\\
%33.1573486328125 -0.750228909395634\\
%33.1878662109375 -0.782978300475221\\
33.2183837890625 -0.79332802463828\\
%33.2489013671875 -0.767365222399059\\
%33.2794189453125 -0.813699721667136\\
33.3099365234375 -0.794315766471942\\
%33.3404541015625 -0.854257107690661\\
%33.3709716796875 -0.857087372364436\\
33.4014892578125 -0.862198440343142\\
%33.4320068359375 -0.861454452952634\\
%33.4625244140625 -0.857399586621372\\
33.4930419921875 -0.904657817777518\\
%33.5235595703125 -0.918056012088456\\
%33.5540771484375 -0.928802660234678\\
33.5845947265625 -0.968617765054235\\
%33.6151123046875 -0.958925801819945\\
%33.6456298828125 -0.986261843409259\\
33.6761474609375 -0.977787587726154\\
%33.7066650390625 -0.971299537817756\\
%33.7371826171875 -1.01785024713717\\
33.7677001953125 -1.04255933545307\\
%33.7982177734375 -1.04146823519225\\
%33.8287353515625 -1.04172894508069\\
33.8592529296875 -1.06278804791097\\
%33.8897705078125 -1.09072258790969\\
%33.9202880859375 -1.05171302141007\\
33.9508056640625 -1.09747439910244\\
%33.9813232421875 -1.10686976036999\\
%34.0118408203125 -1.13005587008411\\
34.0423583984375 -1.16308342786514\\
%34.0728759765625 -1.16355165869373\\
%34.1033935546875 -1.17921738620436\\
34.1339111328125 -1.18729716439701\\
%34.1644287109375 -1.18971163226767\\
%34.1949462890625 -1.22139180042549\\
34.2254638671875 -1.2191921056668\\
%34.2559814453125 -1.25683584825683\\
%34.2864990234375 -1.28340608864005\\
34.3170166015625 -1.29345668920652\\
%34.3475341796875 -1.28152653509654\\
%34.3780517578125 -1.27947835238652\\
34.4085693359375 -1.28887683037608\\
%34.4390869140625 -1.33829398273916\\
%34.4696044921875 -1.34453140908262\\
34.5001220703125 -1.34178166434092\\
%34.5306396484375 -1.35391828149706\\
%34.5611572265625 -1.38597169223887\\
34.5916748046875 -1.40721948081374\\
%34.6221923828125 -1.41431210443807\\
%34.6527099609375 -1.41963885651815\\
34.6832275390625 -1.44685814791473\\
%34.7137451171875 -1.42126770844207\\
%34.7442626953125 -1.45311041215126\\
34.7747802734375 -1.47166181726506\\
%34.8052978515625 -1.48977403779815\\
%34.8358154296875 -1.49643699438905\\
34.8663330078125 -1.50873162562946\\
%34.8968505859375 -1.54163036031441\\
%34.9273681640625 -1.53183381329476\\
34.9578857421875 -1.56747476954632\\
%34.9884033203125 -1.59237153973998\\
%35.0189208984375 -1.57322569997269\\
35.0494384765625 -1.60359769719056\\
%35.0799560546875 -1.63667352848499\\
%35.1104736328125 -1.62450679962058\\
35.1409912109375 -1.62876827959049\\
%35.1715087890625 -1.63626838011123\\
%35.2020263671875 -1.65131640899649\\
35.2325439453125 -1.67114746255206\\
%35.2630615234375 -1.67345557039176\\
%35.2935791015625 -1.73993985738381\\
35.3240966796875 -1.71890768142415\\
%35.3546142578125 -1.71915491688506\\
%35.3851318359375 -1.76306160571212\\
35.4156494140625 -1.75908210008582\\
%35.4461669921875 -1.77855910235402\\
%35.4766845703125 -1.77550531793013\\
35.5072021484375 -1.79151177313454\\
%35.5377197265625 -1.7879004660812\\
%35.5682373046875 -1.8268735394918\\
35.5987548828125 -1.84049341794186\\
%35.6292724609375 -1.8458059720116\\
%35.6597900390625 -1.88234642319627\\
35.6903076171875 -1.87908758443981\\
%35.7208251953125 -1.8879880512493\\
%35.7513427734375 -1.90985942787345\\
35.7818603515625 -1.89961890500258\\
%35.8123779296875 -1.94102458643063\\
%35.8428955078125 -1.95321718569085\\
35.8734130859375 -1.94271535442949\\
%35.9039306640625 -1.97219201202057\\
%35.9344482421875 -1.99116646244508\\
35.9649658203125 -2.00457902811992\\
%35.9954833984375 -2.01644597336933\\
%36.0260009765625 -2.01598357521318\\
36.0565185546875 -2.03530825008449\\
%36.0870361328125 -2.06968627031807\\
%36.1175537109375 -2.06870994232936\\
36.1480712890625 -2.05443654927396\\
%36.1785888671875 -2.10323679394277\\
%36.2091064453125 -2.10679377787303\\
36.2396240234375 -2.10456349880774\\
%36.2701416015625 -2.140542673059\\
%36.3006591796875 -2.1548578677235\\
36.3311767578125 -2.15549532708498\\
%36.3616943359375 -2.17091510507921\\
%36.3922119140625 -2.20112825953615\\
36.4227294921875 -2.18937379765719\\
%36.4532470703125 -2.21068339900423\\
%36.4837646484375 -2.22176310707642\\
36.5142822265625 -2.23115811428\\
%36.5447998046875 -2.24517337014979\\
%36.5753173828125 -2.2576505285018\\
36.6058349609375 -2.25162110077298\\
%36.6363525390625 -2.28097243338726\\
%36.6668701171875 -2.32107734884026\\
36.6973876953125 -2.30639868039406\\
%36.7279052734375 -2.36475073510173\\
%36.7584228515625 -2.33408708635015\\
36.7889404296875 -2.36041821802444\\
%36.8194580078125 -2.37614339705848\\
%36.8499755859375 -2.39141708981707\\
36.8804931640625 -2.39549593634585\\
%36.9110107421875 -2.40748716556427\\
%36.9415283203125 -2.4577234872433\\
36.9720458984375 -2.41760628888156\\
%37.0025634765625 -2.45083659648032\\
%37.0330810546875 -2.44336342029477\\
37.0635986328125 -2.45305145025991\\
%37.0941162109375 -2.47034273834509\\
%37.1246337890625 -2.50179320011007\\
37.1551513671875 -2.52373201375433\\
%37.1856689453125 -2.5244179169951\\
%37.2161865234375 -2.55605936611143\\
37.2467041015625 -2.54966218919439\\
%37.2772216796875 -2.55956883483623\\
%37.3077392578125 -2.56395680690514\\
37.3382568359375 -2.58766010399068\\
%37.3687744140625 -2.58411385862186\\
%37.3992919921875 -2.61660271403776\\
37.4298095703125 -2.6341363363681\\
%37.4603271484375 -2.65904373103737\\
%37.4908447265625 -2.6467225155243\\
37.5213623046875 -2.688557398521\\
%37.5518798828125 -2.69503044656339\\
%37.5823974609375 -2.71652957331477\\
37.6129150390625 -2.69827479868938\\
%37.6434326171875 -2.75035134916551\\
%37.6739501953125 -2.73081967602958\\
37.7044677734375 -2.73444663289648\\
%37.7349853515625 -2.77311807260526\\
%37.7655029296875 -2.77286580384443\\
37.7960205078125 -2.79620257230345\\
%37.8265380859375 -2.7991550261912\\
%37.8570556640625 -2.82193826525972\\
37.8875732421875 -2.84413739258993\\
%37.9180908203125 -2.82831663284848\\
%37.9486083984375 -2.85595914735518\\
37.9791259765625 -2.8614838988544\\
%38.0096435546875 -2.8747626964527\\
%38.0401611328125 -2.87442823499369\\
38.0706787109375 -2.89956546480401\\
%38.1011962890625 -2.92784434932495\\
%38.1317138671875 -2.9358492883661\\
38.1622314453125 -2.93665379806252\\
%38.1927490234375 -2.97394773967206\\
%38.2232666015625 -2.97980356066785\\
38.2537841796875 -2.97468735866594\\
%38.2843017578125 -2.99834142255456\\
%38.3148193359375 -2.97797180041202\\
38.3453369140625 -3.03465228810748\\
%38.3758544921875 -3.04772715588242\\
%38.4063720703125 -3.04777845340736\\
38.4368896484375 -3.07126445521541\\
%38.4674072265625 -3.09448538629972\\
%38.4979248046875 -3.06946692820315\\
38.5284423828125 -3.10467123503831\\
%38.5589599609375 -3.10228091412599\\
%38.5894775390625 -3.13985410526241\\
38.6199951171875 -3.11305536196574\\
%38.6505126953125 -3.1620702789348\\
%38.6810302734375 -3.16098888341314\\
38.7115478515625 -3.15402909365509\\
38.7420654296875 -3.183887535343\\
};
\addplot [frf, forget plot]
table[row sep=crcr]{
38.7420654296875 -3.183887535343\\
38.7725830078125 -3.18526520336091\\
%38.8031005859375 -3.22627071123977\\
%38.8336181640625 -3.23613377785707\\
38.8641357421875 -3.24137688013623\\
%38.8946533203125 -3.2517833766583\\
%38.9251708984375 -3.26392310480158\\
38.9556884765625 -3.30727582848907\\
%38.9862060546875 -3.28758315361212\\
%39.0167236328125 -3.30814903769719\\
39.0472412109375 -3.32604972870196\\
%39.0777587890625 -3.31809382468127\\
%39.1082763671875 -3.33235349556369\\
39.1387939453125 -3.35525153962539\\
%39.1693115234375 -3.36382995009625\\
%39.1998291015625 -3.38124652104166\\
39.2303466796875 -3.39377171134627\\
%39.2608642578125 -3.41002917287923\\
%39.2913818359375 -3.42629470786737\\
39.3218994140625 -3.44468082718424\\
%39.3524169921875 -3.4309211014987\\
%39.3829345703125 -3.45906507848416\\
39.4134521484375 -3.48523127206942\\
%39.4439697265625 -3.48037462167039\\
%39.4744873046875 -3.51349825282947\\
39.5050048828125 -3.51839205736965\\
%39.5355224609375 -3.54929335183613\\
%39.5660400390625 -3.52217062992811\\
39.5965576171875 -3.54329570361354\\
%39.6270751953125 -3.56555100507268\\
%39.6575927734375 -3.57518825104845\\
39.6881103515625 -3.5836802120944\\
%39.7186279296875 -3.60006977164714\\
%39.7491455078125 -3.61990850582484\\
39.7796630859375 -3.63342946838234\\
%39.8101806640625 -3.63600392015148\\
%39.8406982421875 -3.64157563271991\\
39.8712158203125 -3.63694163549447\\
%39.9017333984375 -3.66797434092388\\
%39.9322509765625 -3.70535548741657\\
39.9627685546875 -3.6837059635546\\
%39.9932861328125 -3.69859974447595\\
%40.0238037109375 -3.71859847411782\\
40.0543212890625 -3.74875817550472\\
%40.0848388671875 -3.75637175023399\\
%40.1153564453125 -3.77113345758037\\
40.1458740234375 -3.76123414437308\\
%40.1763916015625 -3.79124663882294\\
%40.2069091796875 -3.80997495752489\\
40.2374267578125 -3.81431375493723\\
%40.2679443359375 -3.83228628569493\\
%40.2984619140625 -3.84606166009521\\
40.3289794921875 -3.85892513454008\\
%40.3594970703125 -3.85563302393891\\
%40.3900146484375 -3.89646592951882\\
40.4205322265625 -3.86104225185015\\
%40.4510498046875 -3.86986546688233\\
%40.4815673828125 -3.89193978529138\\
40.5120849609375 -3.9120286231354\\
%40.5426025390625 -3.93816906926483\\
%40.5731201171875 -3.95286151812161\\
40.6036376953125 -3.94543152086773\\
%40.6341552734375 -3.96240217530201\\
%40.6646728515625 -3.99353729679254\\
40.6951904296875 -3.99062533622844\\
%40.7257080078125 -4.00696317134255\\
%40.7562255859375 -4.01476346590674\\
40.7867431640625 -4.02587120184217\\
%40.8172607421875 -4.05190652395325\\
%40.8477783203125 -4.0594561373817\\
40.8782958984375 -4.07130919074079\\
%40.9088134765625 -4.07898252095616\\
%40.9393310546875 -4.11033155668901\\
40.9698486328125 -4.09870268661974\\
%41.0003662109375 -4.12193340556589\\
%41.0308837890625 -4.11366790546296\\
41.0614013671875 -4.13172875105019\\
%41.0919189453125 -4.15822340707064\\
%41.1224365234375 -4.16946153555185\\
41.1529541015625 -4.17444029906111\\
%41.1834716796875 -4.19215236271238\\
%41.2139892578125 -4.2100097644277\\
41.2445068359375 -4.23153105843984\\
%41.2750244140625 -4.21819770470023\\
%41.3055419921875 -4.23980459986172\\
41.3360595703125 -4.25681807522513\\
%41.3665771484375 -4.24544149993166\\
%41.3970947265625 -4.28155885969894\\
41.4276123046875 -4.28320616346435\\
%41.4581298828125 -4.28985363402369\\
%41.4886474609375 -4.30685833470445\\
41.5191650390625 -4.32820977283519\\
%41.5496826171875 -4.33149831935996\\
%41.5802001953125 -4.35019579568845\\
41.6107177734375 -4.34614180011323\\
%41.6412353515625 -4.34839281064716\\
%41.6717529296875 -4.39807731346126\\
41.7022705078125 -4.38629016796015\\
%41.7327880859375 -4.40248935802776\\
%41.7633056640625 -4.40653630557468\\
41.7938232421875 -4.43759835217678\\
%41.8243408203125 -4.43731530127043\\
%41.8548583984375 -4.44258035229686\\
41.8853759765625 -4.45811070063342\\
%41.9158935546875 -4.49259157323434\\
%41.9464111328125 -4.49820339561021\\
41.9769287109375 -4.48839151176443\\
%42.0074462890625 -4.52623855569444\\
%42.0379638671875 -4.52165896684405\\
42.0684814453125 -4.50761815655166\\
%42.0989990234375 -4.54459270783845\\
%42.1295166015625 -4.55456763837983\\
42.1600341796875 -4.58113208294288\\
%42.1905517578125 -4.57629785918709\\
%42.2210693359375 -4.56675921016906\\
42.2515869140625 -4.59873759717702\\
%42.2821044921875 -4.62289225418258\\
%42.3126220703125 -4.63229158554522\\
42.3431396484375 -4.64980229031088\\
%42.3736572265625 -4.65233174728273\\
%42.4041748046875 -4.66819279908844\\
42.4346923828125 -4.69369497806235\\
%42.4652099609375 -4.70863802386684\\
%42.4957275390625 -4.70837496751021\\
42.5262451171875 -4.72188450208399\\
%42.5567626953125 -4.7290524544283\\
%42.5872802734375 -4.73103908806316\\
42.6177978515625 -4.73732465695225\\
%42.6483154296875 -4.75667685230638\\
%42.6788330078125 -4.78518372402658\\
42.7093505859375 -4.78450336096297\\
%42.7398681640625 -4.80900543049435\\
%42.7703857421875 -4.81507322991316\\
42.8009033203125 -4.82529422043412\\
%42.8314208984375 -4.84266271831128\\
%42.8619384765625 -4.8370932063506\\
42.8924560546875 -4.86502302706366\\
%42.9229736328125 -4.82506586342102\\
%42.9534912109375 -4.87263090864456\\
42.9840087890625 -4.87489345534457\\
%43.0145263671875 -4.89403679750374\\
%43.0450439453125 -4.93113168029753\\
43.0755615234375 -4.93442946700147\\
%43.1060791015625 -4.93546856968322\\
%43.1365966796875 -4.92503381778064\\
43.1671142578125 -4.96731472133577\\
%43.1976318359375 -4.94378523755156\\
%43.2281494140625 -4.99621551409693\\
43.2586669921875 -4.97703804494695\\
%43.2891845703125 -5.02865735377495\\
%43.3197021484375 -5.03042137521573\\
43.3502197265625 -4.98360953903458\\
%43.3807373046875 -5.09886288170965\\
%43.4112548828125 -5.11006572893675\\
43.4417724609375 -5.13242457260937\\
%43.4722900390625 -5.00783613477634\\
%43.5028076171875 -5.2730009827461\\
43.5333251953125 -5.20961814298784\\
%43.5638427734375 -5.09336489242547\\
%43.5943603515625 -5.14467708592012\\
43.6248779296875 -5.20769076779412\\
%43.6553955078125 -5.12743301713271\\
%43.6859130859375 -5.13008794218691\\
43.7164306640625 -5.11639778996601\\
%43.7469482421875 -5.21750940194349\\
%43.7774658203125 -5.22447914357883\\
43.8079833984375 -5.21636339410067\\
%43.8385009765625 -5.21066868815961\\
%43.8690185546875 -5.28252520085911\\
43.8995361328125 -5.28898835171462\\
%43.9300537109375 -5.26080742621565\\
%43.9605712890625 -5.30760143239473\\
43.9910888671875 -5.27332378722384\\
%44.0216064453125 -5.30664109555197\\
%44.0521240234375 -5.31080985820091\\
44.0826416015625 -5.32781319486763\\
%44.1131591796875 -5.3283003434844\\
%44.1436767578125 -5.30339034137643\\
44.1741943359375 -5.35069785049221\\
%44.2047119140625 -5.40268227329827\\
%44.2352294921875 -5.38099469213686\\
44.2657470703125 -5.30979889293445\\
%44.2962646484375 -5.53206235341005\\
%44.3267822265625 -5.50581444690925\\
44.3572998046875 -5.43005644933758\\
%44.3878173828125 -5.4482101992543\\
%44.4183349609375 -5.46579308444248\\
44.4488525390625 -5.4853601681731\\
%44.4793701171875 -5.44603887232188\\
%44.5098876953125 -5.4553780489083\\
44.5404052734375 -5.48612675087958\\
%44.5709228515625 -5.49342394554563\\
%44.6014404296875 -5.53371815165319\\
44.6319580078125 -5.51470882582394\\
%44.6624755859375 -5.54419717806242\\
%44.6929931640625 -5.52981982479834\\
44.7235107421875 -5.56852059029382\\
%44.7540283203125 -5.58821029059114\\
%44.7845458984375 -5.58367529040942\\
44.8150634765625 -5.5973292438484\\
%44.8455810546875 -5.60721335748637\\
%44.8760986328125 -5.6301978939252\\
44.9066162109375 -5.59619700968695\\
%44.9371337890625 -5.63756515791252\\
%44.9676513671875 -5.64040664921487\\
44.9981689453125 -5.66215923103465\\
%45.0286865234375 -5.68371434309155\\
%45.0592041015625 -5.66638872351235\\
45.0897216796875 -5.71718653808001\\
%45.1202392578125 -5.70329621323043\\
%45.1507568359375 -5.71641982680391\\
45.1812744140625 -5.74347932198624\\
%45.2117919921875 -5.74475324725733\\
%45.2423095703125 -5.77446870839145\\
45.2728271484375 -5.7718400040872\\
%45.3033447265625 -5.77068453538989\\
%45.3338623046875 -5.80757808302586\\
45.3643798828125 -5.80908980588093\\
%45.3948974609375 -5.83160203622617\\
%45.4254150390625 -5.84611782641497\\
45.4559326171875 -5.85323047038686\\
%45.4864501953125 -5.84675540313264\\
%45.5169677734375 -5.87352700601649\\
45.5474853515625 -5.8810168685118\\
%45.5780029296875 -5.90723918311065\\
%45.6085205078125 -5.9189830348282\\
45.6390380859375 -5.9163799844445\\
%45.6695556640625 -5.93912360530374\\
%45.7000732421875 -5.94053125932402\\
45.7305908203125 -5.95403825799455\\
%45.7611083984375 -5.96994775218343\\
%45.7916259765625 -5.98708652218527\\
45.8526611328125 -6.01374562921398\\
%45.9136962890625 -6.01231560169326\\
%45.9747314453125 -6.03221060131502\\
46.0357666015625 -6.04317405718288\\
%46.0968017578125 -6.08598370803475\\
%46.1578369140625 -6.11398826659415\\
46.2188720703125 -6.12789753474135\\
%46.2799072265625 -6.16928857223594\\
%46.3409423828125 -6.18672877024497\\
46.4019775390625 -6.20170137149058\\
%46.4630126953125 -6.22853268077803\\
%46.5240478515625 -6.26675871637956\\
46.5850830078125 -6.28049178439439\\
%46.6461181640625 -6.31830443416803\\
%46.7071533203125 -6.31801506411483\\
46.7681884765625 -6.33799834793165\\
%46.8292236328125 -6.36546211980556\\
%46.8902587890625 -6.39462465722779\\
46.9512939453125 -6.42418300044356\\
%47.0123291015625 -6.42507674559997\\
%47.0733642578125 -6.48592611484428\\
47.1343994140625 -6.47437367339647\\
%47.1954345703125 -6.51265390787571\\
%47.2564697265625 -6.53688533679029\\
47.3175048828125 -6.55661822377186\\
%47.3785400390625 -6.59824605228564\\
%47.4395751953125 -6.60953706125567\\
47.5006103515625 -6.60955258337536\\
%47.5616455078125 -6.64282308312268\\
%47.6226806640625 -6.70562764252776\\
47.6837158203125 -6.6927053078945\\
%47.7447509765625 -6.74946415335894\\
%47.8057861328125 -6.72853250398146\\
47.8668212890625 -6.76978458765804\\
%47.9278564453125 -6.77343112143216\\
%47.9888916015625 -6.82063247367136\\
48.0499267578125 -6.81367291433589\\
%48.1109619140625 -6.84726492488006\\
%48.1719970703125 -6.86902991017731\\
48.2330322265625 -6.90869477839023\\
%48.2940673828125 -6.9162338584766\\
%48.3551025390625 -6.96535707555947\\
48.4161376953125 -6.99581627032608\\
%48.4771728515625 -7.01123849609951\\
%48.5382080078125 -6.99490286853275\\
48.5992431640625 -7.0528726870979\\
%48.6602783203125 -7.07722155798997\\
%48.7213134765625 -7.08252165072167\\
48.7823486328125 -7.10839010472638\\
%48.8433837890625 -7.15571669453132\\
%48.9044189453125 -7.1741395229331\\
48.9654541015625 -7.18206571510154\\
%49.0264892578125 -7.22591066925865\\
%49.0875244140625 -7.23947234683367\\
49.1485595703125 -7.26480909129584\\
%49.2095947265625 -7.28736089647429\\
%49.2706298828125 -7.29251794041687\\
49.3316650390625 -7.31762972403262\\
%49.3927001953125 -7.3539454089295\\
%49.4537353515625 -7.39208346775178\\
49.5147705078125 -7.38523930713316\\
%49.5758056640625 -7.44567742786092\\
%49.6368408203125 -7.46881873047616\\
49.6978759765625 -7.48636149291549\\
%49.7589111328125 -7.48677444672587\\
%49.8199462890625 -7.49110580095818\\
49.8809814453125 -7.54296412262096\\
%49.9420166015625 -7.58206054723953\\
%50.0030517578125 -8.03666974753518\\
50.0640869140625 -7.57539504969707\\
%50.1251220703125 -7.59306404464448\\
%50.1861572265625 -7.60974085773302\\
50.2471923828125 -7.67577505248921\\
%50.3082275390625 -7.67056741955635\\
%50.3692626953125 -7.68392956069113\\
50.4302978515625 -7.76401177299639\\
%50.4913330078125 -7.73923615076126\\
%50.5523681640625 -7.78479557085605\\
50.6134033203125 -7.83160558017715\\
%50.6744384765625 -7.81459923108815\\
%50.7354736328125 -7.8421063798931\\
50.7965087890625 -7.87366884759867\\
%50.8575439453125 -7.90780353935868\\
%50.9185791015625 -7.871549644724\\
50.9796142578125 -7.87795920624777\\
%51.0406494140625 -7.91246929623853\\
%51.1016845703125 -8.00195955960965\\
51.1627197265625 -8.00077388497705\\
%51.2237548828125 -8.01536735947156\\
%51.2847900390625 -8.04248993283214\\
51.3458251953125 -8.08096499988406\\
%51.4068603515625 -8.06542190064301\\
%51.4678955078125 -8.0882734933833\\
51.5289306640625 -8.1357664116195\\
%51.5899658203125 -8.1761691026404\\
%51.6510009765625 -8.18937407490848\\
51.7120361328125 -8.19826834428621\\
%51.7730712890625 -8.21976678703359\\
%51.8341064453125 -8.25331799100481\\
51.8951416015625 -8.2837471443795\\
%51.9561767578125 -8.35034150262938\\
%52.0172119140625 -8.36421486680802\\
52.0782470703125 -8.35035727203937\\
%52.1392822265625 -8.41179027343531\\
%52.2003173828125 -8.40516794625868\\
52.2613525390625 -8.39671166051436\\
%52.3223876953125 -8.4804042594614\\
%52.3834228515625 -8.48193621210558\\
52.4444580078125 -8.50109584709988\\
%52.5054931640625 -8.48212058780063\\
%52.5665283203125 -8.53813454925861\\
52.6275634765625 -8.54996886418161\\
%52.6885986328125 -8.55816197152003\\
%52.7496337890625 -8.63224814208755\\
52.8106689453125 -8.65320798669811\\
%52.8717041015625 -8.67601982012605\\
%52.9327392578125 -8.69139104023964\\
52.9937744140625 -8.72019477118662\\
%53.0548095703125 -8.75043105600328\\
%53.1158447265625 -8.71695890269288\\
53.1768798828125 -8.72752345512583\\
%53.2379150390625 -8.78585091665675\\
%53.2989501953125 -8.78875693368835\\
53.3599853515625 -8.8541675105765\\
%53.4210205078125 -8.81755488875073\\
%53.4820556640625 -8.90599041445278\\
53.5430908203125 -8.89633056667221\\
%53.6041259765625 -8.89363856418414\\
%53.6651611328125 -8.90478695845279\\
53.7261962890625 -8.95252803266832\\
%53.7872314453125 -8.99213484064595\\
%53.8482666015625 -9.03399917638669\\
53.9093017578125 -9.04168106825576\\
%53.9703369140625 -9.09371998252849\\
%54.0313720703125 -9.11536610782987\\
54.0924072265625 -9.09392217436938\\
%54.1534423828125 -9.09953242226452\\
%54.2144775390625 -9.14012443438211\\
54.2755126953125 -9.14800500942796\\
%54.3365478515625 -9.1606093861194\\
%54.3975830078125 -9.25855181440048\\
54.4586181640625 -9.25701008669411\\
%54.5196533203125 -9.30497253589041\\
%54.5806884765625 -9.30491814969912\\
54.6417236328125 -9.34122929908019\\
%54.7027587890625 -9.34063697310131\\
%54.7637939453125 -9.40638321295538\\
54.8248291015625 -9.36444835019026\\
%54.8858642578125 -9.46498953301151\\
%54.9468994140625 -9.47504641611368\\
55.0079345703125 -9.46878819266154\\
%55.0689697265625 -9.45440908176551\\
%55.1300048828125 -9.51945447657099\\
55.1910400390625 -9.51563485677975\\
%55.2520751953125 -9.5489731291419\\
%55.3131103515625 -9.60593586396724\\
55.3741455078125 -9.59517856581488\\
%55.4351806640625 -9.64908040471533\\
%55.4962158203125 -9.65094901550619\\
55.5572509765625 -9.69121927257583\\
%55.6182861328125 -9.67557185179926\\
%55.6793212890625 -9.70941992627443\\
55.7403564453125 -9.7372828307854\\
%55.8013916015625 -9.76913792178392\\
%55.8624267578125 -9.7902673010607\\
55.9234619140625 -9.83956151693312\\
%55.9844970703125 -9.85960314620167\\
%56.0455322265625 -9.87654569529828\\
56.1065673828125 -9.93968187032311\\
%56.1676025390625 -9.92890422700809\\
%56.2286376953125 -9.98446967807703\\
56.2896728515625 -9.955462514943\\
%56.3507080078125 -10.012552890617\\
%56.4117431640625 -10.0618415236039\\
56.4727783203125 -10.0507809282571\\
%56.5338134765625 -10.0940627101205\\
%56.5948486328125 -10.0984895751275\\
56.6558837890625 -10.1544276467195\\
%56.7169189453125 -10.171903880465\\
%56.7779541015625 -10.2355222652564\\
56.8389892578125 -10.2068702494349\\
%56.9000244140625 -10.2693367754215\\
%56.9610595703125 -10.3154306143252\\
57.0220947265625 -10.2927484809824\\
%57.0831298828125 -10.312085836911\\
%57.1441650390625 -10.3549282052644\\
57.2052001953125 -10.3745158199469\\
%57.2662353515625 -10.439385874878\\
%57.3272705078125 -10.4394168985537\\
57.3883056640625 -10.4939510600403\\
%57.4493408203125 -10.5101612719738\\
%57.5103759765625 -10.505889704895\\
57.5714111328125 -10.5685050562969\\
%57.6324462890625 -10.5692542773107\\
%57.6934814453125 -10.5760304974487\\
57.7545166015625 -10.6592851723186\\
%57.8155517578125 -10.6197235088793\\
%57.8765869140625 -10.6574033679282\\
57.9376220703125 -10.7205220391258\\
%57.9986572265625 -10.6960076773344\\
%58.0596923828125 -10.7215988628868\\
58.1207275390625 -10.772789474246\\
%58.1817626953125 -10.7864468180075\\
%58.2427978515625 -10.868510971214\\
58.3038330078125 -10.8399968181344\\
%58.3648681640625 -10.9197498210971\\
%58.4259033203125 -10.9139521108405\\
58.4869384765625 -10.9133890407843\\
%58.5479736328125 -10.9754145073738\\
%58.6090087890625 -10.9854380064543\\
58.6700439453125 -10.9632838064637\\
%58.7310791015625 -11.0358657861417\\
%58.7921142578125 -11.044133645003\\
58.8531494140625 -11.0845714854272\\
%58.9141845703125 -11.1059943308009\\
%58.9752197265625 -11.1574692702707\\
59.0362548828125 -11.1463454804538\\
%59.0972900390625 -11.1682094163888\\
%59.1583251953125 -11.2441224030293\\
59.2193603515625 -11.2559479774924\\
%59.2803955078125 -11.2795930774084\\
%59.3414306640625 -11.2803568885256\\
59.4024658203125 -11.3068830680188\\
%59.4635009765625 -11.3165093142731\\
%59.5245361328125 -11.3699726596696\\
59.5855712890625 -11.3994378731165\\
%59.6466064453125 -11.4177984127663\\
%59.7076416015625 -11.4577033995584\\
59.7686767578125 -11.5007859366677\\
%59.8297119140625 -11.5374598537909\\
%59.8907470703125 -11.536250412291\\
59.9517822265625 -11.5601224356575\\
%60.0128173828125 -11.5169949396038\\
%60.0738525390625 -11.6037755770789\\
60.1348876953125 -11.5981444508992\\
%60.1959228515625 -11.6185287543068\\
%60.2569580078125 -11.6672254544207\\
60.3179931640625 -11.6529155999607\\
%60.3790283203125 -11.7228371273263\\
%60.4400634765625 -11.7894020398951\\
60.5010986328125 -11.7854177876579\\
%60.5621337890625 -11.7659912342468\\
%60.6231689453125 -11.8320725989181\\
60.6842041015625 -11.8139438113465\\
%60.7452392578125 -11.8562433159781\\
%60.8062744140625 -11.9142326824721\\
60.8673095703125 -11.9243072475527\\
%60.9283447265625 -11.9462237412487\\
%60.9893798828125 -11.9809458227132\\
61.0504150390625 -12.0435286365136\\
%61.1114501953125 -12.0343772357654\\
%61.1724853515625 -12.0428492268991\\
61.2335205078125 -12.0842857909031\\
%61.2945556640625 -12.0592677578686\\
%61.3555908203125 -12.1186219414358\\
61.4166259765625 -12.1524522203279\\
%61.4776611328125 -12.1640233446789\\
%61.5386962890625 -12.2229025172429\\
61.5997314453125 -12.25046959965\\
%61.6607666015625 -12.2746144099865\\
%61.7218017578125 -12.2878205612863\\
61.7828369140625 -12.2945819972871\\
%61.8438720703125 -12.3099399573349\\
%61.9049072265625 -12.3563381900427\\
61.9659423828125 -12.3899669939789\\
%62.0269775390625 -12.3769923050927\\
%62.0880126953125 -12.4118683923852\\
62.1490478515625 -12.4540811556914\\
%62.2100830078125 -12.4526257607918\\
%62.2711181640625 -12.5010592969381\\
62.3321533203125 -12.5311547060251\\
%62.3931884765625 -12.5283897973103\\
%62.4542236328125 -12.5741552311871\\
62.5152587890625 -12.5589809290994\\
%62.5762939453125 -12.5754431642938\\
%62.6373291015625 -12.6333624447214\\
62.6983642578125 -12.6850470243711\\
%62.7593994140625 -12.6241534838894\\
%62.8204345703125 -12.7263803457473\\
62.8814697265625 -12.6953838192737\\
%62.9425048828125 -12.7188488862201\\
%63.0035400390625 -12.7410632311824\\
63.0645751953125 -12.7720298858634\\
%63.1256103515625 -12.8256687616862\\
%63.1866455078125 -12.8127702933126\\
63.2476806640625 -12.8554084726953\\
%63.3087158203125 -12.888820213454\\
%63.3697509765625 -12.9470193318275\\
63.4307861328125 -12.9151513575753\\
%63.4918212890625 -13.0021469190442\\
%63.5528564453125 -12.9475553231373\\
63.6138916015625 -12.9814281394868\\
%63.6749267578125 -13.0978898931427\\
%63.7359619140625 -13.0827997308752\\
63.7969970703125 -13.0968614341157\\
%63.8580322265625 -13.0327623877902\\
%63.9190673828125 -13.0891050196993\\
63.9801025390625 -13.1132347017234\\
%64.0411376953125 -13.1812019758288\\
%64.1021728515625 -13.2243685119192\\
64.1632080078125 -13.2042686052677\\
%64.2242431640625 -13.2114507619265\\
%64.2852783203125 -13.3032153866244\\
64.3463134765625 -13.2715386918235\\
%64.4073486328125 -13.2545534093261\\
%64.4683837890625 -13.3218223136579\\
64.5294189453125 -13.3536108120254\\
%64.5904541015625 -13.3666544021428\\
%64.6514892578125 -13.3853345350337\\
64.7125244140625 -13.419703769347\\
%64.7735595703125 -13.4721542446079\\
%64.8345947265625 -13.4868284647911\\
64.8956298828125 -13.5327470914735\\
%64.9566650390625 -13.510890338924\\
%65.0177001953125 -13.5122767499111\\
65.0787353515625 -13.5415953641615\\
%65.1397705078125 -13.5994277643114\\
%65.2008056640625 -13.6257004755373\\
65.2618408203125 -13.6054429124913\\
%65.3228759765625 -13.6704946953024\\
%65.3839111328125 -13.7173674450487\\
65.4449462890625 -13.6980137571821\\
%65.5059814453125 -13.677933890959\\
%65.5670166015625 -13.7312962070013\\
65.6280517578125 -13.7459247537571\\
%65.6890869140625 -13.7570437325842\\
%65.7501220703125 -13.8263716664685\\
65.8111572265625 -13.8060655194341\\
%65.8721923828125 -13.8406987306461\\
%65.9332275390625 -13.8379138146463\\
65.9942626953125 -13.8966860233668\\
%66.0552978515625 -13.9305157733593\\
%66.1163330078125 -13.9352131681383\\
66.1773681640625 -13.9636108367881\\
%66.2384033203125 -13.9779814618575\\
%66.2994384765625 -14.0121768202415\\
66.3604736328125 -14.0279658190516\\
%66.4215087890625 -14.0582735751864\\
%66.4825439453125 -14.0574924953383\\
66.5435791015625 -14.1125425428678\\
%66.6046142578125 -14.1774985201545\\
%66.6656494140625 -14.1699086124288\\
66.7266845703125 -14.2021571182863\\
%66.7877197265625 -14.1897518567908\\
%66.8487548828125 -14.2272682114828\\
66.9097900390625 -14.2389359146257\\
%66.9708251953125 -14.2949128590898\\
%67.0318603515625 -14.3019609549713\\
67.0928955078125 -14.3147425486605\\
%67.1539306640625 -14.3595438616169\\
%67.2149658203125 -14.3888987545815\\
67.2760009765625 -14.3611538204197\\
%67.3370361328125 -14.4236357403425\\
%67.3980712890625 -14.4416929097498\\
67.4591064453125 -14.4734111180552\\
%67.5201416015625 -14.5051069892063\\
%67.5811767578125 -14.4909410080875\\
67.6422119140625 -14.54545853361\\
%67.7032470703125 -14.5667563637605\\
%67.7642822265625 -14.6007295441478\\
67.8253173828125 -14.6008814892217\\
%67.8863525390625 -14.6218532356757\\
%67.9473876953125 -14.6510115222546\\
68.0084228515625 -14.6238112086401\\
%68.0694580078125 -14.6694734221255\\
%68.1304931640625 -14.6854191847177\\
68.1915283203125 -14.7297001234232\\
%68.2525634765625 -14.7164443623941\\
%68.3135986328125 -14.7945151158265\\
68.3746337890625 -14.8025444599365\\
%68.4356689453125 -14.8108161105473\\
%68.4967041015625 -14.8502223883938\\
68.5577392578125 -14.8542257118788\\
%68.6187744140625 -14.886102615165\\
%68.6798095703125 -14.9518264735699\\
68.7408447265625 -14.9942588525251\\
%68.8018798828125 -14.9767425406357\\
%68.8629150390625 -14.9610697340473\\
68.9239501953125 -14.9976589233365\\
%68.9849853515625 -15.0811131590394\\
%69.0460205078125 -15.0274542750821\\
69.1070556640625 -15.0852054923812\\
%69.1680908203125 -15.1761707656818\\
%69.2291259765625 -15.1752049963567\\
69.2901611328125 -15.1499774572912\\
%69.3511962890625 -15.2417385916677\\
%69.4122314453125 -15.1976551369894\\
69.4732666015625 -15.2064570114508\\
%69.5343017578125 -15.2045214951526\\
%69.5953369140625 -15.2404337937157\\
69.6563720703125 -15.2224541843996\\
%69.7174072265625 -15.3425319665791\\
%69.7784423828125 -15.4270319947108\\
69.8394775390625 -15.4299770924857\\
%69.9005126953125 -15.4383498238554\\
%69.9615478515625 -15.4412012722038\\
70.0225830078125 -15.449171125778\\
%70.0836181640625 -15.4610965859433\\
%70.1446533203125 -15.5104785933412\\
70.2056884765625 -15.5414442076736\\
%70.2667236328125 -15.5302412554015\\
%70.3277587890625 -15.6162629690078\\
70.3887939453125 -15.6366299691829\\
%70.4498291015625 -15.6320022493318\\
%70.5108642578125 -15.6517546843089\\
70.5718994140625 -15.6682567758011\\
%70.6329345703125 -15.7237135632898\\
%70.6939697265625 -15.8096337578357\\
70.7550048828125 -15.7085217638921\\
%70.8160400390625 -15.8351030794132\\
%70.8770751953125 -15.8176682216133\\
70.9381103515625 -15.8069911848308\\
%70.9991455078125 -15.8949339193367\\
%71.0601806640625 -15.908250159432\\
71.1212158203125 -15.973008771164\\
%71.1822509765625 -15.9273030627756\\
%71.2432861328125 -15.9347655880577\\
71.3043212890625 -16.0229498509179\\
%71.3653564453125 -15.9430718597261\\
%71.4263916015625 -16.0266667543736\\
71.4874267578125 -16.0154152042702\\
%71.5484619140625 -16.1420886526391\\
%71.6094970703125 -16.1153398675839\\
71.6705322265625 -16.1271060757091\\
%71.7315673828125 -16.2048696693816\\
%71.7926025390625 -16.1671057034978\\
71.8536376953125 -16.2454070865529\\
%71.9146728515625 -16.2087352883982\\
%71.9757080078125 -16.2814130924097\\
72.0367431640625 -16.3280237493137\\
%72.0977783203125 -16.360672251407\\
%72.1588134765625 -16.3469036793872\\
72.2198486328125 -16.4036058264306\\
%72.2808837890625 -16.300979173618\\
%72.3419189453125 -16.3692828389783\\
72.4029541015625 -16.5324325149134\\
%72.4639892578125 -16.4403682970459\\
%72.5250244140625 -16.4322638494322\\
72.5860595703125 -16.5114770264246\\
%72.6470947265625 -16.5119571211423\\
%72.7081298828125 -16.4623084236543\\
72.7691650390625 -16.5444499188877\\
%72.8302001953125 -16.6652971699228\\
%72.8912353515625 -16.5175622085944\\
72.9522705078125 -16.6737614982085\\
%73.0133056640625 -16.679646922677\\
%73.0743408203125 -16.6894147636755\\
73.1353759765625 -16.6848952084099\\
%73.1964111328125 -16.7461916431429\\
%73.2574462890625 -16.8238831412145\\
73.3184814453125 -16.8507241502812\\
%73.3795166015625 -16.8407261190073\\
%73.4405517578125 -16.9314612888371\\
73.5015869140625 -16.8942224542563\\
%73.5626220703125 -16.9754837826466\\
%73.6236572265625 -16.9965679098883\\
73.6846923828125 -16.9407316226191\\
%73.7457275390625 -17.0530006385554\\
%73.8067626953125 -16.9082461665723\\
73.8677978515625 -16.9821320696948\\
%73.9288330078125 -17.024209398858\\
%73.9898681640625 -17.1119951667898\\
74.0509033203125 -17.0889209263292\\
%74.1119384765625 -17.1138716502372\\
%74.1729736328125 -17.1581511747088\\
74.2340087890625 -17.165829446682\\
%74.2950439453125 -17.2304931728437\\
%74.3560791015625 -17.1700299126996\\
74.4171142578125 -17.2782345823444\\
%74.4781494140625 -17.2876726886792\\
%74.5391845703125 -17.261937541897\\
74.6002197265625 -17.3989384471464\\
%74.6612548828125 -17.3041314598977\\
%74.7222900390625 -17.3696707104001\\
74.7833251953125 -17.3919511735616\\
%74.8443603515625 -17.443732807571\\
%74.9053955078125 -17.4615778514986\\
74.9664306640625 -17.520394489605\\
%75.0274658203125 -17.4559220693779\\
%75.0885009765625 -17.5479117838191\\
75.1495361328125 -17.5369575359414\\
%75.2105712890625 -17.6843942499287\\
%75.2716064453125 -17.523789164617\\
75.3326416015625 -17.5878895196822\\
%75.3936767578125 -17.597072161749\\
%75.4547119140625 -17.7344845646983\\
75.5157470703125 -17.7489549248355\\
%75.5767822265625 -17.7610710585272\\
%75.6378173828125 -17.7799289089085\\
75.6988525390625 -17.7768122139778\\
%75.7598876953125 -17.8270822817341\\
%75.8209228515625 -17.7924969356866\\
75.8819580078125 -17.9165908213448\\
%75.9429931640625 -17.8893830649555\\
%76.0040283203125 -17.9850199425466\\
76.0650634765625 -17.8726157522537\\
%76.1260986328125 -17.9504383717022\\
%76.1871337890625 -17.9982281142084\\
76.2481689453125 -18.017663977733\\
%76.3092041015625 -18.0525024420896\\
%76.4007568359375 -18.0942138491169\\
76.4923095703125 -18.1393069261011\\
%76.5838623046875 -18.0456442512129\\
%76.6754150390625 -18.2168090358713\\
76.7669677734375 -18.2224309811719\\
%76.8585205078125 -18.2929880194634\\
%76.9500732421875 -18.2846944549195\\
77.0416259765625 -18.3890177356244\\
%77.1331787109375 -18.3896059210689\\
%77.2247314453125 -18.4041387211202\\
77.3162841796875 -18.5009491625371\\
%77.4078369140625 -18.5079196228826\\
%77.4993896484375 -18.5701158925095\\
77.5909423828125 -18.5483510770482\\
%77.6824951171875 -18.6564754578213\\
%77.7740478515625 -18.6464890901182\\
77.8656005859375 -18.7256442225257\\
%77.9571533203125 -18.769242682008\\
%78.0487060546875 -18.8259408039934\\
78.1402587890625 -18.8915477939503\\
%78.2318115234375 -18.9060437116539\\
%78.3233642578125 -18.9047513849646\\
78.4149169921875 -18.9910833348909\\
%78.5064697265625 -19.0044997242141\\
%78.5980224609375 -18.9776850152386\\
78.6895751953125 -19.1027887743652\\
%78.7811279296875 -19.1451083299484\\
%78.8726806640625 -19.1729510995815\\
78.9642333984375 -19.1396777651384\\
%79.0557861328125 -19.2748480300548\\
%79.1473388671875 -19.2516339253071\\
79.2388916015625 -19.3310219596545\\
%79.3304443359375 -19.4239354350958\\
%79.4219970703125 -19.4296416006328\\
79.5135498046875 -19.4938738666095\\
%79.6051025390625 -19.5056570490519\\
%79.6966552734375 -19.5962821543702\\
79.7882080078125 -19.5774713560115\\
%79.8797607421875 -19.6912001214547\\
%79.9713134765625 -19.6581530992543\\
80.0628662109375 -19.7123528352076\\
%80.1544189453125 -19.7225635764653\\
%80.2459716796875 -19.766652579038\\
80.3375244140625 -19.9545466030227\\
%80.4290771484375 -19.9095038077326\\
%80.5206298828125 -19.824536588094\\
80.6121826171875 -20.0058635900298\\
%80.7037353515625 -20.0564769428559\\
%80.7952880859375 -20.0613344488332\\
80.8868408203125 -20.1489100038328\\
%80.9783935546875 -20.1582228149502\\
%81.0699462890625 -20.1021884414046\\
81.1614990234375 -20.2072021467139\\
%81.2530517578125 -20.4136987343847\\
%81.3446044921875 -20.4165475546192\\
81.4361572265625 -20.2948388190719\\
%81.5277099609375 -20.1504886415549\\
%81.6192626953125 -20.4413768782134\\
81.7108154296875 -20.4297581469853\\
%81.8023681640625 -20.5678450875406\\
%81.8939208984375 -20.5869011055457\\
81.9854736328125 -20.5853048001917\\
%82.0770263671875 -20.5867726492506\\
%82.1685791015625 -20.6152381514243\\
82.2601318359375 -20.7310024441005\\
%82.3516845703125 -20.7747541898363\\
%82.4432373046875 -20.7535776878233\\
82.5347900390625 -20.9649696710654\\
%82.6263427734375 -21.05352724788\\
%82.7178955078125 -21.0476793334963\\
82.8094482421875 -20.8899407832769\\
%82.9010009765625 -21.0512913693216\\
%82.9925537109375 -21.1251817823418\\
83.0841064453125 -21.058853262517\\
%83.1756591796875 -21.2331332100911\\
%83.2672119140625 -21.2516967353354\\
83.3587646484375 -21.1426008231532\\
%83.4503173828125 -21.2947416663434\\
%83.5418701171875 -21.3813922693399\\
83.6334228515625 -21.3043635294474\\
%83.7249755859375 -21.4995894292177\\
%83.8165283203125 -21.5380268361824\\
83.9080810546875 -21.2897283802063\\
%83.9996337890625 -21.664859173867\\
%84.0911865234375 -21.6410922685031\\
84.1827392578125 -21.7000824809461\\
%84.2742919921875 -21.7824893581702\\
%84.3658447265625 -21.8386031943835\\
84.4573974609375 -21.7370057578467\\
%84.5489501953125 -21.7599520499548\\
%84.6405029296875 -21.8919597936216\\
84.7320556640625 -22.035162859427\\
%84.8236083984375 -22.1294795107296\\
%84.9151611328125 -22.0440762532882\\
85.0067138671875 -22.0235879660572\\
%85.0982666015625 -22.0378083933983\\
%85.1898193359375 -22.3238140119167\\
85.2813720703125 -22.2715974363208\\
%85.3729248046875 -22.2636785262674\\
%85.4644775390625 -22.2465462497895\\
85.5560302734375 -22.4547037016568\\
%85.6475830078125 -22.4200547239317\\
%85.7391357421875 -22.5346346364813\\
85.8306884765625 -22.571552658451\\
%85.9222412109375 -22.5940642805844\\
%86.0137939453125 -22.5999601265838\\
86.1053466796875 -22.8192127507097\\
%86.1968994140625 -22.755235209793\\
%86.2884521484375 -22.8227758666445\\
86.3800048828125 -22.8419831501171\\
%86.4715576171875 -22.8118585079928\\
%86.5631103515625 -23.0690698356112\\
86.6546630859375 -23.1494000019891\\
%86.7462158203125 -23.1525764278814\\
%86.8377685546875 -23.2649909109809\\
86.9293212890625 -23.2653756364754\\
%87.0208740234375 -23.4046881968727\\
%87.1124267578125 -23.4261416193397\\
87.2039794921875 -23.4453575664576\\
%87.2955322265625 -23.5650997330897\\
%87.3870849609375 -23.4570855283533\\
87.4786376953125 -23.6907125183079\\
%87.5701904296875 -23.5738703170646\\
%87.6617431640625 -23.6599971387688\\
87.7532958984375 -23.8497382693922\\
%87.8448486328125 -23.8968093848577\\
%87.9364013671875 -23.9102712392606\\
88.0279541015625 -24.0158355194331\\
%88.1195068359375 -24.0257002498431\\
%88.2110595703125 -24.170776005379\\
88.3026123046875 -24.2330706421673\\
%88.3941650390625 -24.3404584583034\\
%88.4857177734375 -24.2895842144132\\
88.5772705078125 -24.4877882445659\\
%88.6688232421875 -24.3949166682734\\
%88.7603759765625 -24.5252604941934\\
88.8519287109375 -24.5366158254849\\
%88.9434814453125 -24.6200709923611\\
%89.0350341796875 -24.5898927016275\\
89.1265869140625 -24.830903528203\\
%89.2181396484375 -24.946443092927\\
%89.3096923828125 -24.9997717432998\\
89.4012451171875 -24.9486479955319\\
%89.4927978515625 -24.9298020424853\\
%89.5843505859375 -25.194832323995\\
89.6759033203125 -24.9943096725111\\
%89.7674560546875 -25.3114413242483\\
%89.8590087890625 -25.3767055836973\\
89.9505615234375 -25.3421886572399\\
%90.0421142578125 -25.4131893441157\\
%90.1336669921875 -25.5788813262979\\
90.2252197265625 -25.6917854934108\\
%90.3167724609375 -25.7188409359714\\
%90.4083251953125 -25.6244775467225\\
90.4998779296875 -25.6770232492169\\
%90.5914306640625 -25.7727593907812\\
%90.6829833984375 -25.8969857340928\\
90.7745361328125 -26.2134893202893\\
%90.8660888671875 -26.2072256246354\\
%90.9576416015625 -26.2343785967398\\
91.0491943359375 -26.2092505893682\\
%91.1407470703125 -26.3868011820137\\
%91.2322998046875 -26.5953192039193\\
91.3238525390625 -26.5192324783872\\
%91.4154052734375 -26.8220928656774\\
%91.5069580078125 -26.7499557889668\\
91.5985107421875 -26.8935982811566\\
%91.6900634765625 -27.136670903989\\
%91.7816162109375 -27.2348373646462\\
91.8731689453125 -26.9763236746688\\
%91.9647216796875 -27.1087143546475\\
%92.0562744140625 -27.4640865502317\\
92.1478271484375 -27.3794070667734\\
%92.2393798828125 -27.764705667705\\
%92.3309326171875 -27.7070785302718\\
92.4224853515625 -27.8693025899681\\
%92.5140380859375 -27.8466557379734\\
%92.6055908203125 -28.1610873271698\\
92.6971435546875 -28.301798683708\\
%92.7886962890625 -28.21127454317\\
%92.8802490234375 -28.5816272816587\\
92.9718017578125 -28.6018431951489\\
%93.0633544921875 -28.8416802512314\\
%93.1549072265625 -28.7657198326233\\
93.2464599609375 -29.1173881066602\\
%93.3380126953125 -29.3584928208285\\
%93.4295654296875 -29.3396511659088\\
93.5211181640625 -29.6130258710621\\
%93.6126708984375 -29.9828911186241\\
%93.7042236328125 -29.9989594851298\\
93.7957763671875 -30.4514931090316\\
%93.8873291015625 -30.1782073685533\\
%93.9788818359375 -30.4659238484982\\
94.0704345703125 -30.7428728850737\\
%94.1619873046875 -30.7742702075848\\
%94.2535400390625 -31.1752620211899\\
94.3450927734375 -30.7726857411879\\
%94.4366455078125 -30.983405278174\\
%94.5281982421875 -31.3028382305694\\
94.6197509765625 -31.3413002652616\\
%94.7113037109375 -31.7793248548636\\
%94.8028564453125 -31.9705163024747\\
94.8944091796875 -32.2528961074409\\
%94.9859619140625 -31.6651045716623\\
%95.0775146484375 -32.382382964021\\
95.1690673828125 -32.6618661490405\\
%95.2606201171875 -32.6836134889453\\
%95.3521728515625 -32.9804041837111\\
95.4437255859375 -33.2974147846787\\
%95.5352783203125 -33.2921283386748\\
%95.6268310546875 -33.2592124773748\\
95.7183837890625 -33.3980695447256\\
%95.8099365234375 -33.8564070941942\\
%95.9014892578125 -34.2245241684666\\
95.9930419921875 -34.3765191171812\\
%96.0845947265625 -34.2842297124814\\
%96.1761474609375 -34.3736975551448\\
96.2677001953125 -34.8398237554349\\
%96.3592529296875 -35.0959767809189\\
%96.4508056640625 -35.1818487311093\\
96.5423583984375 -35.8399591245196\\
%96.6339111328125 -35.3902622955579\\
%96.7254638671875 -35.641130085096\\
96.8170166015625 -35.8842358166414\\
%96.9085693359375 -35.9015336320278\\
%97.0001220703125 -36.4244694794279\\
97.0916748046875 -36.9036631082288\\
%97.1832275390625 -37.5787611791238\\
%97.2747802734375 -37.0060929810511\\
97.3663330078125 -37.1913234065788\\
%97.4578857421875 -38.1092590187438\\
%97.5494384765625 -38.121486110957\\
97.6409912109375 -37.9142276831394\\
%97.7325439453125 -37.818937667501\\
%97.8240966796875 -38.5705583967549\\
97.9156494140625 -39.1915617556346\\
%98.0072021484375 -40.2691563822524\\
%98.0987548828125 -39.6995749242718\\
98.1903076171875 -40.3059002925922\\
%98.2818603515625 -40.3865194794237\\
%98.3734130859375 -40.9265795049074\\
98.4649658203125 -40.7688570157437\\
98.5565185546875 -41.3590685907094\\
98.6480712890625 -42.1912571061631\\
%98.7396240234375 -42.0221211857989\\
%98.8311767578125 -42.9123959419882\\
98.9227294921875 -42.6150903711019\\
99.0142822265625 -44.070934439817\\
%99.1058349609375 -44.1413555944404\\
%99.1973876953125 -44.902547979365\\
99.2889404296875 -45.0363845819724\\
99.3804931640625 -45.4720830304904\\
99.4720458984375 -45.1095487125202\\
99.5635986328125 -46.6442524505795\\
99.6551513671875 -46.2758055375091\\
99.7467041015625 -47.8080459942796\\
99.8382568359375 -47.307091809712\\
99.9298095703125 -43.7439876561534\\
100.021362304688 -47.5801267005078\\
100.112915039062 -47.1841379967276\\
100.204467773438 -48.5229012865204\\
100.296020507813 -50.8995175850146\\
100.387573242188 -49.273036586417\\
100.479125976562 -48.5531999884141\\
%100.570678710938 -48.1132679467942\\
%100.662231445313 -48.5252542225353\\
100.753784179688 -47.764902209173\\
100.845336914062 -48.7087725082902\\
100.936889648438 -47.627882551618\\
101.028442382812 -47.9654395203611\\
101.119995117188 -48.3421393085776\\
101.211547851562 -45.7380919181091\\
101.303100585938 -44.4712790541782\\
101.394653320312 -43.7911646944471\\
101.486206054688 -42.2780907670599\\
};
\addplot [frf, forget plot]
table[row sep=crcr]{
101.486206054688 -42.2780907670599\\
101.577758789062 -41.7829973016396\\
101.669311523438 -40.6645875218498\\
%101.760864257812 -40.7391173106072\\
%101.852416992188 -40.1711685595811\\
101.943969726562 -40.1039602628311\\
102.035522460938 -39.1715299385658\\
102.127075195312 -39.2051920168374\\
102.218627929688 -39.3328190865274\\
102.310180664062 -38.5951167901444\\
102.401733398438 -39.7643690934852\\
102.493286132812 -38.8183779095672\\
102.584838867188 -39.3910394400033\\
102.676391601562 -39.0148260755834\\
102.767944335938 -39.4770500766635\\
102.859497070312 -39.321354098696\\
102.951049804688 -39.0118866675595\\
103.042602539062 -38.1621634906942\\
103.134155273438 -38.9315597843628\\
103.225708007812 -37.940979728496\\
103.317260742188 -38.619378450919\\
103.408813476562 -38.1830804745067\\
103.500366210938 -37.9156325183595\\
103.591918945312 -38.2930731816563\\
103.683471679688 -38.0330839607535\\
103.775024414062 -37.9954502834199\\
%103.866577148438 -37.7833009735285\\
%103.958129882812 -37.4223343255985\\
104.049682617188 -37.4368183905335\\
104.141235351562 -37.905162791775\\
104.232788085938 -37.3680168968547\\
104.324340820312 -37.2198627130019\\
104.415893554688 -36.9560226580893\\
%104.507446289063 -36.4044664709013\\
%104.598999023438 -36.2899170230103\\
104.690551757812 -36.0334786769109\\
104.782104492188 -35.8290272525344\\
%104.873657226563 -35.6026514542659\\
%104.965209960938 -35.3174132333057\\
105.056762695312 -34.9017328139108\\
%105.148315429688 -34.8200457924378\\
%105.239868164063 -34.8236862109468\\
105.331420898438 -34.2955808698534\\
105.422973632812 -33.8839976937884\\
105.514526367188 -33.690041223739\\
105.606079101563 -34.0974656392539\\
%105.697631835938 -33.0757074142388\\
%105.789184570312 -32.8275148044954\\
105.880737304688 -33.0721091164746\\
105.972290039063 -32.8857229187687\\
106.063842773438 -32.7585588324104\\
106.155395507812 -32.853720646422\\
106.246948242188 -33.440784732015\\
106.338500976563 -33.5156655507959\\
106.430053710938 -34.1259074069794\\
106.521606445312 -34.0650200335345\\
106.613159179688 -35.0398098376744\\
106.704711914063 -35.6645470310255\\
106.796264648438 -35.9289105357576\\
106.887817382812 -36.4701407871664\\
107.009887695312 -36.8681878678345\\
107.131958007812 -37.8705454519924\\
107.254028320312 -38.7386915291778\\
107.376098632812 -39.5229460461523\\
107.498168945312 -40.399936207506\\
107.620239257812 -40.7069964794382\\
107.742309570312 -42.6078536499371\\
107.864379882812 -44.1140844048069\\
107.986450195312 -47.6304417208891\\
108.108520507812 -50.4661186782128\\
108.230590820312 -57.6869454793876\\
108.352661132812 -51.2513118162582\\
108.474731445312 -47.5618878737456\\
108.596801757812 -49.6888310885746\\
108.718872070312 -44.5320866448378\\
108.840942382812 -45.0595333825582\\
108.963012695312 -40.8199563665332\\
109.085083007812 -41.669548709301\\
109.207153320312 -42.4369532960293\\
109.329223632812 -41.953416849848\\
109.451293945312 -41.7228928350391\\
109.573364257812 -40.7421281033895\\
109.695434570312 -39.6647252328008\\
109.817504882812 -39.2466585265362\\
109.939575195312 -39.1709754770111\\
110.061645507812 -40.1984837383409\\
110.183715820312 -38.8423555619884\\
%110.305786132812 -38.8673008745593\\
%110.427856445312 -38.5574582278816\\
110.549926757812 -37.9478908087079\\
110.671997070312 -38.4357317682542\\
110.794067382812 -38.8339900976833\\
110.916137695312 -37.9549534129168\\
111.038208007813 -39.1557671622465\\
111.160278320313 -40.8659814785563\\
111.282348632813 -39.3681280078005\\
111.404418945313 -40.683602023162\\
111.526489257813 -38.9484756900392\\
111.648559570313 -40.2401641337011\\
111.770629882813 -40.4506174172274\\
111.892700195313 -39.2557526387987\\
112.014770507813 -40.3722650910032\\
112.136840820313 -40.4320901136531\\
112.258911132813 -40.6438003710726\\
112.380981445313 -40.5485851009497\\
112.503051757813 -40.5925369495214\\
112.625122070313 -42.2363262138457\\
112.747192382813 -40.2275774683729\\
112.869262695313 -40.5864783173736\\
112.991333007813 -44.6913203833102\\
113.113403320313 -40.4150452628884\\
113.235473632813 -40.3258623650782\\
113.357543945313 -40.8429570640684\\
113.479614257813 -38.3016831488341\\
113.601684570313 -38.5126698627283\\
113.723754882813 -36.8726664674013\\
113.845825195313 -34.7654201728121\\
113.967895507813 -33.5082273045055\\
114.089965820313 -31.4921058153307\\
114.212036132813 -29.5799663517495\\
114.334106445312 -26.9361254683411\\
114.456176757812 -24.9114449894108\\
114.578247070312 -25.4439708191908\\
114.700317382812 -30.3920479948244\\
114.822387695312 -29.963200336238\\
114.944458007812 -27.9885450924127\\
115.066528320312 -26.0973352421914\\
115.188598632812 -25.6960849576628\\
115.310668945312 -28.7446322025427\\
115.432739257812 -30.8440443577247\\
115.554809570312 -28.0112900707028\\
115.676879882812 -23.8067211338438\\
115.798950195312 -21.0033610393481\\
115.921020507812 -19.6905947410673\\
116.043090820312 -21.6926181096474\\
116.165161132812 -24.9913417034372\\
116.287231445312 -24.114481604605\\
116.409301757812 -22.8694016059198\\
116.531372070312 -24.244909924929\\
116.653442382812 -26.8530680237954\\
116.775512695312 -30.2510646657591\\
116.897583007812 -33.6950444840572\\
117.019653320312 -32.6458506271973\\
117.141723632812 -30.3353014784935\\
117.263793945312 -27.1597723990789\\
117.385864257812 -26.0979128365419\\
117.507934570312 -24.3584737266434\\
117.630004882812 -23.6828610640593\\
117.752075195312 -22.8264483616331\\
117.874145507812 -22.2553435177411\\
117.996215820312 -21.8720385465583\\
118.118286132812 -22.0392170712773\\
118.240356445312 -21.8983641262122\\
118.362426757812 -21.8729795588636\\
118.484497070312 -22.0036989092015\\
118.606567382812 -22.5557673431019\\
118.728637695312 -22.7741595568295\\
118.850708007812 -23.4587612163293\\
118.972778320312 -24.7204923620762\\
119.094848632812 -24.3010596629572\\
119.216918945312 -25.9058856569473\\
119.338989257812 -25.1767091999351\\
119.461059570312 -25.464242353037\\
119.583129882812 -26.054286896841\\
119.705200195312 -26.0977072061228\\
119.827270507812 -24.7010983570876\\
119.949340820312 -23.5408831518366\\
120.071411132812 -21.7418944059696\\
120.193481445312 -20.7501362763974\\
120.315551757812 -17.8530348930117\\
120.437622070312 -16.0757397946036\\
120.559692382812 -13.7645434837689\\
120.681762695312 -12.3817397732108\\
120.803833007812 -10.2951788676741\\
120.925903320312 -7.68998444195356\\
121.047973632812 -4.80288039112424\\
121.170043945312 -1.89029101233875\\
121.292114257812 3.54605529323954\\
121.414184570312 7.51245043070171\\
121.536254882812 4.95725984809332\\
121.658325195312 1.74029206547261\\
121.780395507812 -0.887820485249823\\
121.902465820312 -2.43364115285063\\
122.024536132812 -3.8449241828969\\
122.146606445312 -4.95795230785779\\
122.268676757812 -5.67332000256312\\
122.390747070312 -6.21665598508104\\
122.512817382812 -6.95579891511871\\
122.634887695312 -7.34254219108147\\
122.756958007812 -7.63962776575099\\
122.879028320312 -7.95638189696175\\
123.001098632812 -8.43452993083917\\
123.123168945312 -8.48015120255986\\
123.245239257812 -8.82283033790945\\
123.367309570312 -8.94047979778003\\
123.489379882812 -9.16586656245365\\
123.611450195312 -9.15484014891865\\
123.733520507812 -9.42679023301161\\
123.855590820312 -9.60155492177823\\
123.977661132812 -9.55812701588153\\
124.099731445312 -9.76238042284664\\
124.221801757813 -9.7347172024997\\
124.343872070313 -9.76967751323276\\
124.465942382813 -9.88602769318021\\
124.588012695313 -9.84778955792419\\
124.710083007813 -9.94292599393129\\
124.832153320313 -10.0345793838229\\
124.954223632813 -9.91204955874589\\
125.076293945313 -10.0000157170644\\
125.198364257813 -9.9855118650747\\
125.320434570313 -10.0753605127771\\
125.442504882813 -10.1348640448437\\
125.564575195313 -10.0440883450803\\
125.686645507813 -10.0432022268945\\
125.808715820313 -10.0497702627167\\
125.930786132813 -10.1555296284458\\
126.052856445313 -10.1234138640041\\
126.174926757813 -10.189141982722\\
126.296997070313 -10.2267906250164\\
126.419067382813 -10.1649408973119\\
126.541137695313 -10.1544690788814\\
126.663208007813 -10.1516839831207\\
126.785278320313 -10.1504445915592\\
126.907348632813 -10.0690743443163\\
127.029418945313 -10.0462441731955\\
127.151489257813 -9.93409302764704\\
127.273559570313 -9.94972633551906\\
127.395629882813 -9.7812522920877\\
127.517700195313 -9.81003235822811\\
127.639770507813 -9.64995820286699\\
127.761840820312 -9.59440857317835\\
127.883911132812 -9.45149171987751\\
128.005981445312 -9.43241736238214\\
128.128051757812 -9.35201933288747\\
128.250122070312 -9.32973491249406\\
128.372192382812 -9.29769296381522\\
128.494262695312 -9.23061400553738\\
128.616333007812 -9.12852413296575\\
128.738403320312 -8.84351532943759\\
128.860473632812 -8.41799713006989\\
128.982543945312 -7.68712737823197\\
129.104614257812 -6.65307347406332\\
129.226684570312 -5.7826665511067\\
129.348754882812 -5.26619308011801\\
129.470825195312 -4.93543428152317\\
129.592895507812 -4.80207704122277\\
129.714965820312 -4.69086412242673\\
129.837036132812 -4.55812131698859\\
129.959106445312 -4.41051186349532\\
130.081176757812 -4.24743778654654\\
130.203247070312 -4.06944372864525\\
130.325317382812 -3.90834824760076\\
130.447387695312 -3.73530091792142\\
130.569458007812 -3.54141801734723\\
130.691528320312 -3.35809268505873\\
130.813598632812 -3.10877988803991\\
130.935668945312 -2.86573785879335\\
131.057739257812 -2.60599763944334\\
131.179809570312 -2.37745150285269\\
131.301879882812 -2.1468426740816\\
131.423950195312 -1.88351747930868\\
131.546020507812 -1.57345783830601\\
131.668090820312 -1.33652380756001\\
131.790161132812 -0.997900089207924\\
131.912231445312 -0.653194451344676\\
132.034301757812 -0.317491611888009\\
132.156372070312 -0.0271949993599402\\
132.278442382812 0.294465412642127\\
132.400512695312 0.701133179920262\\
132.522583007812 1.0738825797819\\
132.644653320312 1.46624178169167\\
132.766723632812 1.88882307833582\\
132.888793945312 2.34881693276097\\
133.010864257812 2.82343575589469\\
133.132934570312 3.21365131045771\\
133.255004882812 3.81123157506072\\
133.377075195312 4.29287303452952\\
133.499145507812 4.96304033737024\\
133.621215820312 5.69683866225306\\
133.743286132812 6.33530272081002\\
133.865356445312 7.05792681567654\\
133.987426757812 7.82396505370622\\
134.109497070312 8.72149093386976\\
134.231567382812 9.66172983707679\\
134.353637695312 10.7229374870405\\
134.475708007812 11.4703300961014\\
134.597778320312 12.5308824354225\\
134.719848632812 13.0756272240846\\
134.841918945312 13.5871670903808\\
134.963989257812 13.5948190818878\\
135.086059570312 13.1456965558342\\
135.208129882812 12.4642660313838\\
135.330200195312 11.4461892024158\\
135.452270507812 10.5263745853362\\
135.574340820312 9.54799430194424\\
135.696411132812 8.65574978066175\\
135.818481445312 7.73272754498109\\
135.940551757812 6.91707042615596\\
136.062622070312 6.09315368718642\\
136.184692382812 5.43565231994082\\
136.306762695312 4.77596313485236\\
136.428833007812 4.1484168502123\\
136.550903320312 3.60601165640542\\
136.672973632812 2.98198740994865\\
136.795043945312 2.48020114769558\\
136.917114257812 2.03230194948254\\
137.039184570312 1.49282499500469\\
137.161254882812 1.02505304662303\\
137.283325195312 0.646250311621011\\
137.405395507812 0.163430394313606\\
137.557983398438 -0.351937591293948\\
137.710571289062 -0.773800012783896\\
137.863159179688 -1.22974017004771\\
138.015747070312 -1.63349545964609\\
138.168334960938 -2.05121868888012\\
138.320922851562 -2.44044249193088\\
138.473510742188 -2.82696021004823\\
138.626098632812 -3.18613609557616\\
138.778686523438 -3.52557838059506\\
138.931274414062 -3.89372347960266\\
139.083862304688 -4.19744862234668\\
139.236450195312 -4.49443649241202\\
139.389038085938 -4.80868182299047\\
139.541625976562 -5.12146747206272\\
139.694213867188 -5.42724427771522\\
139.846801757812 -5.71911931743134\\
139.999389648438 -6.01960259279097\\
140.151977539062 -6.29793252860234\\
140.304565429688 -6.53464055072316\\
140.457153320312 -6.85123920755033\\
140.609741210938 -7.07713120579568\\
140.762329101562 -7.33118564412849\\
140.914916992188 -7.58275405348951\\
141.067504882812 -7.8145815631177\\
141.220092773438 -8.06095967599468\\
141.372680664062 -8.29175197355631\\
141.525268554688 -8.5366292746735\\
141.677856445312 -8.78558047193706\\
141.830444335938 -8.98646988277\\
141.983032226562 -9.19736824965548\\
142.135620117188 -9.42688507221347\\
142.288208007812 -9.63272116944059\\
142.440795898438 -9.83216143085652\\
142.593383789062 -10.0208286630555\\
142.745971679688 -10.2170363718994\\
142.898559570312 -10.3848333729508\\
143.051147460938 -10.5889336969539\\
143.203735351562 -10.7951593548524\\
143.356323242188 -10.9679872050573\\
143.508911132812 -11.1739884863557\\
143.661499023438 -11.3438356847149\\
143.814086914062 -11.4996939914932\\
143.966674804688 -11.6625234011001\\
144.119262695312 -11.8054821841765\\
144.271850585938 -11.9964448654894\\
144.424438476562 -12.1721405724685\\
144.577026367188 -12.3314549674069\\
144.729614257812 -12.4767977011021\\
144.882202148438 -12.6087184799752\\
145.034790039062 -12.7870654485461\\
145.187377929688 -12.9234776526715\\
145.339965820312 -13.0784858414984\\
145.492553710938 -13.2224249230957\\
145.645141601562 -13.3788992870287\\
145.797729492188 -13.5076635418414\\
145.950317382812 -13.6445508418381\\
146.102905273438 -13.7943537436102\\
146.255493164062 -13.9375352756632\\
146.408081054688 -14.0579984373118\\
146.560668945312 -14.1811918334099\\
146.713256835938 -14.309933621807\\
146.865844726562 -14.4817015816385\\
147.018432617188 -14.5913870604728\\
147.171020507812 -14.7225032858185\\
147.323608398438 -14.8275315264741\\
147.476196289062 -14.9673866924227\\
147.628784179688 -15.0991467935818\\
147.781372070312 -15.1873707606784\\
147.933959960938 -15.3212772139588\\
148.086547851562 -15.4792843773502\\
148.239135742188 -15.5873998836287\\
148.391723632812 -15.6900807733364\\
148.544311523438 -15.808372553022\\
148.696899414062 -15.9507251676447\\
148.849487304688 -16.0446898743307\\
149.002075195312 -16.1690834557656\\
149.154663085938 -16.2590938046084\\
149.307250976562 -16.4107165288379\\
149.459838867188 -16.5033899186502\\
149.612426757812 -16.6285296434979\\
149.765014648438 -16.7466548854706\\
149.917602539062 -16.9169821598159\\
150.070190429688 -16.9431358563086\\
150.222778320312 -17.0653598983421\\
150.375366210938 -17.1637913194232\\
150.527954101562 -17.2791454895726\\
150.680541992188 -17.3684153285902\\
150.833129882812 -17.4731164541492\\
150.985717773438 -17.5843106916275\\
151.138305664062 -17.6645436676575\\
151.290893554688 -17.7894313269481\\
151.443481445312 -17.8893485310032\\
151.596069335938 -17.9692318773512\\
151.748657226562 -18.0757333671339\\
151.901245117188 -18.1932095980795\\
152.053833007812 -18.2932348517689\\
152.206420898438 -18.3961627828832\\
152.359008789062 -18.4816060942269\\
152.511596679688 -18.5769207941992\\
152.664184570312 -18.6631724978423\\
152.816772460938 -18.7677808410454\\
152.969360351562 -18.8623821749994\\
153.121948242188 -18.9563370455239\\
153.274536132812 -19.0531869920082\\
153.427124023438 -19.1549959222656\\
153.579711914062 -19.1942951641154\\
153.732299804688 -19.3442209217073\\
153.884887695312 -19.4172882861032\\
154.037475585938 -19.5089249350261\\
154.190063476562 -19.5989364270994\\
154.342651367188 -19.7021423224493\\
154.495239257812 -19.7737309710653\\
154.647827148438 -19.8700278895875\\
154.800415039062 -19.937744770789\\
154.953002929688 -20.0705457705646\\
155.105590820312 -20.1455213005903\\
155.258178710938 -20.2409985042208\\
155.410766601562 -20.3416347078434\\
155.563354492188 -20.4065691142806\\
155.715942382812 -20.4905126225699\\
155.868530273438 -20.6108661733674\\
156.021118164062 -20.7111670094906\\
156.173706054688 -20.7900092600725\\
156.326293945312 -20.855954141158\\
156.478881835938 -20.9586850615645\\
156.631469726562 -21.0165036596067\\
156.784057617188 -21.1205759830411\\
156.936645507812 -21.2090665345045\\
157.089233398438 -21.3195331688277\\
157.241821289062 -21.3509732037972\\
157.394409179688 -21.5107192221667\\
157.546997070312 -21.576499552945\\
157.699584960938 -21.6770657397607\\
157.852172851562 -21.7574740900562\\
158.004760742188 -21.8525567916748\\
158.157348632812 -21.9291189312557\\
158.309936523438 -22.0467966427403\\
158.462524414062 -22.1220091137422\\
158.615112304688 -22.2370622273184\\
158.767700195312 -22.3316416083086\\
158.920288085938 -22.4007579839401\\
159.072875976562 -22.5366717358117\\
159.225463867188 -22.6194558906936\\
159.378051757812 -22.7178563485151\\
159.530639648438 -22.8102659267605\\
159.683227539062 -22.9157229949726\\
159.835815429688 -23.0310553722116\\
159.988403320312 -23.1679182263847\\
160.140991210938 -23.2553927500775\\
160.293579101562 -23.4192092721007\\
160.446166992188 -23.4907267800632\\
160.598754882812 -23.6107520853126\\
160.751342773438 -23.7663447985239\\
160.903930664062 -23.8389949888563\\
161.056518554688 -23.9582433057213\\
161.209106445312 -24.0852525910405\\
161.361694335938 -24.1644953161615\\
161.514282226562 -24.1996660359729\\
161.666870117188 -24.2845101275361\\
161.819458007812 -24.2777254572542\\
161.972045898438 -24.3338571596319\\
162.124633789062 -24.3096027916072\\
162.277221679688 -24.2997117178429\\
162.429809570312 -24.2899105579976\\
162.582397460938 -24.2713072718664\\
162.734985351562 -24.3232431003271\\
162.887573242188 -24.3346577588157\\
163.040161132812 -24.3730877487242\\
163.192749023438 -24.3688147885074\\
163.345336914062 -24.4034030489387\\
163.497924804688 -24.4534743033823\\
163.650512695312 -24.4952497881093\\
163.803100585938 -24.5304926491238\\
163.955688476562 -24.5877812409164\\
164.108276367188 -24.6457073557266\\
164.260864257812 -24.7289106689841\\
164.413452148438 -24.7345210449442\\
164.566040039063 -24.8541739125025\\
164.718627929688 -24.8790333120393\\
164.871215820312 -24.9761391794717\\
165.023803710938 -24.9821129124909\\
165.176391601562 -25.0783503366901\\
165.328979492188 -25.1277879985693\\
165.481567382812 -25.2168471354208\\
165.634155273438 -25.2689484287656\\
165.786743164063 -25.3454477783338\\
165.939331054688 -25.4046152830329\\
166.091918945312 -25.4601134085061\\
166.244506835938 -25.5697903047202\\
166.397094726562 -25.5969033308802\\
166.549682617188 -25.6129864518098\\
166.702270507812 -25.7367896433283\\
166.854858398438 -25.7976409913377\\
167.007446289063 -25.8759339897947\\
167.160034179688 -25.9671839149053\\
167.312622070312 -26.018090346119\\
167.465209960938 -26.0750661019586\\
167.617797851562 -26.204556564979\\
167.770385742188 -26.2129042992434\\
167.922973632812 -26.2913580033025\\
168.106079101562 -26.3587266040887\\
168.289184570312 -26.4795599232356\\
168.472290039063 -26.555949835928\\
168.655395507812 -26.6186074391309\\
168.838500976562 -26.7054506718525\\
169.021606445312 -26.8083299471936\\
169.204711914063 -26.899830802414\\
169.387817382812 -26.9621720507789\\
169.570922851562 -27.060047953894\\
169.754028320312 -27.1238353965709\\
169.937133789062 -27.2378559714373\\
170.120239257812 -27.3153848062661\\
170.303344726562 -27.3786583652061\\
170.486450195312 -27.4919608135546\\
170.669555664062 -27.5473986075823\\
170.852661132813 -27.6686437815648\\
171.035766601562 -27.7310077827126\\
171.218872070312 -27.8288748847942\\
171.401977539062 -27.93594256064\\
171.585083007813 -27.9732279000169\\
171.768188476562 -28.0788072769737\\
171.951293945312 -28.1342950065324\\
172.134399414062 -28.2503964154363\\
172.317504882813 -28.3952950871069\\
172.500610351562 -28.4077515704138\\
172.683715820312 -28.5136605271541\\
172.866821289062 -28.6202782879278\\
173.049926757813 -28.6646597192895\\
173.233032226562 -28.7829293376461\\
173.416137695312 -28.8843506924594\\
173.599243164062 -29.011833111837\\
173.782348632813 -29.0858163981412\\
173.965454101562 -29.1500057682164\\
174.148559570312 -29.2472945091192\\
174.331665039062 -29.340534250542\\
174.514770507813 -29.3995216691059\\
174.697875976562 -29.540775574843\\
174.880981445312 -29.6364485292909\\
175.064086914062 -29.7156668261678\\
175.247192382813 -29.8161044114985\\
175.430297851562 -29.9513800967258\\
175.613403320312 -29.9948006630708\\
175.796508789062 -30.0713162771976\\
175.979614257813 -30.1714299833233\\
176.162719726562 -30.25008334275\\
176.345825195312 -30.280773315637\\
176.528930664062 -30.3931774687104\\
176.712036132812 -30.4441181802243\\
176.895141601562 -30.5589657401514\\
177.078247070312 -30.6412013514234\\
177.261352539062 -30.7036791152414\\
177.444458007812 -30.6957927962604\\
177.627563476563 -30.7469717961148\\
177.810668945312 -30.860650889805\\
177.993774414062 -30.8338636718512\\
178.176879882812 -30.9084851295774\\
178.359985351563 -30.9837935281283\\
178.543090820312 -31.0846895896947\\
178.726196289062 -31.1029519962581\\
178.909301757812 -31.2327361699036\\
179.092407226563 -31.2817680377125\\
179.275512695312 -31.3370738240009\\
179.458618164062 -31.4935588874388\\
179.641723632812 -31.523988571829\\
179.824829101563 -31.6295025731179\\
180.007934570312 -31.7004250492294\\
180.191040039062 -31.7647626596459\\
180.374145507812 -31.828315300828\\
180.557250976563 -31.9126644163056\\
180.740356445312 -32.0451455959046\\
180.923461914062 -32.1606001831237\\
181.106567382812 -32.2374763247442\\
181.289672851563 -32.3616347583232\\
181.472778320312 -32.454888424894\\
181.655883789062 -32.5126547920573\\
181.838989257812 -32.6476049633524\\
182.022094726563 -32.7189866132383\\
182.205200195312 -32.8517604385656\\
182.388305664062 -33.0265486668871\\
182.571411132812 -33.1214570831511\\
182.754516601563 -33.1297122602489\\
182.937622070312 -33.2905225111351\\
183.120727539062 -33.4093789048234\\
183.303833007812 -33.3844750361431\\
183.486938476562 -33.5202856791661\\
183.670043945312 -33.5836687289817\\
183.853149414062 -33.639942467103\\
184.036254882812 -33.6378049708204\\
184.219360351562 -33.5410987635555\\
184.402465820313 -33.4332979201747\\
184.585571289062 -33.2397785222918\\
184.768676757812 -32.8256885394834\\
184.951782226562 -33.5039950591372\\
185.134887695313 -33.4901624622201\\
185.317993164062 -33.3107935139342\\
185.501098632812 -33.1602475599425\\
185.684204101562 -33.1340569288926\\
185.867309570313 -33.1005976916161\\
186.050415039062 -33.1619718679672\\
186.233520507812 -33.1709869113869\\
186.416625976562 -33.2047947751807\\
186.599731445313 -33.4278848256089\\
186.782836914062 -33.4409128455754\\
186.965942382812 -33.5491311335668\\
187.149047851562 -33.6739797457052\\
187.332153320313 -33.8239951327471\\
187.515258789062 -33.9890182489537\\
187.698364257812 -34.0382035552436\\
187.881469726562 -34.167159689419\\
188.064575195313 -34.2720404284897\\
188.247680664062 -34.3772429540396\\
188.430786132812 -34.4243270675435\\
188.613891601562 -34.652048068899\\
188.796997070313 -34.6730517812979\\
188.980102539062 -34.8062357318416\\
189.163208007812 -34.9807905466518\\
189.346313476562 -35.0449326972276\\
189.529418945313 -35.1957921355423\\
189.712524414062 -35.1303874261874\\
189.895629882812 -35.5492688874863\\
190.078735351562 -35.7832576739921\\
190.261840820312 -35.7918775831786\\
190.444946289063 -35.8745455096399\\
190.628051757812 -35.8819487974791\\
190.811157226562 -36.1188319652924\\
190.994262695312 -36.12314492771\\
191.177368164063 -36.1997355052494\\
191.360473632812 -36.6294207068894\\
191.543579101562 -36.6295786145792\\
191.726684570312 -36.6250027481916\\
191.909790039063 -36.7811804463675\\
192.092895507812 -36.977801155017\\
192.276000976562 -37.0944042556956\\
192.459106445312 -37.1369047911574\\
192.642211914063 -37.2953563479879\\
192.825317382812 -37.2781488426605\\
193.008422851562 -37.540505111028\\
193.191528320312 -37.6152530494246\\
193.374633789063 -37.6506252816464\\
193.557739257812 -37.80113961639\\
193.740844726562 -38.01235525033\\
193.923950195312 -38.0033138705125\\
194.107055664063 -38.1407671041229\\
194.290161132812 -38.2525574328125\\
194.473266601562 -38.3741932949045\\
194.656372070312 -38.4093490523011\\
194.839477539063 -38.597886193043\\
195.022583007812 -38.860172695351\\
195.205688476562 -38.9275876661096\\
195.388793945312 -39.1089616746328\\
195.571899414063 -39.2220486553428\\
195.755004882812 -39.3127485402528\\
195.938110351562 -39.4041164559071\\
196.121215820312 -39.5289955803585\\
196.304321289063 -39.6474824379618\\
196.487426757812 -39.9581287484072\\
196.670532226562 -40.0073409987274\\
196.853637695312 -40.1162828326389\\
197.036743164062 -40.2909677814267\\
197.219848632813 -40.4289606297798\\
197.402954101562 -40.6843071456949\\
197.586059570312 -40.7647227947953\\
197.769165039062 -40.9028356658564\\
197.952270507813 -41.0252980304441\\
198.135375976562 -41.2437790936983\\
198.318481445312 -41.3466515352209\\
198.501586914062 -41.5509357217395\\
198.715209960938 -41.6835762908059\\
198.928833007813 -41.8074675351467\\
199.142456054688 -41.9852745918766\\
199.356079101562 -42.1330235069208\\
199.569702148438 -42.4063883836272\\
199.783325195312 -42.6160415533558\\
199.996948242188 -42.3756052058574\\
200.210571289062 -43.105975735241\\
200.424194335938 -43.0205019888571\\
200.637817382813 -43.4809355256367\\
200.851440429688 -43.560319982828\\
201.065063476562 -43.7416486419624\\
201.278686523438 -44.0143234636857\\
201.492309570312 -43.8718543176409\\
201.705932617188 -44.2863335224123\\
201.919555664062 -44.3809899425362\\
202.133178710938 -44.3926451351592\\
202.346801757813 -44.7005759944491\\
202.560424804688 -44.6241145247806\\
202.774047851562 -45.0867414707933\\
202.987670898438 -44.7780082251082\\
203.201293945312 -45.0235353365975\\
203.414916992188 -44.8634504929416\\
203.628540039062 -45.0502688263986\\
203.842163085938 -44.7254162107809\\
204.055786132812 -44.9076799090767\\
204.269409179688 -44.7822434618885\\
204.483032226563 -44.4490796284875\\
204.696655273438 -44.3193982155946\\
204.910278320312 -43.8643562775303\\
205.123901367188 -43.5654186741415\\
205.337524414062 -43.1152170128244\\
205.551147460938 -42.6697086343661\\
205.764770507812 -42.2567430454407\\
205.978393554688 -41.8107626562032\\
206.192016601563 -41.6462683880811\\
206.405639648438 -41.3799301245752\\
206.619262695312 -41.1511379754937\\
206.832885742188 -40.9503534784062\\
207.046508789062 -41.0181614800473\\
207.260131835938 -40.7624865236702\\
207.473754882812 -40.9803660793226\\
207.687377929688 -40.9795081013977\\
207.901000976563 -41.2540041675442\\
208.114624023438 -41.5047670310478\\
208.328247070312 -41.4213466207062\\
208.541870117188 -41.8427392959755\\
208.755493164062 -41.9560380030476\\
208.969116210938 -42.1448561115395\\
209.182739257812 -42.518917193509\\
209.396362304688 -42.6999171034679\\
209.609985351563 -43.0046762145332\\
209.823608398438 -43.2590778527383\\
210.037231445312 -43.4432952764352\\
210.250854492188 -43.8380324431781\\
210.464477539062 -44.2013604572193\\
210.678100585938 -44.2841583245731\\
210.891723632812 -44.6680065825176\\
211.105346679688 -44.5431476204364\\
211.318969726562 -45.075758423821\\
211.532592773438 -45.1892033805493\\
211.746215820313 -45.3893879520213\\
211.959838867188 -45.3391975659684\\
212.173461914062 -45.5428720687116\\
212.387084960938 -45.3455348800066\\
212.600708007812 -45.4905124592663\\
212.814331054688 -45.0708136639584\\
213.027954101562 -45.1696760379844\\
213.241577148438 -44.7589239205439\\
213.455200195313 -44.2846566510106\\
213.668823242188 -44.2340176636043\\
213.882446289062 -43.6214402087911\\
214.096069335938 -43.3031573601561\\
214.309692382812 -42.8550186527562\\
214.523315429688 -42.6249644285003\\
214.736938476562 -41.7988113466016\\
214.950561523438 -41.7286162848965\\
215.164184570313 -41.3552644173058\\
215.377807617188 -40.8622949371828\\
215.591430664062 -40.4358391834839\\
215.805053710938 -39.9968649447944\\
216.018676757812 -39.6680054036403\\
216.232299804688 -39.2954543344465\\
216.445922851562 -39.0537188617255\\
216.659545898438 -38.996577250948\\
216.873168945312 -38.7035657652714\\
217.086791992188 -38.5106716391729\\
217.300415039063 -38.3688169684589\\
217.514038085938 -38.2321849926318\\
217.727661132812 -38.1994838287535\\
217.941284179688 -38.1672343121294\\
218.154907226562 -38.1063266653843\\
218.368530273438 -38.2187492713869\\
218.582153320312 -38.2093669120714\\
218.795776367188 -38.1551299472451\\
219.009399414063 -38.1687256112904\\
219.223022460938 -38.2333513286366\\
219.436645507812 -38.2337011753668\\
219.650268554688 -38.27380801491\\
219.863891601562 -38.2735011876003\\
220.077514648438 -38.2742297465092\\
220.291137695312 -38.3222979068558\\
220.504760742188 -38.130914720153\\
220.718383789063 -38.2811079754606\\
220.932006835938 -38.1356866634294\\
221.145629882812 -37.9987678467744\\
221.359252929688 -37.8701324053117\\
221.572875976562 -37.6975583484142\\
221.786499023438 -37.4562140979088\\
222.000122070312 -37.1815052186818\\
222.213745117188 -36.9085676434867\\
222.427368164063 -36.4037645029845\\
222.640991210938 -35.8337293793478\\
222.854614257812 -35.3045914972876\\
223.068237304688 -34.6818481134447\\
223.281860351562 -33.9318982101064\\
223.495483398438 -32.926888029274\\
223.709106445312 -31.8509228528393\\
223.922729492188 -30.6848450479148\\
224.136352539062 -29.936408341338\\
224.349975585938 -29.8712752424545\\
224.563598632813 -30.2581489082513\\
224.777221679688 -30.7264075845926\\
224.990844726562 -30.9426086765157\\
225.204467773438 -30.6833533599103\\
225.418090820312 -30.1716436977919\\
225.631713867188 -29.4556234709482\\
225.845336914062 -28.6364330236122\\
226.058959960938 -28.096730342918\\
226.272583007813 -27.8538856146361\\
226.486206054688 -28.1326339131936\\
226.699829101562 -28.6047820572779\\
226.913452148438 -29.3280701065088\\
227.127075195312 -30.0908595130157\\
227.340698242188 -30.9722364881775\\
227.554321289062 -31.6806539788109\\
227.767944335938 -31.9602860106721\\
227.981567382813 -31.9481076546512\\
228.195190429688 -32.063414406681\\
228.408813476562 -32.2676073362136\\
228.622436523438 -32.6221390510134\\
228.836059570312 -32.5162694931451\\
229.049682617188 -31.920216650215\\
229.293823242188 -30.4126122532788\\
229.537963867188 -28.2797207143005\\
229.782104492188 -25.9753255350562\\
230.026245117188 -24.1482144740263\\
230.270385742188 -22.9009573421831\\
230.514526367188 -22.2173793673735\\
230.758666992188 -21.877460028166\\
231.002807617188 -21.7487888612858\\
231.246948242188 -21.8484345055401\\
231.491088867188 -22.2500231088316\\
231.735229492188 -22.7921501649991\\
231.979370117188 -23.5943309250471\\
232.223510742188 -24.5097460520549\\
232.467651367188 -25.4037868786389\\
232.711791992188 -26.2299901120035\\
232.955932617188 -27.3073362413709\\
233.200073242188 -28.2652570810675\\
233.444213867188 -29.269730221826\\
233.688354492188 -30.2189500953123\\
233.932495117188 -31.2411849793931\\
234.176635742188 -32.1900088822144\\
234.420776367188 -32.9439076913975\\
234.664916992188 -33.644966437048\\
234.909057617188 -34.2702082089884\\
235.153198242188 -34.7623199373714\\
235.397338867188 -35.1745393144904\\
235.641479492188 -35.4593285209898\\
235.885620117188 -35.8150084002735\\
236.129760742188 -36.0040560297981\\
236.373901367188 -36.3060614410042\\
236.618041992188 -36.4538679963629\\
236.862182617188 -36.7764916541564\\
237.106323242188 -37.0168327526254\\
237.350463867188 -37.1537054317395\\
237.594604492188 -37.4095824462258\\
237.838745117188 -37.5553364955882\\
238.082885742188 -37.8340932397084\\
238.327026367188 -37.9365702086428\\
238.571166992188 -38.1710149299869\\
238.815307617188 -38.4304682488332\\
239.059448242188 -38.6142307727133\\
239.303588867188 -38.7564048100111\\
239.547729492188 -38.9394047252193\\
239.791870117188 -39.1415761893688\\
240.036010742188 -39.3160821650863\\
240.280151367188 -39.3402319333759\\
240.524291992188 -39.477166920542\\
240.768432617188 -39.768298700706\\
241.012573242188 -39.8511700463062\\
241.256713867188 -39.9947650279486\\
241.500854492188 -40.0525115217759\\
241.744995117188 -40.1035844040675\\
241.989135742188 -40.2072242045939\\
242.233276367188 -40.2521926308616\\
242.477416992188 -40.3832446242753\\
242.721557617188 -40.4480645093544\\
242.965698242188 -40.4620855913522\\
243.209838867188 -40.4802580839091\\
243.453979492188 -40.5357096545286\\
243.698120117188 -40.5311647766332\\
243.942260742188 -40.4714127405214\\
244.186401367188 -40.4472572871019\\
244.430541992188 -40.5217297911667\\
244.674682617188 -40.5954513857567\\
244.918823242188 -40.5631394903544\\
245.162963867188 -40.5049100832474\\
245.407104492188 -40.6234537297563\\
245.651245117188 -40.5388080473369\\
245.895385742188 -40.4853608779976\\
246.139526367188 -40.4218909710044\\
246.383666992188 -40.5278324308587\\
246.627807617188 -40.4567674569337\\
246.871948242188 -40.5123706417448\\
247.116088867188 -40.5152001341578\\
247.360229492188 -40.575372907189\\
247.604370117188 -40.4419052850046\\
247.848510742188 -40.4353379026646\\
248.092651367188 -40.4927408202588\\
248.336791992188 -40.4021955518532\\
248.580932617188 -40.337711948788\\
248.825073242188 -40.4719706635142\\
249.069213867188 -40.3550632839214\\
249.313354492188 -40.4260551089812\\
249.557495117188 -40.3360015064589\\
249.801635742188 -40.3421631135985\\
250.045776367188 -40.6666648371983\\
250.289916992188 -40.2189292883719\\
250.534057617188 -40.2941919531735\\
250.778198242188 -40.1133856819813\\
251.022338867188 -40.0837815085059\\
251.266479492188 -40.0194278396962\\
251.510620117188 -39.8416802039781\\
251.754760742188 -39.8429381896688\\
251.998901367188 -39.7630656145004\\
252.243041992188 -39.5797988756268\\
252.487182617188 -39.5214209658692\\
252.731323242188 -39.4072247162113\\
252.975463867188 -39.3856364488888\\
253.219604492188 -39.223550484307\\
253.463745117188 -39.1437293818137\\
253.707885742188 -39.1800723245273\\
253.952026367188 -39.0577670925721\\
254.196166992188 -39.0061101185767\\
254.440307617188 -38.932661118075\\
254.684448242188 -38.9508509698699\\
254.928588867188 -38.8686365110537\\
255.172729492188 -38.9005969356903\\
255.416870117188 -38.8277493933784\\
255.661010742188 -38.8537176022082\\
255.905151367188 -38.7607114867234\\
256.149291992188 -38.7234106857339\\
256.393432617188 -38.8127389369733\\
256.637573242188 -38.7064192126438\\
256.881713867188 -38.8683511383334\\
257.125854492188 -38.8074726860189\\
257.369995117188 -38.8853863962092\\
257.614135742188 -38.8757698621019\\
257.858276367188 -38.8013699491794\\
258.102416992188 -38.8828910265307\\
258.346557617188 -38.9512176924694\\
258.590698242188 -38.9631811840811\\
258.834838867188 -39.0680149445166\\
259.078979492188 -39.1259985651743\\
259.323120117188 -39.2654009372315\\
259.567260742188 -39.3533248720139\\
259.841918945312 -39.5213849534003\\
260.116577148438 -39.7642489936944\\
260.391235351562 -39.9343007236667\\
260.665893554688 -40.3347257731144\\
260.940551757812 -40.5457839648597\\
261.215209960938 -40.9235538238127\\
261.489868164062 -41.3362525359894\\
261.764526367188 -41.6812725613497\\
262.039184570312 -41.999904290295\\
262.313842773438 -42.3046773222469\\
262.588500976562 -42.3125323475334\\
262.863159179688 -42.1917461079152\\
263.137817382812 -41.9984476130928\\
263.412475585938 -41.7797549007119\\
263.687133789062 -41.4310015307203\\
263.961791992188 -41.0606537559425\\
264.236450195312 -40.6517604964171\\
264.511108398438 -40.3441922477359\\
264.785766601562 -40.0895260829726\\
265.060424804688 -39.6712407937651\\
265.335083007812 -39.4631844987626\\
265.609741210938 -39.1927738195117\\
265.884399414062 -38.9345810088184\\
266.159057617188 -38.8117009053585\\
266.433715820312 -38.6251913208439\\
266.708374023438 -38.5071067300555\\
266.983032226562 -38.3541059767743\\
267.257690429688 -38.2419557508445\\
267.532348632812 -38.2275550588597\\
267.807006835938 -38.1598777763515\\
268.081665039062 -38.2013260072647\\
268.356323242188 -38.2395859858737\\
268.630981445312 -38.4499098914113\\
268.905639648438 -38.5490413086232\\
269.180297851562 -38.8554338348172\\
269.454956054688 -39.0266464023104\\
269.729614257812 -38.5921336912153\\
270.004272460938 -38.55107732976\\
270.278930664062 -38.7646912787949\\
270.553588867188 -38.6157003836085\\
270.828247070312 -38.6252369110364\\
271.102905273438 -38.4819925674473\\
271.377563476562 -38.2763872578359\\
271.652221679688 -38.0189571630644\\
271.926879882812 -37.8347839256406\\
272.201538085938 -37.6332214236706\\
272.476196289062 -37.406883730289\\
272.750854492188 -37.1937107064692\\
273.025512695312 -36.937394672557\\
};
\addplot [frf]
table[row sep=crcr]{
273.025512695312 -36.937394672557\\
273.300170898438 -36.675004867896\\
273.574829101562 -36.4187994604164\\
273.849487304688 -36.0961630405728\\
274.124145507812 -35.8444197736439\\
274.398803710938 -35.5005803690458\\
274.673461914062 -35.1703075182441\\
274.948120117188 -34.8793978835789\\
275.222778320312 -34.5268241131136\\
275.497436523438 -34.199149426142\\
275.772094726562 -33.8931799469982\\
276.046752929688 -33.6079189633853\\
276.321411132812 -33.3829984653308\\
276.596069335938 -33.2110386905936\\
276.870727539062 -33.1733941839262\\
277.145385742188 -33.1786317534418\\
277.420043945312 -33.2436824248188\\
277.694702148438 -33.3097520960282\\
277.969360351562 -33.4828641921058\\
278.244018554688 -33.6621836611328\\
278.518676757812 -33.8608157576315\\
278.793334960938 -34.024002335369\\
279.067993164062 -34.2189617378359\\
279.342651367188 -34.3644895375376\\
279.617309570312 -34.5246244239581\\
279.891967773438 -34.6262161502549\\
280.166625976562 -34.7634004270994\\
280.441284179688 -34.9005893576496\\
280.715942382812 -34.9929097162614\\
280.990600585938 -35.053967170973\\
281.265258789062 -35.1743483826541\\
281.539916992188 -35.2110581113015\\
281.814575195312 -35.3299159412496\\
282.089233398438 -35.5570670747377\\
282.363891601562 -35.8091520610298\\
282.638549804688 -35.944638682633\\
282.913208007812 -35.989946562142\\
283.187866210938 -36.0930372951921\\
283.462524414062 -36.1004531116746\\
283.737182617188 -36.0631469450404\\
284.011840820312 -35.8702406852595\\
284.286499023438 -35.6621475520279\\
284.561157226562 -35.8532182614752\\
284.835815429688 -36.0267421856365\\
285.110473632812 -36.0583398431692\\
285.385131835938 -35.9462650457725\\
285.659790039062 -35.7387872666605\\
285.934448242188 -35.4909230187903\\
286.209106445312 -35.3381816643573\\
286.483764648438 -35.6026083535909\\
286.758422851562 -35.7252849554478\\
287.033081054688 -35.6502743531209\\
287.307739257812 -35.4700097232753\\
287.582397460938 -35.2882363916696\\
287.857055664062 -35.1470301549759\\
288.131713867188 -34.9734206569882\\
288.406372070312 -34.7890261071387\\
288.681030273438 -34.6181915756245\\
288.955688476562 -34.4944105390014\\
289.230346679688 -34.2782979714087\\
289.505004882812 -34.1163766675386\\
289.779663085938 -33.9164453314853\\
290.054321289062 -33.7620530376355\\
290.359497070312 -33.5715042399244\\
290.664672851562 -33.323898180136\\
290.969848632812 -33.0682856921756\\
291.275024414062 -32.8812032068102\\
291.580200195312 -32.6026735611005\\
291.885375976562 -32.3363555267286\\
292.190551757812 -32.0557582927135\\
292.495727539062 -31.7015486918246\\
292.800903320312 -31.3281949768964\\
293.106079101562 -30.9144062903075\\
293.411254882812 -30.4137869564015\\
293.716430664062 -29.8991712375511\\
294.021606445312 -29.2467667052925\\
294.326782226562 -28.4075696855411\\
294.631958007812 -27.4512652401267\\
294.937133789062 -26.1051898530588\\
295.242309570312 -24.2481723469216\\
295.547485351562 -21.3775938270621\\
295.852661132812 -16.4820283253322\\
296.157836914062 -14.619615478286\\
296.463012695312 -22.7820997779145\\
296.768188476562 -28.8021445926427\\
297.073364257812 -33.7881299867243\\
297.378540039062 -38.5291034339379\\
297.683715820312 -43.5783154772716\\
297.988891601562 -48.8593519154186\\
298.294067382812 -51.052309614127\\
298.599243164062 -48.1037144806268\\
298.904418945312 -45.5050618646308\\
299.209594726562 -43.7719059930062\\
299.514770507812 -42.4458921715888\\
299.819946289062 -41.4441663291632\\
300.125122070312 -40.6552643565482\\
300.430297851562 -39.9943021978995\\
300.735473632812 -39.5109809062631\\
301.040649414062 -39.0034376638197\\
301.345825195312 -38.6164669084945\\
301.651000976562 -38.3449256476227\\
301.956176757812 -37.9999341099257\\
302.261352539062 -37.7408552501199\\
302.566528320312 -37.4975343190581\\
302.871704101562 -37.2110640566338\\
303.176879882812 -37.0579452352923\\
303.482055664062 -36.8226297391507\\
303.787231445312 -36.5939230209501\\
304.092407226562 -36.3859032987074\\
304.397583007812 -36.224752380773\\
304.702758789062 -36.0747554720015\\
305.007934570312 -35.9261815033499\\
305.313110351562 -35.7978431373789\\
305.618286132812 -35.6119516488666\\
305.923461914062 -35.5281926734016\\
306.228637695312 -35.353901139194\\
306.533813476562 -35.171689724656\\
306.838989257812 -34.9868190091165\\
307.144165039062 -34.8507315188404\\
307.449340820312 -34.6860042285551\\
307.754516601562 -34.540596858338\\
308.059692382812 -34.3791404899357\\
308.364868164062 -34.2750824111596\\
308.670043945312 -34.1974480991962\\
308.975219726562 -34.0676433939441\\
309.280395507812 -33.9222712048363\\
309.585571289062 -33.7341477796312\\
309.890747070312 -33.5670952396841\\
310.195922851562 -33.4271994093563\\
310.501098632812 -33.2245496047683\\
310.806274414062 -33.0994862933146\\
311.111450195312 -32.938105897692\\
311.416625976562 -32.8394011272145\\
311.721801757812 -32.6925692884504\\
312.026977539062 -32.576671688364\\
312.332153320312 -32.4248410018169\\
312.637329101562 -32.3239386217988\\
312.942504882812 -32.2413443466451\\
313.247680664062 -32.1578933888263\\
313.552856445312 -32.0896974082411\\
313.858032226562 -32.0046205017202\\
314.163208007812 -31.9538398850536\\
314.468383789062 -31.8990099300717\\
314.773559570312 -31.8434376938199\\
315.078735351562 -31.7894166063995\\
315.383911132812 -31.7025420661962\\
315.689086914062 -31.5651923823992\\
315.994262695312 -31.4745754883423\\
316.299438476562 -31.3238054760985\\
316.604614257812 -31.1482900754642\\
316.909790039062 -30.9944808493548\\
317.214965820312 -30.7891980433435\\
317.520141601562 -30.5970996706151\\
317.825317382812 -30.4319939632568\\
318.130493164062 -30.2189566439265\\
318.435668945312 -29.9840800832091\\
318.740844726562 -29.7510060976464\\
319.046020507812 -29.5248341742711\\
319.351196289062 -29.2666634699962\\
319.656372070312 -29.0169322082201\\
319.961547851562 -28.7354444201533\\
320.266723632812 -28.4769040681912\\
320.571899414062 -28.1852038037864\\
320.907592773438 -27.8683891542773\\
321.243286132812 -27.5258840334112\\
321.578979492188 -27.177936825117\\
321.914672851562 -26.7964151927249\\
322.250366210938 -26.4282646477715\\
322.586059570312 -25.9938284401303\\
322.921752929688 -25.5457032951244\\
323.257446289062 -25.0543161043073\\
323.593139648438 -24.5402666088606\\
323.928833007812 -23.9949543938907\\
324.264526367188 -23.3903740998102\\
324.600219726562 -22.6924766337501\\
324.935913085938 -21.9787635633356\\
325.271606445312 -21.1320334393715\\
325.607299804688 -20.1476603723157\\
325.942993164062 -19.0411761777747\\
326.278686523438 -17.7301093319963\\
326.614379882813 -16.1237332389189\\
326.950073242188 -14.0948910232815\\
327.285766601563 -11.3918555278881\\
327.621459960938 -7.64924823158193\\
327.957153320312 -3.14625618729086\\
328.292846679688 -4.9165269492032\\
328.628540039062 -9.76776822349025\\
328.964233398438 -13.359675138857\\
329.299926757812 -16.1073313398983\\
329.635620117188 -18.2504876926813\\
329.971313476562 -20.0080946957053\\
330.307006835938 -21.6009906904349\\
330.642700195312 -22.9487893583706\\
330.978393554688 -24.2404205101129\\
331.314086914062 -25.3211319604354\\
331.649780273438 -26.3897701055797\\
331.985473632812 -27.4671690666259\\
332.321166992188 -28.5120133150385\\
332.656860351563 -29.0407832412988\\
332.992553710938 -28.548979196707\\
333.328247070313 -29.3367430212518\\
333.663940429688 -30.0635407585176\\
333.999633789062 -30.7533640697175\\
334.335327148438 -31.2794202729213\\
334.671020507812 -31.7797613694134\\
335.006713867188 -32.2247143262524\\
335.342407226562 -32.6159572174883\\
335.678100585938 -32.9198533752272\\
336.013793945312 -33.1914491057481\\
336.349487304688 -33.4141049935029\\
336.685180664062 -33.5095134857309\\
337.020874023438 -33.5205071157542\\
337.356567382812 -33.4807906274235\\
337.692260742188 -33.3954705494604\\
338.027954101562 -33.1960765516294\\
338.363647460938 -32.8847520244012\\
338.699340820313 -32.6036947549868\\
339.035034179688 -32.3677915368286\\
339.370727539063 -32.2154985728543\\
339.706420898438 -32.0569993382764\\
340.042114257812 -31.8637737072911\\
340.377807617188 -31.5945786370969\\
340.713500976562 -31.2855600153616\\
341.049194335938 -30.9396242003663\\
341.384887695312 -30.5217459925584\\
341.720581054688 -30.119503182739\\
342.056274414062 -29.6962771009605\\
342.391967773438 -29.2158218021279\\
342.727661132812 -28.7073133431255\\
343.063354492188 -28.1725232247467\\
343.399047851562 -27.6777510990234\\
343.734741210938 -27.0966233552452\\
344.070434570312 -26.538381480274\\
344.406127929688 -25.9857548513525\\
344.741821289063 -25.3981621321468\\
345.077514648438 -24.843167723002\\
345.413208007813 -24.3508894961766\\
345.748901367188 -23.8730206654892\\
346.084594726562 -23.459698865414\\
346.420288085938 -23.1246028065518\\
346.755981445312 -22.9183831004816\\
347.091674804688 -22.8219498345362\\
347.427368164062 -22.8446354306756\\
347.763061523438 -22.9889310020359\\
348.098754882812 -23.1945216486323\\
348.434448242188 -23.5050645952983\\
348.770141601562 -23.8859846028857\\
349.105834960938 -24.2634904233413\\
349.441528320312 -24.7260521956556\\
349.777221679688 -25.1609778726092\\
350.112915039062 -25.6211229856926\\
350.448608398438 -26.0323376894597\\
350.784301757813 -26.4573472878044\\
351.119995117188 -26.8835185944979\\
351.486206054688 -27.3088678200237\\
351.852416992188 -27.7325346003504\\
352.218627929688 -28.12447501796\\
352.584838867188 -28.4937950500156\\
352.951049804688 -28.8497242471484\\
353.317260742188 -29.1606020616905\\
353.683471679688 -29.4885187433364\\
354.049682617188 -29.7916191434248\\
354.415893554688 -30.0630565819496\\
354.782104492188 -30.296965816548\\
355.148315429688 -30.5455184975219\\
355.514526367188 -30.7743400086079\\
355.880737304688 -30.9897080011298\\
356.246948242188 -31.1934135965849\\
356.613159179688 -31.3976551743029\\
356.979370117188 -31.5711347669772\\
357.345581054688 -31.7210612795883\\
357.711791992188 -31.8709508298973\\
358.078002929688 -32.0199270408115\\
358.444213867188 -32.1360785144421\\
358.810424804688 -32.250061404887\\
359.176635742188 -32.3540403336178\\
359.542846679688 -32.4700218542999\\
359.909057617188 -32.5134777599618\\
360.275268554688 -32.5968922592163\\
360.641479492188 -32.6742570021579\\
361.007690429688 -32.7111841527406\\
361.373901367188 -32.7417721953137\\
361.740112304688 -32.8001496708011\\
362.106323242188 -32.7903398338305\\
362.472534179688 -32.7988819440875\\
362.838745117188 -32.8283570832591\\
363.204956054688 -32.7963533607639\\
363.571166992188 -32.7869722716132\\
363.937377929688 -32.7392159438034\\
364.303588867188 -32.7732682115586\\
364.669799804688 -32.779618730499\\
365.036010742188 -32.722826551876\\
365.402221679688 -32.6044671169167\\
365.768432617188 -32.4437304966954\\
366.134643554688 -32.3543902927151\\
366.500854492188 -32.2999611482089\\
366.867065429688 -32.1536336943626\\
367.233276367188 -31.9992766802902\\
367.599487304688 -31.8380352038661\\
367.965698242188 -31.6832047356852\\
368.331909179688 -31.4550536180026\\
368.698120117188 -31.1569045997344\\
369.064331054688 -30.8361408824196\\
369.430541992188 -30.4884942572148\\
369.796752929688 -30.049690101305\\
370.162963867188 -29.5541511644432\\
370.529174804688 -29.0385668152656\\
370.895385742188 -28.7319823987935\\
371.261596679688 -28.5108397304999\\
371.627807617188 -28.3419831658599\\
371.994018554688 -28.1344410651951\\
372.360229492188 -27.9261310844324\\
372.726440429688 -27.6579196233595\\
373.092651367188 -27.3201922149249\\
373.458862304688 -26.9064695179879\\
373.825073242188 -26.4663564326889\\
374.191284179688 -25.9287022129281\\
374.557495117188 -25.3706322561284\\
374.923706054688 -24.7386943050999\\
375.289916992188 -24.0489661368501\\
375.656127929688 -23.1942088831154\\
376.022338867188 -22.3089408874272\\
376.388549804688 -21.2940531549391\\
376.754760742188 -20.087648706688\\
377.120971679688 -19.1367176742033\\
377.487182617188 -18.1709932347454\\
377.853393554688 -17.3309102857845\\
378.219604492188 -16.8863696652054\\
378.585815429688 -16.6569321116699\\
378.952026367188 -17.0891203000463\\
379.318237304688 -18.0676971205099\\
379.684448242188 -19.4085771466921\\
380.050659179688 -20.8572167376006\\
380.416870117188 -22.4123090994751\\
380.783081054688 -23.7213561729964\\
381.149291992188 -25.030767402792\\
381.515502929688 -26.2030255856653\\
381.912231445312 -27.5143211560448\\
382.308959960938 -28.6324919627558\\
382.705688476562 -29.7340166527822\\
383.102416992188 -30.8459482909328\\
383.499145507812 -31.8733726997812\\
383.895874023438 -33.0089295660569\\
384.292602539062 -34.124887073709\\
384.689331054688 -35.4767284620564\\
385.086059570313 -36.9890006831868\\
385.482788085938 -38.8956323195762\\
385.879516601563 -41.776243103478\\
386.276245117188 -45.5544159807864\\
386.672973632812 -46.1224640519676\\
387.069702148438 -41.4801100130268\\
387.466430664062 -38.2344294270404\\
387.863159179688 -36.3391944540421\\
388.259887695312 -34.8835431446685\\
388.656616210938 -33.3411974605257\\
389.053344726562 -31.8745402687376\\
389.450073242188 -29.486771524404\\
389.846801757812 -26.5734752977838\\
390.243530273438 -22.9426915944201\\
390.640258789063 -20.077336679172\\
391.036987304688 -21.1086053540044\\
391.433715820313 -23.9329705275009\\
391.830444335938 -26.4490457355246\\
392.227172851562 -28.3258414831418\\
392.623901367188 -29.8533075902607\\
393.020629882812 -31.1635042794703\\
393.417358398438 -32.1483524850607\\
393.814086914062 -33.0655866784144\\
394.210815429688 -33.8953038997081\\
394.607543945312 -34.592872816737\\
395.004272460938 -35.2311127933401\\
395.401000976562 -35.8124264100446\\
395.797729492188 -36.3660416054035\\
396.194458007813 -36.8116926992163\\
396.591186523438 -37.2962700234852\\
396.987915039063 -37.6567772219418\\
397.384643554688 -38.1299124287424\\
397.781372070312 -38.4468606666141\\
398.178100585938 -38.8034453056294\\
398.574829101562 -39.1027175631521\\
398.971557617188 -39.4076779875227\\
399.368286132812 -39.6843326327654\\
399.765014648438 -40.0300702364795\\
400.161743164062 -40.4029846078406\\
400.558471679688 -40.7765589187108\\
400.955200195312 -41.0951209024361\\
401.351928710938 -41.3313344861021\\
401.748657226562 -41.6267514369614\\
402.145385742188 -41.8053177597077\\
402.542114257813 -42.0515343669454\\
402.938842773438 -42.3069853552695\\
403.335571289062 -42.520756519061\\
403.732299804688 -42.7059671868308\\
404.129028320312 -42.9912390527792\\
404.525756835938 -43.1875523192057\\
404.922485351562 -43.3292559237537\\
405.319213867188 -43.5168030253137\\
405.715942382812 -43.6938653951643\\
406.112670898438 -43.7913085025191\\
406.509399414062 -43.7997270109052\\
406.906127929688 -43.8403696437369\\
407.302856445312 -43.9610805788\\
407.699584960938 -44.067080190558\\
408.096313476563 -44.2532811140906\\
408.493041992188 -44.3180283123096\\
408.889770507812 -44.2959376465072\\
409.286499023438 -44.3469391814724\\
409.683227539062 -44.1568310783383\\
410.079956054688 -43.8858753291001\\
410.476684570312 -43.1911118067488\\
410.873413085938 -42.4407089355817\\
411.270141601562 -41.8360735183505\\
411.666870117188 -42.8282545986501\\
412.063598632812 -44.3344890763943\\
412.490844726563 -44.425654572046\\
412.918090820312 -43.4045753762419\\
413.345336914062 -41.8925909880907\\
413.772583007812 -40.4011645137716\\
414.199829101562 -39.4455550432977\\
414.627075195313 -39.2731365516379\\
415.054321289062 -39.79072290557\\
415.481567382812 -40.6988467955418\\
415.908813476562 -41.5617186444618\\
416.336059570312 -42.4299603162232\\
416.763305664063 -43.1079004212441\\
417.190551757812 -43.7457304472814\\
417.617797851562 -44.337576123697\\
418.045043945313 -44.8187919141943\\
418.472290039062 -45.2012030463115\\
418.899536132812 -45.5901784445679\\
419.326782226562 -45.9942958857254\\
419.754028320312 -46.2657486562588\\
420.181274414063 -46.536970844986\\
420.608520507812 -46.8967306859171\\
421.035766601562 -47.1654891410497\\
421.463012695313 -47.3893911701809\\
421.890258789062 -47.6544140436116\\
422.317504882813 -47.8187606965558\\
422.744750976562 -48.0255932519046\\
423.171997070312 -48.2690692591375\\
423.599243164063 -48.4097271651533\\
424.026489257812 -48.6711509789083\\
424.453735351562 -48.8201304839656\\
424.880981445312 -48.9449517852933\\
425.308227539062 -49.1252534245891\\
425.735473632813 -49.274414855792\\
426.162719726562 -49.4991889531414\\
426.589965820312 -49.5916458599933\\
427.017211914063 -49.608570173999\\
427.444458007812 -49.8023108206164\\
427.871704101562 -49.879277884434\\
428.298950195312 -49.8876341686962\\
428.726196289062 -50.0094900883447\\
429.153442382813 -50.1806344270391\\
429.580688476562 -50.1838221728623\\
430.007934570312 -50.2504027093744\\
430.435180664062 -50.2773156078471\\
430.862426757812 -50.2454907542768\\
431.289672851563 -50.2611301334994\\
431.716918945312 -50.3036757032161\\
432.144165039062 -50.3706591835063\\
432.571411132813 -50.4452323594154\\
432.998657226562 -50.5361267645673\\
433.425903320312 -50.6617453748936\\
433.853149414062 -50.6952852935236\\
434.280395507812 -50.8665863653473\\
434.707641601563 -50.8798519877927\\
435.134887695312 -50.9148933224369\\
435.562133789062 -50.9307711805967\\
435.989379882812 -50.9201809286935\\
436.416625976562 -50.8949809694618\\
436.843872070313 -50.804922100377\\
437.271118164062 -50.7150558884676\\
437.698364257812 -50.7129086250767\\
438.125610351563 -50.5781068695989\\
438.552856445312 -50.4309464446446\\
438.980102539062 -50.1925369577449\\
439.407348632812 -49.8980422284267\\
439.834594726562 -49.6491348853579\\
440.261840820313 -49.2391191796712\\
440.689086914062 -48.8396256949698\\
441.116333007812 -48.3403114827833\\
441.543579101563 -47.8711049811106\\
441.970825195312 -47.285932215881\\
442.398071289063 -46.6093489038823\\
442.825317382812 -45.9054673186747\\
443.283081054688 -45.0152285102987\\
443.740844726562 -44.0213039226425\\
444.198608398438 -42.8655747803023\\
444.656372070313 -41.4421315023459\\
445.114135742188 -39.8373753489235\\
445.571899414062 -37.9077670126204\\
446.029663085938 -35.6430211111451\\
446.487426757812 -33.284194499113\\
446.945190429688 -31.9425703417507\\
447.402954101562 -32.6189668171281\\
447.860717773438 -34.7568594154124\\
448.318481445312 -36.8910856509494\\
448.776245117188 -38.6496946463989\\
449.234008789063 -40.1952212603416\\
449.691772460938 -41.5485233795174\\
450.149536132812 -42.6508959569281\\
450.607299804688 -43.6744514722631\\
451.065063476562 -44.3653785629943\\
451.522827148438 -44.9643662018045\\
451.980590820312 -45.3159677940551\\
452.438354492188 -45.5692147155703\\
452.896118164062 -45.7968632790467\\
453.353881835938 -46.0985185031812\\
453.811645507813 -46.6520359151643\\
454.269409179688 -47.1992675513495\\
454.727172851562 -47.9053413636763\\
455.184936523438 -48.551016316154\\
455.642700195312 -49.0816595438553\\
456.100463867188 -49.4896968870401\\
456.558227539062 -49.9053271244252\\
457.015991210938 -50.2123844511091\\
457.473754882812 -50.4863014720558\\
457.931518554688 -50.7605576996279\\
458.389282226563 -51.0014396724001\\
458.847045898438 -51.2729762972659\\
459.304809570312 -51.5065927729921\\
459.762573242188 -51.6246215444316\\
460.220336914062 -51.8683145622277\\
460.678100585938 -52.0448082461997\\
461.135864257813 -52.3214319840639\\
461.593627929688 -52.4369089442425\\
462.051391601562 -52.6808668291436\\
462.509155273438 -52.8623596350399\\
462.966918945313 -53.0451268799322\\
463.424682617188 -53.2416214010857\\
463.882446289062 -53.3798225589953\\
464.340209960938 -53.5695963723294\\
464.797973632812 -53.7281869072607\\
465.255737304688 -53.9243789360573\\
465.713500976563 -54.0185157799813\\
466.171264648438 -54.1866733346473\\
466.629028320312 -54.3123386564757\\
467.086791992188 -54.4075605370535\\
467.544555664062 -54.5821346480158\\
468.002319335938 -54.707459155943\\
468.460083007812 -54.8946702298036\\
468.917846679688 -55.0766928696183\\
469.375610351562 -55.2956088141557\\
469.833374023438 -55.4868787046934\\
470.291137695313 -55.7197005681825\\
470.748901367188 -55.9590156022735\\
471.206665039062 -56.1958147256554\\
471.664428710938 -56.359969128611\\
472.122192382812 -56.6048277125771\\
472.579956054688 -56.8052895732985\\
473.037719726562 -57.0703249811197\\
473.526000976562 -57.3201873155991\\
474.014282226562 -57.5851656578797\\
474.502563476562 -57.8740305044663\\
474.990844726562 -58.2372010628472\\
475.479125976562 -58.5371257468719\\
475.967407226562 -58.8295035702069\\
476.455688476562 -59.1700791646547\\
476.943969726562 -59.3822223872072\\
477.432250976562 -59.7387433242906\\
477.920532226562 -60.0341705568812\\
478.408813476562 -60.264404992593\\
478.897094726562 -60.4746835262298\\
479.385375976562 -60.5202290068101\\
479.873657226562 -60.1471375026795\\
480.361938476562 -59.3301710926724\\
480.850219726562 -58.3854009377494\\
481.338500976562 -57.1778035089375\\
481.826782226562 -55.7102000741862\\
482.315063476562 -53.8774672142952\\
482.803344726562 -51.901620948577\\
483.291625976562 -50.0291726947724\\
483.779907226562 -49.0686944152187\\
484.268188476562 -49.3062004321982\\
484.756469726562 -50.347033561616\\
485.244750976562 -51.5409771198621\\
485.733032226562 -52.7011634300555\\
486.221313476562 -53.7503314036556\\
486.709594726562 -54.658558253451\\
487.197875976562 -55.5388349220832\\
487.686157226562 -56.1892355530825\\
488.174438476562 -56.882699665998\\
488.662719726562 -57.4189091321131\\
489.151000976562 -58.0231986340724\\
489.639282226562 -58.5364106192776\\
490.127563476562 -59.0393026515932\\
490.615844726562 -59.5222701456913\\
491.104125976562 -60.0128077420414\\
491.592407226562 -60.5175303918911\\
492.080688476562 -61.0325743425444\\
492.568969726562 -61.5340757336514\\
493.057250976562 -62.2092824410782\\
493.545532226562 -62.9244295658532\\
494.033813476562 -63.9112851584894\\
494.522094726562 -65.5493299465408\\
495.010375976562 -67.7864574061762\\
495.498657226562 -65.0278811289096\\
495.986938476562 -60.8140438314117\\
496.475219726562 -60.2685606847353\\
496.963500976562 -60.3910257312709\\
497.451782226562 -60.7594487147321\\
497.940063476562 -61.0196005993587\\
498.428344726562 -61.2022365464234\\
498.916625976562 -61.3975620116103\\
499.404907226562 -61.4539883278733\\
499.893188476562 -61.5157188121727\\
499.984741210938 -61.4375424585283\\
};
%\addlegendentry{FRF $P(\omega_k)$};
\label{m2t:plantFRF}

% \addplot [
% color=green!65!black,
% solid,
% forget plot
% ]
% table[row sep=crcr]{
% 8.0108642578125 263.036145614564\\
% 8.0413818359375 262.945078512062\\
% 8.0718994140625 262.854338025453\\
% 8.1024169921875 262.763919809447\\
% 8.1329345703125 262.67381961555\\
% 8.1634521484375 262.584033289237\\
% 8.1939697265625 262.494556767221\\
% 8.2244873046875 262.405386074822\\
% 8.2550048828125 262.316517323431\\
% 8.2855224609375 262.227946708054\\
% 8.3160400390625 262.139670504956\\
% 8.3465576171875 262.051685069374\\
% 8.3770751953125 261.963986833314\\
% 8.4075927734375 261.876572303423\\
% 8.4381103515625 261.789438058937\\
% 8.4686279296875 261.702580749686\\
% 8.4991455078125 261.615997094188\\
% 8.5296630859375 261.529683877782\\
% 8.5601806640625 261.44363795084\\
% 8.5906982421875 261.357856227032\\
% 8.6212158203125 261.27233568165\\
% 8.6517333984375 261.187073349983\\
% 8.6822509765625 261.102066325749\\
% 8.7127685546875 261.017311759576\\
% 8.7432861328125 260.932806857529\\
% 8.7738037109375 260.848548879689\\
% 8.8043212890625 260.764535138775\\
% 8.8348388671875 260.680762998807\\
% 8.8653564453125 260.597229873812\\
% 8.8958740234375 260.513933226574\\
% 8.9263916015625 260.430870567419\\
% 8.9569091796875 260.348039453036\\
% 8.9874267578125 260.265437485338\\
% 9.0179443359375 260.183062310359\\
% 9.0484619140625 260.100911617175\\
% 9.0789794921875 260.01898313687\\
% 9.1094970703125 259.937274641521\\
% 9.1400146484375 259.855783943227\\
% 9.1705322265625 259.774508893149\\
% 9.2010498046875 259.693447380598\\
% 9.2315673828125 259.61259733213\\
% 9.2620849609375 259.531956710686\\
% 9.2926025390625 259.451523514742\\
% 9.3231201171875 259.371295777491\\
% 9.3536376953125 259.291271566049\\
% 9.3841552734375 259.211448980679\\
% 9.4146728515625 259.131826154038\\
% 9.4451904296875 259.052401250452\\
% 9.4757080078125 258.9731724652\\
% 9.5062255859375 258.894138023829\\
% 9.5367431640625 258.81529618148\\
% 9.5672607421875 258.736645222235\\
% 9.5977783203125 258.658183458483\\
% 9.6282958984375 258.579909230303\\
% 9.6588134765625 258.501820904866\\
% 9.6893310546875 258.423916875846\\
% 9.7198486328125 258.34619556285\\
% 9.7503662109375 258.268655410873\\
% 9.7808837890625 258.191294889751\\
% 9.8114013671875 258.114112493639\\
% 9.8419189453125 258.037106740499\\
% 9.8724365234375 257.960276171607\\
% 9.9029541015625 257.883619351063\\
% 9.9334716796875 257.807134865321\\
% 9.9639892578125 257.730821322729\\
% 9.9945068359375 257.654677353084\\
% 10.0250244140625 257.578701607188\\
% 10.0555419921875 257.50289275643\\
% 10.0860595703125 257.427249492368\\
% 10.1165771484375 257.351770526323\\
% 10.1470947265625 257.276454588985\\
% 10.1776123046875 257.201300430034\\
% 10.2081298828125 257.126306817756\\
% 10.2386474609375 257.051472538682\\
% 10.2691650390625 256.976796397233\\
% 10.2996826171875 256.902277215368\\
% 10.3302001953125 256.827913832244\\
% 10.3607177734375 256.75370510389\\
% 10.3912353515625 256.679649902875\\
% 10.4217529296875 256.605747117999\\
% 10.4522705078125 256.531995653981\\
% 10.4827880859375 256.45839443116\\
% 10.5133056640625 256.384942385196\\
% 10.5438232421875 256.311638466793\\
% 10.5743408203125 256.238481641405\\
% 10.6048583984375 256.165470888976\\
% 10.6353759765625 256.092605203659\\
% 10.6658935546875 256.019883593566\\
% 10.6964111328125 255.947305080505\\
% 10.7269287109375 255.874868699732\\
% 10.7574462890625 255.80257349971\\
% 10.7879638671875 255.730418541868\\
% 10.8184814453125 255.658402900364\\
% 10.8489990234375 255.586525661864\\
% 10.8795166015625 255.514785925317\\
% 10.9100341796875 255.443182801732\\
% 10.9405517578125 255.371715413971\\
% 10.9710693359375 255.300382896539\\
% 11.0015869140625 255.229184395377\\
% 11.0321044921875 255.158119067665\\
% 11.0626220703125 255.087186081626\\
% 11.0931396484375 255.016384616335\\
% 11.1236572265625 254.945713861535\\
% 11.1541748046875 254.875173017448\\
% 11.1846923828125 254.804761294599\\
% 11.2152099609375 254.734477913643\\
% 11.2457275390625 254.664322105189\\
% 11.2762451171875 254.594293109636\\
% 11.3067626953125 254.524390177005\\
% 11.3372802734375 254.45461256678\\
% 11.3677978515625 254.384959547752\\
% 11.3983154296875 254.315430397859\\
% 11.4288330078125 254.246024404037\\
% 11.4593505859375 254.176740862078\\
% 11.4898681640625 254.107579076474\\
% 11.5203857421875 254.038538360282\\
% 11.5509033203125 253.969618034983\\
% 11.5814208984375 253.900817430344\\
% 11.6119384765625 253.832135884285\\
% 11.6424560546875 253.763572742748\\
% 11.6729736328125 253.695127359568\\
% 11.7034912109375 253.626799096347\\
% 11.7340087890625 253.558587322328\\
% 11.7645263671875 253.490491414282\\
% 11.7950439453125 253.422510756375\\
% 11.8255615234375 253.354644740066\\
% 11.8560791015625 253.286892763984\\
% 11.8865966796875 253.219254233816\\
% 11.9171142578125 253.151728562204\\
% 11.9476318359375 253.084315168627\\
% 11.9781494140625 253.017013479305\\
% 12.0086669921875 252.949822927088\\
% 12.0391845703125 252.882742951356\\
% 12.0697021484375 252.815772997925\\
% 12.1002197265625 252.748912518937\\
% 12.1307373046875 252.682160972775\\
% 12.1612548828125 252.615517823964\\
% 12.1917724609375 252.548982543076\\
% 12.2222900390625 252.482554606643\\
% 12.2528076171875 252.416233497067\\
% 12.2833251953125 252.350018702529\\
% 12.3138427734375 252.283909716907\\
% 12.3443603515625 252.217906039691\\
% 12.3748779296875 252.152007175894\\
% 12.4053955078125 252.086212635981\\
% 12.4359130859375 252.02052193578\\
% 12.4664306640625 251.954934596407\\
% 12.4969482421875 251.889450144187\\
% 12.5274658203125 251.824068110583\\
% 12.5579833984375 251.758788032115\\
% 12.5885009765625 251.69360945029\\
% 12.6190185546875 251.628531911531\\
% 12.6495361328125 251.563554967105\\
% 12.6800537109375 251.498678173053\\
% 12.7105712890625 251.433901090122\\
% 12.7410888671875 251.369223283701\\
% 12.7716064453125 251.30464432375\\
% 12.8021240234375 251.240163784739\\
% 12.8326416015625 251.175781245583\\
% 12.8631591796875 251.111496289579\\
% 12.8936767578125 251.047308504347\\
% 12.9241943359375 250.983217481765\\
% 12.9547119140625 250.919222817914\\
% 12.9852294921875 250.855324113018\\
% 13.0157470703125 250.791520971385\\
% 13.0462646484375 250.727813001352\\
% 13.0767822265625 250.66419981523\\
% 13.1072998046875 250.600681029248\\
% 13.1378173828125 250.537256263501\\
% 13.1683349609375 250.473925141894\\
% 13.1988525390625 250.410687292092\\
% 13.2293701171875 250.347542345473\\
% 13.2598876953125 250.284489937068\\
% 13.2904052734375 250.22152970552\\
% 13.3209228515625 250.158661293033\\
% 13.3514404296875 250.095884345321\\
% 13.3819580078125 250.033198511567\\
% 13.4124755859375 249.970603444369\\
% 13.4429931640625 249.908098799701\\
% 13.4735107421875 249.845684236865\\
% 13.5040283203125 249.783359418446\\
% 13.5345458984375 249.721124010272\\
% 13.5650634765625 249.658977681368\\
% 13.5955810546875 249.596920103915\\
% 13.6260986328125 249.534950953208\\
% 13.6566162109375 249.473069907616\\
% 13.6871337890625 249.411276648541\\
% 13.7176513671875 249.349570860379\\
% 13.7481689453125 249.287952230483\\
% 13.7786865234375 249.226420449121\\
% 13.8092041015625 249.164975209437\\
% 13.8397216796875 249.103616207422\\
% 13.8702392578125 249.042343141868\\
% 13.9007568359375 248.981155714339\\
% 13.9312744140625 248.920053629132\\
% 13.9617919921875 248.85903659324\\
% 13.9923095703125 248.798104316323\\
% 14.0228271484375 248.73725651067\\
% 14.0533447265625 248.676492891169\\
% 14.0838623046875 248.61581317527\\
% 14.1143798828125 248.555217082954\\
% 14.1448974609375 248.494704336704\\
% 14.1754150390625 248.43427466147\\
% 14.2059326171875 248.373927784641\\
% 14.2364501953125 248.313663436008\\
% 14.2669677734375 248.253481347747\\
% 14.2974853515625 248.193381254372\\
% 14.3280029296875 248.133362892722\\
% 14.3585205078125 248.073426001921\\
% 14.3890380859375 248.013570323357\\
% 14.4195556640625 247.953795600648\\
% 14.4500732421875 247.89410157962\\
% 14.4805908203125 247.834488008275\\
% 14.5111083984375 247.77495463677\\
% 14.5416259765625 247.715501217383\\
% 14.5721435546875 247.656127504493\\
% 14.6026611328125 247.596833254555\\
% 14.6331787109375 247.537618226068\\
% 14.6636962890625 247.47848217956\\
% 14.6942138671875 247.419424877553\\
% 14.7247314453125 247.360446084548\\
% 14.7552490234375 247.301545566994\\
% 14.7857666015625 247.242723093268\\
% 14.8162841796875 247.183978433655\\
% 14.8468017578125 247.125311360317\\
% 14.8773193359375 247.066721647278\\
% 14.9078369140625 247.008209070399\\
% 14.9383544921875 246.949773407353\\
% 14.9688720703125 246.891414437611\\
% 14.9993896484375 246.833131942413\\
% 15.0299072265625 246.774925704751\\
% 15.0604248046875 246.716795509348\\
% 15.0909423828125 246.658741142636\\
% 15.1214599609375 246.600762392739\\
% 15.1519775390625 246.542859049449\\
% 15.1824951171875 246.48503090421\\
% 15.2130126953125 246.427277750096\\
% 15.2435302734375 246.369599381794\\
% 15.2740478515625 246.311995595585\\
% 15.3045654296875 246.254466189324\\
% 15.3350830078125 246.197010962421\\
% 15.3656005859375 246.139629715828\\
% 15.3961181640625 246.082322252016\\
% 15.4266357421875 246.025088374958\\
% 15.4571533203125 245.967927890115\\
% 15.4876708984375 245.910840604415\\
% 15.5181884765625 245.853826326238\\
% 15.5487060546875 245.796884865401\\
% 15.5792236328125 245.740016033138\\
% 15.6097412109375 245.683219642086\\
% 15.6402587890625 245.62649550627\\
% 15.6707763671875 245.569843441085\\
% 15.7012939453125 245.51326326328\\
% 15.7318115234375 245.456754790946\\
% 15.7623291015625 245.400317843498\\
% 15.7928466796875 245.343952241659\\
% 15.8233642578125 245.287657807452\\
% 15.8538818359375 245.231434364175\\
% 15.8843994140625 245.175281736394\\
% 15.9149169921875 245.119199749931\\
% 15.9454345703125 245.063188231838\\
% 15.9759521484375 245.007247010398\\
% 16.0064697265625 244.9513759151\\
% 16.0369873046875 244.895574776634\\
% 16.0675048828125 244.839843426871\\
% 16.0980224609375 244.784181698851\\
% 16.1285400390625 244.728589426775\\
% 16.1590576171875 244.673066445987\\
% 16.1895751953125 244.617612592965\\
% 16.2200927734375 244.562227705302\\
% 16.2506103515625 244.506911621705\\
% 16.2811279296875 244.451664181969\\
% 16.3116455078125 244.396485226979\\
% 16.3421630859375 244.341374598686\\
% 16.3726806640625 244.286332140104\\
% 16.4031982421875 244.231357695294\\
% 16.4337158203125 244.176451109352\\
% 16.4642333984375 244.121612228402\\
% 16.4947509765625 244.066840899582\\
% 16.5252685546875 244.012136971032\\
% 16.5557861328125 243.957500291884\\
% 16.5863037109375 243.902930712255\\
% 16.6168212890625 243.848428083228\\
% 16.6473388671875 243.793992256852\\
% 16.6778564453125 243.739623086123\\
% 16.7083740234375 243.685320424978\\
% 16.7388916015625 243.631084128286\\
% 16.7694091796875 243.576914051832\\
% 16.7999267578125 243.522810052317\\
% 16.8304443359375 243.468771987337\\
% 16.8609619140625 243.414799715383\\
% 16.8914794921875 243.360893095826\\
% 16.9219970703125 243.30705198891\\
% 16.9525146484375 243.253276255743\\
% 16.9830322265625 243.199565758284\\
% 17.0135498046875 243.145920359339\\
% 17.0440673828125 243.092339922551\\
% 17.0745849609375 243.038824312388\\
% 17.1051025390625 242.98537339414\\
% 17.1356201171875 242.931987033901\\
% 17.1661376953125 242.878665098573\\
% 17.1966552734375 242.825407455848\\
% 17.2271728515625 242.772213974203\\
% 17.2576904296875 242.719084522893\\
% 17.2882080078125 242.666018971939\\
% 17.3187255859375 242.613017192126\\
% 17.3492431640625 242.560079054991\\
% 17.3797607421875 242.507204432814\\
% 17.4102783203125 242.454393198615\\
% 17.4407958984375 242.401645226141\\
% 17.4713134765625 242.348960389864\\
% 17.5018310546875 242.296338564969\\
% 17.5323486328125 242.243779627349\\
% 17.5628662109375 242.191283453598\\
% 17.5933837890625 242.138849921001\\
% 17.6239013671875 242.086478907531\\
% 17.6544189453125 242.03417029184\\
% 17.6849365234375 241.981923953252\\
% 17.7154541015625 241.929739771754\\
% 17.7459716796875 241.877617627996\\
% 17.7764892578125 241.825557403274\\
% 17.8070068359375 241.773558979535\\
% 17.8375244140625 241.721622239361\\
% 17.8680419921875 241.669747065969\\
% 17.8985595703125 241.6179333432\\
% 17.9290771484375 241.566180955516\\
% 17.9595947265625 241.514489787991\\
% 17.9901123046875 241.462859726307\\
% 18.0206298828125 241.411290656749\\
% 18.0511474609375 241.359782466194\\
% 18.0816650390625 241.308335042112\\
% 18.1121826171875 241.256948272554\\
% 18.1427001953125 241.205622046149\\
% 18.1732177734375 241.1543562521\\
% 18.2037353515625 241.103150780173\\
% 18.2342529296875 241.052005520697\\
% 18.2647705078125 241.000920364556\\
% 18.2952880859375 240.949895203183\\
% 18.3258056640625 240.898929928557\\
% 18.3563232421875 240.848024433194\\
% 18.3868408203125 240.797178610145\\
% 18.4173583984375 240.746392352988\\
% 18.4478759765625 240.695665555827\\
% 18.4783935546875 240.644998113282\\
% 18.5089111328125 240.594389920487\\
% 18.5394287109375 240.543840873085\\
% 18.5699462890625 240.49335086722\\
% 18.6004638671875 240.442919799538\\
% 18.6309814453125 240.392547567174\\
% 18.6614990234375 240.342234067758\\
% 18.6920166015625 240.291979199397\\
% 18.7225341796875 240.241782860683\\
% 18.7530517578125 240.191644950682\\
% 18.7835693359375 240.141565368928\\
% 18.8140869140625 240.091544015423\\
% 18.8446044921875 240.041580790629\\
% 18.8751220703125 239.991675595467\\
% 18.9056396484375 239.941828331309\\
% 18.9361572265625 239.892038899976\\
% 18.9666748046875 239.842307203733\\
% 18.9971923828125 239.792633145284\\
% 19.0277099609375 239.743016627771\\
% 19.0582275390625 239.693457554766\\
% 19.0887451171875 239.64395583027\\
% 19.1192626953125 239.594511358706\\
% 19.1497802734375 239.545124044918\\
% 19.1802978515625 239.495793794163\\
% 19.2108154296875 239.446520512113\\
% 19.2413330078125 239.397304104848\\
% 19.2718505859375 239.348144478851\\
% 19.3023681640625 239.299041541004\\
% 19.3328857421875 239.249995198589\\
% 19.3634033203125 239.201005359278\\
% 19.3939208984375 239.152071931134\\
% 19.4244384765625 239.103194822606\\
% 19.4549560546875 239.054373942525\\
% 19.4854736328125 239.005609200101\\
% 19.5159912109375 238.956900504919\\
% 19.5465087890625 238.908247766935\\
% 19.5770263671875 238.859650896475\\
% 19.6075439453125 238.811109804229\\
% 19.6380615234375 238.762624401251\\
% 19.6685791015625 238.714194598951\\
% 19.6990966796875 238.665820309094\\
% 19.7296142578125 238.617501443799\\
% 19.7601318359375 238.569237915534\\
% 19.7906494140625 238.521029637111\\
% 19.8211669921875 238.472876521686\\
% 19.8516845703125 238.424778482754\\
% 19.8822021484375 238.376735434146\\
% 19.9127197265625 238.328747290028\\
% 19.9432373046875 238.280813964896\\
% 19.9737548828125 238.232935373571\\
% 20.0042724609375 238.185111431204\\
% 20.0347900390625 238.137342053264\\
% 20.0653076171875 238.089627155539\\
% 20.0958251953125 238.041966654134\\
% 20.1263427734375 237.994360465469\\
% 20.1568603515625 237.946808506272\\
% 20.1873779296875 237.899310693579\\
% 20.2178955078125 237.851866944734\\
% 20.2484130859375 237.804477177381\\
% 20.2789306640625 237.757141309465\\
% 20.3094482421875 237.709859259227\\
% 20.3399658203125 237.662630945205\\
% 20.3704833984375 237.615456286227\\
% 20.4010009765625 237.568335201411\\
% 20.4315185546875 237.521267610165\\
% 20.4620361328125 237.474253432178\\
% 20.4925537109375 237.427292587422\\
% 20.5230712890625 237.380384996152\\
% 20.5535888671875 237.333530578895\\
% 20.5841064453125 237.286729256459\\
% 20.6146240234375 237.239980949919\\
% 20.6451416015625 237.193285580626\\
% 20.6756591796875 237.146643070195\\
% 20.7061767578125 237.10005334051\\
% 20.7366943359375 237.053516313714\\
% 20.7672119140625 237.007031912218\\
% 20.7977294921875 236.960600058686\\
% 20.8282470703125 236.914220676043\\
% 20.8587646484375 236.867893687468\\
% 20.8892822265625 236.821619016392\\
% 20.9197998046875 236.775396586498\\
% 20.9503173828125 236.729226321717\\
% 20.9808349609375 236.683108146226\\
% 21.0113525390625 236.637041984448\\
% 21.0418701171875 236.591027761046\\
% 21.0723876953125 236.545065400927\\
% 21.1029052734375 236.499154829236\\
% 21.1334228515625 236.45329597135\\
% 21.1639404296875 236.407488752887\\
% 21.1944580078125 236.361733099694\\
% 21.2249755859375 236.316028937851\\
% 21.2554931640625 236.270376193665\\
% 21.2860107421875 236.224774793673\\
% 21.3165283203125 236.179224664634\\
% 21.3470458984375 236.133725733534\\
% 21.3775634765625 236.088277927578\\
% 21.4080810546875 236.042881174193\\
% 21.4385986328125 235.997535401024\\
% 21.4691162109375 235.952240535931\\
% 21.4996337890625 235.906996506992\\
% 21.5301513671875 235.861803242494\\
% 21.5606689453125 235.81666067094\\
% 21.5911865234375 235.771568721039\\
% 21.6217041015625 235.726527321711\\
% 21.6522216796875 235.681536402081\\
% 21.6827392578125 235.636595891479\\
% 21.7132568359375 235.591705719439\\
% 21.7437744140625 235.546865815698\\
% 21.7742919921875 235.50207611019\\
% 21.8048095703125 235.457336533051\\
% 21.8353271484375 235.412647014612\\
% 21.8658447265625 235.368007485401\\
% 21.8963623046875 235.323417876141\\
% 21.9268798828125 235.278878117745\\
% 21.9573974609375 235.234388141319\\
% 21.9879150390625 235.18994787816\\
% 22.0184326171875 235.145557259751\\
% 22.0489501953125 235.101216217764\\
% 22.0794677734375 235.056924684056\\
% 22.1099853515625 235.012682590667\\
% 22.1405029296875 234.968489869821\\
% 22.1710205078125 234.924346453924\\
% 22.2015380859375 234.880252275562\\
% 22.2320556640625 234.836207267498\\
% 22.2625732421875 234.792211362675\\
% 22.2930908203125 234.748264494211\\
% 22.3236083984375 234.7043665954\\
% 22.3541259765625 234.660517599707\\
% 22.3846435546875 234.616717440773\\
% 22.4151611328125 234.572966052407\\
% 22.4456787109375 234.529263368592\\
% 22.4761962890625 234.485609323474\\
% 22.5067138671875 234.442003851372\\
% 22.5372314453125 234.398446886768\\
% 22.5677490234375 234.35493836431\\
% 22.5982666015625 234.311478218811\\
% 22.6287841796875 234.268066385246\\
% 22.6593017578125 234.224702798751\\
% 22.6898193359375 234.181387394624\\
% 22.7203369140625 234.138120108321\\
% 22.7508544921875 234.094900875458\\
% 22.7813720703125 234.051729631807\\
% 22.8118896484375 234.008606313297\\
% 22.8424072265625 233.96553085601\\
% 22.8729248046875 233.922503196186\\
% 22.9034423828125 233.879523270213\\
% 22.9339599609375 233.836591014635\\
% 22.9644775390625 233.793706366146\\
% 22.9949951171875 233.750869261588\\
% 23.0255126953125 233.708079637954\\
% 23.0560302734375 233.665337432384\\
% 23.0865478515625 233.622642582164\\
% 23.1170654296875 233.579995024729\\
% 23.1475830078125 233.537394697656\\
% 23.1781005859375 233.494841538666\\
% 23.2086181640625 233.452335485624\\
% 23.2391357421875 233.409876476538\\
% 23.2696533203125 233.367464449556\\
% 23.3001708984375 233.325099342966\\
% 23.3306884765625 233.282781095197\\
% 23.3612060546875 233.240509644813\\
% 23.3917236328125 233.198284930522\\
% 23.4222412109375 233.15610689116\\
% 23.4527587890625 233.113975465707\\
% 23.4832763671875 233.071890593273\\
% 23.5137939453125 233.029852213104\\
% 23.5443115234375 232.987860264579\\
% 23.5748291015625 232.945914687208\\
% 23.6053466796875 232.904015420636\\
% 23.6358642578125 232.862162404634\\
% 23.6663818359375 232.820355579109\\
% 23.6968994140625 232.778594884092\\
% 23.7274169921875 232.736880259745\\
% 23.7579345703125 232.695211646356\\
% 23.7884521484375 232.653588984342\\
% 23.8189697265625 232.612012214244\\
% 23.8494873046875 232.57048127673\\
% 23.8800048828125 232.528996112591\\
% 23.9105224609375 232.487556662742\\
% 23.9410400390625 232.446162868224\\
% 23.9715576171875 232.404814670196\\
% 24.0020751953125 232.363512009941\\
% 24.0325927734375 232.322254828863\\
% 24.0631103515625 232.281043068486\\
% 24.0936279296875 232.239876670453\\
% 24.1241455078125 232.198755576525\\
% 24.1546630859375 232.157679728585\\
% 24.1851806640625 232.116649068627\\
% 24.2156982421875 232.075663538769\\
% 24.2462158203125 232.03472308124\\
% 24.2767333984375 231.993827638387\\
% 24.3072509765625 231.95297715267\\
% 24.3377685546875 231.912171566665\\
% 24.3682861328125 231.87141082306\\
% 24.3988037109375 231.830694864657\\
% 24.4293212890625 231.79002363437\\
% 24.4598388671875 231.749397075224\\
% 24.4903564453125 231.708815130356\\
% 24.5208740234375 231.668277743014\\
% 24.5513916015625 231.627784856553\\
% 24.5819091796875 231.587336414441\\
% 24.6124267578125 231.546932360253\\
% 24.6429443359375 231.50657263767\\
% 24.6734619140625 231.466257190485\\
% 24.7039794921875 231.425985962594\\
% 24.7344970703125 231.385758898001\\
% 24.7650146484375 231.345575940817\\
% 24.7955322265625 231.305437035255\\
% 24.8260498046875 231.265342125637\\
% 24.8565673828125 231.225291156385\\
% 24.8870849609375 231.185284072029\\
% 24.9176025390625 231.145320817197\\
% 24.9481201171875 231.105401336625\\
% 24.9786376953125 231.065525575148\\
% 25.0091552734375 231.025693477702\\
% 25.0396728515625 230.985904989326\\
% 25.0701904296875 230.946160055158\\
% 25.1007080078125 230.906458620439\\
% 25.1312255859375 230.866800630506\\
% 25.1617431640625 230.827186030796\\
% 25.1922607421875 230.787614766847\\
% 25.2227783203125 230.748086784293\\
% 25.2532958984375 230.708602028865\\
% 25.2838134765625 230.669160446393\\
% 25.3143310546875 230.629761982802\\
% 25.3448486328125 230.590406584117\\
% 25.3753662109375 230.551094196455\\
% 25.4058837890625 230.511824766027\\
% 25.4364013671875 230.472598239145\\
% 25.4669189453125 230.433414562211\\
% 25.4974365234375 230.394273681721\\
% 25.5279541015625 230.355175544267\\
% 25.5584716796875 230.316120096532\\
% 25.5889892578125 230.277107285294\\
% 25.6195068359375 230.238137057422\\
% 25.6500244140625 230.199209359876\\
% 25.6805419921875 230.160324139709\\
% 25.7110595703125 230.121481344066\\
% 25.7415771484375 230.08268092018\\
% 25.7720947265625 230.043922815377\\
% 25.8026123046875 230.005206977072\\
% 25.8331298828125 229.966533352769\\
% 25.8636474609375 229.927901890062\\
% 25.8941650390625 229.889312536633\\
% 25.9246826171875 229.850765240253\\
% 25.9552001953125 229.812259948782\\
% 25.9857177734375 229.773796610166\\
% 26.0162353515625 229.735375172438\\
% 26.0467529296875 229.69699558372\\
% 26.0772705078125 229.65865779222\\
% 26.1077880859375 229.620361746231\\
% 26.1383056640625 229.582107394133\\
% 26.1688232421875 229.543894684391\\
% 26.1993408203125 229.505723565556\\
% 26.2298583984375 229.467593986264\\
% 26.2603759765625 229.429505895233\\
% 26.2908935546875 229.391459241269\\
% 26.3214111328125 229.353453973258\\
% 26.3519287109375 229.315490040173\\
% 26.3824462890625 229.277567391069\\
% 26.4129638671875 229.239685975082\\
% 26.4434814453125 229.201845741433\\
% 26.4739990234375 229.164046639426\\
% 26.5045166015625 229.126288618443\\
% 26.5350341796875 229.088571627951\\
% 26.5655517578125 229.050895617498\\
% 26.5960693359375 229.013260536711\\
% 26.6265869140625 228.9756663353\\
% 26.6571044921875 228.938112963056\\
% 26.6876220703125 228.900600369847\\
% 26.7181396484375 228.863128505623\\
% 26.7486572265625 228.825697320413\\
% 26.7791748046875 228.788306764327\\
% 26.8096923828125 228.75095678755\\
% 26.8402099609375 228.713647340351\\
% 26.8707275390625 228.676378373073\\
% 26.9012451171875 228.639149836139\\
% 26.9317626953125 228.60196168005\\
% 26.9622802734375 228.564813855385\\
% 26.9927978515625 228.527706312799\\
% 27.0233154296875 228.490639003025\\
% 27.0538330078125 228.453611876874\\
% 27.0843505859375 228.416624885231\\
% 27.1148681640625 228.37967797906\\
% 27.1453857421875 228.342771109399\\
% 27.1759033203125 228.305904227363\\
% 27.2064208984375 228.269077284143\\
% 27.2369384765625 228.232290231002\\
% 27.2674560546875 228.195543019284\\
% 27.2979736328125 228.158835600404\\
% 27.3284912109375 228.122167925851\\
% 27.3590087890625 228.08553994719\\
% 27.3895263671875 228.04895161606\\
% 27.4200439453125 228.012402884174\\
% 27.4505615234375 227.975893703319\\
% 27.4810791015625 227.939424025354\\
% 27.5115966796875 227.902993802212\\
% 27.5421142578125 227.866602985901\\
% 27.5726318359375 227.830251528498\\
% 27.6031494140625 227.793939382155\\
% 27.6336669921875 227.757666499097\\
% 27.6641845703125 227.72143283162\\
% 27.6947021484375 227.685238332092\\
% 27.7252197265625 227.649082952952\\
% 27.7557373046875 227.612966646713\\
% 27.7862548828125 227.576889365957\\
% 27.8167724609375 227.540851063337\\
% 27.8472900390625 227.50485169158\\
% 27.8778076171875 227.46889120348\\
% 27.9083251953125 227.432969551903\\
% 27.9388427734375 227.397086689785\\
% 27.9693603515625 227.361242570135\\
% 27.9998779296875 227.325437146027\\
% 28.0303955078125 227.289670370608\\
% 28.0609130859375 227.253942197095\\
% 28.0914306640625 227.218252578772\\
% 28.1219482421875 227.182601468993\\
% 28.1524658203125 227.146988821182\\
% 28.1829833984375 227.111414588833\\
% 28.2135009765625 227.075878725504\\
% 28.2440185546875 227.040381184826\\
% 28.2745361328125 227.004921920497\\
% 28.3050537109375 226.969500886283\\
% 28.3355712890625 226.934118036016\\
% 28.3660888671875 226.8987733236\\
% 28.3966064453125 226.863466703002\\
% 28.4271240234375 226.82819812826\\
% 28.4576416015625 226.792967553478\\
% 28.4881591796875 226.757774932826\\
% 28.5186767578125 226.722620220543\\
% 28.5491943359375 226.687503370933\\
% 28.5797119140625 226.652424338368\\
% 28.6102294921875 226.617383077287\\
% 28.6407470703125 226.582379542192\\
% 28.6712646484375 226.547413687654\\
% 28.7017822265625 226.51248546831\\
% 28.7322998046875 226.477594838862\\
% 28.7628173828125 226.442741754078\\
% 28.7933349609375 226.40792616879\\
% 28.8238525390625 226.373148037899\\
% 28.8543701171875 226.338407316368\\
% 28.8848876953125 226.303703959226\\
% 28.9154052734375 226.269037921568\\
% 28.9459228515625 226.234409158552\\
% 28.9764404296875 226.199817625402\\
% 29.0069580078125 226.165263277406\\
% 29.0374755859375 226.130746069917\\
% 29.0679931640625 226.096265958351\\
% 29.0985107421875 226.06182289819\\
% 29.1290283203125 226.027416844979\\
% 29.1595458984375 225.993047754325\\
% 29.1900634765625 225.958715581902\\
% 29.2205810546875 225.924420283445\\
% 29.2510986328125 225.890161814756\\
% 29.2816162109375 225.855940131694\\
% 29.3121337890625 225.821755190188\\
% 29.3426513671875 225.787606946225\\
% 29.3731689453125 225.753495355858\\
% 29.4036865234375 225.719420375201\\
% 29.4342041015625 225.685381960433\\
% 29.4647216796875 225.651380067793\\
% 29.4952392578125 225.617414653583\\
% 29.5257568359375 225.583485674168\\
% 29.5562744140625 225.549593085976\\
% 29.5867919921875 225.515736845495\\
% 29.6173095703125 225.481916909276\\
% 29.6478271484375 225.448133233932\\
% 29.6783447265625 225.414385776137\\
% 29.7088623046875 225.380674492628\\
% 29.7393798828125 225.346999340203\\
% 29.7698974609375 225.31336027572\\
% 29.8004150390625 225.279757256099\\
% 29.8309326171875 225.246190238323\\
% 29.8614501953125 225.212659179433\\
% 29.8919677734375 225.179164036533\\
% 29.9224853515625 225.145704766787\\
% 29.9530029296875 225.112281327421\\
% 29.9835205078125 225.078893675719\\
% 30.0140380859375 225.045541769028\\
% 30.0445556640625 225.012225564754\\
% 30.0750732421875 224.978945020364\\
% 30.1055908203125 224.945700093385\\
% 30.1361083984375 224.912490741405\\
% 30.1666259765625 224.879316922069\\
% 30.1971435546875 224.846178593085\\
% 30.2276611328125 224.81307571222\\
% 30.2581787109375 224.7800082373\\
% 30.2886962890625 224.746976126211\\
% 30.3192138671875 224.713979336899\\
% 30.3497314453125 224.681017827369\\
% 30.3802490234375 224.648091555683\\
% 30.4107666015625 224.615200479968\\
% 30.4412841796875 224.582344558403\\
% 30.4718017578125 224.549523749231\\
% 30.5023193359375 224.516738010752\\
% 30.5328369140625 224.483987301327\\
% 30.5633544921875 224.451271579371\\
% 30.5938720703125 224.418590803363\\
% 30.6243896484375 224.385944931838\\
% 30.6549072265625 224.353333923388\\
% 30.6854248046875 224.320757736666\\
% 30.7159423828125 224.288216330383\\
% 30.7464599609375 224.255709663306\\
% 30.7769775390625 224.223237694263\\
% 30.8074951171875 224.190800382138\\
% 30.8380126953125 224.158397685873\\
% 30.8685302734375 224.12602956447\\
% 30.8990478515625 224.093695976985\\
% 30.9295654296875 224.061396882535\\
% 30.9600830078125 224.029132240295\\
% 30.9906005859375 223.996902009494\\
% 31.0211181640625 223.96470614942\\
% 31.0516357421875 223.932544619421\\
% 31.0821533203125 223.900417378898\\
% 31.1126708984375 223.868324387313\\
% 31.1431884765625 223.836265604182\\
% 31.1737060546875 223.80424098908\\
% 31.2042236328125 223.772250501638\\
% 31.2347412109375 223.740294101545\\
% 31.2652587890625 223.708371748545\\
% 31.2957763671875 223.676483402441\\
% 31.3262939453125 223.64462902309\\
% 31.3568115234375 223.612808570409\\
% 31.3873291015625 223.581022004367\\
% 31.4178466796875 223.549269284994\\
% 31.4483642578125 223.517550372373\\
% 31.4788818359375 223.485865226645\\
% 31.5093994140625 223.454213808007\\
% 31.5399169921875 223.422596076712\\
% 31.5704345703125 223.391011993068\\
% 31.6009521484375 223.35946151744\\
% 31.6314697265625 223.327944610249\\
% 31.6619873046875 223.296461231972\\
% 31.6925048828125 223.265011343142\\
% 31.7230224609375 223.233594904345\\
% 31.7535400390625 223.202211876228\\
% 31.7840576171875 223.170862219487\\
% 31.8145751953125 223.139545894878\\
% 31.8450927734375 223.108262863212\\
% 31.8756103515625 223.077013085354\\
% 31.9061279296875 223.045796522225\\
% 31.9366455078125 223.0146131348\\
% 31.9671630859375 222.983462884112\\
% 31.9976806640625 222.952345731245\\
% 32.0281982421875 222.921261637343\\
% 32.0587158203125 222.890210563601\\
% 32.0892333984375 222.85919247127\\
% 32.1197509765625 222.828207321655\\
% 32.1502685546875 222.797255076119\\
% 32.1807861328125 222.766335696076\\
% 32.2113037109375 222.735449142997\\
% 32.2418212890625 222.704595378406\\
% 32.2723388671875 222.673774363882\\
% 32.3028564453125 222.642986061058\\
% 32.3333740234375 222.612230431624\\
% 32.3638916015625 222.581507437321\\
% 32.3944091796875 222.550817039946\\
% 32.4249267578125 222.520159201349\\
% 32.4554443359375 222.489533883435\\
% 32.4859619140625 222.458941048164\\
% 32.5164794921875 222.428380657547\\
% 32.5469970703125 222.397852673654\\
% 32.5775146484375 222.367357058603\\
% 32.6080322265625 222.336893774569\\
% 32.6385498046875 222.306462783782\\
% 32.6690673828125 222.276064048524\\
% 32.6995849609375 222.245697531129\\
% 32.7301025390625 222.215363193988\\
% 32.7606201171875 222.185060999544\\
% 32.7911376953125 222.154790910292\\
% 32.8216552734375 222.124552888783\\
% 32.8521728515625 222.094346897621\\
% 32.8826904296875 222.06417289946\\
% 32.9132080078125 222.034030857013\\
% 32.9437255859375 222.00392073304\\
% 32.9742431640625 221.97384249036\\
% 33.0047607421875 221.943796091841\\
% 33.0352783203125 221.913781500405\\
% 33.0657958984375 221.883798679027\\
% 33.0963134765625 221.853847590736\\
% 33.1268310546875 221.823928198613\\
% 33.1573486328125 221.794040465791\\
% 33.1878662109375 221.764184355459\\
% 33.2183837890625 221.734359830853\\
% 33.2489013671875 221.704566855268\\
% 33.2794189453125 221.674805392047\\
% 33.3099365234375 221.645075404588\\
% 33.3404541015625 221.61537685634\\
% 33.3709716796875 221.585709710805\\
% 33.4014892578125 221.556073931539\\
% 33.4320068359375 221.526469482147\\
% 33.4625244140625 221.496896326289\\
% 33.4930419921875 221.467354427677\\
% 33.5235595703125 221.437843750074\\
% 33.5540771484375 221.408364257295\\
% 33.5845947265625 221.378915913208\\
% 33.6151123046875 221.349498681733\\
% 33.6456298828125 221.320112526843\\
% 33.6761474609375 221.290757412561\\
% 33.7066650390625 221.261433302962\\
% 33.7371826171875 221.232140162173\\
% 33.7677001953125 221.202877954375\\
% 33.7982177734375 221.173646643797\\
% 33.8287353515625 221.144446194723\\
% 33.8592529296875 221.115276571487\\
% 33.8897705078125 221.086137738474\\
% 33.9202880859375 221.057029660123\\
% 33.9508056640625 221.027952300921\\
% 33.9813232421875 220.998905625409\\
% 34.0118408203125 220.96988959818\\
% 34.0423583984375 220.940904183875\\
% 34.0728759765625 220.911949347188\\
% 34.1033935546875 220.883025052867\\
% 34.1339111328125 220.854131265707\\
% 34.1644287109375 220.825267950556\\
% 34.1949462890625 220.796435072314\\
% 34.2254638671875 220.76763259593\\
% 34.2559814453125 220.738860486405\\
% 34.2864990234375 220.710118708791\\
% 34.3170166015625 220.681407228192\\
% 34.3475341796875 220.65272600976\\
% 34.3780517578125 220.624075018701\\
% 34.4085693359375 220.59545422027\\
% 34.4390869140625 220.566863579773\\
% 34.4696044921875 220.538303062566\\
% 34.5001220703125 220.509772634058\\
% 34.5306396484375 220.481272259706\\
% 34.5611572265625 220.452801905018\\
% 34.5916748046875 220.424361535554\\
% 34.6221923828125 220.395951116922\\
% 34.6527099609375 220.367570614784\\
% 34.6832275390625 220.339219994849\\
% 34.7137451171875 220.310899222877\\
% 34.7442626953125 220.28260826468\\
% 34.7747802734375 220.254347086118\\
% 34.8052978515625 220.226115653102\\
% 34.8358154296875 220.197913931594\\
% 34.8663330078125 220.169741887607\\
% 34.8968505859375 220.141599487199\\
% 34.9273681640625 220.113486696484\\
% 34.9578857421875 220.085403481622\\
% 34.9884033203125 220.057349808825\\
% 35.0189208984375 220.029325644355\\
% 35.0494384765625 220.001330954521\\
% 35.0799560546875 219.973365705685\\
% 35.1104736328125 219.945429864258\\
% 35.1409912109375 219.917523396699\\
% 35.1715087890625 219.889646269519\\
% 35.2020263671875 219.861798449277\\
% 35.2325439453125 219.833979902582\\
% 35.2630615234375 219.806190596092\\
% 35.2935791015625 219.778430496516\\
% 35.3240966796875 219.750699570611\\
% 35.3546142578125 219.722997785183\\
% 35.3851318359375 219.69532510709\\
% 35.4156494140625 219.667681503237\\
% 35.4461669921875 219.640066940577\\
% 35.4766845703125 219.612481386116\\
% 35.5072021484375 219.584924806906\\
% 35.5377197265625 219.557397170049\\
% 35.5682373046875 219.529898442698\\
% 35.5987548828125 219.502428592053\\
% 35.6292724609375 219.474987585361\\
% 35.6597900390625 219.447575389924\\
% 35.6903076171875 219.420191973087\\
% 35.7208251953125 219.392837302246\\
% 35.7513427734375 219.365511344848\\
% 35.7818603515625 219.338214068385\\
% 35.8123779296875 219.3109454404\\
% 35.8428955078125 219.283705428485\\
% 35.8734130859375 219.256494000278\\
% 35.9039306640625 219.229311123471\\
% 35.9344482421875 219.202156765799\\
% 35.9649658203125 219.175030895047\\
% 35.9954833984375 219.14793347905\\
% 36.0260009765625 219.120864485692\\
% 36.0565185546875 219.093823882902\\
% 36.0870361328125 219.066811638661\\
% 36.1175537109375 219.039827720996\\
% 36.1480712890625 219.012872097984\\
% 36.1785888671875 218.985944737749\\
% 36.2091064453125 218.959045608463\\
% 36.2396240234375 218.932174678348\\
% 36.2701416015625 218.905331915672\\
% 36.3006591796875 218.878517288752\\
% 36.3311767578125 218.851730765954\\
% 36.3616943359375 218.824972315691\\
% 36.3922119140625 218.798241906422\\
% 36.4227294921875 218.77153950666\\
% 36.4532470703125 218.744865084958\\
% 36.4837646484375 218.718218609923\\
% 36.5142822265625 218.691600050206\\
% 36.5447998046875 218.66500937451\\
% 36.5753173828125 218.63844655158\\
% 36.6058349609375 218.611911550213\\
% 36.6363525390625 218.585404339254\\
% 36.6668701171875 218.558924887592\\
% 36.6973876953125 218.532473164166\\
% 36.7279052734375 218.506049137963\\
% 36.7584228515625 218.479652778016\\
% 36.7889404296875 218.453284053406\\
% 36.8194580078125 218.426942933261\\
% 36.8499755859375 218.400629386759\\
% 36.8804931640625 218.374343383121\\
% 36.9110107421875 218.348084891619\\
% 36.9415283203125 218.321853881569\\
% 36.9720458984375 218.295650322339\\
% 37.0025634765625 218.269474183338\\
% 37.0330810546875 218.243325434027\\
% 37.0635986328125 218.217204043911\\
% 37.0941162109375 218.191109982546\\
% 37.1246337890625 218.16504321953\\
% 37.1551513671875 218.139003724512\\
% 37.1856689453125 218.112991467185\\
% 37.2161865234375 218.087006417292\\
% 37.2467041015625 218.06104854462\\
% 37.2772216796875 218.035117819003\\
% 37.3077392578125 218.009214210326\\
% 37.3382568359375 217.983337688514\\
% 37.3687744140625 217.957488223545\\
% 37.3992919921875 217.931665785439\\
% 37.4298095703125 217.905870344265\\
% 37.4603271484375 217.880101870138\\
% 37.4908447265625 217.85436033322\\
% 37.5213623046875 217.828645703719\\
% 37.5518798828125 217.80295795189\\
% 37.5823974609375 217.777297048034\\
% 37.6129150390625 217.751662962497\\
% 37.6434326171875 217.726055665675\\
% 37.6739501953125 217.700475128006\\
% 37.7044677734375 217.674921319978\\
% 37.7349853515625 217.649394212124\\
% 37.7655029296875 217.623893775021\\
% 37.7960205078125 217.598419979296\\
% 37.8265380859375 217.572972795618\\
% 37.8570556640625 217.547552194706\\
% 37.8875732421875 217.522158147322\\
% 37.9180908203125 217.496790624276\\
% 37.9486083984375 217.471449596424\\
% 37.9791259765625 217.446135034666\\
% 38.0096435546875 217.420846909949\\
% 38.0401611328125 217.395585193267\\
% 38.0706787109375 217.370349855657\\
% 38.1011962890625 217.345140868205\\
% 38.1317138671875 217.319958202041\\
% 38.1622314453125 217.294801828341\\
% 38.1927490234375 217.269671718326\\
% 38.2232666015625 217.244567843264\\
% 38.2537841796875 217.219490174467\\
% 38.2843017578125 217.194438683294\\
% 38.3148193359375 217.16941334115\\
% 38.3453369140625 217.144414119482\\
% 38.3758544921875 217.119440989786\\
% 38.4063720703125 217.094493923602\\
% 38.4368896484375 217.069572892516\\
% 38.4674072265625 217.044677868159\\
% 38.4979248046875 217.019808822207\\
% 38.5284423828125 216.994965726381\\
% 38.5589599609375 216.970148552448\\
% 38.5894775390625 216.94535727222\\
% 38.6199951171875 216.920591857553\\
% 38.6505126953125 216.89585228035\\
% 38.6810302734375 216.871138512557\\
% 38.7115478515625 216.846450526168\\
% 38.7420654296875 216.821788293218\\
% 38.7725830078125 216.797151785791\\
% 38.8031005859375 216.772540976012\\
% 38.8336181640625 216.747955836054\\
% 38.8641357421875 216.723396338133\\
% 38.8946533203125 216.698862454511\\
% 38.9251708984375 216.674354157494\\
% 38.9556884765625 216.649871419433\\
% 38.9862060546875 216.625414212723\\
% 39.0167236328125 216.600982509805\\
% 39.0472412109375 216.576576283164\\
% 39.0777587890625 216.552195505328\\
% 39.1082763671875 216.527840148872\\
% 39.1387939453125 216.503510186415\\
% 39.1693115234375 216.479205590618\\
% 39.1998291015625 216.45492633419\\
% 39.2303466796875 216.430672389882\\
% 39.2608642578125 216.40644373049\\
% 39.2913818359375 216.382240328855\\
% 39.3218994140625 216.358062157861\\
% 39.3524169921875 216.333909190437\\
% 39.3829345703125 216.309781399556\\
% 39.4134521484375 216.285678758234\\
% 39.4439697265625 216.261601239535\\
% 39.4744873046875 216.237548816562\\
% 39.5050048828125 216.213521462466\\
% 39.5355224609375 216.189519150439\\
% 39.5660400390625 216.165541853719\\
% 39.5965576171875 216.141589545589\\
% 39.6270751953125 216.117662199371\\
% 39.6575927734375 216.093759788437\\
% 39.6881103515625 216.069882286199\\
% 39.7186279296875 216.046029666112\\
% 39.7491455078125 216.02220190168\\
% 39.7796630859375 215.998398966444\\
% 39.8101806640625 215.974620833993\\
% 39.8406982421875 215.95086747796\\
% 39.8712158203125 215.927138872018\\
% 39.9017333984375 215.903434989886\\
% 39.9322509765625 215.879755805327\\
% 39.9627685546875 215.856101292146\\
% 39.9932861328125 215.832471424194\\
% 40.0238037109375 215.80886617536\\
% 40.0543212890625 215.785285519583\\
% 40.0848388671875 215.761729430842\\
% 40.1153564453125 215.738197883159\\
% 40.1458740234375 215.714690850598\\
% 40.1763916015625 215.691208307272\\
% 40.2069091796875 215.66775022733\\
% 40.2374267578125 215.644316584969\\
% 40.2679443359375 215.620907354427\\
% 40.2984619140625 215.597522509986\\
% 40.3289794921875 215.574162025971\\
% 40.3594970703125 215.55082587675\\
% 40.3900146484375 215.527514036733\\
% 40.4205322265625 215.504226480374\\
% 40.4510498046875 215.48096318217\\
% 40.4815673828125 215.45772411666\\
% 40.5120849609375 215.434509258426\\
% 40.5426025390625 215.411318582093\\
% 40.5731201171875 215.38815206233\\
% 40.6036376953125 215.365009673846\\
% 40.6341552734375 215.341891391396\\
% 40.6646728515625 215.318797189774\\
% 40.6951904296875 215.295727043821\\
% 40.7257080078125 215.272680928416\\
% 40.7562255859375 215.249658818483\\
% 40.7867431640625 215.226660688989\\
% 40.8172607421875 215.203686514941\\
% 40.8477783203125 215.180736271392\\
% 40.8782958984375 215.157809933435\\
% 40.9088134765625 215.134907476205\\
% 40.9393310546875 215.112028874882\\
% 40.9698486328125 215.089174104684\\
% 41.0003662109375 215.066343140875\\
% 41.0308837890625 215.043535958759\\
% 41.0614013671875 215.020752533685\\
% 41.0919189453125 214.997992841041\\
% 41.1224365234375 214.975256856258\\
% 41.1529541015625 214.95254455481\\
% 41.1834716796875 214.929855912212\\
% 41.2139892578125 214.907190904022\\
% 41.2445068359375 214.88454950584\\
% 41.2750244140625 214.861931693306\\
% 41.3055419921875 214.839337442104\\
% 41.3360595703125 214.816766727959\\
% 41.3665771484375 214.794219526639\\
% 41.3970947265625 214.771695813951\\
% 41.4276123046875 214.749195565747\\
% 41.4581298828125 214.726718757917\\
% 41.4886474609375 214.704265366398\\
% 41.5191650390625 214.681835367163\\
% 41.5496826171875 214.65942873623\\
% 41.5802001953125 214.637045449658\\
% 41.6107177734375 214.614685483547\\
% 41.6412353515625 214.592348814039\\
% 41.6717529296875 214.570035417317\\
% 41.7022705078125 214.547745269606\\
% 41.7327880859375 214.525478347172\\
% 41.7633056640625 214.503234626322\\
% 41.7938232421875 214.481014083405\\
% 41.8243408203125 214.458816694811\\
% 41.8548583984375 214.436642436971\\
% 41.8853759765625 214.414491286359\\
% 41.9158935546875 214.392363219486\\
% 41.9464111328125 214.37025821291\\
% 41.9769287109375 214.348176243224\\
% 42.0074462890625 214.326117287068\\
% 42.0379638671875 214.304081321117\\
% 42.0684814453125 214.282068322093\\
% 42.0989990234375 214.260078266754\\
% 42.1295166015625 214.238111131902\\
% 42.1600341796875 214.216166894379\\
% 42.1905517578125 214.194245531068\\
% 42.2210693359375 214.172347018891\\
% 42.2515869140625 214.150471334814\\
% 42.2821044921875 214.128618455842\\
% 42.3126220703125 214.10678835902\\
% 42.3431396484375 214.084981021436\\
% 42.3736572265625 214.063196420216\\
% 42.4041748046875 214.041434532529\\
% 42.4346923828125 214.019695335583\\
% 42.4652099609375 213.997978806628\\
% 42.4957275390625 213.976284922951\\
% 42.5262451171875 213.954613661884\\
% 42.5567626953125 213.932965000798\\
% 42.5872802734375 213.911338917103\\
% 42.6177978515625 213.88973538825\\
% 42.6483154296875 213.868154391731\\
% 42.6788330078125 213.846595905078\\
% 42.7093505859375 213.825059905864\\
% 42.7398681640625 213.8035463717\\
% 42.7703857421875 213.78205528024\\
% 42.8009033203125 213.760586609176\\
% 42.8314208984375 213.739140336242\\
% 42.8619384765625 213.71771643921\\
% 42.8924560546875 213.696314895894\\
% 42.9229736328125 213.674935684147\\
% 42.9534912109375 213.653578781862\\
% 42.9840087890625 213.632244166973\\
% 43.0145263671875 213.610931817452\\
% 43.0450439453125 213.589641711313\\
% 43.0755615234375 213.568373826608\\
% 43.1060791015625 213.547128141431\\
% 43.1365966796875 213.525904633914\\
% 43.1671142578125 213.504703282229\\
% 43.1976318359375 213.483524064588\\
% 43.2281494140625 213.462366959243\\
% 43.2586669921875 213.441231944486\\
% 43.2891845703125 213.420118998646\\
% 43.3197021484375 213.399028100097\\
% 43.3502197265625 213.377959227246\\
% 43.3807373046875 213.356912358544\\
% 43.4112548828125 213.335887472481\\
% 43.4417724609375 213.314884547585\\
% 43.4722900390625 213.293903562423\\
% 43.5028076171875 213.272944495604\\
% 43.5333251953125 213.252007325775\\
% 43.5638427734375 213.231092031621\\
% 43.5943603515625 213.210198591868\\
% 43.6248779296875 213.189326985282\\
% 43.6553955078125 213.168477190665\\
% 43.6859130859375 213.147649186861\\
% 43.7164306640625 213.126842952752\\
% 43.7469482421875 213.106058467261\\
% 43.7774658203125 213.085295709346\\
% 43.8079833984375 213.064554658009\\
% 43.8385009765625 213.043835292287\\
% 43.8690185546875 213.023137591258\\
% 43.8995361328125 213.00246153404\\
% 43.9300537109375 212.981807099786\\
% 43.9605712890625 212.961174267692\\
% 43.9910888671875 212.940563016991\\
% 44.0216064453125 212.919973326956\\
% 44.0521240234375 212.899405176896\\
% 44.0826416015625 212.878858546162\\
% 44.1131591796875 212.858333414143\\
% 44.1436767578125 212.837829760264\\
% 44.1741943359375 212.817347563992\\
% 44.2047119140625 212.796886804832\\
% 44.2352294921875 212.776447462325\\
% 44.2657470703125 212.756029516055\\
% 44.2962646484375 212.735632945641\\
% 44.3267822265625 212.715257730741\\
% 44.3572998046875 212.694903851052\\
% 44.3878173828125 212.674571286311\\
% 44.4183349609375 212.654260016291\\
% 44.4488525390625 212.633970020803\\
% 44.4793701171875 212.6137012797\\
% 44.5098876953125 212.593453772869\\
% 44.5404052734375 212.573227480238\\
% 44.5709228515625 212.553022381772\\
% 44.6014404296875 212.532838457477\\
% 44.6319580078125 212.51267568739\\
% 44.6624755859375 212.492534051596\\
% 44.6929931640625 212.472413530209\\
% 44.7235107421875 212.452314103388\\
% 44.7540283203125 212.432235751325\\
% 44.7845458984375 212.412178454255\\
% 44.8150634765625 212.392142192446\\
% 44.8455810546875 212.372126946207\\
% 44.8760986328125 212.352132695883\\
% 44.9066162109375 212.332159421859\\
% 44.9371337890625 212.312207104557\\
% 44.9676513671875 212.292275724437\\
% 44.9981689453125 212.272365261995\\
% 45.0286865234375 212.252475697766\\
% 45.0592041015625 212.232607012325\\
% 45.0897216796875 212.212759186282\\
% 45.1202392578125 212.192932200284\\
% 45.1507568359375 212.173126035018\\
% 45.1812744140625 212.153340671208\\
% 45.2117919921875 212.133576089614\\
% 45.2423095703125 212.113832271036\\
% 45.2728271484375 212.094109196309\\
% 45.3033447265625 212.074406846307\\
% 45.3338623046875 212.054725201941\\
% 45.3643798828125 212.035064244161\\
% 45.3948974609375 212.015423953951\\
% 45.4254150390625 211.995804312334\\
% 45.4559326171875 211.976205300374\\
% 45.4864501953125 211.956626899165\\
% 45.5169677734375 211.937069089844\\
% 45.5474853515625 211.917531853584\\
% 45.5780029296875 211.898015171593\\
% 45.6085205078125 211.878519025119\\
% 45.6390380859375 211.859043395446\\
% 45.6695556640625 211.839588263894\\
% 45.7000732421875 211.820153611821\\
% 45.7305908203125 211.800739420624\\
% 45.7611083984375 211.781345671734\\
% 45.7916259765625 211.761972346621\\
% 45.8526611328125 211.723286893784\\
% 45.9136962890625 211.684682914607\\
% 45.9747314453125 211.64616026217\\
% 46.0357666015625 211.607718790146\\
% 46.0968017578125 211.569358352793\\
% 46.1578369140625 211.531078804957\\
% 46.2188720703125 211.49288000207\\
% 46.2799072265625 211.45476180015\\
% 46.3409423828125 211.416724055797\\
% 46.4019775390625 211.378766626197\\
% 46.4630126953125 211.340889369116\\
% 46.5240478515625 211.303092142902\\
% 46.5850830078125 211.265374806485\\
% 46.6461181640625 211.227737219372\\
% 46.7071533203125 211.19017924165\\
% 46.7681884765625 211.152700733982\\
% 46.8292236328125 211.115301557611\\
% 46.8902587890625 211.077981574353\\
% 46.9512939453125 211.040740646599\\
% 47.0123291015625 211.003578637316\\
% 47.0733642578125 210.966495410042\\
% 47.1343994140625 210.929490828888\\
% 47.1954345703125 210.892564758536\\
% 47.2564697265625 210.855717064238\\
% 47.3175048828125 210.818947611818\\
% 47.3785400390625 210.782256267664\\
% 47.4395751953125 210.745642898736\\
% 47.5006103515625 210.709107372559\\
% 47.5616455078125 210.672649557224\\
% 47.6226806640625 210.636269321388\\
% 47.6837158203125 210.59996653427\\
% 47.7447509765625 210.563741065654\\
% 47.8057861328125 210.527592785888\\
% 47.8668212890625 210.491521565878\\
% 47.9278564453125 210.455527277093\\
% 47.9888916015625 210.419609791563\\
% 48.0499267578125 210.383768981875\\
% 48.1109619140625 210.348004721173\\
% 48.1719970703125 210.312316883164\\
% 48.2330322265625 210.276705342106\\
% 48.2940673828125 210.241169972815\\
% 48.3551025390625 210.205710650662\\
% 48.4161376953125 210.170327251571\\
% 48.4771728515625 210.13501965202\\
% 48.5382080078125 210.099787729041\\
% 48.5992431640625 210.064631360215\\
% 48.6602783203125 210.029550423675\\
% 48.7213134765625 209.994544798105\\
% 48.7823486328125 209.959614362737\\
% 48.8433837890625 209.924758997352\\
% 48.9044189453125 209.889978582278\\
% 48.9654541015625 209.855272998391\\
% 49.0264892578125 209.820642127112\\
% 49.0875244140625 209.78608585041\\
% 49.1485595703125 209.751604050793\\
% 49.2095947265625 209.717196611319\\
% 49.2706298828125 209.682863415586\\
% 49.3316650390625 209.648604347735\\
% 49.3927001953125 209.614419292447\\
% 49.4537353515625 209.580308134946\\
% 49.5147705078125 209.546270760995\\
% 49.5758056640625 209.512307056897\\
% 49.6368408203125 209.478416909491\\
% 49.6978759765625 209.444600206157\\
% 49.7589111328125 209.410856834812\\
% 49.8199462890625 209.377186683907\\
% 49.8809814453125 209.343589642429\\
% 49.9420166015625 209.310065599904\\
% 50.0030517578125 209.276614446387\\
% 50.0640869140625 209.243236072469\\
% 50.1251220703125 209.209930369274\\
% 50.1861572265625 209.176697228458\\
% 50.2471923828125 209.143536542208\\
% 50.3082275390625 209.110448203241\\
% 50.3692626953125 209.077432104807\\
% 50.4302978515625 209.044488140684\\
% 50.4913330078125 209.011616205176\\
% 50.5523681640625 208.978816193118\\
% 50.6134033203125 208.946087999873\\
% 50.6744384765625 208.91343152133\\
% 50.7354736328125 208.880846653902\\
% 50.7965087890625 208.84833329453\\
% 50.8575439453125 208.815891340678\\
% 50.9185791015625 208.783520690338\\
% 50.9796142578125 208.751221242021\\
% 51.0406494140625 208.718992894762\\
% 51.1016845703125 208.686835548121\\
% 51.1627197265625 208.654749102177\\
% 51.2237548828125 208.622733457529\\
% 51.2847900390625 208.590788515302\\
% 51.3458251953125 208.558914177134\\
% 51.4068603515625 208.527110345186\\
% 51.4678955078125 208.495376922139\\
% 51.5289306640625 208.463713811188\\
% 51.5899658203125 208.432120916048\\
% 51.6510009765625 208.400598140953\\
% 51.7120361328125 208.369145390648\\
% 51.7730712890625 208.337762570399\\
% 51.8341064453125 208.306449585984\\
% 51.8951416015625 208.275206343696\\
% 51.9561767578125 208.244032750344\\
% 52.0172119140625 208.212928713249\\
% 52.0782470703125 208.181894140246\\
% 52.1392822265625 208.150928939681\\
% 52.2003173828125 208.120033020413\\
% 52.2613525390625 208.089206291813\\
% 52.3223876953125 208.058448663762\\
% 52.3834228515625 208.027760046651\\
% 52.4444580078125 207.997140351381\\
% 52.5054931640625 207.966589489365\\
% 52.5665283203125 207.936107372521\\
% 52.6275634765625 207.905693913278\\
% 52.6885986328125 207.875349024572\\
% 52.7496337890625 207.845072619846\\
% 52.8106689453125 207.81486461305\\
% 52.8717041015625 207.784724918642\\
% 52.9327392578125 207.754653451584\\
% 52.9937744140625 207.724650127345\\
% 53.0548095703125 207.694714861899\\
% 53.1158447265625 207.664847571722\\
% 53.1768798828125 207.635048173799\\
% 53.2379150390625 207.605316585614\\
% 53.2989501953125 207.575652725157\\
% 53.3599853515625 207.546056510921\\
% 53.4210205078125 207.516527861899\\
% 53.4820556640625 207.48706669759\\
% 53.5430908203125 207.45767293799\\
% 53.6041259765625 207.428346503601\\
% 53.6651611328125 207.399087315421\\
% 53.7261962890625 207.369895294952\\
% 53.7872314453125 207.340770364196\\
% 53.8482666015625 207.311712445652\\
% 53.9093017578125 207.28272146232\\
% 53.9703369140625 207.253797337699\\
% 54.0313720703125 207.224939995786\\
% 54.0924072265625 207.196149361077\\
% 54.1534423828125 207.167425358565\\
% 54.2144775390625 207.13876791374\\
% 54.2755126953125 207.110176952592\\
% 54.3365478515625 207.081652401603\\
% 54.3975830078125 207.053194187756\\
% 54.4586181640625 207.024802238527\\
% 54.5196533203125 206.99647648189\\
% 54.5806884765625 206.968216846314\\
% 54.6417236328125 206.940023260762\\
% 54.7027587890625 206.911895654692\\
% 54.7637939453125 206.88383395806\\
% 54.8248291015625 206.855838101311\\
% 54.8858642578125 206.827908015388\\
% 54.9468994140625 206.800043631726\\
% 55.0079345703125 206.772244882253\\
% 55.0689697265625 206.744511699393\\
% 55.1300048828125 206.716844016061\\
% 55.1910400390625 206.689241765662\\
% 55.2520751953125 206.6617048821\\
% 55.3131103515625 206.634233299765\\
% 55.3741455078125 206.606826953543\\
% 55.4351806640625 206.579485778809\\
% 55.4962158203125 206.552209711431\\
% 55.5572509765625 206.524998687769\\
% 55.6182861328125 206.497852644673\\
% 55.6793212890625 206.470771519482\\
% 55.7403564453125 206.443755250031\\
% 55.8013916015625 206.416803774641\\
% 55.8624267578125 206.389917032124\\
% 55.9234619140625 206.363094961784\\
% 55.9844970703125 206.336337503414\\
% 56.0455322265625 206.309644597296\\
% 56.1065673828125 206.283016184201\\
% 56.1676025390625 206.256452205393\\
% 56.2286376953125 206.229952602623\\
% 56.2896728515625 206.203517318132\\
% 56.3507080078125 206.177146294647\\
% 56.4117431640625 206.15083947539\\
% 56.4727783203125 206.124596804065\\
% 56.5338134765625 206.098418224871\\
% 56.5948486328125 206.072303682492\\
% 56.6558837890625 206.0462531221\\
% 56.7169189453125 206.020266489358\\
% 56.7779541015625 205.994343730416\\
% 56.8389892578125 205.968484791913\\
% 56.9000244140625 205.942689620975\\
% 56.9610595703125 205.916958165216\\
% 57.0220947265625 205.89129037274\\
% 57.0831298828125 205.865686192139\\
% 57.1441650390625 205.840145572489\\
% 57.2052001953125 205.81466846336\\
% 57.2662353515625 205.789254814804\\
% 57.3272705078125 205.763904577366\\
% 57.3883056640625 205.738617702074\\
% 57.4493408203125 205.713394140449\\
% 57.5103759765625 205.688233844496\\
% 57.5714111328125 205.66313676671\\
% 57.6324462890625 205.638102860071\\
% 57.6934814453125 205.613132078051\\
% 57.7545166015625 205.588224374607\\
% 57.8155517578125 205.563379704185\\
% 57.8765869140625 205.538598021719\\
% 57.9376220703125 205.513879282629\\
% 57.9986572265625 205.489223442828\\
% 58.0596923828125 205.464630458711\\
% 58.1207275390625 205.440100287166\\
% 58.1817626953125 205.415632885567\\
% 58.2427978515625 205.391228211777\\
% 58.3038330078125 205.366886224148\\
% 58.3648681640625 205.342606881518\\
% 58.4259033203125 205.318390143218\\
% 58.4869384765625 205.294235969063\\
% 58.5479736328125 205.270144319361\\
% 58.6090087890625 205.246115154906\\
% 58.6700439453125 205.222148436982\\
% 58.7310791015625 205.198244127365\\
% 58.7921142578125 205.174402188316\\
% 58.8531494140625 205.150622582587\\
% 58.9141845703125 205.126905273423\\
% 58.9752197265625 205.103250224554\\
% 59.0362548828125 205.079657400204\\
% 59.0972900390625 205.056126765085\\
% 59.1583251953125 205.0326582844\\
% 59.2193603515625 205.009251923842\\
% 59.2803955078125 204.985907649598\\
% 59.3414306640625 204.962625428341\\
% 59.4024658203125 204.939405227241\\
% 59.4635009765625 204.916247013955\\
% 59.5245361328125 204.893150756634\\
% 59.5855712890625 204.870116423921\\
% 59.6466064453125 204.84714398495\\
% 59.7076416015625 204.824233409349\\
% 59.7686767578125 204.801384667238\\
% 59.8297119140625 204.77859772923\\
% 59.8907470703125 204.755872566432\\
% 59.9517822265625 204.733209150446\\
% 60.0128173828125 204.710607453364\\
% 60.0738525390625 204.688067447776\\
% 60.1348876953125 204.665589106766\\
% 60.1959228515625 204.643172403911\\
% 60.2569580078125 204.620817313286\\
% 60.3179931640625 204.59852380946\\
% 60.3790283203125 204.576291867498\\
% 60.4400634765625 204.554121462961\\
% 60.5010986328125 204.532012571908\\
% 60.5621337890625 204.509965170895\\
% 60.6231689453125 204.487979236974\\
% 60.6842041015625 204.466054747696\\
% 60.7452392578125 204.444191681111\\
% 60.8062744140625 204.422390015766\\
% 60.8673095703125 204.400649730707\\
% 60.9283447265625 204.378970805483\\
% 60.9893798828125 204.357353220138\\
% 61.0504150390625 204.33579695522\\
% 61.1114501953125 204.314301991776\\
% 61.1724853515625 204.292868311356\\
% 61.2335205078125 204.271495896009\\
% 61.2945556640625 204.250184728291\\
% 61.3555908203125 204.228934791255\\
% 61.4166259765625 204.207746068462\\
% 61.4776611328125 204.186618543973\\
% 61.5386962890625 204.165552202358\\
% 61.5997314453125 204.144547028687\\
% 61.6607666015625 204.123603008537\\
% 61.7218017578125 204.102720127993\\
% 61.7828369140625 204.081898373644\\
% 61.8438720703125 204.061137732585\\
% 61.9049072265625 204.040438192423\\
% 61.9659423828125 204.019799741269\\
% 62.0269775390625 203.999222367743\\
% 62.0880126953125 203.978706060978\\
% 62.1490478515625 203.958250810613\\
% 62.2100830078125 203.937856606798\\
% 62.2711181640625 203.917523440196\\
% 62.3321533203125 203.897251301982\\
% 62.3931884765625 203.877040183841\\
% 62.4542236328125 203.856890077973\\
% 62.5152587890625 203.836800977091\\
% 62.5762939453125 203.816772874423\\
% 62.6373291015625 203.79680576371\\
% 62.6983642578125 203.776899639213\\
% 62.7593994140625 203.757054495707\\
% 62.8204345703125 203.737270328482\\
% 62.8814697265625 203.71754713335\\
% 62.9425048828125 203.69788490664\\
% 63.0035400390625 203.6782836452\\
% 63.0645751953125 203.658743346398\\
% 63.1256103515625 203.639264008124\\
% 63.1866455078125 203.619845628788\\
% 63.2476806640625 203.600488207325\\
% 63.3087158203125 203.581191743191\\
% 63.3697509765625 203.561956236366\\
% 63.4307861328125 203.542781687357\\
% 63.4918212890625 203.523668097194\\
% 63.5528564453125 203.504615467437\\
% 63.6138916015625 203.485623800169\\
% 63.6749267578125 203.466693098005\\
% 63.7359619140625 203.447823364086\\
% 63.7969970703125 203.429014602085\\
% 63.8580322265625 203.410266816205\\
% 63.9190673828125 203.391580011182\\
% 63.9801025390625 203.372954192282\\
% 64.0411376953125 203.354389365306\\
% 64.1021728515625 203.335885536591\\
% 64.1632080078125 203.317442713006\\
% 64.2242431640625 203.29906090196\\
% 64.2852783203125 203.280740111395\\
% 64.3463134765625 203.262480349795\\
% 64.4073486328125 203.244281626182\\
% 64.4683837890625 203.226143950118\\
% 64.5294189453125 203.208067331706\\
% 64.5904541015625 203.190051781592\\
% 64.6514892578125 203.172097310964\\
% 64.7125244140625 203.154203931556\\
% 64.7735595703125 203.136371655644\\
% 64.8345947265625 203.118600496055\\
% 64.8956298828125 203.10089046616\\
% 64.9566650390625 203.083241579881\\
% 65.0177001953125 203.065653851686\\
% 65.0787353515625 203.048127296597\\
% 65.1397705078125 203.030661930187\\
% 65.2008056640625 203.013257768582\\
% 65.2618408203125 202.995914828461\\
% 65.3228759765625 202.978633127059\\
% 65.3839111328125 202.961412682167\\
% 65.4449462890625 202.944253512135\\
% 65.5059814453125 202.92715563587\\
% 65.5670166015625 202.910119072839\\
% 65.6280517578125 202.893143843071\\
% 65.6890869140625 202.876229967158\\
% 65.7501220703125 202.859377466254\\
% 65.8111572265625 202.842586362077\\
% 65.8721923828125 202.825856676915\\
% 65.9332275390625 202.809188433618\\
% 65.9942626953125 202.792581655611\\
% 66.0552978515625 202.776036366882\\
% 66.1163330078125 202.759552591996\\
% 66.1773681640625 202.743130356086\\
% 66.2384033203125 202.726769684863\\
% 66.2994384765625 202.710470604611\\
% 66.3604736328125 202.694233142189\\
% 66.4215087890625 202.678057325038\\
% 66.4825439453125 202.661943181175\\
% 66.5435791015625 202.645890739198\\
% 66.6046142578125 202.629900028289\\
% 66.6656494140625 202.613971078212\\
% 66.7266845703125 202.598103919316\\
% 66.7877197265625 202.582298582537\\
% 66.8487548828125 202.566555099399\\
% 66.9097900390625 202.550873502013\\
% 66.9708251953125 202.535253823086\\
% 67.0318603515625 202.519696095912\\
% 67.0928955078125 202.50420035438\\
% 67.1539306640625 202.488766632978\\
% 67.2149658203125 202.473394966785\\
% 67.2760009765625 202.458085391485\\
% 67.3370361328125 202.442837943356\\
% 67.3980712890625 202.427652659281\\
% 67.4591064453125 202.412529576746\\
% 67.5201416015625 202.397468733841\\
% 67.5811767578125 202.382470169262\\
% 67.6422119140625 202.367533922314\\
% 67.7032470703125 202.352660032911\\
% 67.7642822265625 202.337848541581\\
% 67.8253173828125 202.32309948946\\
% 67.8863525390625 202.308412918302\\
% 67.9473876953125 202.29378887048\\
% 68.0084228515625 202.279227388978\\
% 68.0694580078125 202.264728517407\\
% 68.1304931640625 202.250292299996\\
% 68.1915283203125 202.235918781599\\
% 68.2525634765625 202.221608007694\\
% 68.3135986328125 202.207360024386\\
% 68.3746337890625 202.19317487841\\
% 68.4356689453125 202.179052617131\\
% 68.4967041015625 202.164993288546\\
% 68.5577392578125 202.150996941287\\
% 68.6187744140625 202.137063624623\\
% 68.6798095703125 202.123193388459\\
% 68.7408447265625 202.109386283342\\
% 68.8018798828125 202.095642360461\\
% 68.8629150390625 202.081961671647\\
% 68.9239501953125 202.068344269378\\
% 68.9849853515625 202.05479020678\\
% 69.0460205078125 202.041299537629\\
% 69.1070556640625 202.027872316352\\
% 69.1680908203125 202.01450859803\\
% 69.2291259765625 202.0012084384\\
% 69.2901611328125 201.987971893858\\
% 69.3511962890625 201.974799021457\\
% 69.4122314453125 201.961689878915\\
% 69.4732666015625 201.948644524613\\
% 69.5343017578125 201.935663017597\\
% 69.5953369140625 201.922745417584\\
% 69.6563720703125 201.909891784961\\
% 69.7174072265625 201.897102180785\\
% 69.7784423828125 201.884376666791\\
% 69.8394775390625 201.871715305388\\
% 69.9005126953125 201.859118159669\\
% 69.9615478515625 201.846585293404\\
% 70.0225830078125 201.834116771048\\
% 70.0836181640625 201.821712657743\\
% 70.1446533203125 201.809373019318\\
% 70.2056884765625 201.797097922295\\
% 70.2667236328125 201.784887433887\\
% 70.3277587890625 201.772741622002\\
% 70.3887939453125 201.760660555247\\
% 70.4498291015625 201.748644302929\\
% 70.5108642578125 201.736692935055\\
% 70.5718994140625 201.724806522341\\
% 70.6329345703125 201.712985136207\\
% 70.6939697265625 201.701228848782\\
% 70.7550048828125 201.689537732912\\
% 70.8160400390625 201.677911862154\\
% 70.8770751953125 201.666351310781\\
% 70.9381103515625 201.654856153789\\
% 70.9991455078125 201.643426466896\\
% 71.0601806640625 201.632062326541\\
% 71.1212158203125 201.620763809895\\
% 71.1822509765625 201.609530994856\\
% 71.2432861328125 201.598363960057\\
% 71.3043212890625 201.587262784865\\
% 71.3653564453125 201.576227549384\\
% 71.4263916015625 201.565258334461\\
% 71.4874267578125 201.554355221684\\
% 71.5484619140625 201.543518293389\\
% 71.6094970703125 201.532747632658\\
% 71.6705322265625 201.522043323329\\
% 71.7315673828125 201.51140544999\\
% 71.7926025390625 201.500834097989\\
% 71.8536376953125 201.490329353432\\
% 71.9146728515625 201.479891303189\\
% 71.9757080078125 201.469520034897\\
% 72.0367431640625 201.459215636957\\
% 72.0977783203125 201.448978198547\\
% 72.1588134765625 201.438807809616\\
% 72.2198486328125 201.428704560892\\
% 72.2808837890625 201.418668543881\\
% 72.3419189453125 201.408699850876\\
% 72.4029541015625 201.398798574954\\
% 72.4639892578125 201.38896480998\\
% 72.5250244140625 201.379198650615\\
% 72.5860595703125 201.369500192313\\
% 72.6470947265625 201.359869531328\\
% 72.7081298828125 201.350306764714\\
% 72.7691650390625 201.340811990333\\
% 72.8302001953125 201.33138530685\\
% 72.8912353515625 201.322026813748\\
% 72.9522705078125 201.312736611318\\
% 73.0133056640625 201.303514800673\\
% 73.0743408203125 201.294361483747\\
% 73.1353759765625 201.285276763294\\
% 73.1964111328125 201.276260742901\\
% 73.2574462890625 201.267313526983\\
% 73.3184814453125 201.25843522079\\
% 73.3795166015625 201.249625930409\\
% 73.4405517578125 201.24088576277\\
% 73.5015869140625 201.232214825645\\
% 73.5626220703125 201.223613227655\\
% 73.6236572265625 201.215081078274\\
% 73.6846923828125 201.206618487829\\
% 73.7457275390625 201.198225567506\\
% 73.8067626953125 201.189902429352\\
% 73.8677978515625 201.181649186284\\
% 73.9288330078125 201.173465952083\\
% 73.9898681640625 201.165352841406\\
% 74.0509033203125 201.157309969783\\
% 74.1119384765625 201.14933745363\\
% 74.1729736328125 201.141435410243\\
% 74.2340087890625 201.133603957806\\
% 74.2950439453125 201.125843215394\\
% 74.3560791015625 201.118153302978\\
% 74.4171142578125 201.11053434143\\
% 74.4781494140625 201.10298645252\\
% 74.5391845703125 201.095509758929\\
% 74.6002197265625 201.088104384248\\
% 74.6612548828125 201.080770452981\\
% 74.7222900390625 201.073508090551\\
% 74.7833251953125 201.066317423304\\
% 74.8443603515625 201.059198578512\\
% 74.9053955078125 201.052151684378\\
% 74.9664306640625 201.045176870039\\
% 75.0274658203125 201.03827426557\\
% 75.0885009765625 201.031444001991\\
% 75.1495361328125 201.024686211267\\
% 75.2105712890625 201.018001026315\\
% 75.2716064453125 201.011388581005\\
% 75.3326416015625 201.00484901017\\
% 75.3936767578125 200.998382449605\\
% 75.4547119140625 200.991989036073\\
% 75.5157470703125 200.985668907309\\
% 75.5767822265625 200.979422202025\\
% 75.6378173828125 200.973249059915\\
% 75.6988525390625 200.967149621658\\
% 75.7598876953125 200.961124028922\\
% 75.8209228515625 200.95517242437\\
% 75.8819580078125 200.949294951664\\
% 75.9429931640625 200.94349175547\\
% 76.0040283203125 200.937762981461\\
% 76.0650634765625 200.932108776322\\
% 76.1260986328125 200.926529287757\\
% 76.1871337890625 200.921024664491\\
% 76.2481689453125 200.915595056275\\
% 76.3092041015625 200.910240613891\\
% 76.4007568359375 200.902350218641\\
% 76.4923095703125 200.894629805134\\
% 76.5838623046875 200.887079895458\\
% 76.6754150390625 200.879701017047\\
% 76.7669677734375 200.87249370271\\
% 76.8585205078125 200.865458490679\\
% 76.9500732421875 200.858595924634\\
% 77.0416259765625 200.851906553756\\
% 77.1331787109375 200.845390932757\\
% 77.2247314453125 200.839049621918\\
% 77.3162841796875 200.832883187138\\
% 77.4078369140625 200.826892199961\\
% 77.4993896484375 200.821077237629\\
% 77.5909423828125 200.815438883113\\
% 77.6824951171875 200.809977725162\\
% 77.7740478515625 200.804694358339\\
% 77.8656005859375 200.799589383066\\
% 77.9571533203125 200.794663405667\\
% 78.0487060546875 200.789917038405\\
% 78.1402587890625 200.785350899537\\
% 78.2318115234375 200.780965613346\\
% 78.3233642578125 200.776761810191\\
% 78.4149169921875 200.772740126551\\
% 78.5064697265625 200.768901205071\\
% 78.5980224609375 200.765245694605\\
% 78.6895751953125 200.761774250263\\
% 78.7811279296875 200.75848753346\\
% 78.8726806640625 200.755386211958\\
% 78.9642333984375 200.752470959918\\
% 79.0557861328125 200.749742457944\\
% 79.1473388671875 200.747201393136\\
% 79.2388916015625 200.744848459133\\
% 79.3304443359375 200.742684356166\\
% 79.4219970703125 200.740709791108\\
% 79.5135498046875 200.738925477522\\
% 79.6051025390625 200.737332135712\\
% 79.6966552734375 200.735930492778\\
% 79.7882080078125 200.734721282659\\
% 79.8797607421875 200.733705246196\\
% 79.9713134765625 200.732883131177\\
% 80.0628662109375 200.732255692391\\
% 80.1544189453125 200.731823691686\\
% 80.2459716796875 200.731587898018\\
% 80.3375244140625 200.731549087511\\
% 80.4290771484375 200.731708043505\\
% 80.5206298828125 200.732065556618\\
% 80.6121826171875 200.732622424802\\
% 80.7037353515625 200.733379453394\\
% 80.7952880859375 200.734337455181\\
% 80.8868408203125 200.735497250453\\
% 80.9783935546875 200.736859667061\\
% 81.0699462890625 200.73842554048\\
% 81.1614990234375 200.740195713865\\
% 81.2530517578125 200.742171038112\\
% 81.3446044921875 200.744352371919\\
% 81.4361572265625 200.746740581847\\
% 81.5277099609375 200.749336542382\\
% 81.6192626953125 200.752141135998\\
% 81.7108154296875 200.755155253216\\
% 81.8023681640625 200.758379792674\\
% 81.8939208984375 200.761815661188\\
% 81.9854736328125 200.76546377381\\
% 82.0770263671875 200.769325053907\\
% 82.1685791015625 200.773400433217\\
% 82.2601318359375 200.777690851915\\
% 82.3516845703125 200.782197258688\\
% 82.4432373046875 200.786920610791\\
% 82.5347900390625 200.791861874129\\
% 82.6263427734375 200.797022023313\\
% 82.7178955078125 200.802402041737\\
% 82.8094482421875 200.808002921648\\
% 82.9010009765625 200.813825664214\\
% 82.9925537109375 200.819871279594\\
% 83.0841064453125 200.826140787017\\
% 83.1756591796875 200.832635214845\\
% 83.2672119140625 200.839355600656\\
% 83.3587646484375 200.846302991309\\
% 83.4503173828125 200.853478443026\\
% 83.5418701171875 200.860883021462\\
% 83.6334228515625 200.868517801783\\
% 83.7249755859375 200.876383868744\\
% 83.8165283203125 200.884482316762\\
% 83.9080810546875 200.892814249998\\
% 83.9996337890625 200.901380782432\\
% 84.0911865234375 200.910183037946\\
% 84.1827392578125 200.919222150399\\
% 84.2742919921875 200.928499263714\\
% 84.3658447265625 200.938015531952\\
% 84.4573974609375 200.947772119396\\
% 84.5489501953125 200.957770200639\\
% 84.6405029296875 200.968010960659\\
% 84.7320556640625 200.978495594907\\
% 84.8236083984375 200.989225309392\\
% 84.9151611328125 201.000201320764\\
% 85.0067138671875 201.011424856403\\
% 85.0982666015625 201.022897154503\\
% 85.1898193359375 201.034619464157\\
% 85.2813720703125 201.046593045452\\
% 85.3729248046875 201.05881916955\\
% 85.4644775390625 201.071299118782\\
% 85.5560302734375 201.084034186737\\
% 85.6475830078125 201.097025678353\\
% 85.7391357421875 201.110274910005\\
% 85.8306884765625 201.123783209602\\
% 85.9222412109375 201.137551916676\\
% 86.0137939453125 201.151582382478\\
% 86.1053466796875 201.165875970069\\
% 86.1968994140625 201.180434054418\\
% 86.2884521484375 201.195258022496\\
% 86.3800048828125 201.210349273371\\
% 86.4715576171875 201.225709218307\\
% 86.5631103515625 201.241339280859\\
% 86.6546630859375 201.257240896974\\
% 86.7462158203125 201.273415515087\\
% 86.8377685546875 201.289864596222\\
% 86.9293212890625 201.306589614091\\
% 87.0208740234375 201.323592055196\\
% 87.1124267578125 201.340873418931\\
% 87.2039794921875 201.35843521768\\
% 87.2955322265625 201.376278976925\\
% 87.3870849609375 201.394406235346\\
% 87.4786376953125 201.412818544925\\
% 87.5701904296875 201.431517471051\\
% 87.6617431640625 201.450504592626\\
% 87.7532958984375 201.469781502172\\
% 87.8448486328125 201.489349805932\\
% 87.9364013671875 201.509211123981\\
% 88.0279541015625 201.529367090334\\
% 88.1195068359375 201.549819353053\\
% 88.2110595703125 201.570569574352\\
% 88.3026123046875 201.591619430714\\
% 88.3941650390625 201.612970612992\\
% 88.4857177734375 201.634624826527\\
% 88.5772705078125 201.656583791251\\
% 88.6688232421875 201.678849241806\\
% 88.7603759765625 201.701422927648\\
% 88.8519287109375 201.724306613165\\
% 88.9434814453125 201.747502077787\\
% 89.0350341796875 201.771011116097\\
% 89.1265869140625 201.794835537948\\
% 89.2181396484375 201.818977168576\\
% 89.3096923828125 201.84343784871\\
% 89.4012451171875 201.868219434694\\
% 89.4927978515625 201.893323798593\\
% 89.5843505859375 201.918752828314\\
% 89.6759033203125 201.94450842772\\
% 89.7674560546875 201.970592516743\\
% 89.8590087890625 201.997007031504\\
% 89.9505615234375 202.023753924425\\
% 90.0421142578125 202.050835164347\\
% 90.1336669921875 202.078252736647\\
% 90.2252197265625 202.10600864335\\
% 90.3167724609375 202.134104903252\\
% 90.4083251953125 202.162543552031\\
% 90.4998779296875 202.191326642366\\
% 90.5914306640625 202.220456244051\\
% 90.6829833984375 202.249934444115\\
% 90.7745361328125 202.279763346935\\
% 90.8660888671875 202.309945074352\\
% 90.9576416015625 202.340481765788\\
% 91.0491943359375 202.371375578361\\
% 91.1407470703125 202.402628687002\\
% 91.2322998046875 202.434243284565\\
% 91.3238525390625 202.466221581946\\
% 91.4154052734375 202.498565808196\\
% 91.5069580078125 202.531278210633\\
% 91.5985107421875 202.564361054956\\
% 91.6900634765625 202.597816625356\\
% 91.7816162109375 202.631647224629\\
% 91.8731689453125 202.665855174287\\
% 91.9647216796875 202.700442814665\\
% 92.0562744140625 202.735412505034\\
% 92.1478271484375 202.770766623707\\
% 92.2393798828125 202.806507568146\\
% 92.3309326171875 202.842637755066\\
% 92.4224853515625 202.879159620544\\
% 92.5140380859375 202.91607562012\\
% 92.6055908203125 202.953388228897\\
% 92.6971435546875 202.991099941647\\
% 92.7886962890625 203.029213272902\\
% 92.8802490234375 203.06773075706\\
% 92.9718017578125 203.106654948474\\
% 93.0633544921875 203.145988421549\\
% 93.1549072265625 203.18573377083\\
% 93.2464599609375 203.225893611098\\
% 93.3380126953125 203.266470577451\\
% 93.4295654296875 203.307467325393\\
% 93.5211181640625 203.348886530914\\
% 93.6126708984375 203.390730890572\\
% 93.7042236328125 203.433003121569\\
% 93.7957763671875 203.475705961825\\
% 93.8873291015625 203.518842170053\\
% 93.9788818359375 203.562414525822\\
% 94.0704345703125 203.606425829625\\
% 94.1619873046875 203.650878902941\\
% 94.2535400390625 203.69577658829\\
% 94.3450927734375 203.741121749292\\
% 94.4366455078125 203.78691727071\\
% 94.5281982421875 203.833166058501\\
% 94.6197509765625 203.879871039859\\
% 94.7113037109375 203.927035163246\\
% 94.8028564453125 203.97466139843\\
% 94.8944091796875 204.022752736507\\
% 94.9859619140625 204.071312189929\\
% 95.0775146484375 204.120342792513\\
% 95.1690673828125 204.169847599461\\
% 95.2606201171875 204.219829687358\\
% 95.3521728515625 204.270292154174\\
% 95.4437255859375 204.321238119254\\
% 95.5352783203125 204.372670723309\\
% 95.6268310546875 204.424593128389\\
% 95.7183837890625 204.477008517858\\
% 95.8099365234375 204.529920096357\\
% 95.9014892578125 204.58333108976\\
% 95.9930419921875 204.637244745125\\
% 96.0845947265625 204.691664330629\\
% 96.1761474609375 204.7465931355\\
% 96.2677001953125 204.802034469944\\
% 96.3592529296875 204.857991665048\\
% 96.4508056640625 204.914468072686\\
% 96.5423583984375 204.971467065417\\
% 96.6339111328125 205.028992036355\\
% 96.7254638671875 205.087046399047\\
% 96.8170166015625 205.145633587327\\
% 96.9085693359375 205.204757055165\\
% 97.0001220703125 205.264420276499\\
% 97.0916748046875 205.324626745058\\
% 97.1832275390625 205.385379974165\\
% 97.2747802734375 205.446683496538\\
% 97.3663330078125 205.508540864059\\
% 97.4578857421875 205.570955647548\\
% 97.5494384765625 205.633931436504\\
% 97.6409912109375 205.69747183884\\
% 97.7325439453125 205.761580480599\\
% 97.8240966796875 205.826261005651\\
% 97.9156494140625 205.891517075371\\
% 98.0072021484375 205.957352368305\\
% 98.0987548828125 206.023770579807\\
% 98.1903076171875 206.090775421664\\
% 98.2818603515625 206.158370621693\\
% 98.3734130859375 206.226559923321\\
% 98.4649658203125 206.29534708515\\
% 98.5565185546875 206.364735880474\\
% 98.6480712890625 206.434730096806\\
% 98.7396240234375 206.505333535349\\
% 98.8311767578125 206.576550010462\\
% 98.9227294921875 206.648383349088\\
% 99.0142822265625 206.72083739016\\
% 99.1058349609375 206.793915983978\\
% 99.1973876953125 206.867622991548\\
% 99.2889404296875 206.941962283904\\
% 99.3804931640625 207.016937741385\\
% 99.4720458984375 207.092553252889\\
% 99.5635986328125 207.168812715084\\
% 99.6551513671875 207.24572003159\\
% 99.7467041015625 207.323279112123\\
% 99.8382568359375 207.401493871596\\
% 99.9298095703125 207.480368229188\\
% 100.021362304688 207.559906107366\\
% 100.112915039062 207.640111430869\\
% 100.204467773438 207.720988125644\\
% 100.296020507813 207.802540117741\\
% 100.387573242188 207.884771332158\\
% 100.479125976562 207.967685691639\\
% 100.570678710938 208.051287115419\\
% 100.662231445313 208.135579517923\\
% 100.753784179688 208.2205668074\\
% 100.845336914062 208.306252884514\\
% 100.936889648438 208.392641640868\\
% 101.028442382812 208.479736957471\\
% 101.119995117188 208.567542703145\\
% 101.211547851562 208.656062732862\\
% 101.303100585938 208.745300886022\\
% 101.394653320312 208.835260984655\\
% 101.486206054688 208.925946831556\\
% 101.577758789062 209.017362208346\\
% 101.669311523438 209.109510873453\\
% 101.760864257812 209.202396560019\\
% 101.852416992188 209.296022973719\\
% 101.943969726562 209.390393790501\\
% 102.035522460938 209.485512654235\\
% 102.127075195312 209.581383174268\\
% 102.218627929688 209.67800892289\\
% 102.310180664062 209.775393432702\\
% 102.401733398438 209.873540193876\\
% 102.493286132812 209.972452651319\\
% 102.584838867188 210.072134201721\\
% 102.676391601562 210.172588190496\\
% 102.767944335938 210.273817908604\\
% 102.859497070312 210.375826589249\\
% 102.951049804688 210.478617404459\\
% 103.042602539062 210.582193461532\\
% 103.134155273438 210.686557799345\\
% 103.225708007812 210.791713384531\\
% 103.317260742188 210.897663107501\\
% 103.408813476562 211.004409778331\\
% 103.500366210938 211.111956122478\\
% 103.591918945312 211.220304776347\\
% 103.683471679688 211.329458282687\\
% 103.775024414062 211.439419085811\\
% 103.866577148438 211.550189526643\\
% 103.958129882812 211.661771837573\\
% 104.049682617188 211.774168137124\\
% 104.141235351562 211.887380424413\\
% 104.232788085938 212.001410573414\\
% 104.324340820312 212.116260326993\\
% 104.415893554688 212.231931290728\\
% 104.507446289063 212.348424926495\\
% 104.598999023438 212.465742545812\\
% 104.690551757812 212.583885302932\\
% 104.782104492188 212.702854187677\\
% 104.873657226563 212.822650018001\\
% 104.965209960938 212.943273432276\\
% 105.056762695312 213.064724881282\\
% 105.148315429688 213.187004619893\\
% 105.239868164063 213.310112698454\\
% 105.331420898438 213.434048953827\\
% 105.422973632812 213.558813000095\\
% 105.514526367188 213.684404218912\\
% 105.606079101563 213.810821749487\\
% 105.697631835938 213.938064478185\\
% 105.789184570312 214.066131027723\\
% 105.880737304688 214.195019745953\\
% 105.972290039063 214.324728694217\\
% 106.063842773438 214.455255635242\\
% 106.155395507812 214.586598020572\\
% 106.246948242188 214.718752977506\\
% 106.338500976563 214.851717295529\\
% 106.430053710938 214.985487412199\\
% 106.521606445312 215.120059398498\\
% 106.613159179688 215.255428943584\\
% 106.704711914063 215.391591338944\\
% 106.796264648438 215.528541461916\\
% 106.887817382812 215.66627375855\\
% 107.009887695312 215.851123092639\\
% 107.131958007812 216.037336699962\\
% 107.254028320312 216.224897488483\\
% 107.376098632812 216.41378671938\\
% 107.498168945312 216.603983915808\\
% 107.620239257812 216.795466766547\\
% 107.742309570312 216.988211024221\\
% 107.864379882812 217.182190397727\\
% 107.986450195312 217.377376438497\\
% 108.108520507812 217.573738420155\\
% 108.230590820312 217.771243211154\\
% 108.352661132812 217.969855139876\\
% 108.474731445312 218.169535851673\\
% 108.596801757812 218.370244157299\\
% 108.718872070312 218.571935872075\\
% 108.840942382812 218.77456364513\\
% 108.963012695312 218.978076777986\\
% 109.085083007812 219.182421031658\\
% 109.207153320312 219.387538421426\\
% 109.329223632812 219.593366998283\\
% 109.451293945312 219.799840616033\\
% 109.573364257812 220.00688868289\\
% 109.695434570312 220.214435896303\\
% 109.817504882812 220.422401959638\\
% 109.939575195312 220.630701279185\\
% 110.061645507812 220.839242639826\\
% 110.183715820312 221.047928857518\\
% 110.305786132812 221.256656406549\\
% 110.427856445312 221.46531501933\\
% 110.549926757812 221.673787256237\\
% 110.671997070312 221.881948042746\\
% 110.794067382812 222.089664170812\\
% 110.916137695312 222.29679376111\\
% 111.038208007813 222.503185682365\\
% 111.160278320313 222.708678923587\\
% 111.282348632813 222.913101914524\\
% 111.404418945313 223.116271789138\\
% 111.526489257813 223.317993586247\\
% 111.648559570313 223.518059380823\\
% 111.770629882813 223.716247338621\\
% 111.892700195313 223.912320685922\\
% 112.014770507813 224.106026585137\\
% 112.136840820313 224.297094905887\\
% 112.258911132813 224.485236879794\\
% 112.380981445313 224.670143625732\\
% 112.503051757813 224.851484530534\\
% 112.625122070313 225.028905468112\\
% 112.747192382813 225.202026837699\\
% 112.869262695313 225.370441399227\\
% 112.991333007813 225.533711880789\\
% 113.113403320313 225.691368329623\\
% 113.235473632813 225.842905173923\\
% 113.357543945313 225.987777958037\\
% 113.479614257813 226.12539970807\\
% 113.601684570313 226.255136878483\\
% 113.723754882813 226.376304822735\\
% 113.845825195313 226.488162722248\\
% 113.967895507813 226.589907897661\\
% 114.089965820313 226.680669414291\\
% 114.212036132813 226.759500879495\\
% 114.334106445312 226.825372312948\\
% 114.456176757812 226.877160951156\\
% 114.578247070312 226.913640824293\\
% 114.700317382812 226.933470915941\\
% 114.822387695312 226.935181683914\\
% 114.944458007812 226.917159681802\\
% 115.066528320312 226.877629975478\\
% 115.188598632812 226.814635994894\\
% 115.310668945312 226.726016398001\\
% 115.432739257812 226.609378448596\\
% 115.554809570312 226.462067321895\\
% 115.676879882812 226.281130648759\\
% 115.798950195312 226.063277490303\\
% 115.921020507812 225.804830798059\\
% 116.043090820312 225.501672261246\\
% 116.165161132812 225.14917827426\\
% 116.287231445312 224.742145580725\\
% 116.409301757812 224.274704978184\\
% 116.531372070312 223.740221323887\\
% 116.653442382812 223.131178009254\\
% 116.775512695312 222.439044139514\\
% 116.897583007812 221.654122982881\\
% 117.019653320312 220.765381028465\\
% 117.141723632812 219.76025851208\\
% 117.263793945312 218.624464994429\\
% 117.385864257812 217.341768203107\\
% 117.507934570312 215.893791895258\\
% 117.630004882812 214.259850377968\\
% 117.752075195312 212.416865361414\\
% 117.874145507812 210.339437034397\\
% 117.996215820312 208.000177130109\\
% 118.118286132812 205.370456470575\\
% 118.240356445312 202.421766333513\\
% 118.362426757812 199.127923395856\\
% 118.484497070312 195.468323988982\\
% 118.606567382812 191.43231376534\\
% 118.728637695312 187.024412797001\\
% 118.850708007812 182.269594892869\\
% 118.972778320312 177.217174415036\\
% 119.094848632812 171.941442552147\\
% 119.216918945312 166.537511688676\\
% 119.338989257812 161.112155992231\\
% 119.461059570312 155.771380266692\\
% 119.583129882812 150.607916586005\\
% 119.705200195312 145.691836062922\\
% 119.827270507812 141.06596743206\\
% 119.949340820312 136.745829016801\\
% 120.071411132812 132.722382670449\\
% 120.193481445312 128.965478005799\\
% 120.315551757812 125.426028262815\\
% 120.437622070312 122.035115262333\\
% 120.559692382812 118.697686921837\\
% 120.681762695312 115.27628485437\\
% 120.803833007812 111.553735731128\\
% 120.925903320312 107.144412280636\\
% 121.047973632812 101.259042718437\\
% 121.170043945312 91.9774601369992\\
% 121.292114257812 73.6843031061573\\
% 121.414184570312 33.1059000410034\\
% 121.536254882812 -12.507228953271\\
% 121.658325195312 -34.0708442442616\\
% 121.780395507812 -44.0798421951575\\
% 121.902465820312 -49.7005220272584\\
% 122.024536132812 -53.3165445388874\\
% 122.146606445312 -55.8571031440028\\
% 122.268676757812 -57.7495626420123\\
% 122.390747070312 -59.2168180305067\\
% 122.512817382812 -60.3867154395374\\
% 122.634887695312 -61.3381955950945\\
% 122.756958007812 -62.1229493927408\\
% 122.879028320312 -62.7764804617255\\
% 123.001098632812 -63.3241575587492\\
% 123.123168945312 -63.7847186316902\\
% 123.245239257812 -64.1723982369118\\
% 123.367309570312 -64.4982725613457\\
% 123.489379882812 -64.7711400789828\\
% 123.611450195312 -64.9981160615358\\
% 123.733520507812 -65.185044880808\\
% 123.855590820312 -65.3367928746147\\
% 123.977661132812 -65.4574608761063\\
% 124.099731445312 -65.550541441143\\
% 124.221801757813 -65.6190372039181\\
% 124.343872070313 -65.6655513874734\\
% 124.465942382813 -65.6923580209131\\
% 124.588012695313 -65.7014571321702\\
% 124.710083007813 -65.6946186552034\\
% 124.832153320313 -65.6734177467974\\
% 124.954223632813 -65.639263483987\\
% 125.076293945313 -65.5934224030141\\
% 125.198364257813 -65.5370379762231\\
% 125.320434570313 -65.4711468593316\\
% 125.442504882813 -65.3966925480354\\
% 125.564575195313 -65.3145369393414\\
% 125.686645507813 -65.2254701855163\\
% 125.808715820313 -65.1302191470001\\
% 125.930786132813 -65.0294546884123\\
% 126.052856445313 -64.9237980137198\\
% 126.174926757813 -64.813826199307\\
% 126.296997070313 -64.7000770544444\\
% 126.419067382813 -64.5830534155281\\
% 126.541137695313 -64.4632269621959\\
% 126.663208007813 -64.3410416287265\\
% 126.785278320313 -64.2169166724225\\
% 126.907348632813 -64.0912494512075\\
% 127.029418945313 -63.9644179549969\\
% 127.151489257813 -63.8367831292595\\
% 127.273559570313 -63.708691024131\\
% 127.395629882813 -63.5804747984588\\
% 127.517700195313 -63.4524566048787\\
% 127.639770507813 -63.3249493795215\\
% 127.761840820312 -63.1982585579326\\
% 127.883911132812 -63.0726837373482\\
% 128.005981445312 -62.948520304462\\
% 128.128051757812 -62.8260610471902\\
% 128.250122070312 -62.705597768783\\
% 128.372192382812 -62.5874229227766\\
% 128.494262695312 -62.4718312878948\\
% 128.616333007812 -62.3591217029529\\
% 128.738403320312 -62.249598883296\\
% 128.860473632812 -62.1435753421679\\
% 128.982543945312 -62.0413734429261\\
% 129.104614257812 -61.9433276111111\\
% 129.226684570312 -61.8497867392085\\
% 129.348754882812 -61.7611168217658\\
% 129.470825195312 -61.6777038642526\\
% 129.592895507812 -61.5999571162465\\
% 129.714965820312 -61.5283126881395\\
% 129.837036132812 -61.4632376212378\\
% 129.959106445312 -61.4052344940962\\
% 130.081176757812 -61.3548466639461\\
% 130.203247070312 -61.3126642617786\\
% 130.325317382812 -61.2793310840373\\
% 130.447387695312 -61.2555525541774\\
% 130.569458007812 -61.2421049652661\\
% 130.691528320312 -61.2398462622358\\
% 130.813598632812 -61.2497286824283\\
% 130.935668945312 -61.2728136491054\\
% 131.057739257812 -61.3102894097676\\
% 131.179809570312 -61.3634920359139\\
% 131.301879882812 -61.4339305623099\\
% 131.423950195312 -61.5233172541416\\
% 131.546020507812 -61.6336042664695\\
% 131.668090820312 -61.7670283254809\\
% 131.790161132812 -61.9261655481073\\
% 131.912231445312 -62.1139991718581\\
% 132.034301757812 -62.3340038565679\\
% 132.156372070312 -62.5902514399006\\
% 132.278442382812 -62.8875447184786\\
% 132.400512695312 -63.2315881924665\\
% 132.522583007812 -63.6292080608032\\
% 132.644653320312 -64.088638550897\\
% 132.766723632812 -64.6198986183272\\
% 132.888793945312 -65.2352932503437\\
% 133.010864257812 -65.950088749842\\
% 133.132934570312 -66.7834341235809\\
% 133.255004882812 -67.7596352220241\\
% 133.377075195312 -68.909941075318\\
% 133.499145507812 -70.2750828008012\\
% 133.621215820312 -71.9089285087858\\
% 133.743286132812 -73.8837991819085\\
% 133.865356445312 -76.2982370047221\\
% 133.987426757812 -79.2882757620265\\
% 134.109497070312 -83.0432472745094\\
% 134.231567382812 -87.8258102735573\\
% 134.353637695312 -93.9898977575297\\
% 134.475708007812 -101.97126093922\\
% 134.597778320312 -112.181604922522\\
% 134.719848632812 -124.69764404737\\
% 134.841918945312 -138.797038959746\\
% 134.963989257812 -152.910243798501\\
% 135.086059570312 -165.460002347641\\
% 135.208129882812 -175.71028635174\\
% 135.330200195312 -183.729681257897\\
% 135.452270507812 -189.9279041498\\
% 135.574340820312 -194.741455985134\\
% 135.696411132812 -198.525613039066\\
% 135.818481445312 -201.544220698877\\
% 135.940551757812 -203.987498827813\\
% 136.062622070312 -205.992109263649\\
% 136.184692382812 -207.657030662771\\
% 136.306762695312 -209.054886847892\\
% 136.428833007812 -210.239756968381\\
% 136.550903320312 -211.252519980787\\
% 136.672973632812 -212.124528780822\\
% 136.795043945312 -212.880162023822\\
% 136.917114257812 -213.538619191122\\
% 137.039184570312 -214.115200691905\\
% 137.161254882812 -214.622233368961\\
% 137.283325195312 -215.069748658396\\
% 137.405395507812 -215.46598592245\\
% 137.557983398438 -215.899460251936\\
% 137.710571289062 -216.274312615782\\
% 137.863159179688 -216.599352905725\\
% 138.015747070312 -216.881763910822\\
% 138.168334960938 -217.12745701655\\
% 138.320922851562 -217.341339150825\\
% 138.473510742188 -217.527515608106\\
% 138.626098632812 -217.689445920227\\
% 138.778686523438 -217.830064914004\\
% 138.931274414062 -217.951877655625\\
% 139.083862304688 -218.057034595022\\
% 139.236450195312 -218.147391545263\\
% 139.389038085938 -218.224557937256\\
% 139.541625976562 -218.289935929585\\
% 139.694213867188 -218.344752326626\\
% 139.846801757812 -218.390084797054\\
% 139.999389648438 -218.426883542359\\
% 140.151977539062 -218.455989308116\\
% 140.304565429688 -218.478148436632\\
% 140.457153320312 -218.494025511446\\
% 140.609741210938 -218.504214030482\\
% 140.762329101562 -218.509245456517\\
% 140.914916992188 -218.509596925126\\
% 141.067504882812 -218.505697836331\\
% 141.220092773438 -218.497935513803\\
% 141.372680664062 -218.486660081643\\
% 141.525268554688 -218.472188681902\\
% 141.677856445312 -218.454809134341\\
% 141.830444335938 -218.434783122471\\
% 141.983032226562 -218.412348975791\\
% 142.135620117188 -218.387724106599\\
% 142.288208007812 -218.361107150302\\
% 142.440795898438 -218.332679850397\\
% 142.593383789062 -218.302608722931\\
% 142.745971679688 -218.271046529868\\
% 142.898559570312 -218.238133586468\\
% 143.051147460938 -218.203998924016\\
% 143.203735351562 -218.168761326214\\
% 143.356323242188 -218.132530254922\\
% 143.508911132812 -218.095406678759\\
% 143.661499023438 -218.057483816219\\
% 143.814086914062 -218.018847803387\\
% 143.966674804688 -217.979578295009\\
% 144.119262695312 -217.939749006524\\
% 144.271850585938 -217.89942820369\\
% 144.424438476562 -217.858679145594\\
% 144.577026367188 -217.81756048613\\
% 144.729614257812 -217.776126638395\\
% 144.882202148438 -217.734428105914\\
% 145.034790039062 -217.692511784157\\
% 145.187377929688 -217.650421235377\\
% 145.339965820312 -217.608196939486\\
% 145.492553710938 -217.565876523343\\
% 145.645141601562 -217.523494970575\\
% 145.797729492188 -217.481084813822\\
% 145.950317382812 -217.438676311097\\
% 146.102905273438 -217.396297607726\\
% 146.255493164062 -217.353974885256\\
% 146.408081054688 -217.311732498488\\
% 146.560668945312 -217.269593101737\\
% 146.713256835938 -217.227577765278\\
% 146.865844726562 -217.185706082831\\
% 147.018432617188 -217.143996270902\\
% 147.171020507812 -217.102465260645\\
% 147.323608398438 -217.061128782909\\
% 147.476196289062 -217.020001447028\\
% 147.628784179688 -216.979096813897\\
% 147.781372070312 -216.938427463775\\
% 147.933959960938 -216.898005059268\\
% 148.086547851562 -216.857840403874\\
% 148.239135742188 -216.817943496435\\
% 148.391723632812 -216.778323581836\\
% 148.544311523438 -216.738989198213\\
% 148.696899414062 -216.69994822099\\
% 148.849487304688 -216.661207903921\\
% 149.002075195312 -216.622774917428\\
% 149.154663085938 -216.584655384387\\
% 149.307250976562 -216.546854913586\\
% 149.459838867188 -216.509378631\\
% 149.612426757812 -216.472231209061\\
% 149.765014648438 -216.435416894053\\
% 149.917602539062 -216.398939531778\\
% 150.070190429688 -216.362802591598\\
% 150.222778320312 -216.327009188993\\
% 150.375366210938 -216.291562106704\\
% 150.527954101562 -216.256463814589\\
% 150.680541992188 -216.221716488262\\
% 150.833129882812 -216.187322026602\\
% 150.985717773438 -216.153282068209\\
% 151.138305664062 -216.119598006874\\
% 151.290893554688 -216.086271006124\\
% 151.443481445312 -216.053302012923\\
% 151.596069335938 -216.020691770551\\
% 151.748657226562 -215.988440830741\\
% 151.901245117188 -215.956549565108\\
% 152.053833007812 -215.925018175922\\
% 152.206420898438 -215.893846706251\\
% 152.359008789062 -215.863035049545\\
% 152.511596679688 -215.832582958652\\
% 152.664184570312 -215.802490054343\\
% 152.816772460938 -215.772755833342\\
% 152.969360351562 -215.743379675921\\
% 153.121948242188 -215.714360853055\\
% 153.274536132812 -215.685698533191\\
% 153.427124023438 -215.657391788643\\
% 153.579711914062 -215.629439601618\\
% 153.732299804688 -215.601840869939\\
% 153.884887695312 -215.574594412424\\
% 154.037475585938 -215.547698973992\\
% 154.190063476562 -215.521153230489\\
% 154.342651367188 -215.494955793239\\
% 154.495239257812 -215.469105213366\\
% 154.647827148438 -215.443599985869\\
% 154.800415039062 -215.418438553488\\
% 154.953002929688 -215.393619310354\\
% 155.105590820312 -215.369140605446\\
% 155.258178710938 -215.345000745866\\
% 155.410766601562 -215.321197999938\\
% 155.563354492188 -215.297730600134\\
% 155.715942382812 -215.274596745854\\
% 155.868530273438 -215.251794606056\\
% 156.021118164062 -215.229322321737\\
% 156.173706054688 -215.207178008293\\
% 156.326293945312 -215.185359757753\\
% 156.478881835938 -215.163865640879\\
% 156.631469726562 -215.142693709175\\
% 156.784057617188 -215.121841996777\\
% 156.936645507812 -215.101308522242\\
% 157.089233398438 -215.081091290245\\
% 157.241821289062 -215.061188293185\\
% 157.394409179688 -215.041597512706\\
% 157.546997070312 -215.022316921129\\
% 157.699584960938 -215.003344482821\\
% 157.852172851562 -214.984678155471\\
% 158.004760742188 -214.966315891321\\
% 158.157348632812 -214.948255638306\\
% 158.309936523438 -214.930495341148\\
% 158.462524414062 -214.913032942386\\
% 158.615112304688 -214.895866383347\\
% 158.767700195312 -214.878993605065\\
% 158.920288085938 -214.862412549153\\
% 159.072875976562 -214.846121158617\\
% 159.225463867188 -214.830117378633\\
% 159.378051757812 -214.814399157282\\
% 159.530639648438 -214.798964446227\\
% 159.683227539062 -214.783811201377\\
% 159.835815429688 -214.768937383484\\
% 159.988403320312 -214.754340958733\\
% 160.140991210938 -214.740019899274\\
% 160.293579101562 -214.725972183737\\
% 160.446166992188 -214.712195797713\\
% 160.598754882812 -214.6986887342\\
% 160.751342773438 -214.685448994028\\
% 160.903930664062 -214.672474586258\\
% 161.056518554688 -214.65976352855\\
% 161.209106445312 -214.647313847514\\
% 161.361694335938 -214.635123579029\\
% 161.514282226562 -214.623190768558\\
% 161.666870117188 -214.61151347142\\
% 161.819458007812 -214.600089753058\\
% 161.972045898438 -214.588917689287\\
% 162.124633789062 -214.577995366517\\
% 162.277221679688 -214.56732088197\\
% 162.429809570312 -214.556892343867\\
% 162.582397460938 -214.546707871618\\
% 162.734985351562 -214.536765595978\\
% 162.887573242188 -214.527063659205\\
% 163.040161132812 -214.517600215199\\
% 163.192749023438 -214.508373429625\\
% 163.345336914062 -214.499381480031\\
% 163.497924804688 -214.490622555953\\
% 163.650512695312 -214.482094859004\\
% 163.803100585938 -214.473796602962\\
% 163.955688476562 -214.465726013838\\
% 164.108276367188 -214.457881329947\\
% 164.260864257812 -214.45026080196\\
% 164.413452148438 -214.442862692953\\
% 164.566040039063 -214.435685278446\\
% 164.718627929688 -214.428726846438\\
% 164.871215820312 -214.42198569743\\
% 165.023803710938 -214.415460144444\\
% 165.176391601562 -214.409148513041\\
% 165.328979492188 -214.403049141319\\
% 165.481567382812 -214.397160379922\\
% 165.634155273438 -214.391480592033\\
% 165.786743164063 -214.386008153364\\
% 165.939331054688 -214.380741452141\\
% 166.091918945312 -214.375678889091\\
% 166.244506835938 -214.370818877413\\
% 166.397094726562 -214.36615984275\\
% 166.549682617188 -214.361700223168\\
% 166.702270507812 -214.357438469111\\
% 166.854858398438 -214.353373043368\\
% 167.007446289063 -214.349502421034\\
% 167.160034179688 -214.345825089461\\
% 167.312622070312 -214.342339548215\\
% 167.465209960938 -214.339044309024\\
% 167.617797851562 -214.335937895731\\
% 167.770385742188 -214.333018844229\\
% 167.922973632812 -214.330285702418\\
% 168.106079101562 -214.327249306126\\
% 168.289184570312 -214.324476080556\\
% 168.472290039063 -214.321963584159\\
% 168.655395507812 -214.319709399706\\
% 168.838500976562 -214.317711134121\\
% 169.021606445312 -214.315966418314\\
% 169.204711914063 -214.314472907001\\
% 169.387817382812 -214.313228278533\\
% 169.570922851562 -214.312230234716\\
% 169.754028320312 -214.31147650062\\
% 169.937133789062 -214.310964824404\\
% 170.120239257812 -214.31069297712\\
% 170.303344726562 -214.310658752527\\
% 170.486450195312 -214.310859966896\\
% 170.669555664062 -214.311294458819\\
% 170.852661132813 -214.311960089015\\
% 171.035766601562 -214.312854740129\\
% 171.218872070312 -214.313976316541\\
% 171.401977539062 -214.315322744163\\
% 171.585083007813 -214.316891970245\\
% 171.768188476562 -214.318681963173\\
% 171.951293945312 -214.320690712271\\
% 172.134399414062 -214.322916227601\\
% 172.317504882813 -214.325356539765\\
% 172.500610351562 -214.328009699703\\
% 172.683715820312 -214.330873778496\\
% 172.866821289062 -214.333946867167\\
% 173.049926757813 -214.337227076484\\
% 173.233032226562 -214.340712536755\\
% 173.416137695312 -214.344401397641\\
% 173.599243164062 -214.348291827953\\
% 173.782348632813 -214.352382015455\\
% 173.965454101562 -214.356670166673\\
% 174.148559570312 -214.3611545067\\
% 174.331665039062 -214.365833279001\\
% 174.514770507813 -214.370704745223\\
% 174.697875976562 -214.375767185\\
% 174.880981445312 -214.38101889577\\
% 175.064086914062 -214.386458192578\\
% 175.247192382813 -214.392083407895\\
% 175.430297851562 -214.397892891431\\
% 175.613403320312 -214.403885009945\\
% 175.796508789062 -214.410058147067\\
% 175.979614257813 -214.416410703112\\
% 176.162719726562 -214.422941094903\\
% 176.345825195312 -214.429647755586\\
% 176.528930664062 -214.436529134461\\
% 176.712036132812 -214.443583696792\\
% 176.895141601562 -214.450809923649\\
% 177.078247070312 -214.458206311717\\
% 177.261352539062 -214.465771373139\\
% 177.444458007812 -214.473503635333\\
% 177.627563476563 -214.481401640834\\
% 177.810668945312 -214.489463947119\\
% 177.993774414062 -214.497689126444\\
% 178.176879882812 -214.506075765681\\
% 178.359985351563 -214.514622466152\\
% 178.543090820312 -214.52332784347\\
% 178.726196289062 -214.532190527382\\
% 178.909301757812 -214.541209161606\\
% 179.092407226563 -214.550382403677\\
% 179.275512695312 -214.559708924794\\
% 179.458618164062 -214.569187409664\\
% 179.641723632812 -214.57881655635\\
% 179.824829101563 -214.588595076127\\
% 180.007934570312 -214.598521693325\\
% 180.191040039062 -214.608595145187\\
% 180.374145507812 -214.618814181725\\
% 180.557250976563 -214.629177565572\\
% 180.740356445312 -214.639684071844\\
% 180.923461914062 -214.650332487993\\
% 181.106567382812 -214.661121613679\\
% 181.289672851563 -214.672050260619\\
% 181.472778320312 -214.68311725246\\
% 181.655883789062 -214.694321424643\\
% 181.838989257812 -214.705661624269\\
% 182.022094726563 -214.717136709964\\
% 182.205200195312 -214.728745551754\\
% 182.388305664062 -214.740487030936\\
% 182.571411132812 -214.752360039947\\
% 182.754516601563 -214.764363482242\\
% 182.937622070312 -214.776496272166\\
% 183.120727539062 -214.788757334836\\
% 183.303833007812 -214.801145606016\\
% 183.486938476562 -214.813660031995\\
% 183.670043945312 -214.826299569479\\
% 183.853149414062 -214.839063185456\\
% 184.036254882812 -214.851949857095\\
% 184.219360351562 -214.864958571627\\
% 184.402465820313 -214.878088326228\\
% 184.585571289062 -214.891338127914\\
% 184.768676757812 -214.904706993424\\
% 184.951782226562 -214.918193949113\\
% 185.134887695313 -214.931798030845\\
% 185.317993164062 -214.945518283887\\
% 185.501098632812 -214.959353762797\\
% 185.684204101562 -214.973303531332\\
% 185.867309570313 -214.98736666233\\
% 186.050415039062 -215.001542237624\\
% 186.233520507812 -215.015829347927\\
% 186.416625976562 -215.03022709274\\
% 186.599731445313 -215.044734580258\\
% 186.782836914062 -215.059350927263\\
% 186.965942382812 -215.074075259034\\
% 187.149047851562 -215.088906709251\\
% 187.332153320313 -215.103844419903\\
% 187.515258789062 -215.118887541194\\
% 187.698364257812 -215.134035231453\\
% 187.881469726562 -215.14928665704\\
% 188.064575195313 -215.164640992263\\
% 188.247680664062 -215.180097419284\\
% 188.430786132812 -215.195655128036\\
% 188.613891601562 -215.211313316136\\
% 188.796997070313 -215.227071188795\\
% 188.980102539062 -215.242927958747\\
% 189.163208007812 -215.258882846151\\
% 189.346313476562 -215.274935078519\\
% 189.529418945313 -215.29108389063\\
% 189.712524414062 -215.307328524454\\
% 189.895629882812 -215.323668229071\\
% 190.078735351562 -215.340102260589\\
% 190.261840820312 -215.356629882078\\
% 190.444946289063 -215.373250363481\\
% 190.628051757812 -215.389962981548\\
% 190.811157226562 -215.406767019755\\
% 190.994262695312 -215.423661768236\\
% 191.177368164063 -215.440646523707\\
% 191.360473632812 -215.457720589397\\
% 191.543579101562 -215.474883274976\\
% 191.726684570312 -215.49213389648\\
% 191.909790039063 -215.509471776253\\
% 192.092895507812 -215.526896242869\\
% 192.276000976562 -215.544406631069\\
% 192.459106445312 -215.562002281691\\
% 192.642211914063 -215.579682541608\\
% 192.825317382812 -215.597446763663\\
% 193.008422851562 -215.615294306598\\
% 193.191528320312 -215.633224535001\\
% 193.374633789063 -215.651236819236\\
% 193.557739257812 -215.66933053538\\
% 193.740844726562 -215.687505065167\\
% 193.923950195312 -215.705759795925\\
% 194.107055664063 -215.724094120515\\
% 194.290161132812 -215.742507437274\\
% 194.473266601562 -215.760999149957\\
% 194.656372070312 -215.779568667673\\
% 194.839477539063 -215.798215404839\\
% 195.022583007812 -215.816938781115\\
% 195.205688476562 -215.835738221349\\
% 195.388793945312 -215.85461315553\\
% 195.571899414063 -215.873563018721\\
% 195.755004882812 -215.892587251018\\
% 195.938110351562 -215.911685297484\\
% 196.121215820312 -215.930856608113\\
% 196.304321289063 -215.950100637761\\
% 196.487426757812 -215.969416846107\\
% 196.670532226562 -215.988804697596\\
% 196.853637695312 -216.008263661391\\
% 197.036743164062 -216.027793211328\\
% 197.219848632813 -216.047392825856\\
% 197.402954101562 -216.067061988004\\
% 197.586059570312 -216.086800185317\\
% 197.769165039062 -216.106606909823\\
% 197.952270507813 -216.126481657977\\
% 198.135375976562 -216.146423930621\\
% 198.318481445312 -216.166433232935\\
% 198.501586914062 -216.186509074395\\
% 198.715209960938 -216.210014339007\\
% 198.928833007813 -216.233608744127\\
% 199.142456054688 -216.257291532494\\
% 199.356079101562 -216.281061954456\\
% 199.569702148438 -216.304919267872\\
% 199.783325195312 -216.328862738029\\
% 199.996948242188 -216.352891637552\\
% 200.210571289062 -216.377005246312\\
% 200.424194335938 -216.401202851347\\
% 200.637817382813 -216.425483746767\\
% 200.851440429688 -216.449847233684\\
% 201.065063476562 -216.474292620114\\
% 201.278686523438 -216.498819220906\\
% 201.492309570312 -216.523426357659\\
% 201.705932617188 -216.548113358638\\
% 201.919555664062 -216.572879558705\\
% 202.133178710938 -216.597724299229\\
% 202.346801757813 -216.622646928021\\
% 202.560424804688 -216.647646799254\\
% 202.774047851562 -216.672723273384\\
% 202.987670898438 -216.697875717088\\
% 203.201293945312 -216.723103503181\\
% 203.414916992188 -216.74840601055\\
% 203.628540039062 -216.773782624082\\
% 203.842163085938 -216.799232734596\\
% 204.055786132812 -216.824755738772\\
% 204.269409179688 -216.850351039083\\
% 204.483032226563 -216.876018043733\\
% 204.696655273438 -216.901756166586\\
% 204.910278320312 -216.927564827104\\
% 205.123901367188 -216.95344345028\\
% 205.337524414062 -216.979391466578\\
% 205.551147460938 -217.005408311866\\
% 205.764770507812 -217.031493427362\\
% 205.978393554688 -217.057646259562\\
% 206.192016601563 -217.083866260191\\
% 206.405639648438 -217.110152886137\\
% 206.619262695312 -217.136505599394\\
% 206.832885742188 -217.162923867004\\
% 207.046508789062 -217.189407161003\\
% 207.260131835938 -217.215954958358\\
% 207.473754882812 -217.242566740918\\
% 207.687377929688 -217.269241995358\\
% 207.901000976563 -217.295980213121\\
% 208.114624023438 -217.322780890369\\
% 208.328247070312 -217.349643527925\\
% 208.541870117188 -217.37656763123\\
% 208.755493164062 -217.40355271028\\
% 208.969116210938 -217.430598279585\\
% 209.182739257812 -217.457703858112\\
% 209.396362304688 -217.484868969241\\
% 209.609985351563 -217.512093140713\\
% 209.823608398438 -217.539375904583\\
% 210.037231445312 -217.566716797167\\
% 210.250854492188 -217.594115359008\\
% 210.464477539062 -217.621571134813\\
% 210.678100585938 -217.649083673422\\
% 210.891723632812 -217.676652527752\\
% 211.105346679688 -217.70427725476\\
% 211.318969726562 -217.731957415393\\
% 211.532592773438 -217.759692574547\\
% 211.746215820313 -217.787482301025\\
% 211.959838867188 -217.815326167493\\
% 212.173461914062 -217.84322375044\\
% 212.387084960938 -217.871174630128\\
% 212.600708007812 -217.899178390569\\
% 212.814331054688 -217.927234619465\\
% 213.027954101562 -217.95534290818\\
% 213.241577148438 -217.983502851698\\
% 213.455200195313 -218.011714048582\\
% 213.668823242188 -218.039976100938\\
% 213.882446289062 -218.068288614374\\
% 214.096069335938 -218.096651197969\\
% 214.309692382812 -218.125063464224\\
% 214.523315429688 -218.15352502904\\
% 214.736938476562 -218.182035511673\\
% 214.950561523438 -218.210594534697\\
% 215.164184570313 -218.239201723974\\
% 215.377807617188 -218.267856708618\\
% 215.591430664062 -218.296559120956\\
% 215.805053710938 -218.325308596502\\
% 216.018676757812 -218.354104773915\\
% 216.232299804688 -218.382947294972\\
% 216.445922851562 -218.411835804532\\
% 216.659545898438 -218.440769950505\\
% 216.873168945312 -218.469749383822\\
% 217.086791992188 -218.498773758395\\
% 217.300415039063 -218.527842731098\\
% 217.514038085938 -218.556955961725\\
% 217.727661132812 -218.58611311297\\
% 217.941284179688 -218.615313850386\\
% 218.154907226562 -218.644557842364\\
% 218.368530273438 -218.673844760098\\
% 218.582153320312 -218.70317427756\\
% 218.795776367188 -218.732546071469\\
% 219.009399414063 -218.761959821263\\
% 219.223022460938 -218.79141520907\\
% 219.436645507812 -218.820911919681\\
% 219.650268554688 -218.850449640526\\
% 219.863891601562 -218.880028061643\\
% 220.077514648438 -218.909646875649\\
% 220.291137695312 -218.93930577772\\
% 220.504760742188 -218.969004465562\\
% 220.718383789063 -218.998742639382\\
% 220.932006835938 -219.028520001867\\
% 221.145629882812 -219.058336258158\\
% 221.359252929688 -219.088191115825\\
% 221.572875976562 -219.118084284838\\
% 221.786499023438 -219.148015477548\\
% 222.000122070312 -219.177984408663\\
% 222.213745117188 -219.207990795222\\
% 222.427368164063 -219.23803435657\\
% 222.640991210938 -219.268114814337\\
% 222.854614257812 -219.298231892418\\
% 223.068237304688 -219.328385316944\\
% 223.281860351562 -219.358574816261\\
% 223.495483398438 -219.388800120915\\
% 223.709106445312 -219.419060963623\\
% 223.922729492188 -219.449357079246\\
% 224.136352539062 -219.479688204784\\
% 224.349975585938 -219.510054079341\\
% 224.563598632813 -219.540454444107\\
% 224.777221679688 -219.570889042341\\
% 224.990844726562 -219.601357619346\\
% 225.204467773438 -219.631859922454\\
% 225.418090820312 -219.662395701\\
% 225.631713867188 -219.692964706306\\
% 225.845336914062 -219.723566691661\\
% 226.058959960938 -219.754201412301\\
% 226.272583007813 -219.784868625389\\
% 226.486206054688 -219.81556809\\
% 226.699829101562 -219.846299567095\\
% 226.913452148438 -219.877062819512\\
% 227.127075195312 -219.907857611938\\
% 227.340698242188 -219.9386837109\\
% 227.554321289062 -219.96954088474\\
% 227.767944335938 -220.000428903598\\
% 227.981567382813 -220.031347539403\\
% 228.195190429688 -220.062296565842\\
% 228.408813476562 -220.093275758357\\
% 228.622436523438 -220.124284894114\\
% 228.836059570312 -220.155323752\\
% 229.049682617188 -220.186392112595\\
% 229.293823242188 -220.221934656951\\
% 229.537963867188 -220.257515128767\\
% 229.782104492188 -220.2931332085\\
% 230.026245117188 -220.328788579382\\
% 230.270385742188 -220.364480927396\\
% 230.514526367188 -220.400209941241\\
% 230.758666992188 -220.435975312308\\
% 231.002807617188 -220.471776734647\\
% 231.246948242188 -220.507613904937\\
% 231.491088867188 -220.543486522461\\
% 231.735229492188 -220.579394289076\\
% 231.979370117188 -220.615336909183\\
% 232.223510742188 -220.651314089706\\
% 232.467651367188 -220.687325540053\\
% 232.711791992188 -220.723370972104\\
% 232.955932617188 -220.759450100172\\
% 233.200073242188 -220.795562640985\\
% 233.444213867188 -220.831708313655\\
% 233.688354492188 -220.867886839652\\
% 233.932495117188 -220.904097942785\\
% 234.176635742188 -220.94034134917\\
% 234.420776367188 -220.976616787211\\
% 234.664916992188 -221.012923987571\\
% 234.909057617188 -221.049262683147\\
% 235.153198242188 -221.085632609056\\
% 235.397338867188 -221.122033502597\\
% 235.641479492188 -221.158465103241\\
% 235.885620117188 -221.194927152599\\
% 236.129760742188 -221.231419394402\\
% 236.373901367188 -221.267941574482\\
% 236.618041992188 -221.304493440746\\
% 236.862182617188 -221.341074743152\\
% 237.106323242188 -221.377685233697\\
% 237.350463867188 -221.414324666383\\
% 237.594604492188 -221.450992797204\\
% 237.838745117188 -221.487689384122\\
% 238.082885742188 -221.524414187052\\
% 238.327026367188 -221.561166967831\\
% 238.571166992188 -221.597947490208\\
% 238.815307617188 -221.634755519819\\
% 239.059448242188 -221.671590824167\\
% 239.303588867188 -221.708453172605\\
% 239.547729492188 -221.745342336315\\
% 239.791870117188 -221.782258088291\\
% 240.036010742188 -221.819200203317\\
% 240.280151367188 -221.856168457952\\
% 240.524291992188 -221.893162630506\\
% 240.768432617188 -221.930182501033\\
% 241.012573242188 -221.967227851297\\
% 241.256713867188 -222.004298464771\\
% 241.500854492188 -222.041394126604\\
% 241.744995117188 -222.078514623616\\
% 241.989135742188 -222.115659744276\\
% 242.233276367188 -222.152829278684\\
% 242.477416992188 -222.190023018553\\
% 242.721557617188 -222.227240757198\\
% 242.965698242188 -222.264482289516\\
% 243.209838867188 -222.301747411967\\
% 243.453979492188 -222.339035922565\\
% 243.698120117188 -222.376347620857\\
% 243.942260742188 -222.413682307905\\
% 244.186401367188 -222.451039786281\\
% 244.430541992188 -222.488419860039\\
% 244.674682617188 -222.525822334709\\
% 244.918823242188 -222.563247017275\\
% 245.162963867188 -222.600693716169\\
% 245.407104492188 -222.638162241246\\
% 245.651245117188 -222.675652403779\\
% 245.895385742188 -222.713164016438\\
% 246.139526367188 -222.750696893278\\
% 246.383666992188 -222.788250849727\\
% 246.627807617188 -222.82582570257\\
% 246.871948242188 -222.863421269933\\
% 247.116088867188 -222.901037371276\\
% 247.360229492188 -222.938673827374\\
% 247.604370117188 -222.976330460304\\
% 247.848510742188 -223.014007093434\\
% 248.092651367188 -223.051703551411\\
% 248.336791992188 -223.089419660147\\
% 248.580932617188 -223.1271552468\\
% 248.825073242188 -223.164910139775\\
% 249.069213867188 -223.202684168697\\
% 249.313354492188 -223.240477164411\\
% 249.557495117188 -223.27828895896\\
% 249.801635742188 -223.316119385581\\
% 250.045776367188 -223.353968278684\\
% 250.289916992188 -223.391835473852\\
% 250.534057617188 -223.429720807819\\
% 250.778198242188 -223.467624118462\\
% 251.022338867188 -223.505545244793\\
% 251.266479492188 -223.54348402694\\
% 251.510620117188 -223.581440306147\\
% 251.754760742188 -223.619413924748\\
% 251.998901367188 -223.657404726172\\
% 252.243041992188 -223.695412554922\\
% 252.487182617188 -223.733437256564\\
% 252.731323242188 -223.771478677724\\
% 252.975463867188 -223.80953666607\\
% 253.219604492188 -223.847611070305\\
% 253.463745117188 -223.885701740156\\
% 253.707885742188 -223.923808526363\\
% 253.952026367188 -223.961931280675\\
% 254.196166992188 -224.000069855826\\
% 254.440307617188 -224.038224105541\\
% 254.684448242188 -224.076393884519\\
% 254.928588867188 -224.114579048419\\
% 255.172729492188 -224.152779453858\\
% 255.416870117188 -224.190994958401\\
% 255.661010742188 -224.229225420542\\
% 255.905151367188 -224.267470699712\\
% 256.149291992188 -224.30573065625\\
% 256.393432617188 -224.34400515141\\
% 256.637573242188 -224.382294047343\\
% 256.881713867188 -224.420597207089\\
% 257.125854492188 -224.458914494575\\
% 257.369995117188 -224.497245774597\\
% 257.614135742188 -224.535590912816\\
% 257.858276367188 -224.57394977575\\
% 258.102416992188 -224.612322230764\\
% 258.346557617188 -224.65070814606\\
% 258.590698242188 -224.689107390676\\
% 258.834838867188 -224.727519834466\\
% 259.078979492188 -224.765945348104\\
% 259.323120117188 -224.804383803065\\
% 259.567260742188 -224.842835071627\\
% 259.841918945312 -224.886107906998\\
% 260.116577148438 -224.929396619403\\
% 260.391235351562 -224.972701030535\\
% 260.665893554688 -225.016020963307\\
% 260.940551757812 -225.059356241848\\
% 261.215209960938 -225.102706691485\\
% 261.489868164062 -225.146072138727\\
% 261.764526367188 -225.18945241126\\
% 262.039184570312 -225.232847337923\\
% 262.313842773438 -225.276256748702\\
% 262.588500976562 -225.319680474719\\
% 262.863159179688 -225.363118348211\\
% 263.137817382812 -225.406570202528\\
% 263.412475585938 -225.45003587211\\
% 263.687133789062 -225.493515192483\\
% 263.961791992188 -225.537008000242\\
% 264.236450195312 -225.580514133041\\
% 264.511108398438 -225.624033429581\\
% 264.785766601562 -225.667565729597\\
% 265.060424804688 -225.711110873847\\
% 265.335083007812 -225.754668704102\\
% 265.609741210938 -225.79823906313\\
% 265.884399414062 -225.841821794692\\
% 266.159057617188 -225.885416743524\\
% 266.433715820312 -225.929023755329\\
% 266.708374023438 -225.972642676764\\
% 266.983032226562 -226.016273355433\\
% 267.257690429688 -226.059915639873\\
% 267.532348632812 -226.103569379543\\
% 267.807006835938 -226.147234424816\\
% 268.081665039062 -226.190910626967\\
% 268.356323242188 -226.234597838163\\
% 268.630981445312 -226.278295911451\\
% 268.905639648438 -226.322004700751\\
% 269.180297851562 -226.365724060843\\
% 269.454956054688 -226.409453847362\\
% 269.729614257812 -226.453193916779\\
% 270.004272460938 -226.496944126403\\
% 270.278930664062 -226.54070433436\\
% 270.553588867188 -226.584474399589\\
% 270.828247070312 -226.628254181836\\
% 271.102905273438 -226.67204354164\\
% 271.377563476562 -226.71584234032\\
% 271.652221679688 -226.759650439974\\
% 271.926879882812 -226.803467703466\\
% 272.201538085938 -226.847293994415\\
% 272.476196289062 -226.891129177191\\
% 272.750854492188 -226.9349731169\\
% 273.025512695312 -226.978825679382\\
% 273.300170898438 -227.022686731195\\
% 273.574829101562 -227.066556139613\\
% 273.849487304688 -227.110433772613\\
% 274.124145507812 -227.154319498869\\
% 274.398803710938 -227.198213187742\\
% 274.673461914062 -227.242114709273\\
% 274.948120117188 -227.286023934172\\
% 275.222778320312 -227.329940733816\\
% 275.497436523438 -227.373864980233\\
% 275.772094726562 -227.417796546098\\
% 276.046752929688 -227.461735304728\\
% 276.321411132812 -227.505681130069\\
% 276.596069335938 -227.549633896687\\
% 276.870727539062 -227.59359347977\\
% 277.145385742188 -227.637559755109\\
% 277.420043945312 -227.681532599094\\
% 277.694702148438 -227.725511888712\\
% 277.969360351562 -227.76949750153\\
% 278.244018554688 -227.813489315698\\
% 278.518676757812 -227.85748720993\\
% 278.793334960938 -227.901491063509\\
% 279.067993164062 -227.945500756267\\
% 279.342651367188 -227.989516168591\\
% 279.617309570312 -228.033537181405\\
% 279.891967773438 -228.077563676168\\
% 280.166625976562 -228.121595534867\\
% 280.441284179688 -228.165632640009\\
% 280.715942382812 -228.209674874614\\
% 280.990600585938 -228.253722122207\\
% 281.265258789062 -228.297774266816\\
% 281.539916992188 -228.341831192961\\
% 281.814575195312 -228.385892785645\\
% 282.089233398438 -228.429958930352\\
% 282.363891601562 -228.474029513042\\
% 282.638549804688 -228.518104420138\\
% 282.913208007812 -228.562183538525\\
% 283.187866210938 -228.606266755541\\
% 283.462524414062 -228.650353958968\\
% 283.737182617188 -228.694445037034\\
% 284.011840820312 -228.738539878398\\
% 284.286499023438 -228.782638372148\\
% 284.561157226562 -228.826740407794\\
% 284.835815429688 -228.870845875262\\
% 285.110473632812 -228.914954664886\\
% 285.385131835938 -228.959066667408\\
% 285.659790039062 -229.003181773962\\
% 285.934448242188 -229.047299876077\\
% 286.209106445312 -229.091420865667\\
% 286.483764648438 -229.135544635024\\
% 286.758422851562 -229.179671076816\\
% 287.033081054688 -229.223800084079\\
% 287.307739257812 -229.267931550209\\
% 287.582397460938 -229.312065368959\\
% 287.857055664062 -229.356201434435\\
% 288.131713867188 -229.400339641085\\
% 288.406372070312 -229.444479883698\\
% 288.681030273438 -229.488622057397\\
% 288.955688476562 -229.532766057631\\
% 289.230346679688 -229.576911780174\\
% 289.505004882812 -229.621059121116\\
% 289.779663085938 -229.665207976859\\
% 290.054321289062 -229.70935824411\\
% 290.359497070312 -229.758415626827\\
% 290.664672851562 -229.807474484153\\
% 290.969848632812 -229.85653467568\\
% 291.275024414062 -229.905596061428\\
% 291.580200195312 -229.954658501845\\
% 291.885375976562 -230.00372185779\\
% 292.190551757812 -230.052785990529\\
% 292.495727539062 -230.101850761722\\
% 292.800903320312 -230.150916033421\\
% 293.106079101562 -230.199981668055\\
% 293.411254882812 -230.249047528421\\
% 293.716430664062 -230.298113477683\\
% 294.021606445312 -230.347179379355\\
% 294.326782226562 -230.396245097297\\
% 294.631958007812 -230.445310495705\\
% 294.937133789062 -230.494375439104\\
% 295.242309570312 -230.543439792338\\
% 295.547485351562 -230.59250342056\\
% 295.852661132812 -230.641566189234\\
% 296.157836914062 -230.690627964108\\
% 296.463012695312 -230.739688611225\\
% 296.768188476562 -230.788747996904\\
% 297.073364257812 -230.837805987733\\
% 297.378540039062 -230.886862450564\\
% 297.683715820312 -230.935917252501\\
% 297.988891601562 -230.984970260898\\
% 298.294067382812 -231.034021343345\\
% 298.599243164062 -231.083070367663\\
% 298.904418945312 -231.132117201895\\
% 299.209594726562 -231.181161714297\\
% 299.514770507812 -231.230203773334\\
% 299.819946289062 -231.279243247671\\
% 300.125122070312 -231.32828000616\\
% 300.430297851562 -231.377313917839\\
% 300.735473632812 -231.426344851922\\
% 301.040649414062 -231.475372677788\\
% 301.345825195312 -231.52439726498\\
% 301.651000976562 -231.573418483192\\
% 301.956176757812 -231.62243620226\\
% 302.261352539062 -231.671450292162\\
% 302.566528320312 -231.720460623\\
% 302.871704101562 -231.769467065005\\
% 303.176879882812 -231.818469488517\\
% 303.482055664062 -231.867467763985\\
% 303.787231445312 -231.916461761957\\
% 304.092407226562 -231.965451353073\\
% 304.397583007812 -232.014436408056\\
% 304.702758789062 -232.06341679771\\
% 305.007934570312 -232.112392392905\\
% 305.313110351562 -232.161363064572\\
% 305.618286132812 -232.210328683701\\
% 305.923461914062 -232.259289121326\\
% 306.228637695312 -232.308244248519\\
% 306.533813476562 -232.357193936391\\
% 306.838989257812 -232.40613805607\\
% 307.144165039062 -232.455076478708\\
% 307.449340820312 -232.504009075462\\
% 307.754516601562 -232.552935717497\\
% 308.059692382812 -232.60185627597\\
% 308.364868164062 -232.650770622026\\
% 308.670043945312 -232.699678626793\\
% 308.975219726562 -232.748580161372\\
% 309.280395507812 -232.797475096827\\
% 309.585571289062 -232.846363304186\\
% 309.890747070312 -232.895244654424\\
% 310.195922851562 -232.944119018463\\
% 310.501098632812 -232.992986267159\\
% 310.806274414062 -233.041846271299\\
% 311.111450195312 -233.090698901593\\
% 311.416625976562 -233.139544028663\\
% 311.721801757812 -233.188381523045\\
% 312.026977539062 -233.237211255164\\
% 312.332153320312 -233.286033095347\\
% 312.637329101562 -233.334846913803\\
% 312.942504882812 -233.383652580619\\
% 313.247680664062 -233.432449965752\\
% 313.552856445312 -233.481238939026\\
% 313.858032226562 -233.530019370113\\
% 314.163208007812 -233.578791128543\\
% 314.468383789062 -233.627554083679\\
% 314.773559570312 -233.676308104721\\
% 315.078735351562 -233.725053060697\\
% 315.383911132812 -233.773788820451\\
% 315.689086914062 -233.822515252638\\
% 315.994262695312 -233.87123222572\\
% 316.299438476562 -233.91993960795\\
% 316.604614257812 -233.968637267375\\
% 316.909790039062 -234.01732507182\\
% 317.214965820312 -234.066002888886\\
% 317.520141601562 -234.114670585939\\
% 317.825317382812 -234.1633280301\\
% 318.130493164062 -234.211975088249\\
% 318.435668945312 -234.260611627003\\
% 318.740844726562 -234.309237512716\\
% 319.046020507812 -234.357852611472\\
% 319.351196289062 -234.406456789074\\
% 319.656372070312 -234.455049911036\\
% 319.961547851562 -234.503631842579\\
% 320.266723632812 -234.552202448622\\
% 320.571899414062 -234.600761593771\\
% 320.907592773438 -234.654163254121\\
% 321.243286132812 -234.707550701481\\
% 321.578979492188 -234.760923754275\\
% 321.914672851562 -234.814282230419\\
% 322.250366210938 -234.867625947314\\
% 322.586059570312 -234.920954721833\\
% 322.921752929688 -234.974268370307\\
% 323.257446289062 -235.027566708513\\
% 323.593139648438 -235.080849551657\\
% 323.928833007812 -235.134116714366\\
% 324.264526367188 -235.187368010674\\
% 324.600219726562 -235.240603254007\\
% 324.935913085938 -235.293822257167\\
% 325.271606445312 -235.347024832327\\
% 325.607299804688 -235.400210791008\\
% 325.942993164062 -235.453379944071\\
% 326.278686523438 -235.506532101704\\
% 326.614379882813 -235.559667073405\\
% 326.950073242188 -235.612784667968\\
% 327.285766601563 -235.665884693474\\
% 327.621459960938 -235.718966957274\\
% 327.957153320312 -235.772031265968\\
% 328.292846679688 -235.825077425407\\
% 328.628540039062 -235.878105240663\\
% 328.964233398438 -235.931114516022\\
% 329.299926757812 -235.984105054972\\
% 329.635620117188 -236.037076660181\\
% 329.971313476562 -236.090029133488\\
% 330.307006835938 -236.142962275886\\
% 330.642700195312 -236.19587588751\\
% 330.978393554688 -236.248769767618\\
% 331.314086914062 -236.301643714576\\
% 331.649780273438 -236.354497525849\\
% 331.985473632812 -236.407330997978\\
% 332.321166992188 -236.460143926569\\
% 332.656860351563 -236.512936106276\\
% 332.992553710938 -236.565707330787\\
% 333.328247070313 -236.618457392805\\
% 333.663940429688 -236.671186084036\\
% 333.999633789062 -236.72389319517\\
% 334.335327148438 -236.776578515866\\
% 334.671020507812 -236.829241834734\\
% 335.006713867188 -236.881882939322\\
% 335.342407226562 -236.9345016161\\
% 335.678100585938 -236.987097650435\\
% 336.013793945312 -237.039670826585\\
% 336.349487304688 -237.092220927672\\
% 336.685180664062 -237.144747735674\\
% 337.020874023438 -237.197251031404\\
% 337.356567382812 -237.249730594487\\
% 337.692260742188 -237.302186203351\\
% 338.027954101562 -237.354617635207\\
% 338.363647460938 -237.407024666025\\
% 338.699340820313 -237.459407070524\\
% 339.035034179688 -237.511764622149\\
% 339.370727539063 -237.564097093053\\
% 339.706420898438 -237.61640425408\\
% 340.042114257812 -237.668685874745\\
% 340.377807617188 -237.720941723214\\
% 340.713500976562 -237.773171566289\\
% 341.049194335938 -237.825375169382\\
% 341.384887695312 -237.8775522965\\
% 341.720581054688 -237.929702710225\\
% 342.056274414062 -237.981826171694\\
% 342.391967773438 -238.033922440575\\
% 342.727661132812 -238.085991275052\\
% 343.063354492188 -238.138032431801\\
% 343.399047851562 -238.19004566597\\
% 343.734741210938 -238.24203073116\\
% 344.070434570312 -238.293987379398\\
% 344.406127929688 -238.345915361125\\
% 344.741821289063 -238.397814425162\\
% 345.077514648438 -238.449684318701\\
% 345.413208007813 -238.50152478727\\
% 345.748901367188 -238.553335574722\\
% 346.084594726562 -238.605116423207\\
% 346.420288085938 -238.656867073147\\
% 346.755981445312 -238.708587263215\\
% 347.091674804688 -238.760276730316\\
% 347.427368164062 -238.811935209554\\
% 347.763061523438 -238.863562434217\\
% 348.098754882812 -238.915158135745\\
% 348.434448242188 -238.966722043712\\
% 348.770141601562 -239.018253885798\\
% 349.105834960938 -239.069753387765\\
% 349.441528320312 -239.121220273428\\
% 349.777221679688 -239.172654264637\\
% 350.112915039062 -239.224055081241\\
% 350.448608398438 -239.275422441071\\
% 350.784301757813 -239.32675605991\\
% 351.119995117188 -239.378055651461\\
% 351.486206054688 -239.433979695321\\
% 351.852416992188 -239.48986252333\\
% 352.218627929688 -239.545703755116\\
% 352.584838867188 -239.601503007201\\
% 352.951049804688 -239.657259892963\\
% 353.317260742188 -239.712974022581\\
% 353.683471679688 -239.768645003002\\
% 354.049682617188 -239.824272437884\\
% 354.415893554688 -239.879855927554\\
% 354.782104492188 -239.935395068959\\
% 355.148315429688 -239.990889455621\\
% 355.514526367188 -240.046338677584\\
% 355.880737304688 -240.101742321368\\
% 356.246948242188 -240.157099969919\\
% 356.613159179688 -240.212411202554\\
% 356.979370117188 -240.267675594914\\
% 357.345581054688 -240.322892718911\\
% 357.711791992188 -240.378062142674\\
% 358.078002929688 -240.433183430495\\
% 358.444213867188 -240.488256142776\\
% 358.810424804688 -240.543279835972\\
% 359.176635742188 -240.598254062538\\
% 359.542846679688 -240.653178370865\\
% 359.909057617188 -240.708052305234\\
% 360.275268554688 -240.762875405746\\
% 360.641479492188 -240.817647208269\\
% 361.007690429688 -240.872367244376\\
% 361.373901367188 -240.927035041283\\
% 361.740112304688 -240.981650121792\\
% 362.106323242188 -241.036212004215\\
% 362.472534179688 -241.090720202329\\
% 362.838745117188 -241.145174225296\\
% 363.204956054688 -241.1995735776\\
% 363.571166992188 -241.253917758987\\
% 363.937377929688 -241.30820626439\\
% 364.303588867188 -241.362438583862\\
% 364.669799804688 -241.416614202506\\
% 365.036010742188 -241.470732600406\\
% 365.402221679688 -241.524793252552\\
% 365.768432617188 -241.578795628768\\
% 366.134643554688 -241.632739193637\\
% 366.500854492188 -241.686623406427\\
% 366.867065429688 -241.74044772101\\
% 367.233276367188 -241.79421158579\\
% 367.599487304688 -241.84791444362\\
% 367.965698242188 -241.901555731722\\
% 368.331909179688 -241.955134881605\\
% 368.698120117188 -242.008651318982\\
% 369.064331054688 -242.062104463687\\
% 369.430541992188 -242.115493729588\\
% 369.796752929688 -242.168818524501\\
% 370.162963867188 -242.222078250095\\
% 370.529174804688 -242.275272301815\\
% 370.895385742188 -242.32840006878\\
% 371.261596679688 -242.381460933691\\
% 371.627807617188 -242.434454272741\\
% 371.994018554688 -242.487379455514\\
% 372.360229492188 -242.540235844893\\
% 372.726440429688 -242.593022796954\\
% 373.092651367188 -242.645739660871\\
% 373.458862304688 -242.69838577881\\
% 373.825073242188 -242.750960485825\\
% 374.191284179688 -242.803463109753\\
% 374.557495117188 -242.855892971107\\
% 374.923706054688 -242.908249382961\\
% 375.289916992188 -242.960531650847\\
% 375.656127929688 -243.012739072632\\
% 376.022338867188 -243.06487093841\\
% 376.388549804688 -243.116926530383\\
% 376.754760742188 -243.168905122738\\
% 377.120971679688 -243.220805981528\\
% 377.487182617188 -243.272628364551\\
% 377.853393554688 -243.324371521217\\
% 378.219604492188 -243.376034692428\\
% 378.585815429688 -243.427617110442\\
% 378.952026367188 -243.479117998746\\
% 379.318237304688 -243.530536571914\\
% 379.684448242188 -243.581872035476\\
% 380.050659179688 -243.633123585775\\
% 380.416870117188 -243.684290409826\\
% 380.783081054688 -243.735371685171\\
% 381.149291992188 -243.786366579731\\
% 381.515502929688 -243.83727425166\\
% 381.912231445312 -243.89232481312\\
% 382.308959960938 -243.947270912999\\
% 382.705688476562 -244.002111440873\\
% 383.102416992188 -244.056845271952\\
% 383.499145507812 -244.111471266845\\
% 383.895874023438 -244.165988271304\\
% 384.292602539062 -244.220395115972\\
% 384.689331054688 -244.274690616117\\
% 385.086059570313 -244.328873571367\\
% 385.482788085938 -244.382942765442\\
% 385.879516601563 -244.436896965876\\
% 386.276245117188 -244.49073492373\\
% 386.672973632812 -244.544455373309\\
% 387.069702148438 -244.598057031862\\
% 387.466430664062 -244.651538599288\\
% 387.863159179688 -244.704898757823\\
% 388.259887695312 -244.758136171733\\
% 388.656616210938 -244.811249486989\\
% 389.053344726562 -244.864237330943\\
% 389.450073242188 -244.917098311995\\
% 389.846801757812 -244.969831019253\\
% 390.243530273438 -245.022434022185\\
% 390.640258789063 -245.074905870268\\
% 391.036987304688 -245.127245092616\\
% 391.433715820313 -245.17945019762\\
% 391.830444335938 -245.231519672569\\
% 392.227172851562 -245.283451983256\\
% 392.623901367188 -245.335245573592\\
% 393.020629882812 -245.386898865202\\
% 393.417358398438 -245.438410257006\\
% 393.814086914062 -245.489778124807\\
% 394.210815429688 -245.541000820858\\
% 394.607543945312 -245.592076673421\\
% 395.004272460938 -245.643003986319\\
% 395.401000976562 -245.693781038477\\
% 395.797729492188 -245.74440608345\\
% 396.194458007813 -245.79487734895\\
% 396.591186523438 -245.845193036346\\
% 396.987915039063 -245.895351320172\\
% 397.384643554688 -245.945350347609\\
% 397.781372070312 -245.995188237955\\
% 398.178100585938 -246.0448630821\\
% 398.574829101562 -246.094372941967\\
% 398.971557617188 -246.143715849957\\
% 399.368286132812 -246.192889808373\\
% 399.765014648438 -246.241892788829\\
% 400.161743164062 -246.290722731653\\
% 400.558471679688 -246.339377545268\\
% 400.955200195312 -246.387855105563\\
% 401.351928710938 -246.436153255255\\
% 401.748657226562 -246.484269803217\\
% 402.145385742188 -246.532202523813\\
% 402.542114257813 -246.579949156199\\
% 402.938842773438 -246.627507403623\\
% 403.335571289062 -246.674874932691\\
% 403.732299804688 -246.722049372632\\
% 404.129028320312 -246.769028314535\\
% 404.525756835938 -246.815809310565\\
% 404.922485351562 -246.862389873179\\
% 405.319213867188 -246.908767474292\\
% 405.715942382812 -246.954939544453\\
% 406.112670898438 -247.000903471975\\
% 406.509399414062 -247.046656602069\\
% 406.906127929688 -247.092196235931\\
% 407.302856445312 -247.137519629828\\
% 407.699584960938 -247.182623994143\\
% 408.096313476563 -247.227506492412\\
% 408.493041992188 -247.272164240322\\
% 408.889770507812 -247.316594304695\\
% 409.286499023438 -247.360793702441\\
% 409.683227539062 -247.404759399487\\
% 410.079956054688 -247.448488309668\\
% 410.476684570312 -247.491977293611\\
% 410.873413085938 -247.535223157564\\
% 411.270141601562 -247.578222652216\\
% 411.666870117188 -247.620972471474\\
% 412.063598632812 -247.663469251213\\
% 412.490844726563 -247.708948100901\\
% 412.918090820312 -247.754125158808\\
% 413.345336914062 -247.798995986904\\
% 413.772583007812 -247.843556061615\\
% 414.199829101562 -247.887800771752\\
% 414.627075195313 -247.931725416367\\
% 415.054321289062 -247.975325202556\\
% 415.481567382812 -248.018595243197\\
% 415.908813476562 -248.06153055461\\
% 416.336059570312 -248.104126054165\\
% 416.763305664063 -248.146376557804\\
% 417.190551757812 -248.188276777497\\
% 417.617797851562 -248.229821318626\\
% 418.045043945313 -248.271004677267\\
% 418.472290039062 -248.311821237426\\
% 418.899536132812 -248.352265268154\\
% 419.326782226562 -248.392330920599\\
% 419.754028320312 -248.43201222496\\
% 420.181274414063 -248.471303087338\\
% 420.608520507812 -248.510197286504\\
% 421.035766601562 -248.54868847055\\
% 421.463012695313 -248.586770153451\\
% 421.890258789062 -248.624435711507\\
% 422.317504882813 -248.661678379675\\
% 422.744750976562 -248.698491247788\\
% 423.171997070312 -248.73486725664\\
% 423.599243164063 -248.770799193969\\
% 424.026489257812 -248.806279690281\\
% 424.453735351562 -248.841301214561\\
% 424.880981445312 -248.875856069833\\
% 425.308227539062 -248.909936388569\\
% 425.735473632813 -248.943534127962\\
% 426.162719726562 -248.976641065025\\
% 426.589965820312 -249.009248791545\\
% 427.017211914063 -249.041348708838\\
% 427.444458007812 -249.072932022361\\
% 427.871704101562 -249.103989736116\\
% 428.298950195312 -249.134512646876\\
% 428.726196289062 -249.164491338203\\
% 429.153442382813 -249.193916174277\\
% 429.580688476562 -249.222777293478\\
% 430.007934570312 -249.251064601785\\
% 430.435180664062 -249.278767765908\\
% 430.862426757812 -249.305876206203\\
% 431.289672851563 -249.332379089317\\
% 431.716918945312 -249.358265320579\\
% 432.144165039062 -249.383523536122\\
% 432.571411132813 -249.40814209471\\
% 432.998657226562 -249.432109069274\\
% 433.425903320312 -249.455412238145\\
% 433.853149414062 -249.478039075949\\
% 434.280395507812 -249.499976744181\\
% 434.707641601563 -249.521212081423\\
% 435.134887695312 -249.541731593193\\
% 435.562133789062 -249.561521441419\\
% 435.989379882812 -249.580567433506\\
% 436.416625976562 -249.598855011\\
% 436.843872070313 -249.616369237806\\
% 437.271118164062 -249.633094787954\\
% 437.698364257812 -249.649015932905\\
% 438.125610351563 -249.664116528338\\
% 438.552856445312 -249.678380000441\\
% 438.980102539062 -249.691789331656\\
% 439.407348632812 -249.704327045841\\
% 439.834594726562 -249.715975192869\\
% 440.261840820313 -249.726715332582\\
% 440.689086914062 -249.736528518115\\
% 441.116333007812 -249.745395278526\\
% 441.543579101563 -249.75329560073\\
% 441.970825195312 -249.760208910676\\
% 442.398071289063 -249.766114053751\\
% 442.825317382812 -249.770989274363\\
% 443.283081054688 -249.775044389572\\
% 443.740844726562 -249.777863152128\\
% 444.198608398438 -249.779416334353\\
% 444.656372070313 -249.779673792428\\
% 445.114135742188 -249.778604430631\\
% 445.571899414062 -249.776176163886\\
% 446.029663085938 -249.77235587857\\
% 446.487426757812 -249.767109391465\\
% 446.945190429688 -249.760401406746\\
% 447.402954101562 -249.752195470923\\
% 447.860717773438 -249.74245392558\\
% 448.318481445312 -249.731137857853\\
% 448.776245117188 -249.718207048443\\
% 449.234008789063 -249.703619917081\\
% 449.691772460938 -249.687333465286\\
% 450.149536132812 -249.669303216241\\
% 450.607299804688 -249.649483151644\\
% 451.065063476562 -249.627825645342\\
% 451.522827148438 -249.604281393574\\
% 451.980590820312 -249.5787993416\\
% 452.438354492188 -249.551326606527\\
% 452.896118164062 -249.52180839609\\
% 453.353881835938 -249.490187923129\\
% 453.811645507813 -249.456406315538\\
% 454.269409179688 -249.42040252137\\
% 454.727172851562 -249.382113208831\\
% 455.184936523438 -249.341472660804\\
% 455.642700195312 -249.298412663617\\
% 456.100463867188 -249.252862389643\\
% 456.558227539062 -249.204748273358\\
% 457.015991210938 -249.153993880451\\
% 457.473754882812 -249.100519769516\\
% 457.931518554688 -249.044243345862\\
% 458.389282226563 -248.985078706917\\
% 458.847045898438 -248.922936478678\\
% 459.304809570312 -248.857723642612\\
% 459.762573242188 -248.789343352366\\
% 460.220336914062 -248.717694739618\\
% 460.678100585938 -248.642672708317\\
% 461.135864257813 -248.564167716517\\
% 461.593627929688 -248.482065544998\\
% 462.051391601562 -248.396247051688\\
% 462.509155273438 -248.306587910992\\
% 462.966918945313 -248.212958336876\\
% 463.424682617188 -248.115222788652\\
% 463.882446289062 -248.013239658176\\
% 464.340209960938 -247.906860937179\\
% 464.797973632812 -247.795931863281\\
% 465.255737304688 -247.680290543169\\
% 465.713500976563 -247.55976755128\\
% 466.171264648438 -247.434185502219\\
% 466.629028320312 -247.303358594971\\
% 467.086791992188 -247.167092126863\\
% 467.544555664062 -247.025181975022\\
% 468.002319335938 -246.87741404294\\
% 468.460083007812 -246.723563669547\\
% 468.917846679688 -246.563394998005\\
% 469.375610351562 -246.396660301212\\
% 469.833374023438 -246.223099260789\\
% 470.291137695313 -246.042438196036\\
% 470.748901367188 -245.854389239175\\
% 471.206665039062 -245.658649452782\\
% 471.664428710938 -245.454899885154\\
% 472.122192382812 -245.242804558955\\
% 472.579956054688 -245.022009388203\\
% 473.037719726562 -244.792141018293\\
% 473.526000976562 -244.536503228506\\
% 474.014282226562 -244.269588279239\\
% 474.502563476562 -243.990860459964\\
% 474.990844726562 -243.699751683587\\
% 475.479125976562 -243.395659200824\\
% 475.967407226562 -243.07794314155\\
% 476.455688476562 -242.745923871088\\
% 476.943969726562 -242.398879149106\\
% 477.432250976562 -242.03604107858\\
% 477.920532226562 -241.656592832453\\
% 478.408813476562 -241.259665145902\\
% 478.897094726562 -240.844332563061\\
% 479.385375976562 -240.409609428287\\
% 479.873657226562 -239.954445614251\\
% 480.361938476562 -239.477721982066\\
% 480.850219726562 -238.978245572882\\
% 481.338500976562 -238.454744536111\\
% 481.826782226562 -237.905862807015\\
% 482.315063476562 -237.330154556469\\
% 482.803344726562 -236.726078448685\\
% 483.291625976562 -236.091991759484\\
% 483.779907226562 -235.426144429076\\
% 484.268188476562 -234.726673150276\\
% 484.756469726562 -233.991595627045\\
% 485.244750976562 -233.218805180325\\
% 485.733032226562 -232.40606593018\\
% 486.221313476562 -231.551008846811\\
% 486.709594726562 -230.651129039923\\
% 487.197875976562 -229.703784747974\\
% 487.686157226562 -228.706198597383\\
% 488.174438476562 -227.655461827831\\
% 488.662719726562 -226.548542322906\\
% 489.151000976562 -225.382297443741\\
% 489.639282226562 -224.153492831989\\
% 490.127563476562 -222.85882851884\\
% 490.615844726562 -221.494973834372\\
% 491.104125976562 -220.058612734881\\
% 491.592407226562 -218.546501223883\\
% 492.080688476562 -216.955538493919\\
% 492.568969726562 -215.28285320791\\
% 493.057250976562 -213.525905907525\\
% 493.545532226562 -211.682607812795\\
% 494.033813476562 -209.751455196021\\
% 494.522094726562 -207.73167702637\\
% 495.010375976562 -205.623391682107\\
% 495.498657226562 -203.427766275888\\
% 495.986938476562 -201.147169693214\\
% 496.475219726562 -198.785308083163\\
% 496.963500976562 -196.347329666012\\
% 497.451782226562 -193.839884829152\\
% 497.940063476562 -191.271128083706\\
% 498.428344726562 -188.650650957846\\
% 498.916625976562 -185.989339464372\\
% 499.404907226562 -183.299156167396\\
% 499.893188476562 -180.592854400374\\
% 499.984741210938 -180.084696744997\\
% };


\end{axis}



\end{tikzpicture}%

 \caption{Measured \gls{FRF} \legref{m2t:plantFRF} and estimated parametric model $\hat{P}$ \legref{m2t:plantModel} of the \glsfirst{AVIS}.}
 \TODO{split photo out of plot, this is no paper...}
 \label{fig:avis-frf}
 \label{fig:avis}
\end{figure}

\subsection{Results}
\label{sec:resultsAvis}
To estimate $\Delta(\omega_k)$ and $\hinfnorm{\Delta}$  using the proposed approach with $\lrm{6,2,2,2}$, an estimation dataset consisting of only a fourth of the excited bins is used.
The full dataset that includes bins not used in the estimation is used to validate the interpolation results.

The measured $\Delta$ is shown in \figref{fig:avisMeas} for the whole frequency band and \figref{fig:avisMeasZoom} for a frequency range containing $\omega_{\star}$.
Two aspects of the proposed method are examined.
First, the interpolation results for a continuous frequency range are validated.
Second, the peak amplitudes of $\Delta(\omega)$ are inspected as these are  indicators for $\infnorm{\Delta}$.

\begin{figure}
 \centering
    \setlength{\figurewidth}{0.75\columnwidth}
    \setlength{\figureheight}{0.68\figurewidth}
    % % This file was created by matlab2tikz v0.4.3 (commit 5727fe79e69f5e2b601173753f6a6749e42dcb5d).
% Copyright (c) 2008--2013, Nico Schlömer <nico.schloemer@gmail.com>
% All rights reserved.
% 
% The latest updates can be retrieved from
%   http://www.mathworks.com/matlabcentral/fileexchange/22022-matlab2tikz
% where you can also make suggestions and rate matlab2tikz.
% 
%
\begin{tikzpicture}

% \pgfplotsset{compat=newest}
\pgfplotsset{tick style={black!30},grid style={black!10}}
\pgfplotsset{every axis legend/.append style={font=\footnotesize}}
\pgfplotsset{every axis label/.append style={font=\footnotesize}}
\pgfplotsset{every tick label/.append style={font=\footnotesize}}
\pgfplotsset{every axis title/.append style={font=\footnotesize}}
\pgfplotsset{every axis post/.style={unbounded coords=jump}}
\pgfplotsset{every axis title/.append style={at={(0.5,0.95)}}}

\definecolor{LPMTrunc}{named}{TangoSkyBlue2}
\definecolor{LPMTruncInit}{named}{TangoSkyBlue2}
\definecolor{RFIR}{named}{TangoScarletRed3}
\definecolor{RFIRInit}{named}{TangoScarletRed1}
\definecolor{existing}{named}{TangoChameleon3}
\definecolor{G0Hat}{named}{TangoOrange2}
\definecolor{GVXI}{named}{G0Hat}
\definecolor{reference}{named}{black}

\definecolor{heuristic}{named}{TangoPlum2}
\definecolor{observed}{named}{TangoOrange3}

\definecolor{best}{named}{TangoAluminium6}

\definecolor{G0HatFill}{named}{TangoOrange1}
\definecolor{GVXIFill}{named}{G0HatFill}
\definecolor{LPMTruncFill}{named}{TangoSkyBlue1}
\definecolor{RFIRFill}{named}{TangoScarletRed1}
\definecolor{existingFill}{named}{TangoChameleon1}
\definecolor{bestFill}{named}{TangoAluminium3}

\definecolor{FRFMean}{named}{TangoPlum3}
\definecolor{FRFSingle}{named}{TangoPlum1}
\definecolor{FRFNoise}{named}{TangoChocolate1}

\pgfplotsset{FRFMean/.style={color=FRFMean,mark=*,mark options={solid},only marks,medsmallmarkers}}
\pgfplotsset{FRFSingle/.style={color=FRFSingle,mark=square*,mark options={solid},only marks,smallmarkers}}
\pgfplotsset{FRFNoise/.style={color=FRFNoise,mark=x,mark options={solid},only marks,smallmarkers}}

\pgfset{number format/1000 sep={\,}}

\pgfplotsset{bandwidth/.style={area style,fill=TangoButter1,draw=TangoButter2,fill opacity=0.5}}
\pgfplotsset{goodestimate/.style={color=TangoChameleon2,line join=round}}
\pgfplotsset{badestimate/.style={color=TangoScarletRed2,solid,line join=round}}
\pgfplotsset{exact/.style={color=black,dashed,line width=0.75pt,line join=round}}

\pgfplotsset{LPMTruncmark/.append style={mark=*,mark options={solid}}}
\pgfplotsset{RFIRmark/.append style={mark=square*,mark options={solid}}}
\pgfplotsset{existingmark/.append style={mark=triangle*,mark options={solid}}}
\pgfplotsset{existingInitmark/.append style={mark=triangle,mark options={solid}}}
\pgfplotsset{LPMTruncInitmark/.append style={mark=o,mark options={solid}}}
\pgfplotsset{RFIRInitmark/.append style={mark=square,mark options={solid}}}
\pgfplotsset{bestmark/.append style={mark=diamond*,mark options={solid}}}
\pgfplotsset{G0Hatmark/.append style={mark=asterisk,mark options={solid}}}

\pgfplotsset{GVXI/.style={color=GVXI,line width=1pt}}
\pgfplotsset{G0Hat/.style={color=G0Hat,line width=1.5pt}}
\pgfplotsset{existing/.style={color=existing}}
\pgfplotsset{LPMTruncInit/.style={color=LPMTruncInit,densely dashed}}
\pgfplotsset{LPMTrunc/.style={color=LPMTrunc}}
\pgfplotsset{RFIRInit/.style={color=RFIRInit,densely dashed}}
\pgfplotsset{RFIR/.style={color=RFIR}}
\pgfplotsset{best/.style={color=best}}

\pgfplotsset{smallmarkers/.append style={mark size=0.75pt}}
\pgfplotsset{medsmallmarkers/.append style={mark size=0.5pt}}
\pgfplotsset{tinymarkers/.append style={mark size=0.25pt}}
\pgfplotsset{extremelytinymarkers/.append style={mark size=0.05pt}}


\tikzset{annotation/.style={align=left,draw=black!0.2,font=\scriptsize,fill=white,fill opacity=0.8}}
% http://tex.stackexchange.com/questions/83487/pgfplotstable-converting-zeros-to-in-a-knitr-inline-table
\pgfplotstableset{%
	zerofill=true,
	after row=[3pt],
                  every head row/.style={before row=\toprule, after row={\\\midrule}},
                  every last row/.style={after row=\bottomrule}
	assign column name/.code={%
        \pgfkeyssetvalue{/pgfplots/table/column name}{\multicolumn{1}{c}{\multirow{2}{*}{#1}} }%
    },
	columns/method/.style={string type,column name={\shortstack{\textsc{Method}\\\phantom{0}}}},
	columns/P000/.style={column type=r,column name={\shortstack[r]{\textsc{Min.}\\{$0\%$}}}},
	columns/P025/.style={column type=r,column name={\shortstack[r]{\vphantom{?}\\\textsc{$25\%$}}}},
	columns/P050/.style={column type=r,column name={\shortstack[r]{\textsc{Median}\\{$50\%$}}}},
	columns/P075/.style={column type=r,column name={\shortstack[r]{\vphantom{?}\\\textsc{$75\%$ }}}},
	columns/P100/.style={column type=r,column name={\shortstack[r]{\textsc{Max.}\\{$100\%$}}}},
	columns/contribution/.style={column name={\shortstack[r]{\textsc{Contribution}\\ \vphantom{0}}},
                    	postproc cell content/.append code={\pgfkeysalso{@cell content/.add={}{\%}}},
	}
}


\begin{axis}[%
width=\figurewidth,
height=\figureheight,
scale only axis,
xmode=log,
xmin=9.9,
xmax=500,
%xminorticks=true,
xlabel={Frequency $\omega$ \axisunit{Hz}},
xmajorgrids,
xminorgrids,
ymin=-75,
ymax=3,
ylabel={Amplitude $\abs{\Delta}$ \axisunit{dB}},
ymajorgrids,
% axis x line*=bottom,
% axis y line*=left,
legend style={draw=none,fill=white,fill opacity=0.75,legend cell align=left,at={(0.02,0.2)},anchor=south west}
]

\addplot [validation]
table[row sep=crcr]{
1.02233886719011 -6.42926358808455\\
1.05285644531519 -6.72341519756003\\
1.08337402344027 -7.15934240938884\\
1.11389160156534 -7.66670118735874\\
1.14440917969042 -8.0548311217139\\
1.1749267578155 -8.14819547220475\\
1.20544433594058 -7.97153450710641\\
1.23596191406566 -7.74742769617103\\
1.26647949219073 -7.9212661836952\\
1.29699707031581 -9.01350611310414\\
1.32751464844089 -11.0554976377829\\
1.35803222656597 -12.568233171941\\
1.38854980469105 -12.3324202718201\\
1.41906738281612 -10.6936487501925\\
1.4495849609412 -10.6385031375762\\
1.48010253906628 -10.5221282406386\\
1.51062011719136 -9.53250891274234\\
1.54113769531644 -9.37600611185479\\
1.57165527344151 -8.57985542709378\\
1.60217285156659 -8.53995492572739\\
1.63269042969167 -7.19384219914582\\
1.66320800781675 -5.84690056019861\\
1.69372558594182 -4.87822816135611\\
1.7242431640669 -4.61935018346071\\
1.75476074219198 -4.05161643885458\\
1.78527832031706 -3.21937044096563\\
1.81579589844214 -2.50817553050695\\
1.84631347656721 -2.47161683816614\\
1.87683105469229 -2.66747811421612\\
1.90734863281737 -2.33035470471987\\
1.93786621094245 -2.73718495113536\\
1.96838378906753 -4.30884318184889\\
1.9989013671926 -3.77050815026837\\
2.02941894531768 -3.24081086682787\\
2.05993652344276 -3.06865013039192\\
2.09045410156784 -2.99437979602965\\
2.12097167969292 0.298459070975298\\
2.15148925781799 -0.0954741980612539\\
2.18200683594307 1.22031674605222\\
2.21252441406815 1.40620970946736\\
2.24304199219323 1.19553423287675\\
2.27355957031831 1.27263553762162\\
2.30407714844338 0.652102017069353\\
2.33459472656846 -0.163855821675099\\
2.36511230469354 0.114403349450413\\
2.39562988281862 0.233734656095976\\
2.4261474609437 0.424639493179939\\
2.45666503906877 -0.300995474139938\\
2.48718261719385 -0.694384127027263\\
2.51770019531893 -5.33713649485577\\
2.54821777344401 -7.73584269450356\\
2.57873535156908 -8.19843667260955\\
2.60925292969416 -8.98451210799976\\
2.63977050781924 -7.73088002785181\\
2.67028808594432 -6.99648071587916\\
2.7008056640694 -3.79364848464706\\
2.73132324219447 -3.37311999464833\\
2.76184082031955 -6.47136956784664\\
2.79235839844463 -6.13473223004206\\
2.82287597656971 -10.3229238744706\\
2.85339355469479 -13.96892891583\\
2.88391113281986 -10.5394996769134\\
2.91442871094494 -6.95929367766729\\
2.94494628907002 -11.7929487165873\\
2.9754638671951 -9.48307890771878\\
3.00598144532018 -12.3682763940339\\
3.03649902344525 -8.0211674114741\\
3.06701660157033 -2.21065088119462\\
3.09753417969541 -6.99077380202976\\
3.12805175782049 -7.45248629676615\\
3.15856933594557 -9.67110439147456\\
3.18908691407064 -6.1031239168571\\
3.21960449219572 -6.49032328101612\\
3.2501220703208 -3.61279698603823\\
3.28063964844588 -4.00909352662336\\
3.31115722657096 0.142104554577713\\
3.34167480469603 -0.745569915846659\\
3.37219238282111 -4.65811315178058\\
3.40270996094619 -6.56686670462682\\
3.43322753907127 -7.03524006011105\\
3.46374511719634 -5.33845646725126\\
3.49426269532142 -5.99204414670072\\
3.5247802734465 -7.43931499596459\\
3.55529785157158 -32.2666292036398\\
3.58581542969666 -16.1737004304792\\
3.61633300782173 -13.2985964114262\\
3.64685058594681 -7.0637620855897\\
3.67736816407189 -6.20346332819827\\
3.70788574219697 -8.36278185783704\\
3.73840332032205 -15.9759144973644\\
3.76892089844712 -13.1617787200225\\
3.7994384765722 -20.6103106406042\\
3.82995605469728 -13.5710392796863\\
3.86047363282236 -14.0642089457609\\
3.89099121094744 -11.3206260375949\\
3.92150878907251 -15.641130351374\\
3.95202636719759 -12.4219629068339\\
3.98254394532267 -10.1436018378376\\
4.01306152344775 -12.2133507387563\\
4.04357910157283 -7.43709019061311\\
4.0740966796979 -7.10770427077688\\
4.10461425782298 -9.00184788110028\\
4.13513183594806 -10.6480869552552\\
4.16564941407314 -18.1993765064251\\
4.19616699219821 -13.4055964614564\\
4.22668457032329 -19.1155731374019\\
4.25720214844837 -20.1426491583263\\
4.28771972657345 -15.5617263295639\\
4.31823730469853 -5.49500460034886\\
4.3487548828236 -11.1030180425527\\
4.37927246094868 -13.3233569635577\\
4.40979003907376 -12.6986929502592\\
4.44030761719884 -12.3323603619061\\
4.47082519532392 -9.88355269618017\\
4.50134277344899 -6.79098780173865\\
4.53186035157407 -3.79313191620548\\
4.56237792969915 -5.3189542548954\\
4.59289550782423 -8.24190758303121\\
4.62341308594931 -6.39565325780956\\
4.65393066407438 -4.81588478659495\\
4.68444824219946 -3.56506858397756\\
4.71496582032454 -1.28065356653872\\
4.74548339844962 -5.11172226852113\\
4.7760009765747 -6.16950152641806\\
4.80651855469977 -8.61880538498718\\
4.83703613282485 -9.42482640792332\\
4.86755371094993 -8.77436057563727\\
4.89807128907501 -7.67836811559732\\
4.92858886720009 -8.16678318156022\\
4.95910644532516 -8.03145869707123\\
4.98962402345024 -6.69429710242002\\
5.02014160157532 -4.77620782336845\\
5.0506591797004 -6.22765442203968\\
5.08117675782547 -9.21580255216588\\
5.11169433595055 -9.10561751670906\\
5.14221191407563 -13.8553762647725\\
5.17272949220071 -10.1983756751981\\
5.20324707032579 -6.24586565459083\\
5.23376464845086 -5.58535167738501\\
5.26428222657594 -5.39918695723958\\
5.29479980470102 -5.27291803067442\\
5.3253173828261 -5.96495260832836\\
5.35583496095118 -3.64048355306585\\
5.38635253907625 -5.01358089491515\\
5.41687011720133 -5.96329449069088\\
5.44738769532641 -9.72378139811065\\
5.47790527345149 -9.84335583920591\\
5.50842285157657 -12.8945017080929\\
5.53894042970164 -11.6287728532426\\
5.56945800782672 -11.1480879284292\\
5.5999755859518 -13.6216891750016\\
5.63049316407688 -21.1858331359525\\
5.66101074220196 -10.9082362651768\\
5.69152832032703 -8.39717894807023\\
5.72204589845211 -6.66837203620895\\
5.75256347657719 -7.13736690292296\\
5.78308105470227 -12.4137880388776\\
5.81359863282735 -18.27005913358\\
5.84411621095242 -13.6342689706835\\
5.8746337890775 -11.9202009492153\\
5.90515136720258 -10.1894282308528\\
5.93566894532766 -8.24759116641542\\
5.96618652345273 -7.62262419253182\\
5.99670410157781 -7.13473278292503\\
6.02722167970289 -5.45411010815906\\
6.05773925782797 -5.58492124865188\\
6.08825683595305 -6.33666503693087\\
6.11877441407812 -6.10317228737625\\
6.1492919922032 -5.56460937035513\\
6.17980957032828 -9.31518041984441\\
6.21032714845336 -8.42768362837614\\
6.24084472657844 -10.3220787825396\\
6.27136230470351 -9.77706016637552\\
6.30187988282859 -15.3341082240842\\
6.33239746095367 -12.7912960847023\\
6.36291503907875 -13.9550835386921\\
6.39343261720383 -8.49779121849269\\
6.4239501953289 -5.30611573644597\\
6.45446777345398 -3.61249387930411\\
6.48498535157906 -3.55648254759717\\
6.51550292970414 -3.25731393430192\\
6.54602050782922 -3.10431720164803\\
6.57653808595429 -3.65530570182517\\
6.60705566407937 -3.11669517362418\\
6.63757324220445 -7.16158520949614\\
6.66809082032953 -7.57396957003215\\
6.6986083984546 -7.48313585847586\\
6.72912597657968 -8.65800035107441\\
6.75964355470476 -8.09524500034161\\
6.79016113282984 -8.27000779222141\\
6.82067871095492 -9.44814987896814\\
6.85119628907999 -10.6218973291529\\
6.88171386720507 -10.6687609357829\\
6.91223144533015 -9.94810071334297\\
6.94274902345523 -9.05454064563497\\
6.97326660158031 -7.50335077701914\\
7.00378417970538 -7.43598295940768\\
7.03430175783046 -6.18389017008712\\
7.06481933595554 -5.61220648754784\\
7.09533691408062 -8.21489523876534\\
7.1258544922057 -7.05473248856299\\
7.15637207033077 -6.20892151072167\\
7.18688964845585 -7.56713505341196\\
7.21740722658093 -5.86830596658172\\
7.24792480470601 -5.44385509621003\\
7.27844238283109 -3.75913436295542\\
7.30895996095616 -2.87111247812004\\
7.33947753908124 -4.02465736406538\\
7.36999511720632 -5.64433091991992\\
7.4005126953314 -6.6515868370409\\
7.43103027345648 -7.07203877238152\\
7.46154785158155 -5.60790393906825\\
7.49206542970663 -4.36895511944192\\
7.52258300783171 -4.65709259177856\\
7.55310058595679 -5.31956089035776\\
7.58361816408186 -5.38359200966454\\
7.61413574220694 -5.91076809789092\\
7.64465332033202 -6.12150350113689\\
7.6751708984571 -5.91022522180742\\
7.70568847658218 -6.4437446916283\\
7.73620605470725 -6.44274490048798\\
7.76672363283233 -5.87455418344257\\
7.79724121095741 -5.6396785330279\\
7.82775878908249 -5.420251462365\\
7.85827636720757 -4.82375055043593\\
7.88879394533264 -5.29163078682774\\
7.91931152345772 -6.0268155417105\\
7.9498291015828 -6.00553541831954\\
7.98034667970788 -5.76138946611394\\
8.01086425783296 -5.74814433741375\\
8.04138183595803 -5.66727886292517\\
8.07189941408311 -5.5920769319211\\
8.10241699220819 -5.70733775118242\\
8.13293457033327 -5.4525143291857\\
8.16345214845835 -5.17366924703924\\
8.19396972658342 -5.35433514106751\\
8.2244873047085 -5.21901763025403\\
8.25500488283358 -5.33182334549747\\
8.28552246095866 -5.44460689631575\\
8.31604003908373 -4.90753793731244\\
8.34655761720881 -5.35754615002151\\
8.37707519533389 -5.12787400609096\\
8.40759277345897 -5.93242912952877\\
8.43811035158405 -6.35535965534615\\
8.46862792970913 -6.64230666931218\\
8.4991455078342 -5.89649859597347\\
8.52966308595928 -6.07914059293108\\
8.56018066408436 -5.76391923694149\\
8.59069824220944 -6.2656143558313\\
8.62121582033451 -6.36731529214489\\
8.65173339845959 -5.62261101272981\\
8.68225097658467 -4.82472937171076\\
8.71276855470975 -4.67555376139723\\
8.74328613283483 -4.39690472028684\\
8.7738037109599 -4.81222246813155\\
8.80432128908498 -4.42039342017409\\
8.83483886721006 -4.49661919081194\\
8.86535644533514 -5.02927566728749\\
8.89587402346022 -6.18753983392133\\
8.92639160158529 -5.65032874375879\\
8.95690917971037 -5.50409847015374\\
8.98742675783545 -5.34664961071184\\
9.01794433596053 -5.17782900411459\\
9.04846191408561 -5.26164889830039\\
9.07897949221068 -5.59523774871258\\
9.10949707033576 -5.59569776600978\\
9.14001464846084 -5.65900197625263\\
9.17053222658592 -5.58842570720105\\
9.20104980471099 -5.3331316381142\\
9.23156738283607 -5.39898708825393\\
9.26208496096115 -5.36798676941129\\
9.29260253908623 -5.1853711851532\\
9.32312011721131 -4.91938456011025\\
9.35363769533638 -4.86299083709054\\
9.38415527346146 -4.86456705384916\\
9.41467285158654 -4.77756919005247\\
9.44519042971162 -4.75829698900259\\
9.4757080078367 -4.90854207541895\\
9.50622558596177 -4.93957157209877\\
9.53674316408685 -4.88793084948418\\
9.56726074221193 -4.36347016066935\\
9.59777832033701 -4.38383185428626\\
9.62829589846209 -4.43637946997711\\
9.65881347658716 -4.60405757142399\\
9.68933105471224 -4.60054658132208\\
9.71984863283732 -4.57507957654246\\
9.7503662109624 -4.5569578916685\\
9.78088378908748 -4.44379844589048\\
9.81140136721255 -4.71325597382685\\
9.84191894533763 -4.76845014950027\\
9.87243652346271 -4.92313399001091\\
9.90295410158779 -4.69385784438015\\
9.93347167971286 -4.74070157191397\\
9.96398925783794 -4.83088571919836\\
9.99450683596302 -4.90393397792138\\
10.0250244140881 -4.79873208293446\\
10.0555419922132 -4.82539459683045\\
10.0860595703383 -4.93566666220948\\
10.1165771484633 -5.03038495123008\\
10.1470947265884 -5.12675287622329\\
10.1776123047135 -4.94979048166351\\
10.2081298828386 -5.07205529987516\\
10.2386474609636 -4.86348582099424\\
10.2691650390887 -5.02969324057881\\
10.2996826172138 -5.02691059296228\\
10.3302001953389 -4.78910582612144\\
10.360717773464 -4.48232043606072\\
10.391235351589 -4.54597640859191\\
10.4217529297141 -4.37846340503319\\
10.4522705078392 -4.52119335867536\\
10.4827880859643 -4.83699687587972\\
10.5133056640893 -5.14040166787396\\
10.5438232422144 -5.52945627970564\\
10.5743408203395 -5.41212059801535\\
10.6048583984646 -5.21547574904872\\
10.6353759765897 -4.93487352228516\\
10.6658935547147 -4.93144176652726\\
10.6964111328398 -4.82905855796844\\
10.7269287109649 -4.82342467086983\\
10.75744628909 -5.06504865525596\\
10.787963867215 -5.58257606872991\\
10.8184814453401 -5.70234981196626\\
10.8489990234652 -5.76841475568273\\
10.8795166015903 -5.31442700708538\\
10.9100341797154 -4.92799912359862\\
10.9405517578404 -5.10770290173406\\
10.9710693359655 -4.85034640134586\\
11.0015869140906 -4.65252006925323\\
11.0321044922157 -5.39886369854605\\
11.0626220703407 -5.14233292344636\\
11.0931396484658 -5.2116048726183\\
11.1236572265909 -5.11540034320331\\
11.154174804716 -5.11196043440356\\
11.1846923828411 -5.10784307197474\\
11.2152099609661 -5.16124672985751\\
11.2457275390912 -4.79593171984067\\
11.2762451172163 -4.75444079125157\\
11.3067626953414 -4.64110564399147\\
11.3372802734664 -4.53534376961716\\
11.3677978515915 -4.33401050726962\\
11.3983154297166 -3.95820936189307\\
11.4288330078417 -3.88520601079637\\
11.4593505859668 -4.11706048592777\\
11.4898681640918 -4.56930748024746\\
11.5203857422169 -5.28191774539903\\
11.550903320342 -5.13409645937219\\
11.5814208984671 -5.32641391485993\\
11.6119384765922 -5.19334068976377\\
11.6424560547172 -5.01608993637791\\
11.6729736328423 -5.26993398802432\\
11.7034912109674 -5.23498177983936\\
11.7340087890925 -5.28919372994176\\
11.7645263672175 -4.89607690861601\\
11.7950439453426 -4.22266600581128\\
11.8255615234677 -3.95676925927557\\
11.8560791015928 -4.0436954000196\\
11.8865966797179 -4.15574511228471\\
11.9171142578429 -4.27741170839869\\
11.947631835968 -4.24738439965387\\
11.9781494140931 -4.24157793099619\\
12.0086669922182 -4.66776935640848\\
12.0391845703432 -4.88070548164637\\
12.0697021484683 -5.18735216754419\\
12.1002197265934 -5.47792505787027\\
12.1307373047185 -5.77616044791466\\
12.1612548828436 -5.73278023807552\\
12.1917724609686 -5.57225374995727\\
12.2222900390937 -5.21130548036143\\
12.2528076172188 -5.01897926145057\\
12.2833251953439 -5.11317656934222\\
12.3138427734689 -5.04326729817024\\
12.344360351594 -5.1755044414266\\
12.3748779297191 -5.20131058854065\\
12.4053955078442 -5.19507121881674\\
12.4359130859693 -5.20265000392618\\
12.4664306640943 -5.1111953962029\\
12.4969482422194 -5.1297168907692\\
12.5274658203445 -5.03486139397756\\
12.5579833984696 -5.05481883560441\\
12.5885009765946 -4.85808357600433\\
12.6190185547197 -4.86085274572918\\
12.6495361328448 -5.01943146367967\\
12.6800537109699 -5.22485072555799\\
12.710571289095 -5.09291056145446\\
12.74108886722 -5.13979155068603\\
12.7716064453451 -5.075024200342\\
12.8021240234702 -4.97549201460254\\
12.8326416015953 -4.9779593943789\\
12.8631591797203 -4.98066333360532\\
12.8936767578454 -5.10238810113702\\
12.9241943359705 -4.92486514545323\\
12.9547119140956 -4.89273228509063\\
12.9852294922207 -4.773132605966\\
13.0157470703457 -4.83043857763528\\
13.0462646484708 -4.96915587838089\\
13.0767822265959 -4.94158764456489\\
13.107299804721 -5.08829780404619\\
13.137817382846 -5.26112558810564\\
13.1683349609711 -5.18476620722157\\
13.1988525390962 -5.15849588718208\\
13.2293701172213 -5.21093580959632\\
13.2598876953464 -5.15154723698248\\
13.2904052734714 -5.41378744831655\\
13.3209228515965 -5.47856228011017\\
13.3514404297216 -5.56033021224727\\
13.3819580078467 -5.3732589663914\\
13.4124755859717 -5.23468457810924\\
13.4429931640968 -5.1729477484605\\
13.4735107422219 -5.17851240915871\\
13.504028320347 -5.09647955843565\\
13.5345458984721 -4.9776144880048\\
13.5650634765971 -5.14971950186828\\
13.5955810547222 -5.16315742722281\\
13.6260986328473 -5.25830632838029\\
13.6566162109724 -5.37437663374197\\
13.6871337890974 -5.43431766600804\\
13.7176513672225 -5.32220347345498\\
13.7481689453476 -5.18977243895023\\
13.7786865234727 -5.17823117099391\\
13.8092041015978 -5.14578413488073\\
13.8397216797228 -5.21673034361112\\
13.8702392578479 -5.2366643822283\\
13.900756835973 -5.31894894195511\\
13.9312744140981 -5.34515575997142\\
13.9617919922232 -5.40296889363992\\
13.9923095703482 -5.4382664408455\\
14.0228271484733 -5.48908075825062\\
14.0533447265984 -5.41892964465825\\
14.0838623047235 -5.4675407170219\\
14.1143798828485 -5.36421924957864\\
14.1448974609736 -5.39240988223958\\
14.1754150390987 -5.40896350195442\\
14.2059326172238 -5.4190566802975\\
14.2364501953489 -5.44553817563934\\
14.2669677734739 -5.49350125512058\\
14.297485351599 -5.58467115818848\\
14.3280029297241 -5.6239762205862\\
14.3585205078492 -5.45130794261581\\
14.3890380859742 -5.35905790409328\\
14.4195556640993 -5.29090120744911\\
14.4500732422244 -5.24626222668337\\
14.4805908203495 -5.34860950081423\\
14.5111083984746 -5.37321212685868\\
14.5416259765996 -5.36550741897298\\
14.5721435547247 -5.3884389992715\\
14.6026611328498 -5.38419978481215\\
14.6331787109749 -5.31549548266867\\
14.6636962890999 -5.42255514245841\\
14.694213867225 -5.44202558342124\\
14.7247314453501 -5.4099141107265\\
14.7552490234752 -5.41836976004657\\
14.7857666016003 -5.41632841197259\\
14.8162841797253 -5.46378694949766\\
14.8468017578504 -5.55111098272329\\
14.8773193359755 -5.56416620082382\\
14.9078369141006 -5.50639592404144\\
14.9383544922256 -5.47303661191205\\
14.9688720703507 -5.45431440033929\\
14.9993896484758 -5.41207515180025\\
15.0299072266009 -5.41208620938221\\
15.060424804726 -5.4213452900367\\
15.090942382851 -5.40530510976373\\
15.1214599609761 -5.40896461569611\\
15.1519775391012 -5.42336045879762\\
15.1824951172263 -5.45084431575282\\
15.2130126953513 -5.53836400760252\\
15.2435302734764 -5.5804276619163\\
15.2740478516015 -5.63915370716643\\
15.3045654297266 -5.66325764172717\\
15.3350830078517 -5.60104690274761\\
15.3656005859767 -5.54787429341707\\
15.3961181641018 -5.53601323116681\\
15.4266357422269 -5.48681523404827\\
15.457153320352 -5.49707194785969\\
15.487670898477 -5.56573417025822\\
15.5181884766021 -5.56872185097774\\
15.5487060547272 -5.57914352109879\\
15.5792236328523 -5.55357761774604\\
15.6097412109774 -5.54337008182284\\
15.6402587891024 -5.52312760221105\\
15.6707763672275 -5.55605886055497\\
15.7012939453526 -5.56353042023215\\
15.7318115234777 -5.62878186708565\\
15.7623291016027 -5.65253063522505\\
15.7928466797278 -5.69836900832706\\
15.8233642578529 -5.72574989913164\\
15.853881835978 -5.67873017176333\\
15.8843994141031 -5.69904552497155\\
15.9149169922281 -5.6455296895669\\
15.9454345703532 -5.61946426300653\\
15.9759521484783 -5.63435931173871\\
16.0064697266034 -5.62897532880135\\
16.0369873047284 -5.60418796352377\\
16.0675048828535 -5.60562013307577\\
16.0980224609786 -5.61728914469001\\
16.1285400391037 -5.63175793614647\\
16.1590576172288 -5.63957507380758\\
16.1895751953538 -5.6505120310145\\
16.2200927734789 -5.68022559050564\\
16.250610351604 -5.71697081383223\\
16.2811279297291 -5.7120318198975\\
16.3116455078542 -5.73713840174901\\
16.3421630859792 -5.75082620851828\\
16.3726806641043 -5.76712006347412\\
16.4031982422294 -5.72572120935234\\
16.4337158203545 -5.73967277331559\\
16.4642333984795 -5.80168518792772\\
16.4947509766046 -5.84738233003316\\
16.5252685547297 -5.90966553687724\\
16.5557861328548 -5.865151271451\\
16.5863037109799 -5.88236629103727\\
16.6168212891049 -5.84111876004135\\
16.64733886723 -5.83618208079986\\
16.6778564453551 -5.84479542862925\\
16.7083740234802 -5.77222424312782\\
16.7388916016052 -5.77645884295805\\
16.7694091797303 -5.79715619866272\\
16.7999267578554 -5.83371833802499\\
16.8304443359805 -5.83564785869163\\
16.8609619141056 -5.87593663461729\\
16.8914794922306 -5.8702671424075\\
16.9219970703557 -5.89457691531663\\
16.9525146484808 -5.88685450879143\\
16.9830322266059 -5.90458814617057\\
17.0135498047309 -5.92197284285788\\
17.044067382856 -5.92233463904529\\
17.0745849609811 -5.89574748376236\\
17.1051025391062 -5.90677775676011\\
17.1356201172313 -5.88931839069983\\
17.1661376953563 -5.87801263879749\\
17.1966552734814 -5.88155770369434\\
17.2271728516065 -5.88769378768347\\
17.2576904297316 -5.92167262608791\\
17.2882080078566 -5.92181579886926\\
17.3187255859817 -5.93174383277761\\
17.3492431641068 -5.94420894343347\\
17.3797607422319 -5.90763071450186\\
17.410278320357 -5.90412989069983\\
17.440795898482 -5.94538307702794\\
17.4713134766071 -5.99618382646463\\
17.5018310547322 -6.04867716213965\\
17.5323486328573 -6.05743865396244\\
17.5628662109823 -6.00945888626967\\
17.5933837891074 -6.00053281433082\\
17.6239013672325 -6.00487097846519\\
17.6544189453576 -5.97872241443486\\
17.6849365234827 -5.98191534977474\\
17.7154541016077 -5.99192244817311\\
17.7459716797328 -6.03126172617749\\
17.7764892578579 -6.05818846695411\\
17.807006835983 -6.07968940570618\\
17.837524414108 -6.08765902220983\\
17.8680419922331 -6.1298909959354\\
17.8985595703582 -6.08894549449604\\
17.9290771484833 -6.05945450645493\\
17.9595947266084 -6.07172369883989\\
17.9901123047334 -6.08729942740121\\
18.0206298828585 -6.07900594406044\\
18.0511474609836 -6.07965391135815\\
18.0816650391087 -6.07733775327381\\
18.1121826172337 -6.05587611104522\\
18.1427001953588 -6.09379611391228\\
18.1732177734839 -6.14590972230906\\
18.203735351609 -6.18032015275298\\
18.2342529297341 -6.19743053001258\\
18.2647705078591 -6.22473026339253\\
18.2952880859842 -6.24040041955482\\
18.3258056641093 -6.24641737725869\\
18.3563232422344 -6.23010575960393\\
18.3868408203594 -6.2244456393102\\
18.4173583984845 -6.21528639787095\\
18.4478759766096 -6.26157010022376\\
18.4783935547347 -6.23170314292861\\
18.5089111328598 -6.2387903267346\\
18.5394287109848 -6.22794406330496\\
18.5699462891099 -6.21555179847337\\
18.600463867235 -6.23157185442159\\
18.6309814453601 -6.232649378188\\
18.6614990234852 -6.25552709554785\\
18.6920166016102 -6.25148451817756\\
18.7225341797353 -6.27423434987662\\
18.7530517578604 -6.27934671711\\
18.7835693359855 -6.30128822961683\\
18.8140869141105 -6.34413668929943\\
18.8446044922356 -6.34089293668109\\
18.8751220703607 -6.3405189248777\\
18.9056396484858 -6.35208707581813\\
18.9361572266109 -6.35873956772315\\
18.9666748047359 -6.37676517900218\\
18.997192382861 -6.36619059781799\\
19.0277099609861 -6.36807865783243\\
19.0582275391112 -6.37530649806405\\
19.0887451172362 -6.36588546883422\\
19.1192626953613 -6.37522784971253\\
19.1497802734864 -6.36930634478904\\
19.1802978516115 -6.36347834946497\\
19.2108154297366 -6.35673344218822\\
19.2413330078616 -6.36316294507424\\
19.2718505859867 -6.37631683420449\\
19.3023681641118 -6.39811237923652\\
19.3328857422369 -6.41283977524915\\
19.3634033203619 -6.43410881358\\
19.393920898487 -6.44160636782203\\
19.4244384766121 -6.44795503505804\\
19.4549560547372 -6.45115678586313\\
19.4854736328623 -6.4661139821975\\
19.5159912109873 -6.46166817296773\\
19.5465087891124 -6.48010517610351\\
19.5770263672375 -6.47640526444513\\
19.6075439453626 -6.4920655706382\\
19.6380615234876 -6.49742027846406\\
19.6685791016127 -6.4994150799958\\
19.6990966797378 -6.49839111411495\\
19.7296142578629 -6.51259351141852\\
19.760131835988 -6.52563634470226\\
19.790649414113 -6.53413963973611\\
19.8211669922381 -6.54073800366791\\
19.8516845703632 -6.54985539087107\\
19.8822021484883 -6.55204167496743\\
19.9127197266133 -6.54517303491969\\
19.9432373047384 -6.55241321451581\\
19.9737548828635 -6.57639513654635\\
20.0042724609886 -6.5831803445605\\
20.0347900391137 -6.57308000170696\\
20.0653076172387 -6.5730657884821\\
20.0958251953638 -6.58189506491021\\
20.1263427734889 -6.58504868742284\\
20.156860351614 -6.58653656475025\\
20.187377929739 -6.59845897901386\\
20.2178955078641 -6.60187416893945\\
20.2484130859892 -6.61194129659486\\
20.2789306641143 -6.61380347180801\\
20.3094482422394 -6.61320633872805\\
20.3399658203644 -6.63161581883156\\
20.3704833984895 -6.64578982114028\\
20.4010009766146 -6.65333167063858\\
20.4315185547397 -6.69319315544931\\
20.4620361328647 -6.69657556096871\\
20.4925537109898 -6.68888470156514\\
20.5230712891149 -6.68159660694675\\
20.55358886724 -6.68468985866025\\
20.5841064453651 -6.68653020194012\\
20.6146240234901 -6.68885499935487\\
20.6451416016152 -6.68903135768966\\
20.6756591797403 -6.68802445239163\\
20.7061767578654 -6.70340304042162\\
20.7366943359904 -6.71626992555952\\
20.7672119141155 -6.71929774836866\\
20.7977294922406 -6.73147679717408\\
20.8282470703657 -6.7542158100897\\
20.8587646484908 -6.76701327133213\\
20.8892822266158 -6.77213674720326\\
20.9197998047409 -6.77810221896186\\
20.950317382866 -6.7814194086319\\
20.9808349609911 -6.78533012573467\\
21.0113525391162 -6.78246516049649\\
21.0418701172412 -6.77477437588215\\
21.0723876953663 -6.77055338554192\\
21.1029052734914 -6.77257688364466\\
21.1334228516165 -6.78253038139934\\
21.1639404297415 -6.79374451560284\\
21.1944580078666 -6.81369296955864\\
21.2249755859917 -6.83778627385288\\
21.2554931641168 -6.84467333279912\\
21.2860107422419 -6.84587373881146\\
21.3165283203669 -6.84914884323274\\
21.347045898492 -6.84421742879607\\
21.3775634766171 -6.85224129079194\\
21.4080810547422 -6.85676431985837\\
21.4385986328672 -6.86177410197166\\
21.4691162109923 -6.86145006351518\\
21.4996337891174 -6.84279690852554\\
21.5301513672425 -6.87076279188864\\
21.5606689453676 -6.87271931142385\\
21.5911865234926 -6.88812207206286\\
21.6217041016177 -6.90316178726403\\
21.6522216797428 -6.90440205205135\\
21.6827392578679 -6.91577792897351\\
21.7132568359929 -6.91823752818186\\
21.743774414118 -6.92547430354898\\
21.7742919922431 -6.93630645295349\\
21.8048095703682 -6.95491909210944\\
21.8353271484933 -6.96921637048888\\
21.8658447266183 -6.9817879549089\\
21.8963623047434 -6.98276244608223\\
21.9268798828685 -6.97531218107963\\
21.9573974609936 -6.97508222223632\\
21.9879150391186 -6.9738384416986\\
22.0184326172437 -6.97990059256239\\
22.0489501953688 -6.99045067981467\\
22.0794677734939 -7.00818609259431\\
22.109985351619 -7.01070069110858\\
22.140502929744 -7.02585784709652\\
22.1710205078691 -7.02847634903549\\
22.2015380859942 -7.02654524935599\\
22.2320556641193 -7.02824055410258\\
22.2625732422443 -7.0133462395022\\
22.2930908203694 -7.02117699598256\\
22.3236083984945 -7.03797964434409\\
22.3541259766196 -7.04105570311776\\
22.3846435547447 -7.0402819263208\\
22.4151611328697 -7.04117982726638\\
22.4456787109948 -7.05968644895813\\
22.4761962891199 -7.0493838464343\\
22.506713867245 -7.05011618562105\\
22.5372314453701 -7.07200267207446\\
22.5677490234951 -7.08112709568741\\
22.5982666016202 -7.08312134409141\\
22.6287841797453 -7.09347964901349\\
22.6593017578704 -7.10101581250456\\
22.6898193359954 -7.10017436278321\\
22.7203369141205 -7.10222919004832\\
22.7508544922456 -7.11729494712398\\
22.7813720703707 -7.11650395469616\\
22.8118896484958 -7.12322478679118\\
22.8424072266208 -7.13195006763567\\
22.8729248047459 -7.13471341407819\\
22.903442382871 -7.14075390767186\\
22.9339599609961 -7.15419134864067\\
22.9644775391211 -7.15968678741831\\
22.9949951172462 -7.16931944039948\\
23.0255126953713 -7.17965873294088\\
23.0560302734964 -7.19355030901454\\
23.0865478516215 -7.20671831614675\\
23.1170654297465 -7.20176209286342\\
23.1475830078716 -7.20865322180862\\
23.1781005859967 -7.20743137240652\\
23.2086181641218 -7.21258997328113\\
23.2391357422468 -7.22881950835591\\
23.2696533203719 -7.24679263859321\\
23.300170898497 -7.2586528838608\\
23.3306884766221 -7.26613128479886\\
23.3612060547472 -7.26181229888698\\
23.3917236328722 -7.24875112940475\\
23.4222412109973 -7.24363120942343\\
23.4527587891224 -7.25279986026101\\
23.4832763672475 -7.24789572614702\\
23.5137939453725 -7.25808977264273\\
23.5443115234976 -7.27210512456548\\
23.5748291016227 -7.27946057051651\\
23.6053466797478 -7.28000955581155\\
23.6358642578729 -7.28853919118893\\
23.6663818359979 -7.28008291710586\\
23.696899414123 -7.29135696194993\\
23.7274169922481 -7.2873546210925\\
23.7579345703732 -7.29618791678848\\
23.7884521484982 -7.30323054291733\\
23.8189697266233 -7.31793565255521\\
23.8494873047484 -7.34276065425297\\
23.8800048828735 -7.35846984458544\\
23.9105224609986 -7.35494623320068\\
23.9410400391236 -7.35873276876328\\
23.9715576172487 -7.3548952743003\\
24.0020751953738 -7.35815218180625\\
24.0325927734989 -7.35309074992762\\
24.0631103516239 -7.35843542365035\\
24.093627929749 -7.36595610101108\\
24.1241455078741 -7.37122227183579\\
24.1546630859992 -7.37414228840402\\
24.1851806641243 -7.37907736715977\\
24.2156982422493 -7.37719493697347\\
24.2462158203744 -7.38011350357215\\
24.2767333984995 -7.39878075559056\\
24.3072509766246 -7.41105618612806\\
24.3377685547496 -7.42058805338843\\
24.3682861328747 -7.42182237530119\\
24.3988037109998 -7.44102406479749\\
24.4293212891249 -7.44141555818214\\
24.45983886725 -7.45247557933726\\
24.490356445375 -7.46267797484848\\
24.5208740235001 -7.47588108698886\\
24.5513916016252 -7.4884816553286\\
24.5819091797503 -7.50179717280531\\
24.6124267578753 -7.50817340508945\\
24.6429443360004 -7.50854097824845\\
24.6734619141255 -7.50248060857274\\
24.7039794922506 -7.50243139380058\\
24.7344970703757 -7.50027815798717\\
24.7650146485007 -7.50103646668282\\
24.7955322266258 -7.50851213384732\\
24.8260498047509 -7.51042445723584\\
24.856567382876 -7.52847449604059\\
24.887084961001 -7.54341887497065\\
24.9176025391261 -7.56182494268461\\
24.9481201172512 -7.57320090884468\\
24.9786376953763 -7.58067581245103\\
25.0091552735014 -7.59239297038164\\
25.0396728516264 -7.58419412402071\\
25.0701904297515 -7.59670385385567\\
25.1007080078766 -7.60855107916501\\
25.1312255860017 -7.63272924867186\\
25.1617431641268 -7.64244966365959\\
25.1922607422518 -7.6574594889841\\
25.2227783203769 -7.66044872204981\\
25.253295898502 -7.66894192280103\\
25.2838134766271 -7.66299104091189\\
25.3143310547521 -7.66484091534852\\
25.3448486328772 -7.66731510109986\\
25.3753662110023 -7.67995521430157\\
25.4058837891274 -7.6904665118206\\
25.4364013672525 -7.70263979724706\\
25.4669189453775 -7.71177946400309\\
25.4974365235026 -7.72445631113624\\
25.5279541016277 -7.73421295777808\\
25.5584716797528 -7.7399697013102\\
25.5889892578778 -7.74684215629043\\
25.6195068360029 -7.75042576651327\\
25.650024414128 -7.75668786818937\\
25.6805419922531 -7.75560407143064\\
25.7110595703782 -7.78007614136271\\
25.7415771485032 -7.77520302382766\\
25.7720947266283 -7.79185145246731\\
25.8026123047534 -7.79686638620075\\
25.8331298828785 -7.79680372763056\\
25.8636474610035 -7.80988601924258\\
25.8941650391286 -7.80730651101237\\
25.9246826172537 -7.8172396550047\\
25.9552001953788 -7.83755511792623\\
25.9857177735039 -7.86516099330669\\
26.0162353516289 -7.88121036507437\\
26.046752929754 -7.88005107652526\\
26.0772705078791 -7.90170007869961\\
26.1077880860042 -7.9209465553032\\
26.1383056641292 -7.92207244584449\\
26.1688232422543 -7.92559548276364\\
26.1993408203794 -7.93528020121875\\
26.2298583985045 -7.93688454641909\\
26.2603759766296 -7.93750229982015\\
26.2908935547546 -7.94180065202818\\
26.3214111328797 -7.95347470778864\\
26.3519287110048 -7.95865728146759\\
26.3824462891299 -7.96702136159263\\
26.4129638672549 -7.97019226610354\\
26.44348144538 -7.96581025901838\\
26.4739990235051 -7.96855162501953\\
26.5045166016302 -7.98817673692906\\
26.5350341797553 -8.00502049826173\\
26.5655517578803 -8.01814961454033\\
26.5960693360054 -8.04470534163363\\
26.6265869141305 -8.08188689784612\\
26.6571044922556 -8.06802631852997\\
26.6876220703806 -8.05092281338244\\
26.7181396485057 -8.01700862031396\\
26.7486572266308 -8.03003852506828\\
26.7791748047559 -8.02987759887827\\
26.809692382881 -8.05010099807225\\
26.840209961006 -8.10564530172286\\
26.8707275391311 -8.11395206163274\\
26.9012451172562 -8.13431629253751\\
26.9317626953813 -8.12470698408521\\
26.9622802735063 -8.13223775966446\\
26.9927978516314 -8.13225250965201\\
27.0233154297565 -8.12781880692251\\
27.0538330078816 -8.14338587796749\\
27.0843505860067 -8.17228541001083\\
27.1148681641317 -8.18661407204661\\
27.1453857422568 -8.20482547873638\\
27.1759033203819 -8.2129487367319\\
27.206420898507 -8.21143688113517\\
27.236938476632 -8.21289096344651\\
27.2674560547571 -8.20206061544616\\
27.2979736328822 -8.20191093991565\\
27.3284912110073 -8.21131744401146\\
27.3590087891324 -8.21884929925517\\
27.3895263672574 -8.2359371332675\\
27.4200439453825 -8.24113645358818\\
27.4505615235076 -8.2539750005069\\
27.4810791016327 -8.25384659680304\\
27.5115966797578 -8.25785999828906\\
27.5421142578828 -8.25717011123305\\
27.5726318360079 -8.27586289061588\\
27.603149414133 -8.28980763136093\\
27.6336669922581 -8.29398015237894\\
27.6641845703831 -8.30251096033305\\
27.6947021485082 -8.31865617288406\\
27.7252197266333 -8.3477910721777\\
27.7557373047584 -8.34915428692284\\
27.7862548828835 -8.35636245098601\\
27.8167724610085 -8.34939420611175\\
27.8472900391336 -8.3533472541327\\
27.8778076172587 -8.33675004690173\\
27.9083251953838 -8.3358787838049\\
27.9388427735088 -8.33872143040196\\
27.9693603516339 -8.35502231349113\\
27.999877929759 -8.31644237241295\\
28.0303955078841 -8.35945011716046\\
28.0609130860092 -8.32685003677994\\
28.0914306641342 -8.40700963039768\\
28.1219482422593 -8.41125674762833\\
28.1524658203844 -8.4190677868383\\
28.1829833985095 -8.42348502087339\\
28.2135009766345 -8.44328235806142\\
28.2440185547596 -8.44952893396618\\
28.2745361328847 -8.45801466412115\\
28.3050537110098 -8.4672846579366\\
28.3355712891349 -8.47804911043806\\
28.3660888672599 -8.48232763662452\\
28.396606445385 -8.49639940904456\\
28.4271240235101 -8.49183172120274\\
28.4576416016352 -8.47906429530411\\
28.4881591797602 -8.48439846232179\\
28.5186767578853 -8.49311885688496\\
28.5491943360104 -8.52510880434124\\
28.5797119141355 -8.54687030161529\\
28.6102294922606 -8.5437693021521\\
28.6407470703856 -8.55502604796288\\
28.6712646485107 -8.54626471202897\\
28.7017822266358 -8.55456642596175\\
28.7322998047609 -8.55275543463966\\
28.7628173828859 -8.56383893191691\\
28.793334961011 -8.5790698258142\\
28.8238525391361 -8.58485627221972\\
28.8543701172612 -8.59396585677558\\
28.8848876953863 -8.59216054604155\\
28.9154052735113 -8.59399094889483\\
28.9459228516364 -8.61402807418017\\
28.9764404297615 -8.63113847916918\\
29.0069580078866 -8.63906443105117\\
29.0374755860116 -8.64527689947482\\
29.0679931641367 -8.64486552314276\\
29.0985107422618 -8.65283027372897\\
29.1290283203869 -8.65353857716525\\
29.159545898512 -8.67395848607356\\
29.190063476637 -8.68153266004924\\
29.2205810547621 -8.69662742371389\\
29.2510986328872 -8.69688982827296\\
29.2816162110123 -8.70956968323549\\
29.3121337891373 -8.71668294853134\\
29.3426513672624 -8.713161376296\\
29.3731689453875 -8.72617120101518\\
29.4036865235126 -8.73362804241106\\
29.4342041016377 -8.74095194164607\\
29.4647216797627 -8.7464308531288\\
29.4952392578878 -8.74315580364942\\
29.5257568360129 -8.74127291770378\\
29.556274414138 -8.74630906650555\\
29.586791992263 -8.74970873336991\\
29.6173095703881 -8.75850087821078\\
29.6478271485132 -8.77020280190658\\
29.6783447266383 -8.77637238722934\\
29.7088623047634 -8.79263887237363\\
29.7393798828884 -8.80264844697143\\
29.7698974610135 -8.79332632857017\\
29.8004150391386 -8.80766628310369\\
29.8309326172637 -8.82046453785506\\
29.8614501953888 -8.83678450455341\\
29.8919677735138 -8.8617155811599\\
29.9224853516389 -8.87268933899077\\
29.953002929764 -8.87509710472568\\
29.9835205078891 -8.87123746696557\\
30.0140380860141 -8.87004269291697\\
30.0445556641392 -8.86950503921162\\
30.0750732422643 -8.872850024954\\
30.1055908203894 -8.88462334307991\\
30.1361083985145 -8.89019697176019\\
30.1666259766395 -8.90224624941936\\
30.1971435547646 -8.90963531431868\\
30.2276611328897 -8.91823600819151\\
30.2581787110148 -8.90557706907265\\
30.2886962891398 -8.90893263978688\\
30.3192138672649 -8.90802916710464\\
30.34973144539 -8.91310612376776\\
30.3802490235151 -8.91358262393271\\
30.4107666016402 -8.925254020171\\
30.4412841797652 -8.93638676796479\\
30.4718017578903 -8.95704538588529\\
30.5023193360154 -8.98218140405527\\
30.5328369141405 -8.99845951825421\\
30.5633544922655 -9.00642249005665\\
30.5938720703906 -9.01459874070878\\
30.6243896485157 -9.02152837510954\\
30.6549072266408 -9.03878209307732\\
30.6854248047659 -9.04562995569444\\
30.7159423828909 -9.06056980565927\\
30.746459961016 -9.07721631644876\\
30.7769775391411 -9.07574109041946\\
30.8074951172662 -9.08027622149439\\
30.8380126953912 -9.08458504572161\\
30.8685302735163 -9.08774961713129\\
30.8990478516414 -9.09254271166435\\
30.9295654297665 -9.10233390654901\\
30.9600830078916 -9.12176473835012\\
30.9906005860166 -9.13293657758425\\
31.0211181641417 -9.14571756285483\\
31.0516357422668 -9.14050196433811\\
31.0821533203919 -9.12462164831345\\
31.1126708985169 -9.13465990256555\\
31.143188476642 -9.13265194634693\\
31.1737060547671 -9.13617691027184\\
31.2042236328922 -9.16251381984893\\
31.2347412110173 -9.18056648955655\\
31.2652587891423 -9.18243653401066\\
31.2957763672674 -9.20784683610509\\
31.3262939453925 -9.2016610756242\\
31.3568115235176 -9.1899232659988\\
31.3873291016426 -9.17877497818677\\
31.4178466797677 -9.2033422241152\\
31.4483642578928 -9.21441773903672\\
31.4788818360179 -9.23213530066363\\
31.509399414143 -9.23434046283484\\
31.539916992268 -9.24157016211291\\
31.5704345703931 -9.24732082248016\\
31.6009521485182 -9.2392117729695\\
31.6314697266433 -9.23883216203774\\
31.6619873047683 -9.24747180191667\\
31.6925048828934 -9.25914521953109\\
31.7230224610185 -9.25383508219613\\
31.7535400391436 -9.25549126967309\\
31.7840576172687 -9.25803178240409\\
31.8145751953937 -9.26584575997481\\
31.8450927735188 -9.27702558689674\\
31.8756103516439 -9.28923028266121\\
31.906127929769 -9.31043663353819\\
31.936645507894 -9.3275416284128\\
31.9671630860191 -9.32934822956156\\
31.9976806641442 -9.32602087934578\\
32.0281982422693 -9.33837181511495\\
32.0587158203944 -9.33783037036739\\
32.0892333985194 -9.34557333409913\\
32.1197509766445 -9.36013062193166\\
32.1502685547696 -9.36405223330644\\
32.1807861328947 -9.35544511910729\\
32.2113037110198 -9.34876125594383\\
32.2418212891448 -9.34737745244075\\
32.2723388672699 -9.34155508091953\\
32.302856445395 -9.36769700471797\\
32.3333740235201 -9.39054719499723\\
32.3638916016451 -9.42030111631573\\
32.3944091797702 -9.42143716072508\\
32.4249267578953 -9.42686933163748\\
32.4554443360204 -9.42825150289571\\
32.4859619141455 -9.4350009021183\\
32.5164794922705 -9.42814059407772\\
32.5469970703956 -9.44843583686702\\
32.5775146485207 -9.45749575198647\\
32.6080322266458 -9.46670950498532\\
32.6385498047708 -9.46864564692731\\
32.6690673828959 -9.47322464462638\\
32.699584961021 -9.46113993109992\\
32.7301025391461 -9.4510716369474\\
32.7606201172712 -9.45017297054841\\
32.7911376953962 -9.46083629586383\\
32.8216552735213 -9.47798724569498\\
32.8521728516464 -9.48633860497438\\
32.8826904297715 -9.50092846702785\\
32.9132080078965 -9.51315680725895\\
32.9437255860216 -9.52418027300416\\
32.9742431641467 -9.54864468184894\\
33.0047607422718 -9.53995941220785\\
33.0352783203969 -9.53624043409718\\
33.0657958985219 -9.53449563632472\\
33.096313476647 -9.54181270828713\\
33.1268310547721 -9.54772708738267\\
33.1573486328972 -9.5515712751577\\
33.1878662110222 -9.56713176636492\\
33.2183837891473 -9.57767791626247\\
33.2489013672724 -9.58786122161854\\
33.2794189453975 -9.56871739838772\\
33.3099365235226 -9.57850899573918\\
33.3404541016476 -9.58568119859069\\
33.3709716797727 -9.5871344939564\\
33.4014892578978 -9.62540384649361\\
33.4320068360229 -9.62842225804695\\
33.4625244141479 -9.62679749481975\\
33.493041992273 -9.63729401174828\\
33.5235595703981 -9.63314877818618\\
33.5540771485232 -9.63956686718433\\
33.5845947266483 -9.65761894986963\\
33.6151123047733 -9.6620825323252\\
33.6456298828984 -9.67010293509975\\
33.6761474610235 -9.68272927415296\\
33.7066650391486 -9.6701028340903\\
33.7371826172736 -9.6902165946089\\
33.7677001953987 -9.69327340312998\\
33.7982177735238 -9.71070496800752\\
33.8287353516489 -9.70990130291369\\
33.859252929774 -9.70950618966384\\
33.889770507899 -9.7186642779638\\
33.9202880860241 -9.71728899530865\\
33.9508056641492 -9.72746554479374\\
33.9813232422743 -9.74092227366305\\
34.0118408203993 -9.74912953580287\\
34.0423583985244 -9.74961425709961\\
34.0728759766495 -9.74654893634926\\
34.1033935547746 -9.74825562056947\\
34.1339111328997 -9.75183667463926\\
34.1644287110247 -9.74322633884071\\
34.1949462891498 -9.73822184808313\\
34.2254638672749 -9.75573390899189\\
34.2559814454 -9.76937002696343\\
34.2864990235251 -9.77544479103932\\
34.3170166016501 -9.77458284061612\\
34.3475341797752 -9.79047767212404\\
34.3780517579003 -9.80911471353153\\
34.4085693360254 -9.828642880762\\
34.4390869141504 -9.82899672498638\\
34.4696044922755 -9.83840344601589\\
34.5001220704006 -9.83509017109145\\
34.5306396485257 -9.83335767978895\\
34.5611572266507 -9.83458192943181\\
34.5916748047758 -9.83440547008314\\
34.6221923829009 -9.83703812319743\\
34.652709961026 -9.8433247763453\\
34.6832275391511 -9.85452563015622\\
34.7137451172761 -9.85560592949344\\
34.7442626954012 -9.86070552485546\\
34.7747802735263 -9.8650693128447\\
34.8052978516514 -9.8757136375026\\
34.8358154297765 -9.88396966293755\\
34.8663330079015 -9.89200010610642\\
34.8968505860266 -9.89096009265052\\
34.9273681641517 -9.90031197298765\\
34.9578857422768 -9.90733923836683\\
34.9884033204018 -9.91290049511673\\
35.0189208985269 -9.90518926603477\\
35.049438476652 -9.90449768714518\\
35.0799560547771 -9.90998734053471\\
35.1104736329022 -9.9125538944453\\
35.1409912110272 -9.91355158267351\\
35.1715087891523 -9.93272523207963\\
35.2020263672774 -9.95554748402662\\
35.2325439454025 -9.97821211377089\\
35.2630615235275 -9.99108182100269\\
35.2935791016526 -9.99309804506856\\
35.3240966797777 -9.99705500597076\\
35.3546142579028 -9.99289490337327\\
35.3851318360279 -10.0051544060798\\
35.4156494141529 -10.0096620307994\\
35.446166992278 -10.0092413500045\\
35.4766845704031 -10.0163855729782\\
35.5072021485282 -10.0198082531495\\
35.5377197266532 -10.018475949197\\
35.5682373047783 -10.014322768102\\
35.5987548829034 -10.020095022638\\
35.6292724610285 -10.0214441007896\\
35.6597900391536 -10.04360542135\\
35.6903076172786 -10.0627611319209\\
35.7208251954037 -10.07161108889\\
35.7513427735288 -10.0845106917118\\
35.7818603516539 -10.0909811891751\\
35.8123779297789 -10.1054736992003\\
35.842895507904 -10.1044956594269\\
35.8734130860291 -10.1097011227357\\
35.9039306641542 -10.1158243044497\\
35.9344482422793 -10.129769793026\\
35.9649658204043 -10.1218926082151\\
35.9954833985294 -10.1216161562203\\
36.0260009766545 -10.1257467432974\\
36.0565185547796 -10.1303002463519\\
36.0870361329047 -10.1366618401975\\
36.1175537110297 -10.1331476551758\\
36.1480712891548 -10.1401643017177\\
36.1785888672799 -10.1624575931005\\
36.209106445405 -10.1698342688413\\
36.23962402353 -10.173985696753\\
36.2701416016551 -10.1816598761566\\
36.3006591797802 -10.1774542130322\\
36.3311767579053 -10.1775367047172\\
36.3616943360303 -10.1728854921008\\
36.3922119141554 -10.173188247925\\
36.4227294922805 -10.1857657594457\\
36.4532470704056 -10.2003842796573\\
36.4837646485307 -10.201613505843\\
36.5142822266557 -10.2171497136617\\
36.5447998047808 -10.2057466131405\\
36.5753173829059 -10.239283043348\\
36.605834961031 -10.2537639260261\\
36.6363525391561 -10.2538359626184\\
36.6668701172811 -10.2345828153342\\
36.6973876954062 -10.2236814313704\\
36.7279052735313 -10.2154258829069\\
36.7584228516564 -10.2333160122906\\
36.7889404297814 -10.2488261433649\\
36.8194580079065 -10.2620235830531\\
36.8499755860316 -10.2760481982024\\
36.8804931641567 -10.2934894078502\\
36.9110107422818 -10.3158033911503\\
36.9415283204068 -10.3336211805737\\
36.9720458985319 -10.3396525793383\\
37.002563476657 -10.3329034309966\\
37.0330810547821 -10.3396508405268\\
37.0635986329071 -10.3394186078481\\
37.0941162110322 -10.3415477750807\\
37.1246337891573 -10.3532590352771\\
37.1551513672824 -10.3600688170553\\
37.1856689454074 -10.3721697502779\\
37.2161865235325 -10.3629345659399\\
37.2467041016576 -10.3773190345354\\
37.2772216797827 -10.3668485293205\\
37.3077392579078 -10.376308717704\\
37.3382568360328 -10.3911461284776\\
37.3687744141579 -10.3856625895726\\
37.399291992283 -10.389925350922\\
37.4298095704081 -10.427367925024\\
37.4603271485332 -10.4172188766079\\
37.4908447266582 -10.3755206537724\\
37.5213623047833 -10.3485177948994\\
37.5518798829084 -10.3675722292568\\
37.5823974610335 -10.3856371776516\\
37.6129150391585 -10.4058916119768\\
37.6434326172836 -10.4110629736415\\
37.6739501954087 -10.4205905618755\\
37.7044677735338 -10.4361584178724\\
37.7349853516589 -10.4436485742218\\
37.7655029297839 -10.4487813891859\\
37.796020507909 -10.4587650722543\\
37.8265380860341 -10.467155297603\\
37.8570556641592 -10.47101558935\\
37.8875732422842 -10.477628080564\\
37.9180908204093 -10.4797778662987\\
37.9486083985344 -10.4886412285333\\
37.9791259766595 -10.4963161128277\\
38.0096435547846 -10.4983196236746\\
38.0401611329096 -10.5054254394556\\
38.0706787110347 -10.5193262097702\\
38.1011962891598 -10.5192625798575\\
38.1317138672849 -10.5166119558783\\
38.1622314454099 -10.5445213659319\\
38.192749023535 -10.560833542158\\
38.2232666016601 -10.5714932177875\\
38.2537841797852 -10.5587403361594\\
38.2843017579103 -10.5656196504514\\
38.3148193360353 -10.5647404764446\\
38.3453369141604 -10.5617871778447\\
38.3758544922855 -10.5711217647254\\
38.4063720704106 -10.5714226133325\\
38.4368896485356 -10.5806440405986\\
38.4674072266607 -10.5833955441366\\
38.4979248047858 -10.5856825416471\\
38.5284423829109 -10.5885662014462\\
38.558959961036 -10.6100414522616\\
38.589477539161 -10.6182354821579\\
38.6199951172861 -10.6213576782013\\
38.6505126954112 -10.620683445244\\
38.6810302735363 -10.627513723006\\
38.7115478516614 -10.6343406295785\\
38.7420654297864 -10.6406743852045\\
38.7725830079115 -10.6464696773525\\
38.8031005860366 -10.6479651399691\\
38.8336181641617 -10.6574566259292\\
38.8641357422867 -10.6653623005574\\
38.8946533204118 -10.6833627234298\\
38.9251708985369 -10.6813995792845\\
38.955688476662 -10.6850120965491\\
38.9862060547871 -10.6930740028357\\
39.0167236329121 -10.700029479476\\
39.0472412110372 -10.7165087266796\\
39.0777587891623 -10.7175313922933\\
39.1082763672874 -10.7282925645679\\
39.1387939454124 -10.7333660724507\\
39.1693115235375 -10.7384073884929\\
39.1998291016626 -10.7440509197513\\
39.2303466797877 -10.7456870496649\\
39.2608642579128 -10.7488220573858\\
39.2913818360378 -10.7586495193893\\
39.3218994141629 -10.7624546196094\\
39.352416992288 -10.7689752236734\\
39.3829345704131 -10.7834819886162\\
39.4134521485381 -10.8091429026094\\
39.4439697266632 -10.8068081809683\\
39.4744873047883 -10.8114772644065\\
39.5050048829134 -10.8144406606777\\
39.5355224610385 -10.8129695151629\\
39.5660400391635 -10.8051997661527\\
39.5965576172886 -10.8012996318132\\
39.6270751954137 -10.8019110244394\\
39.6575927735388 -10.8261864242878\\
39.6881103516638 -10.8465621184125\\
39.7186279297889 -10.8433971894937\\
39.749145507914 -10.8458165293087\\
39.7796630860391 -10.864024835504\\
39.8101806641642 -10.8668216170627\\
39.8406982422892 -10.8698450642582\\
39.8712158204143 -10.8788283971168\\
39.9017333985394 -10.8933032630767\\
39.9322509766645 -10.9034244212532\\
39.9627685547895 -10.9071891449446\\
39.9932861329146 -10.9059542841159\\
40.0238037110397 -10.905400677515\\
40.0543212891648 -10.9100508649328\\
40.0848388672899 -10.9064508735377\\
40.1153564454149 -10.9136388186892\\
40.14587402354 -10.9215499859199\\
40.1763916016651 -10.9339359138275\\
40.2069091797902 -10.9406184036829\\
40.2374267579152 -10.9500800879544\\
40.2679443360403 -10.9566102597553\\
40.2984619141654 -10.9638770571867\\
40.3289794922905 -10.9757430748705\\
40.3594970704156 -10.9785934267301\\
40.3900146485406 -10.9860721639625\\
40.4205322266657 -11.0050758372716\\
40.4510498047908 -11.0325066561372\\
40.4815673829159 -11.0336154685881\\
40.5120849610409 -11.0387466449585\\
40.542602539166 -11.0284594296484\\
40.5731201172911 -11.0254590694999\\
40.6036376954162 -11.0318077612947\\
40.6341552735413 -11.0378235492416\\
40.6646728516663 -11.0472167163118\\
40.6951904297914 -11.0502301215309\\
40.7257080079165 -11.0540353825804\\
40.7562255860416 -11.0517089801505\\
40.7867431641667 -11.054764013641\\
40.8172607422917 -11.0620893761262\\
40.8477783204168 -11.0707047433197\\
40.8782958985419 -11.0914249607423\\
40.908813476667 -11.099862977254\\
40.939331054792 -11.1085948568785\\
40.9698486329171 -11.1138413922412\\
41.0003662110422 -11.1178858467484\\
41.0308837891673 -11.1210393906482\\
41.0614013672923 -11.1170601397596\\
41.0919189454174 -11.1145723970596\\
41.1224365235425 -11.1246886507398\\
41.1529541016676 -11.1296382933969\\
41.1834716797927 -11.1375270420117\\
41.2139892579177 -11.1482431251989\\
41.2445068360428 -11.1556518775834\\
41.2750244141679 -11.158376085946\\
41.305541992293 -11.1542226587964\\
41.3360595704181 -11.1462179932244\\
41.3665771485431 -11.1550164738893\\
41.3970947266682 -11.1592680626888\\
41.4276123047933 -11.1668813611067\\
41.4581298829184 -11.1667688415078\\
41.4886474610434 -11.1735222635423\\
41.5191650391685 -11.1755627000521\\
41.5496826172936 -11.176490099529\\
41.5802001954187 -11.1845520621312\\
41.6107177735438 -11.1974847326284\\
41.6412353516688 -11.2181972414458\\
41.6717529297939 -11.2347851795023\\
41.702270507919 -11.2354446100391\\
41.7327880860441 -11.239983610479\\
41.7633056641691 -11.241056082535\\
41.7938232422942 -11.2455937645763\\
41.8243408204193 -11.247871681794\\
41.8548583985444 -11.2538306862173\\
41.8853759766695 -11.2627993176403\\
41.9158935547945 -11.2666727806948\\
41.9464111329196 -11.2743304810073\\
41.9769287110447 -11.2915694444858\\
42.0074462891698 -11.2953767028036\\
42.0379638672948 -11.3005278027587\\
42.0684814454199 -11.2985066455233\\
42.098999023545 -11.2945634215059\\
42.1295166016701 -11.2960808312204\\
42.1600341797952 -11.3064979005359\\
42.1905517579202 -11.3125802734028\\
42.2210693360453 -11.3276750065062\\
42.2515869141704 -11.3368448401292\\
42.2821044922955 -11.3318813394253\\
42.3126220704205 -11.3280261522881\\
42.3431396485456 -11.3305801023818\\
42.3736572266707 -11.3353127995114\\
42.4041748047958 -11.333818201882\\
42.4346923829209 -11.3532487691204\\
42.4652099610459 -11.3730951192293\\
42.495727539171 -11.3887826316291\\
42.5262451172961 -11.3861258688246\\
42.5567626954212 -11.3890810174025\\
42.5872802735462 -11.3930022306636\\
42.6177978516713 -11.3920136389567\\
42.6483154297964 -11.3943504791733\\
42.6788330079215 -11.3975762428763\\
42.7093505860466 -11.4015295131661\\
42.7398681641716 -11.4093221263598\\
42.7703857422967 -11.4289022518741\\
42.8009033204218 -11.4357831203776\\
42.8314208985469 -11.4445189396509\\
42.8619384766719 -11.4472464006417\\
42.892456054797 -11.4470249942683\\
42.9229736329221 -11.4500595396222\\
42.9534912110472 -11.4577907154178\\
42.9840087891723 -11.4714618197652\\
43.0145263672973 -11.479521875613\\
43.0450439454224 -11.4804072810589\\
43.0755615235475 -11.485078493276\\
43.1060791016726 -11.497539881365\\
43.1365966797976 -11.502978026803\\
43.1671142579227 -11.5009735550506\\
43.1976318360478 -11.5025539231461\\
43.2281494141729 -11.5042480821842\\
43.258666992298 -11.5014123314777\\
43.289184570423 -11.4909404290034\\
43.3197021485481 -11.4798529257475\\
43.3502197266732 -11.4960490718178\\
43.3807373047983 -11.5187741119858\\
43.4112548829234 -11.5377288235055\\
43.4417724610484 -11.5309592464457\\
43.4722900391735 -11.4554132599562\\
43.5028076172986 -11.4145518440437\\
43.5333251954237 -11.4152769671922\\
43.5638427735487 -11.5116410465454\\
43.5943603516738 -11.5477926935889\\
43.6248779297989 -11.5596239676701\\
43.655395507924 -11.5395983005006\\
43.685913086049 -11.5280673049883\\
43.7164306641741 -11.5732727519711\\
43.7469482422992 -11.5907108027756\\
43.7774658204243 -11.6018572751445\\
43.8079833985494 -11.6059928100757\\
43.8385009766744 -11.5891123184666\\
43.8690185547995 -11.5895205443018\\
43.8995361329246 -11.5967208450873\\
43.9300537110497 -11.5907080635172\\
43.9605712891748 -11.5921631058075\\
43.9910888672998 -11.6018966177114\\
44.0216064454249 -11.603663107351\\
44.05212402355 -11.6053369432634\\
44.0826416016751 -11.5966672075728\\
44.1131591798001 -11.5867335504076\\
44.1436767579252 -11.5841944172431\\
44.1741943360503 -11.5946063285557\\
44.2047119141754 -11.6063448772674\\
44.2352294923005 -11.6129219161158\\
44.2657470704255 -11.6307711850835\\
44.2962646485506 -11.7449751820248\\
44.3267822266757 -11.7475485433164\\
44.3572998048008 -11.7027787031217\\
44.3878173829258 -11.718888286545\\
44.4183349610509 -11.6990985410253\\
44.448852539176 -11.6897333270474\\
44.4793701173011 -11.6942245753763\\
44.5098876954262 -11.7081233370317\\
44.5404052735512 -11.7214909733202\\
44.5709228516763 -11.7276296557792\\
44.6014404298014 -11.7377475800292\\
44.6319580079265 -11.7449541074074\\
44.6624755860515 -11.7451841096797\\
44.6929931641766 -11.7469780117677\\
44.7235107423017 -11.7445280664011\\
44.7540283204268 -11.7413062676562\\
44.7845458985519 -11.7373745125412\\
44.8150634766769 -11.7401480862147\\
44.845581054802 -11.7358862925855\\
44.8760986329271 -11.7395468714478\\
44.9066162110522 -11.7518870192259\\
44.9371337891772 -11.7631793266174\\
44.9676513673023 -11.7761567958069\\
44.9981689454274 -11.7804235243166\\
45.0286865235525 -11.7771445278274\\
45.0592041016776 -11.7746951890256\\
45.0897216798026 -11.763297214674\\
45.1202392579277 -11.7658534723957\\
45.1507568360528 -11.7725763003504\\
45.1812744141779 -11.7828484966947\\
45.2117919923029 -11.799428531418\\
45.242309570428 -11.7985549618571\\
45.2728271485531 -11.8107260716752\\
45.3033447266782 -11.8088584873394\\
45.3338623048033 -11.8083612690598\\
45.3643798829283 -11.8158672852878\\
45.3948974610534 -11.8145695871479\\
45.4254150391785 -11.8307045547442\\
45.4559326173036 -11.8345726625595\\
45.4864501954287 -11.8471996103518\\
45.5169677735537 -11.8421059752759\\
45.5474853516788 -11.8404641928342\\
45.5780029298039 -11.8483715357306\\
45.608520507929 -11.8531945788415\\
45.639038086054 -11.8581924481119\\
45.6695556641791 -11.8667071455516\\
45.7000732423042 -11.8775650472298\\
45.7305908204293 -11.878400205914\\
45.7611083985543 -11.8886800606707\\
45.7916259766794 -11.8890788668043\\
45.8526611329296 -11.893884661769\\
45.9136962891797 -11.8952581428263\\
45.9747314454299 -11.9011720483961\\
46.0357666016801 -11.9109492353431\\
46.0968017579302 -11.9195690015907\\
46.1578369141804 -11.9405054520731\\
46.2188720704305 -11.9419561838587\\
46.2799072266807 -11.9468385354592\\
46.3409423829308 -11.9499848753394\\
46.401977539181 -11.9513218792072\\
46.4630126954311 -11.9605999301198\\
46.5240478516813 -11.9764424821172\\
46.5850830079315 -11.9936924831391\\
46.6461181641816 -12.0013786761873\\
46.7071533204318 -12.0078535020864\\
46.7681884766819 -12.0133067261305\\
46.8292236329321 -12.0173344701442\\
46.8902587891822 -12.0300279849008\\
46.9512939454324 -12.0369480040829\\
47.0123291016825 -12.0375777982842\\
47.0733642579327 -12.0531776604335\\
47.1343994141829 -12.0647574741981\\
47.195434570433 -12.0771530006861\\
47.2564697266832 -12.0913840424138\\
47.3175048829333 -12.1062681079781\\
47.3785400391835 -12.1143722682407\\
47.4395751954336 -12.124359679791\\
47.5006103516838 -12.1370721808678\\
47.5616455079339 -12.1443182228633\\
47.6226806641841 -12.1533401665311\\
47.6837158204343 -12.1685720781043\\
47.7447509766844 -12.1735647421614\\
47.8057861329346 -12.1872615514278\\
47.8668212891847 -12.1832275690479\\
47.9278564454349 -12.1702320456817\\
47.988891601685 -12.1643557421324\\
48.0499267579352 -12.1690882909303\\
48.1109619141854 -12.1851540254474\\
48.1719970704355 -12.1910630479362\\
48.2330322266857 -12.2054807008868\\
48.2940673829358 -12.2286551680211\\
48.355102539186 -12.2422668917583\\
48.4161376954361 -12.2566099577521\\
48.4771728516863 -12.2661068459461\\
48.5382080079364 -12.2704611143192\\
48.5992431641866 -12.2762165556422\\
48.6602783204368 -12.2769493615162\\
48.7213134766869 -12.2928496971463\\
48.7823486329371 -12.3048977361997\\
48.8433837891872 -12.3097701386474\\
48.9044189454374 -12.3217063250279\\
48.9654541016875 -12.3281430470789\\
49.0264892579377 -12.3477177074703\\
49.0875244141878 -12.3525891064187\\
49.148559570438 -12.3565619986062\\
49.2095947266882 -12.3658852544917\\
49.2706298829383 -12.3754243418886\\
49.3316650391885 -12.3872116481805\\
49.3927001954386 -12.3862271421646\\
49.4537353516888 -12.3891810629128\\
49.5147705079389 -12.3922115079176\\
49.5758056641891 -12.4068016057079\\
49.6368408204392 -12.4175600881008\\
49.6978759766894 -12.4426486861346\\
49.7589111329396 -12.4504353190385\\
49.8199462891897 -12.4600833761993\\
49.8809814454399 -12.4749448776706\\
49.94201660169 -12.4775910059528\\
50.0030517579402 -12.4093885376904\\
50.0640869141903 -12.4671128162434\\
50.1251220704405 -12.4798280854516\\
50.1861572266907 -12.4949844857803\\
50.2471923829408 -12.4941698761852\\
50.308227539191 -12.5001026128405\\
50.3692626954411 -12.5214995917349\\
50.4302978516913 -12.5251496991204\\
50.4913330079414 -12.5341032174506\\
50.5523681641916 -12.5531506164503\\
50.6134033204417 -12.5558320438032\\
50.6744384766919 -12.5617603561601\\
50.7354736329421 -12.5737236253642\\
50.7965087891922 -12.5840695567328\\
50.8575439454424 -12.585978057992\\
50.9185791016925 -12.5768412674651\\
50.9796142579427 -12.5894950125086\\
51.0406494141928 -12.5919094567189\\
51.101684570443 -12.5928578301189\\
51.1627197266931 -12.6122480972151\\
51.2237548829433 -12.6223470006137\\
51.2847900391935 -12.6474789296359\\
51.3458251954436 -12.6434049710126\\
51.4068603516938 -12.6427554872482\\
51.4678955079439 -12.6477755162984\\
51.5289306641941 -12.6458244162845\\
51.5899658204442 -12.6504050678163\\
51.6510009766944 -12.6594502652503\\
51.7120361329445 -12.670196919862\\
51.7730712891947 -12.670638874225\\
51.8341064454449 -12.677237195993\\
51.895141601695 -12.6903760001254\\
51.9561767579452 -12.6994320754295\\
52.0172119141953 -12.7079788267083\\
52.0782470704455 -12.7174588565406\\
52.1392822266956 -12.7320583161394\\
52.2003173829458 -12.7423546586908\\
52.2613525391959 -12.7408044429325\\
52.3223876954461 -12.7428398260413\\
52.3834228516963 -12.7419717959692\\
52.4444580079464 -12.7463920982368\\
52.5054931641966 -12.7527785865752\\
52.5665283204467 -12.7695178218445\\
52.6275634766969 -12.7783845480251\\
52.688598632947 -12.7867714493985\\
52.7496337891972 -12.7835306893712\\
52.8106689454474 -12.7798618662272\\
52.8717041016975 -12.8118091867875\\
52.9327392579477 -12.8205270195583\\
52.9937744141978 -12.8328274243527\\
53.054809570448 -12.8432390335206\\
53.1158447266981 -12.8441086365424\\
53.1768798829483 -12.851063098443\\
53.2379150391984 -12.847465892565\\
53.2989501954486 -12.8479408835553\\
53.3599853516988 -12.8311586607906\\
53.4210205079489 -12.8179558873522\\
53.4820556641991 -12.8203715296701\\
53.5430908204492 -12.8341592783944\\
53.6041259766994 -12.8510149496867\\
53.6651611329495 -12.855213387825\\
53.7261962891997 -12.8664779527527\\
53.7872314454498 -12.8754857068787\\
53.8482666017 -12.89967605905\\
53.9093017579502 -12.907514810253\\
53.9703369142003 -12.9241687702199\\
54.0313720704505 -12.9368405344799\\
54.0924072267006 -12.9375389556039\\
54.1534423829508 -12.9438352489727\\
54.2144775392009 -12.9513592832539\\
54.2755126954511 -12.9452238415917\\
54.3365478517012 -12.9591744241394\\
54.3975830079514 -12.9641406713698\\
54.4586181642016 -12.9788578037054\\
54.5196533204517 -12.9849842093042\\
54.5806884767019 -13.0141414006761\\
54.641723632952 -13.0168421080803\\
54.7027587892022 -13.0317621925894\\
54.7637939454523 -13.023847077892\\
54.8248291017025 -13.0218818504548\\
54.8858642579526 -13.0087405832102\\
54.9468994142028 -13.0103984380149\\
55.007934570453 -13.0099315032999\\
55.0689697267031 -13.0164468038663\\
55.1300048829533 -13.0195843614664\\
55.1910400392034 -13.0231810768219\\
55.2520751954536 -13.0286115427448\\
55.3131103517037 -13.0273338952898\\
55.3741455079539 -13.0150143245902\\
55.4351806642041 -13.0125421932849\\
55.4962158204542 -13.0131843880792\\
55.5572509767044 -13.0490167294626\\
55.6182861329545 -13.0790511265773\\
55.6793212892047 -13.0839837226124\\
55.7403564454548 -13.0797461108158\\
55.801391601705 -13.0838353628884\\
55.8624267579551 -13.0925252468806\\
55.9234619142053 -13.1064502660002\\
55.9844970704555 -13.1210297404039\\
56.0455322267056 -13.1328728601374\\
56.1065673829558 -13.1353454168021\\
56.1676025392059 -13.1407326537833\\
56.2286376954561 -13.1456799151439\\
56.2896728517062 -13.1513851968254\\
56.3507080079564 -13.1502932492983\\
56.4117431642065 -13.1534391885823\\
56.4727783204567 -13.1492998920135\\
56.5338134767069 -13.1612695207218\\
56.594848632957 -13.1860709065702\\
56.6558837892072 -13.1777214409395\\
56.7169189454573 -13.1899611411206\\
56.7779541017075 -13.1954365073418\\
56.8389892579576 -13.196221751426\\
56.9000244142078 -13.1884250263031\\
56.9610595704579 -13.1860703709668\\
57.0220947267081 -13.1754432993245\\
57.0831298829583 -13.1828728969269\\
57.1441650392084 -13.205707216984\\
57.2052001954586 -13.2281927111665\\
57.2662353517087 -13.2447274851813\\
57.3272705079589 -13.2381800364724\\
57.388305664209 -13.2552576499627\\
57.4493408204592 -13.2561520714656\\
57.5103759767094 -13.2616235183529\\
57.5714111329595 -13.2752266055108\\
57.6324462892097 -13.3010595515693\\
57.6934814454598 -13.3213284042932\\
57.75451660171 -13.3374053925896\\
57.8155517579601 -13.3408744092921\\
57.8765869142103 -13.3410382483336\\
57.9376220704604 -13.343703739252\\
57.9986572267106 -13.3516398530816\\
58.0596923829608 -13.3644027796632\\
58.1207275392109 -13.3777183888876\\
58.1817626954611 -13.3862871073923\\
58.2427978517112 -13.3981626875769\\
58.3038330079614 -13.4039407121683\\
58.3648681642115 -13.4199362429177\\
58.4259033204617 -13.4411559605721\\
58.4869384767118 -13.440352438539\\
58.547973632962 -13.4497252100482\\
58.6090087892122 -13.4618357407089\\
58.6700439454623 -13.4666138116847\\
58.7310791017125 -13.4703211074267\\
58.7921142579626 -13.4752472355564\\
58.8531494142128 -13.4710975689086\\
58.9141845704629 -13.4920565328108\\
58.9752197267131 -13.5058427435335\\
59.0362548829632 -13.5191569073911\\
59.0972900392134 -13.5326853900651\\
59.1583251954636 -13.5431232567942\\
59.2193603517137 -13.5444859863425\\
59.2803955079639 -13.5501942635011\\
59.341430664214 -13.5565427130059\\
59.4024658204642 -13.5612446661219\\
59.4635009767143 -13.5648684255163\\
59.5245361329645 -13.5776252118156\\
59.5855712892146 -13.5835806086087\\
59.6466064454648 -13.5848637565088\\
59.707641601715 -13.5903960801818\\
59.7686767579651 -13.6003226149138\\
59.8297119142153 -13.6015375678356\\
59.8907470704654 -13.610949187513\\
59.9517822267156 -13.6413615599272\\
60.0128173829657 -13.6593935998242\\
60.0738525392159 -13.6753252655313\\
60.1348876954661 -13.6855606176597\\
60.1959228517162 -13.7053776828187\\
60.2569580079664 -13.7136479096623\\
60.3179931642165 -13.7304224046869\\
60.3790283204667 -13.7430867383136\\
60.4400634767168 -13.7527630820605\\
60.501098632967 -13.7649242340827\\
60.5621337892171 -13.7703537389211\\
60.6231689454673 -13.7755521790521\\
60.6842041017175 -13.7871096401038\\
60.7452392579676 -13.7989502872182\\
60.8062744142178 -13.8101143680753\\
60.8673095704679 -13.8324495954952\\
60.9283447267181 -13.8419771292111\\
60.9893798829682 -13.8551265142021\\
61.0504150392184 -13.8576067376732\\
61.1114501954685 -13.8678968838481\\
61.1724853517187 -13.8813130027685\\
61.2335205079689 -13.8933154176417\\
61.294555664219 -13.904371029793\\
61.3555908204692 -13.9240208881063\\
61.4166259767193 -13.9441071868765\\
61.4776611329695 -13.9536891577969\\
61.5386962892196 -13.9431539988851\\
61.5997314454698 -13.9581918512476\\
61.6607666017199 -13.9638104631151\\
61.7218017579701 -13.9698317084889\\
61.7828369142203 -13.9869535761512\\
61.8438720704704 -14.0087715535439\\
61.9049072267206 -14.036879252569\\
61.9659423829707 -14.0463576334278\\
62.0269775392209 -14.0663044001924\\
62.088012695471 -14.0855423130929\\
62.1490478517212 -14.078786658817\\
62.2100830079714 -14.0929076355129\\
62.2711181642215 -14.0909070500368\\
62.3321533204717 -14.0952513958641\\
62.3931884767218 -14.1066288410845\\
62.454223632972 -14.1167584615304\\
62.5152587892221 -14.1200218040176\\
62.5762939454723 -14.1300764693514\\
62.6373291017224 -14.1428744525678\\
62.6983642579726 -14.1429426885816\\
62.7593994142228 -14.1647594547598\\
62.8204345704729 -14.2001101634644\\
62.8814697267231 -14.2098892079997\\
62.9425048829732 -14.2173556769215\\
63.0035400392234 -14.2339964827888\\
63.0645751954735 -14.2352619209001\\
63.1256103517237 -14.2477427576083\\
63.1866455079739 -14.2635379725207\\
63.247680664224 -14.272314091517\\
63.3087158204742 -14.2887671767112\\
63.3697509767243 -14.3028240938376\\
63.4307861329745 -14.3167442788551\\
63.4918212892246 -14.3259582593471\\
63.5528564454748 -14.3371949889573\\
63.6138916017249 -14.360415530399\\
63.6749267579751 -14.3860435623088\\
63.7359619142252 -14.4139530398854\\
63.7969970704754 -14.4107660087157\\
63.8580322267256 -14.4035239483682\\
63.9190673829757 -14.3960413005893\\
63.9801025392259 -14.4100698520426\\
64.041137695476 -14.4201500989367\\
64.1021728517262 -14.4316455352283\\
64.1632080079763 -14.4421937342973\\
64.2242431642265 -14.4640425564804\\
64.2852783204766 -14.4726062616122\\
64.3463134767268 -14.4854093931804\\
64.407348632977 -14.4920311880435\\
64.4683837892271 -14.4912534460741\\
64.5294189454773 -14.495085210115\\
64.5904541017274 -14.506191079284\\
64.6514892579776 -14.5093959642787\\
64.7125244142277 -14.5095612362227\\
64.7735595704779 -14.5167427204429\\
64.8345947267281 -14.5313549302971\\
64.8956298829782 -14.5482344781741\\
64.9566650392284 -14.5691148038316\\
65.0177001954785 -14.5828780909577\\
65.0787353517287 -14.5859243990861\\
65.1397705079788 -14.5945112294074\\
65.200805664229 -14.604294936838\\
65.2618408204792 -14.6130887375269\\
65.3228759767293 -14.6274360697632\\
65.3839111329795 -14.6348102784364\\
65.4449462892296 -14.6440004426391\\
65.5059814454798 -14.6573882049405\\
65.5670166017299 -14.6683324355221\\
65.6280517579801 -14.6866638066647\\
65.6890869142302 -14.7018961855833\\
65.7501220704804 -14.7140227682686\\
65.8111572267305 -14.7176797560455\\
65.8721923829807 -14.718598977177\\
65.9332275392309 -14.7313054108948\\
65.994262695481 -14.7367176564539\\
66.0552978517312 -14.7483051128476\\
66.1163330079813 -14.7470836050286\\
66.1773681642315 -14.7503028635122\\
66.2384033204816 -14.7499455196968\\
66.2994384767318 -14.7660791045296\\
66.360473632982 -14.7807275841633\\
66.4215087892321 -14.7982387619047\\
66.4825439454823 -14.8147773254298\\
66.5435791017324 -14.8165385345561\\
66.6046142579826 -14.8214144922256\\
66.6656494142327 -14.8307909351042\\
66.7266845704829 -14.8394625879465\\
66.787719726733 -14.8454248587394\\
66.8487548829832 -14.8581799467761\\
66.9097900392334 -14.8661128822137\\
66.9708251954835 -14.8716486519428\\
67.0318603517337 -14.8774079765847\\
67.0928955079838 -14.8805896268742\\
67.153930664234 -14.8829460584758\\
67.2149658204841 -14.8993979681749\\
67.2760009767343 -14.9256741052738\\
67.3370361329845 -14.9323957062524\\
67.3980712892346 -14.9512918560011\\
67.4591064454847 -14.9649723800226\\
67.5201416017349 -14.9766775017467\\
67.5811767579851 -14.9806630965285\\
67.6422119142352 -14.9775822201813\\
67.7032470704854 -14.9732375419068\\
67.7642822267355 -14.9705322094751\\
67.8253173829857 -14.9902663101277\\
67.8863525392358 -14.9974225431264\\
67.947387695486 -15.0222735332246\\
68.0084228517362 -15.0370595535117\\
68.0694580079863 -15.0567792268045\\
68.1304931642365 -15.0591280605659\\
68.1915283204866 -15.0616868734784\\
68.2525634767368 -15.0698916046068\\
68.3135986329869 -15.0714825941963\\
68.3746337892371 -15.0758054493676\\
68.4356689454873 -15.0849259240005\\
68.4967041017374 -15.0946941432097\\
68.5577392579876 -15.0999019120495\\
68.6187744142377 -15.1038813091437\\
68.6798095704879 -15.1139782623502\\
68.740844726738 -15.1146630726547\\
68.8018798829882 -15.1145305745644\\
68.8629150392383 -15.1240186278171\\
68.9239501954885 -15.1250619724259\\
68.9849853517387 -15.1259691992257\\
69.0460205079888 -15.1420053196055\\
69.107055664239 -15.1516031920772\\
69.1680908204891 -15.1684049151814\\
69.2291259767393 -15.1917378035457\\
69.2901611329894 -15.2207571401843\\
69.3511962892396 -15.2185204459854\\
69.4122314454897 -15.2181055459577\\
69.4732666017399 -15.2004185832001\\
69.5343017579901 -15.2107384694519\\
69.5953369142402 -15.2163680670623\\
69.6563720704904 -15.2307837021257\\
69.7174072267405 -15.2307834267903\\
69.7784423829907 -15.2391476395897\\
69.8394775392408 -15.244065715223\\
69.900512695491 -15.2435068226818\\
69.9615478517411 -15.248754183352\\
70.0225830079913 -15.2750744579706\\
70.0836181642415 -15.2947303592386\\
70.1446533204916 -15.3056658275032\\
70.2056884767418 -15.3155431087102\\
70.2667236329919 -15.32521390632\\
70.3277587892421 -15.3276565885688\\
70.3887939454922 -15.3316266427962\\
70.4498291017424 -15.3408275811765\\
70.5108642579926 -15.3431376929651\\
70.5718994142427 -15.3509178854283\\
70.6329345704929 -15.3541974102282\\
70.693969726743 -15.3530239733923\\
70.7550048829932 -15.3563359162322\\
70.8160400392433 -15.37865174917\\
70.8770751954935 -15.3971865018968\\
70.9381103517436 -15.4017800428119\\
70.9991455079938 -15.4212864357382\\
71.060180664244 -15.4342100206828\\
71.1212158204941 -15.4443240868898\\
71.1822509767443 -15.4525826052702\\
71.2432861329944 -15.4564380468539\\
71.3043212892446 -15.4979149640722\\
71.3653564454947 -15.5252701523794\\
71.4263916017449 -15.5189207036267\\
71.487426757995 -15.5219292724522\\
71.5484619142452 -15.5128093634773\\
71.6094970704954 -15.5194218036824\\
71.6705322267455 -15.5334696326699\\
71.7315673829957 -15.5482964487699\\
71.7926025392458 -15.5576644303785\\
71.853637695496 -15.5757960811433\\
71.9146728517461 -15.5814472574002\\
71.9757080079963 -15.6005996619293\\
72.0367431642464 -15.6018523746239\\
72.0977783204966 -15.6066105951927\\
72.1588134767468 -15.6212345610207\\
72.2198486329969 -15.6283009690274\\
72.2808837892471 -15.6323838917021\\
72.3419189454972 -15.637690965032\\
72.4029541017474 -15.6348704164182\\
72.4639892579975 -15.6554846351654\\
72.5250244142477 -15.6629759072351\\
72.5860595704979 -15.6712491467515\\
72.647094726748 -15.6831828180124\\
72.7081298829982 -15.6975659823412\\
72.7691650392483 -15.7077817292781\\
72.8302001954985 -15.6961738659201\\
72.8912353517486 -15.7051875391259\\
72.9522705079988 -15.723366250807\\
73.0133056642489 -15.7404959484804\\
73.0743408204991 -15.7452114590663\\
73.1353759767492 -15.7591648720469\\
73.1964111329994 -15.7656445147751\\
73.2574462892496 -15.7749889481917\\
73.3184814454997 -15.7823670822073\\
73.3795166017499 -15.7762770739973\\
73.440551758 -15.7747010765949\\
73.5015869142502 -15.7764395556312\\
73.5626220705003 -15.7782811068292\\
73.6236572267505 -15.8141303897383\\
73.6846923830007 -15.8296791054714\\
73.7457275392508 -15.8645236674296\\
73.806762695501 -15.8617349597957\\
73.8677978517511 -15.878025119049\\
73.9288330080013 -15.8823444344142\\
73.9898681642514 -15.8882283510469\\
74.0509033205016 -15.8897561019401\\
74.1119384767517 -15.8940152592342\\
74.1729736330019 -15.8876505111779\\
74.2340087892521 -15.8901459522791\\
74.2950439455022 -15.8979777019226\\
74.3560791017524 -15.9178760868724\\
74.4171142580025 -15.9222029040939\\
74.4781494142527 -15.9418637422189\\
74.5391845705028 -15.9577556171636\\
74.600219726753 -15.9673334421498\\
74.6612548830032 -15.970710580123\\
74.7222900392533 -15.9727739914626\\
74.7833251955035 -15.9770430312406\\
74.8443603517536 -15.9852815637951\\
74.9053955080038 -15.993765654206\\
74.9664306642539 -15.9947910098182\\
75.0274658205041 -15.9992007948221\\
75.0885009767542 -16.0008055867896\\
75.1495361330044 -16.0087238429827\\
75.2105712892545 -16.019150915104\\
75.2716064455047 -16.0271319580845\\
75.3326416017549 -16.0345958202743\\
75.393676758005 -16.0465968921849\\
75.4547119142552 -16.0533079802353\\
75.5157470705053 -16.0694304868196\\
75.5767822267555 -16.0750462955371\\
75.6378173830056 -16.0874319846062\\
75.6988525392558 -16.0909869250698\\
75.759887695506 -16.093478184487\\
75.8209228517561 -16.105842771056\\
75.8819580080063 -16.1125436785557\\
75.9429931642564 -16.1355411801856\\
76.0040283205066 -16.1328592596956\\
76.0650634767567 -16.1397544554797\\
76.1260986330069 -16.14739533593\\
76.187133789257 -16.154411007084\\
76.2481689455072 -16.1638240491927\\
76.3092041017574 -16.1724774369582\\
76.4007568361326 -16.1820230243212\\
76.4923095705078 -16.2038583319714\\
76.5838623048831 -16.2138171002155\\
76.6754150392583 -16.2136969229105\\
76.7669677736335 -16.2381222991685\\
76.8585205080088 -16.2405182336914\\
76.950073242384 -16.2545883308239\\
77.0416259767592 -16.2730260187597\\
77.1331787111345 -16.2920299770283\\
77.2247314455097 -16.3002343062079\\
77.3162841798849 -16.3241538938608\\
77.4078369142602 -16.3272596901936\\
77.4993896486354 -16.3316328302264\\
77.5909423830106 -16.3369210603253\\
77.6824951173859 -16.3420435530765\\
77.7740478517611 -16.3493016956397\\
77.8656005861363 -16.3610059256494\\
77.9571533205116 -16.3834797757561\\
78.0487060548868 -16.3887093966677\\
78.140258789262 -16.3986652491901\\
78.2318115236373 -16.418785217881\\
78.3233642580125 -16.4360921438614\\
78.4149169923877 -16.4522653341286\\
78.506469726763 -16.4635434703817\\
78.5980224611382 -16.4627912539351\\
78.6895751955134 -16.4678934249473\\
78.7811279298887 -16.4754664935141\\
78.8726806642639 -16.495913951412\\
78.9642333986391 -16.5083826202178\\
79.0557861330144 -16.5119342722052\\
79.1473388673896 -16.526844440976\\
79.2388916017648 -16.5425615726985\\
79.3304443361401 -16.5512501658806\\
79.4219970705153 -16.568550424507\\
79.5135498048905 -16.5819829361323\\
79.6051025392658 -16.6057608138042\\
79.696655273641 -16.6257594409682\\
79.7882080080162 -16.6358547711608\\
79.8797607423915 -16.6435850158069\\
79.9713134767667 -16.6510747893558\\
80.0628662111419 -16.6624428195252\\
80.1544189455172 -16.6683849363725\\
80.2459716798924 -16.682634323588\\
80.3375244142676 -16.6968611173679\\
80.4290771486429 -16.709172298707\\
80.5206298830181 -16.7195513674696\\
80.6121826173933 -16.7209084445598\\
80.7037353517686 -16.7347229319059\\
80.7952880861438 -16.7360531676227\\
80.886840820519 -16.7714233991097\\
80.9783935548943 -16.7945376426816\\
81.0699462892695 -16.8015987047755\\
81.1614990236448 -16.804601436179\\
81.25305175802 -16.7998307319838\\
81.3446044923952 -16.8120318608913\\
81.4361572267704 -16.8022388267785\\
81.5277099611457 -16.7936454896702\\
81.6192626955209 -16.8216956266047\\
81.7108154298961 -16.845000828562\\
81.8023681642714 -16.8502556534478\\
81.8939208986466 -16.8635430777151\\
81.9854736330219 -16.89023496673\\
82.0770263673971 -16.9099025430365\\
82.1685791017723 -16.9369791053793\\
82.2601318361476 -16.9542294601298\\
82.3516845705228 -16.9687000356171\\
82.443237304898 -16.9740189417815\\
82.5347900392732 -17.0003668907411\\
82.6263427736485 -17.0025926095372\\
82.7178955080237 -17.0063719993794\\
82.809448242399 -17.001949089569\\
82.9010009767742 -17.0089186008213\\
82.9925537111494 -17.0049063404482\\
83.0841064455247 -16.9939117251894\\
83.1756591798999 -17.0109179939658\\
83.2672119142751 -17.0236119675827\\
83.3587646486504 -17.0427817352899\\
83.4503173830256 -17.0560381586935\\
83.5418701174008 -17.0717400158658\\
83.6334228517761 -17.0869200624247\\
83.7249755861513 -17.0718968755683\\
83.8165283205265 -17.0731869766181\\
83.9080810549018 -17.0848942099142\\
83.999633789277 -17.0859853850479\\
84.0911865236522 -17.1085744679333\\
84.1827392580275 -17.1175958708069\\
84.2742919924027 -17.1204253751263\\
84.3658447267779 -17.1349323996162\\
84.4573974611532 -17.1488547441732\\
84.5489501955284 -17.1538613480253\\
84.6405029299036 -17.1783016092214\\
84.7320556642789 -17.1859313556654\\
84.8236083986541 -17.1876305549636\\
84.9151611330293 -17.196177820437\\
85.0067138674046 -17.1960678716715\\
85.0982666017798 -17.1913720786471\\
85.189819336155 -17.1824220668982\\
85.2813720705303 -17.184709949302\\
85.3729248049055 -17.1902991374379\\
85.4644775392807 -17.197196852902\\
85.556030273656 -17.1987215092007\\
85.6475830080312 -17.2239227265587\\
85.7391357424064 -17.241003277568\\
85.8306884767817 -17.260125521066\\
85.9222412111569 -17.2803649821679\\
86.0137939455321 -17.2860053100888\\
86.1053466799074 -17.2936781249296\\
86.1968994142826 -17.2920409067861\\
86.2884521486578 -17.2982314920543\\
86.3800048830331 -17.2796762327747\\
86.4715576174083 -17.2866112793984\\
86.5631103517835 -17.2892076100622\\
86.6546630861588 -17.2911061706565\\
86.746215820534 -17.293330592758\\
86.8377685549092 -17.2948468283896\\
86.9293212892845 -17.3068897979834\\
87.0208740236597 -17.3136130487871\\
87.1124267580349 -17.3292877942523\\
87.2039794924102 -17.345461055107\\
87.2955322267854 -17.3505593836923\\
87.3870849611606 -17.3555111269989\\
87.4786376955359 -17.3802937519383\\
87.5701904299111 -17.3948198835425\\
87.6617431642863 -17.4052555090599\\
87.7532958986616 -17.4079796370689\\
87.8448486330368 -17.4236980295919\\
87.936401367412 -17.4244832613225\\
88.0279541017873 -17.4332289129093\\
88.1195068361625 -17.4230602618032\\
88.2110595705377 -17.4260336736929\\
88.302612304913 -17.425971780621\\
88.3941650392882 -17.4516689094841\\
88.4857177736635 -17.4718210269916\\
88.5772705080387 -17.4992126754386\\
88.6688232424139 -17.5264411840245\\
88.7603759767891 -17.5284267316918\\
88.8519287111644 -17.5245306977459\\
88.9434814455396 -17.5137376931423\\
89.0350341799148 -17.5195009560952\\
89.1265869142901 -17.5263088375367\\
89.2181396486653 -17.5377514147065\\
89.3096923830406 -17.5410377019234\\
89.4012451174158 -17.5455393760438\\
89.492797851791 -17.5458882110848\\
89.5843505861663 -17.5490854681176\\
89.6759033205415 -17.5624986880833\\
89.7674560549167 -17.5766817641148\\
89.8590087892919 -17.5609400769027\\
89.9505615236672 -17.5667199163955\\
90.0421142580424 -17.5700719575617\\
90.1336669924177 -17.5690157459325\\
90.2252197267929 -17.5657767359757\\
90.3167724611681 -17.5784739976135\\
90.4083251955434 -17.5936128462717\\
90.4998779299186 -17.5922290833089\\
90.5914306642938 -17.6128059180469\\
90.6829833986691 -17.611713984303\\
90.7745361330443 -17.6230831607701\\
90.8660888674195 -17.6207380609425\\
90.9576416017948 -17.6279487817411\\
91.04919433617 -17.6268299398079\\
91.1407470705452 -17.6270498448168\\
91.2322998049205 -17.6404976560991\\
91.3238525392957 -17.6347055266538\\
91.4154052736709 -17.6391955992455\\
91.5069580080462 -17.6543244182046\\
91.5985107424214 -17.6579641957889\\
91.6900634767966 -17.6718556568253\\
91.7816162111719 -17.6702426177931\\
91.8731689455471 -17.6721444582118\\
91.9647216799223 -17.6756904765321\\
92.0562744142976 -17.6904762901332\\
92.1478271486728 -17.7067775974008\\
92.239379883048 -17.7165521670398\\
92.3309326174233 -17.7142689346065\\
92.4224853517985 -17.719648192215\\
92.5140380861737 -17.7450778469167\\
92.605590820549 -17.7244491818186\\
92.6971435549242 -17.7108641247957\\
92.7886962892994 -17.7172669208805\\
92.8802490236747 -17.7503831648823\\
92.9718017580499 -17.7518813254682\\
93.0633544924251 -17.7631191543983\\
93.1549072268004 -17.7819818386336\\
93.2464599611756 -17.798743419383\\
93.3380126955508 -17.798719059652\\
93.4295654299261 -17.8318036315016\\
93.5211181643013 -17.8643729883956\\
93.6126708986765 -17.8803105031198\\
93.7042236330518 -17.9080859684394\\
93.795776367427 -17.9411534754468\\
93.8873291018022 -17.9570066656761\\
93.9788818361775 -17.9763312260726\\
94.0704345705527 -17.9838628703385\\
94.1619873049279 -17.9923776263892\\
94.2535400393032 -18.0113210914534\\
94.3450927736784 -18.0304833677146\\
94.4366455080536 -18.0516464368554\\
94.5281982424289 -18.0744281653432\\
94.6197509768041 -18.0904240868239\\
94.7113037111793 -18.13043429197\\
94.8028564455546 -18.1493087867307\\
94.8944091799298 -18.1749250540861\\
94.9859619143051 -18.1759242366671\\
95.0775146486803 -18.2022252460205\\
95.1690673830555 -18.2211860936351\\
95.2606201174307 -18.2491407554547\\
95.352172851806 -18.2682636706667\\
95.4437255861812 -18.2888384781377\\
95.5352783205564 -18.3273394850576\\
95.6268310549317 -18.3427867880779\\
95.7183837893069 -18.3672015410795\\
95.8099365236822 -18.3793268946396\\
95.9014892580574 -18.3963246283127\\
95.9930419924326 -18.4130853736954\\
96.0845947268079 -18.4442799422006\\
96.1761474611831 -18.4613516941861\\
96.2677001955583 -18.4907052748496\\
96.3592529299335 -18.5164827669638\\
96.4508056643088 -18.5584016283686\\
96.542358398684 -18.6052453530046\\
96.6339111330593 -18.6169584153326\\
96.7254638674345 -18.6339319743929\\
96.8170166018097 -18.658420911721\\
96.908569336185 -18.673512010414\\
97.0001220705602 -18.6878798335406\\
97.0916748049354 -18.7081303627752\\
97.1832275393106 -18.720476036148\\
97.2747802736859 -18.7237062466612\\
97.3663330080611 -18.7685245324388\\
97.4578857424364 -18.7890453221709\\
97.5494384768116 -18.8237807021972\\
97.6409912111868 -18.8418376108585\\
97.7325439455621 -18.8591882318361\\
97.8240966799373 -18.8733763027575\\
97.9156494143125 -18.9110294772282\\
98.0072021486878 -18.9388071547031\\
98.098754883063 -18.9613985823884\\
98.1903076174382 -18.983046088142\\
98.2818603518135 -18.9959368828605\\
98.3734130861887 -19.0154875240807\\
98.4649658205639 -19.0435305621178\\
98.5565185549392 -19.0650056070876\\
98.6480712893144 -19.0898262661761\\
98.7396240236896 -19.1165606448313\\
98.8311767580649 -19.1446480937803\\
98.9227294924401 -19.1699255901465\\
99.0142822268153 -19.1975391106756\\
99.1058349611906 -19.2367388163441\\
99.1973876955658 -19.2800430778684\\
99.288940429941 -19.3036791865501\\
99.3804931643163 -19.3348826958255\\
99.4720458986915 -19.3664246148972\\
99.5635986330667 -19.3954787079631\\
99.655151367442 -19.409498500082\\
99.7467041018172 -19.4203327386001\\
99.8382568361924 -19.4281115795108\\
99.9298095705677 -19.3888156157757\\
100.021362304943 -19.3811528818709\\
100.112915039318 -19.5087842286458\\
100.204467773693 -19.5610587078265\\
100.296020508069 -19.5984726306424\\
100.387573242444 -19.602442679714\\
100.479125976819 -19.6121376235326\\
100.570678711194 -19.638207945236\\
100.66223144557 -19.6434339317535\\
100.753784179945 -19.6443474974232\\
100.84533691432 -19.6371546611616\\
100.936889648695 -19.6153002960301\\
101.02844238307 -19.5925775642745\\
101.119995117446 -19.5654419193447\\
101.211547851821 -19.5309094615005\\
101.303100586196 -19.4742790833017\\
101.394653320571 -19.4239951201011\\
101.486206054947 -19.3635276323927\\
101.577758789322 -19.3333663646919\\
101.669311523697 -19.3330312379906\\
101.760864258072 -19.3653467541474\\
101.852416992448 -19.4308516111341\\
101.943969726823 -19.5219294892556\\
102.035522461198 -19.5982574835178\\
102.127075195573 -19.6922732565905\\
102.218627929949 -19.8047843134486\\
102.310180664324 -19.8895090953807\\
102.401733398699 -19.9804871785985\\
102.493286133074 -20.054419365736\\
102.584838867449 -20.117047971935\\
102.676391601825 -20.1723897505321\\
102.7679443362 -20.2248845994671\\
102.859497070575 -20.2917463430725\\
102.95104980495 -20.3492341446188\\
103.042602539326 -20.4113092331241\\
103.134155273701 -20.4752704034054\\
103.225708008076 -20.5245046407647\\
103.317260742451 -20.5906875302711\\
103.408813476827 -20.6295406743669\\
103.500366211202 -20.6644920045369\\
103.591918945577 -20.7108641687532\\
103.683471679952 -20.7347876686779\\
103.775024414327 -20.7763466794917\\
103.866577148703 -20.8184438274895\\
103.958129883078 -20.8414972929169\\
104.049682617453 -20.8602272404581\\
104.141235351828 -20.9209343839863\\
104.232788086204 -20.9330792319637\\
104.324340820579 -20.9719099706416\\
104.415893554954 -21.0137783977355\\
104.507446289329 -21.0617378918591\\
104.598999023705 -21.118221425364\\
104.69055175808 -21.1690421746887\\
104.782104492455 -21.2319642089773\\
104.87365722683 -21.29055488632\\
104.965209961206 -21.3456078858411\\
105.056762695581 -21.4216866000957\\
105.148315429956 -21.4930252097747\\
105.239868164331 -21.5700141688359\\
105.331420898706 -21.6849292915055\\
105.422973633082 -21.7651196784811\\
105.514526367457 -21.8927361714327\\
105.606079101832 -22.073061991024\\
105.697631836207 -22.2676788000419\\
105.789184570583 -22.4211064623761\\
105.880737304958 -22.6191829041448\\
105.972290039333 -22.8389449849341\\
106.063842773708 -23.0003108116896\\
106.155395508084 -23.0927691902274\\
106.246948242459 -23.1899688757463\\
106.338500976834 -23.2511519816654\\
106.430053711209 -23.2927073516029\\
106.521606445584 -23.2825195568682\\
106.61315917996 -23.2165791975763\\
106.704711914335 -23.1799578493092\\
106.79626464871 -23.1309926592858\\
106.887817383085 -23.0561416538031\\
107.009887695586 -22.9603093060083\\
107.131958008086 -22.8953125189026\\
107.254028320586 -22.8362824293611\\
107.376098633087 -22.7601578929333\\
107.498168945587 -22.6624405690018\\
107.620239258087 -22.5574727560681\\
107.742309570588 -22.4431492918951\\
107.864379883088 -22.3185970309628\\
107.986450195588 -22.1646454418384\\
108.108520508089 -22.0563294159956\\
108.230590820589 -21.9742600905228\\
108.352661133089 -21.9085529930057\\
108.474731445589 -21.8650013407926\\
108.59680175809 -21.8056881585273\\
108.71887207059 -21.7838166964647\\
108.84094238309 -21.6981250167331\\
108.963012695591 -21.6993506946062\\
109.085083008091 -21.7453344233343\\
109.207153320591 -21.7979385232822\\
109.329223633092 -21.8344389609497\\
109.451293945592 -21.8492673243695\\
109.573364258092 -21.8652976405374\\
109.695434570593 -21.9368694369563\\
109.817504883093 -21.9489200737324\\
109.939575195593 -21.9669914331478\\
110.061645508094 -21.9939799954998\\
110.183715820594 -22.0286831294014\\
110.305786133094 -22.0838789154915\\
110.427856445594 -22.146906108155\\
110.549926758095 -22.191700430398\\
110.671997070595 -22.2475869605188\\
110.794067383095 -22.3369901661763\\
110.916137695596 -22.3503770971593\\
111.038208008096 -22.3824121656829\\
111.160278320596 -22.3480336034042\\
111.282348633097 -22.3276254810265\\
111.404418945597 -22.3662157646047\\
111.526489258097 -22.3786614578001\\
111.648559570598 -22.3820460064822\\
111.770629883098 -22.3486995522165\\
111.892700195598 -22.3262364748279\\
112.014770508099 -22.3532583422048\\
112.136840820599 -22.3299523454872\\
112.258911133099 -22.3295129292015\\
112.380981445599 -22.3622059049101\\
112.5030517581 -22.3518654104091\\
112.6251220706 -22.3463629391338\\
112.7471923831 -22.3788494620588\\
112.869262695601 -22.3829421801505\\
112.991333008101 -22.3126082926145\\
113.113403320601 -22.2761435990527\\
113.235473633102 -22.1763295347624\\
113.357543945602 -22.076448258227\\
113.479614258102 -21.9403757612498\\
113.601684570603 -21.7486878077704\\
113.723754883103 -21.6180742479898\\
113.845825195603 -21.381970975588\\
113.967895508104 -21.0685525628862\\
114.089965820604 -20.6522778828513\\
114.212036133104 -20.0464602518695\\
114.334106445604 -19.0543113134417\\
114.456176758105 -18.1506809931818\\
114.578247070605 -19.4018746060282\\
114.700317383105 -19.9023088734469\\
114.822387695606 -19.915762169076\\
114.944458008106 -19.8782012583533\\
115.066528320606 -19.5350325746155\\
115.188598633107 -19.5343700969269\\
115.310668945607 -19.7931551403956\\
115.432739258107 -20.2623629759811\\
115.554809570608 -19.5177152416018\\
115.676879883108 -18.9553422037753\\
115.798950195608 -17.4630669399094\\
115.921020508108 -17.1614013911413\\
116.043090820609 -18.2043877725222\\
116.165161133109 -19.0175396972492\\
116.287231445609 -19.4944224655748\\
116.40930175811 -19.5854649856777\\
116.53137207061 -19.8991145988349\\
116.65344238311 -20.9418109938652\\
116.775512695611 -22.0219371888879\\
116.897583008111 -22.5313460440667\\
117.019653320611 -22.0395730471499\\
117.141723633112 -21.1321638019157\\
117.263793945612 -20.2318212610424\\
117.385864258112 -19.4588409057675\\
117.507934570613 -18.8426767501708\\
117.630004883113 -18.3927282108188\\
117.752075195613 -18.0064050500102\\
117.874145508113 -17.8110556991018\\
117.996215820614 -17.7033159264283\\
118.118286133114 -17.6754895263914\\
118.240356445614 -17.692219766213\\
118.362426758115 -17.7804551451418\\
118.484497070615 -17.946172836576\\
118.606567383115 -18.0924500011186\\
118.728637695616 -18.261345474998\\
118.850708008116 -18.496078590789\\
118.972778320616 -18.7468654832196\\
119.094848633117 -18.9617787823659\\
119.216918945617 -19.1399373105322\\
119.338989258117 -19.3608928474083\\
119.461059570618 -19.5728029232887\\
119.583129883118 -19.7901532960649\\
119.705200195618 -19.9651519268245\\
119.827270508118 -20.1975370825694\\
119.949340820619 -20.4601290871823\\
120.071411133119 -20.6713747711397\\
120.193481445619 -20.8511488658039\\
120.31555175812 -20.9971459046101\\
120.43762207062 -20.9048802474746\\
120.55969238312 -21.1285300143003\\
120.681762695621 -21.4038732680727\\
120.803833008121 -21.2592502922593\\
120.925903320621 -22.1232247743233\\
121.047973633122 -21.5954300839127\\
121.170043945622 -21.9628455776264\\
121.292114258122 -24.9666486232333\\
121.414184570623 -31.8781111831365\\
121.536254883123 -22.2100359993312\\
121.658325195623 -19.9317104970818\\
121.780395508123 -19.0068955015809\\
121.902465820624 -18.6770472531366\\
122.024536133124 -19.2100090555934\\
122.146606445624 -19.4664068661408\\
122.268676758125 -19.5960899801171\\
122.390747070625 -19.9776292100601\\
122.512817383125 -20.2773280191679\\
122.634887695626 -20.3144861429325\\
122.756958008126 -20.5325902483524\\
122.879028320626 -20.7687051338234\\
123.001098633127 -20.9330301312065\\
123.123168945627 -20.9955779384491\\
123.245239258127 -21.2219638863726\\
123.367309570628 -21.3332432575972\\
123.489379883128 -21.4183998587123\\
123.611450195628 -21.5821742935242\\
123.733520508128 -21.6937464312012\\
123.855590820629 -21.8121323651758\\
123.977661133129 -21.9970829350235\\
124.099731445629 -22.1879236187901\\
124.22180175813 -22.484542903844\\
124.34387207063 -22.5728249801689\\
124.46594238313 -22.725117711853\\
124.588012695631 -22.8725906653509\\
124.710083008131 -22.9877231285186\\
124.832153320631 -23.2387543823525\\
124.954223633132 -23.494255030495\\
125.076293945632 -23.6937111294596\\
125.198364258132 -23.9318934242115\\
125.320434570633 -24.1774653649026\\
125.442504883133 -24.4386256783591\\
125.564575195633 -24.7200808704049\\
125.686645508133 -25.0059888436697\\
125.808715820634 -25.3308171054188\\
125.930786133134 -25.799038927838\\
126.052856445634 -26.1134581345725\\
126.174926758135 -26.3505975721974\\
126.296997070635 -26.7432408376443\\
126.419067383135 -26.7943911403245\\
126.541137695636 -26.799969282769\\
126.663208008136 -26.7044628172012\\
126.785278320636 -26.5319108612609\\
126.907348633137 -26.3062340183025\\
127.029418945637 -26.0712762562626\\
127.151489258137 -25.8009172644594\\
127.273559570637 -25.4946583724826\\
127.395629883138 -25.1854314474985\\
127.517700195638 -24.9993172795692\\
127.639770508138 -24.8887254153735\\
127.761840820639 -24.6117964908049\\
127.883911133139 -24.2375531506739\\
128.005981445639 -23.7942554849427\\
128.12805175814 -23.2707934788278\\
128.25012207064 -22.6330031962652\\
128.37219238314 -21.8308640346635\\
128.494262695641 -20.7973397085687\\
128.616333008141 -19.5765749974543\\
128.738403320641 -18.152965688013\\
128.860473633142 -16.6364631600378\\
128.982543945642 -15.2151956827428\\
129.104614258142 -14.1920018973004\\
129.226684570642 -13.8039368622987\\
129.348754883143 -14.0015895500702\\
129.470825195643 -14.4712007847717\\
129.592895508143 -14.9983798155413\\
129.714965820644 -15.5048994215377\\
129.837036133144 -15.9641996629452\\
129.959106445644 -16.2766883381026\\
130.081176758145 -16.5664830418184\\
130.203247070645 -16.7536651933621\\
130.325317383145 -16.9224996930269\\
130.447387695646 -17.1294430137738\\
130.569458008146 -17.2671157789008\\
130.691528320646 -17.2960365158159\\
130.813598633147 -17.3477598685448\\
130.935668945647 -17.300751686661\\
131.057739258147 -17.2871922956992\\
131.179809570647 -17.2691525998194\\
131.301879883148 -17.2428355765502\\
131.423950195648 -17.1178171138661\\
131.546020508148 -17.0788576242261\\
131.668090820649 -16.9750988661129\\
131.790161133149 -16.8295224269151\\
131.912231445649 -16.6923325414792\\
132.03430175815 -16.5025867104833\\
132.15637207065 -16.2846096729256\\
132.27844238315 -16.0684256984599\\
132.400512695651 -15.8018792851855\\
132.522583008151 -15.6544713795538\\
132.644653320651 -15.4216367391679\\
132.766723633152 -15.0753725731288\\
132.888793945652 -14.6474691278502\\
133.010864258152 -14.3319992495627\\
133.132934570652 -14.0559788344324\\
133.255004883153 -13.5929461511079\\
133.377075195653 -13.0600585360073\\
133.499145508153 -12.7728658707957\\
133.621215820654 -12.3192078880615\\
133.743286133154 -11.8082803098709\\
133.865356445654 -11.0389706608637\\
133.987426758155 -10.5038424749051\\
134.109497070655 -9.70044481059404\\
134.231567383155 -9.03545843949223\\
134.353637695656 -8.29637435894608\\
134.475708008156 -7.67319402835687\\
134.597778320656 -7.80328705761588\\
134.719848633157 -10.4417469517042\\
134.841918945657 -18.7910085137286\\
134.963989258157 -21.0086956695758\\
135.086059570657 -11.9189118292397\\
135.208129883158 -7.70200326732686\\
135.330200195658 -6.45193023499337\\
135.452270508158 -6.24662174278734\\
135.574340820659 -6.69550096094042\\
135.696411133159 -7.50227399993122\\
135.818481445659 -8.20726664961524\\
135.94055175816 -8.69831703159076\\
136.06262207066 -9.22609909562254\\
136.18469238316 -9.76462215931156\\
136.306762695661 -10.2671607665492\\
136.428833008161 -10.735808456644\\
136.550903320661 -11.0582189828663\\
136.672973633161 -11.4043136973236\\
136.795043945662 -11.7284573432523\\
136.917114258162 -12.04786876594\\
137.039184570662 -12.2885494542519\\
137.161254883163 -12.5890206258105\\
137.283325195663 -12.8366997131092\\
137.405395508163 -13.125179185097\\
137.557983398789 -13.4034628082042\\
137.710571289414 -13.6741662991531\\
137.86315918004 -13.9279588359238\\
138.015747070665 -14.13320026349\\
138.16833496129 -14.3306968468652\\
138.320922851916 -14.531274073113\\
138.473510742541 -14.6986104240896\\
138.626098633166 -14.8444829380842\\
138.778686523792 -14.9925738894681\\
138.931274414417 -15.1311629097812\\
139.083862305043 -15.2362277161874\\
139.236450195668 -15.3536754465943\\
139.389038086293 -15.4307878744885\\
139.541625976919 -15.5276809359216\\
139.694213867544 -15.600338219494\\
139.84680175817 -15.7094818948505\\
139.999389648795 -15.8053511965229\\
140.15197753942 -15.9024139985161\\
140.304565430046 -16.0161126336463\\
140.457153320671 -16.1367263618125\\
140.609741211297 -16.2438472857662\\
140.762329101922 -16.3392616233004\\
140.914916992547 -16.4361979141193\\
141.067504883173 -16.5260055061944\\
141.220092773798 -16.6507630880405\\
141.372680664423 -16.7236221216863\\
141.525268555049 -16.8126276612262\\
141.677856445674 -16.9239535184299\\
141.8304443363 -17.0166590965031\\
141.983032226925 -17.1038287739447\\
142.13562011755 -17.1917109839898\\
142.288208008176 -17.2794100569145\\
142.440795898801 -17.3661929610495\\
142.593383789427 -17.4461962588383\\
142.745971680052 -17.5230176667698\\
142.898559570677 -17.5966379077287\\
143.051147461303 -17.681981538056\\
143.203735351928 -17.7644759581858\\
143.356323242554 -17.8376132852787\\
143.508911133179 -17.9193474667213\\
143.661499023804 -18.0036040610673\\
143.81408691443 -18.0787322283999\\
143.966674805055 -18.1535410152769\\
144.119262695681 -18.2382575060922\\
144.271850586306 -18.3029068252492\\
144.424438476931 -18.3633717960924\\
144.577026367557 -18.4134636205109\\
144.729614258182 -18.4625136435885\\
144.882202148807 -18.5394872603648\\
145.034790039433 -18.5989905958322\\
145.187377930058 -18.6508600124826\\
145.339965820684 -18.705542419685\\
145.492553711309 -18.7954064145243\\
145.645141601934 -18.8709969853921\\
145.79772949256 -18.8962544384784\\
145.950317383185 -18.9529750349902\\
146.102905273811 -19.0047622552426\\
146.255493164436 -19.0513610095392\\
146.408081055061 -19.0927601560266\\
146.560668945687 -19.1338779631106\\
146.713256836312 -19.1790492902173\\
146.865844726938 -19.2257132528261\\
147.018432617563 -19.2716689563679\\
147.171020508188 -19.3304340195109\\
147.323608398814 -19.3850092273776\\
147.476196289439 -19.433372217261\\
147.628784180064 -19.4878571360773\\
147.78137207069 -19.5359522680665\\
147.933959961315 -19.5796018199614\\
148.086547851941 -19.6198348524721\\
148.239135742566 -19.6549982163002\\
148.391723633191 -19.6893638383256\\
148.544311523817 -19.7262890398479\\
148.696899414442 -19.7584871784617\\
148.849487305068 -19.793589423116\\
149.002075195693 -19.8364952737644\\
149.154663086318 -19.8770079873005\\
149.307250976944 -19.912026665357\\
149.459838867569 -19.9560958356601\\
149.612426758195 -19.9904072157099\\
149.76501464882 -20.0243033337614\\
149.917602539445 -20.0590268076762\\
150.070190430071 -20.0986544373929\\
150.222778320696 -20.1369714156433\\
150.375366211321 -20.1756646313191\\
150.527954101947 -20.2083549593615\\
150.680541992572 -20.249789477906\\
150.833129883198 -20.2946220311074\\
150.985717773823 -20.3304815687196\\
151.138305664448 -20.3622182489452\\
151.290893555074 -20.3991737587715\\
151.443481445699 -20.4324032980864\\
151.596069336325 -20.4613219857123\\
151.74865722695 -20.4880596606202\\
151.901245117575 -20.5167363074131\\
152.053833008201 -20.5391019530496\\
152.206420898826 -20.5626190825756\\
152.359008789452 -20.594142815508\\
152.511596680077 -20.6317216914835\\
152.664184570702 -20.6633779643705\\
152.816772461328 -20.7023150863649\\
152.969360351953 -20.7301865672219\\
153.121948242578 -20.7565409705854\\
153.274536133204 -20.7839248821319\\
153.427124023829 -20.8055208375434\\
153.579711914455 -20.8286144863481\\
153.73229980508 -20.8524421637533\\
153.884887695705 -20.8854851417367\\
154.037475586331 -20.9122069567244\\
154.190063476956 -20.9440641936927\\
154.342651367582 -20.9748993848333\\
154.495239258207 -21.0048399549107\\
154.647827148832 -21.0314344571549\\
154.800415039458 -21.052390832452\\
154.953002930083 -21.0730297598482\\
155.105590820709 -21.0900394728719\\
155.258178711334 -21.1129524855621\\
155.410766601959 -21.1378502602682\\
155.563354492585 -21.1627319005944\\
155.71594238321 -21.1911785652644\\
155.868530273836 -21.2175342961581\\
156.021118164461 -21.2428869930965\\
156.173706055086 -21.2659633900822\\
156.326293945712 -21.282984079005\\
156.478881836337 -21.3068331435815\\
156.631469726962 -21.3316851655911\\
156.784057617588 -21.3489118103441\\
156.936645508213 -21.3689785699938\\
157.089233398839 -21.3874666298942\\
157.241821289464 -21.4076639591222\\
157.394409180089 -21.4290703698392\\
157.546997070715 -21.444564643963\\
157.69958496134 -21.4599221150919\\
157.852172851966 -21.4789122861225\\
158.004760742591 -21.4964466835007\\
158.157348633216 -21.5079582036637\\
158.309936523842 -21.5245699070292\\
158.462524414467 -21.5424316797278\\
158.615112305093 -21.5567229999497\\
158.767700195718 -21.5803630051312\\
158.920288086343 -21.5955799828776\\
159.072875976969 -21.6178784661608\\
159.225463867594 -21.6328578969728\\
159.378051758219 -21.6537154581002\\
159.530639648845 -21.6656353048226\\
159.68322753947 -21.6830978543417\\
159.835815430096 -21.6979479358052\\
159.988403320721 -21.7181954292087\\
160.140991211346 -21.7390893888215\\
160.293579101972 -21.7720201794289\\
160.446166992597 -21.8105815986036\\
160.598754883223 -21.85107230662\\
160.751342773848 -21.8975111808331\\
160.903930664473 -21.9485729843025\\
161.056518555099 -22.0093964061216\\
161.209106445724 -22.0743802651951\\
161.36169433635 -22.1519279772587\\
161.514282226975 -22.2353329708021\\
161.6668701176 -22.3310667435042\\
161.819458008226 -22.4285297347086\\
161.972045898851 -22.5246655859302\\
162.124633789476 -22.6104784329244\\
162.277221680102 -22.6939058027647\\
162.429809570727 -22.7535814494571\\
162.582397461353 -22.8063100790839\\
162.734985351978 -22.8485664287555\\
162.887573242603 -22.8803798781425\\
163.040161133229 -22.9064302437328\\
163.192749023854 -22.9300214159747\\
163.34533691448 -22.9492059925223\\
163.497924805105 -22.9666607388937\\
163.65051269573 -22.979373277392\\
163.803100586356 -22.9901082875853\\
163.955688476981 -22.9981598050986\\
164.108276367607 -23.0023975480278\\
164.260864258232 -23.0093693838269\\
164.413452148857 -23.0217482864891\\
164.566040039483 -23.025985561937\\
164.718627930108 -23.0467294821258\\
164.871215820734 -23.0604389445464\\
165.023803711359 -23.0662018061062\\
165.176391601984 -23.0736687578131\\
165.32897949261 -23.0843756582094\\
165.481567383235 -23.0899784615355\\
165.63415527386 -23.0963706258334\\
165.786743164486 -23.112145559437\\
165.939331055111 -23.1154132840696\\
166.091918945737 -23.1315186152501\\
166.244506836362 -23.1511334883564\\
166.397094726987 -23.1731324049485\\
166.549682617613 -23.1820805291213\\
166.702270508238 -23.2070562366344\\
166.854858398864 -23.2152507654744\\
167.007446289489 -23.2274874050314\\
167.160034180114 -23.2258170079353\\
167.31262207074 -23.2419172970507\\
167.465209961365 -23.255718562535\\
167.617797851991 -23.2687078060451\\
167.770385742616 -23.2843023403714\\
167.922973633241 -23.2957739002445\\
168.106079101992 -23.3159672587163\\
168.289184570742 -23.3318751739168\\
168.472290039493 -23.3471102564033\\
168.655395508243 -23.3619825842178\\
168.838500976994 -23.371238756814\\
169.021606445744 -23.3844833069974\\
169.204711914495 -23.4027689667226\\
169.387817383245 -23.4246958163482\\
169.570922851996 -23.4480924177474\\
169.754028320746 -23.4684009933866\\
169.937133789496 -23.4901219861256\\
170.120239258247 -23.5108555492287\\
170.303344726997 -23.5264407313906\\
170.486450195748 -23.5383379900373\\
170.669555664498 -23.5486418879037\\
170.852661133249 -23.5634158933879\\
171.035766601999 -23.5775439378897\\
171.21887207075 -23.5993541237627\\
171.4019775395 -23.617071585227\\
171.585083008251 -23.634621393355\\
171.768188477001 -23.6537813175627\\
171.951293945752 -23.6811600137327\\
172.134399414502 -23.6976186481351\\
172.317504883253 -23.7113823408714\\
172.500610352003 -23.7258176034888\\
172.683715820753 -23.7343133992034\\
172.866821289504 -23.7460518580668\\
173.049926758254 -23.7619405292135\\
173.233032227005 -23.7790674564302\\
173.416137695755 -23.8005002125921\\
173.599243164506 -23.823158540568\\
173.782348633256 -23.8494955663499\\
173.965454102007 -23.8730712611235\\
174.148559570757 -23.8954809781151\\
174.331665039508 -23.9187533461015\\
174.514770508258 -23.9432246406519\\
174.697875977009 -23.9635792190085\\
174.880981445759 -23.9814541950066\\
175.06408691451 -24.0013321073261\\
175.24719238326 -24.0200665278451\\
175.43029785201 -24.0465057127364\\
175.613403320761 -24.0787286963986\\
175.796508789511 -24.1116465199786\\
175.979614258262 -24.1466384486814\\
176.162719727012 -24.1784194855243\\
176.345825195763 -24.2047535013312\\
176.528930664513 -24.2328756654323\\
176.712036133264 -24.2569476055661\\
176.895141602014 -24.2855854213191\\
177.078247070765 -24.315517528121\\
177.261352539515 -24.3455895857485\\
177.444458008266 -24.3786815201395\\
177.627563477016 -24.4061037803555\\
177.810668945767 -24.439575876929\\
177.993774414517 -24.4693119015304\\
178.176879883267 -24.4899370456117\\
178.359985352018 -24.5135430732169\\
178.543090820768 -24.5327251030077\\
178.726196289519 -24.5501738108956\\
178.909301758269 -24.5662835784681\\
179.09240722702 -24.5823606436668\\
179.27551269577 -24.592235647493\\
179.458618164521 -24.6043385522095\\
179.641723633271 -24.6241478259952\\
179.824829102022 -24.6307390967904\\
180.007934570772 -24.647654154197\\
180.191040039523 -24.6648715730311\\
180.374145508273 -24.683484482185\\
180.557250977024 -24.69874359768\\
180.740356445774 -24.7116515771418\\
180.923461914524 -24.7251137017697\\
181.106567383275 -24.7389785968541\\
181.289672852025 -24.7567167731532\\
181.472778320776 -24.767133786566\\
181.655883789526 -24.7846809425523\\
181.838989258277 -24.8022866821811\\
182.022094727027 -24.8246612252494\\
182.205200195778 -24.8426436581899\\
182.388305664528 -24.8578013687125\\
182.571411133279 -24.882826245977\\
182.754516602029 -24.9115894819274\\
182.93762207078 -24.9454374636947\\
183.12072753953 -24.9835198961096\\
183.303833008281 -25.024393108576\\
183.486938477031 -25.0700594528575\\
183.670043945781 -25.1243662935762\\
183.853149414532 -25.1929356298111\\
184.036254883282 -25.2621638381008\\
184.219360352033 -25.3390220670184\\
184.402465820783 -25.43708109455\\
184.585571289534 -25.5380222466461\\
184.768676758284 -25.5835872546176\\
184.951782227035 -25.4572909726432\\
185.134887695785 -25.5577805830091\\
185.317993164536 -25.651709055469\\
185.501098633286 -25.6694157305213\\
185.684204102037 -25.7106434007532\\
185.867309570787 -25.7232883318129\\
186.050415039538 -25.7051508108464\\
186.233520508288 -25.6889175782251\\
186.416625977039 -25.6706261777891\\
186.599731445789 -25.6580830824888\\
186.782836914539 -25.6201743033052\\
186.96594238329 -25.6068924379172\\
187.14904785204 -25.5893643887177\\
187.332153320791 -25.5853758397069\\
187.515258789541 -25.5828966281172\\
187.698364258292 -25.563227638571\\
187.881469727042 -25.5738024338616\\
188.064575195793 -25.5779406557575\\
188.247680664543 -25.579836865097\\
188.430786133294 -25.5805543169345\\
188.613891602044 -25.5818926352682\\
188.796997070795 -25.5826654461692\\
188.980102539545 -25.5928955292151\\
189.163208008296 -25.5978762736854\\
189.346313477046 -25.607894771423\\
189.529418945796 -25.6166748506787\\
189.712524414547 -25.6146340145478\\
189.895629883297 -25.5382342488754\\
190.078735352048 -25.5869931711662\\
190.261840820798 -25.5835851798088\\
190.444946289549 -25.6123170182145\\
190.628051758299 -25.6045748754659\\
190.81115722705 -25.6316903267514\\
190.9942626958 -25.63676620346\\
191.177368164551 -25.6421779057007\\
191.360473633301 -25.6566532278291\\
191.543579102052 -25.6730994439858\\
191.726684570802 -25.6773279743572\\
191.909790039553 -25.7008145494373\\
192.092895508303 -25.7121378007259\\
192.276000977053 -25.7197433609411\\
192.459106445804 -25.7316817623027\\
192.642211914554 -25.7450066685756\\
192.825317383305 -25.7612499176677\\
193.008422852055 -25.7775492894327\\
193.191528320806 -25.7861066711185\\
193.374633789556 -25.7937677423259\\
193.557739258307 -25.8041503935884\\
193.740844727057 -25.8177454910014\\
193.923950195808 -25.8305213296145\\
194.107055664558 -25.8448799619479\\
194.290161133309 -25.8565659144047\\
194.473266602059 -25.8690411310737\\
194.65637207081 -25.8794703719033\\
194.83947753956 -25.8902194358855\\
195.02258300831 -25.8991294740188\\
195.205688477061 -25.9111401339409\\
195.388793945811 -25.9173221809924\\
195.571899414562 -25.9277771112129\\
195.755004883312 -25.9400690311443\\
195.938110352063 -25.9506605268931\\
196.121215820813 -25.9620291524761\\
196.304321289564 -25.972975240642\\
196.487426758314 -25.9842179134973\\
196.670532227065 -25.9923176449269\\
196.853637695815 -26.0023280439106\\
197.036743164566 -26.0105065014145\\
197.219848633316 -26.022448789054\\
197.402954102067 -26.0351048424528\\
197.586059570817 -26.0434022200477\\
197.769165039567 -26.0538986462468\\
197.952270508318 -26.0683997087668\\
198.135375977068 -26.083893811323\\
198.318481445819 -26.0953735947681\\
198.501586914569 -26.1067780176725\\
198.715209961445 -26.1229799096948\\
198.92883300832 -26.1365906028629\\
199.142456055196 -26.1514693887699\\
199.356079102072 -26.1650843401563\\
199.569702148947 -26.1833479958917\\
199.783325195823 -26.1874003180323\\
199.996948242698 -26.197971125549\\
200.210571289574 -26.2055524370547\\
200.424194336449 -26.2160416800703\\
200.637817383325 -26.2246365559936\\
200.8514404302 -26.2335131518689\\
201.065063477076 -26.2572946534943\\
201.278686523951 -26.2622531079778\\
201.492309570827 -26.2721224612955\\
201.705932617703 -26.2989586609417\\
201.919555664578 -26.3169613843416\\
202.133178711454 -26.3298495344983\\
202.346801758329 -26.3461186780224\\
202.560424805205 -26.3589660135294\\
202.77404785208 -26.3746658491658\\
202.987670898956 -26.3978088388579\\
203.201293945831 -26.4153078554574\\
203.414916992707 -26.4303659705006\\
203.628540039582 -26.4468999610391\\
203.842163086458 -26.4609588411598\\
204.055786133334 -26.4866969835624\\
204.269409180209 -26.5050623993189\\
204.483032227085 -26.5309499229273\\
204.69665527396 -26.5714845824014\\
204.910278320836 -26.6216276151595\\
205.123901367711 -26.6724824302544\\
205.337524414587 -26.7357570146494\\
205.551147461462 -26.8106849444329\\
205.764770508338 -26.8896224110636\\
205.978393555213 -26.9754363624507\\
206.192016602089 -27.0664660135607\\
206.405639648965 -27.1612338560487\\
206.61926269584 -27.2494565838763\\
206.832885742716 -27.3313953254403\\
207.046508789591 -27.3985368651788\\
207.260131836467 -27.4495113001\\
207.473754883342 -27.4839074773091\\
207.687377930218 -27.5080515474863\\
207.901000977093 -27.5225576579987\\
208.114624023969 -27.5390998397411\\
208.328247070844 -27.5547420329903\\
208.54187011772 -27.5556492695219\\
208.755493164596 -27.5495426633012\\
208.969116211471 -27.5413186422917\\
209.182739258347 -27.5141733480482\\
209.396362305222 -27.4955608867892\\
209.609985352098 -27.4755700926849\\
209.823608398973 -27.4570419229706\\
210.037231445849 -27.4340829920749\\
210.250854492724 -27.4129993569859\\
210.4644775396 -27.3979479000483\\
210.678100586475 -27.3768295575924\\
210.891723633351 -27.3528705919063\\
211.105346680227 -27.3330842869867\\
211.318969727102 -27.3116676510256\\
211.532592773978 -27.2888943031625\\
211.746215820853 -27.2715942505666\\
211.959838867729 -27.2483483687209\\
212.173461914604 -27.2302643048017\\
212.38708496148 -27.2177266980459\\
212.600708008355 -27.2080731734576\\
212.814331055231 -27.1957344839419\\
213.027954102106 -27.188700851754\\
213.241577148982 -27.1826868804275\\
213.455200195858 -27.1740269037343\\
213.668823242733 -27.173846987365\\
213.882446289609 -27.1721736425335\\
214.096069336484 -27.1714905971881\\
214.30969238336 -27.1854675803101\\
214.523315430235 -27.2024608889094\\
214.736938477111 -27.2190271483483\\
214.950561523986 -27.245668848086\\
215.164184570862 -27.2704161377494\\
215.377807617737 -27.3037532928411\\
215.591430664613 -27.344667053569\\
215.805053711489 -27.3952824813291\\
216.018676758364 -27.4586189065624\\
216.23229980524 -27.5386937245273\\
216.445922852115 -27.6198835331164\\
216.659545898991 -27.7118802865929\\
216.873168945866 -27.8107551434481\\
217.086791992742 -27.9126369306412\\
217.300415039617 -28.0100208443233\\
217.514038086493 -28.097446192553\\
217.727661133368 -28.1865449192675\\
217.941284180244 -28.2674207159865\\
218.15490722712 -28.3519042455806\\
218.368530273995 -28.4256832619422\\
218.582153320871 -28.4953713459091\\
218.795776367746 -28.5481464073164\\
219.009399414622 -28.6040386812829\\
219.223022461497 -28.6620389861296\\
219.436645508373 -28.7012478134779\\
219.650268555248 -28.7421193589906\\
219.863891602124 -28.7825919887112\\
220.077514648999 -28.8207246867839\\
220.291137695875 -28.8535322836687\\
220.504760742751 -28.8810061238251\\
220.718383789626 -28.9144870015713\\
220.932006836502 -28.9555463346116\\
221.145629883377 -28.9975684327605\\
221.359252930253 -29.0414858036144\\
221.572875977128 -29.108291501915\\
221.786499024004 -29.1948251798518\\
222.000122070879 -29.309809364635\\
222.213745117755 -29.4592208238904\\
222.42736816463 -29.6679435327817\\
222.640991211506 -29.9420007645837\\
222.854614258382 -30.2894894859672\\
223.068237305257 -30.814671656167\\
223.281860352133 -31.5976895174737\\
223.495483399008 -32.8316216864486\\
223.709106445884 -34.98342527114\\
223.922729492759 -38.6458034839363\\
224.136352539635 -37.3909413067046\\
224.34997558651 -32.281803825365\\
224.563598633386 -29.2033498588298\\
224.777221680261 -27.4144630107267\\
224.990844727137 -26.1754082825253\\
225.204467774013 -25.1354625603619\\
225.418090820888 -24.1751218125225\\
225.631713867764 -23.2699913385118\\
225.845336914639 -22.5080918534398\\
226.058959961515 -22.0382626966916\\
226.27258300839 -21.971537379461\\
226.486206055266 -22.2952755288917\\
226.699829102141 -22.8669180377166\\
226.913452149017 -23.5597008074832\\
227.127075195892 -24.1761433613653\\
227.340698242768 -24.6314102549173\\
227.554321289644 -24.9964032198806\\
227.767944336519 -25.2333891648788\\
227.981567383395 -25.3919473498113\\
228.19519043027 -25.5059935767072\\
228.408813477146 -25.5481656849511\\
228.622436524021 -25.5156372439431\\
228.836059570897 -25.3296965991299\\
229.049682617772 -24.9405380306546\\
229.293823242773 -24.1041363206043\\
229.537963867774 -23.043359484527\\
229.782104492774 -22.0135583789759\\
230.026245117775 -21.2984691063978\\
230.270385742775 -21.0022516176127\\
230.514526367776 -21.1827709300525\\
230.758666992777 -21.6455838535654\\
231.002807617777 -22.3808998240971\\
231.246948242778 -23.3986625295191\\
231.491088867779 -24.7827245335206\\
231.735229492779 -26.5085097182138\\
231.97937011778 -28.608243619637\\
232.22351074278 -31.0833310746554\\
232.467651367781 -33.9982499885102\\
232.711791992782 -37.672194212435\\
232.955932617782 -42.8903590947003\\
233.200073242783 -53.5760670786432\\
233.444213867784 -51.6664728226718\\
233.688354492784 -44.0646004088449\\
233.932495117785 -40.6039475424133\\
234.176635742785 -38.5303948990626\\
234.420776367786 -37.1939759287862\\
234.664916992787 -36.2554148452315\\
234.909057617787 -35.5958282137477\\
235.153198242788 -35.0826345918415\\
235.397338867789 -34.7146199310642\\
235.641479492789 -34.42034399605\\
235.88562011779 -34.173665205041\\
236.12976074279 -33.9847667841263\\
236.373901367791 -33.8178462318915\\
236.618041992792 -33.6676756914554\\
236.862182617792 -33.5298018180964\\
237.106323242793 -33.3942741399289\\
237.350463867794 -33.2701219212307\\
237.594604492794 -33.1609048751908\\
237.838745117795 -33.0642067101126\\
238.082885742795 -32.9656382476146\\
238.327026367796 -32.8674988621433\\
238.571166992797 -32.7757546387726\\
238.815307617797 -32.6847724877649\\
239.059448242798 -32.5977671815089\\
239.303588867799 -32.5298563237388\\
239.547729492799 -32.4557036750036\\
239.7918701178 -32.3820572631882\\
240.0360107428 -32.3235821003398\\
240.280151367801 -32.2687785210967\\
240.524291992802 -32.2119478715865\\
240.768432617802 -32.1638663445376\\
241.012573242803 -32.1125344650442\\
241.256713867804 -32.0605308611085\\
241.500854492804 -32.0026618250506\\
241.744995117805 -31.960265470895\\
241.989135742805 -31.9276999687654\\
242.233276367806 -31.8871187370524\\
242.477416992807 -31.8551942930104\\
242.721557617807 -31.8365730204606\\
242.965698242808 -31.8013962148916\\
243.209838867809 -31.7889419506326\\
243.453979492809 -31.7726598105592\\
243.69812011781 -31.7609441530833\\
243.94226074281 -31.7561173488714\\
244.186401367811 -31.7510092647285\\
244.430541992812 -31.737019435124\\
244.674682617812 -31.7277387154614\\
244.918823242813 -31.718027184345\\
245.162963867814 -31.7064391785748\\
245.407104492814 -31.696508863616\\
245.651245117815 -31.6885510451211\\
245.895385742815 -31.6804468993726\\
246.139526367816 -31.6779267958243\\
246.383666992817 -31.6773306942796\\
246.627807617817 -31.6642377792965\\
246.871948242818 -31.6609946255139\\
247.116088867819 -31.6622907885387\\
247.360229492819 -31.6568660696519\\
247.60437011782 -31.6584100800623\\
247.84851074282 -31.6603874276651\\
248.092651367821 -31.6577416011581\\
248.336791992822 -31.6547909860358\\
248.580932617822 -31.6535517669918\\
248.825073242823 -31.6571133989015\\
249.069213867823 -31.6489621712514\\
249.313354492824 -31.640270409984\\
249.557495117825 -31.6296817518659\\
249.801635742825 -31.6201520641435\\
250.045776367826 -31.6023296149893\\
250.289916992827 -31.6029948762583\\
250.534057617827 -31.6076721596301\\
250.778198242828 -31.6132659023665\\
251.022338867828 -31.6207442632339\\
251.266479492829 -31.6142242987925\\
251.51062011783 -31.6157641928598\\
251.75476074283 -31.6227040230907\\
251.998901367831 -31.6326288023811\\
252.243041992832 -31.6400706379909\\
252.487182617832 -31.6669940095788\\
252.731323242833 -31.6920837348716\\
252.975463867833 -31.720927133598\\
253.219604492834 -31.7572781270082\\
253.463745117835 -31.7909356881523\\
253.707885742835 -31.8172706232898\\
253.952026367836 -31.8469135048284\\
254.196166992837 -31.886395656219\\
254.440307617837 -31.9230828141942\\
254.684448242838 -31.9661337973132\\
254.928588867838 -32.0045900907149\\
255.172729492839 -32.0386956758643\\
255.41687011784 -32.0730869438229\\
255.66101074284 -32.1107812687335\\
255.905151367841 -32.1531860429307\\
256.149291992842 -32.1967646816025\\
256.393432617842 -32.2454621909225\\
256.637573242843 -32.2948147261454\\
256.881713867843 -32.3391099898663\\
257.125854492844 -32.3801786157152\\
257.369995117845 -32.415502264124\\
257.614135742845 -32.4621462191312\\
257.858276367846 -32.5072373096975\\
258.102416992847 -32.5489590249102\\
258.346557617847 -32.6011511101475\\
258.590698242848 -32.6574169002058\\
258.834838867848 -32.6944513761751\\
259.078979492849 -32.7319569865965\\
259.32312011785 -32.7783296030262\\
259.56726074285 -32.8158260255849\\
259.841918945976 -32.8626377453048\\
260.116577149102 -32.8947655614872\\
260.391235352227 -32.9096776247272\\
260.665893555353 -32.9005040922814\\
260.940551758479 -32.8641993867047\\
261.215209961605 -32.809923791365\\
261.48986816473 -32.7154049850321\\
261.764526367856 -32.5950207772664\\
262.039184570982 -32.4508198163815\\
262.313842774107 -32.2827484839697\\
262.588500977233 -32.1176951390869\\
262.863159180359 -31.9616619430734\\
263.137817383484 -31.8236971048482\\
263.41247558661 -31.702923529981\\
263.687133789736 -31.5938800290877\\
263.961791992862 -31.5148401996602\\
264.236450195987 -31.459068347234\\
264.511108399113 -31.4115323205048\\
264.785766602239 -31.3811520668108\\
265.060424805364 -31.3702849436993\\
265.33508300849 -31.3661998335054\\
265.609741211616 -31.3719716090044\\
265.884399414741 -31.3873102134778\\
266.159057617867 -31.4079289272775\\
266.433715820993 -31.4417046055886\\
266.708374024119 -31.477883237812\\
266.983032227244 -31.515001996886\\
267.25769043037 -31.5744838449796\\
267.532348633496 -31.6495573248053\\
267.807006836621 -31.7285759789523\\
268.081665039747 -31.8079767839931\\
268.356323242873 -31.8685868120697\\
268.630981445998 -31.928429614329\\
268.905639649124 -31.9700208243679\\
269.18029785225 -31.94208178739\\
269.454956055376 -31.8517586964512\\
269.729614258501 -31.7518303483009\\
270.004272461627 -31.7418697542213\\
270.278930664753 -31.6763811783865\\
270.553588867878 -31.6219655354463\\
270.828247071004 -31.5618324081255\\
271.10290527413 -31.4604860084322\\
271.377563477255 -31.3633205518805\\
271.652221680381 -31.2759374512369\\
271.926879883507 -31.1977794086093\\
272.201538086633 -31.1235154229857\\
272.476196289758 -31.0505384517837\\
272.750854492884 -30.9867090821069\\
273.02551269601 -30.9031638643672\\
273.300170899135 -30.8213168634645\\
273.574829102261 -30.7368330840059\\
273.849487305387 -30.6593132712058\\
274.124145508512 -30.5777037314298\\
274.398803711638 -30.5068010533427\\
274.673461914764 -30.4361625033428\\
274.94812011789 -30.3727382687226\\
275.222778321015 -30.3270600655842\\
275.497436524141 -30.3082619202705\\
275.772094727267 -30.3226508415922\\
276.046752930392 -30.3825735288352\\
276.321411133518 -30.499709157681\\
276.596069336644 -30.6732516509006\\
276.870727539769 -30.8918160193332\\
277.145385742895 -31.1498563571972\\
277.420043946021 -31.4247082131184\\
277.694702149147 -31.7046643079048\\
277.969360352272 -31.9806176752895\\
278.244018555398 -32.2340460736994\\
278.518676758524 -32.453160525962\\
278.793334961649 -32.6474436785555\\
279.067993164775 -32.7986291914091\\
279.342651367901 -32.9275208141348\\
279.617309571027 -33.0432781014254\\
279.891967774152 -33.1071648635884\\
280.166625977278 -33.1761342591912\\
280.441284180404 -33.2554336713011\\
280.715942383529 -33.3148728146302\\
280.990600586655 -33.3716544590711\\
281.265258789781 -33.4299197149672\\
281.539916992906 -33.5005782215555\\
281.814575196032 -33.5848477170578\\
282.089233399158 -33.6347367214248\\
282.363891602284 -33.6581819255332\\
282.638549805409 -33.633291028784\\
282.913208008535 -33.5595017745265\\
283.187866211661 -33.4396339212769\\
283.462524414786 -33.3275004663656\\
283.737182617912 -33.2413015344317\\
284.011840821038 -33.1106683723955\\
284.286499024163 -33.1395015126716\\
284.561157227289 -33.2360337166924\\
284.835815430415 -33.1547807005442\\
285.110473633541 -33.0465514922481\\
285.385131836666 -32.9783449125313\\
285.659790039792 -32.9104627076679\\
285.934448242918 -32.7813820140968\\
286.209106446043 -32.826119536164\\
286.483764649169 -32.9842698555734\\
286.758422852295 -32.8847376620082\\
287.03308105542 -32.7242487144237\\
287.307739258546 -32.581251075\\
287.582397461672 -32.4713732047647\\
287.857055664798 -32.3735958521348\\
288.131713867923 -32.2811941267091\\
288.406372071049 -32.1999089699416\\
288.681030274175 -32.1257013842256\\
288.9556884773 -32.0553820709881\\
289.230346680426 -31.9788755996804\\
289.505004883552 -31.9043451771843\\
289.779663086677 -31.8332478216684\\
290.054321289803 -31.7560678675447\\
290.359497071054 -31.6590320411136\\
290.664672852305 -31.5660867801773\\
290.969848633556 -31.4586574304016\\
291.275024414806 -31.3549005132883\\
291.580200196057 -31.2477302451148\\
291.885375977308 -31.1245246200439\\
292.190551758559 -30.9851315714468\\
292.495727539809 -30.8148903189988\\
292.80090332106 -30.612679924455\\
293.106079102311 -30.373089361173\\
293.411254883562 -30.0850966235816\\
293.716430664812 -29.7265668040622\\
294.021606446063 -29.2746500923609\\
294.326782227314 -28.6890237583893\\
294.631958008565 -27.8971912111522\\
294.937133789816 -26.7785071857751\\
295.242309571066 -25.0959651913942\\
295.547485352317 -22.3498486316333\\
295.852661133568 -17.39883752054\\
296.157836914819 -15.3008939623463\\
296.46301269607 -22.6357616360339\\
296.76818847732 -27.2504187691132\\
297.073364258571 -29.9755563371438\\
297.378540039822 -31.5729040367582\\
297.683715821073 -32.4792753848596\\
297.988891602323 -32.9923274044692\\
298.294067383574 -33.2781464973955\\
298.599243164825 -33.4301950202043\\
298.904418946076 -33.5043780597446\\
299.209594727327 -33.535612198777\\
299.514770508577 -33.535266310051\\
299.819946289828 -33.5202479958776\\
300.125122071079 -33.4863480048618\\
300.43029785233 -33.4464336416105\\
300.73547363358 -33.408767413472\\
301.040649414831 -33.3814733646013\\
301.345825196082 -33.3580172768075\\
301.651000977333 -33.2995290121013\\
301.956176758584 -33.2742195323567\\
302.261352539834 -33.2389962988292\\
302.566528321085 -33.1862450504273\\
302.871704102336 -33.1439970741835\\
303.176879883587 -33.0883534451104\\
303.482055664837 -33.0331486845993\\
303.787231446088 -32.9845305910387\\
304.092407227339 -32.9402314962597\\
304.39758300859 -32.8986294882912\\
304.702758789841 -32.8567699415032\\
305.007934571091 -32.8160650285511\\
305.313110352342 -32.773987851099\\
305.618286133593 -32.7263401472748\\
305.923461914844 -32.6786272268226\\
306.228637696094 -32.6211510722701\\
306.533813477345 -32.5443228148205\\
306.838989258596 -32.4545103293374\\
307.144165039847 -32.3817589270209\\
307.449340821098 -32.3230195312102\\
307.754516602348 -32.2912163588679\\
308.059692383599 -32.2630614277894\\
308.36486816485 -32.2179993859322\\
308.670043946101 -32.1605398718797\\
308.975219727351 -32.0929679613993\\
309.280395508602 -32.0217001148233\\
309.585571289853 -31.9381378922398\\
309.890747071104 -31.8559241783302\\
310.195922852355 -31.77993339643\\
310.501098633605 -31.7127638207475\\
310.806274414856 -31.6522616409482\\
311.111450196107 -31.6042602530258\\
311.416625977358 -31.5576011057237\\
311.721801758608 -31.51254809785\\
312.026977539859 -31.4775844707087\\
312.33215332111 -31.443634054457\\
312.637329102361 -31.4193988937976\\
312.942504883612 -31.3979018639114\\
313.247680664862 -31.3798196574346\\
313.552856446113 -31.3624078414651\\
313.858032227364 -31.3519859467441\\
314.163208008615 -31.3411400036428\\
314.468383789865 -31.3260044149463\\
314.773559571116 -31.2891871823555\\
315.078735352367 -31.2373563479613\\
315.383911133618 -31.1517861017162\\
315.689086914869 -31.0588708896973\\
315.994262696119 -30.9578844079511\\
316.29943847737 -30.8372932799969\\
316.604614258621 -30.6975482931963\\
316.909790039872 -30.5389766118904\\
317.214965821123 -30.3792247074841\\
317.520141602373 -30.217490383\\
317.825317383624 -30.0511790730548\\
318.130493164875 -29.8886120356925\\
318.435668946126 -29.7077791832112\\
318.740844727376 -29.5137175361653\\
319.046020508627 -29.3123711149201\\
319.351196289878 -29.103215224077\\
319.656372071129 -28.8874291354671\\
319.96154785238 -28.6668465324819\\
320.26672363363 -28.434780606373\\
320.571899414881 -28.1888525385719\\
320.907592774257 -27.915195915873\\
321.243286133633 -27.6276314529518\\
321.578979493009 -27.3143827739018\\
321.914672852385 -26.9798976347166\\
322.25036621176 -26.6199018394193\\
322.586059571136 -26.2310108881271\\
322.921752930512 -25.8209274295122\\
323.257446289888 -25.3727804676947\\
323.593139649264 -24.876352803257\\
323.92883300864 -24.3412991756061\\
324.264526368015 -23.7389548698914\\
324.600219727391 -23.0962192939813\\
324.935913086767 -22.3493375466811\\
325.271606446143 -21.5031494044567\\
325.607299805519 -20.5320256778924\\
325.942993164895 -19.3988981533197\\
326.278686524271 -18.0491992064918\\
326.614379883647 -16.4040976337677\\
326.950073243022 -14.3213921972184\\
327.285766602398 -11.5637541117723\\
327.621459961774 -7.69182218408565\\
327.95715332115 -3.1597019476115\\
328.292846680526 -5.12892954706575\\
328.628540039902 -9.88493668221196\\
328.964233399278 -13.3754423368516\\
329.299926758653 -15.9586518163551\\
329.635620118029 -17.9848634227504\\
329.971313477405 -19.6406023243428\\
330.307006836781 -21.0335901543427\\
330.642700196157 -22.2384510004904\\
330.978393555533 -23.2987411783223\\
331.314086914909 -24.241519576941\\
331.649780274284 -25.1070205879119\\
331.98547363366 -25.9258725344927\\
332.321166993036 -26.7627204751307\\
332.656860352412 -27.3027139228084\\
332.992553711788 -26.8096564979234\\
333.328247071164 -27.1927475326651\\
333.66394043054 -27.6325203916535\\
333.999633789915 -28.0111536866476\\
334.335327149291 -28.3330734208009\\
334.671020508667 -28.6012640731626\\
335.006713868043 -28.8233376641945\\
335.342407227419 -29.0046937610907\\
335.678100586795 -29.1634164643115\\
336.01379394617 -29.2767575086673\\
336.349487305546 -29.3498445532607\\
336.685180664922 -29.3837096651052\\
337.020874024298 -29.3780491092459\\
337.356567383674 -29.3375317670994\\
337.69226074305 -29.2452992887611\\
338.027954102426 -29.0983836985708\\
338.363647461802 -28.9365066420356\\
338.699340821177 -28.7633080996598\\
339.035034180553 -28.6485947475119\\
339.370727539929 -28.5739783199247\\
339.706420899305 -28.4869420281052\\
340.042114258681 -28.3758787475664\\
340.377807618057 -28.2446657866292\\
340.713500977432 -28.0721694226206\\
341.049194336808 -27.8729979698656\\
341.384887696184 -27.6473114145219\\
341.72058105556 -27.3979445606622\\
342.056274414936 -27.1233158190678\\
342.391967774312 -26.8271181333612\\
342.727661133688 -26.5083421312279\\
343.063354493064 -26.1713261306487\\
343.399047852439 -25.8128976667951\\
343.734741211815 -25.4397903930505\\
344.070434571191 -25.0558410098231\\
344.406127930567 -24.6619929383481\\
344.741821289943 -24.2802819226149\\
345.077514649319 -23.9143944639927\\
345.413208008694 -23.5710248911573\\
345.74890136807 -23.2727233152722\\
346.084594727446 -23.0424048755181\\
346.420288086822 -22.8868352927064\\
346.755981446198 -22.8370133756843\\
347.091674805574 -22.9094725208308\\
347.42736816495 -23.0999031703152\\
347.763061524326 -23.4012349379065\\
348.098754883701 -23.8008217315239\\
348.434448243077 -24.2739244823917\\
348.770141602453 -24.8034199031264\\
349.105834961829 -25.3727973427195\\
349.441528321205 -25.9688308949745\\
349.777221680581 -26.5763157447224\\
350.112915039957 -27.1722902025622\\
350.448608399332 -27.7726591550739\\
350.784301758708 -28.3559177311836\\
351.119995118084 -28.9228133084463\\
351.486206055585 -29.5216088820948\\
351.852416993086 -30.0882587669528\\
352.218627930587 -30.6292933344594\\
352.584838868088 -31.1481082020441\\
352.951049805589 -31.6345991725473\\
353.31726074309 -32.1025954395845\\
353.683471680591 -32.5477408553499\\
354.049682618092 -32.9614493482455\\
354.415893555593 -33.3488332821466\\
354.782104493093 -33.7173679794987\\
355.148315430594 -34.0588362061883\\
355.514526368095 -34.3795362669464\\
355.880737305596 -34.689550032634\\
356.246948243097 -34.9812171337147\\
356.613159180598 -35.2441908460169\\
356.979370118099 -35.4808796575572\\
357.3455810556 -35.6983066524537\\
357.711791993101 -35.8850799947554\\
358.078002930602 -36.055376555996\\
358.444213868103 -36.2129971406863\\
358.810424805604 -36.3487232398713\\
359.176635743105 -36.4600629310104\\
359.542846680606 -36.5477904551551\\
359.909057618107 -36.6172244336369\\
360.275268555607 -36.6682189030579\\
360.641479493108 -36.70219436746\\
361.007690430609 -36.7143760373855\\
361.37390136811 -36.723500456211\\
361.740112305611 -36.7030398569879\\
362.106323243112 -36.6620805790713\\
362.472534180613 -36.6027825789601\\
362.838745118114 -36.5257287019429\\
363.204956055615 -36.4301469220365\\
363.571166993116 -36.3468184335352\\
363.937377930617 -36.282406502713\\
364.303588868118 -36.2055348318787\\
364.669799805619 -36.1080627463463\\
365.03601074312 -35.9456803621463\\
365.402221680621 -35.762952840588\\
365.768432618121 -35.5693878660003\\
366.134643555622 -35.3599137309437\\
366.500854493123 -35.1393901763037\\
366.867065430624 -34.9137556164292\\
367.233276368125 -34.6625583956077\\
367.599487305626 -34.3834402068396\\
367.965698243127 -34.0815486579894\\
368.331909180628 -33.7443344562508\\
368.698120118129 -33.3313419475606\\
369.06433105563 -32.9148048382141\\
369.430541993131 -32.4145173220007\\
369.796752930632 -31.8562902794175\\
370.162963868133 -31.2681533943181\\
370.529174805634 -30.7481227799548\\
370.895385743135 -30.3868712090522\\
371.261596680636 -30.1805975065815\\
371.627807618136 -30.0197093343206\\
371.994018555637 -29.8223208814984\\
372.360229493138 -29.5475911760328\\
372.726440430639 -29.2076630744495\\
373.09265136814 -28.7946467740847\\
373.458862305641 -28.3208422192487\\
373.825073243142 -27.7971254376017\\
374.191284180643 -27.2169038523404\\
374.557495118144 -26.5706948638005\\
374.923706055645 -25.8534130231008\\
375.289916993146 -25.0492567212544\\
375.656127930647 -24.1570795944778\\
376.022338868148 -23.1398235017151\\
376.388549805649 -22.027103095464\\
376.75476074315 -20.9202121168766\\
377.12097168065 -19.8270979139697\\
377.487182618151 -18.8259745463552\\
377.853393555652 -17.9255164656964\\
378.219604493153 -17.2632398446437\\
378.585815430654 -17.0225461690173\\
378.952026368155 -17.4486060902526\\
379.318237305656 -18.3984636880513\\
379.684448243157 -19.6835194831936\\
380.050659180658 -21.080008177986\\
380.416870118159 -22.4449857411162\\
380.78308105566 -23.7358779011201\\
381.149291993161 -24.9324520969144\\
381.515502930662 -26.038065356899\\
381.912231446288 -27.1612983956682\\
382.308959961914 -28.2199960497388\\
382.70568847754 -29.2337428203291\\
383.102416993166 -30.2100335240531\\
383.499145508792 -31.1568712037488\\
383.895874024418 -32.1050969969793\\
384.292602540044 -33.0829678246498\\
384.68933105567 -34.115489136795\\
385.086059571296 -35.2861066096516\\
385.482788086922 -36.698281731911\\
385.879516602548 -38.7855384454249\\
386.276245118174 -42.2669027067926\\
386.6729736338 -48.0331081490041\\
387.069702149426 -68.7001771069291\\
387.466430665052 -47.6313191591369\\
387.863159180678 -42.7920377411643\\
388.259887696304 -39.9936521496184\\
388.65661621193 -37.5765938795653\\
389.053344727556 -34.9068132455892\\
389.450073243182 -31.7485677756558\\
389.846801758808 -28.1537474234879\\
390.243530274434 -24.0677185810396\\
390.64025879006 -20.6664410631053\\
391.036987305686 -21.3609653270393\\
391.433715821312 -23.872408369485\\
391.830444336938 -26.0716734904095\\
392.227172852564 -27.7605140897755\\
392.62390136819 -29.1137381025428\\
393.020629883816 -30.1856089187557\\
393.417358399442 -31.0893157990193\\
393.814086915068 -31.8574371245414\\
394.210815430694 -32.5277469224733\\
394.60754394632 -33.1204640895637\\
395.004272461946 -33.6485415810532\\
395.401000977572 -34.1294332720157\\
395.797729493198 -34.5638686287214\\
396.194458008824 -34.9571216015736\\
396.59118652445 -35.3188647756186\\
396.987915040076 -35.6504749859034\\
397.384643555702 -35.9529183056475\\
397.781372071328 -36.2262803238314\\
398.178100586954 -36.485240172576\\
398.57482910258 -36.7194019098008\\
398.971557618206 -36.9389100446053\\
399.368286133832 -37.1412724667525\\
399.765014649458 -37.3438911320791\\
400.161743165084 -37.568997962391\\
400.55847168071 -37.8126741095767\\
400.955200196336 -38.0280735408121\\
401.351928711962 -38.2118754591608\\
401.748657227588 -38.3737696252509\\
402.145385743214 -38.5282149762741\\
402.54211425884 -38.6712036288416\\
402.938842774466 -38.7970785570198\\
403.335571290092 -38.9272305364164\\
403.732299805718 -39.0416914779134\\
404.129028321344 -39.1516623589891\\
404.52575683697 -39.2518567180625\\
404.922485352596 -39.3474097206167\\
405.319213868223 -39.4219603207582\\
405.715942383848 -39.4698954842298\\
406.112670899474 -39.4728453316669\\
406.509399415101 -39.4497028716452\\
406.906127930727 -39.388825719964\\
407.302856446353 -39.342388233487\\
407.699584961979 -39.3231457186807\\
408.096313477605 -39.2653804203458\\
408.493041993231 -39.1548609468586\\
408.889770508857 -39.0767065653304\\
409.286499024483 -38.9485824160539\\
409.683227540109 -38.7435535301224\\
410.079956055735 -38.4470908951747\\
410.476684571361 -38.0149447829115\\
410.873413086987 -37.4111125863634\\
411.270141602613 -36.953921938402\\
411.666870118239 -37.7239974930239\\
412.063598633865 -38.6503296961448\\
412.490844727616 -38.886882877244\\
412.918090821367 -38.7387764708917\\
413.345336915118 -38.4578468705349\\
413.772583008869 -38.4610195759957\\
414.19982910262 -39.3199964552365\\
414.627075196371 -41.3062180361636\\
415.054321290122 -44.2489176187009\\
415.481567383873 -47.7491224199517\\
415.908813477625 -51.3506598026655\\
416.336059571376 -54.1531243545802\\
416.763305665127 -55.009150088623\\
417.190551758878 -54.3628760980515\\
417.617797852629 -53.3515367685409\\
418.04504394638 -52.388640821581\\
418.472290040131 -51.6320488343813\\
418.899536133882 -51.027513324866\\
419.326782227633 -50.5243149155377\\
419.754028321384 -50.1300975043682\\
420.181274415135 -49.7971126314946\\
420.608520508886 -49.5560410669798\\
421.035766602638 -49.34862516327\\
421.463012696389 -49.1860494707182\\
421.89025879014 -49.045587747209\\
422.317504883891 -48.9324682840592\\
422.744750977642 -48.841009797199\\
423.171997071393 -48.7606219457269\\
423.599243165144 -48.7106677175715\\
424.026489258895 -48.6613215555591\\
424.453735352646 -48.6289413037516\\
424.880981446397 -48.6128765880424\\
425.308227540148 -48.5894652925392\\
425.7354736339 -48.5760594631787\\
426.162719727651 -48.5630826831726\\
426.589965821402 -48.5513175680022\\
427.017211915153 -48.5504121569435\\
427.444458008904 -48.5370309963975\\
427.871704102655 -48.5305688691805\\
428.298950196406 -48.5186873254124\\
428.726196290157 -48.4958066915748\\
429.153442383908 -48.4565681246546\\
429.580688477659 -48.4145862625182\\
430.007934571411 -48.3722935596426\\
430.435180665162 -48.316029666287\\
430.862426758913 -48.2802255921729\\
431.289672852664 -48.1727453850128\\
431.716918946415 -48.043526171552\\
432.144165040166 -47.8942918590679\\
432.571411133917 -47.7484286324864\\
432.998657227668 -47.6311539873063\\
433.425903321419 -47.5321988819985\\
433.85314941517 -47.4529925092721\\
434.280395508921 -47.3605446056696\\
434.707641602673 -47.2560441609338\\
435.134887696424 -47.1468173155411\\
435.562133790175 -47.0263045161592\\
435.989379883926 -46.8987071786562\\
436.416625977677 -46.7651235302284\\
436.843872071428 -46.6206164167597\\
437.271118165179 -46.4663078970177\\
437.69836425893 -46.296750948465\\
438.125610352681 -46.1083933044379\\
438.552856446432 -45.8963477745079\\
438.980102540183 -45.6622026679742\\
439.407348633935 -45.3988993075191\\
439.834594727686 -45.0989519635415\\
440.261840821437 -44.7688954115217\\
440.689086915188 -44.4160843830788\\
441.116333008939 -44.0103570285932\\
441.54357910269 -43.5694716580502\\
441.970825196441 -43.0962422054477\\
442.398071290192 -42.5852339354606\\
442.825317383943 -42.0181840229066\\
443.283081055819 -41.3424830411221\\
443.740844727696 -40.5646501098545\\
444.198608399572 -39.668985711628\\
444.656372071448 -38.5947007378569\\
445.114135743324 -37.3286802543585\\
445.5718994152 -35.7898327378224\\
446.029663087076 -33.972096603706\\
446.487426758953 -32.0759422780422\\
446.945190430829 -31.1018895692652\\
447.402954102705 -32.2840644306179\\
447.860717774581 -34.8096550630618\\
448.318481446457 -37.4094438374506\\
448.776245118333 -39.7085668788035\\
449.23400879021 -41.7367094102349\\
449.691772462086 -43.4451148673675\\
450.149536133962 -44.9724983772729\\
450.607299805838 -46.2810806181356\\
451.065063477714 -47.3549739934847\\
451.52282714959 -48.1968891792308\\
451.980590821467 -48.8300968441106\\
452.438354493343 -49.3817173102209\\
452.896118165219 -50.0390803345722\\
453.353881837095 -50.9859129422817\\
453.811645508971 -52.3275520536063\\
454.269409180847 -53.8749727090414\\
454.727172852724 -55.4021064592147\\
455.1849365246 -56.670715667164\\
455.642700196476 -57.6569841040547\\
456.100463868352 -58.3211063900423\\
456.558227540228 -58.7873229258185\\
457.015991212104 -59.1629573964784\\
457.473754883981 -59.4858704678705\\
457.931518555857 -59.7874623686037\\
458.389282227733 -60.0682461240321\\
458.847045899609 -60.3226967172525\\
459.304809571485 -60.5587544335345\\
459.762573243361 -60.8142189453988\\
460.220336915238 -61.070640628845\\
460.678100587114 -61.3396613320296\\
461.13586425899 -61.5883124956794\\
461.593627930866 -61.8020628645127\\
462.051391602742 -62.0028925591825\\
462.509155274619 -62.1436952495631\\
462.966918946495 -62.2206280239178\\
463.424682618371 -62.2827840264439\\
463.882446290247 -62.3032889649814\\
464.340209962123 -62.342565324532\\
464.797973633999 -62.4206090932165\\
465.255737305876 -62.4617711455033\\
465.713500977752 -62.5470093374849\\
466.171264649628 -62.6608875789387\\
466.629028321504 -62.8289480989974\\
467.08679199338 -62.9777795686534\\
467.544555665256 -63.1937747489376\\
468.002319337133 -63.4058642912808\\
468.460083009009 -63.5929727332293\\
468.917846680885 -63.7683029717455\\
469.375610352761 -63.9095891842865\\
469.833374024637 -63.9686003806422\\
470.291137696513 -64.0036798712245\\
470.74890136839 -63.9639335927361\\
471.206665040266 -63.9023608356006\\
471.664428712142 -63.8823414500045\\
472.122192384018 -63.8202781267769\\
472.579956055894 -63.7397352972029\\
473.03771972777 -63.6709894536712\\
473.526000977772 -63.5539136302465\\
474.014282227773 -63.3875265659474\\
474.502563477774 -63.2047045441919\\
474.990844727775 -62.9505977032416\\
475.479125977777 -62.6635276996483\\
475.967407227778 -62.3757849734925\\
476.455688477779 -62.0539026515053\\
476.94396972778 -61.7047862744518\\
477.432250977782 -61.3463759478929\\
477.920532227783 -60.9119616066273\\
478.408813477784 -60.4333574042539\\
478.897094727785 -59.8922460939748\\
479.385375977787 -59.2150309086686\\
479.873657227788 -58.4183035525357\\
480.361938477789 -57.6258478054868\\
480.85021972779 -56.813428666359\\
481.338500977792 -55.9411259187794\\
481.826782227793 -54.9061864287465\\
482.315063477794 -53.682887467622\\
482.803344727795 -52.4419492396781\\
483.291625977797 -51.4101327948431\\
483.779907227798 -51.2622918884686\\
484.268188477799 -52.3457665046296\\
484.7564697278 -54.2631642341765\\
485.244750977802 -56.3191058985258\\
485.733032227803 -58.3229344809274\\
486.221313477804 -60.1922789505956\\
486.709594727805 -61.9530462290162\\
487.197875977807 -63.539119697432\\
487.686157227808 -65.0225973494305\\
488.174438477809 -66.4264117673389\\
488.66271972781 -67.7639529841301\\
489.151000977812 -69.114011942841\\
489.639282227813 -70.4351918880353\\
490.127563477814 -71.7182621048265\\
490.615844727815 -72.882072822806\\
491.104125977817 -74.0395789547055\\
491.592407227818 -75.2084237976216\\
492.080688477819 -76.3568377815157\\
492.56896972782 -77.3959904633593\\
493.057250977822 -78.1044013451051\\
493.545532227823 -78.0566400883827\\
494.033813477824 -76.651186849307\\
494.522094727825 -73.6890071129488\\
495.010375977827 -69.8629497936408\\
495.498657227828 -66.4036007305581\\
495.986938477829 -65.9765270832786\\
496.47521972783 -67.727334016455\\
496.963500977831 -69.5000050708374\\
497.451782227833 -70.9508389344168\\
497.940063477834 -72.0371376515113\\
498.428344727835 -72.7785983477173\\
498.916625977836 -73.2209015128978\\
499.404907227838 -73.4749030708159\\
499.893188477839 -73.6531234028456\\
499.984741212214 -73.6837613945906\\
};
\addlegendentry{LRM (validation)};

\addplot [LRMgrid, smallmarkers, forget plot] 
table[row sep=crcr]{
1.05285644531519 -8.51576097134915\\
1.1749267578155 -8.80836305133198\\
1.29699707031581 -9.85343079125266\\
1.41906738281612 -13.8117544809697\\
1.54113769531644 -22.7753768557106\\
1.66320800781675 -9.94919079190259\\
1.78527832031706 -8.47314560891255\\
1.90734863281737 -9.15334685306522\\
2.02941894531768 -8.50597619742143\\
2.15148925781799 -7.1515785847443\\
2.27355957031831 -9.19372570287112\\
2.39562988281862 -6.32364323334548\\
2.51770019531893 -5.99977992981235\\
2.63977050781924 -4.05377714555459\\
2.76184082031955 -5.62407559364448\\
2.88391113281986 -8.77426074495116\\
3.00598144532018 -8.52278306403554\\
3.12805175782049 -9.23913401580529\\
3.2501220703208 -8.56581619460292\\
3.37219238282111 -8.62361804201754\\
3.49426269532142 -7.88503487196971\\
3.61633300782173 -8.13314307491237\\
3.73840332032205 -6.74156468032538\\
3.86047363282236 -11.6126126881633\\
3.98254394532267 -27.2666170494525\\
4.10461425782298 -16.7547872276959\\
4.22668457032329 -6.66167654821032\\
4.3487548828236 -7.08055448178681\\
4.47082519532392 -7.0500443042223\\
4.59289550782423 -6.2031078669429\\
4.71496582032454 -5.60539546929709\\
4.83703613282485 -7.04654381658656\\
4.95910644532516 -7.42471692325489\\
5.08117675782547 -8.55350285186501\\
5.20324707032579 -6.35507655408827\\
5.3253173828261 -6.67867944071583\\
5.44738769532641 -13.9648086683378\\
5.56945800782672 -15.7161310320576\\
5.69152832032703 -10.6640946200989\\
5.81359863282735 -10.8307062300968\\
5.93566894532766 -8.30425561749416\\
6.05773925782797 -5.42696708464433\\
6.17980957032828 -9.02567374234013\\
6.30187988282859 -7.83623903877179\\
6.4239501953289 -9.50485211902964\\
6.54602050782922 -11.2693014228415\\
6.66809082032953 -10.5169201199893\\
6.79016113282984 -10.3134705943834\\
6.91223144533015 -11.1995885297395\\
7.03430175783046 -7.80341955645525\\
7.15637207033077 -4.81624877442033\\
7.27844238283109 -3.83617738680005\\
7.4005126953314 -4.16425770547897\\
7.52258300783171 -4.68008937914988\\
7.64465332033202 -5.62375271898077\\
7.76672363283233 -6.01695479863599\\
7.88879394533264 -6.12844468778331\\
8.01086425783296 -5.59364039591219\\
8.13293457033327 -4.96222710130439\\
8.25500488283358 -4.79734910737034\\
8.37707519533389 -4.12246094050909\\
8.4991455078342 -3.94275103955658\\
8.62121582033451 -4.07615674862063\\
8.74328613283483 -4.04669632755679\\
8.86535644533514 -4.44566060039841\\
8.98742675783545 -4.02143547940454\\
9.10949707033576 -4.37988335504161\\
9.23156738283607 -4.69329576736237\\
9.35363769533638 -4.72156786633502\\
9.4757080078367 -4.55081959249111\\
9.59777832033701 -4.62912347895025\\
9.71984863283732 -4.60041142779397\\
9.84191894533763 -4.7603906486741\\
9.96398925783794 -4.86195999932698\\
10.0860595703383 -5.06167736677708\\
10.2081298828386 -5.20872114546677\\
10.3302001953389 -5.36546408255634\\
10.4522705078392 -5.17453300419197\\
10.5743408203395 -5.01465753781764\\
10.6964111328398 -4.42947735817256\\
10.8184814453401 -4.21568083178255\\
10.9405517578404 -4.28310561620077\\
11.0626220703407 -4.26749108451889\\
11.1846923828411 -4.28685248109002\\
11.3067626953414 -4.59266038637833\\
11.4288330078417 -4.82742461460958\\
11.550903320342 -4.83328593094063\\
11.6729736328423 -4.66510376212335\\
11.7950439453426 -4.58475701977346\\
11.9171142578429 -4.81627913426053\\
12.0391845703432 -4.78268653034814\\
12.1612548828436 -5.12385190184955\\
12.2833251953439 -5.34142561315514\\
12.4053955078442 -5.4419604402392\\
12.5274658203445 -5.09730022921855\\
12.6495361328448 -5.08627166543528\\
12.7716064453451 -5.06307452466473\\
12.8936767578454 -5.1575760742586\\
13.0157470703457 -5.2204173596981\\
13.137817382846 -5.23481167163141\\
13.2598876953464 -5.22203315065661\\
13.3819580078467 -5.12967875060843\\
13.504028320347 -5.11845784392619\\
13.6260986328473 -5.21981389048068\\
13.7481689453476 -5.27018467453956\\
13.8702392578479 -5.28541107737703\\
13.9923095703482 -5.2832831583184\\
14.1143798828485 -5.29359694981531\\
14.2364501953489 -5.24956558067282\\
14.3585205078492 -5.27594867991246\\
14.4805908203495 -5.43726196953344\\
14.6026611328498 -5.39493553828316\\
14.7247314453501 -5.41319933299849\\
14.8468017578504 -5.38647668020172\\
14.9688720703507 -5.37553725565994\\
15.090942382851 -5.38114687195622\\
15.2130126953513 -5.39037153790906\\
15.3350830078517 -5.48184277093742\\
15.457153320352 -5.51181374816366\\
15.5792236328523 -5.5751008048195\\
15.7012939453526 -5.58364149487369\\
15.8233642578529 -5.65619398147703\\
15.9454345703532 -5.65758809773024\\
16.0675048828535 -5.66203644991543\\
16.1895751953538 -5.73366953147234\\
16.3116455078542 -5.73964778557212\\
16.4337158203545 -5.76752423311365\\
16.5557861328548 -5.77433346677844\\
16.6778564453551 -5.79503962596715\\
16.7999267578554 -5.80841278120261\\
16.9219970703557 -5.86951510298081\\
17.044067382856 -5.87767728968151\\
17.1661376953563 -5.92487604686454\\
17.2882080078566 -5.93500261818764\\
17.410278320357 -5.95015087954971\\
17.5323486328573 -5.93544209255066\\
17.6544189453576 -5.93459262007315\\
17.7764892578579 -5.98297184298929\\
17.8985595703582 -6.07618921041893\\
18.0206298828585 -6.15022048278979\\
18.1427001953588 -6.16299830993887\\
18.2647705078591 -6.19764594422691\\
18.3868408203594 -6.27523596865552\\
18.5089111328598 -6.25966576756372\\
18.6309814453601 -6.27874720242085\\
18.7530517578604 -6.30396396029499\\
18.8751220703607 -6.34526601892941\\
18.997192382861 -6.37578343397126\\
19.1192626953613 -6.39351807844633\\
19.2413330078616 -6.41077816787953\\
19.3634033203619 -6.42262779289177\\
19.4854736328623 -6.4617686458343\\
19.6075439453626 -6.47081567966717\\
19.7296142578629 -6.48270834228151\\
19.8516845703632 -6.51298944363157\\
19.9737548828635 -6.53214687241672\\
20.0958251953638 -6.56335879516269\\
20.2178955078641 -6.59208026896772\\
20.3399658203644 -6.61436229531341\\
20.4620361328647 -6.64859443282643\\
20.5841064453651 -6.67294093280316\\
20.7061767578654 -6.70291686291387\\
20.8282470703657 -6.73815098434216\\
20.950317382866 -6.76363729386759\\
21.0723876953663 -6.79033148825164\\
21.1944580078666 -6.81910296578411\\
21.3165283203669 -6.85216619198837\\
21.4385986328672 -6.85041525583068\\
21.5606689453676 -6.88830424314835\\
21.6827392578679 -6.92386064878878\\
21.8048095703682 -6.95383241417943\\
21.9268798828685 -6.96710178148118\\
22.0489501953688 -6.96624011618906\\
22.1710205078691 -6.97552072907513\\
22.2930908203694 -6.99319834954144\\
22.4151611328697 -7.00786546745991\\
22.5372314453701 -7.03874690669443\\
22.6593017578704 -7.07005020413806\\
22.7813720703707 -7.11124924815931\\
22.903442382871 -7.18471660784684\\
23.0255126953713 -7.20555902556129\\
23.1475830078716 -7.21810152830574\\
23.2696533203719 -7.25875804271851\\
23.3917236328722 -7.27013782266272\\
23.5137939453725 -7.26693324270684\\
23.6358642578729 -7.27527434781604\\
23.7579345703732 -7.30187347153327\\
23.8800048828735 -7.32560492335136\\
24.0020751953738 -7.35425862093808\\
24.1241455078741 -7.38337581487656\\
24.2462158203744 -7.40547115583149\\
24.3682861328747 -7.41505668716849\\
24.490356445375 -7.45067151420488\\
24.6124267578753 -7.5020090612677\\
24.7344970703757 -7.52403924520758\\
24.856567382876 -7.55056042356534\\
24.9786376953763 -7.58651618450386\\
25.1007080078766 -7.61929602994184\\
25.2227783203769 -7.65118682718048\\
25.3448486328772 -7.67452227306046\\
25.4669189453775 -7.70120709155469\\
25.5889892578778 -7.73380524484207\\
25.7110595703782 -7.74968018466848\\
25.8331298828785 -7.78660579040201\\
25.9552001953788 -7.82160524862212\\
26.0772705078791 -7.8526032124729\\
26.1993408203794 -7.88428014902627\\
26.3214111328797 -7.92884148288783\\
26.44348144538 -7.94483909538548\\
26.5655517578803 -8.0114881577814\\
26.6876220703806 -8.03812676879625\\
26.809692382881 -8.07887118632868\\
26.9317626953813 -8.13128410038672\\
27.0538330078816 -8.19155188139416\\
27.1759033203819 -8.21545187890206\\
27.2979736328822 -8.23056336368739\\
27.4200439453825 -8.2560584842513\\
27.5421142578828 -8.27157345234065\\
27.6641845703831 -8.29433929462471\\
27.7862548828835 -8.32605954526872\\
27.9083251953838 -8.36329694456589\\
28.0303955078841 -8.40181614966411\\
28.1524658203844 -8.43935752561515\\
28.2745361328847 -8.46513155439686\\
28.396606445385 -8.47714891882168\\
28.5186767578853 -8.51052115265219\\
28.6407470703856 -8.54160128638858\\
28.7628173828859 -8.57518065791214\\
28.8848876953863 -8.60473493387951\\
29.0069580078866 -8.62316951160005\\
29.1290283203869 -8.66232383774059\\
29.2510986328872 -8.69473146008664\\
29.3731689453875 -8.71109502236254\\
29.4952392578878 -8.76725985346627\\
29.6173095703881 -8.79610967648807\\
29.7393798828884 -8.82469489569382\\
29.8614501953888 -8.85228345305296\\
29.9835205078891 -8.88842978461759\\
30.1055908203894 -8.93765189753526\\
30.2276611328897 -8.98412269385915\\
30.34973144539 -9.01189339773958\\
30.4718017578903 -8.98395338850139\\
30.5938720703906 -9.03044988992127\\
30.7159423828909 -9.05529484241015\\
30.8380126953912 -9.06516859701355\\
30.9600830078916 -9.10290781309658\\
31.0821533203919 -9.12337382649906\\
31.2042236328922 -9.1596261864259\\
31.3262939453925 -9.17334774558583\\
31.4483642578928 -9.18599102016594\\
31.5704345703931 -9.20444870856255\\
31.6925048828934 -9.23178440895998\\
31.8145751953937 -9.26178839980344\\
31.936645507894 -9.28945824099009\\
32.0587158203944 -9.31871727176974\\
32.1807861328947 -9.34430649520721\\
32.302856445395 -9.4002511028508\\
32.4249267578953 -9.42785391730638\\
32.5469970703956 -9.46982187231367\\
32.6690673828959 -9.49782285763507\\
32.7911376953962 -9.52092834773623\\
32.9132080078965 -9.5396467754066\\
33.0352783203969 -9.55899428067863\\
33.1573486328972 -9.58476404901569\\
33.2794189453975 -9.58972436080523\\
33.4014892578978 -9.62565909098112\\
33.5235595703981 -9.64407169182306\\
33.6456298828984 -9.65503614388012\\
33.7677001953987 -9.67110434330641\\
33.889770507899 -9.70697405110377\\
34.0118408203993 -9.72880299088195\\
34.1339111328997 -9.75089804270493\\
34.2559814454 -9.7804382262272\\
34.3780517579003 -9.82035865476996\\
34.5001220704006 -9.84060859448385\\
34.6221923829009 -9.8522096933051\\
34.7442626954012 -9.86340435985062\\
34.8663330079015 -9.87899593798141\\
34.9884033204018 -9.89622993807853\\
35.1104736329022 -9.91770988073728\\
35.2325439454025 -9.94201734516275\\
35.3546142579028 -9.97374754115356\\
35.4766845704031 -10.0060127338825\\
35.5987548829034 -10.0332059984017\\
35.7208251954037 -10.0674416344402\\
35.842895507904 -10.0968374851338\\
35.9649658204043 -10.1260099762108\\
36.0870361329047 -10.1482748907385\\
36.209106445405 -10.1708244826144\\
36.3311767579053 -10.1949525425665\\
36.4532470704056 -10.212383231758\\
36.5753173829059 -10.2308632857515\\
36.6973876954062 -10.2595010562511\\
36.8194580079065 -10.2775271197324\\
36.9415283204068 -10.2916477857423\\
37.0635986329071 -10.3147704468246\\
37.1856689454074 -10.3261487760549\\
37.3077392579078 -10.3556874128472\\
37.4298095704081 -10.3823613149962\\
37.5518798829084 -10.3979894862377\\
37.6739501954087 -10.4447453525028\\
37.796020507909 -10.4759600835467\\
37.9180908204093 -10.506963508781\\
38.0401611329096 -10.531255945401\\
38.1622314454099 -10.5434559993417\\
38.2843017579103 -10.5703073442768\\
38.4063720704106 -10.5843359776781\\
38.5284423829109 -10.6028823688971\\
38.6505126954112 -10.6234537133569\\
38.7725830079115 -10.6539165714154\\
38.8946533204118 -10.6791679501331\\
39.0167236329121 -10.7049080549742\\
39.1387939454124 -10.7330181602226\\
39.2608642579128 -10.7557033669368\\
39.3829345704131 -10.7874095107624\\
39.5050048829134 -10.8000501765279\\
39.6270751954137 -10.8292649582384\\
39.749145507914 -10.850581265283\\
39.8712158204143 -10.880984954186\\
39.9932861329146 -10.9097355105217\\
40.1153564454149 -10.9376509958957\\
40.2374267579152 -10.9607260666051\\
40.3594970704156 -10.9858760862234\\
40.4815673829159 -11.0147866403839\\
40.6036376954162 -11.0408181969863\\
40.7257080079165 -11.0597534170495\\
40.8477783204168 -11.0860227765451\\
40.9698486329171 -11.108264466877\\
41.0919189454174 -11.1315881432155\\
41.2139892579177 -11.14382303043\\
41.3360595704181 -11.1613749092668\\
41.4581298829184 -11.1818249011629\\
41.5802001954187 -11.2005282081182\\
41.702270507919 -11.2253271007024\\
41.8243408204193 -11.2498094817722\\
41.9464111329196 -11.2768376951478\\
42.0684814454199 -11.30612559203\\
42.1905517579202 -11.3178104115345\\
42.3126220704205 -11.3285941801078\\
42.4346923829209 -11.3547942262321\\
42.5567626954212 -11.3679881283505\\
42.6788330079215 -11.3850720149459\\
42.8009033204218 -11.4093164336546\\
42.9229736329221 -11.438038412197\\
43.0450439454224 -11.4513105529772\\
43.1671142579227 -11.5132021724157\\
43.289184570423 -11.5402481275597\\
43.4112548829234 -11.5599291062051\\
43.5333251954237 -11.5934554350618\\
43.655395507924 -11.5945362890669\\
43.7774658204243 -11.5809604645111\\
43.8995361329246 -11.5850258724599\\
44.0216064454249 -11.5781859252976\\
44.1436767579252 -11.6177455849402\\
44.2657470704255 -11.6462942029872\\
44.3878173829258 -11.6799842092896\\
44.5098876954262 -11.7050343736951\\
44.6319580079265 -11.7159729274833\\
44.7540283204268 -11.7212326847692\\
44.8760986329271 -11.7351278281803\\
44.9981689454274 -11.7441373717256\\
45.1202392579277 -11.7611762849793\\
45.242309570428 -11.768831727596\\
45.3643798829283 -11.7780490565255\\
45.4864501954287 -11.8167491028198\\
45.608520507929 -11.8419490184376\\
45.7305908204293 -11.8643155932216\\
45.9136962891797 -11.8979906626378\\
46.1578369141804 -11.9280790570797\\
46.401977539181 -11.971095195106\\
46.6461181641816 -11.9996489294135\\
46.8902587891822 -12.031313215443\\
47.1343994141829 -12.0612197946974\\
47.3785400391835 -12.0886275783459\\
47.6226806641841 -12.1232382212079\\
47.8668212891847 -12.1583941382345\\
48.1109619141854 -12.1943129787389\\
48.355102539186 -12.2272711903869\\
48.5992431641866 -12.2654062148615\\
48.8433837891872 -12.302905620239\\
49.0875244141878 -12.3378114471634\\
49.3316650391885 -12.3790908780301\\
49.5758056641891 -12.4185808655438\\
49.8199462891897 -12.4590348806585\\
50.0640869141903 -12.4855446574752\\
50.308227539191 -12.5053448832214\\
50.5523681641916 -12.5320417902817\\
50.7965087891922 -12.5648490676892\\
51.0406494141928 -12.589691862358\\
51.2847900391935 -12.6266215793119\\
51.5289306641941 -12.6500822065133\\
51.7730712891947 -12.6756636612954\\
52.0172119141953 -12.7121676203339\\
52.2613525391959 -12.735468034118\\
52.5054931641966 -12.7606986541798\\
52.7496337891972 -12.7812096996562\\
52.9937744141978 -12.8061385510255\\
53.2379150391984 -12.8298626142271\\
53.4820556641991 -12.8545109018493\\
53.7261962891997 -12.8770167459722\\
53.9703369142003 -12.9134917239885\\
54.2144775392009 -12.9394945760832\\
54.4586181642016 -12.9619319573303\\
54.7027587892022 -13.0063740507482\\
54.9468994142028 -13.0293220886471\\
55.1910400392034 -13.0539219384219\\
55.4351806642041 -13.0659029575955\\
55.6793212892047 -13.0952332578508\\
55.9234619142053 -13.1254060089631\\
56.1676025392059 -13.1400656540417\\
56.4117431642065 -13.1593725015852\\
56.6558837892072 -13.1884706145853\\
56.9000244142078 -13.21724542415\\
57.1441650392084 -13.2293298369847\\
57.388305664209 -13.279742839489\\
57.6324462892097 -13.3129960883411\\
57.8765869142103 -13.3405177198492\\
58.1207275392109 -13.3751691787765\\
58.3648681642115 -13.4201018011978\\
58.6090087892122 -13.4699022089007\\
58.8531494142128 -13.487796882946\\
59.0972900392134 -13.5203493387854\\
59.341430664214 -13.5552348127945\\
59.5855712892146 -13.6022779087119\\
59.8297119142153 -13.6268959941415\\
60.0738525392159 -13.6589058081718\\
60.3179931642165 -13.7023165421865\\
60.5621337892171 -13.7589656157352\\
60.8062744142178 -13.7845702090211\\
61.0504150392184 -13.8455220477695\\
61.294555664219 -13.8867541084195\\
61.5386962892196 -13.9451581116606\\
61.7828369142203 -13.9897841379583\\
62.0269775392209 -14.0367814238261\\
62.2711181642215 -14.0802078803511\\
62.5152587892221 -14.1310749644863\\
62.7593994142228 -14.1646294270885\\
63.0035400392234 -14.2228450024671\\
63.247680664224 -14.2653120979304\\
63.4918212892246 -14.2901026229988\\
63.7359619142252 -14.3572881608952\\
63.9801025392259 -14.4249225960768\\
64.2242431642265 -14.4578988378151\\
64.4683837892271 -14.4837126701187\\
64.7125244142277 -14.5206462391121\\
64.9566650392284 -14.5724522065217\\
65.200805664229 -14.5936275211816\\
65.4449462892296 -14.6302143405498\\
65.6890869142302 -14.6603252670405\\
65.9332275392309 -14.7018874145224\\
66.1773681642315 -14.7272270442783\\
66.4215087892321 -14.7706246246545\\
66.6656494142327 -14.8133692837657\\
66.9097900392334 -14.8554740392741\\
67.153930664234 -14.8939612468342\\
67.3980712892346 -14.9491270368534\\
67.6422119142352 -14.9820469471887\\
67.8863525392358 -15.0157455190404\\
68.1304931642365 -15.0492670455551\\
68.3746337892371 -15.0732906316472\\
68.6187744142377 -15.1047773654095\\
68.8629150392383 -15.1276165233227\\
69.107055664239 -15.1569520843676\\
69.3511962892396 -15.1817681393895\\
69.5953369142402 -15.2033358659118\\
69.8394775392408 -15.2350733902838\\
70.0836181642415 -15.2643805441925\\
70.3277587892421 -15.2893204729049\\
70.5718994142427 -15.3398793640504\\
70.8160400392433 -15.3824081428069\\
71.060180664244 -15.4253963058142\\
71.3043212892446 -15.5114061476439\\
71.5484619142452 -15.5499165327136\\
71.7926025392458 -15.5860641655632\\
72.0367431642464 -15.6108332629407\\
72.2808837892471 -15.6343207940886\\
72.5250244142477 -15.6605363386944\\
72.7691650392483 -15.6805122474118\\
73.0133056642489 -15.7307120696834\\
73.2574462892496 -15.7666885104662\\
73.5015869142502 -15.8151905727219\\
73.7457275392508 -15.846830193141\\
73.9898681642514 -15.8947002480303\\
74.2340087892521 -15.9293957075542\\
74.4781494142527 -15.948479074061\\
74.7222900392533 -15.9655673724736\\
74.9664306642539 -15.9940186466958\\
75.2105712892545 -16.0212329401296\\
75.4547119142552 -16.0512676855917\\
75.6988525392558 -16.0766031682368\\
75.9429931642564 -16.1073940449863\\
76.187133789257 -16.1428124879014\\
76.4923095705078 -16.1885509118455\\
76.8585205080088 -16.235261492302\\
77.2247314455097 -16.2839394202537\\
77.5909423830106 -16.328010558998\\
77.9571533205116 -16.3800261345598\\
78.3233642580125 -16.4214389729913\\
78.6895751955134 -16.4750732608747\\
79.0557861330144 -16.5148215589581\\
79.4219970705153 -16.5655012043409\\
79.7882080080162 -16.6239851803776\\
80.1544189455172 -16.6570179047316\\
80.5206298830181 -16.6943130080413\\
80.886840820519 -16.7377318755518\\
81.25305175802 -16.7874063636013\\
81.6192626955209 -16.8380868801473\\
81.9854736330219 -16.9015169215149\\
82.3516845705228 -16.9574244422077\\
82.7178955080237 -16.9916775532299\\
83.0841064455247 -17.0484249484001\\
83.4503173830256 -17.0695182606183\\
83.8165283205265 -17.0969331852074\\
84.1827392580275 -17.1233760833584\\
84.5489501955284 -17.1511310410253\\
84.9151611330293 -17.1908071855618\\
85.2813720705303 -17.2185063793597\\
85.6475830080312 -17.2485667315942\\
86.0137939455321 -17.2744310291739\\
86.3800048830331 -17.28725995204\\
86.746215820534 -17.3232235312054\\
87.1124267580349 -17.3387189006111\\
87.4786376955359 -17.3611798884928\\
87.8448486330368 -17.3693345027049\\
88.2110595705377 -17.4152864167249\\
88.5772705080387 -17.471294708533\\
88.9434814455396 -17.5159814997673\\
89.3096923830406 -17.539192865477\\
89.6759033205415 -17.5717332102335\\
90.0421142580424 -17.5763612814529\\
90.4083251955434 -17.590529377258\\
90.7745361330443 -17.6001169183938\\
91.1407470705452 -17.6273233018451\\
91.5069580080462 -17.6469702608397\\
91.8731689455471 -17.6684503568407\\
92.239379883048 -17.6847710634066\\
92.605590820549 -17.7282243547195\\
92.9718017580499 -17.7596157402754\\
93.3380126955508 -17.8163082551754\\
93.7042236330518 -17.8918189807426\\
94.0704345705527 -17.973110043996\\
94.4366455080536 -18.0596311902951\\
94.8028564455546 -18.1582482644926\\
95.1690673830555 -18.271020742667\\
95.5352783205564 -18.3469237685431\\
95.9014892580574 -18.4255582771544\\
96.2677001955583 -18.5208466467046\\
96.6339111330593 -18.6164988480861\\
97.0001220705602 -18.7016556478789\\
97.3663330080611 -18.7820298538266\\
97.7325439455621 -18.8655890426424\\
98.098754883063 -18.9600998179777\\
98.4649658205639 -19.0487890506876\\
98.8311767580649 -19.1542190874893\\
99.1973876955658 -19.2975542019277\\
99.5635986330667 -19.3965046488655\\
99.9298095705677 -19.5088239761856\\
100.296020508069 -19.5684195496639\\
100.66223144557 -19.5968526769053\\
101.02844238307 -19.5494490868379\\
101.394653320571 -19.4894441252187\\
101.760864258072 -19.5066616479616\\
102.127075195573 -19.7494890435815\\
102.493286133074 -20.0316009191341\\
102.859497070575 -20.2827403715308\\
103.225708008076 -20.4858522279707\\
103.591918945577 -20.6675996729513\\
103.958129883078 -20.8225213137449\\
104.324340820579 -20.995198529678\\
104.69055175808 -21.1661116721619\\
105.056762695581 -21.384456231936\\
105.422973633082 -21.7674467556047\\
105.789184570583 -22.345505525746\\
106.155395508084 -22.9844587293729\\
106.521606445584 -23.3010000792764\\
106.887817383085 -23.1336893087445\\
107.376098633087 -22.6837422728871\\
107.864379883088 -22.2577341509295\\
108.352661133089 -21.9647348489229\\
108.84094238309 -21.7647281663419\\
109.329223633092 -21.7133509077964\\
109.817504883093 -21.8525932066243\\
110.305786133094 -22.0453739441085\\
110.794067383095 -22.2302142557746\\
111.282348633097 -22.3154808448921\\
111.770629883098 -22.3979790687504\\
112.258911133099 -22.3317609634594\\
112.7471923831 -22.353549263288\\
113.235473633102 -22.0997951593312\\
113.723754883103 -21.6426543286258\\
114.212036133104 -21.3446370105663\\
114.700317383105 -20.2945549606548\\
115.188598633107 -18.9025524621205\\
115.676879883108 -18.7822391969878\\
116.165161133109 -19.5991278173388\\
116.65344238311 -20.6869754720283\\
117.141723633112 -20.4514814445391\\
117.630004883113 -18.7604387916593\\
118.118286133114 -18.6749016923984\\
118.606567383115 -19.0184978170887\\
119.094848633117 -19.016796824432\\
119.583129883118 -19.4453748430478\\
120.071411133119 -20.9263699315564\\
120.55969238312 -23.2992949330852\\
121.047973633122 -24.7924869185311\\
121.536254883123 -21.3595338887733\\
122.024536133124 -18.6663825199704\\
122.512817383125 -19.9604846906511\\
123.001098633127 -20.7598764833216\\
123.489379883128 -21.2307274033964\\
123.977661133129 -21.9851264248466\\
124.46594238313 -22.6210565341858\\
124.954223633132 -23.3191369798795\\
125.442504883133 -24.7257512332768\\
125.930786133134 -26.0582282960076\\
126.419067383135 -26.3169376734624\\
126.907348633137 -26.0489045325028\\
127.395629883138 -25.8635433655129\\
127.883911133139 -25.4115906653185\\
128.37219238314 -21.4543249374923\\
128.860473633142 -17.0945358044682\\
129.348754883143 -14.4661517113895\\
129.837036133144 -15.7104618861338\\
130.325317383145 -16.979265896102\\
130.813598633147 -17.4421851617484\\
131.301879883148 -17.2638233906018\\
131.790161133149 -16.8322300354481\\
132.27844238315 -16.2671680001841\\
132.766723633152 -15.1177811293346\\
133.255004883153 -13.5154247578537\\
133.743286133154 -11.2993293181626\\
134.231567383155 -8.91842380684835\\
134.719848633157 -7.89126635042948\\
135.208129883158 -6.54543498145841\\
135.696411133159 -7.8923813403004\\
136.18469238316 -9.90591607028745\\
136.672973633161 -11.6245174946401\\
137.161254883163 -12.7102525239463\\
137.710571289414 -13.5653512920031\\
138.320922851916 -14.4499502715136\\
138.931274414417 -15.018437445642\\
139.541625976919 -15.5152206852844\\
140.15197753942 -15.9346022949631\\
140.762329101922 -16.345302336149\\
141.372680664423 -16.7360516862355\\
141.983032226925 -17.1147166829309\\
142.593383789427 -17.4710051459452\\
143.203735351928 -17.7953275422284\\
143.81408691443 -18.082694419614\\
144.424438476931 -18.3545264173541\\
145.034790039433 -18.6033517667123\\
145.645141601934 -18.8285342732023\\
146.255493164436 -19.0393596451459\\
146.865844726938 -19.2417581250173\\
147.476196289439 -19.4259185802466\\
148.086547851941 -19.5998633648501\\
148.696899414442 -19.767394203946\\
149.307250976944 -19.9274719500316\\
149.917602539445 -20.073075031541\\
150.527954101947 -20.2090280979238\\
151.138305664448 -20.3544053267726\\
151.74865722695 -20.498018295905\\
152.359008789452 -20.6136450639827\\
152.969360351953 -20.7284360555846\\
153.579711914455 -20.8422746738283\\
154.190063476956 -20.9459104884038\\
154.800415039458 -21.0471094723686\\
155.410766601959 -21.1418220674111\\
156.021118164461 -21.2318480487808\\
156.631469726962 -21.3202608183115\\
157.241821289464 -21.4100159513506\\
157.852172851966 -21.4832089503172\\
158.462524414467 -21.5536469078233\\
159.072875976969 -21.6243552501958\\
159.68322753947 -21.6920996133793\\
160.293579101972 -21.7857071304447\\
160.903930664473 -21.9651303691575\\
161.514282226975 -22.2483235015715\\
162.124633789476 -22.6034860459203\\
162.734985351978 -22.8326409517392\\
163.34533691448 -22.9463358341472\\
163.955688476981 -22.9966504215059\\
164.566040039483 -23.0341900293068\\
165.176391601984 -23.0864678802511\\
165.786743164486 -23.1165195030112\\
166.397094726987 -23.1565574761227\\
167.007446289489 -23.209278176061\\
167.617797851991 -23.2691859223546\\
168.289184570742 -23.3278710513551\\
169.021606445744 -23.3956431632985\\
169.754028320746 -23.4590072529347\\
170.486450195748 -23.5371909655469\\
171.21887207075 -23.6075684495858\\
171.951293945752 -23.6722384272707\\
172.683715820753 -23.7345389918314\\
173.416137695755 -23.8111671240904\\
174.148559570757 -23.9012519297152\\
174.880981445759 -23.9822251914932\\
175.613403320761 -24.0872314637091\\
176.345825195763 -24.188601161868\\
177.078247070765 -24.3093001687509\\
177.810668945767 -24.4291846791898\\
178.543090820768 -24.5251861080575\\
179.27551269577 -24.6018325896032\\
180.007934570772 -24.6643296187889\\
180.740356445774 -24.7297039791481\\
181.472778320776 -24.7784459590712\\
182.205200195778 -24.8492921003861\\
182.93762207078 -24.9436305009935\\
183.670043945781 -25.1098257869149\\
184.402465820783 -25.3811686145922\\
185.134887695785 -25.6152026270997\\
185.867309570787 -25.6691933878707\\
186.599731445789 -25.6479465524724\\
187.332153320791 -25.6067359905165\\
188.064575195793 -25.5779213680718\\
188.796997070795 -25.5769352550098\\
189.529418945796 -25.5984172777089\\
190.261840820798 -25.6218981584684\\
190.9942626958 -25.6502075519244\\
191.726684570802 -25.6974502191956\\
192.459106445804 -25.7465695162502\\
193.191528320806 -25.7897463341433\\
193.923950195808 -25.8382749149932\\
194.65637207081 -25.8798573494556\\
195.388793945811 -25.9192611471562\\
196.121215820813 -25.9605400059189\\
196.853637695815 -26.0093779836996\\
197.586059570817 -26.0504215702568\\
198.318481445819 -26.0973930591932\\
199.142456055196 -26.1464067786864\\
199.996948242698 -26.1995493304507\\
200.8514404302 -26.2622953698414\\
201.705932617703 -26.3162142983711\\
202.560424805205 -26.356504409863\\
203.414916992707 -26.4209911614714\\
204.269409180209 -26.5179032535267\\
205.123901367711 -26.6751846988183\\
205.978393555213 -26.9577022575691\\
206.832885742716 -27.3376052321834\\
207.687377930218 -27.5425075734097\\
208.54187011772 -27.5631325021894\\
209.396362305222 -27.5124181606417\\
210.250854492724 -27.4270785942186\\
211.105346680227 -27.3538514690075\\
211.959838867729 -27.2780411540343\\
212.814331055231 -27.2233186887371\\
213.668823242733 -27.1780991952243\\
214.523315430235 -27.1839446965329\\
215.377807617737 -27.2937691857896\\
216.23229980524 -27.5401043096953\\
217.086791992742 -27.9028175305552\\
217.941284180244 -28.2486458654174\\
218.795776367746 -28.5147377860802\\
219.650268555248 -28.6706993164482\\
220.504760742751 -28.7995647883176\\
221.359252930253 -29.3842368429732\\
222.213745117755 -29.9243205010225\\
223.068237305257 -30.9223301310067\\
223.922729492759 -33.4929447454464\\
224.777221680261 -30.3827291405161\\
225.631713867764 -23.6698939579992\\
226.486206055266 -24.1430085824882\\
227.340698242768 -24.2669725556808\\
228.19519043027 -24.2388936404962\\
229.049682617772 -24.0172052461077\\
230.026245117775 -22.8129333070322\\
231.002807617777 -22.2854218237372\\
231.97937011778 -29.8526915246967\\
232.955932617782 -48.8138413967633\\
233.932495117785 -39.8270824395761\\
234.909057617787 -35.8099247900682\\
235.88562011779 -34.4598508674441\\
236.862182617792 -33.4905426365942\\
237.838745117795 -32.9454006945798\\
238.815307617797 -32.7346942242527\\
239.7918701178 -32.3969874389171\\
240.768432617802 -32.1515675902702\\
241.744995117805 -31.9815429418386\\
242.721557617807 -31.8525276332891\\
243.69812011781 -31.7656684622119\\
244.674682617812 -31.7210611659652\\
245.651245117815 -31.6866139071536\\
246.627807617817 -31.6644483346627\\
247.60437011782 -31.6518118391423\\
248.580932617822 -31.6357315725442\\
249.557495117825 -31.6239305472731\\
250.534057617827 -31.595717054836\\
251.51062011783 -31.6103363609421\\
252.487182617832 -31.6714091687634\\
253.463745117835 -31.7629611755192\\
254.440307617837 -31.8889543787424\\
255.41687011784 -32.0662009281048\\
256.393432617842 -32.2321350732615\\
257.369995117845 -32.4078409221811\\
258.346557617847 -32.5997553185717\\
259.32312011785 -32.784010107514\\
260.391235352227 -32.9013758177339\\
261.48986816473 -32.7029816545535\\
262.588500977233 -32.1144494554525\\
263.687133789736 -31.5269403145758\\
264.785766602239 -31.4152507194275\\
265.884399414741 -31.4921154396958\\
266.983032227244 -31.6484743864946\\
268.081665039747 -31.8132014244562\\
269.18029785225 -31.8717654673752\\
270.278930664753 -31.7110673788251\\
271.377563477255 -31.3964192936616\\
272.476196289758 -31.0890719180339\\
273.574829102261 -30.7086695111429\\
274.673461914764 -30.338730331478\\
275.772094727267 -30.366511225513\\
276.870727539769 -30.9267457946188\\
277.969360352272 -31.8907010710358\\
279.067993164775 -32.7716772681902\\
280.166625977278 -33.3095177763669\\
281.265258789781 -33.430999912261\\
282.363891602284 -33.4419619657737\\
283.462524414786 -33.3506366627876\\
284.561157227289 -33.2107344257165\\
285.659790039792 -32.9488948295862\\
286.758422852295 -32.6434237840134\\
287.857055664798 -32.3877033827602\\
288.9556884773 -32.0546271516158\\
290.054321289803 -31.7448426246463\\
291.275024414806 -31.2986199772736\\
292.495727539809 -30.7666570717087\\
293.716430664812 -29.6579614801189\\
294.937133789816 -26.7636303302141\\
296.157836914819 -15.2492693679422\\
297.378540039822 -31.671532157827\\
298.599243164825 -33.4744949085233\\
299.819946289828 -33.5185881353569\\
301.040649414831 -33.3571027260049\\
302.261352539834 -33.1889481446758\\
303.482055664837 -33.0216854940875\\
304.702758789841 -32.8436321355658\\
305.923461914844 -32.6557313494504\\
307.144165039847 -32.4244458434748\\
308.36486816485 -32.2049538611385\\
309.585571289853 -31.9358842013762\\
310.806274414856 -31.6841353650545\\
312.026977539859 -31.501198735649\\
313.247680664862 -31.3960798551599\\
314.468383789865 -31.3756935277722\\
315.689086914869 -31.0984951532329\\
316.909790039872 -30.5855187602214\\
318.130493164875 -29.8825773036885\\
319.351196289878 -29.0725787978221\\
320.571899414881 -27.9365910417285\\
321.914672852385 -26.9046171587165\\
323.257446289888 -25.3923599139098\\
324.600219727391 -23.0939841360213\\
325.942993164895 -19.3261953852889\\
327.285766602398 -11.4070238335127\\
328.628540039902 -9.88163239439484\\
329.971313477405 -19.5823042211181\\
331.314086914909 -24.1251469740183\\
332.656860352412 -26.8792181259669\\
333.999633789915 -28.4219340989963\\
335.342407227419 -29.2787151406842\\
336.685180664922 -29.4753645208183\\
338.027954102426 -29.2262078140409\\
339.370727539929 -28.7536013314246\\
340.713500977432 -28.0561764647846\\
342.056274414936 -27.0542418472117\\
343.399047852439 -25.8019550155767\\
344.741821289943 -24.3274212920937\\
346.084594727446 -23.0138218272038\\
347.42736816495 -23.0613652168039\\
348.770141602453 -24.8199308643241\\
350.112915039957 -27.1855909104627\\
351.486206055585 -29.5095631331017\\
352.951049805589 -31.5977637620872\\
354.415893555593 -33.3275782416762\\
355.880737305596 -34.6737132523814\\
357.3455810556 -35.6821360168129\\
358.810424805604 -36.32600907046\\
360.275268555607 -36.6641176430207\\
361.740112305611 -36.7187466298451\\
363.204956055615 -36.543984361539\\
364.669799805619 -36.1955495580502\\
366.134643555622 -35.3020387985083\\
367.599487305626 -34.015422113019\\
369.06433105563 -32.5443794310504\\
370.529174805634 -31.2515887633916\\
371.994018555637 -29.7964232376459\\
373.458862305641 -27.9985795947493\\
374.923706055645 -25.6796639157662\\
376.388549805649 -22.364422154722\\
377.853393555652 -18.1332335049642\\
379.318237305656 -18.7966970335532\\
380.78308105566 -23.8954984583214\\
382.308959961914 -28.3788404879247\\
383.895874024418 -32.7068023906182\\
385.482788086922 -37.8651630047887\\
387.069702149426 -46.4838283842212\\
388.65661621193 -39.6462474043873\\
390.243530274434 -24.3465468097297\\
391.830444336938 -26.6115920096304\\
393.417358399442 -31.1938045840959\\
395.004272461946 -33.6308919219251\\
396.59118652445 -35.2400803643467\\
398.178100586954 -36.541476877774\\
399.765014649458 -37.383724285665\\
401.351928711962 -38.1760594203737\\
402.938842774466 -38.7953096589441\\
404.52575683697 -39.1595952234936\\
406.112670899474 -39.3007670747762\\
407.699584961979 -39.3996372146033\\
409.286499024483 -39.1682290146481\\
410.873413086987 -38.5289676135328\\
412.490844727616 -37.0794617379519\\
414.19982910262 -40.4428575271993\\
415.908813477625 -54.5669430180257\\
417.617797852629 -52.7240288547026\\
419.326782227633 -50.4699376298143\\
421.035766602638 -49.535599291053\\
422.744750977642 -49.1575507701215\\
424.453735352646 -48.7158195049047\\
426.162719727651 -48.477111052245\\
427.871704102655 -48.5250778346721\\
429.580688477659 -48.3686529653903\\
431.289672852664 -48.1158054793547\\
432.998657227668 -47.7521732115312\\
434.707641602673 -47.3244511371692\\
436.416625977677 -46.768145300618\\
438.125610352681 -46.0302835837447\\
439.834594727686 -45.0415059194616\\
441.54357910269 -43.6920124556802\\
443.283081055819 -41.5168122928759\\
445.114135743324 -37.2771422846517\\
446.945190430829 -31.6455003661985\\
448.776245118333 -39.4123730990357\\
450.607299805838 -45.9215337158108\\
452.438354493343 -50.5963685966063\\
454.269409180847 -54.1497895568124\\
456.100463868352 -57.2391279889476\\
457.931518555857 -59.8177645696367\\
459.762573243361 -60.8563573299529\\
461.593627930866 -61.4674901426064\\
463.424682618371 -62.2689956803014\\
465.255737305876 -62.6881806000906\\
467.08679199338 -63.2032029988667\\
468.917846680885 -63.5518419067061\\
470.74890136839 -63.8631623699693\\
472.579956055894 -63.7176311320205\\
474.502563477774 -63.1359839365038\\
476.455688477779 -61.9648238538173\\
478.408813477784 -60.3602392950833\\
480.361938477789 -57.7624826054087\\
482.315063477794 -53.2442878459422\\
484.268188477799 -52.2117742078977\\
486.221313477804 -60.1480815200763\\
488.174438477809 -66.345683610978\\
490.127563477814 -72.5858658379143\\
492.080688477819 -81.8683472282551\\
494.033813477824 -79.6096677990371\\
495.986938477829 -70.9853815668203\\
497.940063477834 -71.5612031822564\\
499.893188477839 -73.9848974101769\\
};
%\addlegendentry{LRM (estimation)};

% plot split among two traces
\addplot [LRM,interpol, forget plot]
table[row sep=crcr]{
1.05285644531519 -8.51576097134915\\
1.07727050781525 -8.52064084058884\\
1.10168457031531 -8.58669958866869\\
1.12609863281538 -8.62742470767301\\
1.15051269531544 -8.66678191155552\\
1.1749267578155 -8.80836305133198\\
1.19934082031556 -8.89143761920582\\
1.22375488281562 -8.97994023362361\\
1.24816894531569 -9.07449006328557\\
1.27258300781575 -9.17580586522621\\
1.29699707031581 -9.85343079125266\\
1.32141113281587 -10.0860182046446\\
1.34582519531594 -10.3444574747733\\
1.370239257816 -10.6333932853\\
1.39465332031606 -10.958516104874\\
1.41906738281612 -13.8117544809697\\
1.44348144531619 -14.7185117301041\\
1.46789550781625 -15.8030890200477\\
1.49230957031631 -17.1148463927319\\
1.51672363281637 -18.7019076943432\\
1.54113769531644 -22.7753768557106\\
1.5655517578165 -12.284523192581\\
1.58996582031656 -11.5835344438338\\
1.61437988281662 -10.9661468649682\\
1.63879394531668 -10.4241158336254\\
1.66320800781675 -9.94919079190259\\
1.68762207031681 -8.45980060162617\\
1.71203613281687 -8.46254033744981\\
1.73645019531693 -8.46643453004077\\
1.760864257817 -8.47033227150138\\
1.78527832031706 -8.47314560891255\\
1.80969238281712 -8.47381104247995\\
1.83410644531718 -8.47125192956014\\
1.85852050781725 -8.46434232979573\\
1.88293457031731 -8.45187214026663\\
1.90734863281737 -9.15334685306522\\
1.93176269531743 -8.75993026087383\\
1.9561767578175 -8.76227503488138\\
1.98059082031756 -8.72058049167936\\
2.00500488281762 -8.63509291600167\\
2.02941894531768 -8.50597619742143\\
2.05383300781774 -7.18331067332662\\
2.07824707031781 -7.18886766289324\\
2.10266113281787 -7.18572438658089\\
2.12707519531793 -7.17347649094989\\
2.15148925781799 -7.1515785847443\\
2.17590332031806 -7.11932069725628\\
2.20031738281812 -7.0757968894996\\
2.22473144531818 -7.01986351099208\\
2.24914550781824 -6.95008353323362\\
2.27355957031831 -9.19372570287112\\
2.29797363281837 -7.2729923326641\\
2.32238769531843 -7.02027421653713\\
2.34680175781849 -6.77918065326492\\
2.37121582031856 -6.54765407805922\\
2.39562988281862 -6.32364323334548\\
2.42004394531868 -6.10508625470447\\
2.44445800781874 -5.88988126651293\\
2.4688720703188 -5.67584692926346\\
2.49328613281887 -5.46067375859252\\
2.51770019531893 -5.99977992981235\\
2.54211425781899 -4.16407760331072\\
2.56652832031905 -4.14373338665229\\
2.59094238281912 -4.11927694723431\\
2.61535644531918 -4.08965459380124\\
2.63977050781924 -4.05377714555459\\
2.6641845703193 -4.01047511147766\\
2.68859863281937 -3.95844737189287\\
2.71301269531943 -3.89619974165504\\
2.73742675781949 -3.82196832343732\\
2.76184082031955 -5.62407559364448\\
2.78625488281961 -5.49073924655028\\
2.81066894531968 -5.32096420721672\\
2.83508300781974 -5.10772435491901\\
2.8594970703198 -4.84171055237442\\
2.88391113281986 -8.77426074495116\\
2.90832519531993 -7.77981557581546\\
2.93273925781999 -7.98553988155192\\
2.95715332032005 -8.17661979017009\\
2.98156738282011 -8.3551059210431\\
3.00598144532018 -8.52278306403554\\
3.03039550782024 -7.35369870656888\\
3.0548095703203 -7.85989349432907\\
3.07922363282036 -8.34123343208142\\
3.10363769532043 -8.80020151225943\\
3.12805175782049 -9.23913401580529\\
3.15246582032055 -8.03877970933513\\
3.17687988282061 -8.22493915988196\\
3.20129394532067 -8.36951874811064\\
3.22570800782074 -8.48107651513737\\
3.2501220703208 -8.56581619460292\\
3.27453613282086 -8.50637977881627\\
3.29895019532092 -8.55380202756214\\
3.32336425782099 -8.58617165086025\\
3.34777832032105 -8.60844314487036\\
3.37219238282111 -8.62361804201754\\
3.39660644532117 -7.8861798698951\\
3.42102050782124 -7.90216484471307\\
3.4454345703213 -7.90747971706696\\
3.46984863282136 -7.90202445840544\\
3.49426269532142 -7.88503487196971\\
3.51867675782148 -7.85505735148354\\
3.54309082032155 -7.80986592323467\\
3.56750488282161 -7.74631875718785\\
3.59191894532167 -7.66014614655649\\
3.61633300782173 -8.13314307491237\\
3.6407470703218 -7.78299945111843\\
3.66516113282186 -7.44445508484893\\
3.68957519532192 -7.16560344285659\\
3.71398925782198 -6.74505496149823\\
3.73840332032205 -6.74156468032538\\
3.76281738282211 -6.58167351996417\\
3.78723144532217 -6.45030400784088\\
3.81164550782223 -6.34474243217545\\
3.8360595703223 -6.26357255515518\\
3.86047363282236 -11.6126126881633\\
3.88488769532242 -11.8681015331735\\
3.90930175782248 -12.0524447438469\\
3.93371582032254 -12.1383352181713\\
3.95812988282261 -12.0950695578038\\
3.98254394532267 -27.2666170494525\\
4.00695800782273 -14.4747336473432\\
4.03137207032279 -15.1153808003747\\
4.05578613282286 -15.7245527152835\\
4.08020019532292 -16.2822125955877\\
4.10461425782298 -16.7547872276959\\
4.12902832032304 -7.77236272356356\\
4.15344238282311 -7.49841645738718\\
4.17785644532317 -7.2246466834734\\
4.20227050782323 -6.94729245318558\\
4.22668457032329 -6.66167654821032\\
4.25109863282336 -6.36229731445462\\
4.27551269532342 -6.04267094723031\\
4.29992675782348 -5.69497901074919\\
4.32434082032354 -5.30950863277394\\
4.3487548828236 -7.08055448178681\\
4.37316894532367 -6.78459385982251\\
4.39758300782373 -6.52241294456297\\
4.42199707032379 -6.28455433201827\\
4.44641113282385 -6.06366453514318\\
4.47082519532392 -7.0500443042223\\
4.49523925782398 -6.6199264216732\\
4.51965332032404 -6.4981711580549\\
4.5440673828241 -6.3526659633759\\
4.56848144532417 -6.12284658287172\\
4.59289550782423 -6.2031078669429\\
4.61730957032429 -6.12163500697602\\
4.64172363282435 -6.01519235551564\\
4.66613769532441 -5.87897401747711\\
4.69055175782448 -5.74202983697802\\
4.71496582032454 -5.60539546929709\\
4.7393798828246 -5.46947755656709\\
4.76379394532466 -5.33424333662879\\
4.78820800782473 -5.19933961738406\\
4.81262207032479 -5.06416276453444\\
4.83703613282485 -7.04654381658656\\
4.86145019532491 -7.00586854513887\\
4.88586425782498 -6.95498467990245\\
4.91027832032504 -6.89440877875739\\
4.9346923828251 -6.82431047716597\\
4.95910644532516 -7.42471692325489\\
4.98352050782522 -7.45187249999401\\
5.00793457032529 -7.47495205098528\\
5.03234863282535 -7.49430805014373\\
5.05676269532541 -7.51020358716994\\
5.08117675782547 -8.55350285186501\\
5.10559082032554 -6.93522944806273\\
5.1300048828256 -6.74278624991382\\
5.15441894532566 -6.58446197229119\\
5.17883300782572 -6.45623169194926\\
5.20324707032579 -6.35507655408827\\
5.22766113282585 -6.27877132666424\\
5.25207519532591 -6.22573326027941\\
5.27648925782597 -6.19491224012904\\
5.30090332032604 -6.18570723872335\\
5.3253173828261 -6.67867944071583\\
5.34973144532616 -6.82966907697931\\
5.37414550782622 -7.01711675043765\\
5.39855957032629 -7.24459220285473\\
5.42297363282635 -7.51660876373734\\
5.44738769532641 -13.9648086683378\\
5.47180175782647 -13.8670595456319\\
5.49621582032653 -13.7581083180007\\
5.5206298828266 -13.6374522455173\\
5.54504394532666 -13.5041100653607\\
5.56945800782672 -15.7161310320576\\
5.59387207032678 -11.7019324391889\\
5.61828613282685 -11.4169754107917\\
5.64270019532691 -11.1515010550114\\
5.66711425782697 -10.9017578518826\\
5.69152832032703 -10.6640946200989\\
5.7159423828271 -10.4348693216541\\
5.74035644532716 -10.2103137127801\\
5.76477050782722 -9.98635326858943\\
5.78918457032728 -9.7583693593977\\
5.81359863282735 -10.8307062300968\\
5.83801269532741 -8.68749422056135\\
5.86242675782747 -8.57960117244153\\
5.88684082032753 -8.48092646930365\\
5.91125488282759 -8.38965522146282\\
5.93566894532766 -8.30425561749416\\
5.96008300782772 -5.39486597720281\\
5.98449707032778 -5.3779700694451\\
6.00891113282784 -5.3796197675079\\
6.03332519532791 -5.39688024479642\\
6.05773925782797 -5.42696708464433\\
6.08215332032803 -5.46713342978086\\
6.10656738282809 -5.5145728171226\\
6.13098144532816 -5.56633558772518\\
6.15539550782822 -5.61925961298135\\
6.17980957032828 -9.02567374234013\\
6.20422363282834 -7.67549085017527\\
6.2286376953284 -7.70239962940258\\
6.25305175782847 -7.73796355906143\\
6.27746582032853 -7.78242765011703\\
6.30187988282859 -7.83623903877179\\
6.32629394532865 -7.90006033398163\\
6.35070800782872 -7.9747941051823\\
6.37512207032878 -8.06162045826107\\
6.39953613282884 -8.16205085949275\\
6.4239501953289 -9.50485211902964\\
6.44836425782897 -9.74453118305394\\
6.47277832032903 -9.98931743088366\\
6.49719238282909 -10.2412867298719\\
6.52160644532915 -10.5025089828187\\
6.54602050782922 -11.2693014228415\\
6.57043457032928 -10.6362162902408\\
6.59484863282934 -10.5729034576439\\
6.6192626953294 -10.5326591757755\\
6.64367675782946 -10.514192910794\\
6.66809082032953 -10.5169201199893\\
6.69250488282959 -10.1270672524776\\
6.71691894532965 -10.1302801771576\\
6.74133300782971 -10.1618709250044\\
6.76574707032978 -10.2223699928412\\
6.79016113282984 -10.3134705943834\\
6.8145751953299 -10.4381999029148\\
6.83898925782996 -10.6012429465223\\
6.86340332033003 -10.8094904764718\\
6.88781738283009 -11.0729448636438\\
6.91223144533015 -11.1995885297395\\
6.93664550783021 -8.62130174639645\\
6.96105957033028 -8.37687138903311\\
6.98547363283034 -8.16303682476689\\
7.0098876953304 -7.97358412931919\\
7.03430175783046 -7.80341955645525\\
7.05871582033052 -5.94393207593481\\
7.08312988283059 -5.62745114839635\\
7.10754394533065 -5.33771979390218\\
7.13195800783071 -5.06892870566287\\
7.15637207033077 -4.81624877442033\\
7.18078613283084 -4.08559737808599\\
7.2052001953309 -4.03402768497449\\
7.22961425783096 -3.97570580669856\\
7.25402832033102 -3.89117212394569\\
7.27844238283109 -3.83617738680005\\
7.30285644533115 -3.7533283322208\\
7.32727050783121 -3.66048179799509\\
7.35168457033127 -3.55662743470833\\
7.37609863283133 -3.44086376282218\\
7.4005126953314 -4.16425770547897\\
7.42492675783146 -4.1752559744474\\
7.44934082033152 -4.30024330920929\\
7.47375488283158 -4.35941077722515\\
7.49816894533165 -4.41398456925538\\
7.52258300783171 -4.68008937914988\\
7.54699707033177 -4.77970737515449\\
7.57141113283183 -4.87188084376822\\
7.5958251953319 -4.95774274511257\\
7.62023925783196 -5.03825340387272\\
7.64465332033202 -5.62375271898077\\
7.66906738283208 -5.68742136641816\\
7.69348144533214 -5.7438562484341\\
7.71789550783221 -5.79352347954602\\
7.74230957033227 -5.83671651958952\\
7.76672363283233 -6.01695479863599\\
7.79113769533239 -6.02088707444813\\
7.81555175783246 -6.01414899574559\\
7.83996582033252 -5.99733149598489\\
7.86437988283258 -5.97096181712334\\
7.88879394533264 -6.12844468778331\\
7.91320800783271 -5.68431569159645\\
7.93762207033277 -5.66253067597148\\
7.96203613283283 -5.64031080053513\\
7.98645019533289 -5.61742862073027\\
8.01086425783296 -5.59364039591219\\
8.03527832033302 -5.22092548834581\\
8.05969238283308 -5.15145538524274\\
8.08410644533314 -5.08588044642522\\
8.10852050783321 -5.02312315164289\\
8.13293457033327 -4.96222710130439\\
8.15734863283333 -4.9023082296373\\
8.18176269533339 -4.84251266781808\\
8.20617675783345 -4.78197705755412\\
8.23059082033352 -4.71978749420566\\
8.25500488283358 -4.79734910737034\\
8.27941894533364 -4.53504421467312\\
8.3038330078337 -4.43024774188285\\
8.32824707033377 -4.32683914029633\\
8.35266113283383 -4.22439911401432\\
8.37707519533389 -4.12246094050909\\
8.40148925783395 -4.02049315593774\\
8.42590332033402 -3.91787633459955\\
8.45031738283408 -3.81387128072487\\
8.47473144533414 -3.70757451075724\\
8.4991455078342 -3.94275103955658\\
8.52355957033426 -3.87828929316271\\
8.54797363283433 -3.81319171839783\\
8.57238769533439 -3.74688292874623\\
8.59680175783445 -3.67864083607049\\
8.62121582033451 -4.07615674862063\\
8.64562988283458 -4.04532904446569\\
8.67004394533464 -4.00178462165718\\
8.6944580078347 -3.94091009507872\\
8.71887207033476 -3.85526618522772\\
8.74328613283483 -4.04669632755679\\
8.76770019533489 -4.0135377410175\\
8.79211425783495 -3.97407063274068\\
8.81652832033501 -3.92767940967576\\
8.84094238283508 -3.87367747365903\\
8.86535644533514 -4.44566060039841\\
8.8897705078352 -4.01456798404047\\
8.91418457033526 -4.00987013639639\\
8.93859863283532 -4.00984344359\\
8.96301269533539 -4.01385223287787\\
8.98742675783545 -4.02143547940454\\
9.01184082033551 -4.03225879077945\\
9.03625488283557 -4.04608170441435\\
9.06066894533564 -4.06273479965114\\
9.0850830078357 -4.08210319618809\\
9.10949707033576 -4.37988335504161\\
9.13391113283582 -4.39526317783753\\
9.15832519533589 -4.41160259913244\\
9.18273925783595 -4.42879782093223\\
9.20715332033601 -4.44675415015917\\
9.23156738283607 -4.69329576736237\\
9.25598144533613 -4.68540326464449\\
9.2803955078362 -4.67597978212285\\
9.30480957033626 -4.66491965182399\\
9.32922363283632 -4.65207280594063\\
9.35363769533638 -4.72156786633502\\
9.37805175783645 -4.63649497852884\\
9.40246582033651 -4.61087176458369\\
9.42687988283657 -4.5882150367849\\
9.45129394533663 -4.56826334325115\\
9.4757080078367 -4.55081959249111\\
9.50012207033676 -4.53574317777287\\
9.52453613283682 -4.52294504464578\\
9.54895019533688 -4.51238527546002\\
9.57336425783694 -4.50407298060946\\
9.59777832033701 -4.62912347895025\\
9.62219238283707 -4.54466134986109\\
9.64660644533713 -4.56072204051281\\
9.67102050783719 -4.5750293955283\\
9.69543457033726 -4.58812846217677\\
9.71984863283732 -4.60041142779397\\
9.74426269533738 -4.61217422629278\\
9.76867675783745 -4.62365115635322\\
9.79309082033751 -4.63503718416251\\
9.81750488283757 -4.64650312674593\\
9.84191894533763 -4.7603906486741\\
9.86633300783769 -4.77110260978714\\
9.89074707033776 -4.78253779983436\\
9.91516113283782 -4.7948532880423\\
9.93957519533788 -4.80824264321865\\
9.96398925783794 -4.86195999932698\\
9.988403320338 -4.8781703947696\\
10.0128173828381 -4.89484749445415\\
10.0372314453381 -4.9120414222657\\
10.0616455078382 -4.92979911822255\\
10.0860595703383 -5.06167736677708\\
10.1104736328383 -5.08825181311721\\
10.1348876953384 -5.11543845132701\\
10.1593017578384 -5.14320940147098\\
10.1837158203385 -5.17149436825662\\
10.2081298828386 -5.20872114546677\\
10.2325439453386 -5.2182398793708\\
10.2569580078387 -5.22650711293755\\
10.2813720703388 -5.23325306154163\\
10.3057861328388 -5.23814690572732\\
10.3302001953389 -5.36546408255634\\
10.3546142578389 -5.3303878672412\\
10.379028320339 -5.29332549271618\\
10.4034423828391 -5.25461678236047\\
10.4278564453391 -5.21452324527058\\
10.4522705078392 -5.17453300419197\\
10.4766845703393 -5.124427562736\\
10.5010986328393 -5.07630362606926\\
10.5255126953394 -5.02943668302203\\
10.5499267578394 -4.98315932221442\\
10.5743408203395 -5.01465753781764\\
10.5987548828396 -4.50376517633998\\
10.6231689453396 -4.48073624960335\\
10.6475830078397 -4.46093075898074\\
10.6719970703398 -4.4439532816142\\
10.6964111328398 -4.42947735817256\\
10.7208251953399 -4.31218038557046\\
10.7452392578399 -4.28380019019619\\
10.76965332034 -4.25841269268403\\
10.7940673828401 -4.23577246751694\\
10.8184814453401 -4.21568083178255\\
10.8428955078402 -4.19798161812844\\
10.8673095703403 -4.18255842384508\\
10.8917236328403 -4.16933332892097\\
10.9161376953404 -4.15826716786694\\
10.9405517578404 -4.28310561620077\\
10.9649658203405 -4.23292242976993\\
10.9893798828406 -4.24137853788045\\
11.0137939453406 -4.24996755516025\\
11.0382080078407 -4.24426012910999\\
11.0626220703407 -4.26749108451889\\
11.0870361328408 -4.24328876823222\\
11.1114501953409 -4.25318685187989\\
11.1358642578409 -4.26366438247436\\
11.160278320341 -4.27484061313771\\
11.1846923828411 -4.28685248109002\\
11.2091064453411 -4.29986023400164\\
11.2335205078412 -4.31405465516275\\
11.2579345703412 -4.32966658032797\\
11.2823486328413 -4.34697971787085\\
11.3067626953414 -4.59266038637833\\
11.3311767578414 -4.61136204761391\\
11.3555908203415 -4.62983916296179\\
11.3800048828416 -4.64829560297392\\
11.4044189453416 -4.66695795242777\\
11.4288330078417 -4.82742461460958\\
11.4532470703417 -4.77712674747369\\
11.4776611328418 -4.79373208157097\\
11.5020751953419 -4.80862092070822\\
11.5264892578419 -4.8218089578894\\
11.550903320342 -4.83328593094063\\
11.5753173828421 -4.74501922409416\\
11.5997314453421 -4.72254050120182\\
11.6241455078422 -4.70197207934643\\
11.6485595703422 -4.68293184108938\\
11.6729736328423 -4.66510376212335\\
11.6973876953424 -4.60597327645445\\
11.7218017578424 -4.59758605310248\\
11.7462158203425 -4.59144401780424\\
11.7706298828426 -4.58721956849092\\
11.7950439453426 -4.58475701977346\\
11.8194580078427 -4.5840367356347\\
11.8438720703427 -4.58516312282723\\
11.8682861328428 -4.5883706858537\\
11.8927001953429 -4.59404726568368\\
11.9171142578429 -4.81627913426053\\
11.941528320343 -4.79648422697687\\
11.9659423828431 -4.79085497152209\\
11.9903564453431 -4.78674890869644\\
12.0147705078432 -4.78404912262994\\
12.0391845703432 -4.78268653034814\\
12.0635986328433 -4.78263466668704\\
12.0880126953434 -4.78390789350635\\
12.1124267578434 -4.78656277248143\\
12.1368408203435 -4.79070279623738\\
12.1612548828436 -5.12385190184955\\
12.1856689453436 -5.13045217304744\\
12.2100830078437 -5.13658758847021\\
12.2344970703437 -5.14230585643713\\
12.2589111328438 -5.14764497518206\\
12.2833251953439 -5.34142561315514\\
12.3077392578439 -5.34835245236059\\
12.332153320344 -5.35394627194523\\
12.3565673828441 -5.35827717858007\\
12.3809814453441 -5.36139299097823\\
12.4053955078442 -5.4419604402392\\
12.4298095703442 -5.16273155671593\\
12.4542236328443 -5.14336094267333\\
12.4786376953444 -5.12614303955962\\
12.5030517578444 -5.110850207353\\
12.5274658203445 -5.09730022921855\\
12.5518798828446 -5.08534971040308\\
12.5762939453446 -5.07488949040226\\
12.6007080078447 -5.06584180904258\\
12.6251220703447 -5.05815912621568\\
12.6495361328448 -5.08627166543528\\
12.6739501953449 -5.0683420676641\\
12.6983642578449 -5.05032768657884\\
12.722778320345 -5.03202426826124\\
12.747192382845 -5.01321230944973\\
12.7716064453451 -5.06307452466473\\
12.7960205078452 -5.0528386776358\\
12.8204345703452 -5.04187620140851\\
12.8448486328453 -5.0300074075177\\
12.8692626953454 -5.01701305487109\\
12.8936767578454 -5.1575760742586\\
12.9180908203455 -5.16644147988183\\
12.9425048828455 -5.17510715644067\\
12.9669189453456 -5.18366852306764\\
12.9913330078457 -5.19221669884746\\
13.0157470703457 -5.2204173596981\\
13.0401611328458 -5.22344637843054\\
13.0645751953459 -5.22643565237934\\
13.0889892578459 -5.22941868996418\\
13.113403320346 -5.2324248883782\\
13.137817382846 -5.23481167163141\\
13.1622314453461 -5.22165796962423\\
13.1866455078462 -5.22301753468957\\
13.2110595703462 -5.22349464431591\\
13.2354736328463 -5.22315322486952\\
13.2598876953464 -5.22203315065661\\
13.2843017578464 -5.16379809459829\\
13.3087158203465 -5.15519430975348\\
13.3331298828465 -5.14670317186096\\
13.3575439453466 -5.13822850970098\\
13.3819580078467 -5.12967875060843\\
13.4063720703467 -5.09881405592296\\
13.4307861328468 -5.10318912679577\\
13.4552001953469 -5.10266440004017\\
13.4796142578469 -5.0928788516041\\
13.504028320347 -5.11845784392619\\
13.528442382847 -5.12438707017299\\
13.5528564453471 -5.13081280835189\\
13.5772705078472 -5.13779121064772\\
13.6016845703472 -5.14538651245516\\
13.6260986328473 -5.21981389048068\\
13.6505126953474 -5.23109482362992\\
13.6749267578474 -5.24210631497851\\
13.6993408203475 -5.25283716946274\\
13.7237548828475 -5.2632577796299\\
13.7481689453476 -5.27018467453956\\
13.7725830078477 -5.26576584931399\\
13.7969970703477 -5.26153562517436\\
13.8214111328478 -5.25744552076338\\
13.8458251953479 -5.25345119915517\\
13.8702392578479 -5.28541107737703\\
13.894653320348 -5.27920659108827\\
13.919067382848 -5.27265330028467\\
13.9434814453481 -5.26576051840283\\
13.9678955078482 -5.25853409760174\\
13.9923095703482 -5.2832831583184\\
14.0167236328483 -5.27707241644373\\
14.0411376953484 -5.26977328086826\\
14.0655517578484 -5.26135781695615\\
14.0899658203485 -5.25178680101817\\
14.1143798828485 -5.29359694981531\\
14.1387939453486 -5.25977060454579\\
14.1632080078487 -5.25563488599266\\
14.1876220703487 -5.25263175333708\\
14.2120361328488 -5.25064080112674\\
14.2364501953489 -5.24956558067282\\
14.2608642578489 -5.24932867097863\\
14.285278320349 -5.24986793703653\\
14.309692382849 -5.25113364833663\\
14.3341064453491 -5.25308622456083\\
14.3585205078492 -5.27594867991246\\
14.3829345703492 -5.2797087402239\\
14.4073486328493 -5.28412949582525\\
14.4317626953494 -5.28937500562955\\
14.4561767578494 -5.29566596366368\\
14.4805908203495 -5.43726196953344\\
14.5050048828495 -5.38178173226487\\
14.5294189453496 -5.38522437302026\\
14.5538330078497 -5.38855888780876\\
14.5782470703497 -5.39179423135346\\
14.6026611328498 -5.39493553828316\\
14.6270751953498 -5.39798422320393\\
14.6514892578499 -5.40093791798313\\
14.67590332035 -5.40379024908032\\
14.70031738285 -5.40653044758756\\
14.7247314453501 -5.41319933299849\\
14.7491455078502 -5.40975146583037\\
14.7735595703502 -5.40422791995644\\
14.7979736328503 -5.39784409892974\\
14.8223876953503 -5.39192440611833\\
14.8468017578504 -5.38647668020172\\
14.8712158203505 -5.36342280890977\\
14.8956298828505 -5.36658284703879\\
14.9200439453506 -5.36964273403595\\
14.9444580078507 -5.36949312837953\\
14.9688720703507 -5.37553725565994\\
14.9932861328508 -5.37840785677815\\
15.0177001953508 -5.38125027945773\\
15.0421142578509 -5.38408112978908\\
15.066528320351 -5.38691676987702\\
15.090942382851 -5.38114687195622\\
15.1153564453511 -5.37483003691978\\
15.1397705078512 -5.36823097718025\\
15.1641845703512 -5.3615183939429\\
15.1885986328513 -5.35490022262542\\
15.2130126953513 -5.39037153790906\\
15.2374267578514 -5.39536110059299\\
15.2618408203515 -5.40157346290158\\
15.2862548828515 -5.40888415249316\\
15.3106689453516 -5.41717520027754\\
15.3350830078517 -5.48184277093742\\
15.3594970703517 -5.48318622966883\\
15.3839111328518 -5.49073929387151\\
15.4083251953518 -5.49803111518912\\
15.4327392578519 -5.50505600882548\\
15.457153320352 -5.51181374816366\\
15.481567382852 -5.51830885573253\\
15.5059814453521 -5.52455009145558\\
15.5303955078522 -5.53055011497622\\
15.5548095703522 -5.53632530353588\\
15.5792236328523 -5.5751008048195\\
15.6036376953523 -5.56563546431244\\
15.6280517578524 -5.5698612697206\\
15.6524658203525 -5.57426288630683\\
15.6768798828525 -5.5788522192741\\
15.7012939453526 -5.58364149487369\\
15.7257080078527 -5.58864340053071\\
15.7501220703527 -5.59387118687869\\
15.7745361328528 -5.59933873822126\\
15.7989501953528 -5.60506061313077\\
15.8233642578529 -5.65619398147703\\
15.847778320353 -5.65206262813683\\
15.872192382853 -5.65410248249651\\
15.8966064453531 -5.65571484541573\\
15.9210205078532 -5.65688364684041\\
15.9454345703532 -5.65758809773024\\
15.9698486328533 -5.63646639659777\\
15.9942626953533 -5.64287175684018\\
16.0186767578534 -5.64927702119451\\
16.0430908203535 -5.65514609784418\\
16.0675048828535 -5.66203644991543\\
16.0919189453536 -5.66837528428783\\
16.1163330078537 -5.67468249381534\\
16.1407470703537 -5.68095867491604\\
16.1651611328538 -5.68720722440975\\
16.1895751953538 -5.73366953147234\\
16.2139892578539 -5.72095759564337\\
16.238403320354 -5.72577423144548\\
16.262817382854 -5.73047878935239\\
16.2872314453541 -5.73509630345217\\
16.3116455078542 -5.73964778557212\\
16.3360595703542 -5.73458430207421\\
16.3604736328543 -5.74385918593418\\
16.3848876953543 -5.75244755319056\\
16.4093017578544 -5.75750782895568\\
16.4337158203545 -5.76752423311365\\
16.4581298828545 -5.7740198189216\\
16.4825439453546 -5.77797733749884\\
16.5069580078546 -5.77712968629243\\
16.5313720703547 -5.77589116770537\\
16.5557861328548 -5.77433346677844\\
16.5802001953548 -5.77252757894416\\
16.6046142578549 -5.77054553315821\\
16.629028320355 -5.76846201127489\\
16.653442382855 -5.76635588225878\\
16.6778564453551 -5.79503962596715\\
16.7022705078551 -5.79794394304156\\
16.7266845703552 -5.80104732427571\\
16.7510986328553 -5.8043672823045\\
16.7755126953553 -5.80657096815827\\
16.7999267578554 -5.80841278120261\\
16.8243408203555 -5.81011300244285\\
16.8487548828555 -5.81166605408094\\
16.8731689453556 -5.8130664707283\\
16.8975830078557 -5.81430900101287\\
16.9219970703557 -5.86951510298081\\
16.9464111328558 -5.87252739236476\\
16.9708251953558 -5.87544143984036\\
16.9952392578559 -5.87826221833848\\
17.019653320356 -5.87868074428377\\
17.044067382856 -5.87767728968151\\
17.0684814453561 -5.87688353478353\\
17.0928955078561 -5.87629859564299\\
17.1173095703562 -5.87592458000466\\
17.1417236328563 -5.87576711459826\\
17.1661376953563 -5.92487604686454\\
17.1905517578564 -5.91600348049576\\
17.2149658203565 -5.92077898098557\\
17.2393798828565 -5.92260904762458\\
17.2637939453566 -5.92184818698763\\
17.2882080078566 -5.93500261818764\\
17.3126220703567 -5.93977803287112\\
17.3370361328568 -5.94462111275789\\
17.3614501953568 -5.94865377639491\\
17.3858642578569 -5.94969376146202\\
17.410278320357 -5.95015087954971\\
17.434692382857 -5.94509692423117\\
17.4591064453571 -5.94352264828495\\
17.4835205078571 -5.94136682871124\\
17.5079345703572 -5.93865720108045\\
17.5323486328573 -5.93544209255066\\
17.5567626953573 -5.93180163295511\\
17.5811767578574 -5.92786476932406\\
17.6055908203575 -5.92383494098459\\
17.6300048828575 -5.92002842767818\\
17.6544189453576 -5.93459262007315\\
17.6788330078576 -5.93389587173812\\
17.7032470703577 -5.93290230893831\\
17.7276611328578 -5.93157427243949\\
17.7520751953578 -5.92986725741167\\
17.7764892578579 -5.98297184298929\\
17.800903320358 -5.99021491070829\\
17.825317382858 -5.99736840138365\\
17.8497314453581 -6.00443192776561\\
17.8741455078581 -6.0114045622521\\
17.8985595703582 -6.07618921041893\\
17.9229736328583 -6.08453227130553\\
17.9473876953583 -6.09232162971193\\
17.9718017578584 -6.09959507453277\\
17.9962158203585 -6.10637210822495\\
18.0206298828585 -6.15022048278979\\
18.0450439453586 -6.14305053834295\\
18.0694580078586 -6.14806857592697\\
18.0938720703587 -6.15305081490771\\
18.1182861328588 -6.15802056398718\\
18.1427001953588 -6.16299830993887\\
18.1671142578589 -6.16800210331189\\
18.191528320359 -6.17304787775123\\
18.215942382859 -6.1781497153421\\
18.2403564453591 -6.18332006793406\\
18.2647705078591 -6.19764594422691\\
18.2891845703592 -6.19750632596913\\
18.3135986328593 -6.19742633112463\\
18.3380126953593 -6.19744920424711\\
18.3624267578594 -6.19761723814349\\
18.3868408203594 -6.27523596865552\\
18.4112548828595 -6.25989002206501\\
18.4356689453596 -6.25910479997509\\
18.4600830078596 -6.25883196724425\\
18.4844970703597 -6.25902970106353\\
18.5089111328598 -6.25966576756372\\
18.5333251953598 -6.26071567640548\\
18.5577392578599 -6.26216130116904\\
18.58215332036 -6.26398984028947\\
18.60656738286 -6.26619303096004\\
18.6309814453601 -6.27874720242085\\
18.6553955078601 -6.27434625892101\\
18.6798095703602 -6.28194901172174\\
18.7042236328603 -6.28941306698113\\
18.7286376953603 -6.29553227201188\\
18.7530517578604 -6.30396396029499\\
18.7774658203604 -6.31106198008115\\
18.8018798828605 -6.318044640165\\
18.8262939453606 -6.32491125639831\\
18.8507080078606 -6.33165913664311\\
18.8751220703607 -6.34526601892941\\
18.8995361328608 -6.35184805988848\\
18.9239501953608 -6.35832718833797\\
18.9483642578609 -6.36467909620347\\
18.9727783203609 -6.37088079632696\\
18.997192382861 -6.37578343397126\\
19.0216064453611 -6.37685831910738\\
19.0460205078611 -6.38101969087052\\
19.0704345703612 -6.38519013766251\\
19.0948486328613 -6.38935925031694\\
19.1192626953613 -6.39351807844633\\
19.1436767578614 -6.39765891047\\
19.1680908203614 -6.40177509189078\\
19.1925048828615 -6.40550885865741\\
19.2169189453616 -6.40814264193881\\
19.2413330078616 -6.41077816787953\\
19.2657470703617 -6.41226427890246\\
19.2901611328618 -6.41464564137056\\
19.3145751953618 -6.41715599704844\\
19.3389892578619 -6.4198122494077\\
19.3634033203619 -6.42262779289177\\
19.387817382862 -6.42561279003939\\
19.4122314453621 -6.42877442923276\\
19.4366455078621 -6.43211716789631\\
19.4610595703622 -6.43564296479144\\
19.4854736328623 -6.4617686458343\\
19.5098876953623 -6.46079679096215\\
19.5343017578624 -6.46321068108489\\
19.5587158203624 -6.46567488249838\\
19.5831298828625 -6.46820465016202\\
19.6075439453626 -6.47081567966717\\
19.6319580078626 -6.47018078874208\\
19.6563720703627 -6.47311509939794\\
19.6807861328628 -6.47617559899669\\
19.7052001953628 -6.47937059210915\\
19.7296142578629 -6.48270834228151\\
19.7540283203629 -6.48619713861586\\
19.778442382863 -6.48984535931368\\
19.8028564453631 -6.49366153307585\\
19.8272705078631 -6.49765439919094\\
19.8516845703632 -6.51298944363157\\
19.8760986328633 -6.50913397953911\\
19.9005126953633 -6.51469687935452\\
19.9249267578634 -6.52038678308071\\
19.9493408203634 -6.52620366223465\\
19.9737548828635 -6.53214687241672\\
19.9981689453636 -6.53821501328315\\
20.0225830078636 -6.54440576957268\\
20.0469970703637 -6.55071573069051\\
20.0714111328638 -6.55714018624332\\
20.0958251953638 -6.56335879516269\\
20.1202392578639 -6.568978942581\\
20.1446533203639 -6.57462107533843\\
20.169067382864 -6.58027650422076\\
20.1934814453641 -6.58593645592845\\
20.2178955078641 -6.59208026896772\\
20.2423095703642 -6.59687794240756\\
20.2667236328642 -6.60167969866762\\
20.2911376953643 -6.60637784808341\\
20.3155517578644 -6.61033193933241\\
20.3399658203644 -6.61436229531341\\
20.3643798828645 -6.61848937758225\\
20.3887939453646 -6.62273243597565\\
20.4132080078646 -6.627109502667\\
20.4376220703647 -6.63163738124592\\
20.4620361328647 -6.64859443282643\\
20.4864501953648 -6.65004414596774\\
20.5108642578649 -6.65572546456258\\
20.5352783203649 -6.66143606690366\\
20.559692382865 -6.6671748352324\\
20.5841064453651 -6.67294093280316\\
20.6085205078651 -6.6787337696959\\
20.6329345703652 -6.68455297301557\\
20.6573486328652 -6.69039836089786\\
20.6817626953653 -6.69626991982142\\
20.7061767578654 -6.70291686291387\\
20.7305908203654 -6.70894524000227\\
20.7550048828655 -6.71500372977533\\
20.7794189453656 -6.72109082514362\\
20.8038330078656 -6.72720499717758\\
20.8282470703657 -6.73815098434216\\
20.8526611328657 -6.74332302595593\\
20.8770751953658 -6.74847971489999\\
20.9014892578659 -6.75358204855297\\
20.9259033203659 -6.75863338692432\\
20.950317382866 -6.76363729386759\\
20.9747314453661 -6.76641143806518\\
20.9991455078661 -6.77244053338364\\
21.0235595703662 -6.77840407425629\\
21.0479736328662 -6.78325964758938\\
21.0723876953663 -6.79033148825164\\
21.0968017578664 -6.79620052025047\\
21.1212158203664 -6.80200799824576\\
21.1456298828665 -6.80774526455554\\
21.1700439453666 -6.81340332204303\\
21.1944580078666 -6.81910296578411\\
21.2188720703667 -6.82369040133563\\
21.2432861328667 -6.82807373224654\\
21.2677001953668 -6.83223154973399\\
21.2921142578669 -6.8361398111594\\
21.3165283203669 -6.85216619198837\\
21.340942382867 -6.84230171872241\\
21.3653564453671 -6.84472663989561\\
21.3897705078671 -6.84688684991994\\
21.4141845703672 -6.84878230187365\\
21.4385986328672 -6.85041525583068\\
21.4630126953673 -6.8517922866119\\
21.4874267578674 -6.8529277177833\\
21.5118408203674 -6.85384951586423\\
21.5362548828675 -6.85460946898826\\
21.5606689453676 -6.88830424314835\\
21.5850830078676 -6.89250422185239\\
21.6094970703677 -6.89675783930136\\
21.6339111328677 -6.90107794263002\\
21.6583251953678 -6.90547714466561\\
21.6827392578679 -6.92386064878878\\
21.7071533203679 -6.92968428423899\\
21.731567382868 -6.93621056242552\\
21.7559814453681 -6.94240156513393\\
21.7803955078681 -6.9482719135321\\
21.8048095703682 -6.95383241417943\\
21.8292236328682 -6.9590908299204\\
21.8536376953683 -6.96241650556181\\
21.8780517578684 -6.9639208669775\\
21.9024658203684 -6.96548127682917\\
21.9268798828685 -6.96710178148118\\
21.9512939453686 -6.9687848632496\\
21.9757080078686 -6.96782843159997\\
22.0001220703687 -6.96682276217717\\
22.0245361328687 -6.96630948534187\\
22.0489501953688 -6.96624011618906\\
22.0733642578689 -6.96657454961456\\
22.0977783203689 -6.96727976984533\\
22.122192382869 -6.96832874360689\\
22.1466064453691 -6.96969949501317\\
22.1710205078691 -6.97552072907513\\
22.1954345703692 -6.97743466469984\\
22.2198486328692 -6.97959678716438\\
22.2442626953693 -6.98199910213799\\
22.2686767578694 -6.98463573183153\\
22.2930908203694 -6.99319834954144\\
22.3175048828695 -6.99371363728352\\
22.3419189453695 -6.99465875346061\\
22.3663330078696 -6.99586492780929\\
22.3907470703697 -6.99744557787307\\
22.4151611328697 -7.00786546745991\\
22.4395751953698 -7.00954326140811\\
22.4639892578699 -7.01669523353195\\
22.4884033203699 -7.02158322610865\\
22.51281738287 -7.02679964664065\\
22.5372314453701 -7.03874690669443\\
22.5616455078701 -7.04599796983405\\
22.5860595703702 -7.05183859187787\\
22.6104736328702 -7.05778987149512\\
22.6348876953703 -7.06385857181556\\
22.6593017578704 -7.07005020413806\\
22.6837158203704 -7.0763692302632\\
22.7081298828705 -7.08281922310766\\
22.7325439453705 -7.08940299539989\\
22.7569580078706 -7.09612270372497\\
22.7813720703707 -7.11124924815931\\
22.8057861328707 -7.12012822231605\\
22.8302001953708 -7.12907688311572\\
22.8546142578709 -7.13806713136631\\
22.8790283203709 -7.14707136827252\\
22.903442382871 -7.18471660784684\\
22.927856445371 -7.16738921657873\\
22.9522705078711 -7.17703792811221\\
22.9766845703712 -7.18664583513555\\
23.0010986328712 -7.196169145358\\
23.0255126953713 -7.20555902556129\\
23.0499267578714 -7.18562645490562\\
23.0743408203714 -7.19356049304565\\
23.0987548828715 -7.20161002060161\\
23.1231689453715 -7.20978614441628\\
23.1475830078716 -7.21810152830574\\
23.1719970703717 -7.22657051692062\\
23.1964111328717 -7.23520915370773\\
23.2208251953718 -7.24403496214529\\
23.2452392578719 -7.25306623523187\\
23.2696533203719 -7.25875804271851\\
23.294067382872 -7.26038446005765\\
23.318481445372 -7.26337028280449\\
23.3428955078721 -7.26599690236816\\
23.3673095703722 -7.26825627294977\\
23.3917236328722 -7.27013782266272\\
23.4161376953723 -7.25639869870906\\
23.4405517578724 -7.25899844340313\\
23.4649658203724 -7.2616178656607\\
23.4893798828725 -7.26426123560924\\
23.5137939453725 -7.26693324270684\\
23.5382080078726 -7.26324433804785\\
23.5626220703727 -7.26600420046304\\
23.5870361328727 -7.26892840441883\\
23.6114501953728 -7.2720181021682\\
23.6358642578729 -7.27527434781604\\
23.6602783203729 -7.27869787804593\\
23.684692382873 -7.2822888145493\\
23.709106445373 -7.28604627418321\\
23.7335205078731 -7.2899678767036\\
23.7579345703732 -7.30187347153327\\
23.7823486328732 -7.30625904257914\\
23.8067626953733 -7.31067459373918\\
23.8311767578734 -7.31509979357816\\
23.8555908203734 -7.3195128775622\\
23.8800048828735 -7.32560492335136\\
23.9044189453735 -7.32568423389915\\
23.9288330078736 -7.33266965785907\\
23.9532470703737 -7.33976399187839\\
23.9776611328737 -7.34468509106745\\
24.0020751953738 -7.35425862093808\\
24.0264892578738 -7.3615721029791\\
24.0509033203739 -7.36890045549904\\
24.075317382874 -7.37492424039539\\
24.099731445374 -7.37926453244125\\
24.1241455078741 -7.38337581487656\\
24.1485595703742 -7.38726684290071\\
24.1729736328742 -7.39095013050604\\
24.1973876953743 -7.39444267224144\\
24.2218017578744 -7.39776677247096\\
24.2462158203744 -7.40547115583149\\
24.2706298828745 -7.38720091775701\\
24.2950439453745 -7.39381252542785\\
24.3194580078746 -7.40070887084971\\
24.3438720703747 -7.4078133812161\\
24.3682861328747 -7.41505668716849\\
24.3927001953748 -7.4223775483967\\
24.4171142578749 -7.42972314245816\\
24.4415283203749 -7.43704888170055\\
24.465942382875 -7.44431790899108\\
24.490356445375 -7.45067151420488\\
24.5147705078751 -7.45662987028311\\
24.5391845703752 -7.4625084879404\\
24.5635986328752 -7.46833915298032\\
24.5880126953753 -7.47415022211254\\
24.6124267578753 -7.5020090612677\\
24.6368408203754 -7.50390214026123\\
24.6612548828755 -7.50879618768801\\
24.6856689453755 -7.51378704493652\\
24.7100830078756 -7.51886971716158\\
24.7344970703757 -7.52403924520758\\
24.7589111328757 -7.52929075264922\\
24.7833251953758 -7.53461947179073\\
24.8077392578758 -7.54002075515706\\
24.8321533203759 -7.54549007706987\\
24.856567382876 -7.55056042356534\\
24.880981445376 -7.55546067928248\\
24.9053955078761 -7.56043592278519\\
24.9298095703762 -7.56549437633072\\
24.9542236328762 -7.57064324511299\\
24.9786376953763 -7.58651618450386\\
25.0030517578763 -7.59276153505527\\
25.0274658203764 -7.598982674578\\
25.0518798828765 -7.60518163595947\\
25.0762939453765 -7.61135947469688\\
25.1007080078766 -7.61929602994184\\
25.1251220703767 -7.62551477380106\\
25.1495361328767 -7.63165726061999\\
25.1739501953768 -7.63771621113102\\
25.1983642578768 -7.6436836397254\\
25.2227783203769 -7.65118682718048\\
25.247192382877 -7.65657531617143\\
25.271606445377 -7.66177458086639\\
25.2960205078771 -7.66677620027218\\
25.3204345703772 -7.6715118857054\\
25.3448486328772 -7.67452227306046\\
25.3692626953773 -7.67739639397092\\
25.3936767578773 -7.68016426175893\\
25.4180908203774 -7.68285760482433\\
25.4425048828775 -7.6855099552036\\
25.4669189453775 -7.70120709155469\\
25.4913330078776 -7.70701415343689\\
25.5157470703777 -7.71278480469539\\
25.5401611328777 -7.71852159832218\\
25.5645751953778 -7.72422803291295\\
25.5889892578778 -7.73380524484207\\
25.6134033203779 -7.72298782470881\\
25.637817382878 -7.72932256675313\\
25.662231445378 -7.73587307897913\\
25.6866455078781 -7.74265325968105\\
25.7110595703782 -7.74968018466848\\
25.7354736328782 -7.7556056800297\\
25.7598876953783 -7.7629844695482\\
25.7843017578783 -7.77060565803606\\
25.8087158203784 -7.77847681153594\\
25.8331298828785 -7.78660579040201\\
25.8575439453785 -7.79500058294462\\
25.8819580078786 -7.80366899685652\\
25.9063720703787 -7.81017508796418\\
25.9307861328787 -7.8159108399156\\
25.9552001953788 -7.82160524862212\\
25.9796142578788 -7.82726938806212\\
26.0040283203789 -7.83291682380474\\
26.028442382879 -7.83856412481572\\
26.052856445379 -7.84423150279702\\
26.0772705078791 -7.8526032124729\\
26.1016845703791 -7.85476349998669\\
26.1260986328792 -7.86206415856009\\
26.1505126953793 -7.86942280227299\\
26.1749267578793 -7.8764192012402\\
26.1993408203794 -7.88428014902627\\
26.2237548828795 -7.89176351426192\\
26.2481689453795 -7.89927409606753\\
26.2725830078796 -7.90680563192882\\
26.2969970703796 -7.91435234858278\\
26.3214111328797 -7.92884148288783\\
26.3458251953798 -7.92537727707304\\
26.3702392578798 -7.92994042745033\\
26.3946533203799 -7.93468950079136\\
26.41906738288 -7.93964776075512\\
26.44348144538 -7.94483909538548\\
26.4678955078801 -7.95028825201842\\
26.4923095703802 -7.95602103181488\\
26.5167236328802 -7.96206444127671\\
26.5411376953803 -7.96844679301671\\
26.5655517578803 -8.0114881577814\\
26.5899658203804 -8.01227233274938\\
26.6143798828805 -8.01870472375509\\
26.6387939453805 -8.02514456466025\\
26.6632080078806 -8.03050487489071\\
26.6876220703806 -8.03812676879625\\
26.7120361328807 -8.04471026381952\\
26.7364501953808 -8.05138367357711\\
26.7608642578808 -8.05816902142089\\
26.7852783203809 -8.06508899638362\\
26.809692382881 -8.07887118632868\\
26.834106445381 -8.08638429731604\\
26.8585205078811 -8.09398281140204\\
26.8829345703811 -8.10167247833999\\
26.9073486328812 -8.10945749227636\\
26.9317626953813 -8.13128410038672\\
26.9561767578813 -8.13869640318353\\
26.9805908203814 -8.14587278593353\\
27.0050048828815 -8.15281566097303\\
27.0294189453815 -8.15953034479816\\
27.0538330078816 -8.19155188139416\\
27.0782470703816 -8.1960556880523\\
27.1026611328817 -8.20000083459593\\
27.1270751953818 -8.2034271407764\\
27.1514892578818 -8.20638066652037\\
27.1759033203819 -8.21545187890206\\
27.200317382882 -8.2135711968362\\
27.224731445382 -8.21803136740095\\
27.2491455078821 -8.22149903798658\\
27.2735595703821 -8.22322776074657\\
27.2979736328822 -8.23056336368739\\
27.3223876953823 -8.2345036259922\\
27.3468017578823 -8.23834769634345\\
27.3712158203824 -8.24210897354544\\
27.3956298828825 -8.24580109208529\\
27.4200439453825 -8.2560584842513\\
27.4444580078826 -8.2539660504786\\
27.4688720703826 -8.258325099525\\
27.4932861328827 -8.26270510318534\\
27.5177001953828 -8.26711750114418\\
27.5421142578828 -8.27157345234065\\
27.5665283203829 -8.27530801582992\\
27.590942382883 -8.27996234824906\\
27.615356445383 -8.28467855812062\\
27.6397705078831 -8.28946729879902\\
27.6641845703831 -8.29433929462471\\
27.6885986328832 -8.29930541423749\\
27.7130126953833 -8.30437674498791\\
27.7374267578833 -8.30956466890422\\
27.7618408203834 -8.31488094083556\\
27.7862548828835 -8.32605954526872\\
27.8106689453835 -8.32983440175104\\
27.8350830078836 -8.33695627047194\\
27.8594970703836 -8.34270770258689\\
27.8839111328837 -8.3486688637015\\
27.9083251953838 -8.36329694456589\\
27.9327392578838 -8.36743993666676\\
27.9571533203839 -8.37634133739402\\
27.981567382884 -8.38504273927754\\
28.005981445384 -8.39353609021572\\
28.0303955078841 -8.40181614966411\\
28.0548095703841 -8.40987995265732\\
28.0792236328842 -8.41772637054601\\
28.1036376953843 -8.42535575078529\\
28.1280517578843 -8.43276962136497\\
28.1524658203844 -8.43935752561515\\
28.1768798828844 -8.44479778473539\\
28.2012939453845 -8.45013731251947\\
28.2257080078846 -8.45540032376022\\
28.2501220703846 -8.46060528390348\\
28.2745361328847 -8.46513155439686\\
28.2989501953848 -8.46692140611515\\
28.3233642578848 -8.46899111463983\\
28.3477783203849 -8.47139688212695\\
28.372192382885 -8.47412091122203\\
28.396606445385 -8.47714891882168\\
28.4210205078851 -8.48046953630256\\
28.4454345703851 -8.48407380915512\\
28.4698486328852 -8.48795478258188\\
28.4942626953853 -8.49210715896868\\
28.5186767578853 -8.51052115265219\\
28.5430908203854 -8.5146814740167\\
28.5675048828854 -8.51903909172995\\
28.5919189453855 -8.52359457823547\\
28.6163330078856 -8.52834873920574\\
28.6407470703856 -8.54160128638858\\
28.6651611328857 -8.5462868637228\\
28.6895751953858 -8.55107724470304\\
28.7139892578858 -8.55597915957634\\
28.7384033203859 -8.56099911552911\\
28.7628173828859 -8.57518065791214\\
28.787231445386 -8.58017489554203\\
28.8116455078861 -8.58523821933198\\
28.8360595703861 -8.59038350059598\\
28.8604736328862 -8.59562288762072\\
28.8848876953863 -8.60473493387951\\
28.9093017578863 -8.59589147863488\\
28.9337158203864 -8.60240701143448\\
28.9581298828864 -8.60912314185202\\
28.9825439453865 -8.61604260663341\\
29.0069580078866 -8.62316951160005\\
29.0313720703866 -8.63050965650842\\
29.0557861328867 -8.63807093555323\\
29.0802001953868 -8.64586383640903\\
29.1046142578868 -8.65390206898019\\
29.1290283203869 -8.66232383774059\\
29.1534423828869 -8.66833002174741\\
29.177856445387 -8.67429981798546\\
29.2022705078871 -8.68021878558687\\
29.2266845703871 -8.68606994309386\\
29.2510986328872 -8.69473146008664\\
29.2755126953873 -8.69791606297849\\
29.2999267578873 -8.70076230642263\\
29.3243408203874 -8.70327879735601\\
29.3487548828874 -8.70547848959563\\
29.3731689453875 -8.71109502236254\\
29.3975830078876 -8.71256921954068\\
29.4219970703876 -8.71407658244954\\
29.4464111328877 -8.71566857406856\\
29.4708251953878 -8.71739822772395\\
29.4952392578878 -8.76725985346627\\
29.5196533203879 -8.77061168845876\\
29.5440673828879 -8.77383408091094\\
29.568481445388 -8.77693952706073\\
29.5928955078881 -8.7799409489615\\
29.6173095703881 -8.79610967648807\\
29.6417236328882 -8.79972779420649\\
29.6661376953883 -8.80333955993979\\
29.6905517578883 -8.8069636660934\\
29.7149658203884 -8.81061949468449\\
29.7393798828884 -8.82469489569382\\
29.7637939453885 -8.82622890658121\\
29.7882080078886 -8.832539147989\\
29.8126220703886 -8.83898674814486\\
29.8370361328887 -8.84553913450856\\
29.8614501953888 -8.85228345305296\\
29.8858642578888 -8.85449182734527\\
29.9102783203889 -8.8629006669135\\
29.9346923828889 -8.8713668370732\\
29.959106445389 -8.87988041659111\\
29.9835205078891 -8.88842978461759\\
30.0079345703891 -8.89700137642444\\
30.0323486328892 -8.90557948864125\\
30.0567626953892 -8.9141461278445\\
30.0811767578893 -8.9226809001961\\
30.1055908203894 -8.93765189753526\\
30.1300048828894 -8.94662968628325\\
30.1544189453895 -8.95544274567231\\
30.1788330078896 -8.96405665102367\\
30.2032470703896 -8.97243257256133\\
30.2276611328897 -8.98412269385915\\
30.2520751953898 -8.98879535858845\\
30.2764892578898 -8.99327130269768\\
30.3009033203899 -8.99756796816877\\
30.3253173828899 -9.00170147581781\\
30.34973144539 -9.01189339773958\\
30.3741455078901 -8.96723304976086\\
30.3985595703901 -8.97124758273503\\
30.4229736328902 -8.97536661188599\\
30.4473876953902 -8.97959912853906\\
30.4718017578903 -8.98395338850139\\
30.4962158203904 -8.98843698509012\\
30.5206298828904 -8.99305690445118\\
30.5450439453905 -8.99781956612901\\
30.5694580078906 -9.00273085126332\\
30.5938720703906 -9.03044988992127\\
30.6182861328907 -9.03124821481879\\
30.6427001953907 -9.03229945901069\\
30.6671142578908 -9.0336477953004\\
30.6915283203909 -9.03533530604363\\
30.7159423828909 -9.05529484241015\\
30.740356445391 -9.05380647702634\\
30.7647705078911 -9.05656252518128\\
30.7891845703911 -9.05936471588751\\
30.8135986328912 -9.06149111239046\\
30.8380126953912 -9.06516859701355\\
30.8624267578913 -9.06820076962907\\
30.8868408203914 -9.07134044509814\\
30.9112548828914 -9.07460369008948\\
30.9356689453915 -9.07800724072968\\
30.9600830078916 -9.10290781309658\\
30.9844970703916 -9.10555943442915\\
31.0089111328917 -9.11016834665531\\
31.0333251953917 -9.11394723599199\\
31.0577392578918 -9.11726055043397\\
31.0821533203919 -9.12337382649906\\
31.1065673828919 -9.12758226985579\\
31.130981445392 -9.13170745388487\\
31.1553955078921 -9.13576072159765\\
31.1798095703921 -9.13975534982933\\
31.2042236328922 -9.1596261864259\\
31.2286376953922 -9.16278861741779\\
31.2530517578923 -9.16581111890389\\
31.2774658203924 -9.16870121024567\\
31.3018798828924 -9.17146623794298\\
31.3262939453925 -9.17334774558583\\
31.3507080078926 -9.17522461865599\\
31.3751220703926 -9.17715186142692\\
31.3995361328927 -9.17913687898374\\
31.4239501953927 -9.18118715935776\\
31.4483642578928 -9.18599102016594\\
31.4727783203929 -9.18850063395735\\
31.4971923828929 -9.19107153165885\\
31.521606445393 -9.19371297648456\\
31.5460205078931 -9.19643457509386\\
31.5704345703931 -9.20444870856255\\
31.5948486328932 -9.20492723471096\\
31.6192626953932 -9.20522142743118\\
31.6436767578933 -9.20535894082013\\
31.6680908203934 -9.2053729793173\\
31.6925048828934 -9.23178440895998\\
31.7169189453935 -9.23437628886052\\
31.7413330078936 -9.2371536424593\\
31.7657470703936 -9.24014560873388\\
31.7901611328937 -9.24337512715181\\
31.8145751953937 -9.26178839980344\\
31.8389892578938 -9.2658264513683\\
31.8634033203939 -9.27002690504889\\
31.8878173828939 -9.27440141501586\\
31.912231445394 -9.27896188929918\\
31.936645507894 -9.28945824099009\\
31.9610595703941 -9.29217663200552\\
31.9854736328942 -9.29485358188629\\
32.0098876953942 -9.29751045290647\\
32.0343017578943 -9.30016963817661\\
32.0587158203944 -9.31871727176974\\
32.0831298828944 -9.32368757367163\\
32.1075439453945 -9.3288131729787\\
32.1319580078946 -9.33411116844155\\
32.1563720703946 -9.33939394131613\\
32.1807861328947 -9.34430649520721\\
32.2052001953947 -9.34932280636497\\
32.2296142578948 -9.35445741406949\\
32.2540283203949 -9.3597256864897\\
32.2784423828949 -9.36514396571999\\
32.302856445395 -9.4002511028508\\
32.3272705078951 -9.40263681044291\\
32.3516845703951 -9.40909547329466\\
32.3760986328952 -9.41544564021314\\
32.4005126953952 -9.42169572387343\\
32.4249267578953 -9.42785391730638\\
32.4493408203954 -9.43392822959095\\
32.4737548828954 -9.43992652076003\\
32.4981689453955 -9.44585653624131\\
32.5225830078955 -9.4517259411582\\
32.5469970703956 -9.46982187231367\\
32.5714111328957 -9.47527299609811\\
32.5958251953957 -9.48062843064321\\
32.6202392578958 -9.48589408075748\\
32.6446533203959 -9.49107521937185\\
32.6690673828959 -9.49782285763507\\
32.693481445396 -9.50213115514305\\
32.717895507896 -9.5063620357173\\
32.7423095703961 -9.51052302725759\\
32.7667236328962 -9.51462138489654\\
32.7911376953962 -9.52092834773623\\
32.8155517578963 -9.52459726860531\\
32.8399658203964 -9.52817420735414\\
32.8643798828964 -9.5316802401739\\
32.8887939453965 -9.5351346652522\\
32.9132080078965 -9.5396467754066\\
32.9376220703966 -9.54291883673653\\
32.9620361328967 -9.54619659128468\\
32.9864501953967 -9.54948763128522\\
33.0108642578968 -9.55279865739163\\
33.0352783203969 -9.55899428067863\\
33.0596923828969 -9.5620175763749\\
33.084106445397 -9.56498581851088\\
33.108520507897 -9.56791198549712\\
33.1329345703971 -9.57080921800542\\
33.1573486328972 -9.58476404901569\\
33.1817626953972 -9.57640671252051\\
33.2061767578973 -9.57969140834831\\
33.2305908203974 -9.58300119075085\\
33.2550048828974 -9.58634320304208\\
33.2794189453975 -9.58972436080523\\
33.3038330078975 -9.59315141551571\\
33.3282470703976 -9.59663101251346\\
33.3526611328977 -9.60016974451645\\
33.3770751953977 -9.60377420170499\\
33.4014892578978 -9.62565909098112\\
33.4259033203979 -9.62942386323107\\
33.4503173828979 -9.63316268217523\\
33.474731445398 -9.63687651263876\\
33.499145507898 -9.64056623430184\\
33.5235595703981 -9.64407169182306\\
33.5479736328982 -9.64111770180057\\
33.5723876953982 -9.64455417531877\\
33.5968017578983 -9.64801504054134\\
33.6212158203983 -9.65150677984576\\
33.6456298828984 -9.65503614388012\\
33.6700439453985 -9.6586101681163\\
33.6944580078985 -9.66223619379633\\
33.7188720703986 -9.66592189298586\\
33.7432861328987 -9.6687034282063\\
33.7677001953987 -9.67110434330641\\
33.7921142578988 -9.6736355816322\\
33.8165283203988 -9.67631521412954\\
33.8409423828989 -9.67916272190092\\
33.865356445399 -9.6821992240757\\
33.889770507899 -9.70697405110377\\
33.9141845703991 -9.71003879170433\\
33.9385986328992 -9.71317382012603\\
33.9630126953992 -9.71639446789993\\
33.9874267578993 -9.71971793938934\\
34.0118408203993 -9.72880299088195\\
34.0362548828994 -9.73155551307531\\
34.0606689453995 -9.734517290317\\
34.0850830078995 -9.73774776538221\\
34.1094970703996 -9.74131002949395\\
34.1339111328997 -9.75089804270493\\
34.1583251953997 -9.75612267946155\\
34.1827392578998 -9.76169746135849\\
34.2071533203999 -9.76767706873369\\
34.2315673828999 -9.77411707737781\\
34.2559814454 -9.7804382262272\\
34.2803955079 -9.78625294424825\\
34.3048095704001 -9.79210875985808\\
34.3292236329002 -9.79799360199695\\
34.3536376954002 -9.8038940077854\\
34.3780517579003 -9.82035865476996\\
34.4024658204003 -9.82221799910934\\
34.4268798829004 -9.82697449910779\\
34.4512939454005 -9.83162711904367\\
34.4757080079005 -9.83617296128796\\
34.5001220704006 -9.84060859448385\\
34.5245361329006 -9.84493009675339\\
34.5489501954007 -9.84709105236044\\
34.5733642579008 -9.84887541261003\\
34.5977783204008 -9.85056991633309\\
34.6221923829009 -9.8522096933051\\
34.646606445401 -9.85382719480697\\
34.671020507901 -9.85545237464743\\
34.6954345704011 -9.8571129144982\\
34.7198486329012 -9.85883447437527\\
34.7442626954012 -9.86340435985062\\
34.7686767579013 -9.86436775055029\\
34.7930908204013 -9.86547676432633\\
34.8175048829014 -9.86675707345654\\
34.8419189454015 -9.86823476433597\\
34.8663330079015 -9.87899593798141\\
34.8907470704016 -9.88141948165435\\
34.9151611329017 -9.88400754098484\\
34.9395751954017 -9.88676733629347\\
34.9639892579018 -9.88970569526663\\
34.9884033204018 -9.89622993807853\\
35.0128173829019 -9.89990395324241\\
35.037231445402 -9.90372843374894\\
35.061645507902 -9.90770237740509\\
35.0860595704021 -9.91182459400716\\
35.1104736329022 -9.91770988073728\\
35.1348876954022 -9.92120485931179\\
35.1593017579023 -9.92616126021579\\
35.1837158204023 -9.93098746604545\\
35.2081298829024 -9.93570065946614\\
35.2325439454025 -9.94201734516275\\
35.2569580079025 -9.94763593204874\\
35.2813720704026 -9.95342257725213\\
35.3057861329027 -9.9593774506975\\
35.3302001954027 -9.96550002645387\\
35.3546142579028 -9.97374754115356\\
35.3790283204028 -9.98046886585604\\
35.4034423829029 -9.98728262801808\\
35.427856445403 -9.99417131124682\\
35.452270507903 -10.000590672796\\
35.4766845704031 -10.0060127338825\\
35.5010986329032 -10.0114522075458\\
35.5255126954032 -10.0169125981489\\
35.5499267579033 -10.0223971961871\\
35.5743408204033 -10.0279090948622\\
35.5987548829034 -10.0332059984017\\
35.6231689454035 -10.0382136074396\\
35.6475830079035 -10.0432338526151\\
35.6719970704036 -10.0482759221916\\
35.6964111329037 -10.0533486579336\\
35.7208251954037 -10.0674416344402\\
35.7452392579038 -10.0714597462671\\
35.7696533204038 -10.0755163631288\\
35.7940673829039 -10.0796418707881\\
35.818481445404 -10.083864975748\\
35.842895507904 -10.0968374851338\\
35.8673095704041 -10.1014168002835\\
35.8917236329041 -10.1059956573992\\
35.9161376954042 -10.1105866562229\\
35.9405517579043 -10.115201009628\\
35.9649658204043 -10.1260099762108\\
35.9893798829044 -10.1301295987727\\
36.0137939454045 -10.1348271921743\\
36.0382080079045 -10.1394133785702\\
36.0626220704046 -10.1438938818077\\
36.0870361329047 -10.1482748907385\\
36.1114501954047 -10.1525628965254\\
36.1358642579048 -10.1567645691421\\
36.1602783204048 -10.1608866648024\\
36.1846923829049 -10.1649359577062\\
36.209106445405 -10.1708244826144\\
36.233520507905 -10.1742650759412\\
36.2579345704051 -10.1776749599849\\
36.2823486329051 -10.1810631666263\\
36.3067626954052 -10.1844385608556\\
36.3311767579053 -10.1949525425665\\
36.3555908204053 -10.1955916738801\\
36.3800048829054 -10.1998011734162\\
36.4044189454055 -10.2038727313713\\
36.4288330079055 -10.2065833481258\\
36.4532470704056 -10.212383231758\\
36.4776611329056 -10.2165951622591\\
36.5020751954057 -10.2208380516359\\
36.5264892579058 -10.2251296949276\\
36.5509033204058 -10.2286107443056\\
36.5753173829059 -10.2308632857515\\
36.599731445406 -10.2329657866724\\
36.624145507906 -10.2349278277908\\
36.6485595704061 -10.23676265888\\
36.6729736329061 -10.2384883799505\\
36.6973876954062 -10.2595010562511\\
36.7218017579063 -10.2621891652851\\
36.7462158204063 -10.265769444789\\
36.7706298829064 -10.2688490291174\\
36.7950439454065 -10.2719184986789\\
36.8194580079065 -10.2775271197324\\
36.8438720704066 -10.2700082444699\\
36.8682861329066 -10.2754579583052\\
36.8927001954067 -10.280909700202\\
36.9171142579068 -10.2863206263586\\
36.9415283204068 -10.2916477857423\\
36.9659423829069 -10.2968498983946\\
36.990356445407 -10.3018893763483\\
37.014770507907 -10.3067344316087\\
37.0391845704071 -10.3108367878479\\
37.0635986329071 -10.3147704468246\\
37.0880126954072 -10.3150964349044\\
37.1124267579073 -10.3175336540202\\
37.1368408204073 -10.3201881031838\\
37.1612548829074 -10.3230599245574\\
37.1856689454074 -10.3261487760549\\
37.2100830079075 -10.3294542710428\\
37.2344970704076 -10.3329762738469\\
37.2589111329076 -10.3367150968014\\
37.2833251954077 -10.3406716308233\\
37.3077392579078 -10.3556874128472\\
37.3321533204078 -10.358657380366\\
37.3565673829079 -10.3617907584307\\
37.380981445408 -10.3651076023568\\
37.405395507908 -10.3686267851452\\
37.4298095704081 -10.3823613149962\\
37.4542236329081 -10.3733343721074\\
37.4786376954082 -10.3789908313216\\
37.5030517579083 -10.3849875201415\\
37.5274658204083 -10.3913223575896\\
37.5518798829084 -10.3979894862377\\
37.5762939454085 -10.4049787399855\\
37.6007080079085 -10.4122751067952\\
37.6251220704086 -10.4198582050857\\
37.6495361329086 -10.4277017988256\\
37.6739501954087 -10.4447453525028\\
37.6983642579088 -10.4514075956529\\
37.7227783204088 -10.4581466330949\\
37.7471923829089 -10.4649406316025\\
37.7716064454089 -10.471231825417\\
37.796020507909 -10.4759600835467\\
37.8204345704091 -10.4806730143217\\
37.8448486329091 -10.4853770624333\\
37.8692626954092 -10.4900778778818\\
37.8936767579093 -10.4947803447255\\
37.9180908204093 -10.506963508781\\
37.9425048829094 -10.5110270964983\\
37.9669189454094 -10.5150900378118\\
37.9913330079095 -10.5191533793922\\
38.0157470704096 -10.5232179005479\\
38.0401611329096 -10.531255945401\\
38.0645751954097 -10.5343219764595\\
38.0889892579098 -10.5364439255039\\
38.1134033204098 -10.5386682128398\\
38.1378173829099 -10.5410031518609\\
38.1622314454099 -10.5434559993417\\
38.18664550791 -10.5460332574692\\
38.2110595704101 -10.5487408920919\\
38.2354736329101 -10.5515844911786\\
38.2598876954102 -10.5545693801547\\
38.2843017579103 -10.5703073442768\\
38.3087158204103 -10.5721223941542\\
38.3331298829104 -10.5750410466646\\
38.3575439454104 -10.5780455954294\\
38.3819580079105 -10.5811420420754\\
38.4063720704106 -10.5843359776781\\
38.4307861329106 -10.5876326203608\\
38.4552001954107 -10.5910368460222\\
38.4796142579108 -10.5945532133146\\
38.5040283204108 -10.5981859837893\\
38.5284423829109 -10.6028823688971\\
38.5528564454109 -10.6029595278929\\
38.577270507911 -10.6080068787325\\
38.6016845704111 -10.6131083835402\\
38.6260986329111 -10.6182588338297\\
38.6505126954112 -10.6234537133569\\
38.6749267579113 -10.6286890859571\\
38.6993408204113 -10.6339615044872\\
38.7237548829114 -10.6392679363813\\
38.7481689454114 -10.6446057023895\\
38.7725830079115 -10.6539165714154\\
38.7969970704116 -10.6527733619782\\
38.8214111329116 -10.6592988923023\\
38.8458251954117 -10.6658823748814\\
38.8702392579118 -10.6725098951346\\
38.8946533204118 -10.6791679501331\\
38.9190673829119 -10.6838200689559\\
38.9434814454119 -10.6890454126656\\
38.967895507912 -10.6942979048804\\
38.9923095704121 -10.6995831965744\\
39.0167236329121 -10.7049080549742\\
39.0411376954122 -10.7093465316982\\
39.0655517579123 -10.7151792318989\\
39.0899658204123 -10.7210691481417\\
39.1143798829124 -10.7267918094983\\
39.1387939454124 -10.7330181602226\\
39.1632080079125 -10.735294632774\\
39.1876220704126 -10.7404176219526\\
39.2120361329126 -10.7455278316368\\
39.2364501954127 -10.7506236944225\\
39.2608642579128 -10.7557033669368\\
39.2852783204128 -10.7607646376476\\
39.3096923829129 -10.7658047996202\\
39.3341064454129 -10.7708204723305\\
39.358520507913 -10.775807348203\\
39.3829345704131 -10.7874095107624\\
39.4073486329131 -10.7889709522188\\
39.4317626954132 -10.7920898125801\\
39.4561767579133 -10.794981482258\\
39.4805908204133 -10.7976378113547\\
39.5050048829134 -10.8000501765279\\
39.5294189454134 -10.8022104218948\\
39.5538330079135 -10.804112753019\\
39.5782470704136 -10.8057573893299\\
39.6026611329136 -10.8071574507595\\
39.6270751954137 -10.8292649582384\\
39.6514892579137 -10.8294356592007\\
39.6759033204138 -10.8346088612843\\
39.7003173829139 -10.8398579100044\\
39.7247314454139 -10.8451822307424\\
39.749145507914 -10.850581265283\\
39.7735595704141 -10.8560544714364\\
39.7979736329141 -10.8616013204241\\
39.8223876954142 -10.8672212911459\\
39.8468017579142 -10.8729138601816\\
39.8712158204143 -10.880984954186\\
39.8956298829144 -10.8869389898114\\
39.9200439454144 -10.8929013371107\\
39.9444580079145 -10.8988687197062\\
39.9688720704146 -10.9048373530615\\
39.9932861329146 -10.9097355105217\\
40.0177001954147 -10.9143259232836\\
40.0421142579147 -10.9189572294247\\
40.0665283204148 -10.9236354135904\\
40.0909423829149 -10.9283660203134\\
40.1153564454149 -10.9376509958957\\
40.139770507915 -10.9421960699922\\
40.1641845704151 -10.9467873583175\\
40.1885986329151 -10.9514267094456\\
40.2130126954152 -10.9561157194148\\
40.2374267579152 -10.9607260666051\\
40.2618408204153 -10.964928976406\\
40.2862548829154 -10.9692232041323\\
40.3106689454154 -10.9736135003616\\
40.3350830079155 -10.9781041258109\\
40.3594970704156 -10.9858760862234\\
40.3839111329156 -10.9908194069672\\
40.4083251954157 -10.9958841400183\\
40.4327392579157 -11.0010648792407\\
40.4571533204158 -11.0063549871739\\
40.4815673829159 -11.0147866403839\\
40.5059814454159 -11.0207806352298\\
40.530395507916 -11.0268638334211\\
40.5548095704161 -11.0325949106797\\
40.5792236329161 -11.0367425497094\\
40.6036376954162 -11.0408181969863\\
40.6280517579162 -11.0443006481565\\
40.6524658204163 -11.048234457537\\
40.6768798829164 -11.052115145163\\
40.7012939454164 -11.055951881289\\
40.7257080079165 -11.0597534170495\\
40.7501220704166 -11.0635281212001\\
40.7745361329166 -11.0672840161016\\
40.7989501954167 -11.0710288126194\\
40.8233642579167 -11.074769943737\\
40.8477783204168 -11.0860227765451\\
40.8721923829169 -11.0898380242716\\
40.8966064454169 -11.093603869557\\
40.921020507917 -11.0973244369106\\
40.9454345704171 -11.10100363898\\
40.9698486329171 -11.108264466877\\
40.9942626954172 -11.1114216446758\\
41.0186767579172 -11.1145819014665\\
41.0430908204173 -11.1177578810867\\
41.0675048829174 -11.1209614116059\\
41.0919189454174 -11.1315881432155\\
41.1163330079175 -11.1310131186935\\
41.1407470704175 -11.1344141380529\\
41.1651611329176 -11.1376752152124\\
41.1895751954177 -11.1408071801355\\
41.2139892579177 -11.14382303043\\
41.2384033204178 -11.1467381055602\\
41.2628173829179 -11.1495703340428\\
41.2872314454179 -11.1523405549308\\
41.311645507918 -11.1550729142149\\
41.3360595704181 -11.1613749092668\\
41.3604736329181 -11.1647989232413\\
41.3848876954182 -11.1683191471008\\
41.4093017579182 -11.1719690021496\\
41.4337158204183 -11.1757804038904\\
41.4581298829184 -11.1818249011629\\
41.4825439454184 -11.1844156166522\\
41.5069580079185 -11.1883877046825\\
41.5313720704186 -11.1923970440565\\
41.5557861329186 -11.1964436648198\\
41.5802001954187 -11.2005282081182\\
41.6046142579187 -11.2046519231679\\
41.6290283204188 -11.20881666842\\
41.6534423829189 -11.2130249173305\\
41.6778564454189 -11.2172797691662\\
41.702270507919 -11.2253271007024\\
41.726684570419 -11.229370584257\\
41.7510986329191 -11.2334017136343\\
41.7755126954192 -11.2374261652599\\
41.7999267579192 -11.2414499050528\\
41.8243408204193 -11.2498094817722\\
41.8487548829194 -11.2533102146207\\
41.8731689454194 -11.2567680389062\\
41.8975830079195 -11.260189388225\\
41.9219970704195 -11.2635814802152\\
41.9464111329196 -11.2768376951478\\
41.9708251954197 -11.2813407301734\\
41.9952392579197 -11.2858295242672\\
42.0196533204198 -11.2903014331034\\
42.0440673829199 -11.2947536616384\\
42.0684814454199 -11.30612559203\\
42.09289550792 -11.304971987399\\
42.11730957042 -11.3083208601798\\
42.1417236329201 -11.3115739051804\\
42.1661376954202 -11.3147356292116\\
42.1905517579202 -11.3178104115345\\
42.2149658204203 -11.3208025214149\\
42.2393798829204 -11.3234967887233\\
42.2637939454204 -11.3251630812774\\
42.2882080079205 -11.3268568616139\\
42.3126220704205 -11.3285941801078\\
42.3370361329206 -11.3303908544014\\
42.3614501954207 -11.3322627797309\\
42.3858642579207 -11.3342262414844\\
42.4102783204208 -11.3362982396187\\
42.4346923829209 -11.3547942262321\\
42.4591064454209 -11.3571530320309\\
42.483520507921 -11.3594504937536\\
42.507934570421 -11.3616905235712\\
42.5323486329211 -11.363878248293\\
42.5567626954212 -11.3679881283505\\
42.5811767579212 -11.3698821977215\\
42.6055908204213 -11.3717550754362\\
42.6300048829214 -11.373621060176\\
42.6544189454214 -11.3754977191705\\
42.6788330079215 -11.3850720149459\\
42.7032470704215 -11.3881040747471\\
42.7276611329216 -11.3912750206185\\
42.7520751954217 -11.3946030008177\\
42.7764892579217 -11.3981081639713\\
42.8009033204218 -11.4093164336546\\
42.8253173829218 -11.4133825210548\\
42.8497314454219 -11.4178330535187\\
42.874145507922 -11.4227353633532\\
42.898559570422 -11.4281657678247\\
42.9229736329221 -11.438038412197\\
42.9473876954222 -11.4364861501098\\
42.9718017579222 -11.4401640609144\\
42.9962158204223 -11.443858170886\\
43.0206298829223 -11.4475721319804\\
43.0450439454224 -11.4513105529772\\
43.0694580079225 -11.4550792941929\\
43.0938720704225 -11.4588858753202\\
43.1182861329226 -11.4627400488382\\
43.1427001954227 -11.4666546203632\\
43.1671142579227 -11.5132021724157\\
43.1915283204228 -11.5178369697539\\
43.2159423829228 -11.521953056767\\
43.2403564454229 -11.5254796193011\\
43.264770507923 -11.5283381910883\\
43.289184570423 -11.5402481275597\\
43.3135986329231 -11.5446442294301\\
43.3380126954232 -11.54881005722\\
43.3624267579232 -11.5527384426071\\
43.3868408204233 -11.5564221638935\\
43.4112548829234 -11.5599291062051\\
43.4356689454234 -11.561448244093\\
43.4600830079235 -11.5627604816314\\
43.4844970704235 -11.5638810518366\\
43.5089111329236 -11.5648242201925\\
43.5333251954237 -11.5934554350618\\
43.5577392579237 -11.5926664954921\\
43.5821533204238 -11.593277642159\\
43.6065673829239 -11.5937876599286\\
43.6309814454239 -11.5942050742721\\
43.655395507924 -11.5945362890669\\
43.679809570424 -11.59143053112\\
43.7042236329241 -11.5884574345454\\
43.7286376954242 -11.5857572905562\\
43.7530517579242 -11.5832736263807\\
43.7774658204243 -11.5809604645111\\
43.8018798829243 -11.5787793087603\\
43.8262939454244 -11.5766969122795\\
43.8507080079245 -11.5746835451348\\
43.8751220704245 -11.5727115665976\\
43.8995361329246 -11.5850258724599\\
43.9239501954247 -11.58220667812\\
43.9483642579247 -11.5805719046881\\
43.9727783204248 -11.5788327228605\\
43.9971923829248 -11.5774000990971\\
44.0216064454249 -11.5781859252976\\
44.046020507925 -11.5781352219232\\
44.070434570425 -11.57847575688\\
44.0948486329251 -11.579226052077\\
44.1192626954252 -11.58041347147\\
44.1436767579252 -11.6177455849402\\
44.1680908204253 -11.6208437621464\\
44.1925048829253 -11.6241819033361\\
44.2169189454254 -11.6277836974747\\
44.2413330079255 -11.631674236349\\
44.2657470704255 -11.6462942029872\\
44.2901611329256 -11.6502525734469\\
44.3145751954257 -11.6542595616306\\
44.3389892579257 -11.6583128508635\\
44.3634033204258 -11.662409528261\\
44.3878173829258 -11.6799842092896\\
44.4122314454259 -11.682715496087\\
44.436645507926 -11.6855106615641\\
44.461059570426 -11.688370870675\\
44.4854736329261 -11.6912973357167\\
44.5098876954262 -11.7050343736951\\
44.5343017579262 -11.7067698649762\\
44.5587158204263 -11.708599728405\\
44.5831298829263 -11.7105211594522\\
44.6075439454264 -11.7125304127915\\
44.6319580079265 -11.7159729274833\\
44.6563720704265 -11.7166876788427\\
44.6807861329266 -11.7176427674752\\
44.7052001954267 -11.7188152224266\\
44.7296142579267 -11.7201868440459\\
44.7540283204268 -11.7212326847692\\
44.7784423829268 -11.7217910235725\\
44.8028564454269 -11.7224793076675\\
44.827270507927 -11.7232932374911\\
44.851684570427 -11.7242299838518\\
44.8760986329271 -11.7351278281803\\
44.9005126954271 -11.736483051635\\
44.9249267579272 -11.7379318238004\\
44.9493408204273 -11.7394780733027\\
44.9737548829273 -11.7411260725784\\
44.9981689454274 -11.7441373717256\\
45.0225830079275 -11.7461201123049\\
45.0469970704275 -11.7482081099234\\
45.0714111329276 -11.7504096398422\\
45.0958251954277 -11.7527335295564\\
45.1202392579277 -11.7611762849793\\
45.1446533204278 -11.7561325526469\\
45.1690673829278 -11.7591283116466\\
45.1934814454279 -11.7622452760927\\
45.217895507928 -11.765480596635\\
45.242309570428 -11.768831727596\\
45.2667236329281 -11.7559428465374\\
45.2911376954281 -11.7613203656865\\
45.3155517579282 -11.7668089988419\\
45.3399658204283 -11.7723907176088\\
45.3643798829283 -11.7780490565255\\
45.3887939454284 -11.7837690057074\\
45.4132080079285 -11.7895369073362\\
45.4376220704285 -11.7953403565925\\
45.4620361329286 -11.8011681074303\\
45.4864501954287 -11.8167491028198\\
45.5108642579287 -11.8205660322074\\
45.5352783204288 -11.8244221928877\\
45.5596923829288 -11.8283207767435\\
45.5841064454289 -11.832264984736\\
45.608520507929 -11.8419490184376\\
45.632934570429 -11.845829348263\\
45.6573486329291 -11.8497273628638\\
45.6817626954291 -11.8536417980292\\
45.7061767579292 -11.8575713022228\\
45.7305908204293 -11.8643155932216\\
45.7672119141794 -11.8688034889295\\
45.8038330079295 -11.8732233324081\\
45.8404541016796 -11.8775839458056\\
45.8770751954296 -11.8818940690496\\
45.9136962891797 -11.8979906626378\\
45.9625244141799 -11.9010621643436\\
46.01135253918 -11.9080202801902\\
46.0601806641801 -11.9148320004842\\
46.1090087891802 -11.9215132563112\\
46.1578369141804 -11.9280790570797\\
46.2066650391805 -11.934543594511\\
46.2554931641806 -11.9409203349767\\
46.3043212891807 -11.9472221012724\\
46.3531494141809 -11.953461144663\\
46.401977539181 -11.971095195106\\
46.4508056641811 -11.9766770744825\\
46.4996337891812 -11.9821510237443\\
46.5484619141814 -11.987540414236\\
46.5972900391815 -11.992867316704\\
46.6461181641816 -11.9996489294135\\
46.6949462891817 -12.0046933146611\\
46.7437744141819 -12.0095817752988\\
46.792602539182 -12.0143275232138\\
46.8414306641821 -12.0189450080837\\
46.8902587891822 -12.031313215443\\
46.9390869141824 -12.0370429426009\\
46.9879150391825 -12.0426860239002\\
47.0367431641826 -12.0482507976935\\
47.0855712891827 -12.0537466730827\\
47.1343994141829 -12.0612197946974\\
47.183227539183 -12.0597035072789\\
47.2320556641831 -12.0667820644107\\
47.2808837891832 -12.0739652258899\\
47.3297119141834 -12.0812487068816\\
47.3785400391835 -12.0886275783459\\
47.4273681641836 -12.09609613474\\
47.4761962891837 -12.1035760499499\\
47.5250244141839 -12.1101160164626\\
47.573852539184 -12.1166687499936\\
47.6226806641841 -12.1232382212079\\
47.6715087891842 -12.129828389217\\
47.7203369141843 -12.1364432049563\\
47.7691650391845 -12.1430866145104\\
47.8179931641846 -12.14976256238\\
47.8668212891847 -12.1583941382345\\
47.9156494141848 -12.1631230402612\\
47.964477539185 -12.1709276793909\\
48.0133056641851 -12.1781501897477\\
48.0621337891852 -12.1848324691478\\
48.1109619141854 -12.1943129787389\\
48.1597900391855 -12.202072153397\\
48.2086181641856 -12.2086494169174\\
48.2574462891857 -12.2148514898788\\
48.3062744141859 -12.221054410181\\
48.355102539186 -12.2272711903869\\
48.4039306641861 -12.2335149096107\\
48.4527587891862 -12.2397987458518\\
48.5015869141863 -12.2461360146953\\
48.5504150391865 -12.2525402132628\\
48.5992431641866 -12.2654062148615\\
48.6480712891867 -12.2726294612282\\
48.6968994141868 -12.2798787786287\\
48.745727539187 -12.2871553678764\\
48.7945556641871 -12.2944603323036\\
48.8433837891872 -12.302905620239\\
48.8922119141873 -12.3101492221496\\
48.9410400391875 -12.3173717758141\\
48.9898681641876 -12.3245724597801\\
49.0386962891877 -12.3315019449286\\
49.0875244141878 -12.3378114471634\\
49.136352539188 -12.34420445313\\
49.1851806641881 -12.3506897033342\\
49.2340087891882 -12.3572749869369\\
49.2828369141883 -12.3639672245108\\
49.3316650391885 -12.3790908780301\\
49.3804931641886 -12.3855522016176\\
49.4293212891887 -12.3919592497147\\
49.4781494141888 -12.398318789269\\
49.526977539189 -12.4046373480398\\
49.5758056641891 -12.4185808655438\\
49.6246337891892 -12.4244117898039\\
49.6734619141893 -12.4301583500813\\
49.7222900391895 -12.435829375469\\
49.7711181641896 -12.4414343612369\\
49.8199462891897 -12.4590348806585\\
49.8687744141898 -12.4570082054771\\
49.91760253919 -12.4642652146112\\
49.9664306641901 -12.4714456058291\\
50.0152587891902 -12.4785416348323\\
50.0640869141903 -12.4855446574752\\
50.1129150391905 -12.4834086004142\\
50.1617431641906 -12.4890569707269\\
50.2105712891907 -12.4945961821869\\
50.2593994141908 -12.5000256980706\\
50.308227539191 -12.5053448832214\\
50.3570556641911 -12.5105529398577\\
50.4058837891912 -12.5156488338944\\
50.4547119141913 -12.5206312079136\\
50.5035400391915 -12.525498275603\\
50.5523681641916 -12.5320417902817\\
50.6011962891917 -12.5368045845207\\
50.6500244141918 -12.5415549970446\\
50.698852539192 -12.546306025579\\
50.7476806641921 -12.5510715278648\\
50.7965087891922 -12.5648490676892\\
50.8453369141923 -12.5673159415901\\
50.8941650391925 -12.5727477300733\\
50.9429931641926 -12.5765476693291\\
50.9918212891927 -12.5802591064995\\
51.0406494141928 -12.589691862358\\
51.089477539193 -12.5954342327713\\
51.1383056641931 -12.6012335415924\\
51.1871337891932 -12.6070880908985\\
51.2359619141933 -12.6129958425255\\
51.2847900391935 -12.6266215793119\\
51.3336181641936 -12.6292513364253\\
51.3824462891937 -12.6344833197828\\
51.4312744141938 -12.6396955785266\\
51.480102539194 -12.6448934611469\\
51.5289306641941 -12.6500822065133\\
51.5777587891942 -12.6552669553582\\
51.6265869141943 -12.6604527621632\\
51.6754150391945 -12.6656446073188\\
51.7242431641946 -12.6708474094659\\
51.7730712891947 -12.6756636612954\\
51.8218994141948 -12.6801723379323\\
51.8707275391949 -12.6847432060941\\
51.9195556641951 -12.6893833647337\\
51.9683837891952 -12.6940997816893\\
52.0172119141953 -12.7121676203339\\
52.0660400391954 -12.7113641822223\\
52.1148681641956 -12.7175240940987\\
52.1636962891957 -12.7236026012293\\
52.2125244141958 -12.7295438887758\\
52.2613525391959 -12.735468034118\\
52.3101806641961 -12.7403827991225\\
52.3590087891962 -12.7456590063376\\
52.4078369141963 -12.7508071999455\\
52.4566650391965 -12.7558215683507\\
52.5054931641966 -12.7606986541798\\
52.5543212891967 -12.7654371067421\\
52.6031494141968 -12.7700374523574\\
52.6519775391969 -12.7743298138721\\
52.7008056641971 -12.7778020678336\\
52.7496337891972 -12.7812096996562\\
52.7984619141973 -12.7845704032693\\
52.8472900391974 -12.7879013226316\\
52.8961181641976 -12.7912190830347\\
52.9449462891977 -12.7945398242423\\
52.9937744141978 -12.8061385510255\\
53.0426025391979 -12.8047837490042\\
53.0914306641981 -12.8111098226994\\
53.1402587891982 -12.8174029909579\\
53.1890869141983 -12.8235263614357\\
53.2379150391984 -12.8298626142271\\
53.2867431641986 -12.8360184296971\\
53.3355712891987 -12.8421197046851\\
53.3843994141988 -12.8481636929714\\
53.4332275391989 -12.8518828403267\\
53.4820556641991 -12.8545109018493\\
53.5308837891992 -12.8568929634941\\
53.5797119141993 -12.8590368252681\\
53.6285400391995 -12.8609537264329\\
53.6773681641996 -12.8626614507186\\
53.7261962891997 -12.8770167459722\\
53.7750244141998 -12.8816502244497\\
53.8238525391999 -12.8863181198353\\
53.8726806642001 -12.8910374742375\\
53.9215087892002 -12.8958295246479\\
53.9703369142003 -12.9134917239885\\
54.0191650392004 -12.9176533666782\\
54.0679931642006 -12.9230833387873\\
54.1168212892007 -12.9285340917808\\
54.1656494142008 -12.9340050915081\\
54.2144775392009 -12.9394945760832\\
54.2633056642011 -12.9354646129456\\
54.3121337892012 -12.9420656822075\\
54.3609619142013 -12.9486997342704\\
54.4097900392014 -12.9553330934847\\
54.4586181642016 -12.9619319573303\\
54.5074462892017 -12.9684633099749\\
54.5562744142018 -12.9748958573371\\
54.6051025392019 -12.9812009357066\\
54.6539306642021 -12.9873533454598\\
54.7027587892022 -13.0063740507482\\
54.7515869142023 -13.0100815192043\\
54.8004150392024 -13.0136303194789\\
54.8492431642026 -13.0170508041651\\
54.8980712892027 -13.0203733841062\\
54.9468994142028 -13.0293220886471\\
54.9957275392029 -13.0319781176152\\
55.044555664203 -13.0346856126529\\
55.0933837892032 -13.0374724992353\\
55.1422119142033 -13.040363247965\\
55.1910400392034 -13.0539219384219\\
55.2398681642036 -13.043931711463\\
55.2886962892037 -13.049390757865\\
55.3375244142038 -13.054878708865\\
55.3863525392039 -13.0603858188162\\
55.4351806642041 -13.0659029575955\\
55.4840087892042 -13.0714215884256\\
55.5328369142043 -13.076933745079\\
55.5816650392044 -13.0824320086882\\
55.6304931642046 -13.0879094843577\\
55.6793212892047 -13.0952332578508\\
55.7281494142048 -13.1007475137123\\
55.7769775392049 -13.1061605146276\\
55.8258056642051 -13.1114659524012\\
55.8746337892052 -13.116658401448\\
55.9234619142053 -13.1254060089631\\
55.9722900392054 -13.1259775685789\\
56.0211181642055 -13.1295568464676\\
56.0699462892057 -13.1330876444464\\
56.1187744142058 -13.136585370816\\
56.1676025392059 -13.1400656540417\\
56.216430664206 -13.1435445260881\\
56.2652587892062 -13.1470386160124\\
56.3140869142063 -13.1505653579229\\
56.3629150392064 -13.1541432178339\\
56.4117431642065 -13.1593725015852\\
56.4605712892067 -13.1605833777451\\
56.5093994142068 -13.1669779687829\\
56.5582275392069 -13.1710414572038\\
56.607055664207 -13.1753113807473\\
56.6558837892072 -13.1884706145853\\
56.7047119142073 -13.196085277199\\
56.7535400392074 -13.2025567917531\\
56.8023681642076 -13.2074097960392\\
56.8511962892077 -13.212303514636\\
56.9000244142078 -13.21724542415\\
56.9488525392079 -13.2134703778671\\
56.997680664208 -13.2173884049409\\
57.0465087892082 -13.2213261280667\\
57.0953369142083 -13.2253006389532\\
57.1441650392084 -13.2293298369847\\
57.1929931642085 -13.2334325775328\\
57.2418212892087 -13.2376288348354\\
57.2906494142088 -13.2419398813936\\
57.3394775392089 -13.246388485992\\
57.388305664209 -13.279742839489\\
57.4371337892092 -13.2859186768774\\
57.4859619142093 -13.2922114193753\\
57.5347900392094 -13.2986314642428\\
57.5836181642095 -13.3051898237896\\
57.6324462892097 -13.3129960883411\\
57.6812744142098 -13.317414750724\\
57.7301025392099 -13.3230741307465\\
57.77893066421 -13.3288067329923\\
57.8277587892102 -13.3346192027338\\
57.8765869142103 -13.3405177198492\\
57.9254150392104 -13.3343975189738\\
57.9742431642105 -13.3447439201241\\
58.0230712892107 -13.3549923067956\\
58.0718994142108 -13.3650844683926\\
58.1207275392109 -13.3751691787765\\
58.169555664211 -13.3850840353213\\
58.2183837892112 -13.3948735660747\\
58.2672119142113 -13.4035984030627\\
58.3160400392114 -13.4118575050527\\
58.3648681642115 -13.4201018011978\\
58.4136962892117 -13.4283222687601\\
58.4625244142118 -13.4365101355048\\
58.5113525392119 -13.4446567713642\\
58.560180664212 -13.4527536028154\\
58.6090087892122 -13.4699022089007\\
58.6578369142123 -13.4595310371049\\
58.7066650392124 -13.4665536199092\\
58.7554931642125 -13.4736075248354\\
58.8043212892127 -13.4806895994477\\
58.8531494142128 -13.487796882946\\
58.9019775392129 -13.4933557524494\\
58.950805664213 -13.5000384355099\\
58.9996337892132 -13.506762006998\\
59.0484619142133 -13.5135309337777\\
59.0972900392134 -13.5203493387854\\
59.1461181642135 -13.527221046738\\
59.1949462892136 -13.5341496230631\\
59.2437744142138 -13.5411384071907\\
59.2926025392139 -13.5481905411363\\
59.341430664214 -13.5552348127945\\
59.3902587892142 -13.5599850895049\\
59.4390869142143 -13.5648492020133\\
59.4879150392144 -13.5698722972162\\
59.5367431642145 -13.5751011395769\\
59.5855712892146 -13.6022779087119\\
59.6343994142148 -13.6037008332311\\
59.6832275392149 -13.608859414044\\
59.732055664215 -13.6144615583554\\
59.7808837892152 -13.6204816092635\\
59.8297119142153 -13.6268959941415\\
59.8785400392154 -13.6309524850514\\
59.9273681642155 -13.637581393815\\
59.9761962892157 -13.6444443502226\\
60.0250244142158 -13.6515496318931\\
60.0738525392159 -13.6589058081718\\
60.122680664216 -13.6665217492426\\
60.1715087892161 -13.6744066351543\\
60.2203369142163 -13.6825699646832\\
60.2691650392164 -13.6910215639451\\
60.3179931642165 -13.7023165421865\\
60.3668212892166 -13.7102095057088\\
60.4156494142168 -13.7184756852797\\
60.4644775392169 -13.7271450107339\\
60.513305664217 -13.736246755866\\
60.5621337892171 -13.7589656157352\\
60.6109619142173 -13.7474415088063\\
60.6597900392174 -13.7559272964579\\
60.7086181642175 -13.7649690615543\\
60.7574462892176 -13.7745311974812\\
60.8062744142178 -13.7845702090211\\
60.8551025392179 -13.7950360923554\\
60.903930664218 -13.8058738685887\\
60.9527587892181 -13.8170251739327\\
61.0015869142183 -13.8284298161485\\
61.0504150392184 -13.8455220477695\\
61.0992431642185 -13.8527467845939\\
61.1480712892186 -13.8611128341075\\
61.1968994142188 -13.8695619706128\\
61.2457275392189 -13.8781054853204\\
61.294555664219 -13.8867541084195\\
61.3433837892191 -13.8955179226133\\
61.3922119142193 -13.9044062704706\\
61.4410400392194 -13.9134276562746\\
61.4898681642195 -13.9225896432826\\
61.5386962892196 -13.9451581116606\\
61.5875244142198 -13.950226188396\\
61.6363525392199 -13.9602616593767\\
61.68518066422 -13.9702001396703\\
61.7340087892201 -13.9800409422871\\
61.7828369142203 -13.9897841379583\\
61.8316650392204 -13.9994303555637\\
61.8804931642205 -14.0089805939724\\
61.9293212892206 -14.0184360414269\\
61.9781494142208 -14.0277978985678\\
62.0269775392209 -14.0367814238261\\
62.075805664221 -14.0455440192669\\
62.1246337892211 -14.0542926447409\\
62.1734619142213 -14.0630375081241\\
62.2222900392214 -14.0717878883575\\
62.2711181642215 -14.0802078803511\\
62.3199462892216 -14.0882779788924\\
62.3687744142218 -14.0962973647244\\
62.4176025392219 -14.1042737748811\\
62.466430664222 -14.1122145332652\\
62.5152587892221 -14.1310749644863\\
62.5640869142223 -14.1379905392104\\
62.6129150392224 -14.1450535685926\\
62.6617431642225 -14.1518565302082\\
62.7105712892226 -14.1583865953172\\
62.7593994142228 -14.1646294270885\\
62.8082275392229 -14.1705689433352\\
62.857055664223 -14.1761870768255\\
62.9058837892231 -14.1814635544179\\
62.9547119142233 -14.186375735859\\
63.0035400392234 -14.2228450024671\\
63.0523681642235 -14.2310273083963\\
63.1011962892236 -14.2390480963961\\
63.1500244142238 -14.2469070434938\\
63.1988525392239 -14.2546044543722\\
63.247680664224 -14.2653120979304\\
63.2965087892241 -14.2507886038838\\
63.3453369142242 -14.2603387336014\\
63.3941650392244 -14.2700765859049\\
63.4429931642245 -14.2799991435858\\
63.4918212892246 -14.2901026229988\\
63.5406494142247 -14.3003823961457\\
63.5894775392249 -14.3108329077419\\
63.638305664225 -14.3214475877264\\
63.6871337892251 -14.3322187600164\\
63.7359619142252 -14.3572881608952\\
63.7847900392254 -14.3657151589389\\
63.8336181642255 -14.3740723109814\\
63.8824462892256 -14.3823658435443\\
63.9312744142258 -14.390601020608\\
63.9801025392259 -14.4249225960768\\
64.028930664226 -14.4331747304508\\
64.0777587892261 -14.4412163574164\\
64.1265869142263 -14.4482920023249\\
64.1754150392264 -14.4530015362334\\
64.2242431642265 -14.4578988378151\\
64.2730712892266 -14.4608499087006\\
64.3218994142267 -14.4664810125858\\
64.3707275392269 -14.4721544883068\\
64.419555664227 -14.4778921823675\\
64.4683837892271 -14.4837126701187\\
64.5172119142272 -14.4896320759252\\
64.5660400392274 -14.4956646759976\\
64.6148681642275 -14.5018233476026\\
64.6636962892276 -14.5081199082614\\
64.7125244142277 -14.5206462391121\\
64.7613525392279 -14.5268640640436\\
64.810180664228 -14.5332284753472\\
64.8590087892281 -14.5397453345237\\
64.9078369142282 -14.5464205791255\\
64.9566650392284 -14.5724522065217\\
65.0054931642285 -14.569060206987\\
65.0543212892286 -14.5750195339334\\
65.1031494142288 -14.5811107755844\\
65.1519775392289 -14.5873180056537\\
65.200805664229 -14.5936275211816\\
65.2496337892291 -14.6000274726681\\
65.2984619142292 -14.6065075642682\\
65.3472900392294 -14.6130588092978\\
65.3961181642295 -14.6196733296909\\
65.4449462892296 -14.6302143405498\\
65.4937744142297 -14.6361129795022\\
65.5426025392298 -14.6421167592917\\
65.59143066423 -14.6482357917973\\
65.6402587892301 -14.6544776851598\\
65.6890869142302 -14.6603252670405\\
65.7379150392304 -14.6659293964267\\
65.7867431642305 -14.671705709458\\
65.8355712892306 -14.6776655832513\\
65.8843994142307 -14.6838191674181\\
65.9332275392309 -14.7018874145224\\
65.982055664231 -14.6959274896238\\
66.0308837892311 -14.7036512743779\\
66.0797119142312 -14.7114437866217\\
66.1285400392314 -14.7193027485973\\
66.1773681642315 -14.7272270442783\\
66.2261962892316 -14.7352164268488\\
66.2750244142317 -14.7432712979719\\
66.3238525392319 -14.7513925407624\\
66.372680664232 -14.7595813930502\\
66.4215087892321 -14.7706246246545\\
66.4703369142322 -14.7765104096212\\
66.5191650392323 -14.78566403318\\
66.5679931642325 -14.794830570919\\
66.6168212892326 -14.8030821484153\\
66.6656494142327 -14.8133692837657\\
66.7144775392329 -14.8226244485799\\
66.763305664233 -14.8311206991771\\
66.8121337892331 -14.8393249080556\\
66.8609619142332 -14.847442044541\\
66.9097900392334 -14.8554740392741\\
66.9586181642335 -14.8623407448226\\
67.0074462892336 -14.8702814340539\\
67.0562744142337 -14.8781967065323\\
67.1051025392339 -14.8860891648698\\
67.153930664234 -14.8939612468342\\
67.2027587892341 -14.9018152117323\\
67.2515869142342 -14.9096531250221\\
67.3004150392344 -14.9174768409685\\
67.3492431642345 -14.9252879831388\\
67.3980712892346 -14.9491270368534\\
67.4468994142347 -14.9554785878083\\
67.4957275392348 -14.9623961460787\\
67.544555664235 -14.9691264042132\\
67.5933837892351 -14.9756746784413\\
67.6422119142352 -14.9820469471887\\
67.6910400392353 -14.9882502697637\\
67.7398681642355 -14.994293444861\\
67.7886962892356 -15.0001880546644\\
67.8375244142357 -15.0059501333367\\
67.8863525392358 -15.0157455190404\\
67.935180664236 -15.0200905204198\\
67.9840087892361 -15.0273796201627\\
68.0328369142362 -15.0330262368926\\
68.0816650392363 -15.038569464037\\
68.1304931642365 -15.0492670455551\\
68.1793212892366 -15.0516519287751\\
68.2281494142367 -15.05731756624\\
68.2769775392369 -15.0628184079162\\
68.325805664237 -15.0681457847384\\
68.3746337892371 -15.0732906316472\\
68.4234619142372 -15.0782436013964\\
68.4722900392373 -15.0829952564415\\
68.5211181642375 -15.0875363724163\\
68.5699462892376 -15.0918583992591\\
68.6187744142377 -15.1047773654095\\
68.6676025392378 -15.10550901417\\
68.716430664238 -15.1110435078812\\
68.7652587892381 -15.1165798810476\\
68.8140869142382 -15.1221070100023\\
68.8629150392383 -15.1276165233227\\
68.9117431642385 -15.1331027122878\\
68.9605712892386 -15.1385624341543\\
69.0093994142387 -15.143995016481\\
69.0582275392388 -15.1494021697093\\
69.107055664239 -15.1569520843676\\
69.1558837892391 -15.1616806570904\\
69.2047119142392 -15.1664302683866\\
69.2535400392393 -15.1712180391245\\
69.3023681642395 -15.1760630628195\\
69.3511962892396 -15.1817681393895\\
69.4000244142397 -15.1784323498709\\
69.4488525392398 -15.1844431391896\\
69.49768066424 -15.1906071321836\\
69.5465087892401 -15.1969095646012\\
69.5953369142402 -15.2033358659118\\
69.6441650392403 -15.2095658616225\\
69.6929931642404 -15.2158600871211\\
69.7418212892406 -15.2222081834426\\
69.7906494142407 -15.2286118120268\\
69.8394775392408 -15.2350733902838\\
69.888305664241 -15.2415959230696\\
69.9371337892411 -15.2481828704445\\
69.9859619142412 -15.2542385264204\\
70.0347900392413 -15.2592677511819\\
70.0836181642415 -15.2643805441925\\
70.1324462892416 -15.2696026801508\\
70.1812744142417 -15.2748578689141\\
70.2301025392418 -15.2796130416527\\
70.278930664242 -15.2844269408748\\
70.3277587892421 -15.2893204729049\\
70.3765869142422 -15.294316279274\\
70.4254150392423 -15.2994390438496\\
70.4742431642425 -15.3047158402411\\
70.5230712892426 -15.3101765254892\\
70.5718994142427 -15.3398793640504\\
70.6207275392428 -15.3459234164017\\
70.6695556642429 -15.3518873175874\\
70.7183837892431 -15.3577791852512\\
70.7672119142432 -15.363609922262\\
70.8160400392433 -15.3824081428069\\
70.8648681642435 -15.389517414874\\
70.9136962892436 -15.3969098153274\\
70.9625244142437 -15.4046876539035\\
71.0113525392438 -15.412977538013\\
71.060180664244 -15.4253963058142\\
71.1090087892441 -15.4348789649848\\
71.1578369142442 -15.4451531099172\\
71.2066650392443 -15.4563829716005\\
71.2554931642445 -15.4687283203602\\
71.3043212892446 -15.5114061476439\\
71.3531494142447 -15.5208582181771\\
71.4019775392448 -15.5292440435549\\
71.450805664245 -15.5367486557628\\
71.4996337892451 -15.543577751154\\
71.5484619142452 -15.5499165327136\\
71.5972900392453 -15.5559143516016\\
71.6461181642454 -15.5616837344674\\
71.6949462892456 -15.567305671328\\
71.7437744142457 -15.5728365030494\\
71.7926025392458 -15.5860641655632\\
71.8414306642459 -15.5921966995049\\
71.8902587892461 -15.5973150199882\\
71.9390869142462 -15.6019303006932\\
71.9879150392463 -15.6064262745622\\
72.0367431642464 -15.6108332629407\\
72.0855712892466 -15.6151803119654\\
72.1343994142467 -15.6194954845267\\
72.1832275392468 -15.6238061570773\\
72.2320556642469 -15.6281393182762\\
72.2808837892471 -15.6343207940886\\
72.3297119142472 -15.6383485439872\\
72.3785400392473 -15.6425016935108\\
72.4273681642474 -15.6468011740499\\
72.4761962892476 -15.6512677400187\\
72.5250244142477 -15.6605363386944\\
72.5738525392478 -15.6544176768376\\
72.6226806642479 -15.6606269414577\\
72.6715087892481 -15.6670546913866\\
72.7203369142482 -15.6736878443181\\
72.7691650392483 -15.6805122474118\\
72.8179931642484 -15.6875126326813\\
72.8668212892486 -15.6946725492317\\
72.9156494142487 -15.7019742704182\\
72.9644775392488 -15.7093986744305\\
73.0133056642489 -15.7307120696834\\
73.0621337892491 -15.7306749727654\\
73.1109619142492 -15.7392124789843\\
73.1597900392493 -15.7480583692243\\
73.2086181642494 -15.7572166799168\\
73.2574462892496 -15.7666885104662\\
73.3062744142497 -15.7764712428156\\
73.3551025392498 -15.7860731026195\\
73.4039306642499 -15.7956581384152\\
73.4527587892501 -15.8053738227194\\
73.5015869142502 -15.8151905727219\\
73.5504150392503 -15.8086329361664\\
73.5992431642504 -15.8181537467385\\
73.6480712892506 -15.8277102732765\\
73.6968994142507 -15.8372776041791\\
73.7457275392508 -15.846830193141\\
73.7945556642509 -15.8563419268613\\
73.843383789251 -15.8657861928147\\
73.8922119142512 -15.875135946379\\
73.9410400392513 -15.8843637766519\\
73.9898681642514 -15.8947002480303\\
74.0386962892515 -15.9026144113833\\
74.0875244142517 -15.9102778929785\\
74.1363525392518 -15.9176783958117\\
74.1851806642519 -15.9237107868214\\
74.2340087892521 -15.9293957075542\\
74.2828369142522 -15.9348856211463\\
74.3316650392523 -15.9401929864171\\
74.3804931642524 -15.94284282842\\
74.4293212892526 -15.9455951512248\\
74.4781494142527 -15.948479074061\\
74.5269775392528 -15.9465713721016\\
74.5758056642529 -15.9511998588954\\
74.6246337892531 -15.9559118779782\\
74.6734619142532 -15.960702351769\\
74.7222900392533 -15.9655673724736\\
74.7711181642534 -15.9705039613749\\
74.8199462892536 -15.9755098830184\\
74.8687744142537 -15.9805835008362\\
74.9176025392538 -15.9857236642759\\
74.9664306642539 -15.9940186466958\\
75.015258789254 -15.9982386792777\\
75.0640869142542 -16.0040045781606\\
75.1129150392543 -16.0097549079873\\
75.1617431642544 -16.0154959630199\\
75.2105712892545 -16.0212329401296\\
75.2593994142547 -16.0269701682481\\
75.3082275392548 -16.0327112823838\\
75.3570556642549 -16.0384593571675\\
75.4058837892551 -16.0442170104599\\
75.4547119142552 -16.0512676855917\\
75.5035400392553 -16.05537831787\\
75.5523681642554 -16.0604426178754\\
75.6011962892556 -16.0651853470584\\
75.6500244142557 -16.0700422782685\\
75.6988525392558 -16.0766031682368\\
75.7476806642559 -16.0809140738875\\
75.7965087892561 -16.087364455899\\
75.8453369142562 -16.0932988979728\\
75.8941650392563 -16.0990290733154\\
75.9429931642564 -16.1073940449863\\
75.9918212892565 -16.1142761366105\\
76.0406494142567 -16.1212514786458\\
76.0894775392568 -16.1283147406739\\
76.1383056642569 -16.1354607846882\\
76.187133789257 -16.1428124879014\\
76.2481689455072 -16.1512298116132\\
76.3092041017574 -16.1596746158525\\
76.3702392580075 -16.1681417252201\\
76.4312744142577 -16.1766256504932\\
76.4923095705078 -16.1885509118455\\
76.565551758008 -16.1981149060787\\
76.6387939455082 -16.2075528012145\\
76.7120361330084 -16.2168636294034\\
76.7852783205086 -16.2260463656256\\
76.8585205080088 -16.235261492302\\
76.9317626955089 -16.2441102597667\\
77.0050048830091 -16.2529085849404\\
77.0782470705093 -16.2615911265697\\
77.1514892580095 -16.2701927027027\\
77.2247314455097 -16.2839394202537\\
77.2979736330099 -16.290378426599\\
77.3712158205101 -16.2997939769533\\
77.4444580080103 -16.3092049485972\\
77.5177001955104 -16.318610684483\\
77.5909423830106 -16.328010558998\\
77.6641845705108 -16.3374039870481\\
77.737426758011 -16.3467904363007\\
77.8106689455112 -16.3561694437195\\
77.8839111330114 -16.3655406379634\\
77.9571533205116 -16.3800261345598\\
78.0303955080118 -16.3871444162362\\
78.1036376955119 -16.3956992672634\\
78.1768798830121 -16.4042568221756\\
78.2501220705123 -16.4128319975037\\
78.3233642580125 -16.4214389729913\\
78.3966064455127 -16.4300911340939\\
78.4698486330129 -16.4388010173656\\
78.5430908205131 -16.4475802585821\\
78.6163330080132 -16.4564395435786\\
78.6895751955134 -16.4750732608747\\
78.7628173830136 -16.4761308425788\\
78.8360595705138 -16.4858446551852\\
78.909301758014 -16.4955317111249\\
78.9825439455142 -16.5051909632836\\
79.0557861330144 -16.5148215589581\\
79.1290283205145 -16.5244225598651\\
79.2022705080147 -16.5339927057605\\
79.2755126955149 -16.543530211193\\
79.3487548830151 -16.5530325870355\\
79.4219970705153 -16.5655012043409\\
79.4952392580155 -16.5746574706277\\
79.5684814455157 -16.583734684156\\
79.6417236330159 -16.5927331974486\\
79.7149658205161 -16.6016543584376\\
79.7882080080162 -16.6239851803776\\
79.8614501955164 -16.6278268248282\\
79.9346923830166 -16.6354253773779\\
80.0079345705168 -16.6428244778959\\
80.081176758017 -16.6500224392959\\
80.1544189455172 -16.6570179047316\\
80.2276611330174 -16.6634303260595\\
80.3009033205175 -16.6703999218861\\
80.3741455080177 -16.6767896220379\\
80.4473876955179 -16.6829850640058\\
80.5206298830181 -16.6943130080413\\
80.5938720705183 -16.7020929411758\\
80.6671142580185 -16.709925944284\\
80.7403564455187 -16.7178277672056\\
80.8135986330189 -16.7258149105518\\
80.886840820519 -16.7377318755518\\
80.9600830080192 -16.7457139213429\\
81.0333251955194 -16.7538024101177\\
81.1065673830196 -16.7620162707254\\
81.1798095705198 -16.7703751995488\\
81.25305175802 -16.7874063636013\\
81.3262939455202 -16.7934506233159\\
81.3995361330203 -16.8043935465824\\
81.4727783205205 -16.815490078461\\
81.5460205080207 -16.826726078467\\
81.6192626955209 -16.8380868801473\\
81.6925048830211 -16.8495572089388\\
81.7657470705213 -16.8611210747344\\
81.8389892580215 -16.8727616346482\\
81.9122314455217 -16.8844610207286\\
81.9854736330219 -16.9015169215149\\
82.058715820522 -16.9128745240949\\
82.1319580080222 -16.9241602288947\\
82.2052001955224 -16.9353543057074\\
82.2784423830226 -16.9464359932001\\
82.3516845705228 -16.9574244422077\\
82.424926758023 -16.9630281202016\\
82.4981689455231 -16.9703552550007\\
82.5714111330233 -16.977559607102\\
82.6446533205235 -16.9846607334094\\
82.7178955080237 -16.9916775532299\\
82.7911376955239 -16.9986284001963\\
82.8643798830241 -17.0055310689758\\
82.9376220705243 -17.0124028567656\\
83.0108642580245 -17.0192605995981\\
83.0841064455247 -17.0484249484001\\
83.1573486330248 -17.0501979713184\\
83.230590820525 -17.0548589035374\\
83.3038330080252 -17.0596367200272\\
83.3770751955254 -17.0645253441443\\
83.4503173830256 -17.0695182606183\\
83.5235595705258 -17.0733852122305\\
83.596801758026 -17.079211259697\\
83.6700439455262 -17.0850555150714\\
83.7432861330263 -17.0903984333046\\
83.8165283205265 -17.0969331852074\\
83.8897705080267 -17.1027244668979\\
83.9630126955269 -17.1078183226335\\
84.0362548830271 -17.1129556178931\\
84.1094970705273 -17.1181403323679\\
84.1827392580275 -17.1233760833584\\
84.2559814455277 -17.1267911465109\\
84.3292236330278 -17.1328452091475\\
84.402465820528 -17.1389209938275\\
84.4757080080282 -17.1448928501411\\
84.5489501955284 -17.1511310410253\\
84.6221923830286 -17.1572621670915\\
84.6954345705288 -17.1634086898829\\
84.768676758029 -17.1695691403634\\
84.8419189455291 -17.1757420586784\\
84.9151611330293 -17.1908071855618\\
84.9884033205295 -17.191248692673\\
85.0616455080297 -17.1981313715666\\
85.1348876955299 -17.2049769948297\\
85.2081298830301 -17.2117727365634\\
85.2813720705303 -17.2185063793597\\
85.3546142580305 -17.2251661812487\\
85.4278564455306 -17.2317407719856\\
85.5010986330308 -17.2382190723034\\
85.574340820531 -17.2437320344595\\
85.6475830080312 -17.2485667315942\\
85.7208251955314 -17.2533298631088\\
85.7940673830316 -17.2580262578903\\
85.8673095705318 -17.2626604233831\\
85.9405517580319 -17.2672365621304\\
86.0137939455321 -17.2744310291739\\
86.0870361330323 -17.2765340368529\\
86.1602783205325 -17.2793145386713\\
86.2335205080327 -17.2820188977341\\
86.3067626955329 -17.2846622678869\\
86.3800048830331 -17.28725995204\\
86.4532470705333 -17.2898274957063\\
86.5264892580334 -17.2923807897876\\
86.5997314455336 -17.294936183744\\
86.6729736330338 -17.2975106105295\\
86.746215820534 -17.3232235312054\\
86.8194580080342 -17.3224840945716\\
86.8927001955344 -17.3265461816389\\
86.9659423830346 -17.3305976173616\\
87.0391845705347 -17.3346507551445\\
87.1124267580349 -17.3387189006111\\
87.1856689455352 -17.3428164136274\\
87.2589111330353 -17.3469588195872\\
87.3321533205355 -17.3511629300719\\
87.4053955080357 -17.3554469726794\\
87.4786376955359 -17.3611798884928\\
87.5518798830361 -17.3589399366064\\
87.6251220705362 -17.3618740158349\\
87.6983642580364 -17.3646022109491\\
87.7716064455366 -17.3668194737525\\
87.8448486330368 -17.3693345027049\\
87.918090820537 -17.3712747334344\\
87.9913330080372 -17.3728817332952\\
88.0645751955374 -17.3741159754347\\
88.1378173830375 -17.3749421956742\\
88.2110595705377 -17.4152864167249\\
88.284301758038 -17.4230606129229\\
88.3575439455381 -17.4324450245905\\
88.4307861330383 -17.441432561443\\
88.5040283205385 -17.4506704883864\\
88.5772705080387 -17.471294708533\\
88.6505126955389 -17.4833359584804\\
88.723754883039 -17.4922030320666\\
88.7969970705392 -17.5006107288846\\
88.8702392580394 -17.5085403198062\\
88.9434814455396 -17.5159814997673\\
89.0167236330398 -17.5220484199502\\
89.08996582054 -17.5269997629674\\
89.1632080080402 -17.5314762926794\\
89.2364501955404 -17.5355257426707\\
89.3096923830406 -17.539192865477\\
89.3829345705408 -17.5425194426758\\
89.4561767580409 -17.5455444268704\\
89.5294189455411 -17.5483041690692\\
89.6026611330413 -17.5508326993361\\
89.6759033205415 -17.5717332102335\\
89.7491455080417 -17.5669315870941\\
89.8223876955419 -17.5693405216422\\
89.8956298830421 -17.5717041697246\\
89.9688720705422 -17.5740251490475\\
90.0421142580424 -17.5763612814529\\
90.1153564455426 -17.5782096410533\\
90.1885986330428 -17.5810314028313\\
90.261840820543 -17.5834112429088\\
90.3350830080432 -17.5858427235748\\
90.4083251955434 -17.590529377258\\
90.4815673830436 -17.5835421846286\\
90.5548095705437 -17.5874121582743\\
90.6280517580439 -17.5914704437218\\
90.7012939455441 -17.5957085108083\\
90.7745361330443 -17.6001169183938\\
90.8477783205445 -17.6046853463309\\
90.9210205080447 -17.6094026346503\\
90.9942626955449 -17.6142568297997\\
91.067504883045 -17.6192352376912\\
91.1407470705452 -17.6273233018451\\
91.2139892580454 -17.6304218437277\\
91.2872314455456 -17.6345079297362\\
91.3604736330458 -17.6386235875693\\
91.433715820546 -17.6427753234733\\
91.5069580080462 -17.6469702608397\\
91.5802001955464 -17.6512160826737\\
91.6534423830465 -17.6555209939519\\
91.7266845705467 -17.6598937008236\\
91.7999267580469 -17.6643434042327\\
91.8731689455471 -17.6684503568407\\
91.9464111330473 -17.6711187016451\\
92.0196533205475 -17.673995948242\\
92.0928955080477 -17.677116816749\\
92.1661376955478 -17.6805172771552\\
92.239379883048 -17.6847710634066\\
92.3126220705482 -17.6874489040923\\
92.3858642580484 -17.6905879769357\\
92.4591064455486 -17.6942520779894\\
92.5323486330488 -17.6985055155632\\
92.605590820549 -17.7282243547195\\
92.6788330080492 -17.7215450881646\\
92.7520751955493 -17.7299362684437\\
92.8253173830495 -17.7390806928932\\
92.8985595705497 -17.7489780690976\\
92.9718017580499 -17.7596157402754\\
93.0450439455501 -17.7709691786988\\
93.1182861330503 -17.7822823272193\\
93.1915283205505 -17.7930597963709\\
93.2647705080506 -17.8044015274876\\
93.3380126955508 -17.8163082551754\\
93.411254883051 -17.8287793271713\\
93.4844970705512 -17.8418126027653\\
93.5577392580514 -17.8554043488314\\
93.6309814455516 -17.8695491336767\\
93.7042236330518 -17.8918189807426\\
93.777465820552 -17.907569043523\\
93.8507080080521 -17.9236306973621\\
93.9239501955524 -17.9399767657262\\
93.9971923830525 -17.956580933899\\
94.0704345705527 -17.973110043996\\
94.1436767580529 -17.9892806296152\\
94.2169189455531 -18.0056159077743\\
94.2901611330533 -18.0221152043455\\
94.3634033205534 -18.0387784931678\\
94.4366455080536 -18.0596311902951\\
94.5098876955538 -18.0764064355224\\
94.583129883054 -18.0933178416179\\
94.6563720705542 -18.1103736651793\\
94.7296142580544 -18.1275809918297\\
94.8028564455546 -18.1582482644926\\
94.8760986330548 -18.1744188568178\\
94.9493408205549 -18.1909053420216\\
95.0225830080552 -18.2076961694486\\
95.0958251955553 -18.2247793398231\\
95.1690673830555 -18.271020742667\\
95.2423095705557 -18.2727341896898\\
95.3155517580559 -18.2912239985922\\
95.3887939455561 -18.3097551197022\\
95.4620361330562 -18.3283237358763\\
95.5352783205564 -18.3469237685431\\
95.6085205080566 -18.3491930398473\\
95.6817626955568 -18.368038924188\\
95.755004883057 -18.387060428727\\
95.8282470705572 -18.4062399729111\\
95.9014892580574 -18.4255582771544\\
95.9747314455576 -18.4449942703324\\
96.0479736330578 -18.464524994873\\
96.1212158205579 -18.4835353385509\\
96.1944580080581 -18.502211084032\\
96.2677001955583 -18.5208466467046\\
96.3409423830585 -18.5394243502504\\
96.4141845705587 -18.5579282607543\\
96.4874267580589 -18.5763441778093\\
96.5606689455591 -18.5946596154167\\
96.6339111330593 -18.6164988480861\\
96.7071533205595 -18.6334083268916\\
96.7803955080596 -18.6500696868536\\
96.8536376955598 -18.6665008494836\\
96.92687988306 -18.6827226289188\\
97.0001220705602 -18.7016556478789\\
97.0733642580604 -18.7168941538127\\
97.1466064455606 -18.732008609196\\
97.2198486330608 -18.7470282918469\\
97.2930908205609 -18.7619849933107\\
97.3663330080611 -18.7820298538266\\
97.4395751955613 -18.7971433886464\\
97.5128173830615 -18.8122528969169\\
97.5860595705617 -18.8273894128845\\
97.6593017580618 -18.8425859465822\\
97.7325439455621 -18.8655890426424\\
97.8057861330623 -18.8821950216567\\
97.8790283205624 -18.8987980102078\\
97.9522705080626 -18.9154012004919\\
98.0255126955628 -18.9320083499338\\
98.098754883063 -18.9600998179777\\
98.1719970705632 -18.9685090466092\\
98.2452392580634 -18.9880639419467\\
98.3184814455636 -19.0079608932783\\
98.3917236330637 -19.0282020548963\\
98.4649658205639 -19.0487890506876\\
98.5382080080641 -19.0697228454434\\
98.6114501955643 -19.0910035774571\\
98.6846923830645 -19.112123839291\\
98.7579345705647 -19.1330694991287\\
98.8311767580649 -19.1542190874893\\
98.9044189455651 -19.1755623536767\\
98.9776611330652 -19.1970880790165\\
99.0509033205654 -19.218783945124\\
99.1241455080656 -19.2406363720699\\
99.1973876955658 -19.2975542019277\\
99.270629883066 -19.3156255974203\\
99.3438720705662 -19.3334252786606\\
99.4171142580664 -19.3509783768097\\
99.4903564455665 -19.3683123878016\\
99.5635986330667 -19.3965046488655\\
99.6368408205669 -19.4147322960976\\
99.7100830080671 -19.4326413921942\\
99.7833251955673 -19.4502327468421\\
99.8565673830675 -19.4675061510854\\
99.9298095705677 -19.5088239761856\\
100.003051758068 -19.523206228913\\
100.076293945568 -19.5361502323391\\
100.149536133068 -19.5480435879017\\
100.222778320568 -19.5588237300632\\
100.296020508069 -19.5684195496639\\
100.369262695569 -19.5767500528156\\
100.442504883069 -19.5837228844894\\
100.515747070569 -19.5892327385477\\
100.588989258069 -19.5931597087231\\
100.66223144557 -19.5968526769053\\
100.73547363307 -19.5826871776923\\
100.80871582057 -19.5762720331991\\
100.88195800807 -19.5685245973481\\
100.95520019557 -19.5595381010741\\
101.02844238307 -19.5494490868379\\
101.101684570571 -19.5159848347452\\
101.174926758071 -19.5063340677876\\
101.248168945571 -19.498326065243\\
101.321411133071 -19.4925073128389\\
101.394653320571 -19.4894441252187\\
101.467895508072 -19.4083680821685\\
101.541137695572 -19.4208976117826\\
101.614379883072 -19.4421125855552\\
101.687622070572 -19.4710765416154\\
101.760864258072 -19.5066616479616\\
101.834106445573 -19.5154187160943\\
101.907348633073 -19.572371805718\\
101.980590820573 -19.631089455333\\
102.053833008073 -19.6903979556875\\
102.127075195573 -19.7494890435815\\
102.200317383073 -19.7689267641269\\
102.273559570574 -19.8398548428988\\
102.346801758074 -19.9074338110486\\
102.420043945574 -19.9713524185374\\
102.493286133074 -20.0316009191341\\
102.566528320574 -20.0708241745735\\
102.639770508075 -20.1290013673917\\
102.713012695575 -20.1835527307591\\
102.786254883075 -20.2347003639762\\
102.859497070575 -20.2827403715308\\
102.932739258075 -20.3280029310544\\
103.005981445576 -20.3708256771398\\
103.079223633076 -20.4115370518109\\
103.152465820576 -20.4493770998358\\
103.225708008076 -20.4858522279707\\
103.298950195576 -20.5212441668437\\
103.372192383076 -20.5557849210624\\
103.445434570577 -20.5896809682916\\
103.518676758077 -20.6231163342203\\
103.591918945577 -20.6675996729513\\
103.665161133077 -20.696942634281\\
103.738403320577 -20.7258465568399\\
103.811645508078 -20.7544954106039\\
103.884887695578 -20.7830701279587\\
103.958129883078 -20.8225213137449\\
104.031372070578 -20.851012971388\\
104.104614258078 -20.879833643394\\
104.177856445579 -20.9091785123993\\
104.251098633079 -20.9392507616604\\
104.324340820579 -20.995198529678\\
104.397583008079 -21.0243809152705\\
104.470825195579 -21.057758310616\\
104.544067383079 -21.092286595174\\
104.61730957058 -21.1270428969096\\
104.69055175808 -21.1661116721619\\
104.76379394558 -21.2025554665714\\
104.83703613308 -21.2423791856374\\
104.91027832058 -21.2854447779269\\
104.983520508081 -21.3325118135315\\
105.056762695581 -21.384456231936\\
105.130004883081 -21.4422757989053\\
105.203247070581 -21.5070863701282\\
105.276489258081 -21.5801022369367\\
105.349731445581 -21.6625910551843\\
105.422973633082 -21.7674467556047\\
105.496215820582 -21.8657823707944\\
105.569458008082 -21.9729308450046\\
105.642700195582 -22.0887701920869\\
105.715942383082 -22.2126871817524\\
105.789184570583 -22.345505525746\\
105.862426758083 -22.4799831568937\\
105.935668945583 -22.6177907150122\\
106.008911133083 -22.7542551991182\\
106.082153320583 -22.8732110852716\\
106.155395508084 -22.9844587293729\\
106.228637695584 -23.0846669167483\\
106.301879883084 -23.1706260837891\\
106.375122070584 -23.2395402834248\\
106.448364258084 -23.2857484552527\\
106.521606445584 -23.3010000792764\\
106.594848633085 -23.2962327885104\\
106.668090820585 -23.2734164438949\\
106.741333008085 -23.2351433220521\\
106.814575195585 -23.1843265855658\\
106.887817383085 -23.1336893087445\\
106.985473633086 -23.044849262152\\
107.083129883086 -22.9491552680113\\
107.180786133086 -22.8504368869649\\
107.278442383086 -22.7515765122838\\
107.376098633087 -22.6837422728871\\
107.473754883087 -22.5599031854727\\
107.571411133087 -22.476949623263\\
107.669067383087 -22.3989292956702\\
107.766723633088 -22.3258869391559\\
107.864379883088 -22.2577341509295\\
107.962036133088 -22.1766789347649\\
108.059692383088 -22.1168139534391\\
108.157348633089 -22.0616597632217\\
108.255004883089 -22.0110267563873\\
108.352661133089 -21.9647348489229\\
108.450317383089 -21.8884821233781\\
108.54797363309 -21.8499056124508\\
108.64562988309 -21.8164519170701\\
108.74328613309 -21.7880725965637\\
108.84094238309 -21.7647281663419\\
108.938598633091 -21.733586355951\\
109.036254883091 -21.7198153845877\\
109.133911133091 -21.7119344651504\\
109.231567383091 -21.7098290264627\\
109.329223633092 -21.7133509077964\\
109.426879883092 -21.7223143209043\\
109.524536133092 -21.7364913050467\\
109.622192383092 -21.7556070418012\\
109.719848633093 -21.7793354462327\\
109.817504883093 -21.8525932066243\\
109.915161133093 -21.8852459490489\\
110.012817383093 -21.9208981976068\\
110.110473633094 -21.9590040580599\\
110.208129883094 -21.9989659413737\\
110.305786133094 -22.0453739441085\\
110.403442383094 -22.0757931095866\\
110.501098633095 -22.1052904357824\\
110.598754883095 -22.1337171469636\\
110.696411133095 -22.1609464170591\\
110.794067383095 -22.2302142557746\\
110.891723633096 -22.2490620403883\\
110.989379883096 -22.2661158133471\\
111.087036133096 -22.2813997060573\\
111.184692383096 -22.2949327188999\\
111.282348633097 -22.3154808448921\\
111.380004883097 -22.3242260623124\\
111.477661133097 -22.3311209781417\\
111.575317383097 -22.3362759123847\\
111.672973633098 -22.339773702398\\
111.770629883098 -22.3979790687504\\
111.868286133098 -22.3691436381119\\
111.965942383098 -22.3604452979991\\
112.063598633099 -22.3509448220362\\
112.161254883099 -22.3412225029924\\
112.258911133099 -22.3317609634594\\
112.356567383099 -22.3229543467529\\
112.4542236331 -22.3151223324238\\
112.5518798831 -22.308526066878\\
112.6495361331 -22.3033842245843\\
112.7471923831 -22.353549263288\\
112.844848633101 -22.2623719898876\\
112.942504883101 -22.234654572854\\
113.040161133101 -22.1992437305321\\
113.137817383101 -22.1547953330771\\
113.235473633102 -22.0997951593312\\
113.333129883102 -22.0045053060352\\
113.430786133102 -21.9344048348825\\
113.528442383102 -21.8519766120613\\
113.626098633103 -21.7391269627786\\
113.723754883103 -21.6426543286258\\
113.821411133103 -21.5117531070317\\
113.919067383103 -21.3607002353497\\
114.016723633104 -21.1877554179847\\
114.114379883104 -20.9916787176712\\
114.212036133104 -21.3446370105663\\
114.309692383104 -21.0760968228828\\
114.407348633105 -20.9084326478196\\
114.505004883105 -20.7228530411548\\
114.602661133105 -20.4932589665005\\
114.700317383105 -20.2945549606548\\
114.797973633106 -19.6937977385572\\
114.895629883106 -19.4866097585529\\
114.993286133106 -19.2828143692174\\
115.090942383106 -19.0865753079751\\
115.188598633107 -18.9025524621205\\
115.286254883107 -18.7357570331608\\
115.383911133107 -18.5913443232091\\
115.481567383107 -18.4743563937379\\
115.579223633108 -18.3894419894261\\
115.676879883108 -18.7822391969878\\
115.774536133108 -18.8387774753517\\
115.872192383108 -18.9345749103575\\
115.969848633109 -19.0350496181497\\
116.067504883109 -19.1492816124145\\
116.165161133109 -19.5991278173388\\
116.262817383109 -19.7592090391524\\
116.36047363311 -19.9080573994672\\
116.45812988311 -20.0455605027423\\
116.55578613311 -20.1710525708139\\
116.65344238311 -20.6869754720283\\
116.751098633111 -20.7116310974102\\
116.848754883111 -20.7007396575709\\
116.946411133111 -20.6526283002851\\
117.044067383111 -20.5649467978644\\
117.141723633112 -20.4514814445391\\
117.239379883112 -19.8309967127063\\
117.337036133112 -19.559330902161\\
117.434692383112 -19.288384579848\\
117.532348633113 -19.020922952955\\
117.630004883113 -18.7604387916593\\
117.727661133113 -18.3988162405631\\
117.825317383113 -18.2780272049302\\
117.922973633114 -18.066396416864\\
118.020629883114 -17.881834345754\\
118.118286133114 -18.6749016923984\\
118.215942383114 -18.5582188931787\\
118.313598633115 -18.6641091657901\\
118.411254883115 -18.7755961890854\\
118.508911133115 -18.8934512610231\\
118.606567383115 -19.0184978170887\\
118.704223633116 -18.6955261593451\\
118.801879883116 -18.7794194030133\\
118.899536133116 -18.8601235089619\\
118.997192383116 -18.9388758050256\\
119.094848633117 -19.016796824432\\
119.192504883117 -19.0774030259246\\
119.290161133117 -19.174345344634\\
119.387817383117 -19.2560671349426\\
119.485473633118 -19.3412351068164\\
119.583129883118 -19.4453748430478\\
119.680786133118 -19.5248301229297\\
119.778442383118 -19.6012284424705\\
119.876098633119 -19.6754797452496\\
119.973754883119 -19.7485026225362\\
120.071411133119 -20.9263699315564\\
120.169067383119 -21.1141743150151\\
120.26672363312 -21.3103695721111\\
120.36437988312 -21.5175426628207\\
120.46203613312 -21.7388938232741\\
120.55969238312 -23.2992949330852\\
120.657348633121 -23.3580832111928\\
120.755004883121 -23.6694563863809\\
120.852661133121 -24.0174005886613\\
120.950317383121 -24.4001978124568\\
121.047973633122 -24.7924869185311\\
121.145629883122 -25.0963236320129\\
121.243286133122 -25.0302649717633\\
121.340942383122 -24.0566771527288\\
121.438598633123 -21.8662256187\\
121.536254883123 -21.3595338887733\\
121.633911133123 -16.8061244685619\\
121.731567383123 -17.0553352812717\\
121.829223633124 -17.6114000008746\\
121.926879883124 -18.1729145618165\\
122.024536133124 -18.6663825199704\\
122.122192383124 -19.0533217943912\\
122.219848633125 -19.3271885530001\\
122.317504883125 -19.5642209895491\\
122.415161133125 -19.773106110552\\
122.512817383125 -19.9604846906511\\
122.610473633126 -20.1313838515508\\
122.708129883126 -20.2896298692743\\
122.805786133126 -20.4381694535319\\
122.903442383126 -20.5793050765013\\
123.001098633127 -20.7598764833216\\
123.098754883127 -20.742066556937\\
123.196411133127 -20.8628771240998\\
123.294067383127 -20.9839701329342\\
123.391723633128 -21.1063089546809\\
123.489379883128 -21.2307274033964\\
123.587036133128 -21.3579538894496\\
123.684692383128 -21.4886282263896\\
123.782348633129 -21.6233128739556\\
123.880004883129 -21.7624998609678\\
123.977661133129 -21.9851264248466\\
124.075317383129 -22.0941862720337\\
124.17297363313 -22.2089002490393\\
124.27062988313 -22.330190622887\\
124.36828613313 -22.4590009869388\\
124.46594238313 -22.6210565341858\\
124.563598633131 -22.6792676065346\\
124.661254883131 -22.8219185371911\\
124.758911133131 -22.9756154322621\\
124.856567383131 -23.1411239493965\\
124.954223633132 -23.3191369798795\\
125.051879883132 -23.5102220158728\\
125.149536133132 -23.7147493690375\\
125.247192383132 -23.9327968373507\\
125.344848633133 -24.1640265003222\\
125.442504883133 -24.7257512332768\\
125.540161133133 -24.8959692500399\\
125.637817383133 -25.0617602501706\\
125.735473633134 -25.222017984879\\
125.833129883134 -25.3755199857531\\
125.930786133134 -26.0582282960076\\
126.028442383134 -25.9928457185197\\
126.126098633135 -26.1032985651822\\
126.223754883135 -26.1942615453701\\
126.321411133135 -26.2655125081443\\
126.419067383135 -26.3169376734624\\
126.516723633136 -26.2829009430235\\
126.614379883136 -26.245162613537\\
126.712036133136 -26.1940248070554\\
126.809692383136 -26.128915778156\\
126.907348633137 -26.0489045325028\\
127.005004883137 -25.9526997119355\\
127.102661133137 -25.8386403941694\\
127.200317383137 -25.7046804241624\\
127.297973633138 -25.5483681132762\\
127.395629883138 -25.8635433655129\\
127.493286133138 -25.6409930012238\\
127.590942383138 -25.3756621903725\\
127.688598633139 -25.0641485887277\\
127.786254883139 -24.703103684275\\
127.883911133139 -25.4115906653185\\
127.981567383139 -23.8992042063739\\
128.07922363314 -23.3799037973706\\
128.17687988314 -22.7977399497804\\
128.27453613314 -22.1546527132335\\
128.37219238314 -21.4543249374923\\
128.469848633141 -20.5198797201517\\
128.567504883141 -19.671621592567\\
128.665161133141 -18.8018887372849\\
128.762817383141 -17.9331867750394\\
128.860473633142 -17.0945358044682\\
128.958129883142 -15.9127240815869\\
129.055786133142 -15.2026202490689\\
129.153442383142 -14.723357195482\\
129.251098633143 -14.4877043593592\\
129.348754883143 -14.4661517113895\\
129.446411133143 -14.6029282396529\\
129.544067383143 -14.8382362265815\\
129.641723633144 -15.122768066982\\
129.739379883144 -15.4219219537768\\
129.837036133144 -15.7104618861338\\
129.934692383144 -15.9673361640885\\
130.032348633145 -16.2008029130042\\
130.130004883145 -16.4083937548252\\
130.227661133145 -16.589656390422\\
130.325317383145 -16.979265896102\\
130.422973633146 -17.0799726023233\\
130.520629883146 -17.2046566586495\\
130.618286133146 -17.3056157666388\\
130.715942383146 -17.3843618292132\\
130.813598633147 -17.4421851617484\\
130.911254883147 -17.1916117952373\\
131.008911133147 -17.2312177220413\\
131.106567383147 -17.2564230671704\\
131.204223633148 -17.267300564831\\
131.301879883148 -17.2638233906018\\
131.399536133148 -16.9419277212824\\
131.497192383148 -16.9325240080715\\
131.594848633149 -16.9113805627285\\
131.692504883149 -16.8781081654253\\
131.790161133149 -16.8322300354481\\
131.887817383149 -16.7731746953895\\
131.985473633149 -16.6969754800818\\
132.08312988315 -16.5708936182848\\
132.18078613315 -16.4274111012578\\
132.27844238315 -16.2671680001841\\
132.376098633151 -16.0909787787218\\
132.473754883151 -15.899803266138\\
132.571411133151 -15.6947142067627\\
132.669067383151 -15.4373758539929\\
132.766723633152 -15.1177811293346\\
132.864379883152 -14.7742565810944\\
132.962036133152 -14.4074523662403\\
133.059692383152 -14.0180140999344\\
133.157348633152 -13.6064597098942\\
133.255004883153 -13.5154247578537\\
133.352661133153 -13.0678401959821\\
133.450317383153 -12.6577331472878\\
133.547973633154 -12.225269959073\\
133.645629883154 -11.7717990422537\\
133.743286133154 -11.2993293181626\\
133.840942383154 -10.8106600691651\\
133.938598633154 -10.3095147360818\\
134.036254883155 -9.80065631690968\\
134.133911133155 -9.28994209752506\\
134.231567383155 -8.91842380684835\\
134.329223633156 -8.44916184469105\\
134.426879883156 -8.01963970286266\\
134.524536133156 -7.6438715847101\\
134.622192383156 -7.33192122063849\\
134.719848633157 -7.89126635042948\\
134.817504883157 -8.27342529802382\\
134.915161133157 -9.09566249489649\\
135.012817383157 -9.09760342021616\\
135.110473633157 -6.96163204031052\\
135.208129883158 -6.54543498145841\\
135.305786133158 -6.51468653067565\\
135.403442383158 -6.7140040409331\\
135.501098633159 -7.05822804855501\\
135.598754883159 -7.46404955292866\\
135.696411133159 -7.8923813403004\\
135.794067383159 -8.31459909903549\\
135.891723633159 -8.73676288349662\\
135.98937988316 -9.14448193842372\\
136.08703613316 -9.53466115683233\\
136.18469238316 -9.90591607028745\\
136.28234863316 -10.2578894280728\\
136.380004883161 -10.5908398364903\\
136.477661133161 -10.9053876581063\\
136.575317383161 -11.2023556569935\\
136.672973633161 -11.6245174946401\\
136.770629883162 -11.8643238097173\\
136.868286133162 -12.0905324363588\\
136.965942383162 -12.3066577321652\\
137.063598633162 -12.5130947392277\\
137.161254883163 -12.7102525239463\\
137.271118164413 -12.8381199383924\\
137.380981445663 -13.0316087967494\\
137.490844726914 -13.2170994848084\\
137.600708008164 -13.3949081101033\\
137.710571289414 -13.5653512920031\\
137.832641601914 -13.7464750091584\\
137.954711914415 -13.9193131228276\\
138.076782226915 -14.0842741234669\\
138.198852539415 -14.2417520067602\\
138.320922851916 -14.4499502715136\\
138.442993164416 -14.5720395712883\\
138.565063476916 -14.6898663927398\\
138.687133789417 -14.8032203427047\\
138.809204101917 -14.9125995910005\\
138.931274414417 -15.018437445642\\
139.053344726918 -15.1211115584657\\
139.175415039418 -15.2209516573083\\
139.297485351918 -15.3182460718744\\
139.419555664418 -15.4132472666423\\
139.541625976919 -15.5152206852844\\
139.663696289419 -15.6001332640581\\
139.785766601919 -15.6841939958243\\
139.90783691442 -15.7678768716299\\
140.02990722692 -15.8513274427321\\
140.15197753942 -15.9346022949631\\
140.274047851921 -16.0176929316258\\
140.396118164421 -16.100546048714\\
140.518188476921 -16.1830799140404\\
140.640258789422 -16.2651970411073\\
140.762329101922 -16.345302336149\\
140.884399414422 -16.4225524731895\\
141.006469726923 -16.500327583832\\
141.128540039423 -16.5771861597671\\
141.250610351923 -16.6535712252085\\
141.372680664423 -16.7360516862355\\
141.494750976924 -16.8127387685251\\
141.616821289424 -16.8886492927675\\
141.738891601924 -16.9637386643146\\
141.860961914425 -17.0379666277654\\
141.983032226925 -17.1147166829309\\
142.105102539425 -17.1875085103726\\
142.227172851926 -17.2592744872313\\
142.349243164426 -17.329989916979\\
142.471313476926 -17.3996337840923\\
142.593383789427 -17.4710051459452\\
142.715454101927 -17.5376277450432\\
142.837524414427 -17.6037952752771\\
142.959594726928 -17.6687961506341\\
143.081665039428 -17.7326372226559\\
143.203735351928 -17.7953275422284\\
143.325805664429 -17.8490063381359\\
143.447875976929 -17.9088696015722\\
143.569946289429 -17.9677625349548\\
143.692016601929 -18.0256994964089\\
143.81408691443 -18.082694419614\\
143.93615722693 -18.1378519863428\\
144.05822753943 -18.1933696546877\\
144.180297851931 -18.2479814895632\\
144.302368164431 -18.3015185618239\\
144.424438476931 -18.3545264173541\\
144.546508789432 -18.403102664584\\
144.668579101932 -18.4544517690971\\
144.790649414432 -18.5049386127623\\
144.912719726932 -18.5545695654803\\
145.034790039433 -18.6033517667123\\
145.156860351933 -18.6497851521606\\
145.278930664433 -18.6954386221736\\
145.401000976934 -18.7404277069881\\
145.523071289434 -18.7847831977875\\
145.645141601934 -18.8285342732023\\
145.767211914435 -18.8717084960908\\
145.889282226935 -18.9143318034644\\
146.011352539435 -18.9564284897895\\
146.133422851936 -18.998021183957\\
146.255493164436 -19.0393596451459\\
146.377563476936 -19.0789603604754\\
146.499633789437 -19.1180974536593\\
146.621704101937 -19.1568038491315\\
146.743774414437 -19.1951116855674\\
146.865844726938 -19.2417581250173\\
146.987915039438 -19.2749256901558\\
147.109985351938 -19.3132905505187\\
147.232055664438 -19.3509397916869\\
147.354125976939 -19.3863750678942\\
147.476196289439 -19.4259185802466\\
147.598266601939 -19.455351846134\\
147.72033691444 -19.4919027297116\\
147.84240722694 -19.5281652188606\\
147.96447753944 -19.5641485819827\\
148.086547851941 -19.5998633648501\\
148.208618164441 -19.6343782266591\\
148.330688476941 -19.6681886509718\\
148.452758789442 -19.7016219712836\\
148.574829101942 -19.7346872239484\\
148.696899414442 -19.767394203946\\
148.818969726943 -19.7997537867344\\
148.941040039443 -19.8317783946608\\
149.063110351943 -19.8634826873157\\
149.185180664443 -19.8948846073757\\
149.307250976944 -19.9274719500316\\
149.429321289444 -19.9560207274517\\
149.551391601944 -19.985735312262\\
149.673461914445 -20.0151409386357\\
149.795532226945 -20.0442501105366\\
149.917602539445 -20.073075031541\\
150.039672851946 -20.0979596524114\\
150.161743164446 -20.1260571134545\\
150.283813476946 -20.1539191202211\\
150.405883789447 -20.1815690606584\\
150.527954101947 -20.2090280979238\\
150.650024414447 -20.2363154285975\\
150.772094726948 -20.2634485058335\\
150.894165039448 -20.2904432328684\\
151.016235351948 -20.3173141313743\\
151.138305664448 -20.3544053267726\\
151.260375976949 -20.3817949490082\\
151.382446289449 -20.4088932524846\\
151.504516601949 -20.4356974372811\\
151.62658691445 -20.4622051180734\\
151.74865722695 -20.498018295905\\
151.87072753945 -20.5160447957414\\
151.992797851951 -20.5408534968329\\
152.114868164451 -20.5653855451669\\
152.236938476951 -20.5896474384606\\
152.359008789452 -20.6136450639827\\
152.481079101952 -20.6373837513502\\
152.603149414452 -20.6608683217221\\
152.725219726952 -20.6836310705327\\
152.847290039453 -20.7061372024506\\
152.969360351953 -20.7284360555846\\
153.091430664453 -20.7505362293849\\
153.213500976954 -20.77244545463\\
153.335571289454 -20.7941706968445\\
153.457641601954 -20.8157182456363\\
153.579711914455 -20.8422746738283\\
153.701782226955 -20.8625304337212\\
153.823852539455 -20.8832888838214\\
153.945922851956 -20.9035252437403\\
154.067993164456 -20.9236011282697\\
154.190063476956 -20.9459104884038\\
154.312133789457 -20.9662496992261\\
154.434204101957 -20.9863949661183\\
154.556274414457 -21.0063508830069\\
154.678344726957 -21.0261223425456\\
154.800415039458 -21.0471094723686\\
154.922485351958 -21.0645073130767\\
155.044555664458 -21.0839809645029\\
155.166625976959 -21.1033642267861\\
155.288696289459 -21.1226477360009\\
155.410766601959 -21.1418220674111\\
155.53283691446 -21.1595709330269\\
155.65490722696 -21.1777512767882\\
155.77697753946 -21.1958605405628\\
155.899047851961 -21.2138945285345\\
156.021118164461 -21.2318480487808\\
156.143188476961 -21.2494363220378\\
156.265258789461 -21.2673497054176\\
156.387329101962 -21.285130795409\\
156.509399414462 -21.3027217655493\\
156.631469726962 -21.3202608183115\\
156.753540039463 -21.3375920515437\\
156.875610351963 -21.3547557277061\\
156.997680664463 -21.3717432743214\\
157.119750976964 -21.3885464256321\\
157.241821289464 -21.4100159513506\\
157.363891601964 -21.4228972628789\\
157.485961914465 -21.4384405066519\\
157.608032226965 -21.4536714955231\\
157.730102539465 -21.4685923046702\\
157.852172851966 -21.4832089503172\\
157.974243164466 -21.4975321173796\\
158.096313476966 -21.5115779941871\\
158.218383789467 -21.525369223936\\
158.340454101967 -21.5389359811451\\
158.462524414467 -21.5536469078233\\
158.584594726967 -21.5655507481574\\
158.706665039468 -21.5773590537051\\
158.828735351968 -21.5891613007126\\
158.950805664468 -21.6010628238826\\
159.072875976969 -21.6243552501958\\
159.194946289469 -21.6352693706611\\
159.317016601969 -21.6486706500789\\
159.43908691447 -21.6624968596652\\
159.56115722697 -21.6769091229628\\
159.68322753947 -21.6920996133793\\
159.805297851971 -21.7067176698508\\
159.927368164471 -21.7237481001571\\
160.049438476971 -21.7423240717887\\
160.171508789471 -21.7628294856066\\
160.293579101972 -21.7857071304447\\
160.415649414472 -21.8114558279555\\
160.537719726972 -21.8406201574924\\
160.659790039473 -21.8737695051429\\
160.781860351973 -21.9114630629275\\
160.903930664473 -21.9651303691575\\
161.026000976974 -22.0077254653322\\
161.148071289474 -22.0604177957458\\
161.270141601974 -22.1176497987681\\
161.392211914475 -22.1778798401748\\
161.514282226975 -22.2483235015715\\
161.636352539475 -22.3184915119283\\
161.758422851976 -22.3902966972154\\
161.880493164476 -22.4620456898599\\
162.002563476976 -22.5319679436481\\
162.124633789476 -22.6034860459203\\
162.246704101977 -22.642179375282\\
162.368774414477 -22.6980665009673\\
162.490844726977 -22.7486175722159\\
162.612915039478 -22.7934925520265\\
162.734985351978 -22.8326409517392\\
162.857055664478 -22.8619233880637\\
162.979125976979 -22.8871883309815\\
163.101196289479 -22.9094011850319\\
163.223266601979 -22.928988943877\\
163.34533691448 -22.9463358341472\\
163.46740722698 -22.9541689091284\\
163.58947753948 -22.966069986466\\
163.711547851981 -22.9769710321232\\
163.833618164481 -22.9871035204826\\
163.955688476981 -22.9966504215059\\
164.077758789481 -23.0035298647263\\
164.199829101982 -23.0112944223976\\
164.321899414482 -23.018952020676\\
164.443969726982 -23.026567828177\\
164.566040039483 -23.0341900293068\\
164.688110351983 -23.0418536321814\\
164.810180664483 -23.0495834634916\\
164.932250976984 -23.0573965148408\\
165.054321289484 -23.0653037758527\\
165.176391601984 -23.0864678802511\\
165.298461914485 -23.0877329403478\\
165.420532226985 -23.094673957733\\
165.542602539485 -23.1017820034071\\
165.664672851986 -23.1090623291994\\
165.786743164486 -23.1165195030112\\
165.908813476986 -23.1241575851579\\
166.030883789486 -23.1319802656441\\
166.152954101987 -23.1399909713725\\
166.275024414487 -23.1481929501283\\
166.397094726987 -23.1565574761227\\
166.519165039488 -23.1650115504382\\
166.641235351988 -23.1736349721391\\
166.763305664488 -23.1824310938119\\
166.885375976989 -23.1914035247189\\
167.007446289489 -23.209278176061\\
167.129516601989 -23.2186699428357\\
167.25158691449 -23.2282286997093\\
167.37365722699 -23.237954425828\\
167.49572753949 -23.2478476135526\\
167.617797851991 -23.2691859223546\\
167.752075195741 -23.2806824851756\\
167.886352539491 -23.292301429197\\
168.020629883242 -23.3040352603725\\
168.154907226992 -23.315877394966\\
168.289184570742 -23.3278710513551\\
168.435668945743 -23.3407850578963\\
168.582153320743 -23.3538080115127\\
168.728637695743 -23.3669385846534\\
168.875122070744 -23.3801757827311\\
169.021606445744 -23.3956431632985\\
169.168090820745 -23.4006287208099\\
169.314575195745 -23.4150339648442\\
169.461059570745 -23.4295753908447\\
169.607543945746 -23.4442375486444\\
169.754028320746 -23.4590072529347\\
169.900512695746 -23.4728911565327\\
170.046997070747 -23.4888256315306\\
170.193481445747 -23.5038561223881\\
170.339965820747 -23.5189573280117\\
170.486450195748 -23.5371909655469\\
170.632934570748 -23.5503544236215\\
170.779418945749 -23.564599330366\\
170.925903320749 -23.5788816850879\\
171.072387695749 -23.5932039800523\\
171.21887207075 -23.6075684495858\\
171.36535644575 -23.6168871247783\\
171.51184082075 -23.6305845539497\\
171.658325195751 -23.6443697215733\\
171.804809570751 -23.6582513560038\\
171.951293945752 -23.6722384272707\\
172.097778320752 -23.6780032860183\\
172.244262695752 -23.6917540720536\\
172.390747070753 -23.7057525760801\\
172.537231445753 -23.7200106251289\\
172.683715820753 -23.7345389918314\\
172.830200195754 -23.7493474093001\\
172.976684570754 -23.7644445870524\\
173.123168945755 -23.7798382280417\\
173.269653320755 -23.7955350468441\\
173.416137695755 -23.8111671240904\\
173.562622070756 -23.8265309715324\\
173.709106445756 -23.8423270103922\\
173.855590820756 -23.8585676285493\\
174.002075195757 -23.8752594202748\\
174.148559570757 -23.9012519297152\\
174.295043945758 -23.9109441557663\\
174.441528320758 -23.9280345708648\\
174.588012695758 -23.9455947600682\\
174.734497070759 -23.9636510487856\\
174.880981445759 -23.9822251914932\\
175.027465820759 -24.0013333777302\\
175.17395019576 -24.0209852472963\\
175.32043457076 -24.0411829568342\\
175.466918945761 -24.0619203453587\\
175.613403320761 -24.0872314637091\\
175.759887695761 -24.1026116353075\\
175.906372070762 -24.12353302537\\
176.052856445762 -24.1448399560597\\
176.199340820762 -24.166531033052\\
176.345825195763 -24.188601161868\\
176.492309570763 -24.211040439967\\
176.638793945764 -24.2338328001572\\
176.785278320764 -24.2569543726677\\
176.931762695764 -24.280371541895\\
177.078247070765 -24.3093001687509\\
177.224731445765 -24.3331889739749\\
177.371215820765 -24.3569669983635\\
177.517700195766 -24.3804984875645\\
177.664184570766 -24.4036382382747\\
177.810668945767 -24.4291846791898\\
177.957153320767 -24.4470286730392\\
178.103637695767 -24.4675470193196\\
178.250122070768 -24.4874399682544\\
178.396606445768 -24.5066647671632\\
178.543090820768 -24.5251861080575\\
178.689575195769 -24.541403981239\\
178.836059570769 -24.5573459399526\\
178.98254394577 -24.5727241187492\\
179.12902832077 -24.5875486413436\\
179.27551269577 -24.6018325896032\\
179.421997070771 -24.6142964782607\\
179.568481445771 -24.6275020583767\\
179.714965820771 -24.6402217337947\\
179.861450195772 -24.6524857459519\\
180.007934570772 -24.6643296187889\\
180.154418945773 -24.6757948002543\\
180.300903320773 -24.6869294030195\\
180.447387695773 -24.697789051323\\
180.593872070774 -24.708437839557\\
180.740356445774 -24.7297039791481\\
180.886840820774 -24.7398202076319\\
181.033325195775 -24.7495617575806\\
181.179809570775 -24.7590027505397\\
181.326293945775 -24.7682405426476\\
181.472778320776 -24.7784459590712\\
181.619262695776 -24.787880841383\\
181.765747070777 -24.7976864375073\\
181.912231445777 -24.8080861899445\\
182.058715820777 -24.8193483219144\\
182.205200195778 -24.8492921003861\\
182.351684570778 -24.8604629954569\\
182.498168945779 -24.8783529318227\\
182.644653320779 -24.8979479350064\\
182.791137695779 -24.9195826724383\\
182.93762207078 -24.9436305009935\\
183.08410644578 -24.9673775130092\\
183.23059082078 -24.9971136018297\\
183.377075195781 -25.0305173744802\\
183.523559570781 -25.0679883312047\\
183.670043945781 -25.1098257869149\\
183.816528320782 -25.1561470195202\\
183.963012695782 -25.2067885619509\\
184.109497070783 -25.2612045708458\\
184.255981445783 -25.3183881567679\\
184.402465820783 -25.3811686145922\\
184.548950195784 -25.4330477987896\\
184.695434570784 -25.4877277105607\\
184.841918945785 -25.5373334072017\\
184.988403320785 -25.5801916890106\\
185.134887695785 -25.6152026270997\\
185.281372070786 -25.6304673153397\\
185.427856445786 -25.6481616714772\\
185.574340820786 -25.6599504541897\\
185.720825195787 -25.6666643021144\\
185.867309570787 -25.6691933878707\\
186.013793945788 -25.6684078155481\\
186.160278320788 -25.6651055655574\\
186.306762695788 -25.6599845852813\\
186.453247070789 -25.6536331476404\\
186.599731445789 -25.6479465524724\\
186.746215820789 -25.6395250555969\\
186.89270019579 -25.6304558454113\\
187.03918457079 -25.6218755632063\\
187.18566894579 -25.6139369572639\\
187.332153320791 -25.6067359905165\\
187.478637695791 -25.5961853454445\\
187.625122070792 -25.5901953179269\\
187.771606445792 -25.5851578822067\\
187.918090820792 -25.5810724933905\\
188.064575195793 -25.5779213680718\\
188.211059570793 -25.575674896927\\
188.357543945793 -25.5742957609301\\
188.504028320794 -25.5737420089566\\
188.650512695794 -25.5739693196391\\
188.796997070795 -25.5769352550098\\
188.943481445795 -25.5785788243073\\
189.089965820795 -25.5808826236254\\
189.236450195796 -25.5838049511635\\
189.382934570796 -25.5873067901232\\
189.529418945796 -25.5984172777089\\
189.675903320797 -25.6020652985409\\
189.822387695797 -25.6064992896759\\
189.968872070798 -25.6112491491447\\
190.115356445798 -25.6162329490185\\
190.261840820798 -25.6218981584684\\
190.408325195799 -25.6231811223179\\
190.554809570799 -25.6293996823029\\
190.701293945799 -25.635995960276\\
190.8477783208 -25.6429412772729\\
190.9942626958 -25.6502075519244\\
191.140747070801 -25.6577676283614\\
191.287231445801 -25.6655955177943\\
191.433715820801 -25.6736665664242\\
191.580200195802 -25.681957561899\\
191.726684570802 -25.6974502191956\\
191.873168945802 -25.7059937830132\\
192.019653320803 -25.7145762748461\\
192.166137695803 -25.7231973995461\\
192.312622070804 -25.7318580205529\\
192.459106445804 -25.7465695162502\\
192.605590820804 -25.7507567517839\\
192.752075195805 -25.7605660299138\\
192.898559570805 -25.7703383373275\\
193.045043945805 -25.7800670525414\\
193.191528320806 -25.7897463341433\\
193.338012695806 -25.7993710019491\\
193.484497070807 -25.8089364391316\\
193.630981445807 -25.8184385111937\\
193.777465820807 -25.8278734985488\\
193.923950195808 -25.8382749149932\\
194.070434570808 -25.8468168469201\\
194.216918945808 -25.8550405972659\\
194.363403320809 -25.8632884918026\\
194.509887695809 -25.8715606984672\\
194.65637207081 -25.8798573494556\\
194.80285644581 -25.8859613042867\\
194.94934082081 -25.8942127451569\\
195.095825195811 -25.9025150140189\\
195.242309570811 -25.9108654350468\\
195.388793945811 -25.9192611471562\\
195.535278320812 -25.9276990889953\\
195.681762695812 -25.9359934412844\\
195.828247070813 -25.944141667577\\
195.974731445813 -25.9523225376284\\
196.121215820813 -25.9605400059189\\
196.267700195814 -25.9687982795162\\
196.414184570814 -25.9771018297274\\
196.560668945814 -25.9854554034268\\
196.707153320815 -25.9938640338059\\
196.853637695815 -26.0093779836996\\
197.000122070816 -26.0174925702725\\
197.146606445816 -26.025554161189\\
197.293090820816 -26.0335728059497\\
197.439575195817 -26.0415604703952\\
197.586059570817 -26.0504215702568\\
197.732543945817 -26.0578339773733\\
197.879028320818 -26.0651974582944\\
198.025512695818 -26.0725348584754\\
198.171997070819 -26.0798737014687\\
198.318481445819 -26.0973930591932\\
198.483276367694 -26.1054703486451\\
198.64807128957 -26.1155076847115\\
198.812866211445 -26.1256567587125\\
198.977661133321 -26.1359449233756\\
199.142456055196 -26.1464067786864\\
199.313354492696 -26.151200963585\\
199.484252930197 -26.1629628377545\\
199.655151367697 -26.1749430130654\\
199.826049805198 -26.1871395105798\\
199.996948242698 -26.1995493304507\\
200.167846680199 -26.212168273529\\
200.338745117699 -26.2249907308928\\
200.509643555199 -26.2380094349583\\
200.6805419927 -26.2507006049256\\
200.8514404302 -26.2622953698414\\
201.022338867701 -26.2738442918624\\
201.193237305201 -26.2844878510271\\
201.364135742702 -26.2950322597634\\
201.535034180202 -26.3055985277281\\
201.705932617703 -26.3162142983711\\
201.876831055203 -26.3147954607911\\
202.047729492703 -26.3247927183584\\
202.218627930204 -26.3350254645128\\
202.389526367704 -26.3455675189283\\
202.560424805205 -26.356504409863\\
202.731323242705 -26.3679351068187\\
202.902221680206 -26.3799739646576\\
203.073120117706 -26.392752877561\\
203.244018555206 -26.4064236241412\\
203.414916992707 -26.4209911614714\\
203.585815430207 -26.4368332142566\\
203.756713867708 -26.4542274023556\\
203.927612305208 -26.47345094146\\
204.098510742709 -26.4948138947506\\
204.269409180209 -26.5179032535267\\
204.44030761771 -26.5375488009396\\
204.61120605521 -26.5608190614993\\
204.78210449271 -26.5886164448564\\
204.953002930211 -26.6219114356629\\
205.123901367711 -26.6751846988183\\
205.294799805212 -26.7181976692983\\
205.465698242712 -26.7670309543096\\
205.636596680213 -26.822166138626\\
205.807495117713 -26.883823528508\\
205.978393555213 -26.9577022575691\\
206.149291992714 -27.0300411229092\\
206.320190430214 -27.105680183573\\
206.491088867715 -27.1824569765396\\
206.661987305215 -27.2577116449854\\
206.832885742716 -27.3376052321834\\
207.003784180216 -27.3867942607501\\
207.174682617716 -27.4391617757752\\
207.345581055217 -27.4826709906513\\
207.516479492717 -27.5170408672292\\
207.687377930218 -27.5425075734097\\
207.858276367718 -27.5525086185773\\
208.029174805219 -27.5631092135597\\
208.200073242719 -27.5678000132879\\
208.37097168022 -27.5675100432854\\
208.54187011772 -27.5631325021894\\
208.71276855522 -27.5554865183595\\
208.883666992721 -27.5452966413813\\
209.054565430221 -27.5331856293085\\
209.225463867722 -27.5196762659479\\
209.396362305222 -27.5124181606417\\
209.567260742723 -27.4900076533244\\
209.738159180223 -27.4747814860117\\
209.909057617724 -27.4591151624668\\
210.079956055224 -27.4431692819916\\
210.250854492724 -27.4270785942186\\
210.421752930225 -27.4109569165823\\
210.592651367725 -27.3949013188407\\
210.763549805226 -27.3789956676837\\
210.934448242726 -27.3633136248114\\
211.105346680227 -27.3538514690075\\
211.276245117727 -27.3374945252454\\
211.447143555227 -27.3214236220134\\
211.618041992728 -27.3057279168407\\
211.788940430228 -27.2904986542326\\
211.959838867729 -27.2780411540343\\
212.130737305229 -27.2622180085083\\
212.30163574273 -27.2469765278673\\
212.47253418023 -27.23247382682\\
212.64343261773 -27.2188793095021\\
212.814331055231 -27.2233186887371\\
212.985229492731 -27.2111584692776\\
213.156127930232 -27.2001526328786\\
213.327026367732 -27.1905761578352\\
213.497924805233 -27.1827239653315\\
213.668823242733 -27.1780991952243\\
213.839721680234 -27.1617370527863\\
214.010620117734 -27.1622329804537\\
214.181518555234 -27.1658681064301\\
214.352416992735 -27.1729902483863\\
214.523315430235 -27.1839446965329\\
214.694213867736 -27.1952516410933\\
214.865112305236 -27.2119539713234\\
215.036010742737 -27.2336951937391\\
215.206909180237 -27.260862187316\\
215.377807617737 -27.2937691857896\\
215.548706055238 -27.3326298237291\\
215.719604492738 -27.3757437946323\\
215.890502930239 -27.4244890564406\\
216.061401367739 -27.4793621289583\\
216.23229980524 -27.5401043096953\\
216.40319824274 -27.6046761275059\\
216.574096680241 -27.6745215358017\\
216.744995117741 -27.748128200567\\
216.915893555241 -27.82457003535\\
217.086791992742 -27.9028175305552\\
217.257690430242 -27.9616950041485\\
217.428588867743 -28.0352303998829\\
217.599487305243 -28.1079895919856\\
217.770385742744 -28.1793206034293\\
217.941284180244 -28.2486458654174\\
218.112182617745 -28.2983711457712\\
218.283081055245 -28.3567146074938\\
218.453979492745 -28.4121826221737\\
218.624877930246 -28.4648209128678\\
218.795776367746 -28.5147377860802\\
218.966674805247 -28.5090598947309\\
219.137573242747 -28.5507437094554\\
219.308471680248 -28.5913481789121\\
219.479370117748 -28.6312153957023\\
219.650268555248 -28.6706993164482\\
219.821166992749 -28.5493482880209\\
219.992065430249 -28.6070622954905\\
220.16296386775 -28.6676892553864\\
220.33386230525 -28.7316808977301\\
220.504760742751 -28.7995647883176\\
220.675659180251 -28.8719611204367\\
220.846557617751 -28.949604277504\\
221.017456055252 -29.0333707733932\\
221.188354492752 -29.1243158281438\\
221.359252930253 -29.3842368429732\\
221.530151367753 -29.3545429861389\\
221.701049805254 -29.4734997644688\\
221.871948242754 -29.6063020770268\\
222.042846680254 -29.7555048333682\\
222.213745117755 -29.9243205010225\\
222.384643555255 -29.8122146787287\\
222.555541992756 -30.0115703872789\\
222.726440430256 -30.2532450476807\\
222.897338867757 -30.5505329582682\\
223.068237305257 -30.9223301310067\\
223.239135742758 -31.2438919046084\\
223.410034180258 -31.6576466325286\\
223.580932617758 -32.1581113180654\\
223.751831055259 -32.7654484572605\\
223.922729492759 -33.4929447454464\\
224.09362793026 -33.4708083908835\\
224.26452636776 -33.527253572519\\
224.435424805261 -33.077308312566\\
224.606323242761 -31.9896011021988\\
224.777221680261 -30.3827291405161\\
224.948120117762 -25.6640413380784\\
225.119018555262 -24.6417182864768\\
225.289916992763 -24.0526299854679\\
225.460815430263 -23.7672933268141\\
225.631713867764 -23.6698939579992\\
225.802612305264 -23.5681979993685\\
225.973510742764 -23.7406419199652\\
226.144409180265 -23.8271930239619\\
226.315307617765 -23.9209152006545\\
226.486206055266 -24.1430085824882\\
226.657104492766 -24.0877587912236\\
226.828002930267 -24.1643343306155\\
226.998901367767 -24.2182987856847\\
227.169799805268 -24.2519082853404\\
227.340698242768 -24.2669725556808\\
227.511596680268 -24.2648905957008\\
227.682495117769 -24.2467002837893\\
227.853393555269 -24.2131271361305\\
228.02429199277 -24.1646276695296\\
228.19519043027 -24.2388936404962\\
228.366088867771 -24.1702515536319\\
228.536987305271 -24.0854219986047\\
228.707885742772 -23.9842535707094\\
228.878784180272 -23.866483678079\\
229.049682617772 -24.0172052461077\\
229.244995117773 -23.5438930548199\\
229.440307617773 -23.3825486289014\\
229.635620117774 -23.2047625153584\\
229.830932617774 -23.0131453957665\\
230.026245117775 -22.8129333070322\\
230.221557617775 -22.6139258377015\\
230.416870117776 -22.3869454855474\\
230.612182617776 -22.211432579123\\
230.807495117777 -22.1508664155606\\
231.002807617777 -22.2854218237372\\
231.198120117778 -22.7083403142483\\
231.393432617778 -23.4980745459122\\
231.588745117779 -24.6858875096399\\
231.784057617779 -26.2486628162683\\
231.97937011778 -29.8526915246967\\
232.17468261778 -32.2599355890742\\
232.369995117781 -34.9658361819926\\
232.565307617781 -38.0892614823916\\
232.760620117782 -41.8610635455055\\
232.955932617782 -48.8138413967633\\
233.151245117783 -51.4407994126277\\
233.346557617783 -47.4821871640755\\
233.541870117784 -44.171909217649\\
233.737182617784 -41.5506890733268\\
233.932495117785 -39.8270824395761\\
234.127807617785 -38.5027049855728\\
234.323120117786 -37.5820840708975\\
234.518432617786 -36.862978345797\\
234.713745117787 -36.2850210427657\\
234.909057617787 -35.8099247900682\\
235.104370117788 -35.412178901986\\
235.299682617788 -35.0741062625836\\
235.494995117789 -34.7830598688053\\
235.690307617789 -34.5297471096381\\
235.88562011779 -34.4598508674441\\
236.08093261779 -34.1507886605827\\
236.276245117791 -33.9577478735685\\
236.471557617791 -33.7852922199116\\
236.666870117792 -33.6303839366994\\
236.862182617792 -33.4905426365942\\
237.057495117793 -33.3338781985935\\
237.252807617793 -33.223837131464\\
237.448120117794 -33.1231944532818\\
237.643432617794 -33.0307247596205\\
237.838745117795 -32.9454006945798\\
238.034057617795 -32.8663563339975\\
238.229370117796 -32.7928581299352\\
238.424682617796 -32.7242816783948\\
238.619995117797 -32.660093017104\\
238.815307617797 -32.7346942242527\\
239.010620117798 -32.6232664402804\\
239.205932617798 -32.5639080077562\\
239.401245117799 -32.5064263380507\\
239.596557617799 -32.4507882225118\\
239.7918701178 -32.3969874389171\\
239.9871826178 -32.3263537053905\\
240.182495117801 -32.2793993683422\\
240.377807617801 -32.2346518109877\\
240.573120117802 -32.1920576662692\\
240.768432617802 -32.1515675902702\\
240.963745117803 -32.1082359257616\\
241.159057617803 -32.0738031021286\\
241.354370117804 -32.0412606766681\\
241.549682617804 -32.009736692787\\
241.744995117805 -31.9815429418386\\
241.940307617805 -31.9457579231408\\
242.135620117806 -31.9198022732373\\
242.330932617806 -31.8956690403303\\
242.526245117807 -31.8732727986792\\
242.721557617807 -31.8525276332891\\
242.916870117808 -31.8257840023605\\
243.112182617808 -31.8085286047899\\
243.307495117809 -31.7928172599159\\
243.502807617809 -31.7785609490031\\
243.69812011781 -31.7656684622119\\
243.89343261781 -31.7540463063934\\
244.088745117811 -31.7435987594502\\
244.284057617811 -31.7342281134441\\
244.479370117812 -31.725835139664\\
244.674682617812 -31.7210611659652\\
244.869995117813 -31.7107723954783\\
245.065307617813 -31.7038736289862\\
245.260620117814 -31.6975849403747\\
245.455932617814 -31.6918493717496\\
245.651245117815 -31.6866139071536\\
245.846557617815 -31.6801053328646\\
246.041870117816 -31.6756518005554\\
246.237182617816 -31.6715928803419\\
246.432495117817 -31.6678755528561\\
246.627807617817 -31.6644483346627\\
246.823120117818 -31.6612613695805\\
247.018432617818 -31.6582665494522\\
247.213745117819 -31.6554176779345\\
247.409057617819 -31.6526706937267\\
247.60437011782 -31.6518118391423\\
247.79968261782 -31.6489599144045\\
247.994995117821 -31.6458686816515\\
248.190307617821 -31.6424612518679\\
248.385620117822 -31.6390818588744\\
248.580932617822 -31.6357315725442\\
248.776245117823 -31.632416408036\\
248.971557617823 -31.6291479475622\\
249.166870117824 -31.6259440680115\\
249.362182617824 -31.6228297846719\\
249.557495117825 -31.6239305472731\\
249.752807617825 -31.6153980778093\\
249.948120117826 -31.6084124471519\\
250.143432617826 -31.6026302032813\\
250.338745117827 -31.5983283013279\\
250.534057617827 -31.595717054836\\
250.729370117828 -31.5949462110098\\
250.924682617828 -31.59611305037\\
251.119995117829 -31.5992709807578\\
251.315307617829 -31.6044376920633\\
251.51062011783 -31.6103363609421\\
251.70593261783 -31.6171430977537\\
251.901245117831 -31.6258902383107\\
252.096557617831 -31.6366356411305\\
252.291870117832 -31.6494127952218\\
252.487182617832 -31.6714091687634\\
252.682495117833 -31.6874131905298\\
252.877807617833 -31.7052295536897\\
253.073120117834 -31.7248431015911\\
253.268432617834 -31.7436302058261\\
253.463745117835 -31.7629611755192\\
253.659057617835 -31.7818801455167\\
253.854370117836 -31.8062691277588\\
254.049682617836 -31.8317260454014\\
254.244995117837 -31.8581765710387\\
254.440307617837 -31.8889543787424\\
254.635620117838 -31.9195063089354\\
254.830932617838 -31.951429895657\\
255.026245117839 -31.9846316799181\\
255.221557617839 -32.0190055053129\\
255.41687011784 -32.0662009281048\\
255.61218261784 -32.0985705589706\\
255.807495117841 -32.1313500486038\\
256.002807617841 -32.1644608236099\\
256.198120117842 -32.1980633077784\\
256.393432617842 -32.2321350732615\\
256.588745117843 -32.2639433380782\\
256.784057617843 -32.2992763243178\\
256.979370117844 -32.3350676628531\\
257.174682617844 -32.3712729142109\\
257.369995117845 -32.4078409221811\\
257.565307617845 -32.444711369714\\
257.760620117846 -32.4818116836968\\
257.955932617846 -32.5190531196816\\
258.151245117847 -32.5563258159957\\
258.346557617847 -32.5997553185717\\
258.541870117848 -32.6364959133125\\
258.737182617848 -32.6748310300667\\
258.932495117849 -32.711371052712\\
259.127807617849 -32.7464134556855\\
259.32312011785 -32.784010107514\\
259.536743164725 -32.8182160110293\\
259.750366211601 -32.8492454818822\\
259.963989258476 -32.8746508334309\\
260.177612305352 -32.8926927311937\\
260.391235352227 -32.9013758177339\\
260.610961914728 -32.8695182614746\\
260.830688477229 -32.8527530184028\\
261.050415039729 -32.8205117439068\\
261.27014160223 -32.7709716175247\\
261.48986816473 -32.7029816545535\\
261.709594727231 -32.565983680472\\
261.929321289731 -32.4719543619443\\
262.149047852232 -32.3634824564317\\
262.368774414732 -32.2430563444825\\
262.588500977233 -32.1144494554525\\
262.808227539734 -31.9314580747326\\
263.027954102234 -31.8016340939599\\
263.247680664735 -31.6889141404166\\
263.467407227235 -31.5970213693188\\
263.687133789736 -31.5269403145758\\
};
\addplot [LRM,interpol, forget plot]
table[row sep=crcr]{
263.687133789736 -31.5269403145758\\
263.906860352236 -31.3901801529065\\
264.126586914737 -31.3726967041074\\
264.346313477238 -31.3743828557768\\
264.566040039738 -31.3900487807429\\
264.785766602239 -31.4152507194275\\
265.005493164739 -31.3668341954863\\
265.22521972724 -31.3955861977957\\
265.44494628974 -31.4270144727208\\
265.664672852241 -31.4595576597279\\
265.884399414741 -31.4921154396958\\
266.104125977242 -31.4885854454063\\
266.323852539743 -31.532983885469\\
266.543579102243 -31.5748036488828\\
266.763305664744 -31.610805455277\\
266.983032227244 -31.6484743864946\\
267.202758789745 -31.6797512113575\\
267.422485352245 -31.7071679593001\\
267.642211914746 -31.7307086089437\\
267.861938477246 -31.7503973252178\\
268.081665039747 -31.8132014244562\\
268.301391602248 -31.8279805090147\\
268.521118164748 -31.83642152425\\
268.740844727249 -31.8385477499292\\
268.960571289749 -31.8344365623643\\
269.18029785225 -31.8717654673752\\
269.40002441475 -31.8512303525285\\
269.619750977251 -31.8239521202652\\
269.839477539752 -31.7904847506851\\
270.059204102252 -31.7513561387807\\
270.278930664753 -31.7110673788251\\
270.498657227253 -31.6064709467951\\
270.718383789754 -31.5586374300233\\
270.938110352254 -31.5075795722879\\
271.157836914755 -31.4534557785115\\
271.377563477255 -31.3964192936616\\
271.597290039756 -31.3366278926152\\
271.817016602257 -31.2742541880012\\
272.036743164757 -31.2094968735071\\
272.256469727258 -31.1425932336514\\
272.476196289758 -31.0890719180339\\
272.695922852259 -31.0115320543469\\
272.915649414759 -30.9316613390779\\
273.13537597726 -30.8500122713362\\
273.35510253976 -30.7673287597232\\
273.574829102261 -30.7086695111429\\
273.794555664762 -30.6183629276532\\
274.014282227262 -30.5384760079144\\
274.234008789763 -30.4634415570376\\
274.453735352263 -30.3956978928331\\
274.673461914764 -30.338730331478\\
274.893188477265 -30.29537962555\\
275.112915039765 -30.2693114208008\\
275.332641602266 -30.2639842220723\\
275.552368164766 -30.2825143326447\\
275.772094727267 -30.366511225513\\
275.991821289767 -30.4253353336643\\
276.211547852268 -30.5104419769731\\
276.431274414768 -30.6220220042445\\
276.651000977269 -30.75899260293\\
276.870727539769 -30.9267457946188\\
277.09045410227 -31.0870642225141\\
277.310180664771 -31.2778137942507\\
277.529907227271 -31.4786004226747\\
277.749633789772 -31.6844886671856\\
277.969360352272 -31.8907010710358\\
278.189086914773 -32.0910342804145\\
278.408813477273 -32.2806163391763\\
278.628540039774 -32.458226129084\\
278.848266602274 -32.6222238185827\\
279.067993164775 -32.7716772681902\\
279.287719727276 -32.9062420786643\\
279.507446289776 -33.0260340769032\\
279.727172852277 -33.1315076310141\\
279.946899414777 -33.2233478491007\\
280.166625977278 -33.3095177763669\\
280.386352539778 -33.3215651917666\\
280.606079102279 -33.3605146509859\\
280.82580566478 -33.391012958911\\
281.04553222728 -33.4141838914651\\
281.265258789781 -33.430999912261\\
281.484985352281 -33.4422904859273\\
281.704711914782 -33.4487534455852\\
281.924438477282 -33.4509673387629\\
282.144165039783 -33.4494034414\\
282.363891602284 -33.4419619657737\\
282.583618164784 -33.4297287119019\\
282.803344727285 -33.414345505503\\
283.023071289785 -33.3960754906565\\
283.242797852286 -33.3751319429633\\
283.462524414786 -33.3506366627876\\
283.682250977287 -33.3223937242144\\
283.901977539787 -33.2916480026282\\
284.121704102288 -33.2585393630247\\
284.341430664789 -33.2231851275214\\
284.561157227289 -33.2107344257165\\
284.78088378979 -33.1496865592908\\
285.00061035229 -33.1003919997783\\
285.220336914791 -33.050467245522\\
285.440063477291 -32.9999596585823\\
285.659790039792 -32.9488948295862\\
285.879516602293 -32.848037147798\\
286.099243164793 -32.7968679514239\\
286.318969727294 -32.745735470739\\
286.538696289794 -32.6946051558981\\
286.758422852295 -32.6434237840134\\
286.978149414795 -32.5754816277196\\
287.197875977296 -32.5286837955977\\
287.417602539796 -32.4819190956031\\
287.637329102297 -32.434993128376\\
287.857055664798 -32.3877033827602\\
288.076782227298 -32.2550797828972\\
288.296508789799 -32.2051167233424\\
288.516235352299 -32.155189596999\\
288.7359619148 -32.1050987984326\\
288.9556884773 -32.0546271516158\\
289.175415039801 -31.9738571308458\\
289.395141602301 -31.9171792083696\\
289.614868164802 -31.8602894714774\\
289.834594727303 -31.8029458533005\\
290.054321289803 -31.7448426246463\\
290.298461914804 -31.6082849521614\\
290.542602539804 -31.5331844310269\\
290.786743164805 -31.4571160035533\\
291.030883789806 -31.3792618441292\\
291.275024414806 -31.2986199772736\\
291.519165039807 -31.2139751536187\\
291.763305664807 -31.1238573952107\\
292.007446289808 -31.026483138498\\
292.251586914809 -30.9050291310336\\
292.495727539809 -30.7666570717087\\
292.73986816481 -30.5666960858435\\
292.984008789811 -30.3870054882966\\
293.228149414811 -30.181011406833\\
293.472290039812 -29.9413411983721\\
293.716430664812 -29.6579614801189\\
293.960571289813 -29.3168937906834\\
294.204711914814 -28.8981302088426\\
294.448852539814 -28.3721108128153\\
294.692993164815 -27.6840513690468\\
294.937133789816 -26.7636303302141\\
295.181274414816 -25.4903578021996\\
295.425415039817 -23.6258542069692\\
295.669555664818 -20.7099954273824\\
295.913696289818 -16.0288407328474\\
296.157836914819 -15.2492693679422\\
296.401977539819 -21.4057627916235\\
296.64611816482 -25.7455089436762\\
296.890258789821 -28.576912826375\\
297.134399414821 -30.4375947079392\\
297.378540039822 -31.671532157827\\
297.622680664823 -32.4084728592741\\
297.866821289823 -32.8932181118017\\
298.110961914824 -33.1920011978561\\
298.355102539824 -33.3714928546647\\
298.599243164825 -33.4744949085233\\
298.843383789826 -33.5158724870739\\
299.087524414826 -33.5406138636428\\
299.331665039827 -33.5451354421137\\
299.575805664827 -33.5362993701218\\
299.819946289828 -33.5185881353569\\
300.064086914829 -33.4795979701353\\
300.308227539829 -33.4512520626684\\
300.55236816483 -33.4209085870257\\
300.796508789831 -33.3893553263563\\
301.040649414831 -33.3571027260049\\
301.284790039832 -33.3232254109695\\
301.528930664832 -33.2894685092037\\
301.773071289833 -33.2558127445549\\
302.017211914834 -33.2223045214756\\
302.261352539834 -33.1889481446758\\
302.505493164835 -33.1557187986147\\
302.749633789836 -33.1225715075595\\
302.993774414836 -33.0894473413309\\
303.237915039837 -33.056277717714\\
303.482055664837 -33.0216854940875\\
303.726196289838 -32.983863232595\\
303.970336914839 -32.9459563618625\\
304.214477539839 -32.9079314341627\\
304.45861816484 -32.8697512642119\\
304.702758789841 -32.8436321355658\\
304.946899414841 -32.8024818954701\\
305.191040039842 -32.7609300924174\\
305.435180664842 -32.7189750367724\\
305.679321289843 -32.6766139233611\\
305.923461914844 -32.6557313494504\\
306.167602539844 -32.6082211504967\\
306.411743164845 -32.5599760387791\\
306.655883789846 -32.5110837539898\\
306.900024414846 -32.4616349088099\\
307.144165039847 -32.4244458434748\\
307.388305664847 -32.3711359168351\\
307.632446289848 -32.3173163682046\\
307.876586914849 -32.263181266477\\
308.120727539849 -32.2089362393291\\
308.36486816485 -32.2049538611385\\
308.609008789851 -32.1311897302511\\
308.853149414851 -32.0825602509631\\
309.097290039852 -32.0337722417422\\
309.341430664852 -31.9848651305663\\
309.585571289853 -31.9358842013762\\
309.829711914854 -31.8830240110258\\
310.073852539854 -31.8328604039185\\
310.317993164855 -31.7828833805301\\
310.562133789855 -31.7332459831131\\
310.806274414856 -31.6841353650545\\
311.050415039857 -31.6357814166532\\
311.294555664857 -31.5884676436258\\
311.538696289858 -31.5425447659461\\
311.782836914859 -31.4984474405379\\
312.026977539859 -31.501198735649\\
312.27111816486 -31.4717078033009\\
312.51525878986 -31.4454278279899\\
312.759399414861 -31.4223163283608\\
313.003540039862 -31.4021311859213\\
313.247680664862 -31.3960798551599\\
313.491821289863 -31.3811630503362\\
313.735961914864 -31.3660614981352\\
313.980102539864 -31.3490881324226\\
314.224243164865 -31.3283357647962\\
314.468383789865 -31.3756935277722\\
314.712524414866 -31.335602779636\\
314.956665039867 -31.2858893993193\\
315.200805664867 -31.2263966456722\\
315.444946289868 -31.1572719891703\\
315.689086914869 -31.0984951532329\\
315.933227539869 -31.0130468556391\\
316.17736816487 -30.9193775608406\\
316.42150878987 -30.8179209071942\\
316.665649414871 -30.7048025593507\\
316.909790039872 -30.5855187602214\\
317.153930664872 -30.3712095047247\\
317.398071289873 -30.2543912725284\\
317.642211914874 -30.1342415015008\\
317.886352539874 -30.0104238299193\\
318.130493164875 -29.8825773036885\\
318.374633789876 -29.6109387978503\\
318.618774414876 -29.4848907737494\\
318.862915039877 -29.3534303572781\\
319.107055664877 -29.2161388422004\\
319.351196289878 -29.0725787978221\\
319.595336914879 -28.6180548490587\\
319.839477539879 -28.4509542274686\\
320.08361816488 -28.2826043235247\\
320.32775878988 -28.111637763318\\
320.571899414881 -27.9365910417285\\
320.840454102382 -27.7375603010522\\
321.109008789882 -27.529863329788\\
321.377563477383 -27.3116110057561\\
321.646118164884 -27.0809468686519\\
321.914672852385 -26.9046171587165\\
322.183227539885 -26.6330208799723\\
322.451782227386 -26.345412306559\\
322.720336914887 -26.039523982397\\
322.988891602387 -25.7128196402546\\
323.257446289888 -25.3923599139098\\
323.526000977389 -25.0025051310942\\
323.794555664889 -24.5826798071282\\
324.06311035239 -24.1287690553215\\
324.331665039891 -23.6358475411594\\
324.600219727391 -23.0939841360213\\
324.868774414892 -22.4814836429474\\
325.137329102393 -21.8215607111337\\
325.405883789893 -21.0867773393439\\
325.674438477394 -20.2618189422431\\
325.942993164895 -19.3261953852889\\
326.211547852396 -18.2516459753371\\
326.480102539896 -16.9977549338679\\
326.748657227397 -15.5041882239135\\
327.017211914897 -13.6763878545014\\
327.285766602398 -11.4070238335127\\
327.554321289899 -8.35997967298158\\
327.8228759774 -4.48258857831752\\
328.0914306649 -2.66900933908408\\
328.359985352401 -6.1406046532835\\
328.628540039902 -9.88163239439484\\
328.897094727402 -12.7230704074531\\
329.165649414903 -14.9633056691102\\
329.434204102404 -16.7685949721971\\
329.702758789904 -18.2738188747992\\
329.971313477405 -19.5823042211181\\
330.239868164906 -20.7090592130187\\
330.508422852406 -21.7040080959356\\
330.776977539907 -22.5922258366463\\
331.045532227408 -23.3921383358124\\
331.314086914909 -24.1251469740183\\
331.582641602409 -24.7660671793578\\
331.85119628991 -25.3469639030171\\
332.119750977411 -25.8744017453809\\
332.388305664911 -26.3537184841152\\
332.656860352412 -26.8792181259669\\
332.925415039913 -27.2562547697682\\
333.193969727413 -27.6009585507574\\
333.462524414914 -27.9085698320192\\
333.731079102415 -28.1815021632327\\
333.999633789915 -28.4219340989963\\
334.268188477416 -28.6318571359229\\
334.536743164917 -28.813110918343\\
334.805297852417 -28.9674088493113\\
335.073852539918 -29.0963565253087\\
335.342407227419 -29.2787151406842\\
335.61096191492 -29.2976473378284\\
335.87951660242 -29.371169464992\\
336.148071289921 -29.4247068891603\\
336.416625977421 -29.4591566605525\\
336.685180664922 -29.4753645208183\\
336.953735352423 -29.394211863616\\
337.222290039924 -29.3721689469725\\
337.490844727424 -29.3363086309808\\
337.759399414925 -29.2874224030713\\
338.027954102426 -29.2262078140409\\
338.296508789926 -29.1532716924917\\
338.565063477427 -29.0691337501123\\
338.833618164928 -28.9742304323991\\
339.102172852428 -28.8689189013188\\
339.370727539929 -28.7536013314246\\
339.63928222743 -28.6228674396465\\
339.90783691493 -28.4818456792588\\
340.176391602431 -28.3306861441749\\
340.444946289932 -28.169501615806\\
340.713500977432 -28.0561764647846\\
340.982055664933 -27.8750538874737\\
341.250610352434 -27.6826352696805\\
341.519165039935 -27.4789723879767\\
341.787719727435 -27.2641315681809\\
342.056274414936 -27.0542418472117\\
342.324829102437 -26.8147599102878\\
342.593383789937 -26.5645280126138\\
342.861938477438 -26.3039929059389\\
343.130493164939 -26.0337959101438\\
343.399047852439 -25.8019550155767\\
343.66760253994 -25.5127223673254\\
343.936157227441 -25.2156682451297\\
344.204711914941 -24.9131315598963\\
344.473266602442 -24.6082878814406\\
344.741821289943 -24.3274212920937\\
345.010375977444 -24.0117155796884\\
345.278930664944 -23.7162977966733\\
345.547485352445 -23.4443804576632\\
345.816040039946 -23.2064188012125\\
346.084594727446 -23.0138218272038\\
346.353149414947 -22.8739892926612\\
346.621704102448 -22.8043275185913\\
346.890258789948 -22.8100682685208\\
347.158813477449 -22.8957686162038\\
347.42736816495 -23.0613652168039\\
347.69592285245 -23.3021812576249\\
347.964477539951 -23.6097470788875\\
348.233032227452 -23.9731488691939\\
348.501586914952 -24.3804949117148\\
348.770141602453 -24.8199308643241\\
349.038696289954 -25.265450159137\\
349.307250977454 -25.7375884599029\\
349.575805664955 -26.218692950328\\
349.844360352456 -26.7029002128966\\
350.112915039957 -27.1855909104627\\
350.387573243082 -27.6640403183771\\
350.662231446208 -28.1421760699594\\
350.936889649334 -28.6100494437229\\
351.211547852459 -29.0661887115235\\
351.486206055585 -29.5095631331017\\
351.779174805586 -29.9394104380551\\
352.072143555586 -30.3775951899398\\
352.365112305587 -30.8000593864217\\
352.658081055588 -31.2067654891404\\
352.951049805589 -31.5977637620872\\
353.24401855559 -31.969272032095\\
353.53698730559 -32.3315549617326\\
353.829956055591 -32.6776883627824\\
354.122924805592 -33.0071231058596\\
354.415893555593 -33.3275782416762\\
354.708862305593 -33.6261890640932\\
355.001831055594 -33.9101221791179\\
355.294799805595 -34.1792614986344\\
355.587768555595 -34.4337462124186\\
355.880737305596 -34.6737132523814\\
356.173706055597 -34.8993000435364\\
356.466674805598 -35.1106475555929\\
356.759643555599 -35.307903456329\\
357.052612305599 -35.4912252007751\\
357.3455810556 -35.6821360168129\\
357.638549805601 -35.8353540586449\\
357.931518555601 -35.9764637686751\\
358.224487305602 -36.1051626728411\\
358.517456055603 -36.2216233279158\\
358.810424805604 -36.32600907046\\
359.103393555604 -36.4184749886253\\
359.396362305605 -36.4991686740706\\
359.689331055606 -36.5651152027635\\
359.982299805607 -36.6200178468277\\
360.275268555607 -36.6641176430207\\
360.568237305608 -36.6906302387305\\
360.861206055609 -36.7131146328909\\
361.15417480561 -36.725261615683\\
361.44714355561 -36.7271266612125\\
361.740112305611 -36.7187466298451\\
362.033081055612 -36.7001397899267\\
362.326049805613 -36.6713056632307\\
362.619018555613 -36.6322246954289\\
362.911987305614 -36.5828577506984\\
363.204956055615 -36.543984361539\\
363.497924805616 -36.4740653337146\\
363.790893555616 -36.3924080502107\\
364.083862305617 -36.2988126691656\\
364.376831055618 -36.1930775015996\\
364.669799805619 -36.1955495580502\\
364.96276855562 -35.99025012398\\
365.25573730562 -35.8397932775778\\
365.548706055621 -35.6747022294139\\
365.841674805622 -35.487626755101\\
366.134643555622 -35.3020387985083\\
366.427612305623 -34.9862301039562\\
366.720581055624 -34.7619937514439\\
367.013549805625 -34.5244596717551\\
367.306518555625 -34.2750637612326\\
367.599487305626 -34.015422113019\\
367.892456055627 -33.7178130172524\\
368.185424805628 -33.4225906317978\\
368.478393555628 -33.1276267893983\\
368.771362305629 -32.8345592587474\\
369.06433105563 -32.5443794310504\\
369.357299805631 -32.2574773893608\\
369.650268555631 -31.9737185317095\\
369.943237305632 -31.6925339230823\\
370.236206055633 -31.4130105698216\\
370.529174805634 -31.2515887633916\\
370.822143555635 -30.9632000014435\\
371.115112305635 -30.6696765811269\\
371.408081055636 -30.3694701353106\\
371.701049805637 -30.0610227217277\\
371.994018555637 -29.7964232376459\\
372.286987305638 -29.4313947606944\\
372.579956055639 -29.1004023411138\\
372.87292480564 -28.7526303161626\\
373.16589355564 -28.386073143502\\
373.458862305641 -27.9985795947493\\
373.751831055642 -27.5773469090124\\
374.044799805643 -27.1470211196967\\
374.337768555643 -26.6856640226189\\
374.630737305644 -26.1881239947283\\
374.923706055645 -25.6796639157662\\
375.216674805646 -25.0674646311406\\
375.509643555646 -24.4624443667724\\
375.802612305647 -23.8132210253199\\
376.095581055648 -23.1152484690795\\
376.388549805649 -22.364422154722\\
376.681518555649 -21.5582705890852\\
376.97448730565 -20.6983949583796\\
377.267456055651 -19.7952459191616\\
377.560424805652 -18.8768494906261\\
377.853393555652 -18.1332335049642\\
378.146362305653 -17.4137966969697\\
378.439331055654 -16.961932587336\\
378.732299805655 -16.903959234589\\
379.025268555655 -17.284651300579\\
379.318237305656 -18.7966970335532\\
379.611206055657 -19.6536681372098\\
379.904174805658 -20.7442012759451\\
380.197143555658 -21.831159474442\\
380.490112305659 -22.8853039095827\\
380.78308105566 -23.8954984583214\\
381.088256836911 -24.8459493963922\\
381.393432618161 -25.792006070038\\
381.698608399412 -26.6918444321274\\
382.003784180663 -27.5519108517281\\
382.308959961914 -28.3788404879247\\
382.626342774415 -29.2107693359928\\
382.943725586915 -30.0212953280937\\
383.261108399416 -30.8177800157237\\
383.578491211917 -31.6078544760138\\
383.895874024418 -32.7068023906182\\
384.213256836919 -33.5684579924465\\
384.530639649419 -34.4606617914409\\
384.84802246192 -35.3996996253404\\
385.165405274421 -36.4072235086906\\
385.482788086922 -37.8651630047887\\
385.800170899423 -38.243691176615\\
386.117553711923 -39.6394121590549\\
386.434936524424 -41.3270051584028\\
386.752319336925 -43.4877200216787\\
387.069702149426 -46.4838283842212\\
387.387084961927 -48.4669429564688\\
387.704467774427 -51.4909893473592\\
388.021850586928 -48.5537603020973\\
388.339233399429 -43.2242641432109\\
388.65661621193 -39.6462474043873\\
388.973999024431 -36.1434947902515\\
389.291381836932 -32.8944538157663\\
389.608764649432 -29.6852281532279\\
389.926147461933 -26.4007298069855\\
390.243530274434 -24.3465468097297\\
390.560913086935 -22.1634570185329\\
390.878295899436 -22.2814716906714\\
391.195678711936 -23.6282884050623\\
391.513061524437 -25.1542500813222\\
391.830444336938 -26.6115920096304\\
392.147827149439 -27.8200611402083\\
392.46520996194 -28.8468399498258\\
392.782592774441 -29.7270670027906\\
393.099975586941 -30.4908750768059\\
393.417358399442 -31.1938045840959\\
393.734741211943 -31.6991950531363\\
394.052124024444 -32.2531279335237\\
394.369506836945 -32.7543845979595\\
394.686889649445 -33.2114440327549\\
395.004272461946 -33.6308919219251\\
395.321655274447 -33.9174282720969\\
395.639038086948 -34.2803687278445\\
395.956420899449 -34.6202693552566\\
396.273803711949 -34.9395208863939\\
396.59118652445 -35.2400803643467\\
396.908569336951 -35.5235695801456\\
397.225952149452 -35.7913492754464\\
397.543334961953 -36.0445755785854\\
397.860717774453 -36.2842432533861\\
398.178100586954 -36.541476877774\\
398.495483399455 -36.6916378160028\\
398.812866211956 -36.8762592393276\\
399.130249024457 -37.0526882790419\\
399.447631836957 -37.2216400616833\\
399.765014649458 -37.383724285665\\
400.082397461959 -37.5394556429212\\
400.39978027446 -37.6892617527058\\
400.717163086961 -37.8334890028161\\
401.034545899461 -37.9724066104647\\
401.351928711962 -38.1760594203737\\
401.669311524463 -38.2958973435838\\
401.986694336964 -38.4096678883328\\
402.304077149465 -38.5175964252784\\
402.621459961966 -38.6198681427565\\
402.938842774466 -38.7953096589441\\
403.256225586967 -38.8476511512178\\
403.573608399468 -38.9353469024334\\
403.890991211969 -39.0166321769446\\
404.20837402447 -39.0914226881009\\
404.52575683697 -39.1595952234936\\
404.843139649471 -39.1309083108298\\
405.160522461972 -39.1824631821072\\
405.477905274473 -39.2281025132172\\
405.795288086974 -39.2676239086492\\
406.112670899474 -39.3007670747762\\
406.430053711975 -39.3272040475097\\
406.747436524476 -39.3465268654251\\
407.064819336977 -39.3582319401668\\
407.382202149478 -39.36170011118\\
407.699584961979 -39.3996372146033\\
408.016967774479 -39.3732196509615\\
408.33435058698 -39.334557279368\\
408.651733399481 -39.2825013048735\\
408.969116211982 -39.2157103224617\\
409.286499024483 -39.1682290146481\\
409.603881836983 -39.0424671711854\\
409.921264649484 -38.9472948269128\\
410.238647461985 -38.8234499693598\\
410.556030274486 -38.6622706034757\\
410.873413086987 -38.5289676135328\\
411.196899415113 -38.2318571900353\\
411.520385743238 -37.9875872578004\\
411.843872071364 -37.7090919260506\\
412.16735839949 -37.4010055893031\\
412.490844727616 -37.0794617379519\\
412.832641602617 -36.7700083290574\\
413.174438477618 -36.5927118893658\\
413.516235352618 -36.7323916065294\\
413.858032227619 -37.4306165886071\\
414.19982910262 -40.4428575271993\\
414.541625977621 -42.7293913616314\\
414.883422852622 -45.50691458291\\
415.225219727623 -48.5399350799598\\
415.567016602624 -51.6946126790585\\
415.908813477625 -54.5669430180257\\
416.250610352625 -52.9841459210685\\
416.592407227626 -53.8834815310112\\
416.934204102627 -53.8496946861136\\
417.276000977628 -53.3477586730357\\
417.617797852629 -52.7240288547026\\
417.95959472763 -52.1281063125047\\
418.301391602631 -51.6047720520425\\
418.643188477632 -51.1586946941094\\
418.984985352632 -50.7819345028706\\
419.326782227633 -50.4699376298143\\
419.668579102634 -50.1854557298562\\
420.010375977635 -49.94234846083\\
420.352172852636 -49.7339503688101\\
420.693969727637 -49.5547063147049\\
421.035766602638 -49.535599291053\\
421.377563477639 -49.4029255201814\\
421.719360352639 -49.2862837120479\\
422.06115722764 -49.1832460581371\\
422.402954102641 -49.0918230219599\\
422.744750977642 -49.1575507701215\\
423.086547852643 -49.0200000327893\\
423.428344727644 -48.9331197424792\\
423.770141602645 -48.8539954034726\\
424.111938477645 -48.7817989156655\\
424.453735352646 -48.7158195049047\\
424.795532227647 -48.6335506921294\\
425.137329102648 -48.5891601688481\\
425.479125977649 -48.5484889997575\\
425.82092285265 -48.5029489022962\\
426.162719727651 -48.477111052245\\
426.504516602652 -48.4459201680597\\
426.846313477652 -48.4174693938378\\
427.188110352653 -48.3916064587865\\
427.529907227654 -48.3682084396033\\
427.871704102655 -48.5250778346721\\
428.213500977656 -48.4894265560501\\
428.555297852657 -48.4521680364098\\
428.897094727658 -48.4131876905674\\
429.238891602658 -48.3723689115802\\
429.580688477659 -48.3686529653903\\
429.92248535266 -48.3199898628531\\
430.264282227661 -48.2685113789059\\
430.606079102662 -48.2141752829648\\
430.947875977663 -48.1569268368118\\
431.289672852664 -48.1158054793547\\
431.631469727665 -48.023071618607\\
431.973266602665 -47.9599666392112\\
432.315063477666 -47.8938847126361\\
432.656860352667 -47.8246755271853\\
432.998657227668 -47.7521732115312\\
433.340454102669 -47.6761949055102\\
433.68225097767 -47.5965391077703\\
434.024047852671 -47.5129837662516\\
434.365844727672 -47.4239296987549\\
434.707641602673 -47.3244511371692\\
435.049438477674 -47.2197574505095\\
435.391235352674 -47.1095925348679\\
435.733032227675 -46.9936680417101\\
436.074829102676 -46.8716591965666\\
436.416625977677 -46.768145300618\\
436.758422852678 -46.6299706500065\\
437.100219727679 -46.4925536526991\\
437.44201660268 -46.3472574241709\\
437.78381347768 -46.1934154687795\\
438.125610352681 -46.0302835837447\\
438.467407227682 -45.8570279553889\\
438.809204102683 -45.672710955227\\
439.151000977684 -45.4762740934985\\
439.492797852685 -45.2665174359569\\
439.834594727686 -45.0415059194616\\
440.176391602687 -44.7994744858576\\
440.518188477687 -44.539196455775\\
440.859985352688 -44.2585853668917\\
441.201782227689 -43.9552163881448\\
441.54357910269 -43.6920124556802\\
441.891479493316 -43.3048180612711\\
442.239379883942 -42.9209300789418\\
442.587280274568 -42.4993253463119\\
442.935180665193 -42.0338244091193\\
443.283081055819 -41.5168122928759\\
443.64929199332 -40.7081314333229\\
444.015502930821 -40.0122863169775\\
444.381713868322 -39.2222052918925\\
444.747924805823 -38.3181909833723\\
445.114135743324 -37.2771422846517\\
445.480346680825 -35.9120316207529\\
445.846557618326 -34.554762921376\\
446.212768555827 -33.1439526104379\\
446.578979493328 -31.9180510441885\\
446.945190430829 -31.6455003661985\\
447.31140136833 -32.2971319390672\\
447.677612305831 -33.9523575442481\\
448.043823243332 -35.8543621751459\\
448.410034180832 -37.704044019432\\
448.776245118333 -39.4123730990357\\
449.142456055834 -40.8899091277447\\
449.508666993335 -42.3021818219503\\
449.874877930836 -43.599578906752\\
450.241088868337 -44.8008452623491\\
450.607299805838 -45.9215337158108\\
450.973510743339 -46.974201443615\\
451.33972168084 -47.9607841635332\\
451.705932618341 -48.8830708067676\\
452.072143555842 -49.7598645520731\\
452.438354493343 -50.5963685966063\\
452.804565430844 -51.3031337427791\\
453.170776368345 -52.057672391208\\
453.536987305846 -52.7823467900204\\
453.903198243346 -53.4792069863024\\
454.269409180847 -54.1497895568124\\
454.635620118348 -54.7951849567342\\
455.001831055849 -55.4160920259847\\
455.36804199335 -56.0128641353916\\
455.734252930851 -56.5855506566113\\
456.100463868352 -57.2391279889476\\
456.466674805853 -57.7361651915164\\
456.832885743354 -58.2047869076713\\
457.199096680855 -58.6448097542249\\
457.565307618356 -59.0561455841852\\
457.931518555857 -59.8177645696367\\
458.297729493358 -60.0402865984415\\
458.663940430859 -60.2519943855037\\
459.03015136836 -60.4540882069848\\
459.396362305861 -60.6473619991709\\
459.762573243361 -60.8563573299529\\
460.128784180863 -60.9168525004793\\
460.494995118363 -61.0570661080145\\
460.861206055864 -61.1956498955652\\
461.227416993365 -61.3325178760509\\
461.593627930866 -61.4674901426064\\
461.959838868367 -61.6003443049065\\
462.326049805868 -61.7308465573205\\
462.692260743369 -61.8587699517882\\
463.05847168087 -61.9839046702426\\
463.424682618371 -62.2689956803014\\
463.790893555872 -62.36834914417\\
464.157104493373 -62.4537752060036\\
464.523315430874 -62.5351738357248\\
464.889526368375 -62.6131277839386\\
465.255737305876 -62.6881806000906\\
465.621948243377 -62.7608439484116\\
465.988159180877 -62.8316042270888\\
466.354370118378 -62.9009285326691\\
466.720581055879 -62.9692699470028\\
467.08679199338 -63.2032029988667\\
467.453002930881 -63.2725247088381\\
467.819213868382 -63.3381398113234\\
468.185424805883 -63.400039562823\\
468.551635743384 -63.458175946347\\
468.917846680885 -63.5518419067061\\
469.284057618386 -63.6015070642069\\
469.650268555887 -63.6461509802125\\
470.016479493388 -63.6855354586505\\
470.382690430889 -63.7193694052605\\
470.74890136839 -63.8631623699693\\
471.115112305891 -63.7887888582447\\
471.481323243391 -63.7885294562987\\
471.847534180892 -63.7770288534816\\
472.213745118393 -63.7536227947759\\
472.579956055894 -63.7176311320205\\
472.96447754027 -63.5991116501064\\
473.348999024646 -63.5072896474409\\
473.733520509022 -63.3999162162655\\
474.118041993398 -63.2763795913778\\
474.502563477774 -63.1359839365038\\
474.893188477775 -62.8780414946009\\
475.283813477776 -62.6779902736674\\
475.674438477777 -62.4598606616242\\
476.065063477778 -62.2225804103177\\
476.455688477779 -61.9648238538173\\
476.84631347778 -61.6849740559078\\
477.236938477781 -61.3810758663442\\
477.627563477782 -61.0507777985939\\
478.018188477783 -60.6912603752362\\
478.408813477784 -60.3602392950833\\
478.799438477785 -59.9292760310339\\
479.190063477786 -59.4572665272005\\
479.580688477787 -58.9385688719608\\
479.971313477788 -58.3665264710708\\
480.361938477789 -57.7624826054087\\
480.75256347779 -56.8860391905645\\
481.143188477791 -56.0885720209228\\
481.533813477792 -55.2082120720235\\
481.924438477793 -54.250217209119\\
482.315063477794 -53.2442878459422\\
482.705688477795 -52.2561983582309\\
483.096313477796 -51.344421039986\\
483.486938477797 -50.874598475821\\
483.877563477798 -51.1584273330709\\
484.268188477799 -52.2117742078977\\
484.6588134778 -53.6423287879385\\
485.049438477801 -55.3255920191286\\
485.440063477802 -57.008090635908\\
485.830688477803 -58.6210288875222\\
486.221313477804 -60.1480815200763\\
486.611938477805 -61.2870197551027\\
487.002563477806 -62.616597277202\\
487.393188477807 -63.8939223576649\\
487.783813477808 -65.1324745761456\\
488.174438477809 -66.345683610978\\
488.56506347781 -67.547120615612\\
488.955688477811 -68.7510486438429\\
489.346313477812 -69.9732604421379\\
489.736938477813 -71.2322687135339\\
490.127563477814 -72.5858658379143\\
490.518188477815 -74.0010119183783\\
490.908813477816 -75.5485765607897\\
491.299438477817 -77.2868174532931\\
491.690063477818 -79.2957399341971\\
492.080688477819 -81.8683472282551\\
492.47131347782 -84.6351852330009\\
492.861938477821 -86.5885968145214\\
493.252563477822 -85.2789962163177\\
493.643188477823 -82.060360703984\\
494.033813477824 -79.6096677990371\\
494.424438477825 -76.0740962884984\\
494.815063477826 -74.2503371919168\\
495.205688477827 -72.5426610729544\\
495.596313477828 -70.814964312724\\
495.986938477829 -70.9853815668203\\
496.37756347783 -69.2237297117882\\
496.768188477831 -69.6987196253334\\
497.158813477832 -70.3009130384147\\
497.549438477833 -70.8687105147534\\
497.940063477834 -71.5612031822564\\
498.330688477835 -71.9020613721735\\
498.721313477836 -72.5220891719219\\
499.111938477837 -73.0735890087122\\
499.502563477838 -73.5597245028472\\
499.893188477839 -73.9848974101769\\
};

%\addlegendentry{LRM (estimation) : interpolated};

\addplot [LRMgrid,smallmarkers,interpol]
table[row sep=crcr]{
1000 42\\ % placeholder
1001 42\\
};

\addlegendentry{LRM (estimation)};

\draw (axis cs:280,0) rectangle (axis cs:350,-35);


\end{axis}
\end{tikzpicture}%

    \TODOfig{split plots}
 \caption{Measured (dual-Youla) $\Delta$ for the \gls{AVIS} using the \gls{LRM} and the proposed interpolation. The data in the rectangle is also shown in \figref{fig:avisMeasZoom}.}
\label{fig:avisMeas}
\end{figure}

\begin{figure}
 \centering
    \setlength{\figurewidth}{0.75\columnwidth}
    \setlength{\figureheight}{0.68\figurewidth}
   % This file was created by matlab2tikz v0.4.3 (commit 5727fe79e69f5e2b601173753f6a6749e42dcb5d).
% Copyright (c) 2008--2013, Nico Schlömer <nico.schloemer@gmail.com>
% All rights reserved.
% 
% The latest updates can be retrieved from
%   http://www.mathworks.com/matlabcentral/fileexchange/22022-matlab2tikz
% where you can also make suggestions and rate matlab2tikz.
% 
%
\begin{tikzpicture}[spy using outlines={circle, magnification=5,connect spies,anchor=center,line width=0.5pt}]

\begin{axis}[%
width=\figurewidth,
height=\figureheight,
scale only axis,
%xmode=log,
xtick={296,310,328},
ytick={-2.5,-10,-14,-30},
xmin=280.166625977278,
xmax=348.770141602453,
xminorticks=true,
xlabel={Frequency $\omega$ \axisunit{Hz}},
xmajorgrids,
xminorgrids,
ymin=-35,
ymax=0,
ylabel={Amplitude $\abs{\Delta}$ \axisunit{dB}},
ymajorgrids,
% zmin=-1,
% zmax=1,
% zmajorgrids,
% axis x line*=bottom,
% axis y line*=left,
% axis z line*=left,
%legend style={at={(0.05,0.98)},anchor=north west,draw=black,fill=white,legend cell align=left}
]

% \addplot [SA]
% table[]{\thisDir/data/avis4/SA.tsv};
% \addlegendentry{ETFE}

% \addplot [LPMgrid]
% table[]{\thisDir/data/avis4/LPM-frf.tsv};
% %\addlegendentry{LPM}
% \label{leg:avis4-LPM}

\addplot [LRMgrid]
table[]{\thisDir/data/avis4/LRM-frf.tsv};
%\addlegendentry{LRM}
\label{leg:avis4-LRM}

\addplot [LRM,interpol]
table[]{\thisDir/data/avis4/LRM-intergrid.tsv};
%\addlegendentry{LRM (int)}

\addplot [LRMgrid,interpol]
table[]{\thisDir/data/avis4/LRM-intergrid-placeholder.tsv};
%\addlegendentry{LRM (estimation)}
\label{leg:avis4-LRMi}

\addplot [validation]
table[]{\thisDir/data/avis4/LRM-validation.tsv};
%\addlegendentry{LRM (validation)}
\label{leg:avis4-validation}

\addplot [LRM,hinfnorm]
table[]{\thisDir/data/avis4/LRM-hinfnorm.tsv};
\label{leg:avis4-LRM-hinf}

\addplot [color=validation,hinfnorm]
table[]{\thisDir/data/avis4/LRM-validation-hinfnorm.tsv};
\label{leg:avis4-valid-hinf}

% \addplot [LPM,hinfnorm]
% table[]{\thisDir/data/avis4/LPM-hinfnorm.tsv};
% \label{leg:avis4-LPM-hinf}

%\addlegendentry{LRM (estimation)}

% \addplot3 [
% color=white,
% line width=2.0pt,
% mark size=3.5pt,
% only marks,
% mark=square*,
% mark options={solid,fill=black,draw=mycolor2}]
% table[] {\thisDir/data/avis4/peak.tsv};

% \coordinate (peak1) at (axis cs:296.157836914819, -15.3008939623463);
% \coordinate (mpeak1) at (axis cs:310,-20);

\coordinate (peak2) at (axis cs:327.95715332115, -3.1597019476115);
\coordinate (mpeak2) at (axis cs:310,-10);

\end{axis}

%\spy [black, size=2cm] on (peak1) in node[fill=white] at (mpeak1);
%\spy [black, size=2cm] on (peak2) in node[fill=white] at (mpeak2);
%NOTE: spy seems to be incompatible with pgfplots layer settings...

\end{tikzpicture}%

 \caption{Measured $\Delta$ for the \gls{AVIS} using the \gls{LRM}~\legref{leg:avis4-LRMi} and the proposed interpolation. Clearly, the \gls{LRM} leads to a higher peak value~\legref{leg:avis4-LRM-hinf}, especially at $328\unit{Hz}$. The validity of the LRM local parametric model is confirmed by the validation measurement at the dense frequency grid~\legref{leg:avis4-validation}, which reveals excellent interpolation properties and a similar peak value~\legref{leg:avis4-valid-hinf}.}
\label{fig:avisMeasZoom}
\end{figure}

\vspace{-2em}
\paragraph*{Interpolation performance}
\label{sec:avis-interpol}
\figref{fig:avisMeas} and \figref{fig:avisMeasZoom} show that the estimation and validation data for \gls{LRM} are generally in agreement.
In the frequency range shown in \figref{fig:avisMeasZoom}, the relative difference between both is at most $2\%$.

\vspace{-2em}
\paragraph*{Peak value estimates}
\label{sec:avis-peak-value}
Near $296\unit{Hz}$, a modest improvement of the peak estimate from $-15.3 \unit{dB}$ (at-grid) to $-14.0 \unit{dB}$ is obtained by the interpolation.
The estimate near $\omega_{\star} \approx 328\unit{Hz}$ is improved more substantially.
The proposed method \eqref{eq:gammaInterpolOptim} yields $\gamma_{\mathrm{IG}}=-2.5\unit{dB}$.
$\gamma_{\mathrm{FRF}}\approx -9.9\unit{dB}$ for the estimation data, which means that a $7.4\unit{dB}$ improvement is achieved by the proposed method.
This peak agrees well with the validation data, where the nearest (at-grid) value has an amplitude of $-3.2\unit{dB}$ (i.e. $8\%$ difference).

This suggests that the local rational models with interpolation are a very good approximation to the actual $\Delta$ and also to estimate $\hinfnorm{\Delta}$ even if the actual peak does not coincide with the discrete frequency grid.

\section{Conclusion and Further Research}
\label{sec:conclusion}
A reliable \Hinf{} norm estimation technique for high performance, non-conservative robust control design is investigated.
In this paper, a new approach is presented that employs so-called local parametric \gls{LRM} models that lead to an enhanced estimate of $\norm{\Delta}$ of a \gls{SISO} error system by interpolating between neighboring local models.
It is also shown that the \gls{LPM} is not a worthwhile alternative to the \gls{LRM} to estimate $\norm{\Delta}$.
The technique is illustrated on a simulation example where the \gls{LRM}-based interpolation yields a reliable estimate of the $\Hinf{}$ norm.
It is observed that the measurement time is reduced by almost an order of magnitude compared to techniques that only examine $\Delta(\omega)$ on the \gls{DFT} frequency grid.
Using measurements on an \gls{AVIS}, it is illustrated that the interpolation-based results can substantially improve $\hinfnorm{\Delta}$ and/or reduce the measurement time four-fold.
Furthermore, the results are validated by a detailed validation measurement.
Ongoing research focuses on extending the technique to \gls{MIMO} systems and the study of its stochastic properties.
