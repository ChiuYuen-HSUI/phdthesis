\chapter{Conclusions}
\myEpigraph{Life can only be understood backwards; but it must be lived forwards.}{Søren Kierkegaard}{}
\tikzsetfigurename{ch06fig}

\section{Conclusions of this research}

  \subsection{Excitation design}
  In this thesis, multisine signals are proposed to measure \gls{SISO} systems within an \gls{LTI} framework.
  Particularly, the proposed multisine signals employ a quasi-logarithmic frequency grid and a power spectrum in $1/f$ akin to pink noise to ensure a good quality of the \gls{FRF} over the whole excited frequency band.
  As such, they are more robust excitation signal than e.g. typical general-use signals such as linear grid multisines or white noise excitation with the same total signal power and duration.
  In this thesis it has been proven that when the user has prior knowledge about the expected minimal damping $\damping[\min]$ of the poles present in the system, this sets the required logarithmic frequency grid spacing to attain a guaranteed relative uncertainty of the identified \gls{FRF} near the resonance peaks.
  Concretely, the grid spacing $\alpha \isdef \frac{f_{k+1}}{f_k}$ of the (quasi-)logarithmic grid should conform to
  \begin{equation}
    \alpha \leq 1 + \damping[\min]
  \end{equation}
  to attain a reliable estimate of the resonances present in the system.
  The choice to use the equality $\alpha = 1+ \damping[\min]$ is the optimal choice within the framework presented in \chapref{sec:excitation}. 
  Decreasing $ \alpha$ further comes at the cost of more excited frequency bins and consequently a lower power per bin, or, a reduced \gls{SNR} per bin if the overall signal power is limited.

  In practical terms of a layman user, this means that the design of a quasi-logarithmic multisines discussed in this thesis, requires only limited prior knowledge from the user:
  \begin{itemize}
    \item the allowable measurement time (or equivalently, the minimal relevant frequency),
    \item the allowable total excitation power,
    \item a broad frequency band of interest that could span multiple decades, and,\
    \item the minimal expected relative damping $\damping[\min]$ in the system,
    \item the assumption that the resonances are well-separated in the frequency domain.
  \end{itemize}

  \subsection{Non-parametric estimation}
  \subsection{Initialization of parametric estimation}


\section{Future perspectives}
  \subsection{Excitation design}
  In this thesis, quasi-logarithmic multisines were proposed as a robust excitation signal that provide efficient estimation of sharp resonance peaks in the \gls{SISO} case.
  Obviously, a first important extension of this research is towards the \gls{MIMO} case which is more involved.
  To measure \gls{MIMO} systems using multisines, one of two typical approaches is often followed when measuring all channels at the same time~\citep[Section 2.7]{Pintelon2012}:
  \begin{itemize}
    \item Zippered multisines, where the frequency grids of the different input channels are interspersed. This has the downside that frequency resolution is traded in for computational simplicity and that the different entries in the \gls{FRF} matrix are known on different frequency grids.
    \item Orthogonal multisines, which are computationally more involved to process, require multiple experiments but allow for finer frequency resolution.
  \end{itemize}
  Due to the intrinsic sparsity of quasi-logarithmic grids, however, zippered quasi-logarithmic multisines might offer an easy and elegant  approach to measure \gls{MIMO} systems with a moderate number of inputs.
  %TODO: for ny x nu MIMO system, alpha_MIMO = 1 + alpha_SISO/nu to satisfy the density of frequency lines. This means: lowest guaranteed observable resonance is at higher frequencies than for SISO (obviously).

  Also, in the related chapter, it has been assumed that the system is an \gls{LTI} system.
  For weakly nonlinear systems, the proposed approach is likely to yield approximately valid results.

  \subsection{Non-parametric estimators}
    \TODO{Order selection / BW selection}
    \TODO{MIMO LRM}
    The \gls{LRM} has only been explored for \gls{SISO} systems in this thesis.
    This technique can obviously be extended towards \gls{MIMO} systems, e.g. as the work by \citet{vanRietschoten2015MSc}.
    As explored there, the choice of parametrization remains an open question for the \gls{MIMO} case.
    Also, unlike the \gls{SISO} case, it becomes more important and tedious to select good local models since the dimensionality of the local estimation problem grows with the system dimensions.

    As the non-parametric methods such as \gls{LPM} and \gls{LRM} were originally intended to serve the same purpose as windows in classical spectral analysis, this naturally leads to other applications of these methods where windows are otherwise used to reduce the effect of spectral leakage.

    In particular, for slowly time-varying systems, windows can be used in combination with the \gls{STFT} to obtain a rough approximation of either the output spectrum or the transfer characteristic of time-varying systems.
    In this context, the \gls{LPM} has already been used~\citep{Lataire2012}.
    Preliminary tests using the \gls{LRM} in lieu of the \gls{LPM} have proven fruitful.
    Similarly, for \gls{LPV} systems, the \gls{LRM} allows to extract non-parametric models from \gls{LPV} systems where the scheduling parameter is fixed during the experiment~\citep[Chapter 4]{vanderMaas2016PhD}.
    Given these successes of \gls{LRM} to estimate a preliminary model of both parameter-varying and time-varying models in more complicated experimental settings.
    E.g. it could prove useful to extend the \gls{LRM} also towards measurements of \gls{LPV} systems where the scheduling parameter is allowed to vary continuously during the experiment.

  \subsection{Non-parametric estimation and interpolation}
  \TODO{Link between local models and weighting function design}

  \subsection{Parametric initialization}
  In \chapref{sec:initvals} it was already illustrated that smoother can provide improved initial estimates of a system. 
  Intrinsically, this stems from the fact that the effective \gls{SNR} of the transfer function is improved by means of the smoothness assumption and, on the other hand by the reduction in the the dimensionality in the parameter space.
  By no means were the  smoothers studied in \chapref{sec:initvals} chosen in an optimal way, such that it remains an open topic to determine which smoother has the best chance to produce good initial values in practical settings.

  On the other hand, the \gls{LRM} and its variants discussed in \chapref{sec:nonparametric} provide some local information regarding the shape of the \gls{FRF} and hence also the transfer function.
  The usefulness has already been illustrated in \chapref{sec:hinf} in connection to the computation of the \Hinf norm of a system.
  Intrinsically, these local rational methods approximate the transfer function by a local model, or, equivalently a set of local poles and zeros.
  Especially for resonant poles, one can expect that these local poles are closely linked to the global (physical) poles of a system.
  As such, those could provide a valuable initialization strategy if the redundant local pole/zero information of neighboring windows can be reduced to a single global pole/zero initial value.
  Techniques in model order reduction are likely to provide a good starting point to simplify the local pole/zero estimates to a global parametric model which could then be used to serve as initial estimate for the non-convex optimization steps.

As noted in \remref{rem:initvals:orders:RFIR}, the selection of the model order used in \gls{RFIR} is still an open topic.
There are a few approaches that we think are worthwhile to investigate.
On the one hand, the truncation methods explained in \secref{sec:nparam:trunc} could lead to a initial estimate of the impulse response length and hence the required model complexity for \gls{RFIR}.
On the other hand, one could devise crude model order selection procedures, e.g.  estimating \gls{RFIR} models with increasing complexity until increasing the model order offers no improvement in the non-parametric model.
Alternatively, more flexible model structures (e.g. \gls{RARX}) could be tried.
This doesn't do away with the model order selection of the initial model, but it will improve the computational efficiency since less parameters are needed to approximate a particular system.

  
