\chapter{Conclusions}
\myEpigraph{Life can only be understood backwards; but it must be lived forwards.}{Søren Kierkegaard}{}
\tikzsetfigurename{ch06fig}

In this thesis, user-friendly system identification methods have been developed.
These methods require little to no user interaction and are designed such that the method work for a broad range of systems.
As such, many of the obtained improvements can be obtained without bothering the user for further information and tuning.
Throughout this dissertation, so-called `Guidelines' have been scattered that offer a bird's eye view of the work from a practical point of view.
In the following sections, the main results of this work are summarized and perspectives are sketched for further (and ongoing) research paths.

\section{Results of this Research}

  \subsection{Excitation Design}
  In this thesis, multisine signals are proposed to measure \gls{SISO} systems within a \gls{LTI} framework.
  Particularly, the proposed multisine signals employ a quasi-logarithmic frequency grid and a power spectrum in $1/f$ akin to pink noise to ensure a good quality of the \gls{FRF} over the whole excited frequency band.
  As such, for a constant total signal power and duration, they are a more robust excitation signal than e.g. typical general-use signals such as linear grid multisines or white noise excitation.
  In this thesis it has been proven that when the user knows the minimal damping $\damping[\min]$ of the poles of the system, this sets the required logarithmic frequency grid spacing to attain a guaranteed (relative) uncertainty of the identified \gls{FRF} near the resonance peaks.
  Concretely, the grid spacing $\alpha \isdef \frac{f_{k+1}}{f_k}$ of the (quasi-)logarithmic grid with frequency resolution $\Delta f$ should conform to
  \begin{equation}
    \alpha \leq 1 + \damping[\min]
  \end{equation}
  to attain a reliable estimate of the resonances present in the system.
  The choice to use the equality $\alpha = 1+ \damping[\min]$ is the optimal choice within the framework presented in \chapref{sec:excitation} such that resonances with $\damping \geq \damping[\min]$ and $\wn \geq \omegaLinLog = \frac{2 \pi \Delta f}{\alpha-1} = \frac{2 \pi \Delta f}{\damping[\min]}$ are well-represented in the \gls{FRF}.
  Decreasing $ \alpha$ further comes at the cost of more excited frequency bins and consequently a lower power per bin, or, a reduced \gls{SNR} per bin if the overall signal power is limited.

  In practical terms of a layman user, this means that the design of a quasi-logarithmic multisines--- as discussed in this thesis--- requires only limited prior knowledge:
  \begin{itemize}
    \item the available measurement time (or equivalently, the minimal relevant frequency $\omegaLinLog$),
    \item the allowable total excitation power,
    \item a broad frequency band of interest that could span multiple decades, and,\
    \item the minimal expected relative damping $\damping[\min]$ in the system,
    \item the assumption that the resonances are well-separated in the frequency domain.
  \end{itemize}

  \subsection{Non-parametric Estimation}
  
  \paragraph{Leakage reduction}
  Different non-parametric estimators, specifically from the family of local estimators, i. e. \gls{LRIC}, \gls{LRM} and \gls{LPM} are compared in this work.
  It has been shown that the most general \gls{LRIC} is very sensitive to the use of the `correct' local model order.
  The pre-existing \gls{LRM} and \gls{LPM}, however, are not as sensitive.
  It has been shown that the inherent bias of the \gls{LRM} becomes the dominant source of errors, in contrast to the variance, when an \gls{SNR} of less than $20\unit{dB}$ is available in the frequency bands of interest.
  For resonant systems, the \gls{LRM} is superior to the \gls{LPM} in terms of both bias and \gls{RMS} error in almost all cases.
  The \gls{LPM} is slightly superior for very noisy datasets ($\SNR < 20 \unit{dB}$) that have a very limited frequency resolution.

  \paragraph{Smoothing (noise reduction)}
  In addition, an approach has been developed that allows to reduce the variance of the estimated \gls{FRF} when it is captured in a long data record.
  This method works by examining the estimated impulse response, i.e. the inverse \gls{DFT} of the \gls{FRF}, and estimating a truncation time $\truncTime$ beyond which the impulse response measurement is dominated by the noise.
  By truncating past this point, and transforming the result back into the frequency domain, an \gls{FRF} is obtained that has a reduced \gls{RMS} error by reducing the variance due to the noise.
  Two complementary approaches to determine $\truncTime$ have been developed.
  Both methods estimate the noise level on the impulse response by assuming the last $10\%$ of the samples are only noise.
  The first method exploits the fact that many physical systems exhibit an exponential decay.
  This exponential decay is estimated roughly, such that the truncation point is determined as the instant where decay is below the noise level.
  The second method divides the impulse response into segments.
  Starting from the last segment, it is tested statistically whether the preceding segment is significantly different from the noise, such that $\truncTime$ is estimated.
  For physical systems that exhibit an exponentially decaying impulse response, the exponential fitting method is to be preferred.

  \paragraph{Peak value estimation}
  It has also been illustrated that the use of such local models allows to extract the resonance peak and thereby the $\Hinf$ gain of a system.
  In that setting, the \gls{LPM} is not reliable due to its limited model flexibility.
  The \gls{LRM}, on the other hand, has been shown to yield estimates of the $\Hinf$ gain that improved greatly compared to pure \gls{FRF} measurements.
  For systems that consist of complex dynamic behavior, this allows one to extract reasonable peak values without the hassle of building a global parametric model.
  The use of such local models, in practice, has been validated and can provide good estimates using a dataset that is four times smaller than what would be required for other \gls{FRF}-based methods.
  As such, the measurement time (and hence cost) can also be reduced by the same amount.


  \subsection{Initialization of Parametric Estimation}
  To obtain a parametric model estimate from measured data often requires solving a non-convex optimization problem.
  However, for such optimization problems, often an initial parameter estimate is required to allow convergence to the global optimum, or a local optimum nearby.
  In this thesis, it was investigated whether convergence to a good optimum can be attained more easily by means of \gls{FRF} smoothers; in particular when many data points are only a poor \gls{SNR} is attained.
  A few practical smoothers have been studied, i.e. the time-truncated \gls{LPM}, and the \gls{RFIR}.
  It has been shown that especially the \gls{RFIR} provides smoothed \glspl{FRF} that aid in convergence to a good optimum than the existing state-of-the-art initializers (\gls{BTLS} and \gls{GTLS}).
  These smoothers allow one to obtain good models from noisier data than what existing techniques.
  The improvement of the initial values has been verified both on simulations, measurements, and a simulation stress test consisting of different filters.
  While overall the \gls{RFIR} provides the best starting values from the different smoothers, combining multiple initial values remains a very valuable approach to make a system identification toolbox robust for the measurement data.

\section{Future Perspectives}
  \subsection{Excitation Design}
  In this thesis, quasi-logarithmic multisines were proposed as a robust excitation signal that provide efficient estimation of sharp resonance peaks in the \gls{SISO} case.
  Obviously, a first important extension of this research is towards the \gls{MIMO} case which is more involved.
  To measure \gls{MIMO} systems using multisines, one of two typical approaches is often followed when measuring all channels at the same time~\citep[Section 2.7]{Pintelon2012}:
  \begin{itemize}
    \item Zippered multisines, where the frequency grids of the different input channels are interspersed. This has the downside that frequency resolution is traded in for computational simplicity and that the different entries in the \gls{FRF} matrix are known on different frequency grids.
    \item Orthogonal multisines, which are computationally more involved to process, require multiple experiments but allow for finer frequency resolution.
  \end{itemize}
  Due to the intrinsic sparsity of quasi-logarithmic grids, however, zippered quasi-logarithmic multisines might offer an easy and elegant  approach to measure \gls{MIMO} systems with a moderate number of inputs.
  The use of \gls{LRM}/\gls{LPM} may help to overcome that the frequency grids for the different input/output channels do not coincide for zippered signals.

  Also, in the related chapter, it has been assumed that the system is an \gls{LTI} system.
  For weakly nonlinear systems, the proposed approach is likely to yield approximately valid results.

  \subsection{Non-parametric Estimators}
    The \gls{LRM} has only been explored for \gls{SISO} systems in this thesis.
    This technique can obviously be extended towards \gls{MIMO} systems, e.g. as the work by \citet{vanRietschoten2015MSc} has shown.
    As explored there, the choice of parametrization remains an open question for the \gls{MIMO} case.
    Also, unlike the \gls{SISO} case, it becomes more important and tedious to select good local models since the dimensionality of the local estimation problem grows with the system dimensions.
    Extension to the \gls{MIMO} case, moreover, may directly lead to an approach that is able to deal with concatenated data records similarly to what the \gls{LPM} is able to provide.

    As has been seen in \chapref{sec:nonparametric}, the \gls{LRIC} is affected by a high variance.
    This is due to pole-zero cancellations that arise when the local bandwidth does not contain data that is informative enough to estimate models of the required complexity.
    As such, future research needs to be undertaken to overcome such hurdles, by means of model order selection or reduction of those local models.
    If these hurdles can be overcome, the bias of local modeling can be reduced significantly.
    This is important for short datasets with very noisy data.

    Specifically for the \gls{LRM}, currently a model structure that is equivalent to rational forms in continuous time, i.e. the (local) transfer function model is isomorphic with
    \begin{equation}
      \LocalModel[\CT]{G}(\dW) 
      = \frac{\LocalModel{B}(\dW)}{\LocalModel{A}(\dW)}
      = \frac{\Sum_i  b_i \dW^{i}}{\Sum_i  a_i \dW^{i}}
    \end{equation}
    There have already been some experiments with other model structures such as local discrete time models where
    \begin{equation}
      \LocalModel[\DT]{G}(\dW) 
      = \frac{\LocalModel{B}(\dW)}{\LocalModel{A}(\dW)}
      = \frac{\Sum_i  b_i \exp (ji\dW\Ts)}{\Sum_i  a_i \exp (ji\dW\Ts) }
    \end{equation}
    but these have proven more difficult to estimate numerically and hence unreliable~\citep[Section 2.4]{Pintelon2006BJ1}.
    A possible approach for this would be to use orthogonal polynomials~\citep{Forsythe1957,Bultheel2005} with complex coefficients.

    As the non-parametric methods such as \gls{LPM} and \gls{LRM} were originally intended to serve the same purpose as windows in classical spectral analysis, this naturally leads to other applications of these methods where windows are otherwise used to reduce the effect of spectral leakage.

    In particular, for slowly time-varying systems, windows can be used in combination with the \gls{STFT} to obtain a rough approximation of either the output spectrum or the transfer characteristic of (slowly) time-varying systems.
    In this context, the \gls{LPM} has already been used~\citep{Lataire2012}.
    Preliminary tests using the \gls{LRM} in lieu of the \gls{LPM} have proven fruitful.
    Similarly, for \gls{LPV} systems, the \gls{LRM} allows to extract non-parametric models from \gls{LPV} systems where the scheduling parameter is fixed during the experiment~\citep{vanderMaas2015CDC}.
    Given the successful application of \gls{LRM} in a frozen parameter-varying system, it seems worthwhile to investigate if the\gls{LRM} can be extended to work in more challenging experimental settings.
    E.g. it could prove useful to extend the \gls{LRM} towards measurements of \gls{LPV} systems where the scheduling parameter is allowed to vary continuously during the experiment.

    Another application where the \gls{LPM} is developed further, is the handling of missing data~\citep{Ugryumova2015}.
    In particular, the frequency domain relationship between the input $U$ and output $Y$ is augmented by a term $M_Y$ that accounts for the missing output data samples:
    \begin{equation}
      Y(\omega_k) = G(\omega_k) U(\omega_k) + T(\omega_k) + M_Y(\omega_k) + \mathrm{noise}\text{,}
    \end{equation}
    since $G(\omega_k)$ is captured by local polynomials, this allows to construct a design matrix as discussed in \chapref{sec:nonparametric} which only depends on $U$ and the time instances where the data $y(t)$ is missing.
    For the \gls{LRM}, on the other hand, the design matrix depends on the output $Y$ which is not known.
    Consequently, extending the \gls{LRM} to handle missing data is non-trivial and could be an interesting topic for further research.

    Finally, a comparison between different approaches for non-parametric modeling~\citep{Hagg2012,Stenman2001ASFRF,Gevers2011lpm} could be made.
    This could be done from a theoretical point of view to find correspondences between different approaches (e.g. both \gls{TRIMM}~\citep{Hagg2012} and \gls{LRM} are generalizations of the \gls{ETFE} for specific settings).
    A comparison from a qualitative point of view would be of more practical use, as this would allow to find the best operating regions of each approach.
    Eventually, this could even lead to hybrid methods that either select the best method for a specific setting or use results from different methods to overcome weaknesses of individual methods such as high computational complexity, high bias, high variance, \ldots
    When regarding the \gls{LRM} as a specific form of local regression~\citep{Loader1999}, this allows to reuse the ideas from that community.
    In particular, local regression often uses a so-called kernel that weights the least-squares fit (where the \gls{LRM} implicitly uses a uniform kernel).
    In view of \gls{FRF} estimation, a worthwhile endeavor could be to study the influence of these kernels on both the \gls{FRF} estimate and the estimated noise level.
    Moreover, this could even lead to the design of new kernels.

  \subsection{Non-parametric Estimation and Interpolation}
  As with the regular \gls{LRM}, ongoing research focusses on the estimation approach of the $\Hinf{}$ gain towards the \gls{MIMO} case.
  In that context, the design of weighting filter also becomes even more important~\citep{Boeren2013}.
  Hence, the link between local models and the design of weighting filters and parametric overbounds is essential and should be confirmed further in a detailed study.

  Another possibility for future research is to approach the problem of $\Hinf$ gain estimation purely from the aspect of local regression such that combining multiple local models is no longer necessary.

  \subsection{Construction of Initial Values}
  In \chapref{sec:initvals} it was already illustrated that smoothers can provide improved initial estimates of a system. 
  Intrinsically, this stems from the fact that the effective \gls{SNR} of the transfer function is improved by means of the smoothness assumption, and on the other hand by the reduction in the dimensionality in the parameter space.
  By no means were the  smoothers studied in \chapref{sec:initvals} chosen in an optimal way, such that it remains an open topic to determine which smoother has the best chance to produce good initial values in practical settings.

  On the other hand, the \gls{LRM} and its variants discussed in \chapref{sec:nonparametric} provide some local information regarding the shape of the \gls{FRF} and hence also the transfer function.
  The usefulness has already been illustrated in \chapref{sec:hinf} in connection to the computation of the \Hinf norm of a system.
  Intrinsically, these local rational methods approximate the transfer function by a local model, or, equivalently a set of local poles and zeros.
  Especially for resonant poles, one can expect that these local poles are closely linked to the global (physical) poles of a system.
  As such, those could provide a valuable initialization strategy if the redundant local pole/zero information of neighboring windows can be reduced to a single global pole/zero initial value.
  Techniques in model order reduction are likely to provide a good starting point to simplify the local pole/zero estimates to a global parametric model which could then be used to serve as initial estimate for the non-convex optimization steps.

As noted in \remref{rem:initvals:orders:RFIR}, the selection of the model order used in \gls{RFIR} is still an open topic.
There are a few approaches that are worthwhile to investigate.
On the one hand, the truncation methods explained in \secref{sec:nonparametric:truncation} could lead to an initial estimate of the impulse response length and hence the required model complexity for \gls{RFIR}.
On the other hand, one could devise crude model order selection procedures, e.g.  estimating \gls{RFIR} models with increasing complexity until increasing the model order offers no improvement in the non-parametric model.
Alternatively, more flexible model structures (e.g. \gls{RARX}) could be tried.
This does not do away with the model order selection of the initial model, but it will improve the computational efficiency since less parameters are needed to approximate a particular system.

  
