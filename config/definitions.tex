\newcommand{\photobox}[1]{\frame{#1}}

\newcommand{\TOCInsert}[2]{\addcontentsline{toc}{#1}{#2}}
\newcommand{\chapterNoNum}[1]{\chapter*{#1} \TOCInsert{chapter}{#1}}
\newcommand{\sectionNoNum}[1]{\section*{#1} \TOCInsert{section}{#1}}

\newcommand{\disclaimer}[1]{
    \begin{centering}
    \small\textsf{#1}
    \end{centering}
}

%http://tex.stackexchange.com/questions/31186/how-to-move-a-paragraph-to-the-bottom-of-the-page-without-vspace
\newenvironment{bottompar}{\par\vspace*{\fill}}{\clearpage}

\newcommand{\researchBasedOn}[2][This chapter]{\disclaimer{#1 is based on~\citet{#2}.}}

%% Referencing
\newcommand{\algoref}[1]{Algorithm~\ref{#1}}
\newcommand{\appref}[1]{Appendix~\ref{#1}}
\newcommand{\assref}[1]{Assumption~\ref{#1}}
\newcommand{\chapref}[1]{Chapter~\ref{#1}}
\newcommand{\defref}[1]{Definition~\ref{#1}}
\newcommand{\remref}[1]{Remark~\ref{#1}}
\newcommand{\egref}[1]{Example~\ref{#1}}
\newcommand{\figref}[1]{Figure~\ref{#1}}
\newcommand{\guideref}[1]{Guideline~\ref{#1}}
\newcommand{\legref}[2][1]{(\ref{#2})}
\newcommand{\lstref}[1]{Listing~\ref{#1}}
\newcommand{\propref}[1]{Property~\ref{#1}}
\newcommand{\secref}[1]{Section~\ref{#1}}
\newcommand{\tabref}[1]{Table~\ref{#1}}
\newcommand{\thmref}[1]{Theorem~\ref{#1}}

%% Common names
\newcommand{\MATLAB}{\textsc{Matlab}\textregistered\xspace}
\newcommand{\Simulink}{\mbox{Simulink\textregistered}\xspace}
\newcommand{\labview}{\mbox{Lab\textsc{View}\textregistered}\xspace}
\newcommand{\FDIDENT}{Frequency Domain System Identification Toolbox}
\newcommand{\BK}{Br\"{u}el \& Kj\ae{}r}
\newcommand{\bruelkjaer}{\BK{}}
\newcommand{\HP}{Hewlett~Packard\xspace}
\newcommand{\YK}{Youla-Ku\v{c}era}
\newcommand{\IEEEfloat}{\textsc{IEEE~754}\xspace}

%% Formatting
\newcommand{\mathbold}[1]{\mathbf{#1}} % used instead of fixmath (can only be used with CM math font!!!)
\newcommand{\latin}[1]{\emph{#1}}
\newcommand{\term}[1]{\emph{#1}}
\newcommand{\deemph}[1]{{\color{gray}{#1}}}
\newcommand{\FIXME}[1]{{\color{green} #1}}
\newcommand{\TODO}[1]{{\color{red} #1}}
\newcommand{\TODOfig}[1][figure]{\parbox[h][\figureheight][c]{\figurewidth}{\centering \vfill \TODO{#1} \vfill}}
\newcommand{\TODOtikzfig}[1][TikZ figure]{\begin{tikzpicture}\node[rectangle, draw=red,minimum width=\figurewidth,minimum height=\figureheight] {\TODO{#1}};\end{tikzpicture}}

\newenvironment{personalNote}{\color{blue}}{}

%% Unit formatting
\newcommand{\unit}[1]{\ensuremath{\,\mathrm{#1}}}
\newcommand{\axisunit}[1]{\ensuremath{\,\left[\mathrm{#1}\right]}}
\newcommand{\noaxisunit}{ }

%% Statistical operators
\DeclareMathOperator{\Probability}{P}
\DeclareMathOperator{\Expectancy}{E}
\DeclareMathOperator{\Bias}{Bias}
\DeclareMathOperator{\MSE}{MSE}
\DeclareMathOperator{\RootMeanSquare}{RMS}
\DeclareMathOperator{\lcm}{lcm}

\DeclareMathOperator{\RealPart}{\mathrm{Re}}
\DeclareMathOperator{\ImagPart}{\mathrm{Im}}
\DeclareMathOperator{\RealImagPart}{\stackrel{\mathrm{Re}}{\mathrm{Im}}}
\DeclareMathOperator{\Diagonal}{\mathrm{diag}}
\DeclareMathOperator{\LogLikelihood}{\ell}

\DeclareMathOperator{\DFT}{\mathrm{DFT}}
\DeclareMathOperator{\IDFT}{\mathrm{IDFT}}

\DeclareMathOperator*{\argmin}{argmin}
\DeclareMathOperator*{\Arg}{arg}

\DeclareMathOperator{\ExpectedValue}{\mathrm{E}}
\DeclareMathOperator{\Variance}{\mathrm{Var}}
\DeclareMathOperator{\Covariance}{\mathrm{Cov}}
\DeclareMathOperator{\BiasValue}{\mathrm{bias}}

\newcommand{\Prob}[1]{\ensuremath{\mathrm{P}}\!\left( #1 \right)}
\newcommand{\E}[2][]{\ExpectedValue_{#1} \left[ #2 \right]}
\newcommand{\Var}[2][]{\Variance_{#1} \left[ #2 \right]}
\newcommand{\var}[2][]{\Variance_{#1} \left[ #2 \right]}
\newcommand{\Cov}[3][]{\Covariance_{#1} \left[ #2 , #3 \right]}

\newcommand{\expectedValue}[2][]{\ensuremath{\E[#1]{#2}}}
\newcommand{\bias}[1]{\ensuremath{\Bias \left( #1 \right)}}
\newcommand{\mse}[1]{\ensuremath{\MSE \left( #1 \right)}}
\newcommand{\rms}[1]{\ensuremath{\RootMeanSquare \left( #1 \right)}}

\newcommand{\statisticalHypothesis}[1]{\ensuremath{\mathrm{H}_{\mathrm{#1}}}}
\newcommand{\nullHypothesis}{\statisticalHypothesis{0}}
\newcommand{\altHypothesis}{\statisticalHypothesis{a}}

%% Sampled statistics
\newcommand{\sampleMean}[1]{\ensuremath{\estimated{\mathrm{m}}_{#1}}}
\newcommand{\sampleBias}[1]{\ensuremath{\estimated{\mathrm{b}}_{#1}}}
\newcommand{\sampleVariance}[1]{\ensuremath{\estimated{\sigma}^2_{#1}}}
\newcommand{\sampleError}[1]{\ensuremath{\estimated{\mathrm{e}}^2_{#1}}}
\newcommand{\sampleStd}[1]{\ensuremath{\estimated{\sigma}_{#1}}}

\newcommand{\noiseStd}{\ensuremath{\sampleStd{\mathrm{n}}}}
\newcommand{\crest}[1]{\ensuremath{\mathrm{Cr}\left( #1 \right)}}

%% Counters
\newcommand{\numberOf}{\ensuremath{\#}}
\newcommand{\numel}[1]{\ensuremath{n_{#1}}}
\newcommand{\order}[1]{\ensuremath{N_{#1}}}

\newcommand{\nWind}{\ensuremath{\numel{\Window}}}
\newcommand{\nBW}{\ensuremath{\numel{\BW}}}
\newcommand{\nth}[1][]{\ensuremath{\numel{\theta}}}
\newcommand{\nMC}{\ensuremath{\numel{\mathrm{MC}}}}
\newcommand{\nSegments}{\ensuremath{\numel{\mathrm{S}}}}
\newcommand{\Nbc}{\ensuremath{\order{\mathrm{bc}}}}

%% Calculus operations
\newcommand{\pd}{\partial}
\newcommand{\dd}{\ensuremath{\mathrm{d}}}
\newcommand{\derivativeNotation}[4][]{\frac{{#4}^{#1} #2}{#4 #3}}
\newcommand{\PartialDerivative}[3][]{\derivativeNotation[#1]{#2}{#3}{\pd}}
\newcommand{\FullDerivative}   [3][]{\derivativeNotation[#1]{#2}{#3}{\dd}}
\newcommand{\evaluate}[2]{\ensuremath{\left. #1 \right|_{#2}}}

%% Matrix operations
\newcommand{\I}{\ensuremath{\mathrm{I}}}
\newcommand{\Identity}[1]{\ensuremath{\I_{#1}}}
\newcommand{\Zero}[1]{\ensuremath{\mathrm{0}_{#1}}}

\newcommand{\CostFunc}[1]{\ensuremath{L_{#1}}}
\newcommand{\LSCost}[1]{\ensuremath{L_{\mathrm{LS}}\left( #1 \right)}}
\newcommand{\ELSCost}[2][]{\ensuremath{{E}_{\mathrm{LS#1}}\!\left( #2 \right)}}


\newcommand{\TT}{\ensuremath{\mathrm{T}}}
% http://tex.stackexchange.com/questions/240691/disable-some-xunicode-macros-in-case-of-t1-font-encoding
% redefining seems to work, but it depends on the font used !!!!!!!
\ifcsname C\endcsname
      \renewcommand{\C}{\ensuremath{\mathrm{C}}}
  \else%
       \newcommand{\C}{\ensuremath{\mathrm{C}}}
  \fi
  \ifcsname T\endcsname
     \renewcommand{\T}{\TT}
  \else
     \newcommand{\T}{\TT}
  \fi

\newcommand{\HT}{\ensuremath{\mathrm{H}}}

\newcommand{\inv}[1]{\ensuremath{#1^{-1}}}
\newcommand{\pinv}[1]{\ensuremath{#1^{+}}}
\newcommand{\conj}[1]{\ensuremath{\overline{#1}}}

\newcommand{\hadamard}{\ensuremath{\circ}}
\newcommand{\conv}{\ensuremath{*}}
\newcommand{\kron}{\ensuremath{\otimes}}

\newcommand{\real}[1]{\ensuremath{\RealPart \left[#1\right]}}
\newcommand{\imag}[1]{\ensuremath{\ImagPart \left[#1\right]}}
\newcommand{\realimag}[1]{\ensuremath{\RealImagPart \left[#1\right]}}
\newcommand{\diag}[1]{\ensuremath{\Diagonal \left({#1}\right)}}
\newcommand{\trace}[1]{\ensuremath{\mathrm{tr}\; #1}}
\newcommand{\ReIm}[1]{\ensuremath{#1_{\RealNumbers}}}
\newcommand{\ReImImRe}[1]{\ensuremath{#1_{\RealNumbers\RealNumbers}}}
\newcommand{\ReImTheta}{\ensuremath{\vartheta}}

%% Linear Algebra operations
\newcommand{\OuterProduct} [3][\TT]{{#2}{#3}^{#1}} %AB^T
\newcommand{\OuterProductC}[2]{\OuterProduct[\HT]{#1}{#2}}
\newcommand{\OuterProductR}[2]{\OuterProduct[\TT ]{#1}{#2}}
\newcommand{\OuterProductSelf}[2][\TT]{\OuterProduct[#1]{#2}{#2}} %AA^T

\newcommand{\InnerProduct} [3][\TT]{{#2}^{#1}{#3}}  % A^T A
\newcommand{\InnerProductC}[2]{\InnerProduct[\HT]{#1}{#2}}
\newcommand{\InnerProductR}[2]{\InnerProduct[\TT]{#1}{#2}}
\newcommand{\InnerProductSelf}[2][\TT]{\InnerProduct[#1]{#2}{#2}} %A^T A

\newcommand{\OuterQuadraticForm}[3][\TT]{#2  #3 #2^{#1}} % A B A^T
\newcommand{\InnerQuadraticForm}[3][\TT]{#2^{#1} #3 #2} % A^T B A
\newcommand{\QuadraticForm}[3][\TT]{\InnerQuadraticForm[#1]{#2}{#3}}

\newcommand{\KTK}[2][\HT]{\ensuremath{ \inv{ \left( #2^{#1} #2 \right)} #2^{#1}}}


%% Numbers sets
\newcommand{\Negative}[1]{\ensuremath{{#1}^{-}}}
\newcommand{\Positive}[1]{\ensuremath{{#1}^{+}}}
\newcommand{\Nonzero }[1]{\ensuremath{{#1}_{0}}}

\newcommand{\NaturalNumbers}{\ensuremath{\mathbb{N}}}
\newcommand{\NaturalNumbersWithoutZero}{\ensuremath{\NaturalNumbers_{0}}}
\newcommand{\IntegerNumbers}{\ensuremath{\mathbb{Z}}}
\newcommand{\RationalNumbers}{\ensuremath{\mathbb{Q}}}
\newcommand{\RealNumbers}{\ensuremath{\mathbb{R}}}
\newcommand{\ComplexNumbers}{\ensuremath{\mathbb{C}}}

\newcommand{\MatrixSet}[2]{\ensuremath{{#1}^{#2}}}
\newcommand{\RealMatrix}[1]{\ensuremath{\MatrixSet{\RealNumbers}{#1}}}
\newcommand{\ComplexMatrix}[1]{\ensuremath{\MatrixSet{\ComplexNumbers}{#1}}}

\newcommand{\EvenNaturalNumbers}{\ensuremath{2\NaturalNumbers}}
\newcommand{\OddNaturalNumbers}{\ensuremath{2\NaturalNumbers + 1}}

\newcommand{\GenericField}{\ensuremath{\mathbb{F}}}
\newcommand{\BoundedField}{\ensuremath{\mathbb{B}}}

\newcommand{\Integers}{\IntegerNumbers}
\newcommand{\PositiveRealNumbers}{\ensuremath{\Positive{\RealNumbers}}}
\newcommand{\NegativeRealNumbers}{\ensuremath{\Negative{\RealNumbers}}}
\newcommand{\RealNumbersWithoutZero}{\ensuremath{\Nonzero{\RealNumbers}}}

\newcommand{\NN}{\NaturalNumbers}
\newcommand{\ZN}{\IntegerNumbers}
\newcommand{\QN}{\RationalNumbers}
\newcommand{\RR}{\RealNumbers}
\newcommand{\CN}{\ComplexNumbers}
\newcommand{\FF}{\GenericField}

\newcommand{\BB}{\BoundedField}
\newcommand{\Bound}{\BB}

%% Special Equality symbols
\newcommand{\isdef}{\ensuremath{\triangleq}}
\newcommand{\qeq}{\ensuremath{\stackrel{?}{=}}}
\newcommand{\distributedAs}{\ensuremath{\sim}}
\newcommand{\distributed}[1][]{\ensuremath{\stackrel{#1}{\sim}}}
\newcommand{\equals}[1][]{\ensuremath{\stackrel{#1}{=}}}
\newcommand{\approxequals}[1][]{\ensuremath{\stackrel{#1}{\approx}}}

%% Unary operators
\newcommand{\round}[1]{ \left\lfloor #1 \right\rceil}
\newcommand{\floor}[1]{ \left\lfloor     #1 \right\rfloor}
\newcommand{\ceil }[1]{ \left\lceil      #1 \right\rceil}
\newcommand{\abs }[1]{ \left|      #1 \right|}
\newcommand{\norm}[2][]{\ensuremath{\left\| #2 \right\|_{#1}}}
\newcommand{\infnorm} [1]{\ensuremath{\norm[\infty]{#1}}}

\newcommand{\svdMax}[1]{\ensuremath{ \overline{\sigma} \left( #1 \right)  }}

%% Distributions
\DeclareMathOperator{\NormalDistribution}{\mathcal{N}}
\DeclareMathOperator{\FDistribution}{\mathcal{F}}
\DeclareMathOperator{\DiracDistribution}{\delta}
\DeclareMathOperator{\CircularNormalDistribution}{\mathcal{C}\!\NormalDistribution}
\DeclareMathOperator{\ComplexCircularNormalDistribution}{\CN\!\CircularNormalDistribution}

\newcommand{\Dirac}[1]{\DiracDistribution\left(#1\right)}
\newcommand{\Normal}[2]{\NormalDistribution\left(#1,#2\right)}
\newcommand{\CircularNormal}[3]{\CircularNormalDistribution\!\left(#1,#2,#3\right)}
\newcommand{\ComplexCircularNormal}[2]{\ComplexCircularNormalDistribution\!\left(#1,#2\right)}

\newcommand{\dirac}[1]{\ensuremath{\delta \left( #1 \right)}}

%% Interval Specifications
\newcommand{\LeftOpenInterval}[1]{\ensuremath{\left] #1 \right]}}
\newcommand{\RightOpenInterval}[1]{\ensuremath{\left[ #1 \right[}}
\newcommand{\OpenInterval}[1]{\ensuremath{\left] #1 \right[}}
\newcommand{\ClosedInterval}[1]{\ensuremath{\left[ #1 \right]}}

%% 
\newcommand{\Hardy}{\ensuremath{\mathcal{H}}}
\newcommand{\Hinf}{\ensuremath{\Hardy_{\infty}}}

\newcommand{\MC}{\ensuremath{\mathrm{MC}}}

% http://tex.stackexchange.com/questions/183955/sum-within-a-fraction
\newcommand{\Sum}{\sum\limits}
\newcommand{\costFunc}[1]{\ensuremath{V_{#1}}}
\newcommand{\validationDistance}[1]{\ensuremath{D_{#1}}}
\newcommand{\model}[2][]{\ensuremath{\hat{G}_{\mathrm{#2}}^{\mathrm{#1}}}}

\newcommand{\code}[1]{\texttt{#1}}

\newcommand{\AICc}{\ensuremath{\mathrm{AICc}}}
\newcommand{\AIC}{\ensuremath{\mathrm{AIC}}}
\newcommand{\CT}{\ensuremath{\mathrm{CT}}}
\newcommand{\DOF}{\ensuremath{\mathrm{DOF}}}
\newcommand{\DT}{\ensuremath{\mathrm{DT}}}
\newcommand{\excited }{\ensuremath{\mathrm{exc}}}
\newcommand{\interest}{\ensuremath{\mathrm{int}}}
\newcommand{\LMM}{\ensuremath{\mathrm{local}}}
\newcommand{\LPM}{\ensuremath{\mathrm{LPM}}}
\newcommand{\LRIC}{\ensuremath{\mathrm{LRIC}}}
\newcommand{\LRM}{\ensuremath{\mathrm{LRM}}}
\newcommand{\PRESS}{\ensuremath{\mathrm{PRESS}}}
\newcommand{\RMSE}{\ensuremath{\mathrm{RMSE}}}
\newcommand{\SNR}{\ensuremath{\mathrm{SNR}}}
\newcommand{\trunc}{\ensuremath{\mathrm{trunc}}}
\newcommand{\ETFE}{\ensuremath{\mathrm{ETFE}}}


% \newcommand{\AvisMatrixDiagonal}[2]{
%     \begin{bmatrix}
%       #1 & \deemph{0} & \deemph{0} & \deemph{0}&  \deemph{0}& \deemph{0}\\
%       \deemph{0} & #1 & \deemph{0} & \deemph{0} & \deemph{0}& \deemph{0} \\
%       \deemph{0} & \deemph{0} & #1 & \deemph{0} & \deemph{0} & \deemph{0}\\
%        \deemph{0}& \deemph{0} &  \deemph{0} & #2 & \deemph{0}& \deemph{0}\\
%        \deemph{0}& \deemph{0} &  \deemph{0} &  \deemph{0} & #2 & \deemph{0} \\
%       \deemph{0} & \deemph{0} & \deemph{0} &\deemph{0}  & \deemph{0} & #2\\
%     \end{bmatrix}
% }
\newcommand{\AvisMatrixDiagonal}[2]{
    \begin{bmatrix}
    #1 & \deemph{0}\\
    \deemph{0} & #2\\
    \end{bmatrix}
    \kron
    \Identity{3}
}
\newcommand{\geomean}[2]{\ensuremath{\sqrt{#1 #2}}}

\newcommand{\FisherInformation}{\ensuremath{\mathrm{Fi}}}
\newcommand{\Fi}[1]{\ensuremath{\FisherInformation \left(#1\right)}}

\newcommand{\CramerRao}{\ensuremath{\mathrm{CR}}}
\newcommand{\CR}[1]{\ensuremath{\CramerRao \left(#1\right)}}

\newcommand{\jac}[1]{\ensuremath{J_{#1}}}

\newcommand{\loglikelihood}[1]{\ensuremath{\LogLikelihood\left( #1 \right) }}
\newcommand{\fc}{\ensuremath{f_{\mathrm{c}}}}
%% Sets
\newcommand{\given}[1]{\left| #1 \right.}
\newcommand{\setdiff}{\ensuremath{\backslash}}
\newcommand{\without}{\setdiff}

\newcommand{\ignoring}[2]{\ensuremath{ {#2}_{[#1]}}} 

\newcommand{\decade}{\ensuremath{\mathrm{dec.}}}

\newcommand{\eps }{\varepsilon}

\newcommand{\wallclocktime}[1][]{\ensuremath{t_{\mathrm{wall-clock}}}}
\newcommand{\cputime}[1][]{\ensuremath{t_{\mathrm{CPU}}}}

%%
\newcommand{\trueSymbol}{\ensuremath{\circ}}
\newcommand{\true}[1]{\ensuremath{{#1}_{\trueSymbol}}}
\newcommand{\starred}[1]{\ensuremath{#1_{\star}}}

%% Common symbols
\newcommand{\BW}{\ensuremath{\mathrm{BW}}}
\newcommand{\wdB}{\ensuremath{\BW_{3\unit{dB}}}}
\newcommand{\dc}[1][]{\ensuremath{K_{#1}}}
\newcommand{\timeconst}[1][]{\ensuremath{\tau_{#1}}}
\newcommand{\damping}[1][]{\ensuremath{\xi_{#1}}}
\newcommand{\wn}[1][]{\ensuremath{\omega_{\mathrm{n}#1}}}
\newcommand{\wc}[1][]{\ensuremath{\omega_{\mathrm{c}#1}}}
\newcommand{\wz}[1][]{\ensuremath{\omega_{\mathrm{z}#1}}}
\newcommand{\pole}[1][]{\ensuremath{{p}_{#1}}}
\newcommand{\multiplicity}[1][]{\ensuremath{\nu_{#1}}}
\newcommand{\wM}[1][]{\ensuremath{\omega_{\mathrm{M}#1}}}
\newcommand{\wR}{\ensuremath{\omega_{\mathrm{R}}}}
\newcommand{\wNom}{\ensuremath{\omega_{\mathrm{nom}}}}
\newcommand{\omegaLinLog}{\ensuremath{\omega_{\ell \ell}}}
\newcommand{\fR}{\ensuremath{f_{\mathrm{R}}}}
\newcommand{\fs}{\ensuremath{f_{\mathrm{s}}}}
\newcommand{\Tm}{\ensuremath{T_{\mathrm{meas}}}}
\newcommand{\Ts}{\ensuremath{T_{\mathrm{s}}}}
\newcommand{\Tw}[1][]{\ensuremath{\wn^{#1}\Ts^{#1}}}
\newcommand{\wInterest}{\ensuremath{\omega_{\mathrm{I}}}}
\newcommand{\snorm}[1][]{\frac{s^{#1}}{\wn^{#1}}}

\newcommand{\ICDF}[1]{\ensuremath{\mathrm{CDF}^{-1}_{#1}}}
\newcommand{\pdf}[2][]{\ensuremath{f_{#1} \left(#2 \right)}}
\newcommand{\placeBetweenGrid}[1]{\ensuremath{\mathrm{misalign} \left( #1 \right) }}
\newcommand{\infgrid}[1]{\ensuremath{\vec{#1}}}
\newcommand{\powerspec}[1]{\ensuremath{\phi_{#1}}}


\newcommand{\truncTime}{\ensuremath{t_{\trunc}}}

\newcommand{\successRate}{\ensuremath{\eta}}
\newcommand{\toleranceSymbol}{\ensuremath{\varepsilon}}
\newcommand{\absoluteTolerance}{\ensuremath{\toleranceSymbol_{\mathrm{A}}}}
\newcommand{\relativeTolerance}{\ensuremath{\toleranceSymbol_{\mathrm{R}}}}

% \newcommand{\fMax}[1]{\ensuremath{f_{#1 \max}}}
% \newcommand{\fMin}[1]{\ensuremath{f_{#1 \min}}}

\newcommand{\fMax}[1]{\ensuremath{\overline{f_{#1}}}}
\newcommand{\fMin}[1]{\ensuremath{\underline{f_{#1}}}}

\newcommand{\kMin}{\ensuremath{k_{\min}}}
\newcommand{\kMax}{\ensuremath{k_{\max}}}

\newcommand{\LandauBigO}{\ensuremath{\mathcal{O}}}
\newcommand{\bigO}[1]{\ensuremath{\LandauBigO \left( #1 \right)}}

%\renewcommand{\fMax}[1]{\ensuremath{f_{#1+1/2}}}
%\renewcommand{\fMin}[1]{\ensuremath{f_{#1-1/2}}}
  
\newcommand{\leqByElement}{\ensuremath{\preceq}}
\newcommand{\geqByElement}{\ensuremath{\succeq}}

\newcommand{\placeholder}{\ensuremath{\bullet}}


\newcommand{\DTmethod}{zero-order-hold}
\newcommand{\ContTime}{\CT}
\newcommand{\DiscTime}{\DT}

%% Formatting
\newcommand{\emspace}{\ensuremath{\qquad}}

\newcommand{\f}{\ensuremath{\mathbb{f}}} % alternative frequency: mho symbol
\newcommand{\idx}[2]{\ensuremath{\left( #1 \right)_{#2}}}
\newcommand{\segm}[2]{\ensuremath{{#1}^{[#2]}}}

\newcommand{\atIter}[2][]{\ensuremath{{#2}^{[#1]}}}
\newcommand{\atSimulation}[1]{\mathnormal{[#1]}}

\newcommand{\exponent}[1]{\ensuremath{e^{#1}}}

\newcommand{\LOGGRID}{\ensuremath{\mathrm{LG}}}
\newcommand{\QLOGGRID}{\ensuremath{\mathrm{QLG}}}
\newcommand{\GRID}{\ensuremath{\mathrm{GRD}}}


\newcommand{\num}[1]{\pgfmathprintnumber{#1}}

\usepackage{pifont}
\newcommand{\circledNum}[1]{\ding{\numexpr201 + #1}}

%% Hinf stuff
\newcommand{\HardySpace}[1][]    {\ensuremath{\mathcal{H}_{#1}}}
\newcommand{\LebesgueSpace}[1][] {\ensuremath{\mathcal{L}_{#1}}}

\newcommand{\RealRational}{\ensuremath{\mathcal{R}}}

\renewcommand{\Hinf}{\texorpdfstring{\HardySpace[\infty]}{Hinf}}
% \renewcommand{\Hinf}{\ensuremath{\mathcal{H}_{\infty}}}
\newcommand{\RHinf}{\RealRational\Hinf}

\newcommand{\worstCase}[1]             {\ensuremath{{#1_{\text{wc}}}}}
\newcommand{\Robust}[1]       {\ensuremath{{#1}_\text{R}}}
\newcommand{\ControlRelevant}[1]       {\ensuremath{{#1}_\text{CR}}}
\newcommand{\RobustControlRelevant}[1] {\ensuremath{{#1}_\text{RCR}}}
\newcommand{\RobustStability}[1]       {\ensuremath{{#1}_\text{RS}}}
\newcommand{\RobustPerformance}[1]     {\ensuremath{{#1}_\text{RP}}}

\newcommand{\ControlObjective} {\ensuremath{\mathcal{J}}}
\newcommand{\estimated} [1]    {\ensuremath{\hat{#1}}}

\newcommand{\tuple} [1] {\ensuremath{ \left( {#1} \right) }}
\newcommand{\mat}   [1] {\ensuremath{ \left[ \begin{matrix} #1 \end{matrix} \right] }}
\newcommand{\poly}  [1] {\ensuremath{\widetilde{#1}}}

\newenvironment{Matrix}{\left[ \begin{matrix}}{\end{matrix} \right]}

\newcommand{\ctr}[1]{\ensuremath{\widetilde{#1}}}

\newcommand{\ezbrace}[2]{\ensuremath{\overbrace{\color{black}#1}^{\color{gray!75} #2}}}

\newcommand{\singularValue}[1]{\ensuremath{\mathbf{\sigma} \left( #1 \right) }}


\newcommand{\lpm}[1]{\ensuremath{\LPM \left( #1 \right) }}
\newcommand{\lrm}[1]{\ensuremath{\LRM \left( #1 \right) }}
\newcommand{\lric}[1]{\ensuremath{\LRIC \left( #1 \right) }}

% \newcommand{\stable}{\ensuremath{\;\mbox{stable}}}
% \newcommand{\stablePart}[1]{S}
% \newcommand{\unstablePart}[1]{}

\newcommand{\stateSpace}[4]{\mat{ \begin{array}{c|c} #1 & #2 \\ \hline #3 & #4 \end{array} }}

\newcommand{\dW}{\ensuremath{\delta}}
\newcommand{\dw}[2]{\ensuremath{\dW_{#1#2}}}


\newcommand{\Crit}{\ensuremath{\mathcal{J}}}
\newcommand{\Controller}{\ensuremath{C}}
\newcommand{\ModelSet}[1][\color{modelset}]{{\ensuremath{#1\mathcal{P}}}}
\newcommand{\ModelUncertainty}{\ensuremath{{\color{delta}{\Delta}}}}
\newcommand{\Plant}{\ensuremath{P}}
\newcommand{\nominalModel}{{\color{plant}{\estimated\Plant}}}
\newcommand{\estimatedPlant}{\nominalModel}
\newcommand{\truePlant}{{\color{plant}{\true\Plant}}}

\newcommand{\experimental}[1]{\ensuremath{#1^{\mathrm{exp}}}}
\newcommand{\robust}[1]{\ensuremath{#1^{\mathrm{R}}}}
\newcommand{\controlrelevant}[1]{\ensuremath{#1^{\mathrm{CR}}}}
\newcommand{\robustcontrolrelevant}[1]{\ensuremath{#1^{\mathrm{RCR}}}}
\newcommand{\ClosedLoop}[1]{\ensuremath{\mathrm{T}\left(#1\right)}}

\newcommand{\disturbance}[1]{\ensuremath{F_{\mathrm{d(#1)}}}}
\newcommand{\externalDisturbance}{\disturbance{ext}}
\newcommand{\systemDisturbance}{\disturbance{pl}}


% \newcommand{\LocalModel}[1]{\ensuremath{\widetilde{\mathbold{#1}}}}


\newcommand{\LocalModel}[2][]{\ensuremath{\mathbold{\widetilde{#2}}_{#1} }}
\newcommand{\leftAdjacent}{\ensuremath{k_{\mathrm{L}}}}
\newcommand{\rightAdjacent}{\ensuremath{k_{\mathrm{R}}}}


\newcommand{\LocalVector}[1]{\ensuremath{\mathbold{#1}}}
\newcommand{\LocalMatrix}[1]{\LocalVector{#1}}
\newcommand{\Window}{\ensuremath{\mathbb{W}}}
\newcommand{\LocalWindow}[1][]{\ensuremath{\Omega_{#1}}}
\newcommand{\LocalShifts}[1]{\ensuremath{\mathcal{R}_{#1}}}


\newcommand{\Frequencies}{\ensuremath{\Omega}}
\newcommand{\DFTGrid}{\ensuremath{\Omega_{\mathrm{FRF}}}}
\newcommand{\InterGrid}{\ensuremath{\Omega}}
\newcommand{\IG}{\ensuremath{\mathrm{IG}}}

\newcommand{\gammaIntergrid}{\ensuremath{\gamma_{\IG}}}
\newcommand{\gammaAtGrid}{\ensuremath{\gamma_{\mathrm{FRF}}}}
