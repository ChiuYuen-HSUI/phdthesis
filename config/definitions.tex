
\newcommand{\TODO}[1]{{\color{red} #1}}
\newcommand{\TOCInsert}[2]{\addcontentsline{toc}{#1}{#2}}
\newcommand{\chapterNoNum}[1]{\chapter*{#1} \TOCInsert{chapter}{#1}}
\newcommand{\sectionNoNum}[1]{\section*{#1} \TOCInsert{section}{#1}}

\newcommand{\researchBasedOn}[2][chapter]{\begin{centering}This #1 is based on~\citet{#2}.\end{centering}}

\newcommand{\figref}[1]{Figure~\ref{#1}}

\newcommand{\matlab}{\textsc{Matlab}\texttrademark}
\newcommand{\labview}{\mbox{LabVIEW\texttrademark}}
\newcommand{\FDIDENT}{Frequency Domain System Identification Toolbox}

\DeclareMathOperator{\Expectancy}{E}
\DeclareMathOperator{\Bias}{Bias}
\DeclareMathOperator{\MSE}{MSE}
\DeclareMathOperator{\RootMeanSquare}{RMS}

\DeclareMathOperator{\RealPart}{\mathrm{Re}}
\DeclareMathOperator{\ImagPart}{\mathrm{Im}}
\DeclareMathOperator{\RealImagPart}{\stackrel{\mathrm{Re}}{\mathrm{Im}}}
\DeclareMathOperator{\Diagonal}{\mathrm{diag}}

\newcommand{\unit}[1]{\ensuremath{\,\textrm{#1}}}
\newcommand{\I}{\ensuremath{\mathrm{I}}}
\renewcommand{\T}{\ensuremath{\mathrm{T}}} %TODO: where is the original definition of \T ?
\newcommand{\HT}{\ensuremath{\mathrm{H}}}
\renewcommand{\C}{\ensuremath{\mathrm{C}}}

\newcommand{\real}[1]{\ensuremath{\RealPart \left[#1\right]}}
\newcommand{\imag}[1]{\ensuremath{\ImagPart \left[#1\right]}}
\newcommand{\realimag}[1]{\ensuremath{\RealImagPart \left[#1\right]}}
\newcommand{\diag}[1]{\ensuremath{\Diagonal \left({#1}\right)}}

\newcommand{\mean}[2]{\ensuremath{\sqrt{#1 #2}}}

\newcommand{\norm}[2][]{\ensuremath{\left\| #2 \right\|_{#1}}}
\newcommand{\FisherInformation}{\ensuremath{\mathrm{Fi}}}
\newcommand{\Fi}[1]{\ensuremath{\FisherInformation \left(#1\right)}}

\newcommand{\CramerRao}{\ensuremath{\mathrm{CR}}}
\newcommand{\CR}[1]{\ensuremath{\CramerRao \left(#1\right)}}

\newcommand{\jac}[1]{\ensuremath{J_{#1}}}
\newcommand{\interest}{\ensuremath{\mathrm{int}}}
\newcommand{\excited }{\ensuremath{\mathrm{exc}}}
\newcommand{\loglikelihood}[1]{\ensuremath{\ell\left( #1 \right) }}
\newcommand{\fc}{\ensuremath{f_{\mathrm{c}}}}
%% Sets
\newcommand{\set}[1]{\ensuremath{\left\{#1\right\}}}
\newcommand{\given}[1]{\left| #1 \right.}

%% Numbers sets
\newcommand{\NaturalNumbers} {\ensuremath{\mathbb{N}}}
\newcommand{\IntegerNumbers} {\ensuremath{\mathbb{Z}}}
\newcommand{\RationalNumbers}{\ensuremath{\mathbb{Q}}}
\newcommand{\RealNumbers}    {\ensuremath{\mathbb{R}}}
\newcommand{\ComplexNumbers} {\ensuremath{\mathbb{C}}}
\newcommand{\Integers}       {\IntegerNumbers}

\newcommand{\NN}{\NaturalNumbers}
\newcommand{\ZN}{\IntegerNumbers}
\newcommand{\QN}{\RationalNumbers}
% \newcommand{\RN}{\RealNumbers}
\newcommand{\CN}{\ComplexNumbers}

\newcommand{\Negative}[1]{\ensuremath{{#1}^{-}}}
\newcommand{\Positive}[1]{\ensuremath{{#1}^{+}}}
\newcommand{\Nonzero }[1]{\ensuremath{{#1}_{0}}}

\newcommand{\distributed}[1][]{\ensuremath{\stackrel{#1}{\sim}}}
\newcommand{\equals}[1][]{\ensuremath{\stackrel{#1}{=}}}
\newcommand{\approxequals}[1][]{\ensuremath{\stackrel{#1}{\approx}}}

%% Calculus operators
\newcommand{\pd}{\partial}
\newcommand{\dd}{\mathrm{d}}
\newcommand{\derivativeNotation}[4][]{\frac{{#4}^{#1} #2}{#4 #3}}
\newcommand{\PartialDerivative}[3][]{\derivativeNotation[#1]{#2}{#3}{\pd}}
\newcommand{\FullDerivative}   [3][]{\derivativeNotation[#1]{#2}{#3}{\dd}}

%% Statistics
\DeclareMathOperator{\ExpectedValue}{\mathrm{E}}
\DeclareMathOperator{\Variance}{\mathrm{Var}}
\DeclareMathOperator{\Covariance}{\mathrm{Cov}}
\DeclareMathOperator{\BiasValue}{\mathrm{bias}}

\newcommand{\E}[2][]{\ExpectedValue_{#1} \left[ #2 \right]}
\newcommand{\Var}[2][]{\Variance_{#1} \left[ #2 \right]}
\newcommand{\Cov}[3][]{\Covariance_{#1} \left[ #2 , #3 \right]}

\newcommand{\round}[2][\QN]{ \left[_{#1} #2 \right]}
\newcommand{\floor}[1]     { \left\lfloor     #1 \right\rfloor}
\newcommand{\ceil }[1]     { \left\lceil      #1 \right\rceil}
\newcommand{\abs }[1]      { \left|      #1 \right|}

%% Statistical Distributions
\DeclareMathOperator{\NormalDistribution}{\mathcal{N}}
\DeclareMathOperator{\DiracDistribution}{\delta}
\DeclareMathOperator{\CircularNormalDistribution}{\mathcal{C}\!\NormalDistribution}
\DeclareMathOperator{\ComplexCircularNormalDistribution}{\CN\!\CircularNormalDistribution}

%%
\newcommand{\mat}[1]{\ensuremath{\mathbf{#1}}}

 % Helpers
\newcommand{\Dirac}[1]{\DiracDistribution\left(#1\right)}
\newcommand{\Normal}[2]{\NormalDistribution\left(#1,#2\right)}
\newcommand{\CircularNormal}[3]{\CircularNormalDistribution\!\left(#1,#2,#3\right)}
\newcommand{\ComplexCircularNormal}[2]{\ComplexCircularNormalDistribution\!\left(#1,#2\right)}

\newcommand{\bias}[1]{\ensuremath{\Bias \left( #1 \right)}}
\newcommand{\mse}[1]{\ensuremath{\MSE \left( #1 \right)}}
\newcommand{\rms}[1]{\ensuremath{\RootMeanSquare \left( #1 \right)}}

\newcommand{\wdB}{\ensuremath{\mathrm{BW}_{3\unit{dB}}}}

\newcommand{\dc}{\ensuremath{G_{\mathrm{dc}}}}
\newcommand{\damping}{\ensuremath{\xi}}
\newcommand{\wn}{\ensuremath{\omega_{\mathrm{n}}}}
\newcommand{\wM}{\ensuremath{\omega_{\mathrm{M}}}}
\newcommand{\wR}{\ensuremath{\omega_{\mathrm{R}}}}
\newcommand{\wNom}{\ensuremath{\omega_{\mathrm{nom}}}}
\newcommand{\fR}{\ensuremath{f_{\mathrm{R}}}}
\newcommand{\fs}{\ensuremath{f_{\mathrm{s}}}}
\newcommand{\Ts}{\ensuremath{T_{\mathrm{s}}}}
\newcommand{\Tw}[1][]{\ensuremath{\wn^{#1}\Ts^{#1}}}
\newcommand{\wInterest}{\ensuremath{\omega_{\mathrm{I}}}}
\newcommand{\snorm}[1][]{\frac{s^{#1}}{\wn^{#1}}}
\newcommand{\nth}[1][]{\ensuremath{n_{\theta}}}
\newcommand{\pdf}[2][]{\ensuremath{f_{#1} \left(#2 \right)}}
\newcommand{\placeBetweenGrid}[1]{\ensuremath{\mathrm{misalign} \left( #1 \right) }}

% \newcommand{\fMax}[1]{\ensuremath{f_{#1 \max}}}
% \newcommand{\fMin}[1]{\ensuremath{f_{#1 \min}}}

\newcommand{\fMax}[1]{\ensuremath{\overline{f_{#1}}}}
\newcommand{\fMin}[1]{\ensuremath{\underline{f_{#1}}}}

%\renewcommand{\fMax}[1]{\ensuremath{f_{#1+1/2}}}
%\renewcommand{\fMin}[1]{\ensuremath{f_{#1-1/2}}}

\newcommand{\DT}{\ensuremath{\mathrm{DT}}}
\newcommand{\CT}{\ensuremath{\mathrm{CT}}}

\newcommand{\OuterProduct} [3][\T]{{#2}{#3}^{#1}}
\newcommand{\OuterProductC}[2]{\OuterProduct[\HT]{#1}{#2}}
\newcommand{\OuterProductR}[2]{\OuterProduct[\T ]{#1}{#2}}
\newcommand{\OuterProductSelf}[2][\T]{\OuterProduct[#1]{#2}{#2}}

\newcommand{\InnerProduct} [3][\T]{{#2}^{#1}{#3}}
\newcommand{\InnerProductC}[2]{\InnerProduct[\HT]{#1}{#2}}
\newcommand{\InnerProductR}[2]{\InnerProduct[\T]{#1}{#2}}
\newcommand{\InnerProductSelf}[2][\T]{\InnerProduct[#1]{#2}{#2}}

\newcommand{\OuterQuadraticForm}[3][\T]{#2  #3 #2^{#1}}
\newcommand{\InnerQuadraticForm}[3][\T]{#2^{#1} #3 #2}
\newcommand{\QuadraticForm}[3][\T]{\InnerQuadraticForm[#1]{#2}{#3}}

\newcommand{\bruelkjaer}{Brüel \& Kj\ae r }

\newcommand{\isdef}{\ensuremath{\triangleq}}

\newcommand{\emspace}{\ensuremath{\qquad}}

\newcommand{\numberOf}{\ensuremath{\#}}
\newcommand{\f}{\ensuremath{\mathbb{f}}}

\newcommand{\LOGGRID}{\ensuremath{\mathrm{LG}}}
\newcommand{\QLOGGRID}{\ensuremath{\mathrm{QLG}}}
\newcommand{\GRID}{\ensuremath{\mathrm{GRD}}}
