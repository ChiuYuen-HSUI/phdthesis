\thispagestyle{empty}
\begin{bottompar}

\textsc{isbn:} 978-9-4619743-2-7 \\
\textsc{doi:} \href{http://dx.doi.org/10.5281/zenodo.60402}{\texttt{10.5281/zenodo.60402}}\\
\textsc{url:} \url{https://github.com/egeerardyn/phdthesis}\\
\textsc{revision:} \texttt{\VCRevision}

University Press\\
Leegstraat 15, 9060 Zelzate, Belgium\\
\url{http://www.universitypress.be}\\

\copyright{} August 2016 Egon Geerardyn.\\

All rights reserved. No parts of this document may be reproduced or transmitted in any form or by any means without the prior written permission of the author.

% The IEEE does not require individuals working on a thesis to obtain a formal reuse license, however, you may print out this statement to be used as a permission grant: 

% Requirements to be followed when using any portion (e.g., figure, graph, table, or textual material) of an IEEE copyrighted paper in a thesis:

% 1) In the case of textual material (e.g., using short quotes or referring to the work within these papers) users must give full credit to the original source (author, paper, publication) followed by the IEEE copyright line © 2011 IEEE. 
% 2) In the case of illustrations or tabular material, we require that the copyright line © [Year of original publication] IEEE appear prominently with each reprinted figure and/or table. 
% 3) If a substantial portion of the original paper is to be used, and if you are not the senior author, also obtain the senior author’s approval. 

%% Own (Egon Geerardyn's) comments:
% 1) copyright info is present in the list of publication, clear attribution is given in both the introduction and text,
% 2) no figures/tables were used verbatim from the paper (re-rendered for book format, often different data sets, ...), i.e. no reprints
% 3) senior author  has agreed with reuse of the material from these paper (providing proper attribution in the corresponding sections)

\end{bottompar}
