

%% Theorem styling
\theoremstyle{definition}
\newtheorem{definition}{Definition}[chapter]

\theoremstyle{plain}
\newtheorem{assumption}{Assumption}[chapter]

\theoremstyle{plain}
\newtheorem{conjecture}{Conjecture}[chapter]

\theoremstyle{plain}
\newtheorem{example}{Example}[chapter]

\theoremstyle{remark}
\newtheorem{remark}{Remark}[chapter]

\theoremstyle{definition}
\newtheorem{property}{Property}[chapter]

\theoremstyle{definition}
\newtheorem{theorem}{Theorem}[chapter]

\theoremstyle{definition}
\newtheorem{lemma}{Lemma}[chapter]

%%
\usepackage[nobreak]{mdframed}


\mdfdefinestyle{guidelinestyle}{% 
  linecolor=black,
  linewidth=1pt,
  frametitlerule=true,
  frametitlefont=\sffamily\bfseries,
  frametitlebackgroundcolor=gray!20,
  innertopmargin=\topskip
} 

% http://tex.stackexchange.com/questions/296965/adding-an-mdtheorem-i-e-with-frametitle-to-a-list-of-theorems/297839#297839
\makeatletter
\define@key{thmdef}{mdthm}[{}]{%
\thmt@trytwice{\def\thmt@theoremdefiner{\mdtheorem[#1]}}{}}
\makeatother
\declaretheorem[mdthm={style=guidelinestyle},numberwithin=chapter]{guideline}

% http://tex.stackexchange.com/questions/180747/editing-format-of-list-of-theorems-in-thmtools
\makeatletter
\def\ll@guideline{%
  \protect\numberline{\csname the\thmt@envname\endcsname}%
  \ifx\@empty\thmt@shortoptarg
    \thmt@thmname
  \else
    \thmt@shortoptarg
  \fi}
\def\l@thmt@theorem{} 
\makeatother
