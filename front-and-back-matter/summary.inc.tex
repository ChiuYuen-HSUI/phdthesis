\chapter{Summary}
\NoThumbs
To understand, simulate and control physical systems, good mathematical models are required. 
System identification is a set of powerful tools that distill mathematical models from measurements of the physical world. 
To this end, one has to:
\begin{enumerate}
\item perform experiments to observe the system in a meaningful way,
\item suppress unwanted effects and extract the features of interest from the measurements,
\item fit a model to the measurements.
\end{enumerate}

Unfortunately, the effective use of system identification hinges strongly on the skill of the user. This doctorate focuses on easy-to-use system identification tools. 
These should allow both beginners to obtain good models, and seasoned practitioners to obtain better models with ease.

The first part of this work focuses on the design of a robust experiment. 
In contrast to optimal designs, only a few assumptions about the system, e.g. the minimal damping and the frequency bands of interest, are made. 
Therefore, such a constructed multisine signal can be used to measure many different systems efficiently.

The second part of this thesis describes non-parametric estimators for the frequency response function (FRF) that estimate and suppress leakage and noise effects. 
Extensions of the Local Polynomial and Local Rational Method (LPM and LRM) are investigated. These approximate the input-output measurements in local frequency domain windows by either polynomial or rational models.

Such local models can be used to great effect e.g. for the flexible dynamics of mechanical structures.
 These local models have been used to obtain a more detailed view of the many resonance peaks of an active vibration isolation system than typical FRFs provide and unlike high-order models, they don’t incur laborious model-order selection.
Consequently, the measurement time and modeling effort can be reduced significantly.

The final part of the thesis considers the use of FRF smoothers to obtain better initial values to fit parametric models to the measurement data. 
In particular for low signal-to-noise ratios, local modeling or regularization provide a considerable improvements. 
This allows users to obtain better models, even from poor measurements.
