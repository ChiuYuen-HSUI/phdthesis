
\subsubsection{Truncation Via a Statistical Test}\label{se:truncstattest}

\newcommand{\segm}[2]{\ensuremath{{#1}^{[#2]}}}
\newcommand{\RMS}{\ensuremath{\,\mathrm{rms}\,}}
\newcommand{\rms}[1]{\ensuremath{\RMS\left(#1\right)}}
\newcommand{\floor}[1]{\ensuremath{\lfloor #1 \rfloor}}

To distinguish the part of the measured impulse response where the actual impulse response dominates from the part where the noise dominates, the estimated impulse response $\hat{g}_\mathrm{poly}$ is split into smaller segments. 
One can then gradually determine whether these segments are noise or the actual impulse response.

% 1) cut into NS segments of minimally NM samples
% 2) assume last segment == noise
%    start from back
% 3) F-test on Var, Var==RMS if dc == 0 (i.e. noise)
%    H0: s²(1) =  s²(2)
%    H1: s²(1) >  s²(2)
%    alpha >> ==> beta <<
% 4) if H1 and RMS(signal)/RMS(noise) > f
%    robustness, SNR
%    select segment i
% 5) refine until wanted resolution is obtained

\begin{enumerate}
  \item The impulse response $\hat{g}_\mathrm{poly}(t)$ is split into $N_S$ segments of approximate equal lengths
$L_S = \floor{\frac{N}{N_S}}$. 
If necessary, the length of the first segment is reduced to allow for the others to have equal lengths.
The $i^{\text{th}}$ segment is denoted $\segm{\hat{g}}{i}$, for $i \in \left\{1,\ldots,N_S\right\}$.
  By Assumption \ref{ass:decay90perctime}, the last segment, $\segm{\hat{g}}{N_S}$, is noise when $L_S \leqslant 0.1N$.
  
  \item Iteration is then carried out from the last segment ($i = N_S$) towards the beginning of the impulse response. To determine whether the preceding segment $\segm{\hat{g}}{i-1}$ is completely dominated by noise,  both an $F$-test and an absolute criterion on the RMS are used.

  The $F$-test is suited to compare the variances of different segments $\segm{\hat{g}}{i}$ and $\segm{\hat{g}}{i-1}$ \cite{Parsons1974}, while accounting for the degrees of freedom in each segment.
  For zero-mean noise, these variances can be related to the RMS value of each noise segment.
  This has the advantage that a segment in which the impulse response still has a significant amplitude is likely to have a large RMS value, which is more likely to be detected.
  Let $s^2(i)$ denote the mean square value of segment $\segm{\hat{g}}{i}$, \emph{viz}:
  \begin{equation}
    s^2(i) = \frac{1}{L_S} \sum_{j=1}^{L_S} \left(\segm{\hat{g}}{i}(j)\right)^2
           = \left( \rms{\hat{g}^{[i]}} \right)^2
           \text{.}
  \end{equation}
  Starting with the last segment ($i = N_S$), the following null hypothesis and the alternative hypothesis are formulated:
  \begin{align}
     H_0 :&\,\, s^2(i-1) = s^2(i)\\
     H_1 :&\,\, s^2(i-1) > s^2(i)
     \text{.}
  \end{align}
  As segment $\segm{\hat{g}}{i}$ has been tested and been classified as noise, the null hypothesis states that the segment $\segm{\hat{g}}{i-1}$ is also noise.
  The alternative case occurs when $\segm{\hat{g}}{i-1}$ has a significant change, which indicates the presence of the signal.
  Using the $F$-test on the RMS values of the two segments with a high confidence level of $\alpha=0.99$ and the aforementioned hypotheses, we can determine whether $\segm{\hat{g}}{i-1}$ is likely to be part of the actual impulse response.
  The high level of confidence ($\alpha=0.99$) ensures that the probability of a Type~II error is smaller than $1-\alpha$ \cite{Parsons1974}.
  In our case, such an error means that a part of the actual impulse response would falsely be regarded as noise, which could significantly increase the bias as information of the system is discarded.
  A Type~I error is less detrimental since, in that case, noise is falsely classified as a signal component and kept in the impulse response, thereby causing a sub-optimal value of the variance of the estimate.
  As the LPM samples are correlated over a short frequency span, the actual noise present in $\hat{g}$ may be slightly non-stationary.
  To cope with this, one can introduce other criteria which must be satisfied together with the outcome of the $F$-test.
  A criterion that shows good results is to check whether the segment $\segm{\hat{g}}{i-1}$ has an RMS value that is at least a factor $\kappa$ larger than the RMS of the noise.
  Even for a moderate $\kappa = 1.1$, a large improvement in the detection was observed. 
%  \JL{Q: doesn't this increase the probability of a Type II error?}
%  \EG{A: This slightly increases for small $\kappa$ and increases significantly for large values of $\kappa$. But it is needed to overcome nonstationarity of the noise.}

  \item This procedure is repeated until a segment $\segm{\hat{g}}{i-1}$ is found that is dominated by the actual impulse response according to the outcome of the $F$-test and the absolute RMS value of the segment.
  At that point, it is very likely that the transition from noise to the actual impulse response happens within the segment $\segm{\hat{g}}{i-1}$.
  One now has the choice to accept the last sample of $\segm{\hat{g}}{i-1}$ to be the last meaningful sample $t_{\mathrm{trunc}}$ for the impulse response.
  The accuracy of this estimate is limited by the length of the segment $L_S$.

  \item
  A more accurate estimate can be obtained by dividing the segment $\segm{\hat{g}}{i-1}$ -- which may contain both the signal and noise -- yet again into smaller segments. The procedure described above is then repeated until a satisfactory, accurate $t_{\mathrm{trunc}}$ is obtained, or until the subsegments have a length that is too short ($L_{S\min} = 10$) to guarantee the accuracy of the RMS values. %rely on the RMS values to be accurate.
  To start this refinement, the last $\segm{\hat{g}}{i}$ should be used to compare the RMS with the $F$-test, since the last subsegment of $\segm{\hat{g}}{i-1}$ cannot be asserted to be dominated by noise.

\end{enumerate}
%In Fig.~\ref{FRF_truncate_stat_EG}, this procedure is shown on the system depicted in \eqref{eq:systemSimulations}. 
This procedure is illustrated in Fig.~\ref{FRF_truncate_stat_EG} for the system described by the following simulation equations, which are relevant to the LPM outlined in Section \ref{se:LPMFRFest}:
\begin{subequations}
\label{eq:systemSimulations}
\begin{align}
y_0(t)  &= 1.5371y_0(t-1)    -0.9025y_0(t-2) + u(t)
\\
y(t) &= y_0(t) + e(t),
\end{align}
\end{subequations}
where $e(t)$ is a white noise sequence, such that the SNR of the output signal is 78.2~dB.
% \eqref{eq:systemSimulations}. 
The figure shows segments of the estimated impulse response. The last two segments are dominated by the noise, while all samples at $t<80$ have large system contributions. The algorithm outlined above selects $t_{\mathrm{trunc}} = 80$, beyond which all samples are set to zero, resulting in a smooth estimate.  %after which all samples are fixed to a zero value for the smoothed estimate.

\begin{figure}[tbh] %top bottom here
\centering
 \setlength\figureheight{0.6\columnwidth}
 \setlength\figurewidth{0.8\columnwidth}
  % This file was created by matlab2tikz v0.2.2.
% Copyright (c) 2008--2012, Nico Schlömer <nico.schloemer@gmail.com>
% All rights reserved.
%
% The latest updates can be retrieved from
%   http://www.mathworks.com/matlabcentral/fileexchange/22022-matlab2tikz
% where you can also make suggestions and rate matlab2tikz.
%
%
%
\begin{tikzpicture}

\begin{axis}[%
width={\figurewidth},
height={\figureheight},
scale only axis,
xmin=0, xmax=160,
xtick={0,40,120,160},
extra x ticks={80},
extra x tick labels={$\truncTime$},
xlabel={Time $t$ \axisunit{samples}},
ymin=-1.3, ymax=1.7,
ylabel={Impulse Response $g$ \noaxisunit},
xmajorgrids,
legend style={nodes=right},
unbounded coords=jump]
\addplot [G0Hat]
coordinates{
 (0,1)(1,1.5371322893124)(2,1.46027567484678)(3,0.857375)(4,-9.19403442267708e-17)(5,-0.7737809375)(6,-1.18940366388567)(7,-1.12993348069139)(8,-0.663420431289062)(9,2.58473797920544e-16)(10,0.598736939238379)(11,0.920337882107388)(12,0.874320988002018)(13,0.513342083279504)(14,-2.97288235695525e-16)(15,-0.463291230159753)(16,-0.712139909233819)(17,-0.676532913772128)(18,-0.397214318458218)(19,3.70688722772794e-16)(20,0.358485922408542)(21,0.551040286598109)(22,0.523488272268203)(23,0.307356867725023)(24,-3.69062419514066e-16)(25,-0.277389573121834)(26,-0.426384469564153)(27,-0.405065246085946)(28,-0.237826885255332)(29,3.25911173049143e-16)(30,0.214638763942937)(31,0.329928174594791)(32,0.313431765865051)(33,0.184025910235575)(34,-3.30708767662391e-16)(35,-0.166083383987607)(36,-0.255292132245621)(37,-0.242527525633339)(38,-0.142395741346374)(39,2.46276523480082e-16)(40,0.128512156565103)(41,0.19754018542539)(42,0.187663176154121)(43,0.110183110235005)(44,-1.57683653460861e-16)(45,-0.0994402569870922)(46,-0.152852829872382)(47,-0.145210188378763)(48,-0.085257590334308)(49,1.08691267791672e-16)(50,0.0769449752767131)(51,0.11827460599818)(52,0.112360875698271)(53,0.0659706981778718)(54,-9.03953561309789e-18)(55,-0.0595385551055293)(56,-0.0915186355117147)(57,-0.086942703736129)(58,-0.0510468686836032)(59,-3.37119113007212e-17)(60,0.0460697989869518)(61,0.0708153755849754)(62,0.0672746068057266)(63,0.0394990939064379)(64,6.96464370550376e-17)(65,-0.0356479322505601)(66,-0.0547955877095567)(67,-0.052055808324079)(68,-0.030563645913324)(69,-6.6223576915183e-17)(70,0.0275836904367748)(71,0.0423997812287643)(72,0.0402797921673261)(73,0.0236495665882299)(74,7.47171997937755e-17)(75,-0.0213437338458773)(76,-0.032808142468988)(77,-0.0311677353455387)(78,-0.0182995838061092)(79,-1.10264207974723e-16)(80,0.0165153743850134)(81,0.0253863152372871)(82,0.0241169994754228)(83,0.014159869113351)(84,1.09491713926827e-16)(85,-0.0127792818747991)(86,-0.0196434468039785)(87,-0.0186612744637796)(88,-0.010956636797406)(89,-9.
04008618451361e-17)(90,0.00988836470965878)(91,0.0151997246837138)(92,0.0144397384495282)(93,0.00847803669294386)(94,1.089870940485e-16)(95,-0.00765142811538167)(96,-0.011761257215506)(97,-0.0111731943547307)(98,-0.00656014318042556)(99,-6.94431491476966e-17)(100,0.00592052922033389)(101,0.0091006366343929)(102,0.00864560480267328)(103,0.00507611374028394)(104,8.28330459778925e-17)(105,-0.00458119265060614)(106,-0.00704189914680743)(107,-0.00668980418946712)(108,-0.00392780004881355)(109,-4.17417836406919e-17)(110,0.00354483954405415)(111,0.00544888732359714)(112,0.0051764429574173)(113,0.00303925680408351)(114,2.99646375420681e-17)(115,-0.00274292926568532)(116,-0.00421624514158486)(117,-0.00400543288450561)(118,-0.00235171897916694)(119,-1.51313965697508e-17)(120,0.00212242637869816)(121,0.00326245011838534)(122,0.00309932761246606)(123,0.00181971531643633)(124,1.34170018845081e-17)(125,-0.00164229307308376)(126,-0.00252442171115117)(127,-0.00239820062559361)(128,-0.00140806102353516)(129,2.98195402915553e-17)(130,0.00127077507374057)(131,0.00195334939829986)(132,0.00185568192838489)(133,0.00108953077884828)(134,8.01099519434303e-17)(135,-0.000983301527910587)(136,-0.00151146452868156)(137,-0.00143589130224749)(138,-0.000843058147492346)(139,-1.34370239721807e-17)(140,0.000760859978111791)(141,0.00116954244000115)(142,0.00111106531800114)(143,0.000652342323733666)(144,5.27699411142e-17)(145,-0.000588738947169537)(146,-0.000904969645670134)(147,-0.00085972116338667)(148,-0.000504770054829573)(149,-5.87227774375603e-17)(150,0.000455554974483582)(151,0.000700248260835645)(152,0.000665235847793916)(153,0.000390581446247955)(154,6.47222614612674e-17)(155,-0.000352499755238696)(156,-0.000541838755752168)(157,-0.000514746817964564)(158,-0.000302224477647856)(159,-5.72914901895746e-17)(160,0.000272757591077104)(161,0.000419264500399693)
};
\addlegendentry{True Value};

\addplot [color=lightgray!80!black,solid]
coordinates{
 (0,1.0530788305713)(1,1.5401750859347)(2,1.45802774931507)(3,0.86442459854892)(4,-0.00639046554012114)(5,-0.770899909268136)(6,-1.18642954089334)(7,-1.0988342169306)(8,-0.68506381347042)(9,0.0399708891574246)(10,0.582041248283753)(11,0.959424942889007)(12,0.868352149881463)(13,0.534339065002973)(14,-0.0219200884705951)(15,-0.449211924918872)(16,-0.721946250583923)(17,-0.669426778593483)(18,-0.373376001506991)(19,-0.0022709676474749)(20,0.350322200100093)(21,0.571403202780426)(22,0.554205136082806)(23,0.29909565187015)(24,-0.0237986799431506)(25,-0.304203990253186)(26,-0.42813701795974)(27,-0.367058311722921)(28,-0.262557190612986)(29,-0.0274661695854109)(30,0.214175682648701)(31,0.310962412196269)(32,0.311037958803447)(33,0.209104758359952)(34,-0.00671904544159164)(35,-0.140494255746866)(36,-0.244750270738285)(37,-0.262875783709227)(38,-0.132322023640012)(39,0.00465465753686413)(40,0.115106772116674)(41,0.203431970436252)(42,0.182567654517998)(43,0.102159013124422)(44,-0.0107930930701593)(45,-0.0875488023510008)(46,-0.176735503357052)(47,-0.154047657658991)(48,-0.0947508213911722)(49,0.0205217625161969)(50,0.0997857617656283)(51,0.0911535793373973)(52,0.0962798164812225)(53,0.100578841992491)(54,0.0218501379371979)(55,-0.0411361498567127)(56,-0.0979354125787488)(57,-0.0941558858726104)(58,-0.0578038943961262)(59,-0.00247128921399623)(60,0.0371278788838963)(61,0.0426006567812478)(62,0.0583295959870679)(63,0.0287880172912928)(64,0.0301903935276118)(65,-0.0449464000912856)(66,-0.0431775699553004)(67,-0.0460879462916128)(68,-0.0109525225196768)(69,0.0259314140642792)(70,0.0393948296460815)(71,0.0734869744237198)(72,0.0126729865460558)(73,0.0210051418814731)(74,-0.00869414240387733)(75,-0.0037755930378919)(76,-0.0454813480527951)(77,-0.0345628267483219)(78,-0.0235173536721714)(79,-0.0191423884800511)(80,0.0139275469015572)(81,-0.0249685037417402)(82,0.0453234250443737)(83,-0.0363632237008142)(84,0.0526588455311864)(85,-0.00871880635556728)(86,0.00463223578390932)(87,-0.00295275845366561)(88,0.00226087007098701)(89,0.
0223031498517009)(90,0.00930507847920215)(91,0.0424493033614078)(92,0.0210554804986187)(93,0.0137926092452841)(94,-0.00879273684783215)(95,-0.0427631467397175)(96,0.0169742643303203)(97,-0.0102483098273445)(98,0.00702007059731766)(99,0.00389680308259893)(100,0.0200203386030399)(101,-0.0149723885519309)(102,0.0206932320549459)(103,-0.00990016880749287)(104,0.0122169096634572)(105,0.0326093021489499)(106,-0.0158280378400942)(107,-0.00384069310189821)(108,-0.0189978612360518)(109,0.00361300963961247)(110,-0.0086414779648314)(111,0.00967489049278187)(112,0.0127767263747594)(113,0.0223467406790162)(114,0.0220451436115617)(115,-0.0158533266301636)(116,-0.0327964043145136)(117,0.00160788974757907)(118,-0.015110510620568)(119,0.0213032326112286)(120,0.0023400304500529)(121,-0.00470808405409158)(122,-0.0116699533631736)(123,0.0439627790305935)(124,0.00405244981816977)(125,-0.00984448738347365)(126,0.0188834343586767)(127,0.016776623851269)(128,-0.0230014411021799)(129,-0.0257812870197883)(130,-0.0115513955393953)(131,0.0228515414156505)(132,-0.0018548544080748)(133,0.0170199205246556)(134,-0.00242844944846287)(135,-0.0103573115285161)(136,0.00143309952240079)(137,-0.0141897295156295)(138,0.0240191779599644)(139,0.00108024180330641)(140,0.0228388372432649)(141,0.0271013681331801)(142,0.00534558063780137)(143,-0.00249336000305963)(144,0.00561667226430895)(145,-0.00532193212122067)(146,0.00774716772437572)(147,-0.0153978136373772)(148,0.0101929461603683)(149,-0.0187856514764983)(150,0.0042812902722015)(151,0.0204403243691005)(152,0.00316252335774176)(153,0.027806894789268)(154,-0.0023256911611946)(155,0.0479062715493156)(156,0.0293056334746769)(157,-0.00149486623807802)(158,-0.000558253064721262)(159,0.0156180853939494)(160,0.0517576654594346)(161,-0.00540006188689709)
};
\addlegendentry{$\estimated g$};

\addplot [
color=black,
solid,
line width=1pt
]
coordinates{
 (0,1.0530788305713)(1,1.5401750859347)(2,1.45802774931507)(3,0.86442459854892)(4,-0.00639046554012114)(5,-0.770899909268136)(6,-1.18642954089334)(7,-1.0988342169306)(8,-0.68506381347042)(9,0.0399708891574246)(10,0.582041248283753)(11,0.959424942889007)(12,0.868352149881463)(13,0.534339065002973)(14,-0.0219200884705951)(15,-0.449211924918872)(16,-0.721946250583923)(17,-0.669426778593483)(18,-0.373376001506991)(19,-0.0022709676474749)(20,0.350322200100093)(21,0.571403202780426)(22,0.554205136082806)(23,0.29909565187015)(24,-0.0237986799431506)(25,-0.304203990253186)(26,-0.42813701795974)(27,-0.367058311722921)(28,-0.262557190612986)(29,-0.0274661695854109)(30,0.214175682648701)(31,0.310962412196269)(32,0.311037958803447)(33,0.209104758359952)(34,-0.00671904544159164)(35,-0.140494255746866)(36,-0.244750270738285)(37,-0.262875783709227)(38,-0.132322023640012)(39,0.00465465753686413)(40,0.115106772116674)(41,0.203431970436252)(42,0.182567654517998)(43,0.102159013124422)(44,-0.0107930930701593)(45,-0.0875488023510008)(46,-0.176735503357052)(47,-0.154047657658991)(48,-0.0947508213911722)(49,0.0205217625161969)(50,0.0997857617656283)(51,0.0911535793373973)(52,0.0962798164812225)(53,0.100578841992491)(54,0.0218501379371979)(55,-0.0411361498567127)(56,-0.0979354125787488)(57,-0.0941558858726104)(58,-0.0578038943961262)(59,-0.00247128921399623)(60,0.0371278788838963)(61,0.0426006567812478)(62,0.0583295959870679)(63,0.0287880172912928)(64,0.0301903935276118)(65,-0.0449464000912856)(66,-0.0431775699553004)(67,-0.0460879462916128)(68,-0.0109525225196768)(69,0.0259314140642792)(70,0.0393948296460815)(71,0.0734869744237198)(72,0.0126729865460558)(73,0.0210051418814731)(74,-0.00869414240387733)(75,-0.0037755930378919)(76,-0.0454813480527951)(77,-0.0345628267483219)(78,-0.0235173536721714)(79,-0.0191423884800511)(80,0)(81,0)(82,0)(83,0)(84,0)(85,0)(86,0)(87,0)(88,0)(89,0)(90,0)(91,0)(92,0)(93,0)(94,0)(95,0)(96,0)(97,0)(98,0)(99,0)(100,0)(101,0)(102,0)(103,0)(104,0)(105,0)(106,0)(107,0)(108,0)(109,0)(110,0)(111,0)(112,0)(113,0)(
114,0)(115,0)(116,0)(117,0)(118,0)(119,0)(120,0)(121,0)(122,0)(123,0)(124,0)(125,0)(126,0)(127,0)(128,0)(129,0)(130,0)(131,0)(132,0)(133,0)(134,0)(135,0)(136,0)(137,0)(138,0)(139,0)(140,0)(141,0)(142,0)(143,0)(144,0)(145,0)(146,0)(147,0)(148,0)(149,0)(150,0)(151,0)(152,0)(153,0)(154,0)(155,0)(156,0)(157,0)(158,0)(159,0)(160,0)(161,0)
};
\addlegendentry{$\estimated g_{\trunc}$};

\addplot [
color=gray,
densely dotted
]
coordinates{
 (0,0.628802731910641)(1,0.628802731910641)(2,0.628802731910641)(3,0.628802731910641)(4,0.628802731910641)(5,0.628802731910641)(6,0.628802731910641)(7,0.628802731910641)(8,0.628802731910641)(9,0.628802731910641)(10,0.628802731910641)(11,0.628802731910641)(12,0.628802731910641)(13,0.628802731910641)(14,0.628802731910641)(15,0.628802731910641)(16,0.628802731910641)(17,0.628802731910641)(18,0.628802731910641)(19,0.628802731910641)(20,0.628802731910641)(21,0.628802731910641)(22,0.628802731910641)(23,0.628802731910641)(24,0.628802731910641)(25,0.628802731910641)(26,0.628802731910641)(27,0.628802731910641)(28,0.628802731910641)(29,0.628802731910641)(30,0.628802731910641)(31,0.628802731910641)(32,0.628802731910641)(33,0.628802731910641)(34,0.628802731910641)(35,0.628802731910641)(36,0.628802731910641)(37,0.628802731910641)(38,0.628802731910641)(39,0.0795043572643008)(40,0.0795043572643008)(41,0.0795043572643008)(42,0.0795043572643008)(43,0.0795043572643008)(44,0.0795043572643008)(45,0.0795043572643008)(46,0.0795043572643008)(47,0.0795043572643008)(48,0.0795043572643008)(49,0.0795043572643008)(50,0.0795043572643008)(51,0.0795043572643008)(52,0.0795043572643008)(53,0.0795043572643008)(54,0.0795043572643008)(55,0.0795043572643008)(56,0.0795043572643008)(57,0.0795043572643008)(58,0.0795043572643008)(59,0.0795043572643008)(60,0.0795043572643008)(61,0.0795043572643008)(62,0.0795043572643008)(63,0.0795043572643008)(64,0.0795043572643008)(65,0.0795043572643008)(66,0.0795043572643008)(67,0.0795043572643008)(68,0.0795043572643008)(69,0.0795043572643008)(70,0.0795043572643008)(71,0.0795043572643008)(72,0.0795043572643008)(73,0.0795043572643008)(74,0.0795043572643008)(75,0.0795043572643008)(76,0.0795043572643008)(77,0.0795043572643008)(78,0.0795043572643008)(79,0.0795043572643008)(80,0.0214125686888731)(81,0.0214125686888731)(82,0.0214125686888731)(83,0.0214125686888731)(84,0.0214125686888731)(85,0.0214125686888731)(86,0.0214125686888731)(87,0.0214125686888731)(88,0.0214125686888731)(89,0.0214125686888731)(90,0.0214125686888731)(
91,0.0214125686888731)(92,0.0214125686888731)(93,0.0214125686888731)(94,0.0214125686888731)(95,0.0214125686888731)(96,0.0214125686888731)(97,0.0214125686888731)(98,0.0214125686888731)(99,0.0214125686888731)(100,0.0214125686888731)(101,0.0214125686888731)(102,0.0214125686888731)(103,0.0214125686888731)(104,0.0214125686888731)(105,0.0214125686888731)(106,0.0214125686888731)(107,0.0214125686888731)(108,0.0214125686888731)(109,0.0214125686888731)(110,0.0214125686888731)(111,0.0214125686888731)(112,0.0214125686888731)(113,0.0214125686888731)(114,0.0214125686888731)(115,0.0214125686888731)(116,0.0214125686888731)(117,0.0214125686888731)(118,0.0214125686888731)(119,0.0214125686888731)(120,0.0214125686888731)(121,0.0193285829589688)(122,0.0193285829589688)(123,0.0193285829589688)(124,0.0193285829589688)(125,0.0193285829589688)(126,0.0193285829589688)(127,0.0193285829589688)(128,0.0193285829589688)(129,0.0193285829589688)(130,0.0193285829589688)(131,0.0193285829589688)(132,0.0193285829589688)(133,0.0193285829589688)(134,0.0193285829589688)(135,0.0193285829589688)(136,0.0193285829589688)(137,0.0193285829589688)(138,0.0193285829589688)(139,0.0193285829589688)(140,0.0193285829589688)(141,0.0193285829589688)(142,0.0193285829589688)(143,0.0193285829589688)(144,0.0193285829589688)(145,0.0193285829589688)(146,0.0193285829589688)(147,0.0193285829589688)(148,0.0193285829589688)(149,0.0193285829589688)(150,0.0193285829589688)(151,0.0193285829589688)(152,0.0193285829589688)(153,0.0193285829589688)(154,0.0193285829589688)(155,0.0193285829589688)(156,0.0193285829589688)(157,0.0193285829589688)(158,0.0193285829589688)(159,0.0193285829589688)(160,0.0193285829589688)(161,0.0193285829589688)
};
\addlegendentry{$\mathrm{RMS} (\segm{\estimated g}{i})$};

% \addplot [color=black,dotted,forget plot]
% table[row sep=crcr]{
% 39	-1.5\\
% 39	 2.0\\
% nan nan\\
% 80	-1.5\\
% 80	 2.0\\
% nan nan\\
% 121 -1.5\\
% 121 2\\
% };

\addplot [truncationline] table[row sep=crcr]{
80	-1.5\\
80	 2.0\\
};

\node[annotation] at (axis cs: 20,-1.1) {Segment 1};
\node[annotation] at (axis cs: 60,-1.1) {Segment 2};
\node[annotation] at (axis cs: 100,-1.1) {Segment 3};
\node[annotation] at (axis cs: 140,-1.1) {Segment 4}; 

\end{axis}


\end{tikzpicture}%

\caption{Schematic of statistical determination of a smooth (improved) FRF. The different segments are separated by dotted lines.}
\label{FRF_truncate_stat_EG}
\end{figure}
