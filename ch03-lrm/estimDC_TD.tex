% !TEX root =  LPMSmooth_ML_EG_JS_JL.tex

\subsection{Estimating the DC Value of the FRF}\label{se:DCvalueEst}


The correct mean value of the impulse response accounts for (i.e. estimates) the DC value of the FRF. A time-domain method is proposed for estimating the mean value of the impulse response in equation \eqref{eq:impRespiDFT}. %The first is a frequency-domain technique and the second is a time-domain technique.





According to Assumption \ref{ass:imprespdecay}, the true impulse response tends   to zero asymptotically; and by Assumption \ref{ass:decay90perctime}, the last $10\%$ of the estimated impulse response is noise, plus a constant value due to an inaccuracy in the average value of the impulse response. The correct estimate of the mean value is obtained by shifting the whole signal such that the last $10\%$ of the impulse response is centered around 0.
To this end, the following procedure is executed:


\begin{enumerate}
\item Compute the mean value $m_{g10}$ of the last 10\% of the impulse response $\hat g_{\mathrm{poly}\setminus \mathrm{DC
}}(t)$, estimated by the LPM, as %$\hat g_\mathrm{poly}(t)$ as
\begin{align}
m_{g10} = \frac{1}{\lceil0.1N - 1\rceil}\sum_{t=\lfloor0.9N\rfloor}^{N-1}\hat g_{\mathrm{poly}\setminus \mathrm{DC
}}(t)
\end{align}

\item Next, subtract the computed mean value $m_{g10}$ from $\hat g_{\mathrm{poly}\setminus \mathrm{DC
}}(t)$ to obtain the improved impulse response $\hat g_\mathrm{poly}(t)$, \emph{viz}:

%\JL{
\begin{align}
\hat g_\mathrm{poly}(t) = \hat g_{\mathrm{poly}\setminus \mathrm{DC
}}(t) - m_{g10},\ \text{for}\ t=0,1,\dots,N-1
\end{align}
%}

\end{enumerate}