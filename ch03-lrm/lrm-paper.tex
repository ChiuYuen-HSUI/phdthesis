\section{Introduction}
Estimating a non-parametric \gls{FRF} is often a key step in the analysis and/or design of physical systems.
However, obtaining a good-quality non-parametric \gls{FRF} from a measured input-output data set can prove difficult due to the presence of noise and measurement time limitations to observe lightly-damped dynamics well.


\citep{Pintelon2010LPM1},\citep{Pintelon2010LPM2},\citep{McKelvey2012LRM}


\paragraph*{Contributions}
In this paper, the existing \gls{LRM} is extended to the \gls{LRIC} that does not suffer from  the bias that is present in the \gls{LRM}.
An extensive study is carried out to compare the performance of the \gls{LPM}, \gls{LRM} and \gls{LRIC} to describe resonant systems under varying conditions of the signal-to-noise ratio.
Also, a pragmatic model-order-selection tool for the \gls{LPM} and \gls{LRM} is introduced.

\paragraph*{Contents}
\secref{sec:theory} introduces theory of estimating \glspl{FRF} and the relevant local modeling techniques.
In \secref{sec:simulations} the performance of the different modeling methods is investigated by means of simulations.
The methods are then illustrated on measurements of mechanical systems in \secref{sec:measurements}.
Finally, conclusions are presented in \secref{sec:conclusion}.



\TODO{
\begin{itemize}
    \item choice $s$ or $z$
    \item bounds RMSE, Bias, StdDev
    \item derivation Bias
    \item comparison LPM/LRM
      \begin{itemize}
        \item Good LPM window
        \item Good LRM window
      \end{itemize}
    \item influence of noise filter: mismatch with model
    \item comparison LRM/LRik (LBTLS)
\end{itemize}
}



\section{Local modeling approaches}
\label{sec:theory}

Consider a discrete-time generalized output-error \gls{LTI} set-up, excited by the input signal $u[n]$.
For an infinite data length, such a system can be described in the time domain by
\begin{equation}
  y[n] = \ezbrace{\true{G}(z) u[n]}{\true{y}[n]} + \ezbrace{\true{H}(z) \true{e}[n]}{v[n]} \text{ with } n \in \ZZ
  \label{eq:output-error-TD-infinite}
\end{equation}
where $v[n]$ is filtered white noise, $v[n] = \true{H}(z) \true{e}[n]$ where $z^{-1}$ is the backward shift operator, i.e. $z^ {-1}x[n] = x[n-1]$.
The transfer functions $\true{G}$ and $\true{H}$ are rational functions that are stable and causal.

During typical measurements, however, $y[n]$ and $u[n]$ are only measured over a limited time span, i.e. $n \in \set{0,1,\ldots,N-1}$.
This introduces transient terms $t_{\bullet}$ in this relationship~\citep{Pintelon1997ARB}:
\begin{equation}
y[n] = \true{G}(z) u[n] + \true{H}(z) \true{e}[n] + \ezbrace{t_G[n] + t_H[n]}{t[n]}
\label{eq:output-error-TD-finite}
\end{equation}
where $t_G[n]$ and $t_H[n]$ are due to the different state of respectively $\true{G}$ and $\true{H}$ at the beginning and end of the measurement record~\citep{Pintelon1997ARB}.
Both terms can be lumped together as $t[n]$, which is often~\citep{Pintelon2010LPM1} dominated by $t_G[n]$.

\begin{definition}\label{def:DFT}
The $N$-points \glsfirst{DFT} of a signal $x(n\Ts) = x[n]$ with $n \in \set{0,\ldots,N-1}$ is
\begin{equation}
  X \left[ k \right] =
  X \left( \omega_k \right)
  \isdef
  \frac{1}{\sqrt{N}}\sum_{n=0}^{N-1} x \left( n \Ts \right)  \exponent{- j \omega_k n \Ts }
  \label{eq:DFT}
\end{equation}
where $\omega_k \isdef \tfrac{2 \pi k }{N \Ts}$, $k \in \set{0,\ldots,N-1}$ and $\Ts$ is the sampling time.
\end{definition}

By applying the \gls{DFT}~\eqref{eq:DFT} to both sides of \eqref{eq:output-error-TD-finite}, one obtains its frequency-domain counterpart:
\begin{align}
    Y \left( \omega_k \right) 
    & = \true{G} \left( \omega_k \right) U \left( \omega_k \right) 
      + \true{H} \left( \omega_k \right) E \left( \omega_k \right)
      + T \left( \omega_k \right)\\
      &= \true{G} \left( \omega_k \right) U \left( \omega_k \right)  + V(\omega_k) + T(\omega_k)
      \text{.}
  \label{eq:output-error-FD}
\end{align}

In this paper, we focus on using local modeling to separate the different terms, i.e.
\begin{itemize}
  \item the transfer function $\true{G}$,
  \item the leakage term $T$, and
  \item the noise term $V$.
\end{itemize}

\subsection{Local modeling}
Local modeling methods  exploit the `smoothness' (or conversely `roughness') of the different terms of \eqref{eq:output-error-FD} over the frequency $\omega$.
In particular, we will assume that $U(\omega)$ is `rough' over the frequency~\citep{Schoukens2009LPM}, e.g. as is the case for noise excitations or random phase multisines (see further).
On the other hand, it is well-known that the transfer functions $\true{G}$, $\true{H}$ and hence also the corresponding transient contributions $T = T_G + T_H$ are smooth over the frequency.
As such, these smooth contributions can be approximated well around each frequency $\omega_k$ using a local rational model:
\begin{align}
  \true{G}(\omega_{k}+d) 
  &\approx
  \frac{\sum_{i=0}^{N_B} b_i d^i}
            {1 + \sum_{i=1}^{N_A} a_i d^i}
    &\isdef
    \frac{\LocalModel[k]{B}(d)}%
           {\LocalModel[k]{A}(d)} 
           = \LocalModel[k]{G}(d)
  \text{,}
  \label{eq:LRM:model:G}
  \\
  \true{T}(\omega_k + d) &\approx
  \frac{\sum_{i=0}^{N_C} c_i d^i}
            {1 + \sum_{i=1}^{N_A} a_i d^i}
    &\isdef 
      \frac{\LocalModel[k]{C}(d)}%
           {\LocalModel[k]{A}(d)}
      = \LocalModel[k]{T}(d)
  \text{.}
  \label{eq:LRM:model:T}
\end{align}

In these expressions, we denote $\LocalModel[k]{G}$ and $\LocalModel[k]{T}$ for the local model for respectively transfer function and the transient.
Note that such local quantities (which depend on the frequency bin $k$) are denoted in bold throughout this paper.
To alleviate the notations, the subscript $k$ is often omitted.

To estimate the local parameters $\LocalVector{\theta} = \mat{a_1\; \cdots\; b_0 \cdots\; c_0\; \cdots}^{\T}$ in \eqref{eq:LRM:model:G} and \eqref{eq:LRM:model:T}, we consider a local window
\begin{equation}
  \LocalWindow[k] 
  \isdef
  \Set{
    \omega_{k+r} 
    | 
    r \in \LocalShifts{k}
  }
\end{equation}
with $\LocalShifts{k} \isdef \Set{-\numel{W}, -\numel{W}+1, \ldots, 0, \ldots, \numel{W}}$
such that the local window around $\omega_k$ consists of $\numel{\LocalWindow} = 2 \numel{W} + 1$ bins. 
If one denotes the input/output spectra in such a local frequency window $\LocalWindow[k]$ as
\begin{equation}
  \LocalVector{U}_k \isdef 
  \mat{
    U(\omega_{k-N_W})\\
    \vdots\\
    U(\omega_{k})\\
    \vdots\\
    U(\omega_{k+N_W})\\
  }
  \text{ and }
  \LocalVector{Y}_k \isdef 
  \mat{
    Y(\omega_{k-N_W})\\
    \vdots\\
    Y(\omega_{k})\\
    \vdots\\
    Y(\omega_{k+N_W})\\
  }
  \text{,}
\end{equation}
and similarly for the other quantities, equation \eqref{eq:output-error-FD} limited to $\LocalWindow[k]$ can be written as
$
  \LocalVector{Y}_k  = {\true{\LocalVector{G}}}_k \hadamard \LocalVector{U}_k + \LocalVector{T}_k + \LocalVector{V}_k
$, where $\hadamard$ denotes the entry-wise product (or Hadamard product).
Substituting $\true{\LocalVector{G}}$ and $\LocalVector{T}$ with the local models $\LocalModel{G}$ and $\LocalModel{T}$ and neglecting the influence of $\LocalVector{V}$ yields
\begin{equation}
  \LocalVector{Y} 
  \approx 
  \hat{\LocalVector{Y}} 
    =
      \LocalModel{G} \hadamard \LocalVector{U} + \LocalModel{T}
      \text{.}
      \label{eq:local-system-equations}
\end{equation}
Note that this encompasses $\numel{\Window}$ complex equations in the $\numel{\theta} = \numel{A} + \numel{B} + \numel{C} + 2$  unknown complex model parameters ($a_i$, $b_i$ and $c_i$ in \eqref{eq:LRM:model:G} and \eqref{eq:LRM:model:T}).
Consequently, a necessary condition to compute the model parameters is that
\begin{equation}
  \DOF = \numel{\Window} - \numel{\theta}
       = 2 \numel{W} - \numel{A} - \numel{B} - \numel{C} - 1
\end{equation}
is positive.
To effectively compute the local model parameters, the equation error in \eqref{eq:local-system-equations} is minimized by formulating a quadratic cost function:
\begin{equation}
  \LocalVector{\CostFunc{\LRIC}} 
    \isdef 
      \left( \LocalVector{Y}  -  \LocalModel{G} \hadamard \LocalVector{U} - \LocalModel{T} \right)^{\HT} 
      \left( \LocalVector{Y}  -  \LocalModel{G} \hadamard \LocalVector{U} - \LocalModel{T} \right)
   \label{eq:costfunc:LRIC}
   \text{.}
\end{equation}
Due to the presence of the denominator of $\LocalModel{G} = \frac{\LocalModel{B}}{\LocalModel{A}}$, the equation error is not linear in the parameters $a_i$ and hence \eqref{eq:costfunc:LRIC} requires time-consuming iterative optimization schemes to obtain a parameter estimate.

\subsection{The Local Rational Method}
The \gls{LRM} as first introduced by \citep{McKelvey2012LRM}, overcomes the computational burden of an iterative procedure by weighting the equation error by the denominator polynomial $\LocalModel{A}$  akin to the procedure in \citep{Levy1959}.
I.e. the \gls{LRM} procedure tries to minimize the equation error in
\begin{equation}
  \LocalModel{A}\LocalVector{Y} \approx \LocalModel{B} \LocalVector{U}  - \LocalVector{C}
  \text{,}
\end{equation}
for which the corresponding cost function is
\begin{equation}
  \LocalVector{\CostFunc{\LRM}}
  \isdef 
  \left( \LocalModel{A} \LocalVector{Y}  -  \LocalModel{B} \hadamard \LocalVector{U} - \LocalModel{C} \right)^{\HT} 
      \left( \LocalModel{A} \LocalVector{Y}  -  \LocalModel{B} \hadamard \LocalVector{U} - \LocalModel{C} \right)
      \text{.}
\end{equation}
Equivalently, this can be rewritten as a linear regression problem
\begin{equation}
  \LocalVector{Y} = \LocalVector{K} \LocalVector{\theta} + \LocalModel{V}
\end{equation}
where $\LocalVector{K}$ is the so-called design matrix (or observation matrix):
\begin{align}
  \label{eq:design-matrix}
  \LocalVector{K} 
    & \isdef 
  \mat{
     \LocalVector{K}_A &
     \LocalVector{K}_B & 
     \LocalVector{K}_C
  }\\
  \LocalVector{K}_A 
    &\isdef
    \mat{
      \LocalVector{Y} \hadamard \LocalVector{d}^1 &
      \cdots &
      \LocalVector{Y} \hadamard \LocalVector{d}^{\numel{A}}
    }\\
  \LocalVector{K}_B 
    &\isdef
    \mat{
      \LocalVector{U} \hadamard \LocalVector{d}^0 &
      \cdots &
      \LocalVector{U} \hadamard \LocalVector{d}^{\numel{B}}
    }\\
  \LocalVector{K}_C
    &\isdef
    \mat{
      \LocalVector{d}^0 &
      \cdots &
      \LocalVector{d}^{\numel{C}}
    }\\
    \LocalVector{d}^{n} & \isdef
    \mat{
      (-\numel{W})^{n} &
      \cdots &
      (\numel{W})^{n}
    }^{\T}
    \text{ with $n \in \mathbb{Z}$.}
\end{align}
and $\LocalModel{V}$ are the model residuals.
This formulation facilitates to solve the problem in a one-step approach:
\begin{equation}
  \LocalVector{\theta}_{\LRM} 
    \isdef \pinv{\LocalVector{K}} \LocalVector{Y}
    = \left( \LocalVector{K}^{\HT} \LocalVector{K} \right)^{-1} \LocalVector{K}^{\HT} \LocalVector{Y}
\end{equation}
where $\pinv{K}$ denotes the Moore-Penroose pseudo-inverse of $K$.

\TODO{hier voortschrijven}

\subsection{The Local Rational Method with Iterative Cost}

\subsection{Notes}

Since it is well-known that the Vandemonde structure in the design matrix \eqref{eq:design-matrix} easily leads to ill-conditioned systems, additional measures are taken to improve numerical conditioning.
To improve numerical conditioning~\citep{Pintelon2005} of the estimation problem, we substitute $d = \dw{r}{k}$ in equations \eqref{eq:LRM:model:G,eq:LRM:model:T}
\begin{equation}
\dw{r}{k} \isdef \frac{\omega_{k+r} - \omega_k}{\Delta \omega_k}
\label{eq:freqScaling}
\end{equation}
where
\begin{equation}
  \Delta \omega_k \isdef
  \max
  \Set{
    \abs{ \omega_k - \omega_j} |  \vphantom{\frac{a}{b}}  \omega_j \in \LocalWindow[k]
  }
\end{equation}
such that $\abs{\dw{r}{k}} \leq 1$ when $r \in \LocalShifts{k}$.

\TODO{define local cost}
\TODO{define local design matrix}


\section{Theory: bias}
\label{sec:biascalc}
Since the local design matrix \eqref{eq:designMatrix} contains the noisy output spectrum $\LocalVector{Y}$ when $\numel{A} > 0$, the \gls{LRM} is expected to be a biased estimator.
In this section, we will derive expressions for the bias such that it can be quantified how important this bias is in different situations.

To determine the bias, an approach similar to \citep[Appendix A]{Guillaume1995} is followed.
Let us denote the expected value of the cost function over all measurements $\LocalVector{Z}$ as
\begin{align}
  \ELSCost{\theta}
              & \isdef 
                   \E[\LocalVector{Z}]%
                         {    
                                \LSCost{\LocalVector{\theta}, \LocalVector{Z}}} \\
              & = \ELSCost[0]{\theta} + \ELSCost[n]{\theta} \\
    \ELSCost[0]{\theta} & \isdef 
            \frac{1}{F} 
                \sum_{r} 
                     \left| 
                             \LocalModel{A}(r, \LocalVector{\theta}) \true{\LocalVector{Y}}
                          - \LocalModel{B}(r, \LocalVector{\theta}) \true{\LocalVector{U}}
                          - \LocalModel{C}(r, \LocalVector{\theta}) 
                    \right|^2 \\
    \ELSCost[n]{\theta} & \isdef 
              \frac{1}{F} 
                     \sum_{r} 
                              \left| \LocalModel{A}(r,\LocalVector{\theta}) \right|^2 
                              \sigma_V^2
\end{align}
Note that this function is a quadratic function of $\LocalVector{\theta}$ and hence it can be represented exactly by its second order Taylor series around $\true\theta$:
\begin{multline}
  \ELSCost{\theta} = \ELSCost{\true\theta} 
  + \PartialDerivative{\ELSCost{\true\theta}}{\true\theta}  \left( \theta - \true\theta \right)\\
  + \frac{1}{2} \left( \theta - \true\theta \right)^{\T} 
  \PartialDerivative[2]{\ELSCost{\true\theta}}{\true\theta} 
   \left( \theta - \true\theta \right)
   \text{.}
\end{multline}
The minimizer $\estim\theta$ of this function can be obtained by equating the derivative
\begin{equation}
  \PartialDerivative{\ELSCost{\estim\theta}}{\estim\theta} = \left( \PartialDerivative{\ELSCost{\true\theta}}{\true\theta} \right)^{\T} + \PartialDerivative[2]{\ELSCost{\true\theta}}{\true\theta} 
   \left( \estim\theta - \true\theta \right)
\end{equation}
to zero.
This is equivalent to 
\begin{equation}
  \left( \estim\theta - \true\theta \right) = - \left(   \PartialDerivative[2]{\ELSCost{\true\theta}}{\true\theta} \right)^{-1}  \left( \PartialDerivative{\ELSCost{\true\theta}}{\true\theta} \right)^{\T}
  \text{.}
\end{equation}
which expresses the bias $\Delta\theta = \left( \estim\theta - \true\theta \right)$ of the estimated parameters.
Since $\ELSCost[0]{\theta}$ is minimized in $\true\theta$, this simplifies to
\begin{equation}
  \Delta\theta = - \left(   \PartialDerivative[2]{\ELSCost{\true\theta}}{\true\theta} \right)^{-1}  \left( \PartialDerivative{\ELSCost[n]{\true\theta}}{\true\theta} \right)^{\T}
  \text{.}
\end{equation}

Do note that the estimated parameters are complex values but that $\ELSCost{\theta}$ is a real-valued function.
This implies that the derivatives with respect to $\theta$ have to be calculated with respect to the real and imaginary part of $\theta$ separately as explained in~\citep{Messerschmitt2006} and~\citep[Section 15.9]{Pintelon2012}.

\TODO{write out subexpressions}
As a first order approximation, the bias on the \gls{FRF} can be written as
\begin{equation}
  \Delta \LocalModel{G}(\omega) \approx \PartialDerivative{\LocalModel{G}(\theta, \omega)}{\theta} \Delta \theta
\end{equation}

\section{Order selection}
Selecting suitable values of $\numel{W}$, $\numel{B}$, $\numel{A}$ and $\numel{C}$ obviously affects the quality of the obtained local models.
For local rational models, there are a few insights from parametric system identification that can be retained to make these choices easier.
In particular $\numel{C} \leq \max\set{\numel{B},\numel{A}}$ and from statistical point of view~\citep{Mahata2006}, the degrees of freedom should remain positive such that $2N_W + 1 \geq N_B + N_A + N_C + 2$.

Different heuristics exist to select the local bandwidth , e.g.~\citep{Fan1995,Thummala2012LPMBW,Stenman2000ASETFE} in the framework of local modeling approaches.
As such, we focus mainly on the selection of the model order hyper-parameters ($N_A$, $N_B$ and $N_C$) for a fixed value of $N$.

In this paper, we take an approach that decides amongst a group of candidate models, which model is a good model: i.e. one that captures that system dynamics but remains parsimonious.
To do so, one needs to define a criterion that both takes `goodness-of-fit' (e.g. value of the cost function, $R^2$, \ldots) and model complexity into account to ensure that the model captures only the systematic behavior and not random variations of the particular measurement.
The latter is often done using `cross-validation'.

In statistical learning, many relevant cross-validation approaches exist.
Within the system identification community a few simple approaches have been widely adopted~\citep[Chapter 11]{Pintelon2012} to ensure that one does not overmodel the data.
\paragraph{Plug-in methods}
In plug-in methods, a goodness-of-fit criterion (e.g. the log-likelihood function $\mathcal{L}$ of the estimated model $M$ given the estimation dataset $Z$), is penalized with a term accounting for the complexity of the model~\citep{Burnham2002}.
E.g. a common choice is the \gls{AIC}~\citep{Akaike1974} or the \gls{AICc}~\citep{Hurvich1989}, which are defined as
\begin{equation}
  L_{\mathrm{AIC(c)}} = - 2  \mathcal{L}(M | Z) + 2 \numel{\theta} + c \frac{2 \numel{\theta} (\numel{\theta} + 1)}{\numel{\Window} - \numel{\theta} - 1}
\end{equation}
where $\numel{\Window}$ is the number of  estimated parameters in the model $M(\theta)$.
For \gls{AIC} $c=0$, and $c=1$ produces the \gls{AICc}.
In the context of local modeling, the \gls{AIC} is a poor choice since it relies on asymptotic arguments of the data set, i.e. $\numel{Z} \to \infty$, which is obviously an unwarranted choice.
The best model among a set of $q$ possible models $\Set{M_1, \ldots, M_q}$ is then
\begin{equation}
  M_{\star} \isdef 
  \arg_{M} \min 
  \Set{
    L_{\mathrm{AIC(c)}}(M)
  }
\end{equation}
To select the best model $M_{\star}$ with such a criterion, the model $M_k$ among a set of models $\set{M_1, M_2, \ldots}$ that produces the smallest value of $L_{\mathrm{AICc}}$ is retained.

\begin{remark}
It should be noted that due to $\mathcal{L}(M | Z)$ , the value of $L_{\mathrm{AIC(c)}}(M,Z)$ is tightly coupled with the particular dataset $Z$.
Practically, this means that such criteria can only be used to compare local models that estimated from the same local dataset (i.e. they have the same $\numel{\Window}$ and center frequency).
\end{remark}

\paragraph{Holdout method}
The so-called `holdout' cross-validation method ensures a good model by first splitting the dataset into two smaller disjoint datasets: e.g. $70\%$ of the data  is used to estimate the model and a goodness-of-fit criterion (e.g. cost function) is used on the remaining data to determine which model has the best out-of-sample performance.
By using independent datasets and assessing the performance of the model on (unseen) validation data, overfitted models can be rejected.

\begin{remark}
Unfortunately, these typical approaches either assume that an abundantly large dataset is available or  assume the models are estimated on  a fixed dataset.
As such, these methods are not ideal for local modeling where the (local) dataset is rather small by design and could change by means of the local bandwidth $N_W$.
\end{remark}

\paragraph{Leave-out cross-validation}
Instead, we use a better-suited cross-validation approach known as \gls{LOOCV} that is related to the \gls{AIC}~\citep{Stone1977}.
Plainly, \gls{LOOCV} cross-validates the model by in turn leaving out each data point, estimating the local model parameters and observing the difference between the omitted data point and the estimate.
In effect, such an approach calculates the \gls{PRESS} for each local model:
\begin{equation}
 \PRESS = \frac{1}{2N_W+1} \sum_{r=-N_W}^{N_W} 
                          \abs{\ignoring{r}{\LocalVector{e}}}^2
                          \text{,}
\label{eq:PRESS-general}
\end{equation}
where $\ignoring{r}{\LocalVector{e}} = \LocalVector{Y}_{r} - \ignoring{r}{\LocalVector{\estimated{Y}}}$, i.e. the difference between the observed (local) output spectrum at the $r^{\text{th}}$ line and the predicted output spectrum at the same line when the corresponding data point is removed from the estimation procedure.
The model with the smallest \gls{PRESS} is retained.
In general, this means that \gls{LOOCV} requires estimating $2N_W + 1$ additional models (one for each omitted data point) and hence can be very time-consuming.

For linear models (i.e. $\LocalVector{Y} = \LocalVector{K} \LocalVector{\theta}$, such as for the \gls{LRM} and \gls{LPM}), however, the \gls{PRESS} can be calculated in a single step without estimating any additional models~\citep[Sec.~12.3.2]{Seber2003}:
\begin{equation}
\PRESS = \frac{\LocalVector{E}^{\HT} \LocalVector{W}^{-1} \LocalVector{E}}{2N_W + 1}
\end{equation}
where $\LocalVector{E}$ is the residual vector of the full estimation and $\LocalVector{W}$ is a diagonal weighting matrix with $\LocalVector{W}_{ii} = (1 - \LocalVector{H}_{ii})^{\HT} (1 - \LocalVector{H}_{ii})$ and
$\LocalVector{H} = \LocalVector{K} \pinv{\LocalVector{K}}$ is the so-called `hat-matrix' that is known to be idempotent.

\section{Simulation}
\label{sec:simulations}

In the simulations, we consider a discrete-time second order system with transfer function
\begin{equation}
G_0(z) = \frac{0.64587 z + 0.64143}
                             {47.9426 z^2 - 51.2955 z + 46.9933}
\end{equation}


\begin{figure}
 \centering
  \begin{tikzpicture}[scale=1,>=stealth]
    \matrix[ampersand replacement=\&, row sep=0.3cm, column sep=0.4cm] {
%% FIRST ROW
        \node (E) {$\true{E}$}; \&
        \node[block] (H0) {$\true{H}'$}; \&
        \node[amplifier] (K) {$\kappa$}; \&
        \node[none] (v) {};\\
%% SECOND ROW
        \node[input] (U) {$U$}; \&
        \node[block] (G0) {$\true{G}$}; \&
        \&
        \node[sum]    (sum)  {}; \& 
        \node (Y) {$Y$};\\
    };
    \draw [connector] (E)   -- (H0);
    \draw [connector] (U)   -- (G0);
    \draw [connector] (G0)  -- node[above] {$\true{Y}$} (sum);
    \draw [connector] (H0)  -- (K);
    \draw [connector] (K) -| node[above] {$V$} (sum);
    \draw [connector] (sum) -- (Y);
    % \draw [connector] (y)    -- ++(0,-1.5em) -| node[left,very near end,name=min] {{$-$}} (sum2);

    \begin{pgfonlayer}{background}
      \node[groupbox=black, inner sep=5pt,fit=(H0) (K)] (H0act) {};
      \node at (H0act.north) [color=black, above] {$\true{H}$};
   \end{pgfonlayer}
\end{tikzpicture}

  \caption{Block schematic used in the simulations}
\end{figure}

\TODO{H0}    

\TODO{SNR}
The \gls{SNR} in the system bandwidth $\BW$ is defined as
\begin{equation}
  \SNR_{\BW} \isdef
  \frac{\int_{\BW} \abs{Y_0(\omega)}^2 \dd{\omega}}
            {\int_{\BW} \abs{V(\omega)}^2 \dd{\omega}}
\end{equation}
where $\BW$ is the frequency range where

\TODO{plot example spectrums}


  % \subsection{Implications of the model structure}
  \subsection{Implications of the noise level}
  % \subsection{Implications of the noise coloring}

\TODO{insert figure: Y(Y,E,T) vs freq}
\TODO{insert figure: err vs SNR}

  \section{Measurements}
  \label{sec:measurements}
   \subsection{AVIS}
   \subsection{F16}

  \section{Conclusion}
  \label{sec:conclusion}

  \begin{itemize}
    \item LRM: extension to LRIC
    \item LRM: derivation of bias
    \item LRM vs LPM: at which settings is LRM better than LPM
    \item LRIC: much slower, performance is somewhat better, but at the cost of a lot of computational power and much harder model selection
    \item practical advice: use LRM rather than LPM unless in very noisy conditions (10 dB), only use LRIC as last resort
  \end{itemize}

