\section{Conclusion}\label{se:conclusion}

The paper introduced a novel time domain method to smooth the LPM-estimate of an FRF. It consisted of, after obtaining the FRF from the LPM, computing the associated estimated impulse response via the inverse Discrete Fourier Transform. Then, it was determined statistically at which time index the impulse response had decayed below the noise floor, yielding a point beyond which the response may be set to zero.

The results clearly indicate that the truncation technique lowers the impact of the noise on the estimate of the FRF, resulting in a decreased variance. A bias-variance trade-off is possible by tuning the time beyond which the impulse response is indistinguishable from the noise. %A bias-variance trade-off is possible, by tuning the time at which one considers the impulse response to be indistinguishable from the noise.