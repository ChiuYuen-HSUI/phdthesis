% \section{Simulations}\label{sec:excitation:sim}
% We considered a linear second order system that is to be identified, with the
% only prior knowledge that its resonance frequency $\fR$ lies in a known
% frequency band that may span several decades. 
% Our goal is to identify such a system by means of a single excitation signal, 
% such that the frequency response function (FRF) of the second order systems that are
% equally damped are identified with the same uncertainty near their resonance.

% \subsection{Excitation}
% To excite the system under test, we used quasi-logarithmic (quasi-log) random-phase multisines. 
% Their phase spectrum $\psi_{k}$ is the result of a uniform random process over $[0,2\pi[$.
% A logarithmic spacing of the frequency components causes each decade within the band of interest to contain the same number of frequency components.

% The quasi-log multisines were sampled at $f_s = 65\,536 \unit{Hz}$
% and consisted of $N= 65\,536$ points such that the frequency spacing was $f_0 = 1\unit{Hz}$ for the linear grid.
% This allowed the signal to cover four decades in the frequency domain,
% from which we did not use the first decade.
% Only frequencies between $1\unit{Hz}$ and $16\,384\unit{Hz}$ are excited, such that aliasing was avoided.

% For the quasi-log excitation, a frequency ratio of $\alpha = 1.05$ was chosen
% and the amplitude spectrum was kept constant $A_k = A_{\mathrm{in}}$. This
% amplitude was normalized such that the root mean square (RMS) value of $u[n]$
% was equal to $1$.

% The desired frequency ratio $\alpha$ may not be attainable for low frequencies,  due to a lack of frequency resolution.
% This causes a less dense coverage of the lowest frequency bands.

% \subsection{System under test}
% The systems we studied are low-pass second order systems with a given damping
% $\damping$, resonance frequency $\fR$ and a DC gain of $0 \unit{dB}$. Such systems
% can be represented in the discrete frequency domain by a transfer
% function of the form:
% \begin{equation}
%   G \left( z \right) = \frac{         b_2  z^{2} + b_1 z + b_0}%
%                             {\phantom{a_2} z^{2} + a_1 z + a_0}
%   \text{.}
%   \label{eq:excitation:tf}
% \end{equation}

% These discrete-time systems were constructed using the \matlab\ function \texttt{cheby1}.

% An output error (OE) model was used to describe the system under test with the
% input $u[n]$, the system output $y_0[n]$, the white noise source $e[n]$
% and the measured output $y[n] = y_0[n] + e[n]$.

% \subsection{Simulation setup}
% First, we considered a system with damping $\damping=0.05$ and resonant
% frequency $\fR = 40 \unit{Hz}$.

% The output noise $e[n]$ was white Gaussian noise with a standard deviation
% $\sigma_e = 0.1123$ which was approximately $10$ times smaller than the
% root mean square (RMS) value $y_0[n]$.

% The simulations were repeated $R=1000$ times for each system with different realizations of the multisines $u[n]$ and the noise $e[n]$.
% In order to avoid that all simulations started with zero initial conditions, $N + N_{\mathrm{trans}}$ data points were generated, from which the first $N_{\mathrm{trans}}$ points of each simulation were discarded. 

% \subsection{Results for a single system}
% The outcome of the simulation over these repetitions are displayed in \figref{fig:excitation:FRF1}.
% These results show the averaged estimated FRF ($\hat{G}$), the standard deviation  $\sigma_{\hat{G}}$ on these estimates, the true FRF ($G_0$) and the bias of the estimates ($\Bias (\hat{G})$).

% These graphs show that the bias is more than $20 \unit{dB}$ less than the variance.
% This means that the mean square error (MSE) of the estimated FRFs is mainly caused by the variance $\sigma_{\hat{G}}^2$ and not by a large bias.

% \begin{figure}[hpt]
%   \centering
%   \setlength{\figurewidth}{0.8\columnwidth}
%   \setlength{\figureheight}{5cm}
%   % This file was created by matlab2tikz v0.1.3.
% Copyright (c) 2008--2011, Nico Schlömer <nico.schloemer@gmail.com>
% All rights reserved.
% 
% The latest updates can be retrieved from
%   http://www.mathworks.com/matlabcentral/fileexchange/22022-matlab2tikz
% where you can also make suggestions and rate matlab2tikz.
% 
\begin{tikzpicture}

% defining custom colors
\definecolor{mycolor1}{rgb}{0,1,1}
\definecolor{mycolor2}{rgb}{1,0,1}


\begin{semilogxaxis}[%
scale only axis,
width=\figurewidth,
height=\figureheight,
xmin=1, xmax=16384,
ymin=-160, ymax=40,
xlabel={Frequency $f$ [$\mathrm{Hz}$]},
ylabel={Frequency Response Function [$\mathrm{dB}$]},
ymajorgrids,
legend entries={True System $G_0$,Mean Model $\hat{G}$,$\sqrt{\mathrm{MSE}(\hat{G})}$,$\sigma_{\hat{G}}$,$\mathrm{Bias}(\hat{G})$},
legend style={nodes=right}]
\addplot [
color=red,
solid,
line width=2.0pt
]
coordinates{
 (1,0.55461)(2,2.46521)(3,7.00118)(4,20.0109)(5,4.90495)(6,-1.92258)(7,-6.25462)(8,-9.50375)(9,-12.135)(10,-14.3612)(11,-16.2989)(12,-18.0189)(13,-19.5684)(14,-20.9799)(15,-22.2772)(16,-23.4785)(17,-24.5975)(18,-25.6452)(19,-26.6305)(20,-27.5608)(21,-28.4419)(22,-29.279)(23,-30.0765)(24,-30.8379)(25,-31.5664)(27,-32.9359)(28,-33.5813)(29,-34.2032)(31,-35.3827)(32,-35.9433)(34,-37.0122)(36,-38.0183)(38,-38.9687)(39,-39.4248)(41,-40.3024)(44,-41.5401)(46,-42.3185)(48,-43.0632)(50,-43.7771)(53,-44.7955)(56,-45.7572)(58,-46.3698)(61,-47.2499)(64,-48.0873)(68,-49.1444)(71,-49.8969)(75,-50.8518)(78,-51.5351)(82,-52.406)(86,-53.2353)(91,-54.219)(95,-54.9677)(100,-55.8603)(105,-56.7092)(110,-57.5184)(116,-58.4422)(122,-59.3193)(128,-60.1542)(134,-60.9507)(141,-61.8361)(148,-62.6785)(156,-63.5937)(164,-64.463)(172,-65.2909)(180,-66.0811)(189,-66.9291)(199,-67.8252)(209,-68.6773)(219,-69.4896)(230,-70.3413)(242,-71.2251)(254,-72.0661)(267,-72.9335)(280,-73.7597)(294,-74.6075)(309,-75.4722)(325,-76.3495)(341,-77.1846)(358,-78.03)(376,-78.8825)(395,-79.7391)(415,-80.5975)(436,-81.4553)(458,-82.3108)(481,-83.1623)(505,-84.0086)(530,-84.8483)(557,-85.712)(585,-86.5644)(614,-87.4054)(645,-88.2617)(677,-89.1034)(712,-89.9797)(747,-90.8141)(785,-91.6768)(824,-92.52)(866,-93.3846)(909,-94.2274)(955,-95.0862)(1003,-95.9394)(1053,-96.7858)(1106,-97.6404)(1162,-98.5002)(1220,-99.3482)(1281,-100.198)(1346,-101.06)(1413,-101.906)(1484,-102.761)(1559,-103.62)(1637,-104.472)(1719,-105.324)(1805,-106.177)(1896,-107.035)(1991,-107.89)(2091,-108.747)(2197,-109.612)(2307,-110.467)(2423,-111.327)(2544,-112.181)(2672,-113.043)(2806,-113.903)(2947,-114.765)(3096,-115.634)(3251,-116.496)(3414,-117.361)(3586,-118.231)(3766,-119.099)(3955,-119.97)(4154,-120.844)(4362,-121.717)(4582,-122.599)(4812,-123.479)(5053,-124.361)(5307,-125.249)(5574,-126.141)(5854,-127.037)(6148,-127.937)(6457,-128.843)(6781,-129.753)(7122,-130.672)(7479,-131.595)(7855,-132.529)(8250,-133.472)(8664,-134.424)(9099,-135.387)(9556,-136.363)(10036,-137.354)(10540,-138.361)(11070,-139.387)(11626,-140.433)(12210,-141.502)(12823,-142.598)(13467,-143.724)(14144,-144.888)(14854,-146.09)(15600,-147.34)(16384,-148.647) 
};

\addplot [
color=blue,
dashed,
line width=2.0pt
]
coordinates{
 (1,0.547501)(2,2.44904)(3,6.93359)(4,18.4654)(5,4.82972)(6,-1.94153)(7,-6.26253)(8,-9.50776)(9,-12.1372)(10,-14.3626)(11,-16.2998)(12,-18.0197)(13,-19.5691)(14,-20.9807)(15,-22.2782)(16,-23.4797)(17,-24.5989)(18,-25.647)(19,-26.6326)(20,-27.5633)(21,-28.4448)(22,-29.2824)(23,-30.0803)(24,-30.8422)(25,-31.5713)(27,-32.9419)(28,-33.588)(29,-34.2105)(31,-35.3914)(32,-35.9526)(34,-37.023)(36,-38.0307)(38,-38.9828)(39,-39.4398)(41,-40.3192)(44,-41.5599)(46,-42.3403)(48,-43.0872)(50,-43.8034)(53,-44.8253)(56,-45.7907)(58,-46.406)(61,-47.2901)(64,-48.1319)(68,-49.1951)(71,-49.9523)(75,-50.9141)(78,-51.6026)(82,-52.481)(86,-53.318)(91,-54.312)(95,-55.0693)(100,-55.9732)(105,-56.834)(110,-57.6558)(116,-58.5953)(122,-59.4891)(128,-60.3415)(134,-61.1565)(141,-62.0644)(148,-62.9305)(156,-63.8744)(164,-64.774)(172,-65.6337)(180,-66.4574)(189,-67.345)(199,-68.2875)(209,-69.1887)(219,-70.0526)(230,-70.9643)(242,-71.9173)(254,-72.8315)(267,-73.7828)(280,-74.6976)(294,-75.6466)(309,-76.6262)(325,-77.6337)(341,-78.6071)(358,-79.6088)(376,-80.6374)(395,-81.6924)(415,-82.7737)(436,-83.8821)(458,-85.0188)(481,-86.1855)(505,-87.3847)(530,-88.619)(557,-89.9396)(585,-91.299)(614,-92.6971)(645,-94.1757)(677,-95.6687)(712,-97.2229)(747,-98.6204)(785,-99.838)(824,-100.633)(866,-100.929)(909,-100.754)(955,-100.263)(1003,-99.63)(1053,-98.9678)(1106,-98.3238)(1162,-97.7266)(1220,-97.195)(1281,-96.7191)(1346,-96.2902)(1413,-95.9168)(1484,-95.5824)(1559,-95.2841)(1637,-95.0216)(1719,-94.7877)(1805,-94.5794)(1896,-94.3922)(1991,-94.2259)(2091,-94.0767)(2197,-93.942)(2307,-93.8226)(2423,-93.715)(2544,-93.6189)(2672,-93.5319)(2806,-93.4538)(2947,-93.3834)(3096,-93.3195)(3251,-93.2625)(3414,-93.211)(3586,-93.1642)(3766,-93.1222)(3955,-93.0843)(4154,-93.0499)(4362,-93.0189)(4582,-92.9908)(4812,-92.9654)(5053,-92.9425)(5307,-92.9218)(5574,-92.9029)(5854,-92.8859)(6148,-92.8706)(6457,-92.8566)(6781,-92.844)(7122,-92.8326)(7479,-92.8223)(7855,-92.8129)(8250,-92.8044)(8664,-92.7967)(9099,-92.7898)(9556,-92.7835)(10036,-92.7778)(10540,-92.7726)(11070,-92.7679)(11626,-92.7637)(12210,-92.7598)(12823,-92.7564)(13467,-92.7532)(14144,-92.7504)(14854,-92.7479)(15600,-92.7455)(16384,-92.7435) 
};

\addplot [
color=mycolor1,
dashed,
line width=2.0pt
]
coordinates{
 (1,-23.3363)(2,-20.8159)(3,-15.0692)(4,4.77177)(5,-14.9277)(6,-21.4349)(7,-25.0199)(8,-27.4798)(9,-29.3412)(10,-30.8323)(11,-32.0726)(12,-33.1325)(13,-34.0567)(14,-34.875)(15,-35.6087)(16,-36.2731)(17,-36.8795)(18,-37.437)(19,-37.9523)(20,-38.431)(21,-38.8775)(22,-39.2956)(23,-39.6883)(24,-40.0581)(25,-40.4073)(27,-41.0508)(28,-41.3483)(29,-41.6312)(31,-42.1579)(32,-42.4034)(34,-42.8627)(36,-43.2841)(38,-43.672)(39,-43.8545)(41,-44.1989)(44,-44.6688)(46,-44.9544)(48,-45.2204)(50,-45.4684)(53,-45.8102)(56,-46.1196)(58,-46.3099)(61,-46.5737)(64,-46.8141)(68,-47.1029)(71,-47.2984)(75,-47.5344)(78,-47.695)(82,-47.8898)(86,-48.0651)(91,-48.2605)(95,-48.4002)(100,-48.5569)(105,-48.6962)(110,-48.8206)(116,-48.9529)(122,-49.0694)(128,-49.1724)(134,-49.2638)(141,-49.358)(148,-49.4408)(156,-49.5236)(164,-49.5959)(172,-49.6592)(180,-49.7151)(189,-49.7703)(199,-49.8237)(209,-49.8702)(219,-49.9108)(230,-49.9499)(242,-49.9869)(254,-50.0191)(267,-50.0494)(280,-50.0757)(294,-50.1003)(309,-50.1233)(325,-50.1444)(341,-50.1627)(358,-50.1796)(376,-50.1951)(395,-50.2093)(415,-50.2222)(436,-50.2339)(458,-50.2445)(481,-50.2541)(505,-50.2628)(530,-50.2706)(557,-50.2779)(585,-50.2845)(614,-50.2903)(645,-50.2957)(677,-50.3006)(712,-50.3051)(747,-50.3091)(785,-50.3128)(824,-50.3161)(866,-50.3191)(909,-50.3218)(955,-50.3243)(1003,-50.3266)(1053,-50.3286)(1106,-50.3304)(1162,-50.3321)(1220,-50.3337)(1281,-50.335)(1346,-50.3363)(1413,-50.3374)(1484,-50.3384)(1559,-50.3394)(1637,-50.3402)(1719,-50.341)(1805,-50.3417)(1896,-50.3423)(1991,-50.3429)(2091,-50.3434)(2197,-50.3439)(2307,-50.3443)(2423,-50.3447)(2544,-50.345)(2672,-50.3454)(2806,-50.3456)(2947,-50.3459)(3096,-50.3461)(3251,-50.3464)(3414,-50.3466)(3586,-50.3467)(3766,-50.3469)(3955,-50.347)(4154,-50.3472)(4362,-50.3473)(4582,-50.3474)(4812,-50.3475)(5053,-50.3476)(5307,-50.3477)(5574,-50.3477)(5854,-50.3478)(6148,-50.3479)(6457,-50.3479)(6781,-50.348)(7122,-50.348)(7479,-50.348)(7855,-50.3481)(8250,-50.3481)(8664,-50.3481)(9099,-50.3482)(9556,-50.3482)(10036,-50.3482)(10540,-50.3482)(11070,-50.3483)(11626,-50.3483)(12210,-50.3483)(12823,-50.3483)(13467,-50.3483)(14144,-50.3483)(14854,-50.3483)(15600,-50.3484)(16384,-50.3484) 
};

\addplot [
color=mycolor2,
dash pattern=on 1pt off 3pt on 3pt off 3pt,
line width=2.0pt
]
coordinates{
 (1,-23.3671)(2,-20.996)(3,-15.9647)(4,-4.73098)(5,-15.6697)(6,-21.6008)(7,-25.0747)(8,-27.4994)(9,-29.347)(10,-30.8321)(11,-32.0698)(12,-33.1286)(13,-34.0524)(14,-34.8707)(15,-35.6046)(16,-36.2692)(17,-36.8759)(18,-37.4337)(19,-37.9492)(20,-38.4282)(21,-38.875)(22,-39.2933)(23,-39.6861)(24,-40.0561)(25,-40.4054)(27,-41.0492)(28,-41.3468)(29,-41.6298)(31,-42.1566)(32,-42.4022)(34,-42.8616)(36,-43.283)(38,-43.671)(39,-43.8535)(41,-44.1979)(44,-44.6678)(46,-44.9534)(48,-45.2193)(50,-45.4673)(53,-45.809)(56,-46.1184)(58,-46.3086)(61,-46.5723)(64,-46.8127)(68,-47.1013)(71,-47.2967)(75,-47.5326)(78,-47.6931)(82,-47.8878)(86,-48.063)(91,-48.2583)(95,-48.3979)(100,-48.5544)(105,-48.6937)(110,-48.818)(116,-48.9502)(122,-49.0666)(128,-49.1695)(134,-49.2608)(141,-49.355)(148,-49.4377)(156,-49.5204)(164,-49.5926)(172,-49.6559)(180,-49.7117)(189,-49.7668)(199,-49.8202)(209,-49.8666)(219,-49.9072)(230,-49.9463)(242,-49.9833)(254,-50.0154)(267,-50.0456)(280,-50.0719)(294,-50.0965)(309,-50.1194)(325,-50.1405)(341,-50.1588)(358,-50.1757)(376,-50.1912)(395,-50.2054)(415,-50.2182)(436,-50.2299)(458,-50.2405)(481,-50.2501)(505,-50.2588)(530,-50.2666)(557,-50.2739)(585,-50.2804)(614,-50.2863)(645,-50.2917)(677,-50.2965)(712,-50.3011)(747,-50.305)(785,-50.3087)(824,-50.312)(866,-50.3151)(909,-50.3178)(955,-50.3203)(1003,-50.3225)(1053,-50.3245)(1106,-50.3264)(1162,-50.3281)(1220,-50.3296)(1281,-50.3309)(1346,-50.3322)(1413,-50.3333)(1484,-50.3344)(1559,-50.3353)(1637,-50.3361)(1719,-50.3369)(1805,-50.3376)(1896,-50.3382)(1991,-50.3388)(2091,-50.3393)(2197,-50.3398)(2307,-50.3402)(2423,-50.3406)(2544,-50.3409)(2672,-50.3413)(2806,-50.3416)(2947,-50.3418)(3096,-50.3421)(3251,-50.3423)(3414,-50.3425)(3586,-50.3426)(3766,-50.3428)(3955,-50.3429)(4154,-50.3431)(4362,-50.3432)(4582,-50.3433)(4812,-50.3434)(5053,-50.3435)(5307,-50.3436)(5574,-50.3436)(5854,-50.3437)(6148,-50.3438)(6457,-50.3438)(6781,-50.3439)(7122,-50.3439)(7479,-50.344)(7855,-50.344)(8250,-50.344)(8664,-50.3441)(9099,-50.3441)(9556,-50.3441)(10036,-50.3441)(10540,-50.3442)(11070,-50.3442)(11626,-50.3442)(12210,-50.3442)(12823,-50.3442)(13467,-50.3442)(14144,-50.3442)(14854,-50.3442)(15600,-50.3443)(16384,-50.3443) 
};

\addplot [
color=black,
dash pattern=on 1pt off 3pt on 3pt off 3pt,
line width=2.0pt,
mark=x,
mark options={solid}
]
coordinates{
 (1,-44.2728)(2,-34.6273)(3,-22.3476)(4,4.25581)(5,-22.9442)(6,-35.587)(7,-43.7098)(8,-50.0819)(9,-55.6682)(10,-61.0106)(11,-66.6195)(12,-73.3943)(13,-84.9528)(14,-86.6516)(15,-78.949)(16,-76.0722)(17,-74.5769)(18,-73.7149)(19,-73.202)(20,-72.9025)(21,-72.7419)(22,-72.6757)(23,-72.6759)(24,-72.7235)(25,-72.8057)(27,-73.0399)(28,-73.1804)(29,-73.3311)(31,-73.653)(32,-73.8203)(34,-74.1612)(36,-74.5048)(38,-74.8465)(39,-75.0157)(41,-75.3497)(44,-75.8372)(46,-76.1522)(48,-76.4589)(50,-76.7571)(53,-77.1889)(56,-77.6023)(58,-77.8682)(61,-78.2529)(64,-78.6216)(68,-79.0898)(71,-79.4246)(75,-79.8506)(78,-80.1558)(82,-80.5452)(86,-80.9157)(91,-81.3544)(95,-81.6875)(100,-82.083)(105,-82.4574)(110,-82.8123)(116,-83.2144)(122,-83.593)(128,-83.9498)(134,-84.2869)(141,-84.657)(148,-85.0045)(156,-85.3762)(164,-85.7232)(172,-86.0478)(180,-86.3519)(189,-86.6716)(199,-87.0014)(209,-87.3071)(219,-87.5909)(230,-87.88)(242,-88.1706)(254,-88.4379)(267,-88.7039)(280,-88.9478)(294,-89.1883)(309,-89.4233)(325,-89.6509)(341,-89.8574)(358,-90.0563)(376,-90.2467)(395,-90.4277)(415,-90.5989)(436,-90.76)(458,-90.911)(481,-91.0519)(505,-91.1829)(530,-91.3044)(557,-91.4209)(585,-91.5278)(614,-91.6258)(645,-91.7183)(677,-91.8024)(712,-91.8834)(747,-91.9544)(785,-92.0221)(824,-92.083)(866,-92.1403)(909,-92.1917)(955,-92.2396)(1003,-92.2833)(1053,-92.3229)(1106,-92.3595)(1162,-92.3932)(1220,-92.4235)(1281,-92.4512)(1346,-92.4768)(1413,-92.4998)(1484,-92.5209)(1559,-92.5402)(1637,-92.5576)(1719,-92.5735)(1805,-92.588)(1896,-92.6012)(1991,-92.6132)(2091,-92.6241)(2197,-92.6341)(2307,-92.643)(2423,-92.6512)(2544,-92.6586)(2672,-92.6653)(2806,-92.6715)(2947,-92.677)(3096,-92.6821)(3251,-92.6867)(3414,-92.6908)(3586,-92.6946)(3766,-92.698)(3955,-92.7011)(4154,-92.7039)(4362,-92.7065)(4582,-92.7088)(4812,-92.7109)(5053,-92.7128)(5307,-92.7145)(5574,-92.7161)(5854,-92.7175)(6148,-92.7188)(6457,-92.72)(6781,-92.721)(7122,-92.722)(7479,-92.7229)(7855,-92.7237)(8250,-92.7244)(8664,-92.725)(9099,-92.7256)(9556,-92.7262)(10036,-92.7266)(10540,-92.7271)(11070,-92.7275)(11626,-92.7278)(12210,-92.7282)(12823,-92.7285)(13467,-92.7287)(14144,-92.729)(14854,-92.7292)(15600,-92.7294)(16384,-92.7296) 
};

\end{semilogxaxis}
\end{tikzpicture}

%   \caption{FRF of a second order system with $\fR=40\unit{Hz}$ and $\damping=0.05$ 
%            along with its variance and bias.}%
%   \label{fig:excitation:FRF1}
% \end{figure}

% \subsection{Simulations on a range of second order systems}
% The simulation carried out above was repeated for a more general set of systems.
% We studied systems with both a damping of $\damping=0.05$ (as above) and $\damping=0.2$.
% For these different damping values, systems with resonant frequencies
% $\fR \in \{ 40  \unit{Hz}\!,$
%           $ 80  \unit{Hz}\!,$
%           $200  \unit{Hz}\!,$
%           $400  \unit{Hz}\!,$
%           $800  \unit{Hz}\!,$
%           $  2 \unit{kHz}\!,$
%           $  4 \unit{kHz}\!\}$ were considered.
% The parameters pertaining to sampling and the excitation signal were left unchanged
% from the previous simulations. For the systems with a damping of $0.2$, $\sigma_e$
% was put at $0.0678$, which is approximately $10\%$ of the RMS value of $y_0[n]$.
% To allow an easy comparison with a multisine with an equidistant (linear) frequency grid, we also evaluated the variance for such an excitation signal (which had all other parameters determined similarly to the quasi-log multisines).

% In \figref{fig:excitation:damping005} and \figref{fig:excitation:damping02} the results from these
% simulations are presented in a condensed form. 
% They show the mean estimated FRFs ($\hat{G}$), the true FRFs ($G_0$) and the corresponding standard deviations ($\sigma_{\hat{G}}$) near the resonance frequency of the simulated systems.
% The bias was at least $20\unit{dB}$ smaller than the standard deviation as well, but is omitted from the figures for clarity.

% We can verify that for the smaller damping ($\damping=0.05$ in \figref{fig:excitation:damping005})
% the variance near the resonance is almost identical for all systems at $-39.3\unit{dB} \pm 1.8 \unit{dB}$. 

% For the systems that are damped more ($\damping=0.2$ in \figref{fig:excitation:damping02}),
% we observe that the variance is nearly constant over the frequency range as well.
% The standard deviation lies at around $-49.3 \unit{dB} \pm 2.2\unit{dB}$.
% Systems that have a lower resonance frequency, exhibit a slightly higher variance 
% on their estimated FRF (near the resonance). 

% The logarithmically-spaced excitation provides a much more constant dependency of resonance frequency to the variance than a linearly-spaced multisine, where the variance varies over respectively $35$ and $25 \unit{dB}$.

% \begin{figure}%
%   \centering
%   \setlength{\figurewidth}{0.8\columnwidth}
%   \setlength{\figureheight}{5cm}
%   % This file was created by matlab2tikz v0.1.3.
% Copyright (c) 2008--2011, Nico Schlömer <nico.schloemer@gmail.com>
% All rights reserved.
% 
% The latest updates can be retrieved from
%   http://www.mathworks.com/matlabcentral/fileexchange/22022-matlab2tikz
% where you can also make suggestions and rate matlab2tikz.
% 
\begin{tikzpicture}

% defining custom colors



\begin{semilogxaxis}[%
scale only axis,
unbounded coords=jump,
width=\figurewidth,
height=\figureheight,
xmin=2, xmax=7000,
ymin=-65, ymax=22,
xlabel={Frequency $f$ \axisunit{Hz}},
ylabel={Amplitude $\abs{G(\omega)}$ \axisunit{dB}},
ymajorgrids, xmajorgrids, yminorgrids,
legend entries={True Systems $G_0$,
%                 Mean Models $\hat{G}$,
                $\sigma_{\hat{G}}$ (Linear),
                $\sigma_{\hat{G}}$ (Quasi-Log),
                $\sigma_{\hat{G}}$ (Compensated Log)},
legend style={at={(axis cs:7000,0)}, anchor={north east},nodes=right,font=\footnotesize}]
%%%%%%%%%%%%%%%%%%%%%%%%%%%%%%%%%%%%%%
\addplot [truesys]
table[]{\thisDir/data/damping005-cmp/truesys.tsv};

%%%%%%%%%%%%%%%%%%%%%%%%%%%%%%%%%%%%%%%%%%%%%%%
% \addplot [model, forget plot]
% table[]{\thisDir/data/damping005-cmp/model.tsv};

%%%%%%%%%%%%%%%%%%%%%%%%%%%%%%%%%%%%%%%%%%%%%%%%

\addplot [linms, forget plot]
table[]{\thisDir/data/damping005-cmp/linms.tsv};

\addplot[linmsconnect]
table[]{\thisDir/data/damping005-cmp/linms-connect.tsv};

%%%%%%%%%%%%%%%%%%%%%%%%%%%%%%%%%%%%%%%%%%%%%%%%%

\addplot [qlogms,forget plot]
table[]{\thisDir/data/damping005-cmp/qlogms.tsv};

\addplot [qlogmsconnect]
table[]{\thisDir/data/damping005-cmp/qlogms-connect.tsv};

%%%%%%%%%%%%%%%%%%%%%%%%%%%%%%%%%%%%%%%%%%%%%%%%%%%%%%

\addplot [clogms, forget plot]
table[]{\thisDir/data/damping005-cmp/clogms.tsv};

\addplot [clogmsconnect]
table[]{\thisDir/data/damping005-cmp/clogms-connect.tsv};

\end{semilogxaxis}
\end{tikzpicture}

%   \caption{Comparison of the FRF and corresponding variances for
%            different second order systems with constant damping
%            $\damping=0.05$ but different resonance frequencies around
%            their respective resonance frequencies, for both a quasi-log 
%            and a linear multisine excitation.}%
%   \label{fig:excitation:damping005}
% \end{figure}

% \begin{figure}%
%   \centering
%   \setlength{\figurewidth}{0.8\columnwidth}
%   \setlength{\figureheight}{5cm}
%   % This file was created by matlab2tikz v0.1.3.
% Copyright (c) 2008--2011, Nico Schlömer <nico.schloemer@gmail.com>
% All rights reserved.
% 
% The latest updates can be retrieved from
%   http://www.mathworks.com/matlabcentral/fileexchange/22022-matlab2tikz
% where you can also make suggestions and rate matlab2tikz.
% 
\begin{tikzpicture}

% defining custom colors
\definecolor{mycolor1}{rgb}{0,0.501961,0}
\definecolor{mycolor2}{rgb}{0,1,1}
\definecolor{mycolor3}{rgb}{1,0,1}


\begin{semilogxaxis}[%
scale only axis,
unbounded coords=jump,
width=\figurewidth,
height=\figureheight,
xmin=2, xmax=7000,
ymin=-70, ymax=10,
xlabel={Frequency $f$ \axisunit{Hz}},
ylabel={Amplitude $\abs{G(\omega)}$ \axisunit{dB}},
ymajorgrids, xmajorgrids, yminorgrids,
legend entries={True Systems $G_0$,
                %Mean Model $\hat{G}$,
                $\sigma_{\hat{G}}$ (Linear),
                $\sigma_{\hat{G}}$ (Quasi-Log),
                $\sigma_{\hat{G}}$ (Compensated Log)},
legend style={at={(axis cs:7000,0)}, anchor={north east},nodes=right,font=\footnotesize}]

\addplot [truesys]
table[]{\thisDir/data/damping02-cmp/truesys.tsv};

% \addplot [model]
% table[]{\thisDir/data/damping02-cmp/model.tsv};

\addplot [linms,forget plot]
table[]{\thisDir/data/damping02-cmp/linms.tsv};

\addplot [linmsconnect]
table[]{\thisDir/data/damping02-cmp/linms-connect.tsv};

\addplot [qlogms,forget plot]
table[]{\thisDir/data/damping02-cmp/qlogms.tsv};

\addplot [qlogmsconnect]
table[]{\thisDir/data/damping02-cmp/qlogms-connect.tsv};

\addplot [clogms,forget plot]
table[]{\thisDir/data/damping02-cmp/clogms.tsv};

\addplot [clogmsconnect]
table[]{\thisDir/data/damping02-cmp/clogms-connect.tsv};

\end{semilogxaxis}
\end{tikzpicture}

%   \caption{Comparison of the FRF and corresponding variances for
%            different second order systems with constant damping
%            $\damping=0.2$ but different resonance frequencies around
%            their respective resonance frequencies, for both a quasi-log 
%            and a linear multisine excitation.}%
%   \label{fig:excitation:damping02}
% \end{figure}

% \subsection{Conclusion}
% The variance on the FRF near resonances with identical damping is nearly independent of the resonance frequency when a quasi-log multisine is used as excitation signal. 
% This result was obtained for second order sections and illustrated on a more general system.
